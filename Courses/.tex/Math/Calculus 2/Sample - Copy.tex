%%%%%%%%%%%%%%%%%%%%%%%%%%%%%%%%%%%%%%%%%%%%%%%%%%%%%%%%%%%%%%%%%%%%%%%%%%%%%%%%
% ------------------------- A GENERAL PURPOSE LATEX TEMPLATE -------------------------
%%%%%%%%%%%%%%%%%%%%%%%%%%%%%%%%%%%%%%%%%%%%%%%%%%%%%%%%%%%%%%%%%%%%%%%%%%%%%%%%

% The \documentclass command defines the type of document. 'article' is great for reports, homework, and short papers.
\documentclass{article}

% --- PREAMBLE: PACKAGES AND GLOBAL SETTINGS ---
% The preamble is where you load packages and define commands for the whole document.

% --- Required Packages ---

% For page layout and margins.
\usepackage[
    left=1in,
    textwidth=6.5in, 
    top=1in,
    bottom=1in
]{geometry}

% For including images.
\usepackage{graphicx} 

% For advanced math environments like align*, matrices, etc.
\usepackage{amsmath}  

% For more math symbols like \mathbb, \mathfrak, etc.
\usepackage{amssymb}
\usepackage{amsfonts}

% For customizing lists (like changing the spacing between items).
\usepackage{enumitem} 

% For including external PDF files into your document.
\usepackage{pdfpages} 

% --- Document Information ---
\title{My Document Title}
\author{Your Name}
\date{\today} % \today automatically inserts the current date. You can also type a specific date like "September 23, 2025".


%%%%%%%%%%%%%%%%%%%%%%%%%%%%%%%%%%%%%%%%%%%%%%%%%%%%%%%%%%%%%%%%%%%%%%%%%%%%%%%%
% ------------------------- START OF THE DOCUMENT -------------------------
%%%%%%%%%%%%%%%%%%%%%%%%%%%%%%%%%%%%%%%%%%%%%%%%%%%%%%%%%%%%%%%%%%%%%%%%%%%%%%%%
\begin{document}

% This command generates the title based on the \title, \author, and \date commands above.
\maketitle

% --- SECTION 1: DOCUMENT STRUCTURE AND TEXT ---

\section{Headings and Paragraphs}
This is the first section. Sections are the highest level of organization. You can write your main paragraphs directly under the section title. This is a placeholder paragraph to demonstrate how text flows. The quick brown fox jumps over the lazy dog.

\subsection{This is a Subsection}
Subsections help you break down a major section into smaller, more manageable parts. Notice the numbering automatically becomes 1.1, 1.2, etc. This is another placeholder paragraph.

\subsubsection{This is a Subsubsection}
For even finer detail, you can use a subsubsection. The numbering continues to nest. This is the final level of numbered headings in the standard article class.

\bigskip % This adds a large vertical space, as per your "blue line" rule, to separate major parts.

% --- SECTION 2: LISTS WITH CUSTOM SPACING ---

\section{Creating Lists}
Here we demonstrate how to create lists with the specific spacing you requested.

\subsection{Itemized (Bulleted) Lists}
The \texttt{itemize} environment creates a bulleted list. We add \texttt{[itemsep=5pt]} to add 5 points of vertical space between the main bullet points, as per your "red line" rule.

\begin{itemize}[itemsep=5pt]
    \item This is the first main item in the list.
    \item This is the second main item. Notice the space between this and the first item.
        \begin{itemize}
            \item This is a nested item. Nested lists have their own default spacing.
            \item This is another nested item.
        \end{itemize}
    \item This is the third main item.
\end{itemize}

\subsection{Enumerated (Numbered) Lists}
The \texttt{enumerate} environment works the same way but creates a numbered list. We can apply the same custom spacing.

\begin{enumerate}[itemsep=5pt]
    \item This is the first numbered item.
    \item This is the second numbered item, showing the 5pt spacing.
    \item And this is the third.
\end{enumerate}

\bigskip % "Blue line" spacing before the next major section.

% --- SECTION 3: MATHEMATICAL TYPESETTING ---

\section{Writing Mathematics}
LaTeX is famous for its beautiful math typesetting.

\subsection{Inline and Display Math}
You can place math directly within a line of text (inline math) by wrapping it in single dollar signs, like this: $E = mc^2$. This is useful for simple variables and equations.

For more complex or important equations, use a display math environment. The \texttt{align*} environment (from the \texttt{amsmath} package) is excellent for this. It centers the equations and allows you to align them neatly at the equals signs (using the \texttt{\&} character). The asterisk (*) prevents equation numbering.
\begin{align*}
    f(x) &= x^2 + 2x + 1 \\
    g(x,y) &= \frac{\sqrt{x^2 + y^2}}{x - y} \\
    \int_0^\infty e^{-x^2}\,dx &= \frac{\sqrt{\pi}}{2}
\end{align*}

\bigskip % "Blue line" spacing.

% --- SECTION 4: INCLUDING EXTERNAL FILES ---

\section{Images and PDFs}
Here is how you can include external files like images and other PDFs.

\subsection{Including an Image}
To include an image, it's best to place it inside a \texttt{figure} environment. This allows LaTeX to place it nicely and lets you add a caption.
\begin{figure}[h!] % The [h!] is an optional placement specifier, meaning "place it here, if possible!"
    \centering % This centers the image on the page
    % NOTE: Replace 'your-image-filename.png' with the actual file name. 
    % The file should be in the same folder as this .tex file.
    \includegraphics[width=0.5\textwidth]{your-image-filename.png}
    \caption{This is the caption for the image.}
    \label{fig:myfigure} % A label lets you refer to the figure elsewhere in the text, e.g., see Figure~\ref{fig:myfigure}.
\end{figure}

\subsection{Including a PDF}
To include another PDF document, use the \texttt{\\includepdf} command from the \texttt{pdfpages} package.
\begin{itemize}
    \item The command below will insert all pages from the specified PDF file at this point in the document.
    \item \textbf{IMPORTANT:} The PDF file must be in the same folder as this .tex file.
    \item You must replace \texttt{"Your-PDF-Filename.pdf"} with the actual name of your PDF file.
\end{itemize}

% The \newpage command ensures the PDF starts on a fresh page, which is usually desired.
\newpage 

% The [pages={-}] option tells LaTeX to include ALL pages of the PDF.
% You could also specify pages, e.g., [pages={1, 3-5}] to include page 1 and pages 3 through 5.
\includepdf[pages={-}]{Your-PDF-Filename.pdf}

\end{document}
%%%%%%%%%%%%%%%%%%%%%%%%%%%%%%%%%%%%%%%%%%%%%%%%%%%%%%%%%%%%%%%%%%%%%%%%%%%%%%%%
% ------------------------- END OF THE DOCUMENT -------------------------
%%%%%%%%%%%%%%%%%%%%%%%%%%%%%%%%%%%%%%%%%%%%%%%%%%%%%%%%%%%%%%%%%%%%%%%%%%%%%%%%```