\documentclass{article}
\usepackage{graphicx} % Required for inserting images
\usepackage{amsmath}  % Required for advanced math environments like align*
\usepackage{amssymb}  % For more math symbols

% --- GEOMETRY PACKAGE FOR PAGE LAYOUT ---
\usepackage[
    left=1in,
    textwidth=7in, 
    top=1in,
    bottom=1in
]{geometry}
% -----------------------------------------

\title{Homework 7.8 Improper Integrals}
\author{Tashfeen Omran}
\date{September 2025}

\begin{document}

\maketitle

\section{Improper Integral Problems and Solutions}

%---------------------------------------------------------------
\subsection{Problem 1}
Determine whether the integral $ \int_{1}^{\infty} 3x^{-4} \,dx $ is convergent or divergent. If it is convergent, evaluate it.
\subsubsection*{Solution}
This is a Type 1 improper integral because it has an infinite limit of integration.
\begin{align*}
    \int_{1}^{\infty} 3x^{-4} \,dx &= \lim_{t \to \infty} \int_{1}^{t} 3x^{-4} \,dx \\
    &= \lim_{t \to \infty} \left[ 3 \frac{x^{-3}}{-3} \right]_{1}^{t} \\
    &= \lim_{t \to \infty} \left[ -x^{-3} \right]_{1}^{t} = \lim_{t \to \infty} \left[ -\frac{1}{x^3} \right]_{1}^{t} \\
    &= \lim_{t \to \infty} \left( -\frac{1}{t^3} - \left(-\frac{1}{1^3}\right) \right) \\
    &= \lim_{t \to \infty} \left( -\frac{1}{t^3} + 1 \right) \\
    &= 0 + 1 = 1
\end{align*}
\textbf{Answer:} Convergent. The integral evaluates to \textbf{1}.

%---------------------------------------------------------------
\subsection{Problem 2}
Determine whether the integral $ \int_{-\infty}^{-1} \frac{1}{\sqrt[3]{x}} \,dx $ is convergent or divergent. If it is convergent, evaluate it.
\subsubsection*{Solution}
This is a Type 1 improper integral.
\begin{align*}
    \int_{-\infty}^{-1} x^{-1/3} \,dx &= \lim_{t \to -\infty} \int_{t}^{-1} x^{-1/3} \,dx \\
    &= \lim_{t \to -\infty} \left[ \frac{x^{2/3}}{2/3} \right]_{t}^{-1} = \lim_{t \to -\infty} \left[ \frac{3}{2}x^{2/3} \right]_{t}^{-1} \\
    &= \lim_{t \to -\infty} \left( \frac{3}{2}(-1)^{2/3} - \frac{3}{2}t^{2/3} \right) \\
    &= \frac{3}{2}(1) - \lim_{t \to -\infty} \frac{3}{2}t^{2/3} \\
    &= \frac{3}{2} - \infty = -\infty
\end{align*}
As $t \to -\infty$, $t^{2/3}$ (which is $(\sqrt[3]{t})^2$) approaches $+\infty$. The limit does not exist.
\textbf{Answer:} \textbf{DIVERGES}.

%---------------------------------------------------------------
\subsection{Problem 3}
Determine whether the integral $ \int_{-6}^{\infty} \frac{1}{x+7} \,dx $ is convergent or divergent.
\subsubsection*{Solution}
This is a Type 1 improper integral.
\begin{align*}
    \int_{-6}^{\infty} \frac{1}{x+7} \,dx &= \lim_{t \to \infty} \int_{-6}^{t} \frac{1}{x+7} \,dx \\
    &= \lim_{t \to \infty} \left[ \ln|x+7| \right]_{-6}^{t} \\
    &= \lim_{t \to \infty} \left( \ln|t+7| - \ln|-6+7| \right) \\
    &= \lim_{t \to \infty} \ln(t+7) - \ln(1) \\
    &= \infty - 0 = \infty
\end{align*}
The limit does not exist.
\textbf{Answer:} \textbf{DIVERGES}.

%---------------------------------------------------------------
\subsection{Problem 4}
Determine whether the integral $ \int_{8}^{\infty} \frac{1}{(x-7)^{3/2}} \,dx $ is convergent or divergent.
\subsubsection*{Solution}
This is a Type 1 improper integral.
\begin{align*}
    \int_{8}^{\infty} (x-7)^{-3/2} \,dx &= \lim_{t \to \infty} \int_{8}^{t} (x-7)^{-3/2} \,dx \\
    &= \lim_{t \to \infty} \left[ \frac{(x-7)^{-1/2}}{-1/2} \right]_{8}^{t} = \lim_{t \to \infty} \left[ \frac{-2}{\sqrt{x-7}} \right]_{8}^{t} \\
    &= \lim_{t \to \infty} \left( \frac{-2}{\sqrt{t-7}} - \left(\frac{-2}{\sqrt{8-7}}\right) \right) \\
    &= \lim_{t \to \infty} \left( \frac{-2}{\sqrt{t-7}} + 2 \right) \\
    &= 0 + 2 = 2
\end{align*}
\textbf{Answer:} Convergent. The integral evaluates to \textbf{2}.

%---------------------------------------------------------------
\subsection{Problem 5}
Determine whether the integral $ \int_{0}^{\infty} \frac{1}{\sqrt{1+x}} \,dx $ is convergent or divergent.
\subsubsection*{Solution}
This is a Type 1 improper integral.
\begin{align*}
    \int_{0}^{\infty} (1+x)^{-1/2} \,dx &= \lim_{t \to \infty} \int_{0}^{t} (1+x)^{-1/2} \,dx \\
    &= \lim_{t \to \infty} \left[ \frac{(1+x)^{1/2}}{1/2} \right]_{0}^{t} = \lim_{t \to \infty} \left[ 2\sqrt{1+x} \right]_{0}^{t} \\
    &= \lim_{t \to \infty} \left( 2\sqrt{1+t} - 2\sqrt{1+0} \right) \\
    &= \infty - 2 = \infty
\end{align*}
The limit does not exist.
\textbf{Answer:} \textbf{DIVERGES}.

%---------------------------------------------------------------
\subsection{Problem 6}
Determine whether the integral $ \int_{-\infty}^{0} \frac{x}{(x^2+4)^2} \,dx $ is convergent or divergent.
\subsubsection*{Solution}
This is a Type 1 improper integral. Use u-substitution: $ u = x^2+4 $, $ du = 2x \,dx \implies x\,dx = du/2 $.
\begin{align*}
    \int_{-\infty}^{0} \frac{x}{(x^2+4)^2} \,dx &= \lim_{t \to -\infty} \int_{t}^{0} \frac{x}{(x^2+4)^2} \,dx \\
    &= \lim_{t \to -\infty} \int_{x=t}^{x=0} \frac{1}{u^2} \frac{du}{2} \\
    &= \frac{1}{2} \lim_{t \to -\infty} \left[ -\frac{1}{u} \right]_{x=t}^{x=0} = \frac{1}{2} \lim_{t \to -\infty} \left[ -\frac{1}{x^2+4} \right]_{t}^{0} \\
    &= \frac{1}{2} \lim_{t \to -\infty} \left( -\frac{1}{0^2+4} - \left(-\frac{1}{t^2+4}\right) \right) \\
    &= \frac{1}{2} \lim_{t \to -\infty} \left( -\frac{1}{4} + \frac{1}{t^2+4} \right) \\
    &= \frac{1}{2} \left( -\frac{1}{4} + 0 \right) = -\frac{1}{8}
\end{align*}
\textbf{Answer:} Convergent. The integral evaluates to \textbf{-1/8}.

%---------------------------------------------------------------
\subsection{Problem 7}
Determine whether the integral $ \int_{1}^{\infty} \frac{x^3+x+1}{x^5} \,dx $ is convergent or divergent.
\subsubsection*{Solution}
First, simplify the integrand.
\[ \frac{x^3+x+1}{x^5} = \frac{x^3}{x^5} + \frac{x}{x^5} + \frac{1}{x^5} = x^{-2} + x^{-4} + x^{-5} \]
Now evaluate the integral:
\begin{align*}
    \int_{1}^{\infty} (x^{-2} + x^{-4} + x^{-5}) \,dx &= \lim_{t \to \infty} \int_{1}^{t} (x^{-2} + x^{-4} + x^{-5}) \,dx \\
    &= \lim_{t \to \infty} \left[ -x^{-1} - \frac{x^{-3}}{3} - \frac{x^{-4}}{4} \right]_{1}^{t} \\
    &= \lim_{t \to \infty} \left[ -\frac{1}{x} - \frac{1}{3x^3} - \frac{1}{4x^4} \right]_{1}^{t} \\
    &= \lim_{t \to \infty} \left( \left(-\frac{1}{t} - \frac{1}{3t^3} - \frac{1}{4t^4}\right) - \left(-1 - \frac{1}{3} - \frac{1}{4}\right) \right) \\
    &= (0 - 0 - 0) - \left(-\frac{12}{12} - \frac{4}{12} - \frac{3}{12}\right) = - \left(-\frac{19}{12}\right) = \frac{19}{12}
\end{align*}
\textbf{Answer:} Convergent. The integral evaluates to \textbf{19/12}.

%---------------------------------------------------------------
\subsection{Problem 8}
Determine whether the integral $ \int_{0}^{\infty} \frac{e^x}{(8+e^x)^2} \,dx $ is convergent or divergent.
\subsubsection*{Solution}
Use u-substitution: $ u = 8+e^x $, $ du = e^x \,dx $.
\begin{align*}
    \int_{0}^{\infty} \frac{e^x}{(8+e^x)^2} \,dx &= \lim_{t \to \infty} \int_{0}^{t} \frac{e^x}{(8+e^x)^2} \,dx \\
    &= \lim_{t \to \infty} \left[ -\frac{1}{8+e^x} \right]_{0}^{t} \\
    &= \lim_{t \to \infty} \left( -\frac{1}{8+e^t} - \left(-\frac{1}{8+e^0}\right) \right) \\
    &= \lim_{t \to \infty} \left( -\frac{1}{8+e^t} + \frac{1}{9} \right) \\
    &= 0 + \frac{1}{9} = \frac{1}{9}
\end{align*}
\textbf{Answer:} Convergent. The integral evaluates to \textbf{1/9}.

%---------------------------------------------------------------
\subsection{Problem 9}
Determine whether the integral $ \int_{-\infty}^{\infty} 9xe^{-x^2} \,dx $ is convergent or divergent.
\subsubsection*{Solution}
The integral is over $ (-\infty, \infty) $, so we split it at an arbitrary point, like $ x=0 $.
\[ \int_{-\infty}^{\infty} 9xe^{-x^2} \,dx = \int_{-\infty}^{0} 9xe^{-x^2} \,dx + \int_{0}^{\infty} 9xe^{-x^2} \,dx \]
Evaluate the second integral first. Use u-substitution: $ u = -x^2, du = -2x \,dx \implies 9x\,dx = -\frac{9}{2}du $.
\begin{align*}
    \int_{0}^{\infty} 9xe^{-x^2} \,dx &= \lim_{t \to \infty} \int_{0}^{t} 9xe^{-x^2} \,dx \\
    &= \lim_{t \to \infty} \left[ -\frac{9}{2}e^{-x^2} \right]_{0}^{t} \\
    &= \lim_{t \to \infty} \left( -\frac{9}{2}e^{-t^2} - \left(-\frac{9}{2}e^0\right) \right) = 0 + \frac{9}{2} = \frac{9}{2}
\end{align*}
Now evaluate the first integral:
\begin{align*}
    \int_{-\infty}^{0} 9xe^{-x^2} \,dx &= \lim_{s \to -\infty} \int_{s}^{0} 9xe^{-x^2} \,dx \\
    &= \lim_{s \to -\infty} \left[ -\frac{9}{2}e^{-x^2} \right]_{s}^{0} \\
    &= \lim_{s \to -\infty} \left( -\frac{9}{2}e^{0} - \left(-\frac{9}{2}e^{-s^2}\right) \right) = -\frac{9}{2} - 0 = -\frac{9}{2}
\end{align*}
Since both integrals converge, the original integral converges. The total value is $ \frac{9}{2} + (-\frac{9}{2}) = 0 $.
\textbf{Answer:} Convergent. The integral evaluates to \textbf{0}.
(Note: This could also be solved by recognizing $f(x)=9xe^{-x^2}$ is an odd function.)

%---------------------------------------------------------------
\subsection{Problem 10}
Determine whether the integral $ \int_{-\infty}^{\infty} \frac{x^5}{x^6+1} \,dx $ is convergent or divergent.
\subsubsection*{Solution}
Split the integral at $ x=0 $. Let's evaluate the part from $ [0, \infty) $.
Use u-substitution: $ u = x^6+1, du = 6x^5 \,dx \implies x^5\,dx = du/6 $.
\begin{align*}
    \int_{0}^{\infty} \frac{x^5}{x^6+1} \,dx &= \lim_{t \to \infty} \int_{0}^{t} \frac{x^5}{x^6+1} \,dx \\
    &= \lim_{t \to \infty} \left[ \frac{1}{6}\ln|x^6+1| \right]_{0}^{t} \\
    &= \frac{1}{6} \lim_{t \to \infty} \left( \ln(t^6+1) - \ln(1) \right) = \frac{1}{6}(\infty - 0) = \infty
\end{align*}
Since one part of the split integral diverges, the entire integral diverges.
\textbf{Answer:} \textbf{DIVERGES}.

%---------------------------------------------------------------
\subsection{Problem 11}
Determine whether the integral $ \int_{0}^{\infty} 4\sin^2(\alpha) \,d\alpha $ is convergent or divergent.
\subsubsection*{Solution}
Use the half-angle identity: $ \sin^2(\alpha) = \frac{1-\cos(2\alpha)}{2} $.
\begin{align*}
    \int_{0}^{\infty} 4\sin^2(\alpha) \,d\alpha &= \int_{0}^{\infty} 4\left(\frac{1-\cos(2\alpha)}{2}\right) \,d\alpha = \int_{0}^{\infty} (2-2\cos(2\alpha)) \,d\alpha \\
    &= \lim_{t \to \infty} \int_{0}^{t} (2-2\cos(2\alpha)) \,d\alpha \\
    &= \lim_{t \to \infty} \left[ 2\alpha - \sin(2\alpha) \right]_{0}^{t} \\
    &= \lim_{t \to \infty} ( (2t - \sin(2t)) - (0-0) )
\end{align*}
As $ t \to \infty $, the $2t$ term goes to infinity. The term $ \sin(2t) $ oscillates between -1 and 1, but it cannot stop the $2t$ term from growing infinitely large. The limit does not exist.
\textbf{Answer:} \textbf{DIVERGES}.

%---------------------------------------------------------------
\subsection{Problem 12}
(a) Evaluate the integral: $ \int_{0}^{t} 8\sin^2(\alpha) \,d\alpha $
(b) Determine whether $ \int_{0}^{\infty} 8\sin^2(\alpha) \,d\alpha $ is convergent or divergent.
\subsubsection*{Solution (a)}
This is a standard definite integral. Using the half-angle identity:
\begin{align*}
    \int_{0}^{t} 8\sin^2(\alpha) \,d\alpha &= \int_{0}^{t} 8\left(\frac{1-\cos(2\alpha)}{2}\right) \,d\alpha = \int_{0}^{t} (4-4\cos(2\alpha)) \,d\alpha \\
    &= \left[ 4\alpha - 2\sin(2\alpha) \right]_{0}^{t} \\
    &= (4t - 2\sin(2t)) - (0-0)
\end{align*}
\textbf{Answer (a):} $ 4t - 2\sin(2t) $

\subsubsection*{Solution (b)}
To determine if the improper integral converges, we take the limit of the result from part (a) as $ t \to \infty $.
\[ \lim_{t \to \infty} (4t - 2\sin(2t)) = \infty \]
The limit does not exist.
\textbf{Answer (b):} \textbf{DIVERGES}.

%---------------------------------------------------------------
\subsection{Problem 13}
Determine whether the integral $ \int_{0}^{\infty} \sin(\theta)e^{\cos(\theta)} \,d\theta $ is convergent or divergent.
\subsubsection*{Solution}
Use u-substitution: $ u = \cos(\theta), du = -\sin(\theta) \,d\theta $.
\begin{align*}
    \int_{0}^{\infty} \sin(\theta)e^{\cos(\theta)} \,d\theta &= \lim_{t \to \infty} \int_{0}^{t} \sin(\theta)e^{\cos(\theta)} \,d\theta \\
    &= \lim_{t \to \infty} \left[ -e^{\cos(\theta)} \right]_{0}^{t} \\
    &= \lim_{t \to \infty} \left( -e^{\cos(t)} - (-e^{\cos(0)}) \right) \\
    &= \lim_{t \to \infty} \left( e - e^{\cos(t)} \right)
\end{align*}
As $ t \to \infty $, $ \cos(t) $ oscillates between -1 and 1. Therefore, $ e^{\cos(t)} $ oscillates between $ e^{-1} $ and $ e^1 $. The limit does not settle on a single value.
\textbf{Answer:} \textbf{DIVERGES}.

%---------------------------------------------------------------
\subsection{Problem 14}
Determine whether the integral $ \int_{1}^{\infty} \frac{1}{x^2+x} \,dx $ is convergent or divergent.
\subsubsection*{Solution}
Use partial fraction decomposition: $ \frac{1}{x(x+1)} = \frac{A}{x} + \frac{B}{x+1} \implies 1 = A(x+1) + Bx $.
If $ x=0, A=1 $. If $ x=-1, B=-1 $. So, $ \frac{1}{x^2+x} = \frac{1}{x} - \frac{1}{x+1} $.
\begin{align*}
    \int_{1}^{\infty} \left(\frac{1}{x} - \frac{1}{x+1}\right) \,dx &= \lim_{t \to \infty} \int_{1}^{t} \left(\frac{1}{x} - \frac{1}{x+1}\right) \,dx \\
    &= \lim_{t \to \infty} \left[ \ln|x| - \ln|x+1| \right]_{1}^{t} \\
    &= \lim_{t \to \infty} \left[ \ln\left|\frac{x}{x+1}\right| \right]_{1}^{t} \\
    &= \lim_{t \to \infty} \left( \ln\left(\frac{t}{t+1}\right) - \ln\left(\frac{1}{2}\right) \right) \\
\end{align*}
As $ t \to \infty, \frac{t}{t+1} \to 1 $, so $ \ln(\frac{t}{t+1}) \to \ln(1) = 0 $.
The result is $ 0 - \ln(1/2) = -(-\ln(2)) = \ln(2) $.
\textbf{Answer:} Convergent. The integral evaluates to \textbf{ln(2)}.

%---------------------------------------------------------------
\subsection{Problem 15}
Determine whether the integral $ \int_{2}^{\infty} \frac{dv}{v^2+2v-3} $ is convergent or divergent.
\subsubsection*{Solution}
Use partial fraction decomposition: $ \frac{1}{(v+3)(v-1)} = \frac{A}{v+3} + \frac{B}{v-1} \implies 1 = A(v-1) + B(v+3) $.
If $ v=1, B=1/4 $. If $ v=-3, A=-1/4 $.
\begin{align*}
    \int_{2}^{\infty} \frac{1/4}{v-1} - \frac{1/4}{v+3} \,dv &= \frac{1}{4} \lim_{t \to \infty} \int_{2}^{t} \left(\frac{1}{v-1} - \frac{1}{v+3}\right) \,dv \\
    &= \frac{1}{4} \lim_{t \to \infty} \left[ \ln|v-1| - \ln|v+3| \right]_{2}^{t} \\
    &= \frac{1}{4} \lim_{t \to \infty} \left[ \ln\left|\frac{v-1}{v+3}\right| \right]_{2}^{t} \\
    &= \frac{1}{4} \lim_{t \to \infty} \left( \ln\left(\frac{t-1}{t+3}\right) - \ln\left(\frac{1}{5}\right) \right) \\
    &= \frac{1}{4} ( \ln(1) - \ln(1/5) ) = \frac{1}{4} (0 - (-\ln(5))) = \frac{\ln(5)}{4}
\end{align*}
\textbf{Answer:} Convergent. The integral evaluates to \textbf{ln(5)/4}.

%---------------------------------------------------------------
\subsection{Problem 16}
Determine whether the integral $ \int_{-\infty}^{0} \frac{z}{z^4+81} \,dz $ is convergent or divergent.
\subsubsection*{Solution}
Use u-substitution: $ u = z^2, du = 2z \,dz \implies z\,dz = du/2 $.
\begin{align*}
    \int_{-\infty}^{0} \frac{z}{z^4+81} \,dz &= \lim_{t \to -\infty} \int_{t}^{0} \frac{z}{(z^2)^2+81} \,dz \\
    &= \lim_{t \to -\infty} \frac{1}{2} \int_{z=t}^{z=0} \frac{du}{u^2+81} \\
    &= \frac{1}{2} \lim_{t \to -\infty} \left[ \frac{1}{9}\arctan\left(\frac{u}{9}\right) \right]_{z=t}^{z=0} = \frac{1}{18} \lim_{t \to -\infty} \left[ \arctan\left(\frac{z^2}{9}\right) \right]_{t}^{0} \\
    &= \frac{1}{18} \lim_{t \to -\infty} \left( \arctan(0) - \arctan\left(\frac{t^2}{9}\right) \right)
\end{align*}
As $ t \to -\infty, t^2 \to \infty $, so $ \arctan(t^2/9) \to \pi/2 $.
The result is $ \frac{1}{18}(0 - \frac{\pi}{2}) = -\frac{\pi}{36} $.
\textbf{Answer:} Convergent. The integral evaluates to \textbf{-$\pi$/36}.

%---------------------------------------------------------------
\subsection{Problem 17}
(a) Evaluate the integral: $ \int_{t}^{0} \frac{z}{z^4+36} \,dz $
(b) Determine whether $ \int_{-\infty}^{0} \frac{z}{z^4+36} \,dz $ is convergent or divergent.
\subsubsection*{Solution (a)}
Similar to problem 16, use u-substitution $ u=z^2, du=2z\,dz $.
\begin{align*}
    \int_{t}^{0} \frac{z}{z^4+36} \,dz &= \frac{1}{2} \int_{z=t}^{z=0} \frac{du}{u^2+36} \\
    &= \frac{1}{2} \left[ \frac{1}{6}\arctan\left(\frac{u}{6}\right) \right]_{z=t}^{z=0} = \frac{1}{12} \left[ \arctan\left(\frac{z^2}{6}\right) \right]_{t}^{0} \\
    &= \frac{1}{12} \left( \arctan(0) - \arctan\left(\frac{t^2}{6}\right) \right)
\end{align*}
\textbf{Answer (a):} $ -\frac{1}{12}\arctan\left(\frac{t^2}{6}\right) $

\subsubsection*{Solution (b)}
Take the limit of the result from part (a) as $ t \to -\infty $.
\[ \lim_{t \to -\infty} \left(-\frac{1}{12}\arctan\left(\frac{t^2}{6}\right)\right) = -\frac{1}{12}\left(\frac{\pi}{2}\right) = -\frac{\pi}{24} \]
\textbf{Answer (b):} Convergent. The integral evaluates to \textbf{-$\pi$/24}.

%---------------------------------------------------------------
\subsection{Problem 18}
Determine whether the integral $ \int_{0}^{9} \frac{3}{\sqrt[3]{x-1}} \,dx $ is convergent or divergent.
\subsubsection*{Solution}
This is a Type 2 improper integral because the function has an infinite discontinuity at $ x=1 $, which is within the interval $ [0,9] $. We must split the integral at the point of discontinuity.
\[ \int_{0}^{9} \frac{3}{(x-1)^{1/3}} \,dx = \int_{0}^{1} 3(x-1)^{-1/3} \,dx + \int_{1}^{9} 3(x-1)^{-1/3} \,dx \]
Evaluate the first part:
\begin{align*}
    \lim_{t \to 1^-} \int_{0}^{t} 3(x-1)^{-1/3} \,dx &= \lim_{t \to 1^-} \left[ 3 \frac{(x-1)^{2/3}}{2/3} \right]_{0}^{t} = \lim_{t \to 1^-} \left[ \frac{9}{2}(x-1)^{2/3} \right]_{0}^{t} \\
    &= \lim_{t \to 1^-} \left( \frac{9}{2}(t-1)^{2/3} - \frac{9}{2}(-1)^{2/3} \right) \\
    &= 0 - \frac{9}{2}(1) = -\frac{9}{2}
\end{align*}
This part converges. Now evaluate the second part:
\begin{align*}
    \lim_{s \to 1^+} \int_{s}^{9} 3(x-1)^{-1/3} \,dx &= \lim_{s \to 1^+} \left[ \frac{9}{2}(x-1)^{2/3} \right]_{s}^{9} \\
    &= \lim_{s \to 1^+} \left( \frac{9}{2}(9-1)^{2/3} - \frac{9}{2}(s-1)^{2/3} \right) \\
    &= \frac{9}{2}(8)^{2/3} - 0 = \frac{9}{2}(4) = 18
\end{align*}
This part also converges. Since both parts converge, the original integral converges. The total value is $ -\frac{9}{2} + 18 = \frac{27}{2} $.
\textbf{Answer:} Convergent. The integral evaluates to \textbf{27/2}.

%---------------------------------------------------------------
\section{Analysis of Problems and Techniques}

\subsection{Types of Improper Integrals Encountered}
\begin{enumerate}
    \item \textbf{Type 1: Infinite Intervals}
    \begin{itemize}
        \item Integrals over $ [a, \infty) $: Problems 1, 3, 4, 5, 7, 8, 11, 12, 13, 14, 15.
        \item Integrals over $ (-\infty, b] $: Problems 2, 6, 16, 17.
        \item Integrals over $ (-\infty, \infty) $: Problems 9, 10. These must be split into two separate improper integrals, e.g., $ \int_{-\infty}^{c} f(x) \,dx + \int_{c}^{\infty} f(x) \,dx $. The integral converges only if BOTH parts converge.
    \end{itemize}
    \item \textbf{Type 2: Infinite Discontinuity}
    \begin{itemize}
        \item The integrand has a vertical asymptote at $x=c$ within the interval $[a, b]$. The integral must be split at the discontinuity: $ \int_{a}^{b} f(x) \,dx = \int_{a}^{c} f(x) \,dx + \int_{c}^{b} f(x) \,dx $. This was seen in Problem 18 at $x=1$ on the interval $[0,9]$.
    \end{itemize}
\end{enumerate}

\subsection{Convergence and Divergence Rules (p-Integrals)}
A key tool for quickly assessing convergence is the \textbf{p-integral test}.
\begin{itemize}
    \item For Type 1 integrals: $ \int_{a}^{\infty} \frac{1}{x^p} \,dx $ (where $a > 0$) \textbf{converges if $p > 1$} and \textbf{diverges if $p \le 1$}.
        \begin{itemize}
            \item Problem 1: $p=4 > 1$, converges.
            \item Problem 4: Form is $1/u^{3/2}$, $p=3/2 > 1$, converges.
            \item Problem 2: $p=1/3 \le 1$ on an infinite interval, diverges.
            \item Problem 5: Form is $1/u^{1/2}$, $p=1/2 \le 1$, diverges.
        \end{itemize}
    \item For Type 2 integrals: $ \int_{0}^{a} \frac{1}{x^p} \,dx $ (discontinuity at 0) \textbf{converges if $p < 1$} and \textbf{diverges if $p \ge 1$}.
        \begin{itemize}
            \item Problem 18: Form is $1/u^{1/3}$, $p=1/3 < 1$, converges.
        \end{itemize}
\end{itemize}

\subsection{Techniques and Algebraic Manipulations Used}
\begin{itemize}
    \item \textbf{Limit Definition}: The fundamental technique for all problems was to rewrite the improper integral as a limit of a proper integral.
    \item \textbf{U-Substitution}: Used in Problems 6, 8, 9, 10, 13, 16, 17 to simplify the integrand before integration.
    \item \textbf{Partial Fraction Decomposition}: Necessary for integrating rational functions where the denominator is factorable. Used in Problems 14 and 15.
    \item \textbf{Trigonometric Identities}: The half-angle identity $ \sin^2(\alpha) = (1-\cos(2\alpha))/2 $ was crucial for Problems 11 and 12.
    \item \textbf{Splitting Integrals}: Required for Type 1 integrals over $(-\infty, \infty)$ (Problems 9, 10) and for Type 2 integrals with an interior discontinuity (Problem 18).
    \item \textbf{Recognizing Odd/Even Functions}: In Problem 9, the integrand is an odd function integrated over a symmetric interval $(-\infty, \infty)$. If such an integral converges, its value must be 0. In Problem 10, the integrand was also odd, but it was shown to diverge. \textit{Trick: You must still prove convergence of one half of the integral before concluding the value is 0.}
    \item \textbf{Simplifying the Integrand}: In Problem 7, dividing each term in the numerator by the denominator simplified the problem into a sum of p-integrals.
\end{itemize}

\subsection{Essential Limits to Know}
For evaluating improper integrals, knowledge of limits at infinity is critical.
\begin{itemize}
    \item $ \lim_{x \to \infty} \frac{1}{x^p} = 0 $ for any $ p > 0 $.
    \item $ \lim_{x \to \infty} e^x = \infty $ and $ \lim_{x \to -\infty} e^x = 0 $.
    \item $ \lim_{x \to \infty} \ln(x) = \infty $ and $ \lim_{x \to 0^+} \ln(x) = -\infty $.
    \item $ \lim_{x \to \infty} \arctan(x) = \frac{\pi}{2} $ and $ \lim_{x \to -\infty} \arctan(x) = -\frac{\pi}{2} $.
    \item Limits of oscillating functions like $ \sin(x) $ or $ \cos(x) $ as $ x \to \infty $ do not exist.
\end{itemize}

\subsection{Additional Tricks and Untested Concepts}
The provided problems did not cover all aspects of improper integrals. Here are other important concepts:
\begin{itemize}
    \item \textbf{The Comparison Test}: For an integrand that is difficult to integrate directly, you can compare it to a simpler function. If $f(x) \ge g(x) \ge 0$:
    \begin{itemize}
        \item If $ \int_{a}^{\infty} f(x) \,dx $ converges, then $ \int_{a}^{\infty} g(x) \,dx $ also converges.
        \item If $ \int_{a}^{\infty} g(x) \,dx $ diverges, then $ \int_{a}^{\infty} f(x) \,dx $ also diverges.
        \item \textbf{Example}: To determine if $ \int_{1}^{\infty} \frac{1}{x^2+5} \,dx $ converges, we can note that $ \frac{1}{x^2+5} < \frac{1}{x^2} $. Since $ \int_{1}^{\infty} \frac{1}{x^2} \,dx $ converges (p-integral with p=2), our integral must also converge.
    \end{itemize}
    \item \textbf{The Limit Comparison Test}: If $f(x)$ and $g(x)$ are positive functions and $ \lim_{x \to \infty} \frac{f(x)}{g(x)} = L $, where L is a finite, positive number, then $ \int f(x) \,dx $ and $ \int g(x) \,dx $ either both converge or both diverge.
    \begin{itemize}
        \item \textbf{Example}: For $ \int_{1}^{\infty} \frac{x}{x^3-2} \,dx $, we can compare it to $ g(x) = \frac{x}{x^3} = \frac{1}{x^2} $. Since $ \lim_{x \to \infty} \frac{x/(x^3-2)}{1/x^2} = 1 $, and we know $ \int_{1}^{\infty} \frac{1}{x^2} \,dx $ converges, our integral must also converge.
    \end{itemize}
     \item \textbf{Checking the Domain}: As a trick, always check that the function is defined over the integration interval. A problem like $ \int_{0}^{2} \frac{1}{\sqrt{x-3}} \,dx $ is invalid because the integrand is not real for any value in the interval. The initial (incorrect) reading of Problem 18 as $1/\sqrt{x-1}$ would have made the integral from $[0,1]$ invalid in the real number system.
\end{itemize}

\end{document}