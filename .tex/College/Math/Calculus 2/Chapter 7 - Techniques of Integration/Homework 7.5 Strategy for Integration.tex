\documentclass{article}
\usepackage{graphicx} % Required for inserting images
\usepackage{amsmath}  % Required for advanced math environments like align*
\usepackage{amssymb}  % For more math symbols

% --- GEOMETRY PACKAGE FOR PAGE LAYOUT ---
\usepackage[
    left=1in,
    textwidth=7in, 
    top=1in,
    bottom=1in
]{geometry}
% -----------------------------------------

\title{Homework 7.5 Strategy for Integration}
\author{Tashfeen Omran}
\date{September 2025}

\begin{document}

\maketitle

\section{Integration Problems and Solutions}

%---------------------------------------------------------------
\subsection{Problem 1}
\begin{enumerate}
    \item[(a)] Evaluate the integral: $ \int \frac{3x}{1+x^2} \,dx $
    \subsubsection*{Solution}
    To evaluate the integral of (3x)/(1+x²), u-substitution is used.
    Let $ u = 1+x^2 $, so $ du = 2x \,dx $.
    The integral becomes:
    \[ \int \frac{3}{2u} \,du = \frac{3}{2} \int \frac{1}{u} \,du \]
    This evaluates to:
    \textbf{Answer:} $ \frac{3}{2}\ln|1+x^2| + C $

    \item[(b)] Evaluate the integral: $ \int \frac{3}{1+x^2} \,dx $
    \subsubsection*{Solution}
    The integral of 3/(1+x²) is a standard form.
    It evaluates to:
    \textbf{Answer:} $ 3\tan^{-1}(x) + C $

    \item[(c)] Evaluate the integral: $ \int \frac{3}{1-x^2} \,dx $
    \subsubsection*{Solution}
    For the integral of 3/(1-x²), partial fraction decomposition is applied. The expression can be rewritten as:
    \[ \frac{3}{2} \left[ \int \frac{1}{1-x} \,dx + \int \frac{1}{1+x} \,dx \right] \]
    This integrates to $ -(\frac{3}{2})\ln|1-x| + (\frac{3}{2})\ln|1+x| + C $, which can be simplified to:
    \textbf{Answer:} $ \frac{3}{2}\ln\left|\frac{1+x}{1-x}\right| + C $
\end{enumerate}


%---------------------------------------------------------------
\subsection{Problem 2}
\begin{enumerate}
    \item[(a)] Evaluate the integral: $ \int 7x\sqrt{x^2-1} \,dx $
    \subsubsection*{Solution}
    To solve the integral of 7x√(x²-1), let $ u = x^2-1 $, which gives $ du = 2x \,dx $. The integral becomes:
    \[ \frac{7}{2} \int \sqrt{u} \,du \]
    This evaluates to:
    \textbf{Answer:} $ \frac{7}{3}(x^2-1)^{3/2} + C $

    \item[(b)] Evaluate the integral: $ \int \frac{7}{x\sqrt{x^2-1}} \,dx $
    \subsubsection*{Solution}
    The integral of 7/(x√(x²-1)) is a standard form related to inverse trigonometric functions. The solution is:
    \textbf{Answer:} $ 7\sec^{-1}|x| + C $

    \item[(c)] Evaluate the integral: $ \int \frac{7\sqrt{x^2-1}}{x} \,dx $
    \subsubsection*{Solution}
    For the integral of 7√(x²-1)/x, a trigonometric substitution is appropriate. Let $ x = \sec(\theta) $, so $ dx = \sec(\theta)\tan(\theta) \,d\theta $. The integral simplifies to:
    \[ 7 \int \tan^2(\theta) \,d\theta = 7 \int (\sec^2(\theta) - 1) \,d\theta \]
    This integrates to $ 7(\tan(\theta) - \theta) + C $. Substituting back gives:
    \textbf{Answer:} $ 7(\sqrt{x^2-1} - \sec^{-1}(x)) + C $
\end{enumerate}


%---------------------------------------------------------------
\subsection{Problem 3}
\begin{enumerate}
    \item[(a)] Evaluate the integral: $ \int \frac{2\ln(x)}{x} \,dx $
    \subsubsection*{Solution}
    To integrate 2ln(x)/x, use u-substitution with $ u = \ln(x) $, so $ du = (1/x) \,dx $. The integral becomes:
    \[ 2 \int u \,du \]
    which results in:
    \textbf{Answer:} $ (\ln(x))^2 + C $

    \item[(b)] Evaluate the integral: $ \int 2\ln(2x) \,dx $
    \subsubsection*{Solution}
    The integral of 2ln(2x) requires integration by parts. Let $ u = \ln(2x) $ and $ dv = 2 \,dx $. Then $ du = (1/x) \,dx $ and $ v = 2x $. The formula $ \int u \,dv = uv - \int v \,du $ gives:
    \[ 2x\ln(2x) - \int 2 \,dx \]
    which evaluates to:
    \textbf{Answer:} $ 2x\ln(2x) - 2x + C $

    \item[(c)] Evaluate the integral: $ \int 2x\ln|x| \,dx $
    \subsubsection*{Solution}
    For the integral of 2xln|x|, integration by parts is also used. Let $ u = \ln|x| $ and $ dv = 2x \,dx $. Then $ du = (1/x) \,dx $ and $ v = x^2 $. This leads to:
    \[ x^2\ln|x| - \int x \,dx \]
    which results in:
    \textbf{Answer:} $ x^2\ln|x| - \frac{x^2}{2} + C $
\end{enumerate}


%---------------------------------------------------------------
\subsection{Problem 4}
\begin{enumerate}
    \item[(a)] Evaluate the integral: $ \int 2\sin^2(x) \,dx $
    \subsubsection*{Solution}
    To evaluate the integral of 2sin²(x), the half-angle identity $ \sin^2(x) = (1-\cos(2x))/2 $ is used. The integral becomes:
    \[ \int (1-\cos(2x)) \,dx \]
    which evaluates to:
    \textbf{Answer:} $ x - \frac{1}{2}\sin(2x) + C $

    \item[(b)] Evaluate the integral: $ \int 2\sin^3(x) \,dx $
    \subsubsection*{Solution}
    For the integral of 2sin³(x), rewrite $ \sin^3(x) $ as $ \sin(x)(1-\cos^2(x)) $. Using u-substitution with $ u = \cos(x) $ and $ du = -\sin(x) \,dx $, the integral becomes:
    \[ -2 \int (1-u^2) \,du \]
    This evaluates to:
    \textbf{Answer:} $ -2(\cos(x) - \frac{1}{3}\cos^3(x)) + C $

    \item[(c)] Evaluate the integral: $ \int 2\sin(2x) \,dx $
    \subsubsection*{Solution}
    The integral of 2sin(2x) is a straightforward integration. It evaluates to:
    \textbf{Answer:} $ -\cos(2x) + C $
\end{enumerate}


%---------------------------------------------------------------
\subsection{Problem 5}
Evaluate the integral: $ \int \frac{\cos(x)}{3-\sin(x)} \,dx $
\subsubsection*{Solution}
To solve the integral of cos(x)/(3-sin(x)), a u-substitution is performed. Let $ u = 3-\sin(x) $, so $ du = -\cos(x) \,dx $. The integral becomes:
\[ -\int \frac{1}{u} \,du \]
which evaluates to:
\textbf{Answer:} $ -\ln|3-\sin(x)| + C $


%---------------------------------------------------------------
\subsection{Problem 6}
Evaluate the integral: $ \int \frac{x}{x^4+16} \,dx $
\subsubsection*{Solution}
For the integral of x/(x⁴+16), a substitution is made. Let $ u = x^2 $, so $ du = 2x \,dx $. The integral becomes:
\[ \frac{1}{2} \int \frac{1}{u^2+16} \,du \]
This is a standard arctangent form, and the result is:
\textbf{Answer:} $ \frac{1}{8}\tan^{-1}\left(\frac{x^2}{4}\right) + C $


%---------------------------------------------------------------
\subsection{Problem 7}
Evaluate the integral: $ \int 5t \sin(t)\cos(t) \,dt $
\subsubsection*{Solution}
To evaluate the integral of 5t sin(t)cos(t), the double angle identity $ \sin(2t) = 2\sin(t)\cos(t) $ is used. The integral becomes:
\[ \frac{5}{2} \int t \sin(2t) \,dt \]
Integration by parts is then applied, with $ u = t $ and $ dv = \sin(2t) \,dt $. This yields $ du = dt $ and $ v = -(\frac{1}{2})\cos(2t) $. The final result is:
\textbf{Answer:} $ \frac{5}{2}\left[-\frac{t}{2}\cos(2t) + \frac{1}{4}\sin(2t)\right] + C $


%---------------------------------------------------------------
\subsection{Problem 8}
Evaluate the integral: $ \int \frac{2x-3}{x^2+3x} \,dx $
\subsubsection*{Solution}
For the integral of (2x-3)/(x²+3x), we can split the denominator by factoring: x(x+3). However, a u-substitution is more direct. Let u = x²+3x, then du = (2x+3) dx. The integral becomes ∫(2x-3)/(u) * (du/(2x+3)). The integral of (2x-3)/(x^2+3x) is:
\textbf{Answer:} $ \ln|x^2+3x|+C $


%---------------------------------------------------------------
\subsection{Problem 9}
Evaluate the integral: $ \int x \sec(x)\tan(x) \,dx $
\subsubsection*{Solution}
To integrate x sec(x)tan(x), integration by parts is the appropriate method. Let $ u = x $ and $ dv = \sec(x)\tan(x) \,dx $. Then $ du = dx $ and $ v = \sec(x) $. This gives:
\[ x\sec(x) - \int \sec(x) \,dx \]
The final answer is:
\textbf{Answer:} $ x\sec(x) - \ln|\sec(x)+\tan(x)| + C $


%---------------------------------------------------------------
\subsection{Problem 10}
Evaluate the integral: $ \int 9\theta \tan^2(\theta) \,d\theta $
\subsubsection*{Solution}
To evaluate the integral of 9θ tan²(θ), use the identity $ \tan^2(\theta) = \sec^2(\theta) - 1 $. The integral becomes:
\[ 9 \int \theta(\sec^2(\theta)-1) \,d\theta = 9\left[\int \theta\sec^2(\theta) \,d\theta - \int \theta \,d\theta\right] \]
The first part requires integration by parts with $ u = \theta $ and $ dv = \sec^2(\theta) \,d\theta $, leading to $ \theta\tan(\theta) - \int \tan(\theta) \,d\theta $. The final result is:
\textbf{Answer:} $ 9\left[\theta\tan(\theta) + \ln|\cos(\theta)| - \frac{\theta^2}{2}\right] + C $


%---------------------------------------------------------------
\section{Problem Types and Techniques Used}
\begin{itemize}
    \item \textbf{U-Substitution:} This was a fundamental technique used in problems 1a, 2a, 3a, 4b, 5, 6, and 8.
    \item \textbf{Integration by Parts:} This method was essential for problems 3b, 3c, 7, 9, and 10.
    \item \textbf{Trigonometric Integrals:} Problems 4a, 4b, and 4c involved powers and multiples of trigonometric functions, requiring specific identities.
    \item \textbf{Trigonometric Substitution:} Problem 2c utilized this technique to simplify a radical expression.
    \item \textbf{Partial Fraction Decomposition:} Problem 1c was solved using this algebraic method.
    \item \textbf{Standard Integral Forms:} Problems 1b and 2b were recognizable as basic antiderivatives involving inverse trigonometric functions.
\end{itemize}


%---------------------------------------------------------------
\section{Algebraic Manipulations and Tricks}
\begin{itemize}
    \item \textbf{Trigonometric Identities:}
    \begin{itemize}
        \item Half-angle identity: $ \sin^2(x) = (1-\cos(2x))/2 $ (Problem 4a).
        \item Pythagorean identity: $ \sin^2(x) = 1-\cos^2(x) $ (Problem 4b).
        \item Double angle identity: $ \sin(2t) = 2\sin(t)\cos(t) $ (Problem 7).
        \item Identity for $ \tan^2(\theta) $: $ \tan^2(\theta) = \sec^2(\theta) - 1 $ (Problem 10).
    \end{itemize}
    \item \textbf{Completing the Square and Factoring:} While not explicitly used in the final solutions, these are often considered for rational functions and were part of the initial analysis for problems like 1c and 8.
    \item \textbf{Substitution before Integration by Parts:} In problem 7, a trigonometric identity was used to simplify the integrand before applying integration by parts.
\end{itemize}


\end{document}```