\documentclass{article}
\usepackage{amsmath}
\usepackage{amssymb}
\usepackage[margin=1in]{geometry}

\title{Homework 7.4: Integration by Fraction Decomposition}
\author{Tashfeen Omran}
\date{October 2025}

\begin{document}

\maketitle

\part{Comprehensive Introduction, Context, and Prerequisites}

\section{Core Concepts}
Integration by Partial Fraction Decomposition is not an integration technique itself, but rather an \textbf{algebraic procedure} for rewriting a complex rational function (a polynomial divided by another polynomial) as a sum of simpler, more easily integrable fractions.

The fundamental idea is that any rational function $R(x) = \frac{P(x)}{Q(x)}$, where $P(x)$ and $Q(x)$ are polynomials, can be rewritten as a sum of fractions whose denominators involve the factors of $Q(x)$.

The procedure relies on one crucial prerequisite: the degree of the numerator $P(x)$ must be \textbf{strictly less than} the degree of the denominator $Q(x)$. If it is not, one must first perform polynomial long division.

The form of the decomposition depends entirely on the types of factors present in the denominator, $Q(x)$:

\begin{enumerate}
    \item \textbf{Case 1: Distinct Linear Factors.} If the denominator has a factor $(ax+b)$ that is not repeated, the decomposition will include a term of the form:
    \[ \frac{A}{ax+b} \]
    \item \textbf{Case 2: Repeated Linear Factors.} If the denominator has a factor $(ax+b)^k$ where $k \ge 2$, the decomposition must include a term for each power of the factor, from 1 to $k$:
    \[ \frac{A_1}{ax+b} + \frac{A_2}{(ax+b)^2} + \dots + \frac{A_k}{(ax+b)^k} \]
    \item \textbf{Case 3: Distinct Irreducible Quadratic Factors.} If the denominator has a quadratic factor $(ax^2+bx+c)$ that cannot be factored into linear terms (i.e., $b^2-4ac < 0$), the decomposition will include a term of the form:
    \[ \frac{Ax+B}{ax^2+bx+c} \]
    Note the linear numerator, not just a constant.
    \item \textbf{Case 4: Repeated Irreducible Quadratic Factors.} If the denominator has a factor $(ax^2+bx+c)^k$ where $k \ge 2$, the decomposition must include a term for each power of the factor, from 1 to $k$:
    \[ \frac{A_1x+B_1}{ax^2+bx+c} + \frac{A_2x+B_2}{(ax^2+bx+c)^2} + \dots + \frac{A_kx+B_k}{(ax^2+bx+c)^k} \]
\end{enumerate}

\section{Intuition and Derivation}
The "why" behind partial fractions is best understood by thinking about the reverse process: adding fractions. To add $\frac{2}{x-1} + \frac{3}{x+5}$, we find a common denominator:
\[ \frac{2(x+5)}{(x-1)(x+5)} + \frac{3(x-1)}{(x-1)(x+5)} = \frac{2x+10+3x-3}{(x-1)(x+5)} = \frac{5x+7}{x^2+4x-5} \]
Partial fraction decomposition is simply the assertion that this process can be reversed. If we start with $\frac{5x+7}{x^2+4x-5}$, we can deduce that it must have come from simpler fractions whose denominators were the factors of $x^2+4x-5$, namely $(x-1)$ and $(x+5)$. We assume it came from a sum of the form $\frac{A}{x-1} + \frac{B}{x+5}$, and our task is to find the numerators, $A$ and $B$, that make the equation true. This algebraic restructuring is powerful because while integrating the combined fraction is difficult, integrating the sum of the simple parts is straightforward.

\section{Historical Context and Motivation}
The method of partial fractions was developed in the early 18th century, with significant contributions from Johann Bernoulli and Gottfried Wilhelm Leibniz. The motivation was precisely the problem we face in this chapter: the integration of rational functions. Before this, calculus could handle polynomials and some basic transcendental functions, but a general method for fractions of polynomials was missing. This was a major roadblock, as rational functions appeared frequently in the study of mechanics and celestial motion. By developing this algebraic technique, mathematicians unlocked the ability to integrate a vast new class of functions, which was a critical step in expanding the applicability of calculus to solve real-world physics and engineering problems.

\section{Key Formulas (Procedural Rules)}
The "formulas" for this topic are the rules for setting up the decomposition and the resulting integrals.

\subsection{Decomposition Forms}
\begin{itemize}
    \item Factor $(x-a)$: Term is $\frac{A}{x-a}$
    \item Factor $(x-a)^k$: Terms are $\frac{A_1}{x-a} + \frac{A_2}{(x-a)^2} + \dots + \frac{A_k}{(x-a)^k}$
    \item Factor $(x^2+a^2)$: Term is $\frac{Ax+B}{x^2+a^2}$
    \item Factor $(x^2+a^2)^k$: Terms are $\frac{A_1x+B_1}{x^2+a^2} + \dots + \frac{A_kx+B_k}{(x^2+a^2)^k}$
\end{itemize}

\subsection{Resulting Integral Forms}
After decomposition, you will almost always face integrals of these three types:
\begin{enumerate}
    \item $\displaystyle \int \frac{A}{ax+b} \,dx = \frac{A}{a} \ln|ax+b| + C$
    \item $\displaystyle \int \frac{A}{(ax+b)^k} \,dx \quad (k>1) = \frac{A}{a(1-k)}(ax+b)^{1-k} + C$
    \item $\displaystyle \int \frac{Ax+B}{x^2+a^2} \,dx = \int \frac{Ax}{x^2+a^2} \,dx + \int \frac{B}{x^2+a^2} \,dx = \frac{A}{2}\ln(x^2+a^2) + \frac{B}{a}\arctan\left(\frac{x}{a}\right) + C$
\end{enumerate}

\section{Prerequisites}
Mastery of this topic is impossible without a strong foundation in several key prerequisite skills:
\begin{itemize}
    \item \textbf{Polynomial Factoring:} You must be able to factor the denominator completely. This includes techniques like finding rational roots, difference of squares, sum/difference of cubes, and grouping.
    \item \textbf{Polynomial Long Division:} This is non-negotiable. It is the very first step for any improper rational function (where degree of numerator $\ge$ degree of denominator).
    \item \textbf{Solving Systems of Linear Equations:} After setting up the decomposition, you will generate a system of equations to solve for the unknown coefficients ($A, B, C, \dots$).
    \item \textbf{Completing the Square:} This is essential for handling irreducible quadratic factors, to get them into the form $(x-h)^2+k^2$ which is suitable for an arctangent integral.
    \item \textbf{Basic Integration Rules:} Specifically, the integrals for $\frac{1}{u}$ (logarithm), $u^n$ (power rule), and $\frac{1}{u^2+a^2}$ (arctangent).
\end{itemize}

\part{Detailed Homework Solutions}
\textit{Note: The answers shown in the homework images contained numerous typos. The solutions below are derived from scratch and are correct.}

\section{Problem 1}
\textbf{Write out the form of the partial fraction decomposition. Do not determine the numerical values.}
\subsection*{(a) $\displaystyle\frac{x-42}{x^2+x-42}$}
\begin{enumerate}
    \item \textbf{Factor the denominator:} $x^2+x-42 = (x+7)(x-6)$.
    \item \textbf{Identify factor type:} We have two distinct linear factors.
    \item \textbf{Write the form:}
    \[ \frac{A}{x+7} + \frac{B}{x-6} \]
\end{enumerate}

\subsection*{(b) $\displaystyle\frac{1}{x^2+x^4}$}
\begin{enumerate}
    \item \textbf{Factor the denominator:} $x^4+x^2 = x^2(x^2+1)$.
    \item \textbf{Identify factor type:} We have a repeated linear factor ($x^2$, which is $(x-0)^2$) and an irreducible quadratic factor ($x^2+1$).
    \item \textbf{Write the form:}
    \[ \frac{A}{x} + \frac{B}{x^2} + \frac{Cx+D}{x^2+1} \]
\end{enumerate}

\section{Problem 2}
\textbf{Write out the form of the partial fraction decomposition. Do not determine the numerical values.}
\subsection*{(a) $\displaystyle\frac{x^5+36}{(x^2-x)(x^4+12x^2+36)}$}
\begin{enumerate}
    \item \textbf{Factor the denominator:} $(x^2-x)(x^4+12x^2+36) = x(x-1)(x^2+6)^2$.
    \item \textbf{Identify factor type:} We have two distinct linear factors ($x$ and $x-1$) and a repeated irreducible quadratic factor ($(x^2+6)^2$).
    \item \textbf{Write the form:}
    \[ \frac{A}{x} + \frac{B}{x-1} + \frac{Cx+D}{x^2+6} + \frac{Ex+F}{(x^2+6)^2} \]
\end{enumerate}

\subsection*{(b) $\displaystyle\frac{x^2}{x^2+x-20}$}
\begin{enumerate}
    \item \textbf{Check degrees:} The degree of the numerator (2) is equal to the degree of the denominator (2). This is an improper fraction, so we must perform long division first.
    \[
    \begin{array}{r} 1 \\ x^2+x-20 \overline{) x^2+0x+0} \\ - (x^2+x-20) \\ \hline -x+20 \end{array}
    \]
    So, $\displaystyle\frac{x^2}{x^2+x-20} = 1 + \frac{-x+20}{x^2+x-20}$.
    \item \textbf{Factor the denominator of the remainder:} $x^2+x-20 = (x+5)(x-4)$. These are distinct linear factors.
    \item \textbf{Write the final form:}
    \[ 1 + \frac{A}{x+5} + \frac{B}{x-4} \]
\end{enumerate}

\section{Problem 3}
\textbf{Write out the form of the partial fraction decomposition. Do not determine the numerical values.}
\subsection*{(a) $\displaystyle\frac{x^6}{x^2-4}$}
\begin{enumerate}
    \item \textbf{Check degrees:} Degree of numerator (6) > degree of denominator (2). Perform long division.
    \[
    \frac{x^6}{x^2-4} = \frac{x^6-4x^4+4x^4-16x^2+16x^2-64+64}{x^2-4} = x^4+4x^2+16+\frac{64}{x^2-4}
    \]
    \item \textbf{Factor the denominator of the remainder:} $x^2-4 = (x-2)(x+2)$. Distinct linear factors.
    \item \textbf{Write the final form:}
    \[ x^4+4x^2+16 + \frac{A}{x-2} + \frac{B}{x+2} \]
\end{enumerate}

\subsection*{(b) $\displaystyle\frac{x^4}{(x^2-x+1)(x^2+4)^2}$}
\begin{enumerate}
    \item \textbf{Check degrees:} Degree of numerator (4) < degree of denominator (1+4=5). This is a proper fraction.
    \item \textbf{Factor the denominator:} The denominator is already factored.
    \item \textbf{Identify factor type:} We have a distinct irreducible quadratic factor ($x^2-x+1$, since $b^2-4ac = (-1)^2-4(1)(1) = -3 < 0$) and a repeated irreducible quadratic factor ($(x^2+4)^2$).
    \item \textbf{Write the form:}
    \[ \frac{Ax+B}{x^2-x+1} + \frac{Cx+D}{x^2+4} + \frac{Ex+F}{(x^2+4)^2} \]
\end{enumerate}

\section{Problem 4}
\textbf{Evaluate the integral $\displaystyle\int \frac{3}{(x-1)(x+2)} \,dx$.}
\begin{enumerate}
    \item \textbf{Decomposition Setup:} $\displaystyle\frac{3}{(x-1)(x+2)} = \frac{A}{x-1} + \frac{B}{x+2}$.
    \item \textbf{Solve for coefficients:} Multiply by the common denominator: $3 = A(x+2) + B(x-1)$.
    \begin{itemize}
        \item Let $x=1$: $3 = A(1+2) \implies 3 = 3A \implies A=1$.
        \item Let $x=-2$: $3 = B(-2-1) \implies 3 = -3B \implies B=-1$.
    \end{itemize}
    \item \textbf{Integrate:}
    \begin{align*}
    \int \left( \frac{1}{x-1} - \frac{1}{x+2} \right) \,dx &= \ln|x-1| - \ln|x+2| + C \\
    &= \ln\left|\frac{x-1}{x+2}\right| + C
    \end{align*}
\end{enumerate}
\textbf{Final Answer:} $\displaystyle\ln\left|\frac{x-1}{x+2}\right| + C$

\section{Problem 5}
\textbf{Evaluate the integral $\displaystyle\int \frac{82x+8}{(9x+1)(x-1)} \,dx$.}
\begin{enumerate}
    \item \textbf{Decomposition Setup:} $\displaystyle\frac{82x+8}{(9x+1)(x-1)} = \frac{A}{9x+1} + \frac{B}{x-1}$.
    \item \textbf{Solve for coefficients:} $82x+8 = A(x-1) + B(9x+1)$.
    \begin{itemize}
        \item Let $x=1$: $82(1)+8 = B(9(1)+1) \implies 90 = 10B \implies B=9$.
        \item Let $x=-1/9$: $82(-1/9)+8 = A(-1/9-1) \implies -82/9+72/9 = A(-10/9) \implies -10/9 = -10/9 A \implies A=1$.
    \end{itemize}
    \item \textbf{Integrate:}
    \begin{align*}
    \int \left( \frac{1}{9x+1} + \frac{9}{x-1} \right) \,dx &= \frac{1}{9}\ln|9x+1| + 9\ln|x-1| + C
    \end{align*}
\end{enumerate}
\textbf{Final Answer:} $\displaystyle\frac{1}{9}\ln|9x+1| + 9\ln|x-1| + C$

\section{Problem 6}
\textbf{Evaluate the integral $\displaystyle\int \frac{5t-2}{t+1} \,dt$.}
\begin{enumerate}
    \item \textbf{Long Division:} Degree of numerator (1) = degree of denominator (1).
    \[ \frac{5t-2}{t+1} = \frac{5(t+1) - 5 - 2}{t+1} = 5 - \frac{7}{t+1} \]
    \item \textbf{Integrate:}
    \[ \int \left( 5 - \frac{7}{t+1} \right) \,dt = 5t - 7\ln|t+1| + C \]
\end{enumerate}
\textbf{Final Answer:} $5t - 7\ln|t+1| + C$

\section{Problem 7}
\textbf{Evaluate the integral $\displaystyle\int \frac{2}{(t^2-4)^2} \,dt$.}
\begin{enumerate}
    \item \textbf{Decomposition Setup:} $\displaystyle\frac{2}{((t-2)(t+2))^2} = \frac{2}{(t-2)^2(t+2)^2} = \frac{A}{t-2} + \frac{B}{(t-2)^2} + \frac{C}{t+2} + \frac{D}{(t+2)^2}$.
    \item \textbf{Solve for coefficients:} $2 = A(t-2)(t+2)^2 + B(t+2)^2 + C(t+2)(t-2)^2 + D(t-2)^2$.
    \begin{itemize}
        \item Let $t=2$: $2 = B(4)^2 \implies B=2/16=1/8$.
        \item Let $t=-2$: $2 = D(-4)^2 \implies D=2/16=1/8$.
        \item Equate coefficients of highest power ($t^3$): $0 = At^3 + Ct^3 \implies A+C=0 \implies C=-A$.
        \item Let $t=0$: $2 = A(-2)(2)^2 + B(2)^2 + C(2)(-2)^2 + D(-2)^2 = -8A+4B+8C+4D$.
        \item Substitute known values: $2 = -8A + 4(1/8) + 8C + 4(1/8) = -8A+1/2+8C+1/2 \implies 1 = -8A+8C$.
        \item Substitute $C=-A$: $1 = -8A+8(-A) \implies 1 = -16A \implies A=-1/16$.
        \item Then $C = 1/16$.
    \end{itemize}
    \item \textbf{Integrate:}
    \begin{align*}
    \int \left( \frac{-1/16}{t-2} + \frac{1/8}{(t-2)^2} + \frac{1/16}{t+2} + \frac{1/8}{(t+2)^2} \right) \,dt \\
    &= -\frac{1}{16}\ln|t-2| - \frac{1}{8(t-2)} + \frac{1}{16}\ln|t+2| - \frac{1}{8(t+2)} + C \\
    &= \frac{1}{16}\ln\left|\frac{t+2}{t-2}\right| - \frac{1}{8}\left(\frac{1}{t-2} + \frac{1}{t+2}\right) + C \\
    &= \frac{1}{16}\ln\left|\frac{t+2}{t-2}\right| - \frac{1}{8}\left(\frac{t+2+t-2}{t^2-4}\right) + C \\
    &= \frac{1}{16}\ln\left|\frac{t+2}{t-2}\right| - \frac{2t}{8(t^2-4)} + C = \frac{1}{16}\ln\left|\frac{t+2}{t-2}\right| - \frac{t}{4(t^2-4)} + C
    \end{align*}
\end{enumerate}
\textbf{Final Answer:} $\displaystyle\frac{1}{16}\ln\left|\frac{t+2}{t-2}\right| - \frac{t}{4(t^2-4)} + C$

\section{Problem 8}
\textbf{Evaluate the integral $\displaystyle\int \frac{17}{(x-1)(x^2+16)} \,dx$.}
\begin{enumerate}
    \item \textbf{Decomposition Setup:} $\displaystyle\frac{17}{(x-1)(x^2+16)} = \frac{A}{x-1} + \frac{Bx+C}{x^2+16}$.
    \item \textbf{Solve for coefficients:} $17 = A(x^2+16) + (Bx+C)(x-1)$.
    \begin{itemize}
        \item Let $x=1$: $17 = A(1+16) \implies A=1$.
        \item Expand: $17 = (x^2+16) + (Bx^2-Bx+Cx-C) = (1+B)x^2 + (-B+C)x + (16-C)$.
        \item Equate coefficients:
        \begin{itemize}
            \item $x^2$: $0 = 1+B \implies B=-1$.
            \item $x$: $0 = -B+C \implies C=B \implies C=-1$.
            \item Constant: $17 = 16-C \implies C=-1$. (Consistent)
        \end{itemize}
    \end{itemize}
    \item \textbf{Integrate:}
    \begin{align*}
    \int \left( \frac{1}{x-1} + \frac{-x-1}{x^2+16} \right) \,dx &= \int \frac{1}{x-1} \,dx - \int \frac{x}{x^2+16} \,dx - \int \frac{1}{x^2+16} \,dx \\
    &= \ln|x-1| - \frac{1}{2}\ln(x^2+16) - \frac{1}{4}\arctan\left(\frac{x}{4}\right) + C
    \end{align*}
\end{enumerate}
\textbf{Final Answer:} $\displaystyle\ln|x-1| - \frac{1}{2}\ln(x^2+16) - \frac{1}{4}\arctan\left(\frac{x}{4}\right) + C$

\section{Problem 9}
\textbf{Evaluate the integral $\displaystyle\int \frac{x^2-x+12}{x^3+2x} \,dx$.}
\begin{enumerate}
    \item \textbf{Decomposition Setup:} $\displaystyle\frac{x^2-x+12}{x(x^2+2)} = \frac{A}{x} + \frac{Bx+C}{x^2+2}$.
    \item \textbf{Solve for coefficients:} $x^2-x+12 = A(x^2+2) + (Bx+C)x$.
    \begin{itemize}
        \item Let $x=0$: $12 = A(2) \implies A=6$.
        \item Expand: $x^2-x+12 = (Ax^2+2A) + (Bx^2+Cx) = (A+B)x^2 + Cx + 2A$.
        \item Equate coefficients:
        \begin{itemize}
            \item $x^2$: $1 = A+B \implies 1 = 6+B \implies B=-5$.
            \item $x$: $-1 = C$.
            \item Constant: $12=2A \implies A=6$. (Consistent)
        \end{itemize}
    \end{itemize}
    \item \textbf{Integrate:}
    \begin{align*}
    \int \left( \frac{6}{x} + \frac{-5x-1}{x^2+2} \right) \,dx &= \int \frac{6}{x} \,dx - \int \frac{5x}{x^2+2} \,dx - \int \frac{1}{x^2+2} \,dx \\
    &= 6\ln|x| - \frac{5}{2}\ln(x^2+2) - \frac{1}{\sqrt{2}}\arctan\left(\frac{x}{\sqrt{2}}\right) + C
    \end{align*}
\end{enumerate}
\textbf{Final Answer:} $\displaystyle 6\ln|x| - \frac{5}{2}\ln(x^2+2) - \frac{1}{\sqrt{2}}\arctan\left(\frac{x}{\sqrt{2}}\right) + C$

\section{Problem 10}
\textbf{Evaluate the integral $\displaystyle\int \frac{5x^2+x+5}{(x^2+1)^2} \,dx$.}
\begin{enumerate}
    \item \textbf{Decomposition Setup:} $\displaystyle\frac{5x^2+x+5}{(x^2+1)^2} = \frac{Ax+B}{x^2+1} + \frac{Cx+D}{(x^2+1)^2}$.
    \item \textbf{Solve for coefficients:} $5x^2+x+5 = (Ax+B)(x^2+1) + (Cx+D) = Ax^3+Bx^2+(A+C)x+(B+D)$.
    \begin{itemize}
        \item Equate coefficients:
        \begin{itemize}
            \item $x^3$: $A=0$.
            \item $x^2$: $B=5$.
            \item $x$: $A+C=1 \implies 0+C=1 \implies C=1$.
            \item Constant: $B+D=5 \implies 5+D=5 \implies D=0$.
        \end{itemize}
    \end{itemize}
    \item \textbf{Integrate:}
    \begin{align*}
    \int \left( \frac{5}{x^2+1} + \frac{x}{(x^2+1)^2} \right) \,dx &= \int \frac{5}{x^2+1} \,dx + \int \frac{x}{(x^2+1)^2} \,dx \\
    \text{For the second integral, let } u=x^2+1, du=2x\,dx \\
    &= 5\arctan(x) + \frac{1}{2}\int\frac{1}{u^2}\,du \\
    &= 5\arctan(x) - \frac{1}{2u} + C \\
    &= 5\arctan(x) - \frac{1}{2(x^2+1)} + C
    \end{align*}
\end{enumerate}
\textbf{Final Answer:} $\displaystyle 5\arctan(x) - \frac{1}{2(x^2+1)} + C$

\section{Problem 11}
\textbf{Evaluate the integral $\displaystyle\int \frac{x+12}{x^2+14x+53} \,dx$.}
\begin{enumerate}
    \item \textbf{Analyze denominator:} $b^2-4ac = 14^2 - 4(1)(53) = 196-212 = -16 < 0$. It's an irreducible quadratic. No decomposition needed.
    \item \textbf{Complete the square:} $x^2+14x+53 = (x^2+14x+49) - 49 + 53 = (x+7)^2 + 4$.
    \item \textbf{Rewrite and substitute:}
    \[ \int \frac{x+12}{(x+7)^2+4} \,dx \]
    Let $u = x+7$, so $du=dx$ and $x=u-7$. The numerator becomes $(u-7)+12 = u+5$.
    \item \textbf{Integrate:}
    \begin{align*}
    \int \frac{u+5}{u^2+4} \,du &= \int \frac{u}{u^2+4} \,du + \int \frac{5}{u^2+4} \,du \\
    &= \frac{1}{2}\ln(u^2+4) + 5 \cdot \frac{1}{2}\arctan\left(\frac{u}{2}\right) + C \\
    &= \frac{1}{2}\ln((x+7)^2+4) + \frac{5}{2}\arctan\left(\frac{x+7}{2}\right) + C \\
    &= \frac{1}{2}\ln(x^2+14x+53) + \frac{5}{2}\arctan\left(\frac{x+7}{2}\right) + C
    \end{align*}
\end{enumerate}
\textbf{Final Answer:} $\displaystyle\frac{1}{2}\ln(x^2+14x+53) + \frac{5}{2}\arctan\left(\frac{x+7}{2}\right) + C$

\section{Problem 12}
\textbf{Evaluate the integral $\displaystyle\int_0^1 \frac{x^3+3x}{x^4+6x^2+3} \,dx$.}
\begin{enumerate}
    \item \textbf{Check for simple substitution:} Notice that the derivative of the denominator is $4x^3+12x = 4(x^3+3x)$, which is a multiple of the numerator. This problem is best solved with u-substitution, not partial fractions.
    \item \textbf{Substitute:} Let $u=x^4+6x^2+3$, so $du = (4x^3+12x)dx = 4(x^3+3x)dx$.
    This means $(x^3+3x)dx = \frac{1}{4}du$.
    \item \textbf{Change bounds:}
    \begin{itemize}
        \item When $x=0$, $u = 0^4+6(0)^2+3 = 3$.
        \item When $x=1$, $u = 1^4+6(1)^2+3 = 10$.
    \end{itemize}
    \item \textbf{Integrate:}
    \begin{align*}
    \int_3^{10} \frac{1}{u} \cdot \frac{1}{4}du &= \frac{1}{4} \left[ \ln|u| \right]_3^{10} \\
    &= \frac{1}{4}(\ln(10) - \ln(3)) = \frac{1}{4}\ln\left(\frac{10}{3}\right)
    \end{align*}
\end{enumerate}
\textbf{Final Answer:} $\displaystyle\frac{1}{4}\ln\left(\frac{10}{3}\right)$

\section{Problem 13}
\textbf{Evaluate the integral $\displaystyle\int_0^1 \frac{2}{2x^2+3x+1} \,dx$.}
\begin{enumerate}
    \item \textbf{Decomposition Setup:} Factor denominator $2x^2+3x+1=(2x+1)(x+1)$.
    \[ \frac{2}{(2x+1)(x+1)} = \frac{A}{2x+1} + \frac{B}{x+1} \]
    \item \textbf{Solve for coefficients:} $2 = A(x+1) + B(2x+1)$.
    \begin{itemize}
        \item Let $x=-1$: $2=B(2(-1)+1) \implies 2=-B \implies B=-2$.
        \item Let $x=-1/2$: $2=A(-1/2+1) \implies 2=A(1/2) \implies A=4$.
    \end{itemize}
    \item \textbf{Integrate:}
    \begin{align*}
    \int_0^1 \left( \frac{4}{2x+1} - \frac{2}{x+1} \right) \,dx &= \left[ 4\cdot\frac{1}{2}\ln|2x+1| - 2\ln|x+1| \right]_0^1 \\
    &= \left[ 2\ln|2x+1| - 2\ln|x+1| \right]_0^1 \\
    &= (2\ln(3) - 2\ln(2)) - (2\ln(1) - 2\ln(1)) \\
    &= 2(\ln(3)-\ln(2)) - 0 = 2\ln\left(\frac{3}{2}\right)
    \end{align*}
\end{enumerate}
\textbf{Final Answer:} $\displaystyle 2\ln\left(\frac{3}{2}\right)$

\section{Problem 14}
\textbf{Evaluate the integral $\displaystyle\int \frac{x^2}{x-5} \,dx$.}
\begin{enumerate}
    \item \textbf{Long Division:} Degree of numerator (2) > degree of denominator (1). Using synthetic division for $(x-5)$:
    \[
    \begin{array}{c|ccc}
    5 & 1 & 0 & 0 \\
      &   & 5 & 25 \\
    \hline
      & 1 & 5 & 25 \\
    \end{array}
    \]
    So, $\displaystyle\frac{x^2}{x-5} = x+5 + \frac{25}{x-5}$.
    \item \textbf{Integrate:}
    \[ \int \left( x+5 + \frac{25}{x-5} \right) \,dx = \frac{1}{2}x^2 + 5x + 25\ln|x-5| + C \]
\end{enumerate}
\textbf{Final Answer:} $\displaystyle\frac{1}{2}x^2 + 5x + 25\ln|x-5| + C$

\part{In-Depth Analysis of Problems and Techniques}
\subsection{Problem Types and General Approach}
\begin{itemize}
    \item \textbf{Setup Only Problems (1, 2, 3):} These problems test the foundational skill of correctly identifying factor types in the denominator and applying the corresponding decomposition rule. The key is to factor completely and remember the different numerator forms (constant $A$ for linear, linear $Ax+B$ for quadratic) and the need for multiple terms for repeated factors.
    \item \textbf{Improper Fractions (2b, 3a, 6, 14):} A critical first checkpoint. If the numerator's degree is greater than or equal to the denominator's, polynomial long division is mandatory. The general approach is: Divide, then apply partial fraction decomposition to the rational remainder.
    \item \textbf{Distinct Linear Factors (4, 5, 13):} This is the most straightforward case. The approach is to set up the decomposition with constant numerators, solve for the coefficients (the "Heaviside cover-up method" is very efficient here), and integrate each term to get a sum/difference of logarithms.
    \item \textbf{Repeated Linear Factors (7):} This type is more algebraically intensive. The approach requires a term for each power of the repeated factor. Solving for coefficients usually requires a mix of the cover-up method for the coefficient of the highest power term and creating a system of equations for the rest. Integration yields both logarithmic and power-rule results.
    \item \textbf{Irreducible Quadratic Factors (8, 9):} The hallmark of this type is a quadratic in the denominator that cannot be factored. The approach is to use a linear numerator ($Ax+B$), solve for A and B, and then split the resulting fraction. The part with $Ax$ is solved with a u-substitution (yielding a logarithm), and the part with $B$ is solved using the arctangent formula.
    \item \textbf{Repeated Irreducible Quadratic Factors (10):} This combines the complexity of repeated factors with irreducible quadratics. The setup requires a separate `(Ax+B)`-style numerator for each power of the quadratic factor. The integration can be complex, but in Problem 10, it simplified nicely into one arctangent and one u-substitution.
    \item \textbf{Problems Requiring Completing the Square (11):} This occurs when an irreducible quadratic is not in the standard $x^2+a^2$ form. The approach is to first complete the square to get it into the form $(x-h)^2+k^2$, then use a u-substitution ($u=x-h$), and finally proceed as in the standard irreducible quadratic case.
    \item \textbf{"Trap" Problems (12):} This problem looked like a candidate for partial fractions but was much more easily solved by a simple u-substitution. The general lesson is to always look for simpler methods before committing to a long algebraic procedure. Check if the numerator is related to the derivative of the denominator.
\end{itemize}

\subsection{Key Algebraic and Calculus Manipulations}
\begin{itemize}
    \item \textbf{Polynomial Long Division:} Used in Problems 2b, 3a, 6, and 14. This was crucial because the partial fraction decomposition rules only apply to proper fractions.
    \item \textbf{Factoring Completely:} Used in all setup problems (1, 2, 3) and solution problems (4, 5, 7, 9, 13). This is the absolute first step and determines the entire structure of the problem.
    \item \textbf{Heaviside Cover-Up Method:} A shortcut for finding coefficients for distinct linear factors. For a term $\frac{A}{x-a}$, cover the $(x-a)$ factor in the original fraction and substitute $x=a$ into the rest. This was applicable in problems 4, 5, and 13 to find coefficients quickly.
    \item \textbf{Equating Coefficients:} A more robust method for finding coefficients that works in all cases. After clearing the denominator, expand all terms and create a system of equations by equating the coefficients of like powers of $x$. This was necessary for problems with repeated or quadratic factors like 7, 8, 9, and 10.
    \item \textbf{Completing the Square:} Used in Problem 11 to transform $x^2+14x+53$ into $(x+7)^2+4$, making it ready for an arctangent integration. This is the standard technique for any irreducible quadratic that has a $bx$ term.
    \item \textbf{Splitting the Numerator:} Used in problems 8, 9, and 11. Once you have a term like $\frac{Ax+B}{(x-h)^2+k^2}$, it must be split into two separate integrals: $\int\frac{A(x-h)}{(x-h)^2+k^2}dx$ (for the logarithm part) and $\int\frac{B'}{(x-h)^2+k^2}dx$ (for the arctangent part).
    \item \textbf{U-Substitution:} Appeared in two contexts: as the primary solution method (Problem 12) and as a sub-step for integrating parts of a decomposed fraction, specifically for the logarithmic part of irreducible quadratics (Problems 8, 9, 11) and for higher-power terms (Problem 10).
\end{itemize}

\part{"Cheatsheet" and Tips for Success}
\subsection{Summary of Most Important Formulas/Forms}
\begin{center}
\begin{tabular}{|l|l|l|}
\hline
\textbf{Factor in Denominator} & \textbf{Decomposition Term(s)} & \textbf{Integral Form} \\
\hline
$(x-r)$ & $\displaystyle\frac{A}{x-r}$ & $A\ln|x-r|$ \\
\hline
$(x-r)^k$ & $\displaystyle\sum_{i=1}^k \frac{A_i}{(x-r)^i}$ & Logarithm + Power Rule \\
\hline
$(x^2+a^2)$ & $\displaystyle\frac{Ax+B}{x^2+a^2}$ & Logarithm + Arctangent \\
\hline
$(x^2+a^2)^k$ & $\displaystyle\sum_{i=1}^k \frac{A_ix+B_i}{(x^2+a^2)^i}$ & Complex (Often requires Trig Sub) \\
\hline
\end{tabular}
\end{center}
\textbf{Key Integrals to Memorize:}
\[ \int \frac{1}{u} \,du = \ln|u| + C \quad \quad \int \frac{1}{u^2+a^2} \,du = \frac{1}{a}\arctan\left(\frac{u}{a}\right) + C \]

\subsection{Cheats, Tricks, and Shortcuts}
\begin{itemize}
    \item \textbf{Heaviside Cover-Up:} For distinct linear factors $(x-r)$, find coefficient $A$ by covering $(x-r)$ in the original fraction and plugging $x=r$ into what's left. It's much faster than setting up a full system of equations.
    \item \textbf{Check for U-Sub First:} Before starting a complex decomposition, always check if the numerator is a multiple of the denominator's derivative (like in Problem 12). This can save you 15 minutes of algebra.
    \item \textbf{Strategic Substitution:} When solving for coefficients, plugging in "smart" values of $x$ (like the roots of the factors) can simplify the equations immensely, even if you still need to equate coefficients for the rest.
\end{itemize}

\subsection{Common Pitfalls and Mistakes}
\begin{itemize}
    \item \textbf{Forgetting Long Division:} The #1 mistake. If the fraction is improper, you MUST divide first.
    \item \textbf{Incorrect Decomposition Form:} Using $A$ over a quadratic instead of $Ax+B$. Forgetting to include a term for every power of a repeated factor.
    \item \textbf{Algebraic Errors:} Simple sign mistakes or distribution errors when solving for coefficients are extremely common and will throw off the entire problem. Work carefully.
    \item \textbf{Forgetting Absolute Values:} The integral of $1/u$ is $\ln|u|$, not $\ln(u)$. This is crucial for the domain. (Note: for terms like $\ln(x^2+a^2)$, absolute values are not needed as the argument is always positive).
    \item \textbf{Arctangent Errors:} Forgetting the $\frac{1}{a}$ coefficient in the arctan formula is a frequent mistake.
\end{itemize}

\subsection{How to Recognize Problem Types}
Recognition is 100\% about the denominator.
\begin{enumerate}
    \item Is the fraction proper? If not, divide.
    \item Factor the denominator completely.
    \item Examine the factors:
    \begin{itemize}
        \item Only distinct linear factors? It's the easiest type.
        \item Any repeated linear factors? Prepare for a larger system of equations.
        \item Any irreducible quadratics? Expect to use an $Ax+B$ numerator and prepare to get a log and an arctan.
        \item Any repeated quadratics? Prepare for the most algebraically intensive case.
    \end{itemize}
\end{enumerate}

\part{Conceptual Synthesis and The "Big Picture"}
\subsection{Thematic Connections}
The core theme of integration by partial fractions is \textbf{decomposition of complexity}. It's a powerful illustration of the mathematical strategy of taking a problem that looks unsolvable in its current form and breaking it down into a sum of smaller, manageable pieces that conform to known patterns.

This theme is universal in mathematics and science:
\begin{itemize}
    \item \textbf{Taylor Series:} We decompose a complex function into an infinite sum of simple power functions ($c_n(x-a)^n$).
    \item \textbf{Fourier Series:} We decompose a periodic signal into a sum of simple sines and cosines.
    \item \textbf{Vector Decomposition:} In physics, we break down a force or velocity vector into its orthogonal components ($\vec{F} = F_x\hat{i} + F_y\hat{j}$) to simplify analysis in each dimension.
\end{itemize}
In each case, a complex object is expressed as a linear combination of simpler "basis" elements. For partial fractions, our basis elements are simple rational functions whose integrals we have memorized.

\subsection{Forward and Backward Links}
\begin{itemize}
    \item \textbf{Backward Links (Foundations):} This topic is a culminating point for many algebraic skills. It's a direct application of polynomial factoring, long division, and solving systems of equations. On the calculus side, it relies entirely on the mastery of the most basic integrals learned previously: the power rule, the integral of $1/u$, and the integral for arctangent. Without fluency in these prerequisites, this topic is inaccessible.

    \item \textbf{Forward Links (Applications):} The ability to integrate any rational function is a critical skill that serves as a foundation for more advanced topics:
    \begin{itemize}
        \item \textbf{Differential Equations:} Many first-order differential equations, especially those modeling real-world phenomena like logistic population growth, chemical reaction kinetics, or RC circuits, are solved by separating variables and integrating. This integration step frequently requires partial fractions.
        \item \textbf{Laplace Transforms:} In engineering and physics, the Laplace Transform is used to solve linear ordinary differential equations. The process involves transforming the equation into an algebraic problem, solving for the transformed function (which is often a rational function), and then applying the Inverse Laplace Transform. This inverse transform step often requires first decomposing the rational function using partial fractions.
        \item \textbf{Control Theory:} Analyzing the stability and response of systems in control engineering involves working with transfer functions, which are rational functions in the complex variable 's'. Partial fraction expansion is used to analyze the system's behavior.
    \end{itemize}
\end{itemize}

\part{Real-World Application and Modeling}
\subsection{Concrete Scenarios}
\begin{enumerate}
    \item \textbf{Pharmacokinetics:} When a drug is administered, its concentration in the bloodstream changes over time. Models for this process, such as a two-compartment model (blood plasma and body tissue), often lead to systems of differential equations. Solving for the drug concentration at time $t$ requires integrating a rational function of time, which is done using partial fractions. This allows pharmacologists to predict how long a drug will remain effective and to schedule dosing intervals.
    \item \textbf{Environmental Engineering (Pollutant Cleanup):} Consider a lake with a pollutant that is being removed, while fresh water flows in. The rate of change of the pollutant concentration can be modeled by a differential equation. If the rates depend on the current concentration in a complex way (e.g., the breakdown process is a second-order reaction), solving for the time it takes to reach a safe pollutant level involves an integral solved by partial fractions.
    \item \textbf{Population Dynamics (Logistic Growth):} The simple exponential growth model is unrealistic as it assumes infinite resources. The logistic model introduces a "carrying capacity" $K$. The rate of population growth is proportional to both the current population $P$ and the remaining capacity $(K-P)$. This leads to the differential equation $\frac{dP}{dt} = rP(1-\frac{P}{K})$. To find the population $P$ as a function of time $t$, one must integrate $\frac{1}{P(1-P/K)}$, a classic partial fractions problem.
\end{enumerate}

\subsection{Model Problem Setup}
Let's set up the \textbf{logistic growth model} for a fish population.

\begin{itemize}
    \item \textbf{Scenario:} A small lake is stocked with an initial population of $P_0 = 100$ fish. The lake has a carrying capacity of $K=1000$ fish. The intrinsic growth rate of the fish population is $r=0.2$ per year. We want to find a formula for the population $P(t)$ at any time $t$.
    \item \textbf{Variables:}
    \begin{itemize}
        \item $P(t)$: fish population at time $t$ (in years).
        \item $t$: time in years.
        \item $K = 1000$: carrying capacity.
        \item $r = 0.2$: intrinsic growth rate.
        \item $P_0 = 100$: initial population.
    \end{itemize}
    \item \textbf{Governing Equation:} The logistic differential equation is $\displaystyle \frac{dP}{dt} = rP\left(1 - \frac{P}{K}\right)$.
    \item \textbf{Mathematical Setup:} To solve for $P(t)$, we separate the variables:
    \[ \frac{dP}{P(1 - P/K)} = r \,dt \]
    \[ \frac{dP}{P\frac{K-P}{K}} = r \,dt \]
    \[ \int \frac{K}{P(K-P)} \,dP = \int r \,dt \]
    To solve the left-hand integral, we must use partial fraction decomposition on the integrand $\frac{K}{P(K-P)}$. We would set up:
    \[ \frac{K}{P(K-P)} = \frac{A}{P} + \frac{B}{K-P} \]
    Solving this integral provides the function $P(t)$ that models the fish population over time.
\end{itemize}

\part{Common Variations and Untested Concepts}
The provided homework set was very thorough, covering all four standard cases of partial fraction decomposition. However, one common and important technique that often arises from repeated irreducible quadratic factors was not explicitly required: \textbf{integration using trigonometric substitution as a sub-step}.

The integral in Problem 10, $\int \frac{x}{(x^2+1)^2} dx$, was solvable with a simple u-substitution. A more general problem of this type, which is a classic calculus II problem, requires more work.

\subsubsection*{Untested Concept: Repeated Irreducible Quadratic leading to Trig Substitution}
Sometimes, a term arises from decomposition for which a simple u-substitution is not available.
\textbf{Worked Example:} Evaluate $\displaystyle \int \frac{1}{(x^2+1)^2} \,dx$.

This integral might result from the decomposition of a more complex fraction. It cannot be solved with a simple u-substitution.
\begin{enumerate}
    \item \textbf{Recognize the Form:} The form $(x^2+a^2)^k$ is a strong indicator for trigonometric substitution. Here $a=1$.
    \item \textbf{Perform Trig Substitution:}
    Let $x = \tan(\theta)$. Then $dx = \sec^2(\theta) \,d\theta$.
    The denominator becomes $(x^2+1)^2 = (\tan^2(\theta)+1)^2 = (\sec^2(\theta))^2 = \sec^4(\theta)$.
    \item \textbf{Substitute and Simplify:}
    \[ \int \frac{1}{\sec^4(\theta)} \cdot \sec^2(\theta) \,d\theta = \int \frac{1}{\sec^2(\theta)} \,d\theta = \int \cos^2(\theta) \,d\theta \]
    \item \textbf{Integrate using Half-Angle Identity:}
    \[ \int \frac{1}{2}(1 + \cos(2\theta)) \,d\theta = \frac{1}{2}\left(\theta + \frac{1}{2}\sin(2\theta)\right) + C = \frac{1}{2}(\theta + \sin(\theta)\cos(\theta)) + C \]
    \item \textbf{Convert Back to x:}
    We used $x = \tan(\theta)$, so $\theta = \arctan(x)$. We can form a right triangle where the opposite side is $x$ and the adjacent side is 1. The hypotenuse is $\sqrt{x^2+1}$.
    From the triangle, $\sin(\theta) = \frac{x}{\sqrt{x^2+1}}$ and $\cos(\theta) = \frac{1}{\sqrt{x^2+1}}$.
    \item \textbf{Final Answer:}
    \[ \frac{1}{2}\left(\arctan(x) + \frac{x}{\sqrt{x^2+1}}\frac{1}{\sqrt{x^2+1}}\right) + C = \frac{1}{2}\left(\arctan(x) + \frac{x}{x^2+1}\right) + C \]
\end{enumerate}
This type of problem represents a deeper integration of multiple techniques: partial fractions to isolate the term, and trigonometric substitution to solve the resulting integral.

\part{Advanced Diagnostic Testing: "Find the Flaw"}
Below are five problems with complete solutions. Each solution contains one subtle but critical error. Your task is to find the flaw, explain it, and provide the correct step and final answer.

\subsection{Problem 1}
Evaluate $\displaystyle \int \frac{x^3+2x^2-1}{x^2-4} \,dx$.
\subsubsection*{Flawed Solution}
The denominator factors as $(x-2)(x+2)$. We set up the decomposition:
\[ \frac{x^3+2x^2-1}{(x-2)(x+2)} = \frac{A}{x-2} + \frac{B}{x+2} \]
Clear the denominator: $x^3+2x^2-1 = A(x+2) + B(x-2)$.
Let $x=2$: $8+8-1 = A(4) \implies 15 = 4A \implies A=15/4$.
Let $x=-2$: $-8+8-1 = B(-4) \implies -1 = -4B \implies B=1/4$.
The integral is:
\[ \int \left(\frac{15/4}{x-2} + \frac{1/4}{x+2}\right) \,dx = \frac{15}{4}\ln|x-2| + \frac{1}{4}\ln|x+2| + C \]
\textbf{Final Answer:} $\displaystyle \frac{15}{4}\ln|x-2| + \frac{1}{4}\ln|x+2| + C$

\subsection{Problem 2}
Evaluate $\displaystyle \int \frac{x-5}{x^2+x-2} \,dx$.
\subsubsection*{Flawed Solution}
The denominator factors as $(x+2)(x-1)$. Set up the decomposition:
\[ \frac{x-5}{(x+2)(x-1)} = \frac{A}{x+2} + \frac{B}{x-1} \]
Clear the denominator: $x-5 = A(x-1) + B(x+2)$.
Let $x=1$: $1-5 = B(1+2) \implies -4 = 3B \implies B=-4/3$.
Let $x=-2$: $-2-5 = A(-2-1) \implies -7 = -3A \implies A=7/3$.
The integral is:
\[ \int \left(\frac{7/3}{x+2} - \frac{4/3}{x-1}\right) \,dx = \frac{7}{3}\ln|x+2| + \frac{4}{3}\ln|x-1| + C \]
\textbf{Final Answer:} $\displaystyle \frac{7}{3}\ln|x+2| + \frac{4}{3}\ln|x-1| + C$

\subsection{Problem 3}
Evaluate $\displaystyle \int \frac{5x^2+1}{x(x^2+1)} \,dx$.
\subsubsection*{Flawed Solution}
The denominator is factored. The term $x^2+1$ is an irreducible quadratic. Set up the decomposition:
\[ \frac{5x^2+1}{x(x^2+1)} = \frac{A}{x} + \frac{B}{x^2+1} \]
Clear the denominator: $5x^2+1 = A(x^2+1) + Bx$.
Let $x=0$: $1 = A(1) \implies A=1$.
Substitute $A=1$: $5x^2+1 = 1(x^2+1) + Bx \implies 5x^2+1 = x^2+1+Bx \implies 4x^2 = Bx$.
This gives $B=4x$, which is not a constant. Something is wrong, but we'll proceed by equating coefficients.
$5x^2+0x+1 = Ax^2+Bx+A$.
$x^2$: $A=5$. $x$: $B=0$. Constant: $A=1$. This is a contradiction. The setup must be wrong, but let's assume $A=1, B=4x$ was okay.
\[ \int \left(\frac{1}{x} + \frac{4x}{x^2+1}\right) \,dx = \ln|x| + 2\ln(x^2+1) + C \]
\textbf{Final Answer:} $\displaystyle \ln|x| + 2\ln(x^2+1) + C$

\subsection{Problem 4}
Evaluate $\displaystyle \int \frac{2x}{(x-1)^2} \,dx$.
\subsubsection*{Flawed Solution}
The denominator has a repeated linear factor. Set up the decomposition:
\[ \frac{2x}{(x-1)^2} = \frac{A}{x-1} + \frac{B}{(x-1)^2} \]
Clear the denominator: $2x = A(x-1) + B$.
Let $x=1$: $2(1) = B \implies B=2$.
Equate coefficients of $x$: $2=A$.
The integral is:
\[ \int \left(\frac{2}{x-1} + \frac{2}{(x-1)^2}\right) \,dx = 2\ln|x-1| + 2\ln((x-1)^2) + C \]
\textbf{Final Answer:} $\displaystyle 2\ln|x-1| + 4\ln|x-1| + C = 6\ln|x-1| + C$

\subsection{Problem 5}
Evaluate $\displaystyle \int \frac{1}{x^2+9} \,dx$.
\subsubsection*{Flawed Solution}
This is an irreducible quadratic. The formula for this integral is $\arctan(u)$.
Let $u=x$, $a=3$. The integral is of the form $\int \frac{1}{u^2+a^2} \,du$.
The solution is $\arctan(x/3)+C$.
\textbf{Final Answer:} $\displaystyle \arctan(x/3) + C$


\end{document}