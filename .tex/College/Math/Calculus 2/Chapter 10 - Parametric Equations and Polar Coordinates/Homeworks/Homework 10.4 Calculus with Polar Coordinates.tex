\documentclass{article}
\usepackage{amsmath}
\usepackage{amssymb}
\usepackage{geometry}
\usepackage{graphicx}
\geometry{a4paper, margin=1in}

\title{Homework 10.4 Calculus with Polar Coordinates}
\author{Tashfeen Omran}
\date{October 2025}

\begin{document}

\maketitle

\part{Comprehensive Introduction, Context, and Prerequisites}

\section{Core Concepts}
The topic of calculus with polar coordinates extends the principles of differentiation and integration to a different coordinate system. While the Cartesian coordinate system identifies points using horizontal and vertical distances $(x, y)$, the polar coordinate system uses a distance and an angle $(r, \theta)$.

\begin{itemize}
    \item \textbf{Polar Coordinates $(r, \theta)$:} A point in the plane is defined by its distance $r$ from a central point called the \textbf{pole} (equivalent to the origin) and an angle $\theta$ measured counterclockwise from a fixed ray called the \textbf{polar axis} (equivalent to the positive x-axis).
    \item \textbf{Conversion Formulas:} The relationship between Cartesian and polar coordinates is defined by:
    \[ x = r \cos(\theta) \quad \quad y = r \sin(\theta) \]
    \[ r^2 = x^2 + y^2 \quad \quad \tan(\theta) = \frac{y}{x} \]
    \item \textbf{Area in Polar Coordinates:} Unlike in Cartesian coordinates where area is approximated by summing rectangles, in polar coordinates, we sum the areas of infinitesimally small circular sectors.
    \item \textbf{Arc Length:} We can adapt the arc length formula to find the length of a curve defined by a polar equation.
    \item \textbf{Tangents to Polar Curves:} We can find the slope of a tangent line to a polar curve by treating it as a parametric curve with $\theta$ as the parameter.
\end{itemize}

\section{Intuition and Derivation}

\subsection{Area of a Polar Region}
Imagine a region defined by $r = f(\theta)$ from $\theta = a$ to $\theta = b$. To find its area, we can slice it into many small sectors. The area of a single sector of a circle with radius $r$ and a small angle $d\theta$ is approximately that of a thin triangle, $dA = \frac{1}{2} \times \text{base} \times \text{height} \approx \frac{1}{2} r (r d\theta) = \frac{1}{2}r^2 d\theta$.
By summing (integrating) these infinitesimal sectors from $\theta = a$ to $\theta = b$, we arrive at the formula for the area of a polar region:
\[ A = \int_{a}^{b} \frac{1}{2} r^2 \,d\theta \]

\subsection{Slope of a Tangent Line}
To find the slope $\frac{dy}{dx}$, we view a polar curve $r = f(\theta)$ parametrically.
\[ x = r \cos(\theta) = f(\theta) \cos(\theta) \quad \text{and} \quad y = r \sin(\theta) = f(\theta) \sin(\theta) \]
Using the chain rule, $\frac{dy}{dx} = \frac{dy/d\theta}{dx/d\theta}$. Applying the product rule to $x$ and $y$:
\[ \frac{dy}{d\theta} = \frac{dr}{d\theta}\sin(\theta) + r\cos(\theta) \]
\[ \frac{dx}{d\theta} = \frac{dr}{d\theta}\cos(\theta) - r\sin(\theta) \]
This gives the formula for the slope of the tangent line:
\[ \frac{dy}{dx} = \frac{\frac{dr}{d\theta}\sin(\theta) + r\cos(\theta)}{\frac{dr}{d\theta}\cos(\theta) - r\sin(\theta)} \]
Horizontal tangents occur when $\frac{dy}{d\theta} = 0$ (and $\frac{dx}{d\theta} \neq 0$), and vertical tangents occur when $\frac{dx}{d\theta} = 0$ (and $\frac{dy}{d\theta} \neq 0$).

\subsection{Arc Length of a Polar Curve}
The arc length formula for a parametric curve is $L = \int \sqrt{(\frac{dx}{dt})^2 + (\frac{dy}{dt})^2} \,dt$. Using $\theta$ as the parameter, we substitute the expressions for $\frac{dx}{d\theta}$ and $\frac{dy}{d\theta}$ derived above:
\[ \left(\frac{dx}{d\theta}\right)^2 + \left(\frac{dy}{d\theta}\right)^2 = \left(\frac{dr}{d\theta}\cos(\theta) - r\sin(\theta)\right)^2 + \left(\frac{dr}{d\theta}\sin(\theta) + r\cos(\theta)\right)^2 \]
Expanding and simplifying using the identity $\sin^2(\theta) + \cos^2(\theta) = 1$ yields:
\[ \left(\frac{dx}{d\theta}\right)^2 + \left(\frac{dy}{d\theta}\right)^2 = \left(\frac{dr}{d\theta}\right)^2 + r^2 \]
This leads to the arc length formula for a polar curve from $\theta = a$ to $\theta = b$:
\[ L = \int_{a}^{b} \sqrt{r^2 + \left(\frac{dr}{d\theta}\right)^2} \,d\theta \]

\section{Historical Context and Motivation}
The concepts of polar coordinates have roots tracing back to ancient Greece with Archimedes describing his spiral. [2] However, the modern analytic form was developed in the 17th century. [5] Mathematicians like Isaac Newton and Jacob Bernoulli found that the Cartesian coordinate system was often cumbersome for describing objects or phenomena with radial or circular symmetry, such as planetary orbits or spirals. [1, 4]

The motivation was to create a coordinate system that naturally aligns with the geometry of such problems. [3] Instead of defining a point by its rectangular projections, polar coordinates define a point by its direct distance from a center and its angle of rotation. This simplifies the equations of circles, cardioids, and roses, making calculations like finding area or arc length much more straightforward than they would be in Cartesian form. [3, 27] This principle of choosing a coordinate system to match the problem's symmetry is a fundamental theme that extends to cylindrical and spherical coordinates in multivariable calculus.

\section{Key Formulas}
\begin{itemize}
    \item \textbf{Area of a Polar Region:} $A = \int_{a}^{b} \frac{1}{2} [f(\theta)]^2 \,d\theta = \int_{a}^{b} \frac{1}{2} r^2 \,d\theta$
    \item \textbf{Area Between Two Polar Curves:} For $r_{outer} \ge r_{inner}$, $A = \int_{a}^{b} \frac{1}{2} (r_{outer}^2 - r_{inner}^2) \,d\theta$
    \item \textbf{Arc Length of a Polar Curve:} $L = \int_{a}^{b} \sqrt{r^2 + \left(\frac{dr}{d\theta}\right)^2} \,d\theta$
    \item \textbf{Slope of the Tangent Line:} $\frac{dy}{dx} = \frac{r' \sin(\theta) + r \cos(\theta)}{r' \cos(\theta) - r \sin(\theta)}$, where $r' = \frac{dr}{d\theta}$
\end{itemize}

\section{Prerequisites}
To master calculus with polar coordinates, the following skills are essential:
\begin{itemize}
    \item \textbf{Trigonometry:} Deep familiarity with the unit circle, solving trigonometric equations (e.g., $\cos(\theta) = 1/2$), and trigonometric identities, especially:
    \begin{itemize}
        \item Power-Reducing Formulas: $\sin^2(\theta) = \frac{1 - \cos(2\theta)}{2}$, $\cos^2(\theta) = \frac{1 + \cos(2\theta)}{2}$
        \item Double-Angle Formulas: $\sin(2\theta) = 2\sin(\theta)\cos(\theta)$, $\cos(2\theta) = \cos^2(\theta) - \sin^2(\theta)$
    \end{itemize}
    \item \textbf{Calculus I:} Proficiency in differentiation (product rule, chain rule) and definite integration, including techniques like u-substitution and applying the Fundamental Theorem of Calculus.
    \item \textbf{Algebra:} Skill in manipulating and simplifying complex expressions, especially those involving squares, square roots, and trigonometric functions.
\end{itemize}

\part{Detailed Homework Solutions}
\newpage
\section*{Problem 1}
\textbf{Find the area of the region that is bounded by the given curve and lies in the specified sector.}
\[ r = \sqrt{18\theta}, \quad 0 \le \theta \le \frac{\pi}{2} \]
\textbf{Solution:}
The formula for the area of a polar region is $A = \int_{a}^{b} \frac{1}{2} r^2 \,d\theta$.
\begin{enumerate}
    \item \textbf{Setup the integral:}
    Substitute $r = \sqrt{18\theta}$ and the bounds $a=0, b=\pi/2$.
    \[ A = \int_{0}^{\pi/2} \frac{1}{2} (\sqrt{18\theta})^2 \,d\theta \]
    \item \textbf{Simplify the integrand:}
    \[ A = \int_{0}^{\pi/2} \frac{1}{2} (18\theta) \,d\theta = \int_{0}^{\pi/2} 9\theta \,d\theta \]
    \item \textbf{Integrate:}
    \[ A = 9 \left[ \frac{\theta^2}{2} \right]_{0}^{\pi/2} \]
    \item \textbf{Evaluate using the Fundamental Theorem of Calculus:}
    \[ A = 9 \left( \frac{(\pi/2)^2}{2} - \frac{0^2}{2} \right) = 9 \left( \frac{\pi^2/4}{2} \right) = 9 \left( \frac{\pi^2}{8} \right) \]
\end{enumerate}
\textbf{Final Answer:} $A = \frac{9\pi^2}{8}$

\section*{Problem 2}
\textbf{Find the area of the region that is bounded by the given curve and lies in the specified sector.}
\[ r = 5\sin(\theta) + 5\cos(\theta), \quad 0 \le \theta \le \pi \]
\textbf{Solution:}
\begin{enumerate}
    \item \textbf{Setup the integral:}
    \[ A = \int_{0}^{\pi} \frac{1}{2} (5\sin(\theta) + 5\cos(\theta))^2 \,d\theta \]
    \item \textbf{Simplify the integrand:}
    \[ A = \frac{1}{2} \int_{0}^{\pi} (25\sin^2(\theta) + 50\sin(\theta)\cos(\theta) + 25\cos^2(\theta)) \,d\theta \]
    Using $\sin^2(\theta) + \cos^2(\theta) = 1$ and $2\sin(\theta)\cos(\theta) = \sin(2\theta)$:
    \[ A = \frac{1}{2} \int_{0}^{\pi} (25(\sin^2(\theta) + \cos^2(\theta)) + 25(2\sin(\theta)\cos(\theta))) \,d\theta \]
    \[ A = \frac{1}{2} \int_{0}^{\pi} (25 + 25\sin(2\theta)) \,d\theta \]
    \item \textbf{Integrate:}
    \[ A = \frac{25}{2} \int_{0}^{\pi} (1 + \sin(2\theta)) \,d\theta \]
    \[ A = \frac{25}{2} \left[ \theta - \frac{1}{2}\cos(2\theta) \right]_{0}^{\pi} \]
    \item \textbf{Evaluate:}
    \[ A = \frac{25}{2} \left[ \left(\pi - \frac{1}{2}\cos(2\pi)\right) - \left(0 - \frac{1}{2}\cos(0)\right) \right] \]
    \[ A = \frac{25}{2} \left[ \left(\pi - \frac{1}{2}\right) - \left(-\frac{1}{2}\right) \right] = \frac{25}{2} \left(\pi - \frac{1}{2} + \frac{1}{2}\right) = \frac{25\pi}{2} \]
\end{enumerate}
\textbf{Final Answer:} $A = \frac{25\pi}{2}$

\section*{Problem 3}
\textbf{Find the area of the region that is bounded by the given curve and lies in the specified sector.}
\[ r = \frac{7}{\theta}, \quad \frac{\pi}{2} \le \theta \le 2\pi \]
\textbf{Solution:}
\begin{enumerate}
    \item \textbf{Setup the integral:}
    \[ A = \int_{\pi/2}^{2\pi} \frac{1}{2} \left(\frac{7}{\theta}\right)^2 \,d\theta \]
    \item \textbf{Simplify the integrand:}
    \[ A = \frac{1}{2} \int_{\pi/2}^{2\pi} \frac{49}{\theta^2} \,d\theta = \frac{49}{2} \int_{\pi/2}^{2\pi} \theta^{-2} \,d\theta \]
    \item \textbf{Integrate:}
    \[ A = \frac{49}{2} \left[ -\theta^{-1} \right]_{\pi/2}^{2\pi} = \frac{49}{2} \left[ -\frac{1}{\theta} \right]_{\pi/2}^{2\pi} \]
    \item \textbf{Evaluate:}
    \[ A = \frac{49}{2} \left( -\frac{1}{2\pi} - \left(-\frac{1}{\pi/2}\right) \right) = \frac{49}{2} \left( -\frac{1}{2\pi} + \frac{2}{\pi} \right) \]
    \[ A = \frac{49}{2} \left( -\frac{1}{2\pi} + \frac{4}{2\pi} \right) = \frac{49}{2} \left( \frac{3}{2\pi} \right) = \frac{147}{4\pi} \]
\end{enumerate}
\textbf{Final Answer:} $A = \frac{147}{4\pi}$

\section*{Problem 4}
\textbf{Find the area of the shaded region.}
\[ r^2 = \sin(2\theta) \]
The shaded region is one loop of the lemniscate.
\textbf{Solution:}
\begin{enumerate}
    \item \textbf{Find the bounds:} A loop starts and ends when $r=0$.
    \[ r^2 = 0 \implies \sin(2\theta) = 0 \]
    This occurs when $2\theta = 0, \pi, 2\pi, \dots$. So, $\theta = 0, \pi/2, \pi, \dots$. The first loop is traced from $\theta=0$ to $\theta=\pi/2$.
    \item \textbf{Setup the integral:} The formula is $A = \int \frac{1}{2}r^2 d\theta$. Here, $r^2$ is already given.
    \[ A = \int_{0}^{\pi/2} \frac{1}{2} \sin(2\theta) \,d\theta \]
    \item \textbf{Integrate:}
    \[ A = \frac{1}{2} \left[ -\frac{1}{2}\cos(2\theta) \right]_{0}^{\pi/2} = -\frac{1}{4} \left[ \cos(2\theta) \right]_{0}^{\pi/2} \]
    \item \textbf{Evaluate:}
    \[ A = -\frac{1}{4} (\cos(\pi) - \cos(0)) = -\frac{1}{4} (-1 - 1) = -\frac{1}{4}(-2) = \frac{1}{2} \]
\end{enumerate}
\textbf{Final Answer:} $A = \frac{1}{2}$
\newpage
\section*{Problem 5}
\textbf{Find the area of the shaded region.}
\[ r = 4 + \cos(\theta) \]
The shaded region is from $\theta = \pi/2$ to $\theta = 3\pi/2$.
\textbf{Solution:}
\begin{enumerate}
    \item \textbf{Setup the integral:} The bounds are given by the graph as $\pi/2$ to $3\pi/2$.
    \[ A = \int_{\pi/2}^{3\pi/2} \frac{1}{2} (4 + \cos(\theta))^2 \,d\theta \]
    \item \textbf{Expand the integrand:}
    \[ A = \frac{1}{2} \int_{\pi/2}^{3\pi/2} (16 + 8\cos(\theta) + \cos^2(\theta)) \,d\theta \]
    Use the power-reducing formula $\cos^2(\theta) = \frac{1 + \cos(2\theta)}{2}$.
    \[ A = \frac{1}{2} \int_{\pi/2}^{3\pi/2} \left(16 + 8\cos(\theta) + \frac{1}{2} + \frac{1}{2}\cos(2\theta)\right) \,d\theta \]
    \[ A = \frac{1}{2} \int_{\pi/2}^{3\pi/2} \left(\frac{33}{2} + 8\cos(\theta) + \frac{1}{2}\cos(2\theta)\right) \,d\theta \]
    \item \textbf{Integrate:}
    \[ A = \frac{1}{2} \left[ \frac{33}{2}\theta + 8\sin(\theta) + \frac{1}{4}\sin(2\theta) \right]_{\pi/2}^{3\pi/2} \]
    \item \textbf{Evaluate:}
    \[ A = \frac{1}{2} \left[ \left(\frac{33}{2}\frac{3\pi}{2} + 8\sin(\frac{3\pi}{2}) + \frac{1}{4}\sin(3\pi)\right) - \left(\frac{33}{2}\frac{\pi}{2} + 8\sin(\frac{\pi}{2}) + \frac{1}{4}\sin(\pi)\right) \right] \]
    \[ A = \frac{1}{2} \left[ \left(\frac{99\pi}{4} + 8(-1) + 0\right) - \left(\frac{33\pi}{4} + 8(1) + 0\right) \right] \]
    \[ A = \frac{1}{2} \left[ \frac{99\pi}{4} - 8 - \frac{33\pi}{4} - 8 \right] = \frac{1}{2} \left[ \frac{66\pi}{4} - 16 \right] = \frac{33\pi}{4} - 8 \]
\end{enumerate}
\textbf{Final Answer:} $A = \frac{33\pi}{4} - 8$

\section*{Problem 6}
\textbf{Find the area of the shaded region.}
\[ r = 4 + 3\sin(\theta) \]
The shaded region appears to be from $\theta = -\pi/2$ to $\theta = \pi/2$.
\textbf{Solution:}
\begin{enumerate}
    \item \textbf{Setup the integral:}
    \[ A = \int_{-\pi/2}^{\pi/2} \frac{1}{2} (4 + 3\sin(\theta))^2 \,d\theta \]
    \item \textbf{Expand the integrand:}
    \[ A = \frac{1}{2} \int_{-\pi/2}^{\pi/2} (16 + 24\sin(\theta) + 9\sin^2(\theta)) \,d\theta \]
    Use the power-reducing formula $\sin^2(\theta) = \frac{1 - \cos(2\theta)}{2}$.
    \[ A = \frac{1}{2} \int_{-\pi/2}^{\pi/2} \left(16 + 24\sin(\theta) + 9\left(\frac{1 - \cos(2\theta)}{2}\right)\right) \,d\theta \]
    \[ A = \frac{1}{2} \int_{-\pi/2}^{\pi/2} \left(\frac{32}{2} + \frac{9}{2} + 24\sin(\theta) - \frac{9}{2}\cos(2\theta)\right) \,d\theta \]
    \[ A = \frac{1}{2} \int_{-\pi/2}^{\pi/2} \left(\frac{41}{2} + 24\sin(\theta) - \frac{9}{2}\cos(2\theta)\right) \,d\theta \]
    \item \textbf{Integrate:}
    \[ A = \frac{1}{2} \left[ \frac{41}{2}\theta - 24\cos(\theta) - \frac{9}{4}\sin(2\theta) \right]_{-\pi/2}^{\pi/2} \]
    \item \textbf{Evaluate:}
    \[ A = \frac{1}{2} \left[ \left(\frac{41\pi}{4} - 24\cos(\frac{\pi}{2}) - \frac{9}{4}\sin(\pi)\right) - \left(-\frac{41\pi}{4} - 24\cos(-\frac{\pi}{2}) - \frac{9}{4}\sin(-\pi)\right) \right] \]
    \[ A = \frac{1}{2} \left[ \left(\frac{41\pi}{4} - 0 - 0\right) - \left(-\frac{41\pi}{4} - 0 - 0\right) \right] = \frac{1}{2} \left( \frac{41\pi}{4} + \frac{41\pi}{4} \right) = \frac{1}{2}\left(\frac{82\pi}{4}\right) = \frac{41\pi}{4} \]
\end{enumerate}
\textbf{Final Answer:} $A = \frac{41\pi}{4}$

\section*{Problem 7}
\textbf{Sketch the curve and find the area that it encloses.}
\[ r = 2 - 2\sin(\theta) \]
This is a cardioid. The first graph shown is the correct one. It is symmetric about the y-axis, pointing downwards.
\textbf{Solution:}
\begin{enumerate}
    \item \textbf{Find the bounds:} The entire cardioid is traced as $\theta$ goes from $0$ to $2\pi$.
    \item \textbf{Setup the integral:}
    \[ A = \int_{0}^{2\pi} \frac{1}{2} (2 - 2\sin(\theta))^2 \,d\theta = \int_{0}^{2\pi} \frac{1}{2} (4(1 - \sin(\theta))^2) \,d\theta \]
    \[ A = 2 \int_{0}^{2\pi} (1 - 2\sin(\theta) + \sin^2(\theta)) \,d\theta \]
    \item \textbf{Apply power reduction:}
    \[ A = 2 \int_{0}^{2\pi} \left(1 - 2\sin(\theta) + \frac{1 - \cos(2\theta)}{2}\right) \,d\theta \]
    \[ A = 2 \int_{0}^{2\pi} \left(\frac{3}{2} - 2\sin(\theta) - \frac{1}{2}\cos(2\theta)\right) \,d\theta \]
    \item \textbf{Integrate:}
    \[ A = 2 \left[ \frac{3}{2}\theta + 2\cos(\theta) - \frac{1}{4}\sin(2\theta) \right]_{0}^{2\pi} \]
    \item \textbf{Evaluate:}
    \[ A = 2 \left[ \left(\frac{3}{2}(2\pi) + 2\cos(2\pi) - 0\right) - \left(0 + 2\cos(0) - 0\right) \right] \]
    \[ A = 2 [ (3\pi + 2) - (2) ] = 2(3\pi) = 6\pi \]
\end{enumerate}
\textbf{Final Answer:} The area is $6\pi$.

\section*{Problem 8}
\textbf{Find the area of the region enclosed by one loop of the curve.}
\[ r = \sin(2\theta) \]
This is a four-leaved rose.
\textbf{Solution:}
\begin{enumerate}
    \item \textbf{Find the bounds for one loop:} Set $r=0$.
    \[ \sin(2\theta) = 0 \implies 2\theta = 0, \pi, \dots \implies \theta = 0, \pi/2, \dots \]
    The first loop is traced from $\theta=0$ to $\theta=\pi/2$.
    \item \textbf{Setup the integral:}
    \[ A = \int_{0}^{\pi/2} \frac{1}{2} (\sin(2\theta))^2 \,d\theta = \frac{1}{2} \int_{0}^{\pi/2} \sin^2(2\theta) \,d\theta \]
    \item \textbf{Apply power reduction:} Use $\sin^2(x) = \frac{1-\cos(2x)}{2}$. Here $x=2\theta$.
    \[ A = \frac{1}{2} \int_{0}^{\pi/2} \frac{1 - \cos(4\theta)}{2} \,d\theta = \frac{1}{4} \int_{0}^{\pi/2} (1 - \cos(4\theta)) \,d\theta \]
    \item \textbf{Integrate:}
    \[ A = \frac{1}{4} \left[ \theta - \frac{1}{4}\sin(4\theta) \right]_{0}^{\pi/2} \]
    \item \textbf{Evaluate:}
    \[ A = \frac{1}{4} \left[ \left(\frac{\pi}{2} - \frac{1}{4}\sin(2\pi)\right) - \left(0 - \frac{1}{4}\sin(0)\right) \right] \]
    \[ A = \frac{1}{4} \left( \frac{\pi}{2} - 0 - 0 \right) = \frac{\pi}{8} \]
\end{enumerate}
\textbf{Final Answer:} $A = \frac{\pi}{8}$
\newpage
\section*{Problem 9}
\textbf{Find the area of the region that lies inside the first curve and outside the second curve.}
\[ r = 3\cos(\theta), \quad r = 1 + \cos(\theta) \]
\textbf{Solution:}
\begin{enumerate}
    \item \textbf{Find intersection points:}
    \[ 3\cos(\theta) = 1 + \cos(\theta) \implies 2\cos(\theta) = 1 \implies \cos(\theta) = \frac{1}{2} \]
    This gives $\theta = -\frac{\pi}{3}$ and $\theta = \frac{\pi}{3}$.
    \item \textbf{Identify outer and inner curves:} In the interval $(-\pi/3, \pi/3)$, for $\theta=0$, $r_1=3$ and $r_2=2$. So $r_1=3\cos(\theta)$ is the outer curve.
    \item \textbf{Setup the integral:} Use the formula $A = \int \frac{1}{2}(r_{outer}^2 - r_{inner}^2)d\theta$. By symmetry, we can integrate from $0$ to $\pi/3$ and double the result.
    \[ A = 2 \int_{0}^{\pi/3} \frac{1}{2} \left[ (3\cos(\theta))^2 - (1+\cos(\theta))^2 \right] \,d\theta \]
    \[ A = \int_{0}^{\pi/3} \left[ 9\cos^2(\theta) - (1 + 2\cos(\theta) + \cos^2(\theta)) \right] \,d\theta \]
    \[ A = \int_{0}^{\pi/3} (8\cos^2(\theta) - 2\cos(\theta) - 1) \,d\theta \]
    \item \textbf{Apply power reduction:}
    \[ A = \int_{0}^{\pi/3} \left(8\left(\frac{1+\cos(2\theta)}{2}\right) - 2\cos(\theta) - 1\right) \,d\theta \]
    \[ A = \int_{0}^{\pi/3} (4 + 4\cos(2\theta) - 2\cos(\theta) - 1) \,d\theta = \int_{0}^{\pi/3} (3 + 4\cos(2\theta) - 2\cos(\theta)) \,d\theta \]
    \item \textbf{Integrate:}
    \[ A = \left[ 3\theta + 2\sin(2\theta) - 2\sin(\theta) \right]_{0}^{\pi/3} \]
    \item \textbf{Evaluate:}
    \[ A = \left( 3\frac{\pi}{3} + 2\sin(\frac{2\pi}{3}) - 2\sin(\frac{\pi}{3}) \right) - (0) \]
    \[ A = \pi + 2\left(\frac{\sqrt{3}}{2}\right) - 2\left(\frac{\sqrt{3}}{2}\right) = \pi \]
\end{enumerate}
\textbf{Final Answer:} $A = \pi$

\section*{Problem 10}
\textbf{Find the area of the region that lies inside the first curve and outside the second curve.}
\[ r = 1 + \cos(\theta), \quad r = 2 - \cos(\theta) \]
\textbf{Solution:}
There appears to be a typo in the question, as for all $\theta$, $1 + \cos(\theta) \le 2$ and $2 - \cos(\theta) \ge 1$. A region "inside" the first curve and "outside" the second is not possible as $1+\cos\theta$ is never greater than $2-\cos\theta$. The intended question is likely to find the area inside $r_2 = 2 - \cos(\theta)$ and outside $r_1 = 1 + \cos(\theta)$. I will solve this instead.
\begin{enumerate}
    \item \textbf{Find intersection points:}
    \[ 2 - \cos(\theta) = 1 + \cos(\theta) \implies 1 = 2\cos(\theta) \implies \cos(\theta) = \frac{1}{2} \]
    This gives $\theta = -\frac{\pi}{3}$ and $\theta = \frac{\pi}{3}$.
    \item \textbf{Setup the integral:} $r_2$ is the outer curve. Use symmetry.
    \[ A = 2 \int_{0}^{\pi/3} \frac{1}{2} \left[ (2-\cos(\theta))^2 - (1+\cos(\theta))^2 \right] \,d\theta \]
    \[ A = \int_{0}^{\pi/3} \left[ (4 - 4\cos(\theta) + \cos^2(\theta)) - (1 + 2\cos(\theta) + \cos^2(\theta)) \right] \,d\theta \]
    \[ A = \int_{0}^{\pi/3} (3 - 6\cos(\theta)) \,d\theta \]
    \item \textbf{Integrate:}
    \[ A = \left[ 3\theta - 6\sin(\theta) \right]_{0}^{\pi/3} \]
    \item \textbf{Evaluate:}
    \[ A = \left( 3\frac{\pi}{3} - 6\sin(\frac{\pi}{3}) \right) - (0) = \pi - 6\left(\frac{\sqrt{3}}{2}\right) = \pi - 3\sqrt{3} \]
    Since area cannot be negative, this confirms my suspicion about the curves being swapped. Let's assume the question meant inside $r=2-\cos\theta$ and outside $r=1+\cos\theta$. Wait, at $\theta=0$, $r_1=2$, $r_2=1$. Ah, I swapped them. The first curve $r=1+\cos\theta$ is NOT always smaller. Let's check: $1+\cos\theta > 2-\cos\theta \implies 2\cos\theta > 1 \implies \cos\theta > 1/2$. This is true for $\theta \in (-\pi/3, \pi/3)$. So the original problem statement IS solvable. $r_1$ is outer on this interval.
    
    Let's re-solve with the correct outer curve.
    Outer: $r_1 = 1+\cos\theta$. Inner: $r_2 = 2-\cos\theta$. Bounds: $-\pi/3$ to $\pi/3$.
    \[ A = \int_{-\pi/3}^{\pi/3} \frac{1}{2} \left[ (1+\cos\theta)^2 - (2-\cos\theta)^2 \right] d\theta \]
    \[ A = \frac{1}{2} \int_{-\pi/3}^{\pi/3} \left[ (1+2\cos\theta+\cos^2\theta) - (4-4\cos\theta+\cos^2\theta) \right] d\theta \]
    \[ A = \frac{1}{2} \int_{-\pi/3}^{\pi/3} (6\cos\theta - 3) d\theta \]
    Using symmetry:
    \[ A = \int_{0}^{\pi/3} (6\cos\theta - 3) d\theta = [6\sin\theta - 3\theta]_0^{\pi/3} \]
    \[ A = (6\sin(\pi/3) - 3(\pi/3)) - 0 = 6(\frac{\sqrt{3}}{2}) - \pi = 3\sqrt{3} - \pi \]
\end{enumerate}
\textbf{Final Answer:} $3\sqrt{3} - \pi$

\section*{Problem 11}
\textbf{Find the area of the region that lies inside both curves.}
\[ r = \sin(2\theta), \quad r = \cos(2\theta) \]
\textbf{Solution:}
\begin{enumerate}
    \item \textbf{Find intersection points:}
    \[ \sin(2\theta) = \cos(2\theta) \implies \tan(2\theta) = 1 \]
    $2\theta = \frac{\pi}{4} + n\pi \implies \theta = \frac{\pi}{8} + \frac{n\pi}{2}$. The intersection in the first quadrant is $\theta=\pi/8$.
    \item \textbf{Setup the integral:} The shape has 8-fold symmetry. We can find the area of one small wedge and multiply by 8. From $\theta=0$ to $\theta=\pi/8$, the smaller radius is $r=\sin(2\theta)$. We are finding the area inside BOTH curves.
    The total area is the sum of areas of petals, but we must avoid double counting the overlapping region.
    By symmetry, the total area is 8 times the area from $\theta=0$ to $\theta=\pi/8$ bounded by the smaller curve, which is $r=\sin(2\theta)$ in this interval.
    \[ A = 8 \times \int_{0}^{\pi/8} \frac{1}{2} (\sin(2\theta))^2 \,d\theta = 4 \int_{0}^{\pi/8} \sin^2(2\theta) \,d\theta \]
    \item \textbf{Apply power reduction:}
    \[ A = 4 \int_{0}^{\pi/8} \frac{1-\cos(4\theta)}{2} \,d\theta = 2 \int_{0}^{\pi/8} (1-\cos(4\theta)) \,d\theta \]
    \item \textbf{Integrate:}
    \[ A = 2 \left[ \theta - \frac{1}{4}\sin(4\theta) \right]_{0}^{\pi/8} \]
    \item \textbf{Evaluate:}
    \[ A = 2 \left[ (\frac{\pi}{8} - \frac{1}{4}\sin(\frac{\pi}{2})) - (0) \right] = 2 \left( \frac{\pi}{8} - \frac{1}{4} \right) = \frac{\pi}{4} - \frac{1}{2} \]
\end{enumerate}
\textbf{Final Answer:} $A = \frac{\pi}{4} - \frac{1}{2}$
\newpage
\section*{Problem 12}
\textbf{Find all points of intersection of the given curves. Assume $0 \le \theta \le \pi$.}
\[ r = \cos(\theta), \quad r = \sin(2\theta) \]
\textbf{Solution:}
\begin{enumerate}
    \item \textbf{Check the pole:} The pole is a point of intersection if $r=0$ for some $\theta$ on each curve.
    For $r=\cos(\theta)$, $r=0$ when $\theta = \pi/2$.
    For $r=\sin(2\theta)$, $r=0$ when $2\theta = 0, \pi, 2\pi, \dots \implies \theta = 0, \pi/2, \pi$.
    Since both curves are at the pole when $\theta = \pi/2$, the pole is an intersection point. We can write this as $(0, \pi/2)$.
    \item \textbf{Set equations equal:}
    \[ \cos(\theta) = \sin(2\theta) \]
    Using the double angle identity $\sin(2\theta) = 2\sin(\theta)\cos(\theta)$:
    \[ \cos(\theta) = 2\sin(\theta)\cos(\theta) \]
    \[ \cos(\theta) - 2\sin(\theta)\cos(\theta) = 0 \]
    \[ \cos(\theta) (1 - 2\sin(\theta)) = 0 \]
    \item \textbf{Solve for $\theta$:} This gives two possibilities:
    \begin{itemize}
        \item $\cos(\theta) = 0 \implies \theta = \pi/2$. This corresponds to the pole we already found. At $\theta=\pi/2$, $r=0$.
        \item $1 - 2\sin(\theta) = 0 \implies \sin(\theta) = 1/2$. In the interval $0 \le \theta \le \pi$, this occurs at $\theta = \pi/6$ and $\theta = 5\pi/6$.
    \end{itemize}
    \item \textbf{Find the corresponding r values:}
    \begin{itemize}
        \item For $\theta = \pi/6$: $r = \cos(\pi/6) = \sqrt{3}/2$. Point: $(\sqrt{3}/2, \pi/6)$.
        \item For $\theta = 5\pi/6$: $r = \cos(5\pi/6) = -\sqrt{3}/2$. Point: $(-\sqrt{3}/2, 5\pi/6)$.
    \end{itemize}
\end{enumerate}
\textbf{Final Answer:} The intersection points are POLE, $(\frac{\sqrt{3}}{2}, \frac{\pi}{6})$, and $(-\frac{\sqrt{3}}{2}, \frac{5\pi}{6})$.

\section*{Problem 13}
\textbf{Find all points of intersection of the given curves. Assume $0 \le \theta \le 2\pi$.}
\[ r^2 = 2\cos(2\theta), \quad r = 1 \]
\textbf{Solution:}
\begin{enumerate}
    \item \textbf{Substitute $r=1$ into the first equation:}
    \[ 1^2 = 2\cos(2\theta) \implies \cos(2\theta) = \frac{1}{2} \]
    \item \textbf{Solve for $2\theta$:} We need to find all angles $2\theta$ in the interval $[0, 4\pi]$ whose cosine is $1/2$.
    \[ 2\theta = \frac{\pi}{3}, \frac{5\pi}{3}, \frac{7\pi}{3}, \frac{11\pi}{3} \]
    \item \textbf{Solve for $\theta$:} Divide by 2.
    \[ \theta = \frac{\pi}{6}, \frac{5\pi}{6}, \frac{7\pi}{6}, \frac{11\pi}{6} \]
    \item \textbf{State the points:} For all these angles, the radius is $r=1$.
\end{enumerate}
\textbf{Final Answer:} The intersection points are $(1, \frac{\pi}{6})$, $(1, \frac{5\pi}{6})$, $(1, \frac{7\pi}{6})$, and $(1, \frac{11\pi}{6})$.

\section*{Problem 14}
This problem is identical to Problem 9.
\textbf{Find the area of the shaded region.} (Region inside $r = 3\cos(\theta)$ and outside $r=1+\cos(\theta)$).
\[ r = 3\cos(\theta), \quad r = 1 + \cos(\theta) \]
\textbf{Solution:}
As solved in Problem 9, the intersection points are at $\theta = \pm \pi/3$, and the area is given by the integral:
\[ A = \int_{-\pi/3}^{\pi/3} \frac{1}{2} \left[ (3\cos\theta)^2 - (1+\cos\theta)^2 \right] d\theta = \pi \]
\textbf{Final Answer:} $A = \pi$

\section*{Problem 15}
\textbf{Find the exact length of the polar curve.}
\[ r = \theta^2, \quad 0 \le \theta \le 7\pi \]
\textbf{Solution:}
\begin{enumerate}
    \item \textbf{Find the derivative:} $r = \theta^2 \implies \frac{dr}{d\theta} = 2\theta$.
    \item \textbf{Setup the arc length integral:} $L = \int_{a}^{b} \sqrt{r^2 + (\frac{dr}{d\theta})^2} \,d\theta$.
    \[ L = \int_{0}^{7\pi} \sqrt{(\theta^2)^2 + (2\theta)^2} \,d\theta = \int_{0}^{7\pi} \sqrt{\theta^4 + 4\theta^2} \,d\theta \]
    \item \textbf{Simplify the integrand:} Since $\theta \ge 0$, we can factor out $\sqrt{\theta^2} = \theta$.
    \[ L = \int_{0}^{7\pi} \sqrt{\theta^2(\theta^2 + 4)} \,d\theta = \int_{0}^{7\pi} \theta \sqrt{\theta^2 + 4} \,d\theta \]
    \item \textbf{Integrate using u-substitution:} Let $u = \theta^2 + 4$, so $du = 2\theta d\theta$, or $\frac{1}{2}du = \theta d\theta$.
    Change the bounds: when $\theta=0, u=4$. When $\theta=7\pi, u=(7\pi)^2+4 = 49\pi^2+4$.
    \[ L = \int_{4}^{49\pi^2+4} \frac{1}{2} \sqrt{u} \,du = \frac{1}{2} \int_{4}^{49\pi^2+4} u^{1/2} \,du \]
    \[ L = \frac{1}{2} \left[ \frac{2}{3} u^{3/2} \right]_{4}^{49\pi^2+4} = \frac{1}{3} \left[ u^{3/2} \right]_{4}^{49\pi^2+4} \]
    \item \textbf{Evaluate:}
    \[ L = \frac{1}{3} \left( (49\pi^2+4)^{3/2} - 4^{3/2} \right) = \frac{1}{3} \left( (49\pi^2+4)^{3/2} - 8 \right) \]
\end{enumerate}
\textbf{Final Answer:} $L = \frac{1}{3}((49\pi^2+4)^{3/2} - 8)$

\section*{Problem 16}
\textbf{Find the exact length of the curve. Use a graph to determine the parameter interval.}
\[ r = \cos^2(\theta/2) \]
The curve is a cardioid, traced once for $0 \le \theta \le 2\pi$.
\textbf{Solution:}
\begin{enumerate}
    \item \textbf{Simplify r and find derivative:} Use the identity $\cos^2(x) = \frac{1+\cos(2x)}{2}$.
    \[ r = \frac{1}{2}(1 + \cos(\theta)) \]
    \[ \frac{dr}{d\theta} = -\frac{1}{2}\sin(\theta) \]
    \item \textbf{Setup the arc length integral:}
    \[ L = \int_{0}^{2\pi} \sqrt{\left(\frac{1}{2}(1+\cos\theta)\right)^2 + \left(-\frac{1}{2}\sin\theta\right)^2} \,d\theta \]
    \item \textbf{Simplify the integrand:}
    \[ L = \int_{0}^{2\pi} \sqrt{\frac{1}{4}(1+2\cos\theta+\cos^2\theta) + \frac{1}{4}\sin^2\theta} \,d\theta \]
    \[ L = \frac{1}{2} \int_{0}^{2\pi} \sqrt{1+2\cos\theta+\cos^2\theta+\sin^2\theta} \,d\theta = \frac{1}{2} \int_{0}^{2\pi} \sqrt{2+2\cos\theta} \,d\theta \]
    Use the identity $1+\cos\theta = 2\cos^2(\theta/2)$.
    \[ L = \frac{1}{2} \int_{0}^{2\pi} \sqrt{2(2\cos^2(\theta/2))} \,d\theta = \frac{1}{2} \int_{0}^{2\pi} \sqrt{4\cos^2(\theta/2)} \,d\theta = \int_{0}^{2\pi} |\cos(\theta/2)| \,d\theta \]
    \item \textbf{Integrate over the absolute value:} $\cos(\theta/2)$ is positive for $0 \le \theta < \pi$ and negative for $\pi < \theta \le 2\pi$.
    \[ L = \int_{0}^{\pi} \cos(\theta/2) \,d\theta + \int_{\pi}^{2\pi} -\cos(\theta/2) \,d\theta \]
    The curve is symmetric about the x-axis, so we can calculate the length of the top half (from 0 to $\pi$) and double it.
    \[ L = 2 \int_{0}^{\pi} \cos(\theta/2) \,d\theta = 2 \left[ 2\sin(\theta/2) \right]_{0}^{\pi} = 4 \left[ \sin(\pi/2) - \sin(0) \right] = 4(1 - 0) = 4 \]
\end{enumerate}
\textbf{Final Answer:} $L = 4$
\newpage
\section*{Problem 17}
\textbf{Find the slope of the tangent line to the given polar curve at the point specified by the value of $\theta$.}
\[ r = 8\cos(\theta), \quad \theta = \frac{\pi}{3} \]
\textbf{Solution:}
\begin{enumerate}
    \item \textbf{Find the derivative:} $r' = \frac{dr}{d\theta} = -8\sin(\theta)$.
    \item \textbf{Use the slope formula:}
    \[ \frac{dy}{dx} = \frac{r' \sin(\theta) + r \cos(\theta)}{r' \cos(\theta) - r \sin(\theta)} = \frac{(-8\sin\theta)\sin\theta + (8\cos\theta)\cos\theta}{(-8\sin\theta)\cos\theta - (8\cos\theta)\sin\theta} \]
    \[ \frac{dy}{dx} = \frac{8(\cos^2\theta - \sin^2\theta)}{-16\sin\theta\cos\theta} = \frac{8\cos(2\theta)}{-8\sin(2\theta)} = -\cot(2\theta) \]
    \item \textbf{Evaluate at $\theta = \pi/3$:}
    \[ \frac{dy}{dx} = -\cot(2 \cdot \frac{\pi}{3}) = -\cot(\frac{2\pi}{3}) = -(-\frac{1}{\sqrt{3}}) = \frac{1}{\sqrt{3}} \]
\end{enumerate}
\textbf{Final Answer:} $\frac{1}{\sqrt{3}}$ or $\frac{\sqrt{3}}{3}$

\section*{Problem 18}
\textbf{Find the slope of the tangent line...}
\[ r = \frac{1}{\theta}, \quad \theta = \pi \]
\textbf{Solution:}
\begin{enumerate}
    \item \textbf{Find the derivative:} $r' = \frac{dr}{d\theta} = -1/\theta^2$.
    \item \textbf{Evaluate r and r' at $\theta = \pi$:}
    \[ r(\pi) = 1/\pi, \quad r'(\pi) = -1/\pi^2 \]
    \item \textbf{Use the slope formula:}
    \[ \frac{dy}{dx} = \frac{r' \sin(\theta) + r \cos(\theta)}{r' \cos(\theta) - r \sin(\theta)} \]
    At $\theta = \pi$: $\sin(\pi)=0$, $\cos(\pi)=-1$.
    \[ \frac{dy}{dx} = \frac{(-1/\pi^2)(0) + (1/\pi)(-1)}{(-1/\pi^2)(-1) - (1/\pi)(0)} = \frac{-1/\pi}{1/\pi^2} = -\frac{1}{\pi} \cdot \frac{\pi^2}{1} = -\pi \]
\end{enumerate}
\textbf{Final Answer:} $-\pi$

\section*{Problem 19}
\textbf{Find the slope of the tangent line...}
\[ r = \sin(\theta) + 3\cos(\theta), \quad \theta = \frac{\pi}{2} \]
\textbf{Solution:}
\begin{enumerate}
    \item \textbf{Find the derivative:} $r' = \cos(\theta) - 3\sin(\theta)$.
    \item \textbf{Evaluate r and r' at $\theta = \pi/2$:}
    At $\theta=\pi/2$, $\sin(\pi/2)=1, \cos(\pi/2)=0$.
    \[ r(\pi/2) = 1 + 3(0) = 1 \]
    \[ r'(\pi/2) = 0 - 3(1) = -3 \]
    \item \textbf{Use the slope formula:}
    \[ \frac{dy}{dx} = \frac{r' \sin(\theta) + r \cos(\theta)}{r' \cos(\theta) - r \sin(\theta)} \]
    At $\theta = \pi/2$:
    \[ \frac{dy}{dx} = \frac{(-3)(1) + (1)(0)}{(-3)(0) - (1)(1)} = \frac{-3}{-1} = 3 \]
\end{enumerate}
\textbf{Final Answer:} $3$

\section*{Problem 20}
\textbf{Find the points on the given curve where the tangent line is horizontal or vertical. $0 \le \theta < \pi$.}
\[ r = \sin(\theta) \]
\textbf{Solution:}
\begin{enumerate}
    \item \textbf{Find expressions for $dy/d\theta$ and $dx/d\theta$:}
    $r = \sin(\theta)$, $r' = \cos(\theta)$.
    \[ \frac{dy}{d\theta} = r'\sin\theta + r\cos\theta = (\cos\theta)\sin\theta + (\sin\theta)\cos\theta = 2\sin\theta\cos\theta = \sin(2\theta) \]
    \[ \frac{dx}{d\theta} = r'\cos\theta - r\sin\theta = (\cos\theta)\cos\theta - (\sin\theta)\sin\theta = \cos^2\theta - \sin^2\theta = \cos(2\theta) \]
    \item \textbf{Horizontal tangents:} Set $\frac{dy}{d\theta} = 0$.
    \[ \sin(2\theta) = 0 \implies 2\theta = 0, \pi, 2\pi, \dots \]
    For $0 \le \theta < \pi$, this means $0 \le 2\theta < 2\pi$. So $2\theta = 0, \pi$.
    \[ \theta = 0, \quad \theta = \pi/2 \]
    At these points, $\frac{dx}{d\theta} = \cos(0)=1$ and $\cos(\pi)=-1$, which are non-zero.
    \item \textbf{Vertical tangents:} Set $\frac{dx}{d\theta} = 0$.
    \[ \cos(2\theta) = 0 \implies 2\theta = \pi/2, 3\pi/2, \dots \]
    For $0 \le \theta < \pi$, this means $0 \le 2\theta < 2\pi$. So $2\theta = \pi/2, 3\pi/2$.
    \[ \theta = \pi/4, \quad \theta = 3\pi/4 \]
    At these points, $\frac{dy}{d\theta} = \sin(\pi/2)=1$ and $\sin(3\pi/2)=-1$, which are non-zero.
    \item \textbf{Find the points $(r, \theta)$:}
    \begin{itemize}
        \item \textbf{Horizontal tangents:}
        \begin{itemize}
            \item $\theta = 0 \implies r = \sin(0) = 0$. Point: $(0, 0)$. This is the pole.
            \item $\theta = \pi/2 \implies r = \sin(\pi/2) = 1$. Point: $(1, \pi/2)$.
        \end{itemize}
        \item \textbf{Vertical tangents:}
        \begin{itemize}
            \item $\theta = \pi/4 \implies r = \sin(\pi/4) = \sqrt{2}/2$. Point: $(\sqrt{2}/2, \pi/4)$.
            \item $\theta = 3\pi/4 \implies r = \sin(3\pi/4) = \sqrt{2}/2$. Point: $(\sqrt{2}/2, 3\pi/4)$.
        \end{itemize}
    \end{itemize}
\end{enumerate}
\textbf{Final Answer:}
\begin{itemize}
    \item Horizontal tangents at: $(0, 0)$ and $(1, \pi/2)$.
    \item Vertical tangents at: $(\frac{\sqrt{2}}{2}, \frac{\pi}{4})$ and $(\frac{\sqrt{2}}{2}, \frac{3\pi}{4})$.
\end{itemize}

\part{In-Depth Analysis of Problems and Techniques}
\section{Problem Types and General Approach}
\begin{itemize}
    \item \textbf{Type 1: Area of a Simple Polar Region (Problems 1, 2, 3, 7):} These are the most straightforward problems. The strategy is to directly apply the area formula $A = \int_{a}^{b} \frac{1}{2} r^2 \,d\theta$. The main work involves squaring the function $r=f(\theta)$, simplifying, and performing the integration, which often requires trigonometric identities.
    \item \textbf{Type 2: Area of a Single Loop or Petal (Problems 4, 8):} For curves like roses or lemniscates, the first step is to determine the bounds that trace a single loop. This is done by finding two consecutive angles $\theta$ for which $r=0$. Once the bounds are found, the approach is the same as Type 1.
    \item \textbf{Type 3: Area of a Specified Sector (Problems 5, 6):} Here, a graph visually defines the region. The task is to identify the starting and ending angles from the graph and use them as the bounds of integration in the standard area formula.
    \item \textbf{Type 4: Area Between Two Polar Curves (Problems 9, 10, 11, 14):} The key is to find the points of intersection by setting the two radius equations equal ($r_1=r_2$). These intersection angles become the bounds of integration. It's also crucial to determine which curve is the "outer" one ($r_{outer}$) and which is the "inner" one ($r_{inner}$) within those bounds. The formula is then $A = \int_{a}^{b} \frac{1}{2} (r_{outer}^2 - r_{inner}^2) \,d\theta$.
    \item \textbf{Type 5: Finding Intersection Points (Problems 12, 13):} This is a sub-problem for Type 4 but also a standalone skill. The general approach is two-fold: (1) Set $r_1=r_2$ and solve the resulting trigonometric equation for $\theta$. (2) Separately check if the pole ($r=0$) is an intersection point, as the curves might pass through the pole at different $\theta$ values.
    \item \textbf{Type 6: Arc Length of a Polar Curve (Problems 15, 16):} This type uses a different formula, $L = \int_{a}^{b} \sqrt{r^2 + (dr/d\theta)^2} \,d\theta$. The primary challenge is algebraic: after finding $dr/d\theta$, you must simplify the expression $r^2 + (dr/d\theta)^2$ under the square root, often using trigonometric identities, before you can attempt to integrate.
    \item \textbf{Type 7: Slope of a Tangent Line (Problems 17, 18, 19):} These problems require the formula for $\frac{dy}{dx}$. The approach is to calculate $dr/d\theta$, then substitute $r$, $dr/d\theta$, and the given value of $\theta$ into the formula and simplify.
    \item \textbf{Type 8: Horizontal and Vertical Tangents (Problem 20):} This is an extension of Type 7. Instead of evaluating the slope at a point, you must solve for $\theta$. Find the numerator of the slope formula, $\frac{dy}{d\theta}$, and set it to zero to find angles for horizontal tangents. Then find the denominator, $\frac{dx}{d\theta}$, and set it to zero to find angles for vertical tangents.
\end{itemize}

\section{Key Algebraic and Calculus Manipulations}
\begin{itemize}
    \item \textbf{Trigonometric Power-Reducing Formulas:} This is the most crucial technique for area problems. Squaring a polar function $r$ almost always results in terms like $\sin^2(\theta)$ or $\cos^2(\theta)$, which cannot be integrated directly. The formulas $\sin^2(x) = \frac{1-\cos(2x)}{2}$ and $\cos^2(x) = \frac{1+\cos(2x)}{2}$ are essential. This was used in Problems 2, 5, 6, 7, 8, and 9.
    \item \textbf{Squaring Binomials:} Many polar curves are of the form $r = a \pm b\cos\theta$ or $r = a \pm b\sin\theta$. When calculating area, $(r)^2$ requires expanding a binomial, e.g., $(1+\cos\theta)^2 = 1 + 2\cos\theta + \cos^2\theta$ in Problem 9. This step precedes the use of power-reducing formulas.
    \item \textbf{Solving Trigonometric Equations:} This is fundamental for finding intersection points (Type 4 \& 5) and locations of horizontal/vertical tangents (Type 8). In Problem 12, solving $\cos(\theta) = \sin(2\theta)$ required using the double-angle identity and then factoring to find all solutions.
    \item \textbf{Simplifying Radicals for Arc Length:} In Problem 16, simplifying $\sqrt{r^2 + (dr/d\theta)^2}$ was the main challenge. The expression simplified to $\sqrt{4\cos^2(\theta/2)}$, which correctly becomes $2|\cos(\theta/2)|$. Recognizing the need for an absolute value and splitting the integral accordingly was a key calculus step.
    \item \textbf{U-Substitution:} This standard integration technique was necessary for the arc length in Problem 15, where the integral $\int \theta\sqrt{\theta^2+4}\,d\theta$ was solved by letting $u=\theta^2+4$.
    \item \textbf{Using Symmetry:} For many problems (e.g., Problem 9, 11), the region is symmetric. Calculating the area of half the region (e.g., from $\theta=0$ to an intersection point) and doubling the result can simplify the arithmetic of evaluating the definite integral.
\end{itemize}

\part{"Cheatsheet" and Tips for Success}

\section{Quick-Reference Formulas}
\begin{itemize}
    \item \textbf{Area:} $A = \frac{1}{2} \int_{\alpha}^{\beta} r^2 \,d\theta$
    \item \textbf{Area Between Curves:} $A = \frac{1}{2} \int_{\alpha}^{\beta} (r_{outer}^2 - r_{inner}^2) \,d\theta$
    \item \textbf{Arc Length:} $L = \int_{\alpha}^{\beta} \sqrt{r^2 + (r')^2} \,d\theta$
    \item \textbf{Slope:} $\frac{dy}{dx} = \frac{r' \sin\theta + r \cos\theta}{r' \cos\theta - r \sin\theta}$
\end{itemize}

\section{Tricks and Shortcuts}
\begin{itemize}
    \item \textbf{Use Symmetry:} Always check for symmetry. If a region is symmetric about the x-axis or y-axis, you can integrate over a smaller interval and multiply. For example, for the area in Problem 9, integrating from $0$ to $\pi/3$ and doubling is easier than integrating from $-\pi/3$ to $\pi/3$.
    \item \textbf{Know Your Polar Graphs:} Recognizing that $r=a\cos\theta$ is a circle, $r=a(1\pm\cos\theta)$ is a cardioid, and $r=a\cos(n\theta)$ is a rose will help you set up bounds and visualize the problem correctly without needing to plot many points.
    \item \textbf{Area of a Rose Petal:} The area of one petal of $r=a\cos(n\theta)$ or $r=a\sin(n\theta)$ is always $\frac{\pi a^2}{4n}$ (if n is even) or $\frac{\pi a^2}{2n}$ (total area, n petals), but doing the integral is safer. For Problem 8, $r=\sin(2\theta)$, $a=1, n=2$, area is $\pi(1)^2 / (4*2) = \pi/8$. This is a great way to check your work.
\end{itemize}

\section{Common Pitfalls and How to Avoid Them}
\begin{itemize}
    \item \textbf{Forgetting $\frac{1}{2}$ in the Area Formula:} This is the most common mistake. Always write $\frac{1}{2}$ first when setting up an area integral.
    \item \textbf{Forgetting to Square $r$:} The formula is $\int \frac{1}{2} r^2 \,d\theta$, not $\int \frac{1}{2} r \,d\theta$.
    \item \textbf{Incorrect Bounds:} Finding the wrong bounds is a major source of error. For single loops, solve $r=0$. For areas between curves, solve $r_1=r_2$. Always sketch the graphs to verify your bounds make sense.
    \item \textbf{Mixing up $r_{outer}$ and $r_{inner}$:} For area between curves, test a point in the middle of your interval to see which $r$ is larger. If you get a negative area, you have likely swapped them.
    \item \textbf{Ignoring the Pole:} When finding intersection points, don't forget to check if both curves pass through the pole ($r=0$), as this can happen at different $\theta$ values. (See Problem 12).
    \item \textbf{Algebra Errors:} Be very careful when expanding expressions like $(a+b\cos\theta)^2$ and simplifying expressions under the square root for arc length.
\end{itemize}

\part{Conceptual Synthesis and The "Big Picture"}

\section{Thematic Connections}
The core theme of this topic is \textbf{adapting the tools of calculus to a coordinate system that fits the geometry of the problem}. We have already seen the power of integration as a tool for "summing infinite infinitesimal pieces" to find properties of a whole. In Cartesian coordinates, we summed infinitesimally thin rectangles ($dA = y\,dx$) to find area. Here, we've simply changed the shape of our fundamental "piece" to an infinitesimally thin circular sector ($dA = \frac{1}{2}r^2\,d\theta$).

This theme of choosing an appropriate analytical framework is one of the most powerful ideas in mathematics and science. It connects backward to our choice of using different methods for integration (u-sub vs. parts) based on the structure of the integrand. It connects forward to the use of cylindrical and spherical coordinates in multivariable calculus, which are extensions of polar coordinates used to simplify problems involving cylinders, cones, and spheres. The principle remains the same: describe the world using a language that makes its inherent symmetries obvious.

\section{Forward and Backward Links}
\begin{itemize}
    \item \textbf{Backward Link (Foundation):} This chapter is a direct application of definite integration from Calculus I. Without a solid understanding of setting up and evaluating integrals, none of these problems are possible. Furthermore, it breathes new life into trigonometry; identities that may have seemed abstract are now indispensable computational tools.
    \item \textbf{Forward Link (Application):} Understanding how to formulate area in polar coordinates is a crucial prerequisite for multivariable calculus (Calculus III). When evaluating double integrals over circular domains, we switch to polar coordinates. The differential area element $dA$ in Cartesian is $dx\,dy$, but in polar coordinates, it becomes $dA = r\,dr\,d\theta$. The factor of $r$ in this new area element is a direct consequence of the geometry of polar coordinates that we've explored here. This technique is fundamental for solving problems in physics, probability, and engineering that involve circular regions or radial distributions. [24]
\end{itemize}

\part{Real-World Application and Modeling}

\section{Concrete Scenarios}
While polar coordinates are essential in physics (orbital mechanics, electromagnetism) and engineering (antenna radiation patterns), their principles also appear in finance and economics.
\begin{enumerate}
    \item \textbf{Finance (Volatility Modeling):} In quantitative finance, stochastic processes are used to model asset prices. A model like the Ornstein-Uhlenbeck process, which describes the velocity of a particle, can be used to model mean-reverting interest rates or volatility. When analyzing the joint behavior of the asset price and its volatility, their state can be represented in a 2D "phase space." Using polar coordinates in this space allows analysts to separate the magnitude of deviation from a stable state (the radius, $r$) from its cyclical phase (the angle, $\theta$), which can simplify the analysis of financial derivatives that depend on these dynamics.
    \item \textbf{Economics (Urban Modeling):} Economists modeling the layout of cities sometimes use polar coordinates to represent land value. The pole represents the city center (Central Business District), and any point is described by its distance from the center ($r$) and its direction ($\theta$). A simple model might state that land value $V$ is a function of radius, e.g., $V(r) = V_0 e^{-kr}$, where value is highest at the center and decays exponentially. To find the total value of all land within a certain radius $R$, one would need to integrate this function over a circular area using the polar area element: Total Value = $\int_0^{2\pi} \int_0^R V(r) \, r \, dr \, d\theta$.
    \item \textbf{Statistics (Probability Distributions):} A very important application arises when calculating the normalizing constant for the normal (Gaussian) distribution, which is central to statistics, econometrics, and the Black-Scholes model for option pricing. The integral $\int_{-\infty}^{\infty} e^{-x^2} dx$ has no simple antiderivative. The solution involves squaring the integral and converting the resulting double integral into polar coordinates, which makes it easily solvable. This technique is a cornerstone of mathematical statistics. [24]
\end{enumerate}

\section{Model Problem Setup}
\textbf{Scenario:} An economic analyst is studying urban sprawl. They model the population density of a city as a function that depends only on the distance $r$ from the city center. Their model is given by the function $\rho(r) = \frac{10000}{1 + r^2}$ people per square kilometer, where $r$ is in kilometers. The analyst wants to calculate the total population of the city, which they define as the population living within a 10 km radius of the center.

\textbf{Model Setup:}
\begin{itemize}
    \item \textbf{Variables:}
        \begin{itemize}
            \item $\rho(r)$: Population density as a function of radius $r$.
            \item $r$: Distance from the city center in km.
            \item $\theta$: Angle, which density is independent of.
            \item $R$: The radius of the city, $R=10$ km.
            \item $P$: Total population.
        \end{itemize}
    \item \textbf{Formulation:} To find the total population, we must integrate the density function over the area of the city. Since the problem has circular symmetry, polar coordinates are ideal. The area of a small "polar rectangle" (an annulus sector) is $dA = r \, dr \, d\theta$. The population in this small area is $dP = \rho(r) \, dA = \rho(r) \, r \, dr \, d\theta$.
    \item \textbf{Integral:} To find the total population $P$, we integrate this expression over the entire city disk (from $r=0$ to $R=10$ and $\theta=0$ to $2\pi$).
    \[ P = \int_{0}^{2\pi} \int_{0}^{10} \rho(r) \cdot r \, dr \, d\theta \]
    Substituting the density function:
    \[ P = \int_{0}^{2\pi} \int_{0}^{10} \frac{10000}{1 + r^2} \cdot r \, dr \, d\theta \]
    This integral would then be solved (using a u-substitution for the inner integral) to find the city's total population.
\end{itemize}

\part{Common Variations and Untested Concepts}
The provided homework set was comprehensive but did not include problems dealing with the inner loop of a limaçon, which is a classic polar area problem.

\subsection{Area of the Inner Loop of a Limaçon}
\textbf{Explanation:} A limaçon is a curve of the form $r = a + b\cos\theta$ or $r = a + b\sin\theta$. If $|a/b| < 1$, the graph has both an outer and an inner loop. The inner loop is traced when the radius $r$ becomes negative. To find its area, you must first find the interval of $\theta$ values that trace this loop by finding the angles for which $r$ passes through zero and is negative between them. [6, 8, 9]

\textbf{Worked-Out Example:}
\textbf{Problem:} Find the area of the inner loop of the limaçon $r = 1 + 2\cos\theta$.

\textbf{Solution:}
\begin{enumerate}
    \item \textbf{Find bounds for the inner loop:} The inner loop is traced when $r < 0$. First, find where $r=0$.
    \[ 1 + 2\cos\theta = 0 \implies \cos\theta = -\frac{1}{2} \]
    In the interval $[0, 2\pi]$, the solutions are $\theta = \frac{2\pi}{3}$ and $\theta = \frac{4\pi}{3}$. Between these two angles, $\cos\theta$ is less than $-1/2$, so $r$ is negative. These are our bounds.
    \item \textbf{Setup the area integral:}
    \[ A = \int_{2\pi/3}^{4\pi/3} \frac{1}{2} (1 + 2\cos\theta)^2 \,d\theta \]
    \item \textbf{Expand and simplify:}
    \[ A = \frac{1}{2} \int_{2\pi/3}^{4\pi/3} (1 + 4\cos\theta + 4\cos^2\theta) \,d\theta \]
    Using power reduction, $\cos^2\theta = \frac{1+\cos(2\theta)}{2}$:
    \[ A = \frac{1}{2} \int_{2\pi/3}^{4\pi/3} \left(1 + 4\cos\theta + 4\left(\frac{1+\cos(2\theta)}{2}\right)\right) \,d\theta \]
    \[ A = \frac{1}{2} \int_{2\pi/3}^{4\pi/3} (1 + 4\cos\theta + 2 + 2\cos(2\theta)) \,d\theta = \frac{1}{2} \int_{2\pi/3}^{4\pi/3} (3 + 4\cos\theta + 2\cos(2\theta)) \,d\theta \]
    \item \textbf{Integrate:}
    \[ A = \frac{1}{2} \left[ 3\theta + 4\sin\theta + \sin(2\theta) \right]_{2\pi/3}^{4\pi/3} \]
    \item \textbf{Evaluate:}
    \[ A = \frac{1}{2} \left[ \left(3\frac{4\pi}{3} + 4\sin\frac{4\pi}{3} + \sin\frac{8\pi}{3}\right) - \left(3\frac{2\pi}{3} + 4\sin\frac{2\pi}{3} + \sin\frac{4\pi}{3}\right) \right] \]
    \[ A = \frac{1}{2} \left[ \left(4\pi - 4\frac{\sqrt{3}}{2} + \frac{\sqrt{3}}{2}\right) - \left(2\pi + 4\frac{\sqrt{3}}{2} - \frac{\sqrt{3}}{2}\right) \right] \]
    \[ A = \frac{1}{2} \left[ (4\pi - \frac{3\sqrt{3}}{2}) - (2\pi + \frac{3\sqrt{3}}{2}) \right] = \frac{1}{2} [2\pi - 3\sqrt{3}] = \pi - \frac{3\sqrt{3}}{2} \]
\end{enumerate}

\part{Advanced Diagnostic Testing: "Find the Flaw"}
For each problem below, a flawed solution is presented. Your task is to find the single critical error, explain it, and provide the correct solution.

\subsection{Problem 1}
\textbf{Question:} Find the area enclosed by the cardioid $r = 1 + \sin\theta$.

\textbf{Flawed Solution:}
The area is given by $A = \int_0^{2\pi} r^2 \,d\theta$.
\[ A = \int_0^{2\pi} (1 + \sin\theta)^2 \,d\theta = \int_0^{2\pi} (1 + 2\sin\theta + \sin^2\theta) \,d\theta \]
\[ A = \int_0^{2\pi} \left(1 + 2\sin\theta + \frac{1-\cos(2\theta)}{2}\right) \,d\theta = \int_0^{2\pi} \left(\frac{3}{2} + 2\sin\theta - \frac{1}{2}\cos(2\theta)\right) \,d\theta \]
\[ A = \left[\frac{3}{2}\theta - 2\cos\theta - \frac{1}{4}\sin(2\theta)\right]_0^{2\pi} \]
\[ A = (\frac{3}{2}(2\pi) - 2\cos(2\pi)) - (0 - 2\cos(0)) = (3\pi - 2) - (-2) = 3\pi \]
\textbf{Final Answer:} $3\pi$.

\subsection{Problem 2}
\textbf{Question:} Find the total arc length of the circle $r = 4\cos\theta$.

\textbf{Flawed Solution:}
The circle is traced from $0$ to $2\pi$. We have $r = 4\cos\theta$ and $r' = -4\sin\theta$.
The arc length is $L = \int_0^{2\pi} \sqrt{r^2 + (r')^2} \,d\theta$.
\[ L = \int_0^{2\pi} \sqrt{(4\cos\theta)^2 + (-4\sin\theta)^2} \,d\theta \]
\[ L = \int_0^{2\pi} \sqrt{16\cos^2\theta + 16\sin^2\theta} \,d\theta = \int_0^{2\pi} \sqrt{16} \,d\theta = \int_0^{2\pi} 4 \,d\theta \]
\[ L = [4\theta]_0^{2\pi} = 4(2\pi) - 0 = 8\pi \]
\textbf{Final Answer:} $8\pi$.

\subsection{Problem 3}
\textbf{Question:} Find the area of the region inside the circle $r=2$ and outside the cardioid $r = 2(1-\cos\theta)$.

\textbf{Flawed Solution:}
Find intersections: $2 = 2(1-\cos\theta) \implies 1 = 1-\cos\theta \implies \cos\theta = 0$. So $\theta = \pm \pi/2$.
The circle is the outer curve.
\[ A = \frac{1}{2} \int_{-\pi/2}^{\pi/2} \left( (2(1-\cos\theta))^2 - (2)^2 \right) \,d\theta \]
\[ A = \frac{1}{2} \int_{-\pi/2}^{\pi/2} (4(1-2\cos\theta+\cos^2\theta) - 4) \,d\theta = \int_{-\pi/2}^{\pi/2} (2(1-2\cos\theta+\cos^2\theta) - 2) \,d\theta \]
\[ A = \int_{-\pi/2}^{\pi/2} (-4\cos\theta + 2\cos^2\theta) \,d\theta = \int_{-\pi/2}^{\pi/2} (-4\cos\theta + 1 + \cos(2\theta)) \,d\theta \]
\[ A = [-4\sin\theta + \theta + \frac{1}{2}\sin(2\theta)]_{-\pi/2}^{\pi/2} = (-4 + \frac{\pi}{2}) - (4 - \frac{\pi}{2}) = \pi - 8 \]
\textbf{Final Answer:} $\pi - 8$.

\subsection{Problem 4}
\textbf{Question:} Find the slope of the tangent line to $r = \sin\theta$ at $\theta = \pi/4$.

\textbf{Flawed Solution:}
We have $r = \sin\theta$ and $r' = \cos\theta$.
The slope formula is $\frac{dy}{dx} = \frac{r' \sin\theta + r \cos\theta}{r' \cos\theta + r \sin\theta}$.
At $\theta = \pi/4$, $r = \sin(\pi/4) = \sqrt{2}/2$ and $r' = \cos(\pi/4) = \sqrt{2}/2$.
\[ \frac{dy}{dx} = \frac{(\sqrt{2}/2)(\sqrt{2}/2) + (\sqrt{2}/2)(\sqrt{2}/2)}{(\sqrt{2}/2)(\sqrt{2}/2) + (\sqrt{2}/2)(\sqrt{2}/2)} = \frac{1/2 + 1/2}{1/2 + 1/2} = \frac{1}{1} = 1 \]
\textbf{Final Answer:} $1$.

\subsection{Problem 5}
\textbf{Question:} Find all intersection points of $r = 2\sin\theta$ and $r = 2\cos\theta$.

\textbf{Flawed Solution:}
Set the equations equal: $2\sin\theta = 2\cos\theta \implies \tan\theta = 1$.
The only solution in $[0, \pi)$ is $\theta = \pi/4$.
At this angle, $r = 2\sin(\pi/4) = 2(\sqrt{2}/2) = \sqrt{2}$.
So there is only one intersection point.
\textbf{Final Answer:} $(\sqrt{2}, \pi/4)$.

\end{document}