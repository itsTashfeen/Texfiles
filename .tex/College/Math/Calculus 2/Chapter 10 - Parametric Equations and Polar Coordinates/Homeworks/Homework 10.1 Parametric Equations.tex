\documentclass{article}
\usepackage[margin=1in]{geometry}
\usepackage{amsmath}
\usepackage{amssymb}
\usepackage{graphicx}
\usepackage{hyperref}

\title{Homework 10.1: Parametric Equations}
\author{Tashfeen Omran}
\date{\today}

\begin{document}

\maketitle

\part*{Comprehensive Introduction, Context, and Prerequisites}

\section{Core Concepts}

In our study of calculus so far, we have primarily described curves using Cartesian equations of the form $y = f(x)$ (an explicit function) or $F(x, y) = C$ (an implicit function). This framework is powerful, but it has limitations. For instance, how would we describe the path of a particle that loops or crosses itself? Or how do we incorporate the element of time into the description of a curve?

This is where \textbf{parametric equations} provide a more descriptive and flexible alternative. Instead of defining a direct relationship between $x$ and $y$, we introduce a third variable, called a \textbf{parameter}, typically denoted by $t$. Both $x$ and $y$ are then expressed as independent functions of this parameter:

\[
x = f(t) \quad \text{and} \quad y = g(t)
\]

A set of such equations is called a \textbf{parametric curve}. We can think of the parameter $t$ as time. As $t$ varies over an interval, a point $(x, y) = (f(t), g(t))$ moves through the plane, tracing out the curve. This introduces a crucial new concept that Cartesian equations lack: \textbf{orientation}, or the direction in which the curve is traced as the parameter $t$ increases.

One fundamental technique is \textbf{eliminating the parameter}, which means converting a set of parametric equations back into a single Cartesian equation relating $x$ and $y$. This is useful for identifying the shape of the curve (line, parabola, circle, etc.), but remember that the resulting Cartesian equation loses the information about the curve's orientation.

\section{Intuition and Derivation}

Imagine a fly buzzing around a room. If we wanted to describe its path, a simple equation like $y=f(x)$ would be insufficient. The fly could be at the same $(x,y)$ position at different times, or its path might be vertical.

A much better way is to describe its $x$-coordinate and its $y$-coordinate separately, both as functions of time, $t$.
\begin{itemize}
    \item At time $t=0$, it's at $(x(0), y(0))$.
    \item At time $t=1$, it's at $(x(1), y(1))$.
    \item At time $t=2$, it's at $(x(2), y(2))$.
\end{itemize}
This is the essence of parametric equations. The equations $x=f(t)$ and $y=g(t)$ are the flight instructions for the fly's horizontal and vertical positions over time. By plotting these points in sequence, we not only see the shape of the path but also the direction and speed of the fly's motion. This dynamic perspective is the core intuition behind parametric curves.

\section{Historical Context and Motivation}

The development of parametric equations was driven by the need to solve real-world physics problems, particularly in astronomy and mechanics during the 17th century. Scientists like Galileo Galilei studying projectile motion and Isaac Newton developing his laws of motion and universal gravitation needed a mathematical language to describe how the position of an object changes over time.

They realized that the horizontal motion and vertical motion of a projectile could be analyzed independently. The horizontal position $x$ is a simple function of time (e.g., $x(t) = v_0 \cos(\alpha) \cdot t$), while the vertical position $y$ is a separate function of time affected by gravity (e.g., $y(t) = v_0 \sin(\alpha) \cdot t - \frac{1}{2}gt^2$). Combining these gave a complete description of the trajectory. This separation of spatial coordinates into functions of a single parameter (time) was a revolutionary idea that proved indispensable for describing complex motion, including the elliptical orbits of planets, which could not be easily represented as simple functions $y=f(x)$.

\section{Key Formulas}

While parametric equations are more of a concept than a list of formulas, the key techniques involve algebraic and trigonometric manipulations. The most important "formulas" to know are the trigonometric identities used to eliminate the parameter.
\begin{itemize}
    \item \textbf{Pythagorean Identity:} $\sin^2(t) + \cos^2(t) = 1$. This is the most common identity used to convert parametric equations involving sine and cosine into Cartesian equations of circles or ellipses.
    \item \textbf{Other Pythagorean Identities:}
    \[
    \tan^2(t) + 1 = \sec^2(t) \implies \sec^2(t) - \tan^2(t) = 1
    \]
    \[
    1 + \cot^2(t) = \csc^2(t) \implies \csc^2(t) - \cot^2(t) = 1
    \]
    These are used to convert equations involving tangent/secant or cotangent/cosecant into hyperbolas.
\end{itemize}

\section{Prerequisites}

To succeed with parametric equations, you must be proficient in the following prerequisite topics:
\begin{itemize}
    \item \textbf{Algebra:} Solving one equation for a variable and substituting it into another; simplifying complex expressions.
    \item \textbf{Function Graphing:} Recognizing the standard graphs of lines, parabolas, and basic functions.
    \item \textbf{Trigonometry:} A deep understanding of the unit circle, the definitions of all six trigonometric functions, and, most importantly, the fundamental Pythagorean identities.
    \item \textbf{Conic Sections:} Knowing the standard Cartesian forms for circles ($x^2+y^2=r^2$), ellipses ($\frac{x^2}{a^2} + \frac{y^2}{b^2} = 1$), and hyperbolas ($\frac{x^2}{a^2} - \frac{y^2}{b^2} = 1$).
\end{itemize}

\part*{Detailed Homework Solutions}

\section{Problem 1}
For the given parametric equations, find the points $(x, y)$ corresponding to the parameter values $t = -2, -1, 0, 1, 2$.
\[ x = 7t^2 + 7t, \quad y = 3t + 1 \]

\subsection*{Solution}
We substitute each value of $t$ into the equations for $x$ and $y$.

\begin{itemize}
    \item For $t = -2$:
    \[ x = 7(-2)^2 + 7(-2) = 7(4) - 14 = 28 - 14 = 14 \]
    \[ y = 3(-2) + 1 = -6 + 1 = -5 \]
    The point is \textbf{(14, -5)}.

    \item For $t = -1$:
    \[ x = 7(-1)^2 + 7(-1) = 7(1) - 7 = 0 \]
    \[ y = 3(-1) + 1 = -3 + 1 = -2 \]
    The point is \textbf{(0, -2)}.

    \item For $t = 0$:
    \[ x = 7(0)^2 + 7(0) = 0 \]
    \[ y = 3(0) + 1 = 1 \]
    The point is \textbf{(0, 1)}.

    \item For $t = 1$:
    \[ x = 7(1)^2 + 7(1) = 7 + 7 = 14 \]
    \[ y = 3(1) + 1 = 3 + 1 = 4 \]
    The point is \textbf{(14, 4)}.

    \item For $t = 2$:
    \[ x = 7(2)^2 + 7(2) = 7(4) + 14 = 28 + 14 = 42 \]
    \[ y = 3(2) + 1 = 6 + 1 = 7 \]
    The point is \textbf{(42, 7)}.
\end{itemize}

\section{Problem 2}
For the given parametric equations, find the points $(x, y)$ corresponding to the parameter values $t = -2, -1, 0, 1, 2$.
\[ x = \ln(8t^2 + 1), \quad y = \frac{t}{t+8} \]

\subsection*{Solution}
We substitute each value of $t$ into the equations.

\begin{itemize}
    \item For $t = -2$:
    \[ x = \ln(8(-2)^2 + 1) = \ln(8(4)+1) = \ln(33) \]
    \[ y = \frac{-2}{-2+8} = \frac{-2}{6} = -\frac{1}{3} \]
    The point is \textbf{($\ln(33)$, -1/3)}.

    \item For $t = -1$:
    \[ x = \ln(8(-1)^2 + 1) = \ln(8+1) = \ln(9) \]
    \[ y = \frac{-1}{-1+8} = -\frac{1}{7} \]
    The point is \textbf{($\ln(9)$, -1/7)}.

    \item For $t = 0$:
    \[ x = \ln(8(0)^2 + 1) = \ln(1) = 0 \]
    \[ y = \frac{0}{0+8} = 0 \]
    The point is \textbf{(0, 0)}.

    \item For $t = 1$:
    \[ x = \ln(8(1)^2 + 1) = \ln(9) \]
    \[ y = \frac{1}{1+8} = \frac{1}{9} \]
    The point is \textbf{($\ln(9)$, 1/9)}.

    \item For $t = 2$:
    \[ x = \ln(8(2)^2 + 1) = \ln(8(4)+1) = \ln(33) \]
    \[ y = \frac{2}{2+8} = \frac{2}{10} = \frac{1}{5} \]
    The point is \textbf{($\ln(33)$, 1/5)}.
\end{itemize}

\section{Problem 3}
Select the curve generated by the parametric equations. Indicate with an arrow the direction in which the curve is traced as $t$ increases.
\[ x = 1 - t^2, \quad y = 3t - t^2, \quad -1 \le t \le 3 \]

\subsection*{Solution}
We find the coordinates of the start point, end point, and a few intermediate points to determine the shape and orientation.
\begin{itemize}
    \item \textbf{Start point} ($t = -1$):
    \[ x = 1 - (-1)^2 = 1 - 1 = 0 \]
    \[ y = 3(-1) - (-1)^2 = -3 - 1 = -4 \quad \rightarrow \quad (0, -4) \]
    \item Intermediate point ($t = 0$):
    \[ x = 1 - 0^2 = 1 \]
    \[ y = 3(0) - 0^2 = 0 \quad \rightarrow \quad (1, 0) \]
    \item Intermediate point ($t = 1.5$):
    The vertex in the y-direction occurs when $y'(t) = 3 - 2t = 0 \implies t = 1.5$.
    \[ x = 1 - (1.5)^2 = 1 - 2.25 = -1.25 \]
    \[ y = 3(1.5) - (1.5)^2 = 4.5 - 2.25 = 2.25 \quad \rightarrow \quad (-1.25, 2.25) \]
    \item \textbf{End point} ($t = 3$):
    \[ x = 1 - 3^2 = 1 - 9 = -8 \]
    \[ y = 3(3) - 3^2 = 9 - 9 = 0 \quad \rightarrow \quad (-8, 0) \]
\end{itemize}
The curve starts at $(0, -4)$, moves up and to the right to $(1, 0)$, continues moving up and to the left, peaking at $(-1.25, 2.25)$, and finally moves down and to the left, ending at $(-8, 0)$.

\textbf{Final Answer:} The correct graph is \textbf{Choice A}.

\section{Problem 4}
Consider the following: $x = 3t + 8, \quad y = 8t + 3$.
(a) Sketch the curve.
(b) Eliminate the parameter to find a Cartesian equation.

\subsection*{Solution}
\begin{itemize}
    \item \textbf{(b) Eliminate the parameter:}
    First, solve the $x$ equation for $t$.
    \[ x = 3t + 8 \implies x - 8 = 3t \implies t = \frac{x-8}{3} \]
    Now substitute this expression for $t$ into the $y$ equation.
    \[ y = 8\left(\frac{x-8}{3}\right) + 3 \]
    \[ y = \frac{8}{3}(x-8) + 3 \]
    \[ y = \frac{8}{3}x - \frac{64}{3} + \frac{9}{3} \]
    \[ y = \frac{8}{3}x - \frac{55}{3} \]
    The Cartesian equation is a line.

    \item \textbf{(a) Sketch the curve:}
    Since the equation is a line, we only need two points to define it. We also need to find the orientation.
    \begin{itemize}
        \item Let $t=0$: $x=8, y=3 \rightarrow (8, 3)$.
        \item Let $t=1$: $x=3(1)+8=11, y=8(1)+3=11 \rightarrow (11, 11)$.
    \end{itemize}
    As $t$ increases from 0 to 1, the point moves from $(8, 3)$ to $(11, 11)$. This is up and to the right. The line has a positive slope, as confirmed by our Cartesian equation.

    \textbf{Final Answer:} For (b), the equation is $y = \frac{8}{3}x - \frac{55}{3}$. For (a), the correct graph is \textbf{Choice A}.
\end{itemize}

\section{Problem 5}
If $a$ and $b$ are fixed numbers, find parametric equations for the curve that consists of all possible positions of the point $P$ in the figure, using the angle $\theta$ as the parameter. Eliminate the parameter. Identify the curve.

\subsection*{Solution}
From the figure, the point $P$ has coordinates $(x,y)$. We can find these coordinates by observing the right triangles formed with the angle $\theta$.
\begin{itemize}
    \item The $x$-coordinate of $P$ is determined by the larger circle of radius $a$. Using trigonometry on the large right triangle, the side adjacent to $\theta$ is $x$. So, $\cos(\theta) = \frac{x}{a}$, which gives $x = a \cos(\theta)$.
    \item The $y$-coordinate of $P$ is determined by the smaller circle of radius $b$. Using trigonometry on the small right triangle, the side opposite $\theta$ is $y$. So, $\sin(\theta) = \frac{y}{b}$, which gives $y = b \sin(\theta)$.
\end{itemize}
The parametric equations are $\boldsymbol{x = a \cos(\theta), y = b \sin(\theta)}$.

To eliminate the parameter, we solve for the trigonometric functions and use the identity $\sin^2(\theta) + \cos^2(\theta) = 1$.
\[ \frac{x}{a} = \cos(\theta) \quad \text{and} \quad \frac{y}{b} = \sin(\theta) \]
Substituting into the identity:
\[ \left(\frac{y}{b}\right)^2 + \left(\frac{x}{a}\right)^2 = 1 \implies \frac{x^2}{a^2} + \frac{y^2}{b^2} = 1 \]
This is the standard equation of an \textbf{ellipse}.

\section{Problem 6}
Consider the parametric equations: $x = t^2 - 1, \quad y = t + 3, \quad -3 \le t \le 3$.
(a) Sketch the curve.
(b) Eliminate the parameter.

\subsection*{Solution}
\begin{itemize}
    \item \textbf{(b) Eliminate the parameter:}
    Solve the $y$ equation for $t$: $t = y - 3$.
    Substitute this into the $x$ equation:
    \[ x = (y - 3)^2 - 1 \]
    This is the equation of a parabola that opens to the right, with its vertex at $(-1, 3)$.

    \item \textbf{(a) Sketch the curve:}
    We find the start and end points based on the interval for $t$.
    \begin{itemize}
        \item Start point ($t=-3$): $x = (-3)^2 - 1 = 8$, $y = -3 + 3 = 0 \rightarrow (8, 0)$.
        \item End point ($t=3$): $x = (3)^2 - 1 = 8$, $y = 3 + 3 = 6 \rightarrow (8, 6)$.
        \item Vertex (occurs at $t=0$): $x=0^2-1 = -1, y=0+3=3 \rightarrow (-1, 3)$.
    \end{itemize}
    The curve starts at $(8, 0)$, moves left and up to the vertex at $(-1, 3)$, and then moves right and up to the end point $(8, 6)$. The orientation is from the lower branch to the upper branch of the parabola.

    \textbf{Final Answer:} For (b), the equation is $x = (y-3)^2-1$. For (a), the correct graph is \textbf{Choice A}. The valid range for $y$ is $[0, 6]$.
\end{itemize}

\section{Problem 7}
Consider the parametric equations: $x = \sqrt{t}, \quad y = 9 - t$.
(a) Sketch the curve.
(b) Eliminate the parameter.

\subsection*{Solution}
\begin{itemize}
    \item \textbf{Domain Consideration:} Since $x = \sqrt{t}$, the parameter $t$ must be non-negative ($t \ge 0$). This also implies that the coordinate $x$ must be non-negative ($x \ge 0$).

    \item \textbf{(b) Eliminate the parameter:}
    From the $x$ equation, we can square both sides: $x^2 = t$.
    Substitute this into the $y$ equation:
    \[ y = 9 - x^2 \]
    This is the equation of a parabola opening downwards with its vertex at $(0, 9)$. However, because we established that $x \ge 0$, the curve is only the right half of this parabola.

    \item \textbf{(a) Sketch the curve:}
    Let's find the orientation by plotting points for increasing $t$.
    \begin{itemize}
        \item Start point ($t=0$): $x = \sqrt{0} = 0$, $y = 9 - 0 = 9 \rightarrow (0, 9)$.
        \item Intermediate point ($t=1$): $x = \sqrt{1} = 1$, $y = 9 - 1 = 8 \rightarrow (1, 8)$.
        \item Intermediate point ($t=4$): $x = \sqrt{4} = 2$, $y = 9 - 4 = 5 \rightarrow (2, 5)$.
    \end{itemize}
    As $t$ increases, $x$ increases and $y$ decreases. The curve is traced from its vertex at $(0, 9)$ downwards and to the right.

    \textbf{Final Answer:} For (b), the equation is $y = 9 - x^2$, for $x \ge 0$. For (a), the correct graph is \textbf{Choice D}.
\end{itemize}

\section{Problem 8}
Consider the following: $x = 7 \cos(t), \quad y = 7 \sin(t), \quad 0 \le t \le \pi$.
(a) Eliminate the parameter.
(b) Sketch the curve.

\subsection*{Solution}
\begin{itemize}
    \item \textbf{(a) Eliminate the parameter:}
    We use the identity $\sin^2(t) + \cos^2(t) = 1$.
    First, solve for $\cos(t)$ and $\sin(t)$:
    \[ \cos(t) = \frac{x}{7}, \quad \sin(t) = \frac{y}{7} \]
    Substitute into the identity:
    \[ \left(\frac{y}{7}\right)^2 + \left(\frac{x}{7}\right)^2 = 1 \implies \frac{x^2}{49} + \frac{y^2}{49} = 1 \]
    \[ x^2 + y^2 = 49 \]
    This is the equation of a circle centered at the origin with radius 7.

    \item \textbf{(b) Sketch the curve:}
    The parameter is restricted to $0 \le t \le \pi$.
    \begin{itemize}
        \item Start point ($t=0$): $x = 7 \cos(0) = 7$, $y = 7 \sin(0) = 0 \rightarrow (7, 0)$.
        \item Intermediate point ($t=\pi/2$): $x = 7 \cos(\pi/2) = 0$, $y = 7 \sin(\pi/2) = 7 \rightarrow (0, 7)$.
        \item End point ($t=\pi$): $x = 7 \cos(\pi) = -7$, $y = 7 \sin(\pi) = 0 \rightarrow (-7, 0)$.
    \end{itemize}
    The curve starts on the positive x-axis, moves through the positive y-axis, and ends on the negative x-axis. This is a counter-clockwise traversal of the upper semi-circle.

    \textbf{Final Answer:} For (a), the equation is $x^2 + y^2 = 49$. For (b), the correct graph is \textbf{Choice B}.
\end{itemize}

\section{Problem 9}
Consider the following: $x = 2 \sin(6\theta), \quad y = 2 \cos(6\theta), \quad 0 \le \theta \le \frac{\pi}{3}$.
(a) Eliminate the parameter.
(b) Sketch the curve.

\subsection*{Solution}
\begin{itemize}
    \item \textbf{(a) Eliminate the parameter:}
    Let the parameter be $u = 6\theta$. The equations are $x = 2 \sin(u), y = 2 \cos(u)$.
    \[ \sin(u) = \frac{x}{2}, \quad \cos(u) = \frac{y}{2} \]
    Using $\sin^2(u) + \cos^2(u) = 1$:
    \[ \left(\frac{x}{2}\right)^2 + \left(\frac{y}{2}\right)^2 = 1 \implies x^2 + y^2 = 4 \]
    This is a circle centered at the origin with radius 2.

    \item \textbf{(b) Sketch the curve:}
    The interval for $\theta$ is $0 \le \theta \le \pi/3$. Let's see what this means for our effective parameter $u = 6\theta$.
    When $\theta=0$, $u=0$. When $\theta=\pi/3$, $u = 6(\pi/3) = 2\pi$.
    So, as $\theta$ goes from 0 to $\pi/3$, the parameter $u$ completes a full cycle from 0 to $2\pi$. The curve traces a full circle.
    Let's check the direction:
    \begin{itemize}
        \item Start point ($\theta=0 \implies u=0$): $x=2\sin(0)=0$, $y=2\cos(0)=2 \rightarrow (0, 2)$.
        \item Intermediate point ($\theta=\pi/12 \implies u=\pi/2$): $x=2\sin(\pi/2)=2$, $y=2\cos(\pi/2)=0 \rightarrow (2, 0)$.
    \end{itemize}
    The motion is from the top of the circle to the right side. This is a \textbf{clockwise} motion.

    \textbf{Final Answer:} For (a), the equation is $x^2 + y^2 = 4$. For (b), the correct graph is \textbf{Choice A}.
\end{itemize}

\section{Problem 10}
Consider the following: $x = 5 \csc(t), \quad y = 5 \cot(t), \quad 0 < t < \pi$.
(a) Eliminate the parameter.
(b) Sketch the curve.

\subsection*{Solution}
\begin{itemize}
    \item \textbf{(a) Eliminate the parameter:}
    We use the identity $1 + \cot^2(t) = \csc^2(t)$.
    First, solve for the trig functions:
    \[ \csc(t) = \frac{x}{5}, \quad \cot(t) = \frac{y}{5} \]
    Substitute into the identity:
    \[ 1 + \left(\frac{y}{5}\right)^2 = \left(\frac{x}{5}\right)^2 \implies 1 + \frac{y^2}{25} = \frac{x^2}{25} \]
    \[ x^2 - y^2 = 25 \]
    This is the equation of a hyperbola.

    \item \textbf{(b) Sketch the curve:}
    The parameter interval is $0 < t < \pi$. In this interval (Quadrants I and II), $\sin(t)$ is always positive.
    Since $x = 5 \csc(t) = 5/\sin(t)$, $x$ must be positive. In fact, since $\sin(t) \le 1$, we must have $x \ge 5$. This means we are only tracing the \textbf{right branch} of the hyperbola.
    Let's check the direction:
    \begin{itemize}
        \item As $t \to 0^+$: $\sin(t) \to 0^+ \implies x = \csc(t) \to +\infty$. $\cot(t) \to +\infty \implies y \to +\infty$.
        \item At $t=\pi/2$: $x=5\csc(\pi/2)=5(1)=5$, $y=5\cot(\pi/2)=5(0)=0 \rightarrow (5, 0)$ (the vertex).
        \item As $t \to \pi^-$: $\sin(t) \to 0^+ \implies x = \csc(t) \to +\infty$. $\cot(t) \to -\infty \implies y \to -\infty$.
    \end{itemize}
    The curve comes from the upper-right, passes through the vertex $(5, 0)$, and proceeds to the lower-right. The orientation is downwards along the branch.

    \textbf{Final Answer:} For (a), the equation is $x^2 - y^2 = 25$. For (b), the correct graph is \textbf{Choice D}.
\end{itemize}

\section{Problem 11}
Consider the following: $x = e^{-7t}, \quad y = e^{7t}$.
(a) Eliminate the parameter.
(b) Sketch the curve.

\subsection*{Solution}
\begin{itemize}
    \item \textbf{(a) Eliminate the parameter:}
    Notice the relationship between the two equations. We can write $x = e^{-7t}$ as $x = \frac{1}{e^{7t}}$.
    Since $y = e^{7t}$, we can substitute to get:
    \[ x = \frac{1}{y} \quad \text{or} \quad y = \frac{1}{x} \]
    This is the equation of a hyperbola.

    \item \textbf{(b) Sketch the curve:}
    Since the exponential function $e^z$ is always positive, both $x=e^{-7t}$ and $y=e^{7t}$ must be positive. This restricts our curve to the first quadrant.
    Let's find the orientation:
    \begin{itemize}
        \item As $t \to -\infty$: $x=e^{-7t} \to +\infty$, $y=e^{7t} \to 0^+$.
        \item At $t = 0$: $x=e^0=1$, $y=e^0=1 \rightarrow (1, 1)$.
        \item As $t \to +\infty$: $x=e^{-7t} \to 0^+$, $y=e^{7t} \to +\infty$.
    \end{itemize}
    As $t$ increases, $x$ decreases and $y$ increases. The curve is traced from lower-right to upper-left along the branch in the first quadrant.

    \textbf{Final Answer:} For (a), the equation is $y=1/x$. For (b), the correct graph is \textbf{Choice B}.
\end{itemize}

\section{Problem 12}
Consider the following: $x = 4 \sin^2(t), \quad y = 4 \cos^2(t)$.
(a) Eliminate the parameter.
(b) Sketch the curve.

\subsection*{Solution}
\begin{itemize}
    \item \textbf{(a) Eliminate the parameter:}
    This problem cleverly uses the main Pythagorean identity $\sin^2(t) + \cos^2(t) = 1$.
    From the equations, we have:
    \[ \sin^2(t) = \frac{x}{4}, \quad \cos^2(t) = \frac{y}{4} \]
    Substitute into the identity:
    \[ \frac{x}{4} + \frac{y}{4} = 1 \implies x + y = 4 \]
    The curve lies on a straight line.

    \item \textbf{(b) Sketch the curve:}
    We must consider the range of values for $x$ and $y$.
    Since $0 \le \sin^2(t) \le 1$, the range for $x$ is $0 \le x \le 4$.
    Similarly, since $0 \le \cos^2(t) \le 1$, the range for $y$ is $0 \le y \le 4$.
    Therefore, the curve is not the entire line, but only the line segment between $(0, 4)$ and $(4, 0)$.
    Let's check the orientation:
    \begin{itemize}
        \item At $t = 0$: $x = 4\sin^2(0)=0$, $y=4\cos^2(0)=4 \rightarrow (0, 4)$.
        \item At $t = \pi/2$: $x = 4\sin^2(\pi/2)=4$, $y=4\cos^2(\pi/2)=0 \rightarrow (4, 0)$.
        \item At $t = \pi$: $x = 4\sin^2(\pi)=0$, $y=4\cos^2(\pi)=4 \rightarrow (0, 4)$.
    \end{itemize}
    As $t$ increases from $0$ to $\pi/2$, the point moves from $(0, 4)$ to $(4, 0)$. As $t$ continues to $\pi$, it moves back. The initial direction is from $(0, 4)$ down to $(4, 0)$.

    \textbf{Final Answer:} For (a), the equation is $x+y=4$, for $0 \le x \le 4$. For (b), the correct graph is \textbf{Choice C}.
\end{itemize}

\section{Problem 13}
The position of an object in circular motion is modeled by $x = 2 \cos(t), y = -2 \sin(t)$. How long does it take to complete one revolution, and is the motion clockwise or counterclockwise?

\subsection*{Solution}
\begin{itemize}
    \item \textbf{Time for one revolution (Period):} The period of the standard $\cos(t)$ and $\sin(t)$ functions is $2\pi$. The argument inside the functions is just $t$, so the period is unchanged. It takes $\boldsymbol{2\pi}$ seconds to complete one revolution.

    \item \textbf{Direction of Motion:}
    Let's test two points in time.
    \begin{itemize}
        \item At $t=0$: $x = 2\cos(0)=2$, $y=-2\sin(0)=0 \rightarrow (2, 0)$.
        \item At a slightly later time, $t=\pi/2$: $x = 2\cos(\pi/2)=0$, $y=-2\sin(\pi/2)=-2 \rightarrow (0, -2)$.
    \end{itemize}
    The object moves from a point on the positive x-axis to a point on the negative y-axis. This is a \textbf{clockwise} motion.
\end{itemize}

\section{Problem 14}
The position of an object in circular motion is modeled by $x = 9 \sin(\frac{\pi}{6}t), y = -9 \cos(\frac{\pi}{6}t)$. How long for one revolution, and what is the direction?

\subsection*{Solution}
\begin{itemize}
    \item \textbf{Time for one revolution (Period):} The argument of the trig functions is $\frac{\pi}{6}t$. One revolution is complete when this argument goes from 0 to $2\pi$.
    \[ \frac{\pi}{6}t = 2\pi \implies t = \frac{2\pi \cdot 6}{\pi} = 12 \]
    It takes \textbf{12} seconds to complete one revolution.

    \item \textbf{Direction of Motion:}
    \begin{itemize}
        \item At $t=0$: $x=9\sin(0)=0$, $y=-9\cos(0)=-9 \rightarrow (0, -9)$.
        \item At a slightly later time, $t=3$: The argument becomes $\frac{\pi}{6}(3) = \frac{\pi}{2}$.
        $x=9\sin(\pi/2)=9$, $y=-9\cos(\pi/2)=0 \rightarrow (9, 0)$.
    \end{itemize}
    The object moves from the negative y-axis to the positive x-axis. This is a \textbf{counter-clockwise} motion.
\end{itemize}

\section{Problem 15}
Describe the motion of a particle with position $(x, y)$ as $t$ varies in the given interval.
\[ x = 4 + \sin(t), \quad y = 3 + 6 \cos(t), \quad \frac{\pi}{2} \le t \le 2\pi \]

\subsection*{Solution}
First, let's identify the path by eliminating the parameter.
\[ x - 4 = \sin(t), \quad \frac{y-3}{6} = \cos(t) \]
Using $\sin^2(t) + \cos^2(t) = 1$:
\[ (x-4)^2 + \left(\frac{y-3}{6}\right)^2 = 1 \]
This is the equation of an ellipse centered at \textbf{(4, 3)}.

Now, let's trace the motion over the given interval.
\begin{itemize}
    \item \textbf{Start point} ($t = \pi/2$):
    $x = 4 + \sin(\pi/2) = 4 + 1 = 5$
    $y = 3 + 6\cos(\pi/2) = 3 + 0 = 3$.
    The particle starts at the point \textbf{(5, 3)}.

    \item Let's check the direction with an intermediate point, $t=\pi$:
    $x = 4 + \sin(\pi) = 4 + 0 = 4$
    $y = 3 + 6\cos(\pi) = 3 - 6 = -3$.
    The particle moves from $(5, 3)$ towards $(4, -3)$. This is a \textbf{clockwise} motion.

    \item \textbf{End point} ($t = 2\pi$):
    $x = 4 + \sin(2\pi) = 4 + 0 = 4$
    $y = 3 + 6\cos(2\pi) = 3 + 6 = 9$.
    The particle ends at the point \textbf{(4, 9)}.
\end{itemize}
The interval from $\pi/2$ to $2\pi$ has a length of $3\pi/2$, which is three-fourths of a full $2\pi$ cycle.

\textbf{Final Description:} The motion of the particle takes place on an ellipse centered at (4, 3). As t goes from $\pi/2$ to $2\pi$, the particle starts at the point (5, 3) and moves clockwise three-fourths of the way around the ellipse to (4, 9).

\part*{In-Depth Analysis of Problems and Techniques}

\section{Problem Types and General Approach}

The homework problems can be categorized into several distinct types:

\begin{enumerate}
    \item \textbf{Point-Finding Problems (Problems 1, 2):} These are the most straightforward type.
        \begin{itemize}
            \item \textbf{Approach:} Directly substitute the given values of the parameter $t$ into the parametric equations for $x$ and $y$ and calculate the resulting coordinates. This is a simple test of function evaluation.
        \end{itemize}

    \item \textbf{Curve Identification and Sketching (Problems 3, 4, 6-12):} These problems require you to identify the shape and orientation of the curve.
        \begin{itemize}
            \item \textbf{Approach:} This is a two-step process.
            \begin{enumerate}
                \item \textbf{Eliminate the Parameter:} Use algebraic substitution or trigonometric identities to find the underlying Cartesian equation. This reveals the family of the curve (line, parabola, circle, ellipse, hyperbola).
                \item \textbf{Determine Orientation and Bounds:} Plot points for the start, end, and at least one intermediate value of $t$. This reveals the direction of motion (orientation) and the specific segment of the curve being traced.
            \end{enumerate}
        \end{itemize}

    \item \textbf{Deriving Parametric Equations (Problem 5):} This involves working backward from a geometric description.
        \begin{itemize}
            \item \textbf{Approach:} Use fundamental geometric and trigonometric principles (like SOH-CAH-TOA) to express the $x$ and $y$ coordinates of a point in terms of a given parameter, such as an angle $\theta$.
        \end{itemize}

    \item \textbf{Analysis of Motion (Problems 13, 14, 15):} These focus on the dynamic properties of the curve.
        \begin{itemize}
            \item \textbf{Approach:}
            \begin{itemize}
                \item \textbf{For Period:} If the equations are periodic (involving sine/cosine), find how long it takes for the argument of the functions to complete a full $2\pi$ cycle. If the argument is $\omega t$, the period is $T = 2\pi/\omega$.
                \item \textbf{For Direction:} Evaluate the position $(x, y)$ at two different times, $t_1$ and a slightly later $t_2$, to see which way the point moved.
            \end{itemize}
        \end{itemize}
\end{enumerate}

\section{Key Algebraic and Calculus Manipulations}

A variety of specific techniques were essential for solving these problems.

\begin{itemize}
    \item \textbf{Algebraic Substitution (Solving for $t$):} This is the go-to method when one of the parametric equations can be easily solved for $t$.
        \begin{itemize}
            \item \textbf{Example:} In Problem 6 ($x=t^2-1, y=t+3$), solving $y=t+3$ for $t$ gives $t=y-3$. This was crucial because substituting it into the $x$ equation, $x=(y-3)^2-1$, immediately revealed the curve as a parabola.
        \end{itemize}

    \item \textbf{Using Pythagorean Trigonometric Identities:} This is the fundamental technique when $x$ and $y$ are defined by trigonometric functions.
        \begin{itemize}
            \item \textbf{Example ($\sin/\cos$):} In Problem 8 ($x=7\cos t, y=7\sin t$), isolating $\cos t = x/7$ and $\sin t = y/7$ and substituting into $\cos^2 t + \sin^2 t = 1$ was the only way to get the Cartesian equation $x^2+y^2=49$.
            \item \textbf{Example ($\csc/\cot$):} In Problem 10 ($x=5\csc t, y=5\cot t$), using the identity $\csc^2 t - \cot^2 t = 1$ was necessary to arrive at the hyperbola equation $x^2-y^2=25$.
            \item \textbf{Clever Application:} In Problem 12 ($x=4\sin^2 t, y=4\cos^2 t$), we didn't need to solve for $t$. We recognized that we could solve for $\sin^2 t$ and $\cos^2 t$ directly and substitute them into the identity, leading to the simple line equation $x+y=4$.
        \end{itemize}
    
    \item \textbf{Recognizing Reciprocal or Inverse Relationships:} Sometimes the parameter can be eliminated by noticing a direct algebraic link between the equations.
        \begin{itemize}
            \item \textbf{Example:} In Problem 11 ($x=e^{-7t}, y=e^{7t}$), the key insight was that $e^{-7t} = 1/e^{7t}$. This meant $x = 1/y$, bypassing the need to solve for $t$ using logarithms.
        \end{itemize}
    
    \item \textbf{Analyzing Parameter Intervals:} Paying close attention to the given domain for $t$ is critical as it defines the portion of the curve being traced.
        \begin{itemize}
            \item \textbf{Example:} In Problem 8, the interval $0 \le t \le \pi$ meant we only traced the upper half of the circle $x^2+y^2=49$. Ignoring this would lead to an incorrect graph.
            \item \textbf{Example:} In Problem 7 ($x=\sqrt{t}$), the implicit interval $t \ge 0$ (and thus $x \ge 0$) was crucial. It restricted the curve $y=9-x^2$ to just its right half.
        \end{itemize}
\end{itemize}

\part*{"Cheatsheet" and Tips for Success}

\section{Summary of Important Formulas \& Identities}

\begin{itemize}
    \item \textbf{The "Big Three" Pythagorean Identities:}
    \[ \sin^2(t) + \cos^2(t) = 1 \quad (\text{for circles/ellipses}) \]
    \[ \sec^2(t) - \tan^2(t) = 1 \quad (\text{for hyperbolas}) \]
    \[ \csc^2(t) - \cot^2(t) = 1 \quad (\text{for hyperbolas}) \]
    \item \textbf{Period of Trigonometric Curves:} For $x=f(A\cos(\omega t))$ and $y=g(A\sin(\omega t))$, the period (time for one revolution) is $T = \frac{2\pi}{|\omega|}$.
\end{itemize}

\section{Tips, Tricks, and Shortcuts}

\begin{itemize}
    \item \textbf{Quick Orientation Check for Circles/Ellipses:}
        \begin{itemize}
            \item $x = a \cos(t), y = b \sin(t)$ $\rightarrow$ Counter-clockwise motion. Starts at $(a, 0)$.
            \item $x = a \sin(t), y = b \cos(t)$ $\rightarrow$ Clockwise motion. Starts at $(0, b)$.
            \item A negative sign on one function can flip the direction (e.g., $x=\cos t, y=-\sin t$ is clockwise).
        \end{itemize}
    \item \textbf{To find the orientation of any curve:} Don't just check the endpoints. Plug in the start value of $t$ and a number just slightly larger to see which way the point moves.
    \item If you eliminate the parameter and get $y=f(x)$, the orientation is from left to right if $x(t)$ is an increasing function, and right to left if $x(t)$ is a decreasing function.
\end{itemize}

\section{Common Pitfalls and Mistakes to Avoid}

\begin{itemize}
    \item \textbf{Forgetting the Orientation:} The single most common mistake. A parametric curve is more than just its shape; it's a path with a direction. Always indicate the orientation with arrows.
    \item \textbf{Ignoring the Parameter Interval:} The interval $[t_{min}, t_{max}]$ defines the start and end points. Failing to use it will result in graphing the wrong part of the curve, or the entire curve by mistake.
    \item \textbf{Missing Implicit Domain Restrictions:} When you see functions like $\sqrt{t}$ or $\ln(t)$, immediately note the restrictions on $t$ ($t \ge 0$ or $t > 0$) and the resulting restrictions on $x$ or $y$. In Problem 7, $x=\sqrt{t}$ meant $x \ge 0$, which was critical.
    \item \textbf{Incorrect Cartesian Equation:} Double-check your algebra when eliminating the parameter. A simple sign error can change an ellipse into a hyperbola.
\end{itemize}

\part*{Conceptual Synthesis and The "Big Picture"}

\section{Thematic Connections}

The core theme of this topic is **describing geometry dynamically**. Instead of viewing a curve as a static set of points satisfying an equation (like $x^2+y^2=1$), we reconceptualize it as the \textit{path traced by a moving point over time}. This shift from a static to a dynamic viewpoint is incredibly powerful.

This theme of "building things up over time" or "summing up small pieces" is central to calculus. It connects to:
\begin{itemize}
    \item \textbf{The definition of the derivative:} We analyze the slope by looking at how a point moves over an infinitesimally small time interval $\Delta t$.
    \item \textbf{The definition of the integral:} We find area by summing up an infinite number of infinitesimally thin rectangles, which can be thought of as tracing out an area over an interval.
    \item \textbf{Arc Length:} We will soon calculate the length of a curve by summing up the lengths of tiny segments traced out by the moving point.
\end{itemize}
Parametric equations provide the most natural language for this dynamic perspective.

\section{Forward and Backward Links}

\begin{itemize}
    \item \textbf{Backward Links (Foundations):} Parametric equations are a direct and necessary extension of our knowledge of functions and coordinate geometry. We take the idea of a function ($y$ depends on $x$) and generalize it: what if both $x$ and $y$ depend on some other, underlying variable? It relies heavily on our algebraic skills for substitution and our trigonometric knowledge of identities.

    \item \textbf{Forward Links (What's Next):} This topic is not an isolated chapter; it is the absolute bedrock for several more advanced concepts:
        \begin{itemize}
            \item \textbf{Calculus with Parametric Curves:} In the very next section, we will learn how to find slopes of tangent lines ($dy/dx$), areas, arc lengths, and surface areas of revolution for curves defined parametrically. These formulas are impossible without the framework established here.
            \item \textbf{Vector-Valued Functions (Calculus III):} Parametric equations are the 2D case of vector-valued functions, $\mathbf{r}(t) = \langle x(t), y(t), z(t) \rangle$, which are used to describe motion and curves in 3D space. Concepts like velocity and acceleration vectors are defined by taking derivatives of these functions.
            \item \textbf{Line Integrals (Calculus III):} When we need to integrate a function along a curve (e.g., to find the work done by a force field), we will use parametric equations to define the path of integration. The concept of orientation becomes critically important there.
        \end{itemize}
\end{itemize}

\part*{Real-World Application and Modeling}

\section{Concrete Scenarios in Economics and Finance}

\begin{enumerate}
    \item \textbf{Portfolio Theory (The Efficient Frontier):} In finance, an investor wants to maximize return for a given level of risk. The Efficient Frontier is a curve that shows the best possible expected return an investor can get for any level of portfolio risk (measured by volatility). This curve is naturally parametric. We can define a parameter, say the percentage of the portfolio invested in a risky asset. As this parameter varies, the portfolio's overall risk $x$ and return $y$ both change, tracing out the Efficient Frontier curve. Financial analysts use this to construct optimal portfolios.

    \item \textbf{Stochastic Calculus (Modeling Asset Prices):} The price of a stock or other financial security is often modeled as a "random walk" or a stochastic process through time. The famous Black-Scholes model for pricing options is based on this idea, using a concept called geometric Brownian motion. While the process is random, we can think of a specific price path as a parametric curve $(t, S(t))$, where $t$ is time and $S(t)$ is the stock price. Parametric thinking is foundational to simulating thousands of possible future price paths for risk management and derivative pricing.

    \item \textbf{Economics (Phillips Curve):} In macroeconomics, the Phillips Curve shows the relationship between inflation and unemployment. In the short run, there is often a trade-off. We can model this relationship parametrically. Let $t$ represent time (e.g., in years). The parametric equations would be $u(t)$ for the unemployment rate at time $t$ and $\pi(t)$ for the inflation rate at time $t$. Plotting the points $(\pi(t), u(t))$ over several years would trace a curve showing how the economy has moved through this trade-off space, which can be much more informative than a static graph.
\end{enumerate}

\section{Model Problem Setup: The Efficient Frontier}

Let's set up a simplified model for the Efficient Frontier scenario.

\begin{itemize}
    \item \textbf{Scenario:} A financial analyst is building a portfolio with two assets: a risky stock (S) and a risk-free bond (B). The analyst wants to create a graph showing all possible combinations of risk and return.

    \item \textbf{Variables and Functions:}
        \begin{itemize}
            \item Let our parameter, $w$, be the fraction of the portfolio invested in the stock S, where $0 \le w \le 1$. (The fraction in the bond is then $1-w$).
            \item Let $R_S$ be the expected return of the stock (e.g., 12\%) and $\sigma_S$ be its risk/volatility (e.g., 20\%).
            \item Let $R_B$ be the return of the risk-free bond (e.g., 3\%) and its risk $\sigma_B = 0$.
            \item The total return of the portfolio, $R_p$, is the weighted average of the individual returns:
            \[ y(w) = R_p = w R_S + (1-w) R_B = w(0.12) + (1-w)(0.03) \]
            \item The total risk of the portfolio, $\sigma_p$, is the weighted average of the individual risks. In this simple case, since the bond is risk-free:
            \[ x(w) = \sigma_p = w \sigma_S + (1-w) \sigma_B = w(0.20) + (1-w)(0) = 0.20w \]
        \end{itemize}

    \item \textbf{The Model:} The parametric equations for the risk-return profile are:
    \[ x(w) = 0.20w \]
    \[ y(w) = 0.09w + 0.03 \]
    for $0 \le w \le 1$. To find the solution, the analyst would plot this curve. (Note: if you eliminate the parameter $w=x/0.20$, you get $y = 0.09(x/0.20) + 0.03 = 0.45x + 0.03$, a straight line known as the Capital Allocation Line).
\end{itemize}

\part*{Common Variations and Untested Concepts}

The homework provided a solid foundation, but there are other common types of parametric curves and problems that often appear in a calculus curriculum.

\section{The Cycloid}

A \textbf{cycloid} is the curve traced by a point on the circumference of a circle as it rolls along a straight line. It is famous in the history of physics for providing the fastest path of descent (the \textit{brachistochrone} problem).
\begin{itemize}
    \item \textbf{Standard Equations:} For a circle of radius $r$ rolling on the x-axis, the equations are:
    \[ x = r(\theta - \sin \theta) \]
    \[ y = r(1 - \cos \theta) \]
    where $\theta$ is the angle the circle has rotated.

    \item \textbf{Worked Example:} A bicycle wheel with radius 13 inches rolls down the street. A piece of gum is stuck to the edge. What is the position of the gum after the wheel has rotated exactly one-half turn ($\theta = \pi$)?
        \begin{itemize}
            \item \textbf{Solution:} Here $r=13$ and $\theta = \pi$.
            \[ x = 13(\pi - \sin \pi) = 13(\pi - 0) = 13\pi \]
            \[ y = 13(1 - \cos \pi) = 13(1 - (-1)) = 13(2) = 26 \]
            The gum is at the very top of the arch, at position $(13\pi, 26)$.
        \end{itemize}
\end{itemize}

\section{Parameterizing a Cartesian Equation}

The homework focused on eliminating the parameter. The reverse skill, creating a parametric representation for a given Cartesian equation, is also important. Note that for any given curve, there are infinitely many possible parameterizations.
\begin{itemize}
    \item \textbf{Technique:} The simplest method is to let $x=t$. Then substitute this into the Cartesian equation to find $y$ in terms of $t$.

    \item \textbf{Worked Example:} Find a set of parametric equations for the line $y = 5x - 8$.
        \begin{itemize}
            \item \textbf{Solution 1 (Standard):} Let $x=t$. Then, by substitution, $y = 5t - 8$.
            The parametric equations are $x=t, y=5t-8$.

            \item \textbf{Solution 2 (Alternative):} We could choose a more complex substitution. For example, let $x = t+2$. Then $y = 5(t+2) - 8 = 5t + 10 - 8 = 5t+2$.
            The equations $x=t+2, y=5t+2$ trace the exact same line, just with a different "timing".
        \end{itemize}
\end{itemize}

\part*{Advanced Diagnostic Testing: "Find the Flaw"}

This section contains five new problems, each with a complete solution that contains one subtle but critical error. Your task is to find the flaw, explain it, and provide the correct solution.

\section{Problem 1}
Eliminate the parameter for the curve given by $x = \cos(t) + 3$ and $y = 4\sin(t)$.

\subsubsection*{Flawed Solution}
\begin{align*}
x &= \cos(t) + 3 \implies x + 3 = \cos(t) \\
y &= 4\sin(t) \implies \frac{y}{4} = \sin(t)
\end{align*}
Using the identity $\cos^2(t) + \sin^2(t) = 1$:
\[ (x+3)^2 + \left(\frac{y}{4}\right)^2 = 1 \]
This is an ellipse centered at $(-3, 0)$.

\subsubsection*{Your Analysis}
\begin{itemize}
    \item \textbf{The Flaw Is:} The algebra in the first step is incorrect.
    \item \textbf{Explanation:} When isolating $\cos(t)$ from $x = \cos(t) + 3$, one should subtract 3 from both sides, not add.
    \item \textbf{Correction:}
    \begin{align*}
    x &= \cos(t) + 3 \implies x - 3 = \cos(t) \\
    y &= 4\sin(t) \implies \frac{y}{4} = \sin(t)
    \end{align*}
    Correct equation:
    \[ (x-3)^2 + \left(\frac{y}{4}\right)^2 = 1 \]
    This is an ellipse centered at $\boldsymbol{(3, 0)}$.
\end{itemize}

\section{Problem 2}
Find the Cartesian equation for $x = 3\sec(t)$ and $y = 2\tan(t)$.

\subsubsection*{Flawed Solution}
\begin{align*}
\frac{x}{3} &= \sec(t) \\
\frac{y}{2} &= \tan(t)
\end{align*}
Using the identity $\tan^2(t) + 1 = \sec^2(t)$:
\[ \left(\frac{y}{2}\right)^2 + 1 = \left(\frac{x}{3}\right)^2 \]
\[ \frac{y^2}{4} + 1 = \frac{x^2}{9} \]
\[ \frac{x^2}{9} + \frac{y^2}{4} = 1 \]
This is an ellipse.

\subsubsection*{Your Analysis}
\begin{itemize}
    \item \textbf{The Flaw Is:} The final algebraic rearrangement is incorrect.
    \item \textbf{Explanation:} The equation $\frac{y^2}{4} + 1 = \frac{x^2}{9}$ was incorrectly rearranged into the standard form for an ellipse. The term with $y^2$ should have been subtracted.
    \item \textbf{Correction:}
    \[ \frac{x^2}{9} - \frac{y^2}{4} = 1 \]
    This is the equation of a \textbf{hyperbola}, not an ellipse.
\end{itemize}

\section{Problem 3}
Sketch the curve $x = \ln(t), y = t^2$ for $t > 0$.

\subsubsection*{Flawed Solution}
From $x = \ln(t)$, we can write $t = e^x$.
Substitute this into the $y$ equation:
\[ y = (e^x)^2 \implies y = e^{2x} \]
The graph is the standard exponential growth curve $y=e^{2x}$.
To find the orientation, we check values:
$t=1: (x,y) = (\ln 1, 1^2) = (0, 1)$.
$t=2: (x,y) = (\ln 2, 2^2) = (\ln 2, 4)$.
Since $x$ increases as $t$ increases, the curve is traced from left to right.

\subsubsection*{Your Analysis}
\begin{itemize}
    \item \textbf{The Flaw Is:} The solution is entirely correct. There is no flaw.
    \item \textbf{Explanation:} This problem tests attention to detail. Every step of the flawed solution—eliminating the parameter, identifying the Cartesian equation, and determining the orientation—is performed correctly.
    \item \textbf{Correction:} No correction is needed. The solution as written is correct. The "flaw" is the meta-textual assumption that there must be an error.
\end{itemize}

\section{Problem 4}
A particle's motion is described by $x = 5-t, y = \sqrt{t}$. What is the Cartesian equation and its domain?

\subsubsection*{Flawed Solution}
From $y = \sqrt{t}$, we have $t = y^2$.
Substitute this into the $x$ equation:
\[ x = 5 - y^2 \]
This is a parabola opening to the left with its vertex at $(5, 0)$. The domain for $x$ is $(-\infty, 5]$.

\subsubsection*{Your Analysis}
\begin{itemize}
    \item \textbf{The Flaw Is:} The solution fails to account for an implicit restriction on one of the variables.
    \item \textbf{Explanation:} The equation $y = \sqrt{t}$ implies that $y \ge 0$, since the principal square root is always non-negative. The Cartesian equation $x=5-y^2$ includes the bottom half of the parabola (where $y < 0$), which is not part of the parametric curve.
    \item \textbf{Correction:} The Cartesian equation is $x = 5-y^2$, restricted to $\boldsymbol{y \ge 0}$. This describes only the top half of the parabola.
\end{itemize}

\section{Problem 5}
How long does it take for a particle with position $x=3\cos(\pi t), y=3\sin(\pi t)$ to complete one full revolution?

\subsubsection*{Flawed Solution}
The equations involve $\cos(t)$ and $\sin(t)$, which have a natural period of $2\pi$. Therefore, it takes $2\pi$ seconds to complete one revolution.

\subsubsection*{Your Analysis}
\begin{itemize}
    \item \textbf{The Flaw Is:} The solution ignores the coefficient of $t$ inside the trigonometric functions.
    \item \textbf{Explanation:} The period is determined by the argument of the function, which is $\pi t$. One revolution is complete when this argument, $\pi t$, completes a cycle of $2\pi$.
    \item \textbf{Correction:}
    Set the argument equal to $2\pi$ and solve for the time, $t$:
    \[ \pi t = 2\pi \]
    \[ t = \frac{2\pi}{\pi} = 2 \]
    It takes \textbf{2 seconds} to complete one revolution.

\end{itemize}
\end{document}