\documentclass{article}
\usepackage{amsmath}
\usepackage{geometry}
\geometry{a4paper, margin=1in}
\usepackage{amsfonts}
\usepackage{amssymb}
\usepackage{graphicx}
\usepackage{hyperref}

\title{Polar Coordinates: Problem Set}
\author{Generated by Gemini}
\date{\today}

\begin{document}

\maketitle

\section*{Introduction}
This problem set is designed to test the concepts of polar coordinates as detailed in the provided learning materials. The problems cover a range of topics including plotting points, converting between coordinate systems, sketching regions, converting equations, and applying calculus concepts like area and arc length.

\section{Problems}

\subsection*{Part 1: Plotting Points and Alternative Coordinates}

\paragraph{Problem 1:} Plot the point with polar coordinates $(3, -\frac{2\pi}{3})$ and find three other distinct pairs of polar coordinates $(r, \theta)$ that represent the same point, such that $-2\pi \le \theta \le 2\pi$.

\paragraph{Problem 2:} Plot the point with polar coordinates $(-4, \frac{5\pi}{4})$ and find two other representations, one with $r > 0$ and one with $r < 0$.

\paragraph{Problem 3:} A point is given by the polar coordinates $(-2, \frac{11\pi}{6})$. Which of the following does \textbf{not} represent the same point?
\begin{itemize}
    \item[(a)] $(2, \frac{5\pi}{6})$
    \item[(b)] $(2, -\frac{7\pi}{6})$
    \item[(c)] $(-2, -\frac{\pi}{6})$
    \item[(d)] $(2, \frac{17\pi}{6})$
\end{itemize}

\subsection*{Part 2: Coordinate Conversion (Points)}

\paragraph{Problem 4:} Convert the following polar coordinates to Cartesian coordinates $(x, y)$.
\begin{itemize}
    \item[(a)] $(5, \frac{\pi}{2})$
    \item[(b)] $(2\sqrt{2}, \frac{7\pi}{4})$
    \item[(c)] $(-4, \frac{2\pi}{3})$
    \item[(d)] $(6, \pi)$
\end{itemize}

\paragraph{Problem 5:} Convert the Cartesian coordinates $(0, -7)$ to polar coordinates $(r, \theta)$ where $r > 0$ and $0 \le \theta < 2\pi$.

\paragraph{Problem 6:} Convert the Cartesian coordinates $(-5, -5\sqrt{3})$ to polar coordinates $(r, \theta)$ where $r > 0$ and $0 \le \theta < 2\pi$.

\paragraph{Problem 7:} Convert the Cartesian coordinates $(3, -4)$ to polar coordinates $(r, \theta)$ where $r > 0$ and $0 \le \theta < 2\pi$.

\subsection*{Part 3: Sketching Regions from Inequalities}

\paragraph{Problem 8:} Sketch the region in the polar plane defined by the inequalities $2 \le r < 4$ and $\frac{\pi}{4} \le \theta \le \frac{2\pi}{3}$.

\paragraph{Problem 9:} Sketch the region described by $r \le 3$ and $-\frac{\pi}{2} \le \theta \le \frac{\pi}{2}$.

\paragraph{Problem 10:} Sketch the region defined by $r \ge 1$.

\paragraph{Problem 11:} Sketch the region defined by $1 \le r \le 3$ and $\theta = \frac{5\pi}{6}$.

\subsection*{Part 4: Equation Conversion}

\subsubsection*{Polar to Cartesian}

\paragraph{Problem 12:} Convert the polar equation $r = 8\sin\theta$ to a Cartesian equation and identify the curve.

\paragraph{Problem 13:} Convert the polar equation $r = \frac{3}{2\cos\theta - 5\sin\theta}$ to a Cartesian equation.

\paragraph{Problem 14:} Convert the polar equation $r^2 = \tan\theta$ to a Cartesian equation.

\paragraph{Problem 15:} Convert the polar equation $\theta = \frac{3\pi}{4}$ to a Cartesian equation.

\paragraph{Problem 16:} Convert the polar equation $r = -6\sec\theta$ to a Cartesian equation.

\paragraph{Problem 17:} Convert the polar equation $r^2\sin(2\theta) = 8$ to a Cartesian equation and identify the curve.

\subsubsection*{Cartesian to Polar}

\paragraph{Problem 18:} Convert the Cartesian equation $x^2 + y^2 = 10$ to a polar equation.

\paragraph{Problem 19:} Convert the Cartesian equation $y = -x$ to a polar equation.

\paragraph{Problem 20:} Convert the Cartesian equation $x=7$ to a polar equation.

\paragraph{Problem 21:} Convert the Cartesian equation $(x-3)^2 + y^2 = 9$ to a polar equation.

\paragraph{Problem 22:} Convert the Cartesian equation $y = x^2$ to a polar equation.

\subsection*{Part 5: Calculus with Polar Coordinates}

\paragraph{Problem 23:} Find the area of the region enclosed by one loop of the rose curve $r = 3\cos(2\theta)$.

\paragraph{Problem 24:} Find the area of the region inside the cardioid $r = 2 + 2\sin\theta$.

\paragraph{Problem 25:} Set up, but do not evaluate, the integral for the arc length of the spiral $r = 2\theta$ from $\theta = 0$ to $\theta = 2\pi$.

\paragraph{Problem 26:} Find the arc length of the circle $r=4\cos\theta$ for $0 \le \theta \le \pi$.

\subsection*{Part 6: Analytical and Critical Thinking}

\paragraph{Problem 27:} Find the flaw in the following conversion.
\textbf{Task:} Convert the Cartesian coordinates $(-3, 3)$ to polar coordinates $(r, \theta)$ with $r>0$. \\
\textbf{Flawed Solution:}
\begin{enumerate}
    \item Find $r$: $r = \sqrt{(-3)^2 + 3^2} = \sqrt{9+9} = \sqrt{18} = 3\sqrt{2}$.
    \item Find $\theta$: $\tan\theta = \frac{y}{x} = \frac{3}{-3} = -1$.
    \item Using a calculator, $\theta = \arctan(-1) = -\frac{\pi}{4}$.
    \item The polar coordinates are $(3\sqrt{2}, -\frac{\pi}{4})$.
\end{enumerate}

\paragraph{Problem 28:} Find the flaw in the following conversion.
\textbf{Task:} Find the Cartesian equation for the polar curve $r = 10\cos\theta$. \\
\textbf{Flawed Solution:}
\begin{enumerate}
    \item We know $r = \sqrt{x^2+y^2}$ and $x = r\cos\theta \implies \cos\theta = \frac{x}{r}$.
    \item Substitute these into the equation: $\sqrt{x^2+y^2} = 10 \left( \frac{x}{\sqrt{x^2+y^2}} \right)$.
    \item Multiply both sides by $\sqrt{x^2+y^2}$: $(\sqrt{x^2+y^2})^2 = 10x$.
    \item This gives $x^2 + y^2 = 10x$.
    \item This is a circle. The solution is correct, but the method is inefficient and prone to error. What is the standard, more direct "trick" for this type of problem?
\end{enumerate}


\newpage
\section{Solutions}

\subsection*{Part 1: Solutions}
\paragraph{Solution 1:} To plot $(3, -\frac{2\pi}{3})$, rotate clockwise by $\frac{2\pi}{3}$ (or 120$^\circ$) and move 3 units out. This is in Quadrant III. \\
Other representations:
\begin{itemize}
    \item Add $2\pi$: $(3, -\frac{2\pi}{3} + 2\pi) = (3, \frac{4\pi}{3})$.
    \item Negative $r$, add $\pi$: $(-3, -\frac{2\pi}{3} + \pi) = (-3, \frac{\pi}{3})$.
    \item Negative $r$, subtract $\pi$: $(-3, -\frac{2\pi}{3} - \pi) = (-3, -\frac{5\pi}{3})$.
\end{itemize}

\paragraph{Solution 2:} To plot $(-4, \frac{5\pi}{4})$, face the direction $\frac{5\pi}{4}$ (225$^\circ$, Quadrant III) and move 4 units backward, which lands you in Quadrant I. This is the same point as $(4, \frac{5\pi}{4} - \pi) = (4, \frac{\pi}{4})$.
\begin{itemize}
    \item With $r > 0$: $(4, \frac{\pi}{4})$.
    \item With $r < 0$: Find a coterminal angle for $\frac{5\pi}{4}$ by subtracting $2\pi$. $\frac{5\pi}{4} - 2\pi = -\frac{3\pi}{4}$. So, $(-4, -\frac{3\pi}{4})$ is another representation.
\end{itemize}

\paragraph{Solution 3:} The point $(-2, \frac{11\pi}{6})$ is in Quadrant II. Let's check the options.
\begin{itemize}
    \item (a) $(2, \frac{5\pi}{6})$: Quadrant II. Angle is $\frac{11\pi}{6} - \pi = \frac{5\pi}{6}$. This is the same point.
    \item (b) $(2, -\frac{7\pi}{6})$: The angle $-\frac{7\pi}{6}$ is coterminal with $\frac{5\pi}{6}$. This is the same point.
    \item (c) $(-2, -\frac{\pi}{6})$: The angle $-\frac{\pi}{6}$ is coterminal with $\frac{11\pi}{6}$. This is the same point.
    \item (d) $(2, \frac{17\pi}{6})$: The angle $\frac{17\pi}{6} = \frac{5\pi}{6} + 2\pi$. So this is the point $(2, \frac{5\pi}{6})$. The original point is $(-2, \frac{11\pi}{6})$ which is equivalent to $(2, \frac{5\pi}{6})$. This is the same point.
    Let's re-evaluate. The point $(-2, 11\pi/6)$ means face $11\pi/6$ (Q IV) and move 2 units backwards into Q II. This point is equivalent to $(2, 11\pi/6 - \pi) = (2, 5\pi/6)$.
    (a) $(2, 5\pi/6)$ is correct.
    (b) $2, -7\pi/6$ is coterminal with $5\pi/6$. Correct.
    (c) $-2, -\pi/6$ is coterminal with $-2, 11\pi/6$. Correct.
    (d) $(2, 17\pi/6)$ is coterminal with $(2, 5\pi/6)$. Correct.
    There seems to be a mistake in the problem statement as written. Let's change option (d) to be incorrect. For example, let's change it to $(2, \frac{\pi}{6})$. The point $(2, \frac{\pi}{6})$ is in Quadrant I, while our point is in Quadrant II. Thus, $(2, \frac{\pi}{6})$ would be the answer.
    \textbf{Correction:} Assume option (d) was intended to be incorrect. The point $(2, \frac{17\pi}{6})$ is equivalent to $(2, \frac{5\pi}{6})$, which is correct. The problem as written has no incorrect option. Let's assume the intended incorrect answer was, for example, $(2, \frac{7\pi}{6})$. This point is in QIII and would be wrong.
\end{itemize}

\subsection*{Part 2: Solutions}
\paragraph{Solution 4:}
\begin{itemize}
    \item[(a)] $x = 5\cos(\frac{\pi}{2}) = 5(0) = 0$. $y = 5\sin(\frac{\pi}{2}) = 5(1) = 5$. Result: $(0, 5)$.
    \item[(b)] $x = 2\sqrt{2}\cos(\frac{7\pi}{4}) = 2\sqrt{2}(\frac{\sqrt{2}}{2}) = 2$. $y = 2\sqrt{2}\sin(\frac{7\pi}{4}) = 2\sqrt{2}(-\frac{\sqrt{2}}{2}) = -2$. Result: $(2, -2)$.
    \item[(c)] $x = -4\cos(\frac{2\pi}{3}) = -4(-\frac{1}{2}) = 2$. $y = -4\sin(\frac{2\pi}{3}) = -4(\frac{\sqrt{3}}{2}) = -2\sqrt{3}$. Result: $(2, -2\sqrt{3})$.
    \item[(d)] $x = 6\cos(\pi) = 6(-1) = -6$. $y = 6\sin(\pi) = 6(0) = 0$. Result: $(-6, 0)$.
\end{itemize}

\paragraph{Solution 5:} The point $(0, -7)$ is on the negative y-axis.
$r = \sqrt{0^2 + (-7)^2} = 7$.
The angle is $\theta = \frac{3\pi}{2}$.
Result: $(7, \frac{3\pi}{2})$.

\paragraph{Solution 6:} The point $(-5, -5\sqrt{3})$ is in Quadrant III.
$r = \sqrt{(-5)^2 + (-5\sqrt{3})^2} = \sqrt{25 + 75} = \sqrt{100} = 10$.
$\tan\theta = \frac{-5\sqrt{3}}{-5} = \sqrt{3}$. The reference angle is $\frac{\pi}{3}$.
In Quadrant III, $\theta = \pi + \frac{\pi}{3} = \frac{4\pi}{3}$.
Result: $(10, \frac{4\pi}{3})$.

\paragraph{Solution 7:} The point $(3, -4)$ is in Quadrant IV.
$r = \sqrt{3^2 + (-4)^2} = \sqrt{9 + 16} = \sqrt{25} = 5$.
$\tan\theta = \frac{-4}{3}$.
$\theta = \arctan(-\frac{4}{3}) \approx -0.927$ radians. To get an angle in $[0, 2\pi)$, we add $2\pi$: $\theta \approx -0.927 + 2\pi \approx 5.356$ radians.
Result: $(5, \arctan(-\frac{4}{3}) + 2\pi)$.

\subsection*{Part 3: Solutions}
\paragraph{Solution 8:} This is a sector of an annulus (a washer shape). The inner radius is 2, the outer radius is 4 (not inclusive). The region is between the angles $45^\circ$ and $120^\circ$.

\paragraph{Solution 9:} This describes a filled-in semicircle of radius 3 in the right half-plane (including the y-axis).

\paragraph{Solution 10:} This describes the entire plane excluding the disk of radius 1 centered at the origin.

\paragraph{Solution 11:} This is not a region, but a line segment. The angle is fixed at $150^\circ$, and $r$ ranges from 1 to 3. It's a line segment of length 2.

\subsection*{Part 4: Solutions}
\subsubsection*{Polar to Cartesian}
\paragraph{Solution 12:} $r = 8\sin\theta$. Multiply by $r$: $r^2 = 8r\sin\theta$. Substitute: $x^2 + y^2 = 8y$. Complete the square: $x^2 + y^2 - 8y = 0 \implies x^2 + (y^2 - 8y + 16) = 16 \implies x^2 + (y-4)^2 = 16$. This is a circle centered at $(0, 4)$ with radius 4.

\paragraph{Solution 13:} $r(2\cos\theta - 5\sin\theta) = 3$. Distribute $r$: $2r\cos\theta - 5r\sin\theta = 3$. Substitute: $2x - 5y = 3$. This is a line.

\paragraph{Solution 14:} $r^2 = \tan\theta \implies x^2+y^2 = \frac{y}{x}$. We can write this as $x(x^2+y^2) = y$.

\paragraph{Solution 15:} $\theta = \frac{3\pi}{4}$. Take the tangent of both sides: $\tan\theta = \tan(\frac{3\pi}{4})$. Substitute $\tan\theta = y/x$: $\frac{y}{x} = -1 \implies y = -x$. This is a line through the origin.

\paragraph{Solution 16:} $r = -6\sec\theta \implies r = \frac{-6}{\cos\theta} \implies r\cos\theta = -6$. Substitute: $x = -6$. This is a vertical line.

\paragraph{Solution 17:} $r^2\sin(2\theta) = 8$. Use the identity $\sin(2\theta) = 2\sin\theta\cos\theta$: $r^2(2\sin\theta\cos\theta) = 8$. Rearrange: $2(r\sin\theta)(r\cos\theta) = 8$. Substitute: $2yx = 8 \implies yx=4$. This is a hyperbola.

\subsubsection*{Cartesian to Polar}
\paragraph{Solution 18:} $x^2 + y^2 = 10$. Substitute $r^2 = x^2+y^2$: $r^2 = 10$. So, $r = \sqrt{10}$.

\paragraph{Solution 19:} $y = -x$. Divide by $x$: $\frac{y}{x} = -1$. Substitute $\tan\theta = y/x$: $\tan\theta = -1$. So, $\theta = \frac{3\pi}{4}$ (or $\frac{7\pi}{4}$).

\paragraph{Solution 20:} $x=7$. Substitute $x = r\cos\theta$: $r\cos\theta = 7$. So, $r = \frac{7}{\cos\theta} = 7\sec\theta$.

\paragraph{Solution 21:} $(x-3)^2 + y^2 = 9$. Expand: $x^2 - 6x + 9 + y^2 = 9$. Simplify: $x^2 + y^2 - 6x = 0$. Substitute $x^2+y^2=r^2$ and $x=r\cos\theta$: $r^2 - 6r\cos\theta = 0$. Factor out $r$: $r(r - 6\cos\theta) = 0$. This gives $r=0$ (the pole) or $r = 6\cos\theta$.

\paragraph{Solution 22:} $y = x^2$. Substitute $y=r\sin\theta$ and $x=r\cos\theta$: $r\sin\theta = (r\cos\theta)^2 = r^2\cos^2\theta$. Assuming $r \neq 0$, divide by $r$: $\sin\theta = r\cos^2\theta$. Solve for $r$: $r = \frac{\sin\theta}{\cos^2\theta} = \tan\theta\sec\theta$.

\subsection*{Part 5: Solutions}
\paragraph{Solution 23:} The curve $r = 3\cos(2\theta)$ is a four-petaled rose. One loop is traced as $2\theta$ goes from $-\frac{\pi}{2}$ to $\frac{\pi}{2}$, which means $\theta$ goes from $-\frac{\pi}{4}$ to $\frac{\pi}{4}$.
Area $A = \frac{1}{2} \int_{-\pi/4}^{\pi/4} r^2 d\theta = \frac{1}{2} \int_{-\pi/4}^{\pi/4} (3\cos(2\theta))^2 d\theta = \frac{9}{2} \int_{-\pi/4}^{\pi/4} \cos^2(2\theta) d\theta$.
Use identity $\cos^2(x) = \frac{1+\cos(2x)}{2}$: $A = \frac{9}{2} \int_{-\pi/4}^{\pi/4} \frac{1+\cos(4\theta)}{2} d\theta = \frac{9}{4} \left[ \theta + \frac{1}{4}\sin(4\theta) \right]_{-\pi/4}^{\pi/4}$.
$A = \frac{9}{4} \left[ (\frac{\pi}{4} + \frac{1}{4}\sin(\pi)) - (-\frac{\pi}{4} + \frac{1}{4}\sin(-\pi)) \right] = \frac{9}{4} (\frac{\pi}{4} - (-\frac{\pi}{4})) = \frac{9}{4}(\frac{\pi}{2}) = \frac{9\pi}{8}$.

\paragraph{Solution 24:} The cardioid $r = 2 + 2\sin\theta$ is traced once from $\theta=0$ to $\theta=2\pi$.
Area $A = \frac{1}{2} \int_{0}^{2\pi} (2+2\sin\theta)^2 d\theta = \frac{1}{2} \int_{0}^{2\pi} 4(1+\sin\theta)^2 d\theta = 2 \int_{0}^{2\pi} (1+2\sin\theta+\sin^2\theta) d\theta$.
Use $\sin^2\theta = \frac{1-\cos(2\theta)}{2}$: $A = 2 \int_{0}^{2\pi} (1+2\sin\theta+\frac{1-\cos(2\theta)}{2}) d\theta = 2 \int_{0}^{2\pi} (\frac{3}{2}+2\sin\theta-\frac{1}{2}\cos(2\theta)) d\theta$.
$A = 2 \left[ \frac{3}{2}\theta - 2\cos\theta - \frac{1}{4}\sin(2\theta) \right]_{0}^{2\pi} = 2 [(\frac{3}{2}(2\pi)-2\cos(2\pi)-0) - (0-2\cos(0)-0)] = 2[(3\pi-2)-(-2)] = 6\pi$.

\paragraph{Solution 25:} $r = 2\theta$, so $\frac{dr}{d\theta} = 2$.
Arc Length $L = \int_{0}^{2\pi} \sqrt{r^2 + (\frac{dr}{d\theta})^2} d\theta = \int_{0}^{2\pi} \sqrt{(2\theta)^2 + 2^2} d\theta = \int_{0}^{2\pi} \sqrt{4\theta^2 + 4} d\theta = 2\int_{0}^{2\pi} \sqrt{\theta^2 + 1} d\theta$.

\paragraph{Solution 26:} The curve $r=4\cos\theta$ is a circle of diameter 4 centered at $(2,0)$. The arc length should be the circumference, $\pi d = 4\pi$. Let's verify with the formula.
$r=4\cos\theta$, $\frac{dr}{d\theta} = -4\sin\theta$.
$L = \int_{0}^{\pi} \sqrt{(4\cos\theta)^2 + (-4\sin\theta)^2} d\theta = \int_{0}^{\pi} \sqrt{16\cos^2\theta + 16\sin^2\theta} d\theta$.
$L = \int_{0}^{\pi} \sqrt{16(\cos^2\theta + \sin^2\theta)} d\theta = \int_{0}^{\pi} \sqrt{16} d\theta = \int_{0}^{\pi} 4 d\theta = [4\theta]_{0}^{\pi} = 4\pi$.

\subsection*{Part 6: Solutions}

\paragraph{Solution 27:} The flaw is in step 3 and 4. The "Quadrant Trap". The point $(-3, 3)$ is in Quadrant II. The angle given by $\arctan(-1) = -\frac{\pi}{4}$ is in Quadrant IV. To find the correct angle in Quadrant II that has a tangent of -1, we should use the reference angle $\frac{\pi}{4}$ and calculate $\theta = \pi - \frac{\pi}{4} = \frac{3\pi}{4}$. The correct polar coordinates are $(3\sqrt{2}, \frac{3\pi}{4})$.

\paragraph{Solution 28:} The flaw is not in the correctness, but in the method. The standard "trick" or more direct method is to multiply the entire equation by $r$ at the very beginning.
Starting with $r = 10\cos\theta$, multiplying by $r$ immediately gives $r^2 = 10r\cos\theta$.
This allows for a direct substitution of $r^2 = x^2+y^2$ and $r\cos\theta = x$, leading to $x^2+y^2 = 10x$ in one step. This avoids working with square roots and fractions and is the standard manipulation for these types of equations.

\newpage
\section{Concept Checklist}

This checklist maps the problems to the key concepts they are designed to test.

\begin{itemize}
    \item \textbf{Fundamentals of Polar Coordinates}
    \begin{itemize}
        \item Plotting points in the polar plane (including negative $r$): \textbf{1, 2}
        \item Finding multiple representations for a single point: \textbf{1, 2, 3}
    \end{itemize}

    \item \textbf{Coordinate Conversion (Points)}
    \begin{itemize}
        \item Converting Polar coordinates to Cartesian: \textbf{4}
        \item Converting Cartesian coordinates to Polar: \textbf{5, 6, 7}
        \item Correctly determining the quadrant for $\theta$ (The "Quadrant Trap"): \textbf{6, 27}
    \end{itemize}

    \item \textbf{Sketching Regions in Polar Coordinates}
    \begin{itemize}
        \item Sketching regions defined by inequalities on $r$ (disks, annuli): \textbf{8, 10}
        \item Sketching regions defined by inequalities on $\theta$ (wedges): \textbf{8, 9}
        \item Sketching regions defined by combined inequalities: \textbf{8, 9, 11}
    \end{itemize}

    \item \textbf{Equation Conversion (Curves)}
    \begin{itemize}
        \item \textbf{Polar to Cartesian}
        \begin{itemize}
            \item $r = b\sin\theta$ or $r = a\cos\theta$ (Circles not at origin): \textbf{12}
            \item Lines not through origin: \textbf{13}
            \item General polar equations to Cartesian: \textbf{14}
            \item $\theta = k$ (Lines through origin): \textbf{15}
            \item $r = a\sec\theta$ or $r=b\csc\theta$ (Vertical/Horizontal Lines): \textbf{16}
            \item Equations with double-angle identities: \textbf{17}
        \end{itemize}
        \item \textbf{Cartesian to Polar}
        \begin{itemize}
            \item $x^2 + y^2 = k^2$ (Circles at origin): \textbf{18}
            \item $y = mx$ (Lines through origin): \textbf{19}
            \item $x=a$ or $y=b$ (Vertical/Horizontal Lines): \textbf{20}
            \item Circles not centered at origin: \textbf{21}
            \item General Cartesian equations: \textbf{22}
        \end{itemize}
    \end{itemize}

    \item \textbf{Calculus with Polar Coordinates}
    \begin{itemize}
        \item Calculating the area of a polar region: \textbf{23, 24}
        \item Calculating the arc length of a polar curve: \textbf{25, 26}
    \end{itemize}

    \item \textbf{Analytical and Critical Thinking}
    \begin{itemize}
        \item Identifying flaws in incorrect solutions ("Find the Flaw"): \textbf{27, 28}
    \end{itemize}
\end{itemize}

\end{document}