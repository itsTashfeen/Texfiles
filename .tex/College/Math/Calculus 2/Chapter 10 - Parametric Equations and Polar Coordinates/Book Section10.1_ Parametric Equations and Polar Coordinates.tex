\documentclass{article}
\usepackage{amsmath}
\usepackage{amssymb}
\usepackage{geometry}
\usepackage{graphicx}
\usepackage{hyperref}

\geometry{a4paper, margin=1in}

\title{Homework 10: Parametric Equations and Polar Coordinates}
\author{Tashfeen Omran}
\date{October 2025}

\begin{document}

\maketitle

\section{Comprehensive Introduction, Context, and Prerequisites}

\subsection{Core Concepts}
Parametric equations offer a powerful way to describe curves that may not be easily represented as a single function of the form \(y = f(x)\). Instead of defining a direct relationship between \(x\) and \(y\), we introduce a third variable, a \textbf{parameter}, typically denoted by \(t\). Both \(x\) and \(y\) are expressed as independent functions of this parameter:
\[
x = f(t), \quad y = g(t)
\]
The parameter \(t\) often represents time, allowing us to think of the curve as the path of a moving particle. As \(t\) varies over an interval, the point \((x, y)\) traces out the curve in a specific direction, which we call the \textbf{orientation} of the curve. This method is incredibly useful for describing complex paths, motion, and curves that self-intersect or are not functions in the traditional sense.

\subsection{Intuition and Derivation}
The intuition behind parametric equations is rooted in describing motion. Imagine tracking a satellite. Its position in the sky (azimuth and elevation) is constantly changing with time. It would be unnatural to describe its elevation as a function of its azimuth. Instead, it is far more intuitive to describe both its azimuth and its elevation as functions of a single parameter: time.

The process of converting from a set of parametric equations to a single Cartesian equation (\(y\) in terms of \(x\)) is called \textbf{eliminating the parameter}. This is typically an algebraic process. The goal is to solve one of the parametric equations for \(t\) and substitute that expression into the other equation. For example, if \(x = t+1\) and \(y = t^2\), we can write \(t = x-1\). Substituting this into the equation for \(y\) gives \(y = (x-1)^2\), a parabola. This process helps us identify the underlying shape of the curve, but it comes at the cost of losing the information about the orientation and the "speed" at which the curve is traced.

\subsection{Historical Context and Motivation}
While the roots of describing motion parametrically can be traced back to the epicycles used by Greek astronomers like Apollonius of Perga to model planetary orbits, the modern formulation is tied to the development of analytic geometry and calculus in the 17th century. [4] The need to study the physics of motion—such as the trajectory of a cannonball, the path of a planet, or the shape of a taut wire (the catenary curve)—drove mathematicians like Isaac Newton, Leibniz, and the Bernoulli brothers to use parametric descriptions. These problems were difficult or impossible to describe using explicit functions \(y = f(x)\). For example, the path of a point on a rolling wheel (a cycloid) has vertical tangents, meaning \(dy/dx\) is undefined. A parametric description, where \(x\) and \(y\) are functions of the angle of rotation, handles this situation elegantly and was crucial for solving the famous brachistochrone problem (finding the path of fastest descent).

\subsection{Key Formulas}
\begin{itemize}
    \item \textbf{Line Segment:} From \((x_1, y_1)\) to \((x_2, y_2)\) for \(0 \le t \le 1\):
    \[ x = x_1 + (x_2 - x_1)t, \quad y = y_1 + (y_2 - y_1)t \]
    \item \textbf{Circle:} Center \((h, k)\), radius \(r\), counter-clockwise motion:
    \[ x = h + r\cos(t), \quad y = k + r\sin(t), \quad 0 \le t \le 2\pi \]
    \item \textbf{Ellipse:} Center \((h, k)\), semi-axes \(a\) and \(b\):
    \[ x = h + a\cos(t), \quad y = k + b\sin(t), \quad 0 \le t \le 2\pi \]
    \item \textbf{Projectile Motion (no air resistance):} Initial velocity \(v_0\), launch angle \(\alpha\):
    \[ x = (v_0 \cos\alpha)t, \quad y = (v_0 \sin\alpha)t - \frac{1}{2}gt^2 \]
\end{itemize}

\subsection{Prerequisites}
\begin{itemize}
    \item \textbf{Algebra:} Proficiency in solving equations for a variable, substitution, simplifying complex expressions, and working with exponents and logarithms.
    \item \textbf{Trigonometry:} Deep understanding of the unit circle, graphs of all six trigonometric functions, and fundamental identities, especially the Pythagorean identity \(\sin^2\theta + \cos^2\theta = 1\) and double-angle identities.
    \item \textbf{Functions:} A solid grasp of function notation, domain, range, and the shapes of common functions (lines, parabolas, circles, ellipses, exponential, and logarithmic functions).
    \item \textbf{Analytic Geometry:} Familiarity with the Cartesian coordinate system and the standard equations of conic sections.
\end{itemize}

\section{Detailed Homework Solutions}

\subsection{Problems 1-2: Finding Points}
\subsubsection{Problem 1: \(x = t^2 + t, y = 3^{t+1}\)}
For \(t = -2, -1, 0, 1, 2\):
\begin{itemize}
    \item \(t=-2: x=(-2)^2+(-2)=2, y=3^{-2+1}=3^{-1}=1/3 \implies (2, 1/3)\)
    \item \(t=-1: x=(-1)^2+(-1)=0, y=3^{-1+1}=3^0=1 \implies (0, 1)\)
    \item \(t=0: x=(0)^2+(0)=0, y=3^{0+1}=3^1=3 \implies (0, 3)\)
    \item \(t=1: x=(1)^2+(1)=2, y=3^{1+1}=3^2=9 \implies (2, 9)\)
    \item \(t=2: x=(2)^2+(2)=6, y=3^{2+1}=3^3=27 \implies (6, 27)\)
\end{itemize}

\subsubsection{Problem 2: \(x = \ln(t^2 + 1), y = t/(t+4)\)}
For \(t = -2, -1, 0, 1, 2\):
\begin{itemize}
    \item \(t=-2: x=\ln((-2)^2+1)=\ln(5), y=-2/(-2+4)=-1 \implies (\ln 5, -1)\)
    \item \(t=-1: x=\ln((-1)^2+1)=\ln(2), y=-1/(-1+4)=-1/3 \implies (\ln 2, -1/3)\)
    \item \(t=0: x=\ln(0^2+1)=\ln(1)=0, y=0/(0+4)=0 \implies (0, 0)\)
    \item \(t=1: x=\ln(1^2+1)=\ln(2), y=1/(1+4)=1/5 \implies (\ln 2, 1/5)\)
    \item \(t=2: x=\ln(2^2+1)=\ln(5), y=2/(2+4)=1/3 \implies (\ln 5, 1/3)\)
\end{itemize}

\subsection{Problems 3-6: Sketching Curves}
(Note: Sketches are described based on plotted points and direction.)

\subsubsection{Problem 3: \(x = 1 - t^2, y = 2t - t^2, -1 \le t \le 2\)}
Points: \(t=-1 \to (0, -3)\), \(t=0 \to (1, 0)\), \(t=1 \to (0, 1)\), \(t=2 \to (-3, 0)\).
The curve starts at \((0, -3)\), moves up and right to a vertex-like point at \((1,1)\) (when \(t=1/2\)), then moves up and left through \((0,1)\) and ends at \((-3,0)\).

\subsubsection{Problem 4: \(x = t^3 + t, y = t^2 + 2, -2 \le t \le 2\)}
Points: \(t=-2 \to (-10, 6)\), \(t=-1 \to (-2, 3)\), \(t=0 \to (0, 2)\), \(t=1 \to (2, 3)\), \(t=2 \to (10, 6)\).
The curve starts at \((-10, 6)\), moves to a vertex at \((0, 2)\), and then mirrors its path to end at \((10, 6)\), forming a cusp at the vertex.

\subsubsection{Problem 5: \(x = 2^t - t, y = 2^{-t} + t, -3 \le t \le 3\)}
Points: \(t=-3 \to (3.125, 5)\), \(t=0 \to (1, 1)\), \(t=3 \to (5, 3.125)\).
The curve starts at \((3.125, 5)\) and moves down and to the right, passing through \((1,1)\), and ending at \((5, 3.125)\).

\subsubsection{Problem 6: \(x = \cos^2 t, y = 1 + \cos t, 0 \le t \le \pi\)}
Points: \(t=0 \to (1, 2)\), \(t=\pi/2 \to (0, 1)\), \(t=\pi \to (1, 0)\).
Since \(y-1 = \cos t\), we can substitute into \(x\), giving \(x = (y-1)^2\). This is a parabola opening to the right. The curve starts at the top point \((1, 2)\), moves left to the vertex \((0, 1)\), and then moves right to the bottom point \((1, 0)\).

\subsection{Problems 7-12: Sketch and Eliminate Parameter}

\subsubsection{Problem 7: \(x = 2t - 1, y = \frac{1}{2}t + 1\)}
a. The curve is a straight line. Direction is up and to the right.
b. From \(x = 2t - 1\), we get \(t = (x+1)/2\). Substitute into \(y\):
\[y = \frac{1}{2}\left(\frac{x+1}{2}\right) + 1 = \frac{x+1}{4} + 1 = \frac{x+5}{4}\]
The Cartesian equation is \(y = \frac{1}{4}x + \frac{5}{4}\).

\subsubsection{Problem 8: \(x = 3t + 2, y = 2t + 3\)}
a. The curve is a straight line. Direction is up and to the right.
b. From \(x = 3t + 2\), we get \(t = (x-2)/3\). Substitute into \(y\):
\[y = 2\left(\frac{x-2}{3}\right) + 3 = \frac{2x-4}{3} + \frac{9}{3} = \frac{2x+5}{3}\]
The Cartesian equation is \(y = \frac{2}{3}x + \frac{5}{3}\).

\subsubsection{Problem 9: \(x = t^2 - 3, y = t + 2, -3 \le t \le 3\)}
a. The curve is a segment of a parabola opening to the right. It starts at \(t=-3 \to (6, -1)\) and ends at \(t=3 \to (6, 5)\).
b. From \(y = t + 2\), we get \(t = y - 2\). Substitute into \(x\):
\[x = (y-2)^2 - 3\]
The range of \(y\) is from \(-3+2=-1\) to \(3+2=5\), so the equation is valid for \(-1 \le y \le 5\).

\subsubsection{Problem 10: \(x = \sin t, y = 1 - \cos t, 0 \le t \le 2\pi\)}
a. The curve is a circle traced once counter-clockwise, starting and ending at \((0,0)\).
b. Rearrange: \(x = \sin t\) and \(y-1 = -\cos t\). Square both equations: \(x^2 = \sin^2 t\) and \((y-1)^2 = \cos^2 t\). Add them:
\[x^2 + (y-1)^2 = \sin^2 t + \cos^2 t \implies x^2 + (y-1)^2 = 1\]
This is a circle with center \((0, 1)\) and radius 1.

\subsubsection{Problem 11: \(x = \sqrt{t}, y = 1 - t\)}
a. The curve starts at \(t=0 \to (0, 1)\) and moves down and to the right. Since \(t \ge 0\), \(x \ge 0\).
b. From \(x = \sqrt{t}\), we get \(t = x^2\). Substitute into \(y\):
\[y = 1 - x^2\]
The Cartesian equation is \(y = 1 - x^2\) for \(x \ge 0\).

\subsubsection{Problem 12: \(x = t^2, y = t^3\)}
a. The curve passes through \((0,0)\) with a cusp. For \(t<0\), it's in the lower-right quadrant; for \(t>0\), it's in the upper-right.
b. From \(x=t^2\), \(t = \pm\sqrt{x}\). From \(y=t^3\), \(t = \sqrt[3]{y}\). So, \(\sqrt[3]{y} = \pm\sqrt{x}\). Cubing both sides gives \(y = (\pm x^{1/2})^3 = \pm x^{3/2}\). Squaring this result gives:
\[y^2 = x^3\]

\subsection{Problems 13-22: Eliminate Parameter and Sketch}

\subsubsection{Problem 13: \(x=3\cos t, y=3\sin t, 0 \le t \le \pi\)}
a. \(\frac{x}{3} = \cos t\), \(\frac{y}{3} = \sin t\). Using \(\cos^2 t + \sin^2 t = 1\):
\[\left(\frac{x}{3}\right)^2 + \left(\frac{y}{3}\right)^2 = 1 \implies x^2+y^2=9\]
b. This is the top half of a circle of radius 3, traced counter-clockwise from \((3,0)\) to \((-3,0)\).

\subsubsection{Problem 14: \(x=\sin(4\theta), y=\cos(4\theta), 0 \le \theta \le \pi/2\)}
a. \(x^2+y^2 = \sin^2(4\theta) + \cos^2(4\theta) = 1\). As \(\theta\) goes from 0 to \(\pi/2\), the argument \(4\theta\) goes from 0 to \(2\pi\).
b. This is the full unit circle, \(x^2+y^2=1\), traced once clockwise starting from \((0,1)\).

\subsubsection{Problem 15: \(x=\cos\theta, y=\sec^2\theta, 0 \le \theta < \pi/2\)}
a. Since \(y = \sec^2\theta = 1/\cos^2\theta\), we can substitute \(x=\cos\theta\):
\[y = 1/x^2\]
b. For the given interval, \(x=\cos\theta\) is in \((0, 1]\). The curve is a portion of \(y=1/x^2\) starting at \((1,1)\) and going up as \(x \to 0^+\).

\subsubsection{Problem 16: \(x=\csc t, y=\cot t, 0 < t < \pi\)}
a. Use the identity \(1+\cot^2 t = \csc^2 t\). Substituting gives \(1+y^2=x^2\).
\[x^2 - y^2 = 1\]
b. For \(0 < t < \pi\), \(x = \csc t > 0\). This is the right branch of a hyperbola.

\subsubsection{Problem 17: \(x=e^{-t}, y=e^t\)}
a. \(x = 1/e^t\). Since \(y=e^t\), we have \(x = 1/y\).
\[y = 1/x\]
b. Since \(e^t > 0\) for all \(t\), both \(x\) and \(y\) are positive. This is the branch of the hyperbola \(y=1/x\) in the first quadrant.

\subsubsection{Problem 18: \(x=t+2, y=1/t, t>0\)}
a. From \(x=t+2\), \(t=x-2\). Substitute into \(y\):
\[y = \frac{1}{x-2}\]
b. Since \(t>0\), \(x=t+2 > 2\). This is a portion of the rational function for \(x>2\).

\subsubsection{Problem 19: \(x=\ln t, y=\sqrt{t}, t \ge 1\)}
a. From \(x=\ln t\), \(t = e^x\). Substitute into \(y\):
\[y = \sqrt{e^x} = e^{x/2}\]
b. Since \(t \ge 1\), \(x = \ln t \ge \ln 1 = 0\). The curve is \(y=e^{x/2}\) for \(x \ge 0\).

\subsubsection{Problem 20: \(x=|t|, y=|1-|t||\)}
a. Substitute \(x=|t|\) into the equation for y:
\[y = |1-x|\]
b. Since \(x=|t|\), we have \(x \ge 0\). The curve is the graph of \(y=|1-x|\) for non-negative \(x\).

\subsubsection{Problem 21: \(x=\sin^2 t, y=\cos^2 t\)}
a. Using the identity \(\sin^2 t + \cos^2 t = 1\), we substitute \(x\) and \(y\):
\[x+y=1\]
b. Since both \(x\) and \(y\) are squares of real numbers, \(0 \le x \le 1\) and \(0 \le y \le 1\). The curve is the line segment from \((0,1)\) to \((1,0)\).

\subsubsection{Problem 22: \(x=\sinh t, y=\cosh t\)}
a. Use the identity \(\cosh^2 t - \sinh^2 t = 1\). Substituting gives:
\[y^2 - x^2 = 1\]
b. Since \(y=\cosh t \ge 1\) for all \(t\), this is the upper branch of a hyperbola.

\subsection{Problems 23-24: Circular Motion}

\subsubsection{Problem 23: \(x=5\cos t, y=-5\sin t\)}
The equations are of the form \(x=r\cos\omega t, y=-r\sin\omega t\). This represents clockwise motion. The time to complete one revolution (the period) is \(2\pi/\omega\). Here \(\omega=1\).
\textbf{Time for one revolution:} \(2\pi\) seconds.
\textbf{Direction:} At \(t=0\), point is \((5,0)\). At \(t=\pi/2\), point is \((0,-5)\). The motion is \textbf{clockwise}.

\subsubsection{Problem 24: \(x=3\sin(\pi t/4), y=3\cos(\pi t/4)\)}
This can be rewritten as \(x=3\cos(\pi/2 - \pi t/4), y=3\sin(\pi/2 - \pi t/4)\) to analyze direction, but it's easier to plot points. At \(t=0\), point is \((0,3)\). At \(t=2\), point is \((3,0)\). The motion is \textbf{clockwise}. The angular frequency is \(\omega = \pi/4\). The period is \(T = 2\pi/\omega = 2\pi/(\pi/4) = 8\).
\textbf{Time for one revolution:} 8 seconds.
\textbf{Direction:} \textbf{Clockwise}.

\subsection{Problems 25-28: Describing Motion}

\subsubsection{Problem 25: \(x=5+2\cos(\pi t), y=3+2\sin(\pi t), 1 \le t \le 2\)}
The equations describe a circle centered at \((5,3)\) with radius 2. The argument \(\pi t\) goes from \(\pi\) to \(2\pi\) as \(t\) goes from 1 to 2. This traces the bottom half of the circle, starting at \(t=1 \to (5-2, 3)=(3,3)\) and ending at \(t=2 \to (5+2, 3)=(7,3)\) in a counter-clockwise direction.

\subsubsection{Problem 26: \(x=2+\sin t, y=1+3\cos t, \pi/2 \le t \le 2\pi\)}
This is an ellipse centered at \((2,1)\) since \(\sin t = x-2\) and \(\cos t = (y-1)/3\), so \((x-2)^2 + ((y-1)/3)^2 = 1\). The parameter \(t\) goes from \(\pi/2\) to \(2\pi\).
At \(t=\pi/2\), point is \((3,1)\). At \(t=\pi\), point is \((2,-2)\). At \(t=3\pi/2\), point is \((1,1)\). At \(t=2\pi\), point is \((2,4)\). The particle traces three-quarters of an ellipse clockwise.

\subsubsection{Problem 27: \(x=5\sin t, y=2\cos t, -\pi \le t \le 5\pi\)}
This is an ellipse centered at the origin, \((x/5)^2 + (y/2)^2 = 1\). The interval for \(t\) is \(6\pi\) long, which is \(6\pi / (2\pi) = 3\) full revolutions. The motion is clockwise. At \(t=0\), point is \((0,2)\); at \(t=\pi/2\), point is \((5,0)\). The particle traverses the ellipse clockwise 3 times.

\subsubsection{Problem 28: \(x=\sin t, y=\cos^2 t, -2\pi \le t \le 2\pi\)}
Substitute \(x=\sin t\) into \(y=\cos^2 t = 1-\sin^2 t\). The path is \(y=1-x^2\), a parabola. Since \(x=\sin t\), \(-1 \le x \le 1\). The particle oscillates back and forth along the parabolic arc between \((-1,0)\) and \((1,0)\). The interval is \(4\pi\) long, so it completes two full oscillations.

\subsection{Miscellaneous Problems}

\subsubsection{Problem 29: Curve with range constraints}
Given \(x=f(t)\) with range \([1,4]\) and \(y=g(t)\) with range \([2,3]\). This means for any value of \(t\), \(1 \le x \le 4\) and \(2 \le y \le 3\).
\textbf{Conclusion:} The entire curve is contained within the rectangle defined by these bounds: the rectangle with vertices at \((1,2), (4,2), (4,3), (1,3)\).

\subsubsection{Problem 37: Line Segment Parameterization}
a. Let \(x(t) = x_1 + (x_2-x_1)t\) and \(y(t) = y_1 + (y_2-y_1)t\).
At \(t=0\), \(x(0)=x_1\) and \(y(0)=y_1\), so we are at point \(P_1\).
At \(t=1\), \(x(1)=x_1+x_2-x_1=x_2\) and \(y(1)=y_1+y_2-y_1=y_2\), so we are at point \(P_2\).
For \(0<t<1\), the point is on the line connecting \(P_1\) and \(P_2\). This is the standard linear interpolation formula.
b. For the segment from \((-2,7)\) to \((3,-1)\):
\(x_1=-2, y_1=7, x_2=3, y_2=-1\).
\[x = -2 + (3 - (-2))t = -2 + 5t\]
\[y = 7 + (-1 - 7)t = 7 - 8t\]
For \(0 \le t \le 1\).

\subsubsection{Problem 38: Triangle with Vertices A(1,1), B(4,2), C(1,5)}
Using the formula from 37(a):
\begin{itemize}
    \item \textbf{Side AB:} From (1,1) to (4,2). \(x = 1+3t, y=1+t\), for \(0 \le t \le 1\).
    \item \textbf{Side BC:} From (4,2) to (1,5). \(x = 4-3t, y=2+3t\), for \(0 \le t \le 1\).
    \item \textbf{Side CA:} From (1,5) to (1,1). \(x = 1+0t=1, y=5-4t\), for \(0 \le t \le 1\).
\end{itemize}

\subsubsection{Problem 39: Clockwise circle, origin center, radius 5, period 4\(\pi\)}
Standard clockwise is \(x=r\cos t, y=-r\sin t\). Here \(r=5\). We need the period \(T=2\pi/\omega\) to be \(4\pi\).
\(4\pi = 2\pi/\omega \implies \omega = 1/2\).
Equations: \(x = 5\cos(t/2), y = -5\sin(t/2)\).

\subsubsection{Problem 40: Counter-clockwise circle, center (1,3), radius 1, period 3}
Standard counter-clockwise is \(x=h+r\cos(\omega t), y=k+r\sin(\omega t)\). Here \((h,k)=(1,3)\) and \(r=1\).
Period \(T = 2\pi/\omega = 3 \implies \omega = 2\pi/3\).
Equations: \(x = 1+\cos(2\pi t/3), y=3+\sin(2\pi t/3)\).

\subsubsection{Problem 41: Path on circle \(x^2 + (y-1)^2 = 4\)}
This is a circle centered at \((0,1)\) with radius \(r=2\).
a. Once around clockwise, starting at \((2,1)\). The starting point corresponds to \(t=0\) in the standard parameterization.
Clockwise equations: \(x = h + r\cos t, y = k - r\sin t\).
\(x=0+2\cos t, y=1-2\sin t\). To start at \((2,1)\), we need \(x=2, y=1\). \(2\cos t = 2 \implies \cos t = 1\) and \(1-2\sin t = 1 \implies \sin t = 0\). This is \(t=0\).
So, \(x=2\cos t, y=1-2\sin t\) for \(0 \le t \le 2\pi\).

b. Three times counter-clockwise, starting at \((2,1)\).
Counter-clockwise: \(x=2\cos t, y=1+2\sin t\). The interval for three times is \(3 \times 2\pi = 6\pi\).
So, \(x=2\cos t, y=1+2\sin t\) for \(0 \le t \le 6\pi\).

c. Halfway counter-clockwise, starting at \((0,3)\).
The starting point \((0,3)\) corresponds to \(x=0, y=3\). In \(x=2\cos t, y=1+2\sin t\), this means \(2\cos t = 0\) and \(1+2\sin t = 3 \implies \sin t = 1\). This occurs at \(t=\pi/2\). Halfway around is an interval of \(\pi\).
So, \(x=2\cos t, y=1+2\sin t\) for \(\pi/2 \le t \le 3\pi/2\).

\subsubsection{Problem 42: Ellipse}
a. The equation is \(x^2/a^2 + y^2/b^2 = 1\). This suggests \(x/a = \cos t\) and \(y/b = \sin t\).
Parametric equations: \(x = a\cos t, y = b\sin t\), for \(0 \le t \le 2\pi\).

c. As \(b\) varies, the vertical semi-axis of the ellipse changes. If \(b\) increases, the ellipse becomes taller (more elongated vertically). If \(b\) decreases, it becomes shorter (more compressed vertically). If \(b=a\), it is a circle. If \(b=0\), it is a line segment on the x-axis.

\subsubsection{Problem 45: Four curves, one equation}
i. \(x=t^2, y=t\). Eliminate \(t\): \(x=y^2\). Since \(y=t\), \(y\) can be any real number. This is the full parabola.
ii. \(x=t, y=\sqrt{t}\). Eliminate \(t\): \(y=\sqrt{x}\), so \(x=y^2\). Since \(y=\sqrt{t}\), \(y \ge 0\). This is the top half of the parabola.
iii. \(x=\cos^2 t, y=\cos t\). Eliminate \(t\): \(x=y^2\). Since \(y=\cos t\), \(-1 \le y \le 1\). This is a segment of the parabola.
iv. \(x=3^{2t}, y=3^t\). \(x=(3^t)^2\). Substitute \(y=3^t\): \(x=y^2\). Since \(y=3^t\), \(y > 0\). This is the top half of the parabola for \(x>0\).
b. All four curves lie on the parabola \(x=y^2\). They differ in the portion of the parabola they trace and the direction/speed. (i) is the whole parabola, (ii) is the top half, (iii) oscillates on a segment of it, and (iv) traces the top-right part.

\subsubsection{Problem 55: Intersection and Collision}
Red particle: \(x_R=t+5, y_R=t^2+4t+6\).
Blue particle: \(x_B=2t+1, y_B=2t+6\).

a. Verify intersection points.
For \((1,6)\): Does it lie on the red path? \(1=t+5 \implies t=-4\). \(y_R = (-4)^2+4(-4)+6 = 16-16+6=6\). Yes.
Does it lie on the blue path? \(1=2t+1 \implies t=0\). \(y_B = 2(0)+6=6\). Yes.
The paths intersect at \((1,6)\). It is not a collision because the particles are there at different times (\(t=-4\) for red, \(t=0\) for blue).

For \((6,11)\): Does it lie on the red path? \(6=t+5 \implies t=1\). \(y_R = 1^2+4(1)+6 = 11\). Yes.
Does it lie on the blue path? \(6=2t+1 \implies t=2.5\). \(y_B = 2(2.5)+6=11\). Yes.
The paths intersect at \((6,11)\). Not a collision.

b. Green particle: \(x_G=2t+4, y_G=2t+9\).
Path of blue particle: from \(x=2t+1\), \(t=(x-1)/2\). So \(y=2((x-1)/2)+6 = x-1+6=x+5\). The path is the line \(y=x+5\).
Path of green particle: from \(x=2t+4\), \(t=(x-4)/2\). So \(y=2((x-4)/2)+9 = x-4+9=x+5\). They move on the same path.

Collision: \(x_R=x_G\) and \(y_R=y_G\) for the same \(t\).
\(t+5 = 2t+4 \implies t=1\).
Check \(y\)-values at \(t=1\):
\(y_R(1) = 1^2+4(1)+6=11\).
\(y_G(1) = 2(1)+9=11\).
Yes, they collide at time \(t=1\) at the point \((1+5, 11) = (6,11)\).

\subsubsection{Problem 58: Projectile Motion}
Given \(v_0=500\) m/s, \(\alpha=30^\circ\), \(g=9.8\) m/s\(^2\).
\(x = (500\cos 30^\circ)t = (500 \cdot \sqrt{3}/2)t = 250\sqrt{3}t\).
\(y = (500\sin 30^\circ)t - \frac{1}{2}(9.8)t^2 = (500 \cdot 1/2)t - 4.9t^2 = 250t - 4.9t^2\).

a. When does it hit the ground? When \(y=0\).
\(250t - 4.9t^2 = 0 \implies t(250 - 4.9t) = 0\).
\(t=0\) (start) or \(t = 250/4.9 \approx 51.02\) s.
\textbf{Time to hit ground:} \(\approx 51.02\) seconds.

How far does it go? Find \(x\) at that time.
\(x = 250\sqrt{3} \cdot (250/4.9) \approx 22092\) m.
\textbf{Range:} \(\approx 22.1\) km.

Maximum height? Occurs when \(y'(t)=0\).
\(y'(t) = 250 - 9.8t = 0 \implies t = 250/9.8 \approx 25.51\) s.
Max height is \(y(25.51) = 250(25.51) - 4.9(25.51)^2 \approx 3188.8\) m.
\textbf{Maximum height:} \(\approx 3189\) meters.

c. Eliminate the parameter. From \(x\), \(t = x/(250\sqrt{3})\). Substitute into \(y\):
\[y = 250\left(\frac{x}{250\sqrt{3}}\right) - 4.9\left(\frac{x}{250\sqrt{3}}\right)^2 = \frac{x}{\sqrt{3}} - \frac{4.9}{187500}x^2\]
This is a quadratic equation in \(x\), so the path is a parabola.

\section{In-Depth Analysis of Problems and Techniques}
\subsection{Problem Types and General Approach}
\begin{itemize}
    \item \textbf{Direct Evaluation (1-2):} Substitute parameter values to find coordinates. The approach is simple calculation.
    \item \textbf{Parameter Elimination (7-22):} The core technique is algebraic or trigonometric manipulation. Strategy: (1) Solve one equation for \(t\). (2) Substitute into the other. (3) For trig functions, use identities like \(\sin^2\theta+\cos^2\theta=1\) or \(\sec^2\theta-\tan^2\theta=1\).
    \item \textbf{Motion Analysis (23-28):} Identify the shape of the path, then use the parameter interval to determine the starting point, ending point, direction (clockwise/counter-clockwise), and number of revolutions.
    \item \textbf{Geometric Derivation (37-42, 49-54):} Translate a geometric description or a set of constraints (e.g., center, radius, direction) into algebraic equations for \(x\) and \(y\) in terms of a parameter. This often involves trigonometry and basic geometry.
    \item \textbf{Intersection vs. Collision (55-57):} A critical distinction. For \textbf{intersection}, find points \((x,y)\) that satisfy both sets of equations, but possibly for different parameter values. For \textbf{collision}, you must find a single parameter value \(t\) that produces the same \((x,y)\) in both sets of equations.
    \item \textbf{Application Problems (58):} Apply standard parametric formulas (like those for projectile motion) to a real-world scenario and interpret the results.
    \item \textbf{Family of Curves (59-64):} Exploratory problems where the goal is to understand how a constant (e.g., \(c, a, n\)) in the equations affects the overall shape of the curve.
\end{itemize}

\subsection{Key Algebraic and Calculus Manipulations}
\begin{itemize}
    \item \textbf{Substitution after Solving for t (7, 8, 9, 11, 18, 19, 58c):} The most common algebraic trick. It's crucial for converting from parametric to Cartesian form for non-trigonometric functions.
    \item \textbf{Pythagorean Identity (10, 13, 14, 16, 21, 22, 26, 27):} This is the go-to technique whenever \(x\) and \(y\) are defined by sine/cosine, secant/tangent, or sinh/cosh pairs. It's the key to getting equations for circles, ellipses, and hyperbolas.
    \item \textbf{Direct Relationship (6, 12, 15, 20, 28, 45):} In some cases, you can see a direct algebraic link between \(x\) and \(y\) without solving for \(t\). For example, in Problem 6, seeing that \(x = (\cos t)^2\) and \(y-1=\cos t\) immediately gives \(x=(y-1)^2\).
    \item \textbf{Setting Systems of Equations (55, 56, 57):} For collision problems, this involves setting \(x_1(t) = x_2(t)\) and \(y_1(t) = y_2(t)\) and solving for a common \(t\). For intersection, it's \(x_1(t_1) = x_2(t_2)\) and \(y_1(t_1) = y_2(t_2)\).
\end{itemize}

\section{"Cheatsheet" and Tips for Success}
\subsection{Formula Summary}
\begin{itemize}
    \item \textbf{Circle (CW):} \(x=h+r\cos t, y=k-r\sin t\)
    \item \textbf{Circle (CCW):} \(x=h+r\cos t, y=k+r\sin t\)
    \item \textbf{Period/Frequency:} \(T = 2\pi/\omega\), where \(\omega\) is the coefficient of \(t\).
    \item \textbf{Line Segment:} \(x=(1-t)x_1+tx_2, y=(1-t)y_1+ty_2\) for \(0 \le t \le 1\).
    \item \textbf{Parabola \(y=ax^2+...\):} Often involves one linear and one quadratic equation in \(t\).
    \item \textbf{Parabola \(x=ay^2+...\):} Often involves \(x=t^2\) and \(y=t\).
\end{itemize}
\subsection{Tricks and Shortcuts}
\begin{itemize}
    \item \textbf{Recognizing Shapes:} \((\cos t, \sin t) \to\) Circle/Ellipse. \((t, t^2) \to\) Parabola. \((t, 1/t) \to\) Hyperbola.
    \item \textbf{Direction Check:} Don't solve the whole problem. Just plug in two close values of \(t\), like \(t=0\) and \(t=0.1\), to see which way the point \((x,y)\) moves.
    \item \textbf{Symmetry:} If \(x(t)\) is an even function and \(y(t)\) is an even function, the curve is traced twice. If \(x(t)\) is even and \(y(t)\) is odd (or vice versa), look for symmetry across an axis.
\end{itemize}
\subsection{Common Pitfalls}
\begin{itemize}
    \item \textbf{Ignoring the Interval:} Forgetting to restrict the domain/range of the final Cartesian equation based on the interval for \(t\). (e.g., Problem 11, where \(x \ge 0\)).
    \item \textbf{Intersection vs. Collision:} Confusing the two concepts. Remember, collision requires the same time \(t\).
    \item \textbf{Sign Errors in Direction:} Mixing up clockwise and counter-clockwise. A negative sign on the \(\sin\) term (with a positive \(\cos\)) usually implies clockwise motion.
    \item \textbf{Algebraic Mistakes:} Simple errors in squaring binomials or solving for \(t\) are the most frequent source of incorrect answers.
\end{itemize}

\section{Conceptual Synthesis and The "Big Picture"}
\subsection{Thematic Connections}
The core theme is \textbf{parameterization}: representing a complex object (a curve) by mapping a simpler object (a line segment, the real number line) to it. This idea is fundamental throughout mathematics.
\begin{itemize}
    \item In \textbf{Linear Algebra}, we parameterize lines and planes using vectors. The equation of a line \(\vec{p}(t) = \vec{p}_0 + t\vec{v}\) is exactly what we used in Problem 37.
    \item In \textbf{Differential Geometry}, we study surfaces by parameterizing them with two variables, \((u,v)\).
    \item In \textbf{Complex Analysis}, we evaluate contour integrals by parameterizing the path of integration in the complex plane.
\end{itemize}
This chapter is your first formal introduction to this powerful and unifying concept.

\subsection{Forward and Backward Links}
\textbf{Backward Link:} This topic is a direct extension of function graphing from Precalculus. However, by freeing ourselves from the \(y=f(x)\) constraint, we can now describe shapes like circles (Problem 10) and self-intersecting curves which fail the "vertical line test".

\textbf{Forward Link:} Parametric equations are the absolute bedrock for \textbf{vector calculus} (Calculus III). The position of a particle will be described by a vector function \(\vec{r}(t) = \langle x(t), y(t), z(t) \rangle\). The skills you learn here—describing paths, finding orientation—will be used to define velocity (\(\vec{v}(t) = \vec{r}'(t)\)) and acceleration (\(\vec{a}(t) = \vec{r}''(t)\)), to calculate the length of a curve in space, and to compute line integrals (the work done by a force along a path).

\section{Real-World Application and Modeling}
\subsection{Concrete Scenarios in Finance and Economics}
\begin{itemize}
    \item \textbf{Derivative Pricing and Stochastic Calculus:} The price of a financial asset (like a stock or option) over time is often modeled as a stochastic process. The famous Black-Scholes model, for instance, describes the asset price \(S_t\) using a stochastic differential equation. A simulation of this path, called a Monte Carlo simulation, generates a parametric curve where time is the parameter and the asset price is the output, used to price complex derivatives. The path is \( (t, S(t)) \).
    \item \textbf{Business Cycles and Phase Portraits:} In macroeconomics, the relationship between inflation (\(\pi\)) and unemployment (\(u\)) can be modeled as a system of differential equations. Plotting the solution in the \((\pi, u)\)-plane with time \(t\) as the parameter generates a "phase portrait". This parametric curve can show how an economy might spiral into or out of a recession, or settle into a long-run equilibrium.
    \item \textbf{Yield Curve Modeling:} The financial yield curve plots the interest rate (yield) of bonds against their time to maturity. The evolution of this entire curve over time can be modeled parametrically. For example, the Nelson-Siegel model describes the yield \(y\) as a function of maturity \(\tau\) using parameters that change over time, \(t\). So the state of the economy at time \(t\) is represented by a curve, and the evolution of the economy is the evolution of this parameterized curve.
\end{itemize}

\subsection{Model Problem Setup}
\textbf{Scenario:} Modeling a boom-bust economic cycle. An economist posits that, after a policy shock, the quarterly unemployment rate \(U\) and the quarterly inflation rate \(I\) (as a percentage) follow a path described by these parametric equations, where \(t\) is time in quarters:
\[
U(t) = 5 + 2\cos\left(\frac{\pi t}{10}\right)
\]
\[
I(t) = 3 - 2\sin\left(\frac{\pi t}{10}\right)
\]
\textbf{Model Setup:}
\begin{itemize}
    \item \textbf{Variables:} \(t\) (time in quarters), \(U\) (unemployment \%), \(I\) (inflation \%).
    \item \textbf{Equations:} The given parametric equations.
    \item \textbf{Interpretation:} The center of the cycle is at (5\% unemployment, 3\% inflation). The cycle has an amplitude of 2\% for both variables. The period is \(T = 2\pi / (\pi/10) = 20\) quarters, or 5 years.
    \item \textbf{Question to Solve:} What is the Cartesian equation relating inflation and unemployment during this cycle?
    \item \textbf{Setup for Solution:}
    1. Isolate the trigonometric terms:
       \[ \frac{U-5}{2} = \cos\left(\frac{\pi t}{10}\right) \quad \text{and} \quad \frac{I-3}{-2} = \sin\left(\frac{\pi t}{10}\right) \]
    2. Square and add the equations using the Pythagorean identity:
       \[ \left(\frac{U-5}{2}\right)^2 + \left(\frac{I-3}{-2}\right)^2 = 1 \]
    3. The resulting equation to solve/analyze is:
       \[ \frac{(U-5)^2}{4} + \frac{(I-3)^2}{4} = 1 \implies (U-5)^2 + (I-3)^2 = 4 \]
    This shows the relationship is a circle in the Unemployment-Inflation plane, indicating a cyclical trade-off between the two.
\end{itemize}

\section{Common Variations and Untested Concepts}
My homework set was very thorough but focused exclusively on defining and describing parametric curves. The next logical step, which was not included, is performing \textbf{calculus on parametric curves}.

\subsubsection{Concept 1: Derivatives of Parametric Equations}
The slope of the tangent line to a parametric curve is still \(dy/dx\), but we must compute it using the chain rule, since \(y\) is a function of \(t\) and \(t\) is implicitly a function of \(x\).
\[ \frac{dy}{dx} = \frac{dy/dt}{dx/dt} \]
The second derivative is found by applying this rule again:
\[ \frac{d^2y}{dx^2} = \frac{d}{dx}\left(\frac{dy}{dx}\right) = \frac{\frac{d}{dt}\left(\frac{dy}{dx}\right)}{dx/dt} \]

\textbf{Worked-Out Example:} Find the slope of the tangent line to the curve \(x = t^2, y = t^3 - 3t\) at \(t = 2\).
\begin{enumerate}
    \item Find the derivatives with respect to \(t\):
    \( \frac{dx}{dt} = 2t \)
    \( \frac{dy}{dt} = 3t^2 - 3 \)
    \item Compute \(dy/dx\):
    \[ \frac{dy}{dx} = \frac{3t^2 - 3}{2t} \]
    \item Evaluate at \(t=2\):
    \[ \frac{dy}{dx}\bigg|_{t=2} = \frac{3(2^2) - 3}{2(2)} = \frac{12 - 3}{4} = \frac{9}{4} \]
    At \(t=2\), the point is \((4, 2)\). The slope of the tangent line at this point is \(9/4\).
\end{enumerate}

\section{Advanced Diagnostic Testing: "Find the Flaw"}
For each problem, find the single flaw in the provided solution, explain the error, and provide the correct result.

\subsubsection{Problem 1}
Find the Cartesian equation for the curve \( x = 3\cos t + 1, y = 3\sin t - 2 \).
\textbf{Flawed Solution:}
1. Isolate trig terms: \( x - 1 = 3\cos t \) and \( y + 2 = 3\sin t \).
2. Square both equations: \( (x - 1)^2 = 9\cos^2 t \) and \( (y + 2)^2 = 9\sin^2 t \).
3. Add the equations: \( (x - 1)^2 + (y + 2)^2 = 9\cos^2 t + 9\sin^2 t \).
4. Factor out 9: \( (x - 1)^2 + (y + 2)^2 = 9(\cos^2 t + \sin^2 t) \).
5. Simplify: \( (x - 1)^2 + (y + 2)^2 = 1 \). This is a unit circle.
\newline
\textbf{Flaw Explanation:}
The error is in step 5. When factoring out 9 and using the identity \(\cos^2 t + \sin^2 t = 1\), the right side should be \(9(1)=9\), not 1.
\textbf{Correct Solution:} The correct equation is \( (x - 1)^2 + (y + 2)^2 = 9 \), which is a circle of radius 3 centered at \((1, -2)\).

\subsubsection{Problem 2}
A particle's motion is given by \( x = t^2 - 1, y = t^3 - t \). Find all \(t\) where the tangent line is horizontal.
\textbf{Flawed Solution:}
1. A horizontal tangent occurs when the slope \(dy/dx = 0\).
2. First, find \(dx/dt = 2t\) and \(dy/dt = 3t^2 - 1\).
3. The slope is \( \frac{dy}{dx} = \frac{3t^2 - 1}{2t} \).
4. Set the slope to zero: \( \frac{3t^2 - 1}{2t} = 0 \).
5. This implies the denominator is zero, so \( 2t = 0 \), which gives \( t = 0 \).
\newline
\textbf{Flaw Explanation:}
The error is in step 5. For a fraction to be zero, the \textit{numerator} must be zero, not the denominator. A zero in the denominator leads to an undefined slope (a vertical tangent).
\textbf{Correct Solution:} Set the numerator to zero: \( 3t^2 - 1 = 0 \implies t^2 = 1/3 \implies t = \pm\frac{1}{\sqrt{3}} \).

\subsubsection{Problem 3}
Find the collision points of two particles with paths \( \vec{r_1}(t) = \langle 2t-3, t^2 \rangle \) and \( \vec{r_2}(t) = \langle t, 3t-2 \rangle \).
\textbf{Flawed Solution:}
1. To find a collision, set the x and y components equal.
2. \( x_1 = x_2 \implies 2t - 3 = t \implies t = 3 \).
3. \( y_1 = y_2 \implies t^2 = 3t - 2 \implies t^2 - 3t + 2 = 0 \).
4. Factoring gives \( (t-1)(t-2) = 0 \), so \( t=1 \) or \( t=2 \).
5. Since there is no single value of \(t\) that satisfies both equations, there is no collision.
\newline
\textbf{Flaw Explanation:}
The logic is sound, but the conclusion is based on a false premise. The reasoning that no single value of \(t\) satisfies both is correct. The flaw is subtle: the problem *is* correct in its conclusion that there is no collision. A better "flaw" would be to find intersection instead. Let's re-frame this to be a true flawed problem.

\textit{Re-staged Problem 3:} Find the intersection point of the paths \( \vec{r_1}(t) = \langle t+1, t^2 \rangle \) and \( \vec{r_2}(t) = \langle 3-t, t \rangle \).
\textbf{Flawed Solution:}
1. Set components equal: \( t+1 = 3-t \) and \( t^2 = t \).
2. From the x-components: \( 2t = 2 \implies t=1 \).
3. From the y-components: \( t^2-t = 0 \implies t(t-1)=0 \implies t=0, 1 \).
4. Since \(t=1\) works for both, there is a collision at \(t=1\).
5. The point is \((1+1, 1^2) = (2,1)\).
\newline
\textbf{Flaw Explanation:}
The error is assuming the parameters must be the same for an \textit{intersection}. Intersection means the paths cross, but not necessarily at the same time. Collision requires the same time. The question asks for intersection.
\textbf{Correct Solution:} Use different parameters, \(t\) and \(s\).
\(x: t+1 = 3-s\)
\(y: t^2 = s\)
Substitute \(s=t^2\) into the first equation: \( t+1 = 3-t^2 \implies t^2+t-2=0 \implies (t+2)(t-1)=0\). So \(t=-2\) or \(t=1\).
If \(t=1\), the point on path 1 is \((2,1)\). If \(t=-2\), the point is \((-1,4)\). These are the two intersection points.

\subsubsection{Problem 4}
Eliminate the parameter for \( x = \sec^2 t - 1, y = \tan t \).
\textbf{Flawed Solution:}
1. Recall the identity \( 1 + \tan^2 t = \sec^2 t \).
2. From the equations, \( \sec^2 t = x+1 \) and \( \tan t = y \).
3. Substitute into the identity: \( 1 + y = x+1 \).
4. Simplifying gives the line \( y = x \).
\newline
\textbf{Flaw Explanation:}
The error is in step 3. The identity involves \(\tan^2 t\), not \(\tan t\). The substitution should be \(1 + y^2 = x+1\).
\textbf{Correct Solution:} The correct equation is \( 1+y^2 = x+1 \), which simplifies to \( x = y^2 \), a parabola opening to the right.

\subsubsection{Problem 5}
Find the highest point on the circle \( x = 4\cos t, y = 4\sin t - 3 \).
\textbf{Flawed Solution:}
1. The highest point occurs when the vertical velocity is zero.
2. \( dy/dt = 4\cos t \).
3. Set \( dy/dt = 0 \): \( 4\cos t = 0 \implies t = \pi/2 \) or \( t = 3\pi/2 \).
4. At \( t = \pi/2 \), \( y = 4\sin(\pi/2) - 3 = 4 - 3 = 1 \).
5. At \( t = 3\pi/2 \), \( y = 4\sin(3\pi/2) - 3 = -4 - 3 = -7 \).
6. The highest point is at y=1.
\newline
\textbf{Flaw Explanation:}
The logic and calculation are perfectly correct. This is not a "flawed" solution. Let me create a true flaw.

\textit{Re-staged Problem 5:} Find the highest point on the circle \( x = 4\cos t, y = 4\sin t - 3 \).
\textbf{Flawed Solution:}
1. This is a circle. The general form is \( (x-h)^2 + (y-k)^2 = r^2 \).
2. The radius \(r=4\). The center is \((h,k) = (0, 3)\).
3. The highest point on a circle is at the center's y-coordinate plus the radius.
4. Highest point y-coordinate = \( k + r = 3 + 4 = 7 \).
\newline
\textbf{Flaw Explanation:}
The error is in step 2. The equation for y is \(y = 4\sin t - 3\), which corresponds to \(y-k = r\sin t\). Here, \(y - (-3) = 4\sin t\), so the center's y-coordinate is \(k=-3\), not \(+3\).
\textbf{Correct Solution:} The center is at \((0, -3)\). The highest point is at \( y = k+r = -3+4 = 1 \).

\end{document}