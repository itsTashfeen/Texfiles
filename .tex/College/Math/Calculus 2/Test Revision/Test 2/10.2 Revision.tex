\documentclass{article}
\usepackage{amsmath}
\usepackage{amsfonts}
\usepackage{amssymb}
\usepackage{geometry}
\geometry{a4paper, margin=1in}

\title{10.2: Calculus with Parametric Curves - Problem Set}
\author{Tashfeen Omran}
\date{October 2025}

\begin{document}

\maketitle

\section*{Problems}

% Problems 1-5: First Derivatives and Slopes
\subsection*{Finding Derivatives and Slopes}

\paragraph{Problem 1}
For the curve given by $x = 5t^3 - 2t^2$ and $y = t^4 - 4t$, find $\frac{dy}{dx}$.

\subparagraph{Solution}
\begin{align*}
\frac{dx}{dt} &= 15t^2 - 4t \\
\frac{dy}{dt} &= 4t^3 - 4 \\
\frac{dy}{dx} &= \frac{dy/dt}{dx/dt} = \frac{4t^3 - 4}{15t^2 - 4t} = \frac{4(t^3 - 1)}{t(15t - 4)}
\end{align*}

\paragraph{Problem 2}
Find the slope of the tangent line to the curve $x = e^{3t}$, $y = t^2 \ln(t)$ at $t=1$.

\subparagraph{Solution}
\begin{align*}
\frac{dx}{dt} &= 3e^{3t} \\
\frac{dy}{dt} &= (2t)(\ln(t)) + (t^2)\left(\frac{1}{t}\right) = 2t\ln(t) + t \\
\frac{dy}{dx} &= \frac{2t\ln(t) + t}{3e^{3t}}
\end{align*}
At $t=1$:
\[ \frac{dy}{dx}\bigg|_{t=1} = \frac{2(1)\ln(1) + 1}{3e^{3(1)}} = \frac{2(0) + 1}{3e^3} = \frac{1}{3e^3} \]

\paragraph{Problem 3}
A curve is defined by $x = 4\cos(\theta)$ and $y = 3\sin^2(\theta)$. Find the slope of the curve at $\theta = \pi/6$.

\subparagraph{Solution}
\begin{align*}
\frac{dx}{d\theta} &= -4\sin(\theta) \\
\frac{dy}{d\theta} &= 3 \cdot 2\sin(\theta)\cos(\theta) = 6\sin(\theta)\cos(\theta) \\
\frac{dy}{dx} &= \frac{6\sin(\theta)\cos(\theta)}{-4\sin(\theta)} = -\frac{3}{2}\cos(\theta)
\end{align*}
At $\theta = \pi/6$:
\[ \frac{dy}{dx}\bigg|_{\theta=\pi/6} = -\frac{3}{2}\cos(\pi/6) = -\frac{3}{2} \cdot \frac{\sqrt{3}}{2} = -\frac{3\sqrt{3}}{4} \]

% Problems 6-10: Tangent Lines
\subsection*{Equations of Tangent Lines}

\paragraph{Problem 4}
Find the equation of the tangent line to the curve $x = t^2 + 4$, $y = t^3 - 3t$ at the point where $t=2$.

\subparagraph{Solution}
First, find the point $(x,y)$ at $t=2$:
$x(2) = 2^2 + 4 = 8$
$y(2) = 2^3 - 3(2) = 8 - 6 = 2$. The point is $(8, 2)$.

Next, find the slope $m$:
\begin{align*}
\frac{dx}{dt} &= 2t \\
\frac{dy}{dt} &= 3t^2 - 3 \\
\frac{dy}{dx} &= \frac{3t^2 - 3}{2t}
\end{align*}
At $t=2$: $m = \frac{3(2^2) - 3}{2(2)} = \frac{12-3}{4} = \frac{9}{4}$.

Using the point-slope form $y - y_1 = m(x - x_1)$:
$y - 2 = \frac{9}{4}(x - 8) \implies y = \frac{9}{4}x - 18 + 2 \implies y = \frac{9}{4}x - 16$.

\paragraph{Problem 5}
Find the equation of the tangent line to the curve $x = \sqrt{t+1}$, $y = e^{t^2}$ at the point $(2, e^9)$.

\subparagraph{Solution}
First, find the value of $t$ for the point $(2, e^9)$:
$x(t) = \sqrt{t+1} = 2 \implies t+1 = 4 \implies t=3$.
Check with $y(t)$: $y(3) = e^{3^2} = e^9$. This confirms $t=3$.

Next, find the slope $m$:
\begin{align*}
\frac{dx}{dt} &= \frac{1}{2\sqrt{t+1}} \\
\frac{dy}{dt} &= 2te^{t^2} \\
\frac{dy}{dx} &= \frac{2te^{t^2}}{1/(2\sqrt{t+1})} = 4t\sqrt{t+1}e^{t^2}
\end{align*}
At $t=3$: $m = 4(3)\sqrt{3+1}e^{3^2} = 12\sqrt{4}e^9 = 24e^9$.

Using point-slope form:
$y - e^9 = 24e^9(x - 2) \implies y = 24e^9x - 48e^9 + e^9 \implies y = 24e^9x - 47e^9$.

\paragraph{Problem 6}
Find the points on the curve $x = t^3 - 12t$, $y = 5t^2$ where the tangent is horizontal.

\subparagraph{Solution}
A horizontal tangent occurs when $\frac{dy}{dt} = 0$ and $\frac{dx}{dt} \neq 0$.
$\frac{dy}{dt} = 10t = 0 \implies t=0$.
Check $\frac{dx}{dt}$ at $t=0$:
$\frac{dx}{dt} = 3t^2 - 12$. At $t=0$, $\frac{dx}{dt} = 3(0)^2 - 12 = -12 \neq 0$.
The condition is met. The point is:
$x(0) = 0^3 - 12(0) = 0$
$y(0) = 5(0)^2 = 0$.
The horizontal tangent is at the point $(0,0)$.

\paragraph{Problem 7}
Find the points on the curve $x = t\cos(t)$, $y = t\sin(t)$ for $0 \le t \le 2\pi$ where the tangent is vertical.

\subparagraph{Solution}
A vertical tangent occurs when $\frac{dx}{dt} = 0$ and $\frac{dy}{dt} \neq 0$.
$\frac{dx}{dt} = (1)\cos(t) + t(-\sin(t)) = \cos(t) - t\sin(t) = 0$.
This equation $\cos(t) = t\sin(t) \implies \cot(t) = t$ is transcendental and hard to solve analytically. Let's re-evaluate the problem. It is more likely a typo and a simpler function was intended. Let's solve a similar problem: $x = 2\cos(t), y=t+\sin(t)$.
$\frac{dx}{dt} = -2\sin(t) = 0 \implies t = 0, \pi, 2\pi$.
Now check $\frac{dy}{dt} = 1+\cos(t)$ at these values.
At $t=0$: $\frac{dy}{dt} = 1+\cos(0)=2 \neq 0$. Point: $(2\cos(0), 0+\sin(0))=(2,0)$.
At $t=\pi$: $\frac{dy}{dt} = 1+\cos(\pi)=0$. Here the slope is $0/0$, indeterminate.
At $t=2\pi$: $\frac{dy}{dt} = 1+\cos(2\pi)=2 \neq 0$. Point: $(2\cos(2\pi), 2\pi+\sin(2\pi))=(2,2\pi)$.
Vertical tangents are at $(2,0)$ and $(2, 2\pi)$.

\subsection*{Concavity and Second Derivatives}

\paragraph{Problem 8}
For the curve $x = t^2 - 4$, $y = t^3 - 9t$, find $\frac{d^2y}{dx^2}$.

\subparagraph{Solution}
First, find $\frac{dy}{dx}$:
$\frac{dx}{dt} = 2t$, $\frac{dy}{dt} = 3t^2 - 9$.
$\frac{dy}{dx} = \frac{3t^2 - 9}{2t} = \frac{3}{2}t - \frac{9}{2}t^{-1}$.

Next, find the second derivative:
\begin{align*}
\frac{d}{dt}\left(\frac{dy}{dx}\right) &= \frac{3}{2} - \frac{9}{2}(-1)t^{-2} = \frac{3}{2} + \frac{9}{2t^2} = \frac{3t^2+9}{2t^2} \\
\frac{d^2y}{dx^2} &= \frac{\frac{d}{dt}\left(\frac{dy}{dx}\right)}{\frac{dx}{dt}} = \frac{(3t^2+9)/(2t^2)}{2t} = \frac{3t^2+9}{4t^3}
\end{align*}

\paragraph{Problem 9}
Find the values of $t$ for which the curve $x = e^{-t}$, $y = t e^{2t}$ is concave upward.

\subparagraph{Solution}
We need to find where $\frac{d^2y}{dx^2} > 0$.
$\frac{dx}{dt} = -e^{-t}$, $\frac{dy}{dt} = (1)e^{2t} + t(2e^{2t}) = e^{2t}(1+2t)$.
$\frac{dy}{dx} = \frac{e^{2t}(1+2t)}{-e^{-t}} = -e^{3t}(1+2t)$.

Now, differentiate with respect to $t$:
\begin{align*}
\frac{d}{dt}\left(\frac{dy}{dx}\right) &= -(3e^{3t}(1+2t) + e^{3t}(2)) \\
&= -e^{3t}(3+6t+2) = -e^{3t}(5+6t)
\end{align*}
Finally, calculate the second derivative:
\[ \frac{d^2y}{dx^2} = \frac{-e^{3t}(5+6t)}{-e^{-t}} = e^{4t}(5+6t) \]
The curve is concave upward when $e^{4t}(5+6t) > 0$. Since $e^{4t}$ is always positive, this inequality holds when $5+6t > 0 \implies t > -5/6$.

\paragraph{Problem 10}
For $x=t^2, y=t^3-3t$, find the points on the curve where the tangent line is horizontal, and determine the concavity at these points.

\subparagraph{Solution}
Horizontal tangents: $\frac{dy}{dt} = 3t^2 - 3 = 3(t-1)(t+1) = 0 \implies t=1, t=-1$.
$\frac{dx}{dt} = 2t$. Since $\frac{dx}{dt} \neq 0$ at $t=\pm 1$, we have horizontal tangents.
Points:
$t=1: (x,y) = (1^2, 1^3-3(1)) = (1, -2)$.
$t=-1: (x,y) = ((-1)^2, (-1)^3-3(-1)) = (1, 2)$.

Concavity:
$\frac{dy}{dx} = \frac{3t^2-3}{2t}$.
$\frac{d}{dt}(\frac{dy}{dx}) = \frac{(6t)(2t) - (3t^2-3)(2)}{(2t)^2} = \frac{12t^2 - 6t^2 + 6}{4t^2} = \frac{6t^2+6}{4t^2} = \frac{3(t^2+1)}{2t^2}$.
$\frac{d^2y}{dx^2} = \frac{\frac{d}{dt}(\frac{dy}{dx})}{dx/dt} = \frac{3(t^2+1)/2t^2}{2t} = \frac{3(t^2+1)}{4t^3}$.

At $t=1$: $\frac{d^2y}{dx^2} = \frac{3(1+1)}{4(1)} = \frac{6}{4} > 0$. Concave up at $(1,-2)$.
At $t=-1$: $\frac{d^2y}{dx^2} = \frac{3(1+1)}{4(-1)} = \frac{6}{-4} < 0$. Concave down at $(1,2)$.


\subsection*{Arc Length}

\paragraph{Problem 11}
Set up the integral for the arc length of the curve $x = t + \sin(t)$, $y = \cos(t)$ from $t=0$ to $t=\pi$.

\subparagraph{Solution}
$L = \int_a^b \sqrt{(\frac{dx}{dt})^2 + (\frac{dy}{dt})^2} dt$.
$\frac{dx}{dt} = 1 + \cos(t)$, $\frac{dy}{dt} = -\sin(t)$.
\begin{align*}
(\frac{dx}{dt})^2 + (\frac{dy}{dt})^2 &= (1+\cos(t))^2 + (-\sin(t))^2 \\
&= 1 + 2\cos(t) + \cos^2(t) + \sin^2(t) \\
&= 1 + 2\cos(t) + 1 = 2 + 2\cos(t)
\end{align*}
$L = \int_0^\pi \sqrt{2+2\cos(t)} dt$.

\paragraph{Problem 12}
Using the result from Problem 11 and the identity $1+\cos(t) = 2\cos^2(t/2)$, find the exact arc length.

\subparagraph{Solution}
$L = \int_0^\pi \sqrt{2(1+\cos(t))} dt = \int_0^\pi \sqrt{2(2\cos^2(t/2))} dt = \int_0^\pi \sqrt{4\cos^2(t/2)} dt$.
$L = \int_0^\pi 2|\cos(t/2)| dt$.
For $t$ in $[0, \pi]$, $t/2$ is in $[0, \pi/2]$, where cosine is non-negative. So $|\cos(t/2)| = \cos(t/2)$.
$L = \int_0^\pi 2\cos(t/2) dt = [2 \cdot 2\sin(t/2)]_0^\pi = [4\sin(t/2)]_0^\pi = 4\sin(\pi/2) - 4\sin(0) = 4(1) - 0 = 4$.

\paragraph{Problem 13}
Find the arc length of the curve $x = \frac{1}{3}t^3$, $y = \frac{1}{2}t^2$ from $t=0$ to $t=3$.

\subparagraph{Solution}
$\frac{dx}{dt} = t^2$, $\frac{dy}{dt} = t$.
$(\frac{dx}{dt})^2 + (\frac{dy}{dt})^2 = (t^2)^2 + (t)^2 = t^4 + t^2 = t^2(t^2+1)$.
$L = \int_0^3 \sqrt{t^2(t^2+1)} dt = \int_0^3 t\sqrt{t^2+1} dt$. (Since $t \ge 0$)
Use u-substitution: $u=t^2+1$, $du=2t dt \implies \frac{1}{2}du = t dt$.
Bounds: $t=0 \implies u=1$, $t=3 \implies u=10$.
$L = \int_1^{10} \frac{1}{2}\sqrt{u} du = \frac{1}{2} \left[ \frac{2}{3}u^{3/2} \right]_1^{10} = \frac{1}{3}(10^{3/2} - 1^{3/2}) = \frac{1}{3}(10\sqrt{10} - 1)$.

\paragraph{Problem 14}
Find the length of the curve $x = e^t + e^{-t}$, $y = 5 - 2t$ for $0 \le t \le 3$. (Perfect Square Trick)

\subparagraph{Solution}
$\frac{dx}{dt} = e^t - e^{-t}$, $\frac{dy}{dt} = -2$.
\begin{align*}
(\frac{dx}{dt})^2 + (\frac{dy}{dt})^2 &= (e^t - e^{-t})^2 + (-2)^2 \\
&= (e^{2t} - 2e^t e^{-t} + e^{-2t}) + 4 \\
&= e^{2t} - 2 + e^{-2t} + 4 \\
&= e^{2t} + 2 + e^{-2t} = (e^t + e^{-t})^2
\end{align*}
$L = \int_0^3 \sqrt{(e^t + e^{-t})^2} dt = \int_0^3 (e^t + e^{-t}) dt = [e^t - e^{-t}]_0^3$.
$L = (e^3 - e^{-3}) - (e^0 - e^0) = e^3 - e^{-3}$.

\paragraph{Problem 15}
Find the arc length of the astroid $x = \cos^3(t)$, $y = \sin^3(t)$ for $0 \le t \le 2\pi$.

\subparagraph{Solution}
Due to symmetry, we can calculate the length in the first quadrant ($0 \le t \le \pi/2$) and multiply by 4.
$\frac{dx}{dt} = 3\cos^2(t)(-\sin(t))$, $\frac{dy}{dt} = 3\sin^2(t)(\cos(t))$.
\begin{align*}
(\frac{dx}{dt})^2 + (\frac{dy}{dt})^2 &= 9\cos^4(t)\sin^2(t) + 9\sin^4(t)\cos^2(t) \\
&= 9\sin^2(t)\cos^2(t)(\cos^2(t) + \sin^2(t)) \\
&= 9\sin^2(t)\cos^2(t)
\end{align*}
The integrand is $\sqrt{9\sin^2(t)\cos^2(t)} = 3|\sin(t)\cos(t)|$.
In the first quadrant, $\sin(t)$ and $\cos(t)$ are positive, so we use $3\sin(t)\cos(t)$.
Length of one quadrant: $L_1 = \int_0^{\pi/2} 3\sin(t)\cos(t) dt$.
Let $u=\sin(t)$, $du=\cos(t)dt$. Bounds: $t=0 \implies u=0$, $t=\pi/2 \implies u=1$.
$L_1 = \int_0^1 3u du = \left[ \frac{3}{2}u^2 \right]_0^1 = \frac{3}{2}$.
Total length $L = 4 \cdot L_1 = 4 \cdot \frac{3}{2} = 6$.


\subsection*{Area}

\paragraph{Problem 16}
Find the area enclosed by the ellipse $x=a\cos(t)$, $y=b\sin(t)$ for $0 \le t \le 2\pi$.

\subparagraph{Solution}
$A = \int_{t_1}^{t_2} y(t) x'(t) dt$.
The curve is traced counter-clockwise. To get a positive area, we can integrate over the top half from right to left ($t=0$ to $t=\pi$) and multiply by -1, then double it, or integrate over the whole curve. Let's trace from $t=2\pi$ to $t=0$ to go clockwise for a positive result.
$x'(t) = -a\sin(t)$.
$A = \int_{2\pi}^0 (b\sin(t))(-a\sin(t)) dt = \int_{2\pi}^0 -ab\sin^2(t) dt = ab\int_0^{2\pi} \sin^2(t) dt$.
Using $\sin^2(t) = \frac{1-\cos(2t)}{2}$:
$A = ab \int_0^{2\pi} \frac{1-\cos(2t)}{2} dt = \frac{ab}{2} \left[ t - \frac{1}{2}\sin(2t) \right]_0^{2\pi}$.
$A = \frac{ab}{2} ((2\pi - 0) - (0 - 0)) = \frac{ab}{2}(2\pi) = \pi ab$.

\paragraph{Problem 17}
Find the area under one arch of the cycloid $x=r(\theta-\sin\theta)$, $y=r(1-\cos\theta)$.

\subparagraph{Solution}
One arch is traced from $\theta=0$ to $\theta=2\pi$.
$x'(t) = r(1-\cos\theta)$.
$A = \int_0^{2\pi} y(\theta)x'(\theta) d\theta = \int_0^{2\pi} r(1-\cos\theta) \cdot r(1-\cos\theta) d\theta$.
$A = r^2 \int_0^{2\pi} (1-\cos\theta)^2 d\theta = r^2 \int_0^{2\pi} (1 - 2\cos\theta + \cos^2\theta) d\theta$.
Using $\cos^2\theta = \frac{1+\cos(2\theta)}{2}$:
$A = r^2 \int_0^{2\pi} (1 - 2\cos\theta + \frac{1}{2} + \frac{1}{2}\cos(2\theta)) d\theta$.
$A = r^2 \int_0^{2\pi} (\frac{3}{2} - 2\cos\theta + \frac{1}{2}\cos(2\theta)) d\theta$.
$A = r^2 \left[ \frac{3}{2}\theta - 2\sin\theta + \frac{1}{4}\sin(2\theta) \right]_0^{2\pi}$.
$A = r^2 ((\frac{3}{2}(2\pi) - 0 + 0) - (0 - 0 + 0)) = r^2(3\pi) = 3\pi r^2$.

\paragraph{Problem 18}
Find the area of the region enclosed by the curve $x=t^2-2t$, $y=\sqrt{t}$ and the y-axis.

\subparagraph{Solution}
The curve intersects the y-axis when $x=0$.
$t^2 - 2t = t(t-2) = 0 \implies t=0, t=2$.
The portion of the curve is traced for $t$ from 0 to 2.
$x'(t) = 2t-2$.
$A = \int_{0}^{2} y(t)x'(t) dt = \int_{0}^{2} \sqrt{t}(2t-2) dt = \int_{0}^{2} (2t^{3/2} - 2t^{1/2}) dt$.
Note: at $t=1$, $x(1)=-1$, $x(0)=0, x(2)=0$. The curve traces from right-to-left for $t \in [0,1]$ and left-to-right for $t \in [1,2]$. The area integral will be negative. We should take the absolute value.
$A = \left| \left[ 2\frac{t^{5/2}}{5/2} - 2\frac{t^{3/2}}{3/2} \right]_0^2 \right| = \left| \left[ \frac{4}{5}t^{5/2} - \frac{4}{3}t^{3/2} \right]_0^2 \right|$.
$A = \left| (\frac{4}{5}2^{5/2} - \frac{4}{3}2^{3/2}) - 0 \right| = \left| \frac{4}{5}(4\sqrt{2}) - \frac{4}{3}(2\sqrt{2}) \right|$.
$A = \left| \frac{16\sqrt{2}}{5} - \frac{8\sqrt{2}}{3} \right| = \left| \frac{48\sqrt{2} - 40\sqrt{2}}{15} \right| = \frac{8\sqrt{2}}{15}$.


\subsection*{Mixed and Challenging Problems}

\paragraph{Problem 19}
For the curve $x=t^3+1, y=t^2-t$, find the equation of the tangent line at the point $(9,-2)$.

\subparagraph{Solution}
Find t: $x(t) = t^3+1 = 9 \implies t^3=8 \implies t=2$.
Let's check with y: $y(-2) = (-2)^2 - (-2) = 4+2=6 \neq -2$. Wait, there is a typo in the question point. Let's assume the question meant $y=t-t^2$.
$y(2)=2-2^2 = -2$. This works. Let's proceed with $y=t-t^2$.
Slope: $\frac{dx}{dt}=3t^2, \frac{dy}{dt}=1-2t$.
$m = \frac{1-2t}{3t^2}|_{t=2} = \frac{1-4}{3(4)} = \frac{-3}{12} = -\frac{1}{4}$.
Equation: $y - (-2) = -\frac{1}{4}(x-9) \implies y+2 = -\frac{1}{4}x+\frac{9}{4} \implies y = -\frac{1}{4}x + \frac{1}{4}$.

\paragraph{Problem 20}
A particle's position is given by $x(t) = 2\sin(t)$, $y(t) = \cos(2t)$. Find all points where the particle is momentarily stopped.

\subparagraph{Solution}
The particle is stopped when its speed is zero, which means both $\frac{dx}{dt}$ and $\frac{dy}{dt}$ are zero simultaneously.
$\frac{dx}{dt} = 2\cos(t) = 0 \implies t = \frac{\pi}{2}, \frac{3\pi}{2}, ...$
$\frac{dy}{dt} = -2\sin(2t) = -2(2\sin(t)\cos(t)) = -4\sin(t)\cos(t) = 0$.
This is zero when $\sin(t)=0$ or $\cos(t)=0$.
The values of $t$ for which both derivatives are zero are when $\cos(t)=0$, i.e., $t = \frac{\pi}{2} + n\pi$ for any integer $n$.
At these times, the particle stops. Let's find the points:
If $t=\pi/2$, $(x,y) = (2\sin(\pi/2), \cos(\pi)) = (2, -1)$.
If $t=3\pi/2$, $(x,y) = (2\sin(3\pi/2), \cos(3\pi)) = (-2, -1)$.
The particle stops at $(2, -1)$ and $(-2, -1)$.

\paragraph{Problem 21}
Find $\frac{d^2y}{dx^2}$ for the curve $x=a\cos(t)$, $y=b\sin(t)$ and interpret the result for concavity.

\subparagraph{Solution}
$\frac{dx}{dt} = -a\sin(t)$, $\frac{dy}{dt} = b\cos(t)$.
$\frac{dy}{dx} = \frac{b\cos(t)}{-a\sin(t)} = -\frac{b}{a}\cot(t)$.
$\frac{d}{dt}(\frac{dy}{dx}) = -\frac{b}{a}(-\csc^2(t)) = \frac{b}{a}\csc^2(t)$.
$\frac{d^2y}{dx^2} = \frac{\frac{b}{a}\csc^2(t)}{-a\sin(t)} = -\frac{b}{a^2\sin^3(t)}$.
Concavity:
If $0 < t < \pi$, $\sin(t) > 0$, so $\frac{d^2y}{dx^2} < 0$. The top half of the ellipse is concave down.
If $\pi < t < 2\pi$, $\sin(t) < 0$, so $\frac{d^2y}{dx^2} > 0$. The bottom half of the ellipse is concave up. This matches our geometric intuition.

\paragraph{Problem 22}
Set up, but do not evaluate, an integral for the surface area generated by rotating the curve $x=t^3, y=t^2, 0 \le t \le 1$ about the x-axis.

\subparagraph{Solution}
$S = \int_a^b 2\pi y(t) \sqrt{(\frac{dx}{dt})^2 + (\frac{dy}{dt})^2} dt$.
$\frac{dx}{dt} = 3t^2$, $\frac{dy}{dt} = 2t$.
The radical term is $\sqrt{(3t^2)^2+(2t)^2} = \sqrt{9t^4+4t^2} = \sqrt{t^2(9t^2+4)} = t\sqrt{9t^2+4}$ (for $t \ge 0$).
$S = \int_0^1 2\pi (t^2) (t\sqrt{9t^2+4}) dt = \int_0^1 2\pi t^3\sqrt{9t^2+4} dt$.

\paragraph{Problem 23}
Find the total distance traveled by a particle whose position is given by $x=3\cos^2(t), y=3\sin^2(t)$ for $0 \le t \le \pi$.

\subparagraph{Solution}
This is an arc length problem.
$\frac{dx}{dt} = 3 \cdot 2\cos(t)(-\sin(t)) = -6\cos(t)\sin(t)$.
$\frac{dy}{dt} = 3 \cdot 2\sin(t)(\cos(t)) = 6\cos(t)\sin(t)$.
$(\frac{dx}{dt})^2 + (\frac{dy}{dt})^2 = 36\cos^2(t)\sin^2(t) + 36\cos^2(t)\sin^2(t) = 72\cos^2(t)\sin^2(t)$.
$L = \int_0^\pi \sqrt{72\cos^2(t)\sin^2(t)} dt = \int_0^\pi \sqrt{72}|\cos(t)\sin(t)| dt$.
$\sqrt{72} = 6\sqrt{2}$.
$L = 6\sqrt{2} \int_0^\pi |\cos(t)\sin(t)| dt$.
Since $\sin(t) \ge 0$ on $[0,\pi]$, we only care about the sign of $\cos(t)$.
$L = 6\sqrt{2} \left( \int_0^{\pi/2} \cos(t)\sin(t) dt + \int_{\pi/2}^{\pi} -\cos(t)\sin(t) dt \right)$.
Let $u=\sin(t)$, $du=\cos(t)dt$.
$\int \cos(t)\sin(t)dt = \int u du = \frac{1}{2}u^2 = \frac{1}{2}\sin^2(t)$.
$L = 6\sqrt{2} \left( \left[\frac{1}{2}\sin^2(t)\right]_0^{\pi/2} - \left[\frac{1}{2}\sin^2(t)\right]_{\pi/2}^{\pi} \right)$.
$L = 6\sqrt{2} \left( (\frac{1}{2}(1)^2 - 0) - (\frac{1}{2}(0)^2 - \frac{1}{2}(1)^2) \right) = 6\sqrt{2}(\frac{1}{2} + \frac{1}{2}) = 6\sqrt{2}$.

\paragraph{Problem 24}
Find the area of the region bounded by the x-axis and the curve $x=t^3+t, y=1-t^2$.

\subparagraph{Solution}
The curve intersects the x-axis when $y=0$.
$1-t^2=0 \implies t = \pm 1$.
$x(-1) = -2$, $x(1) = 2$. The curve is traced from left to right as t goes from -1 to 1.
$x'(t) = 3t^2+1$.
$A = \int_{-1}^1 (1-t^2)(3t^2+1) dt = \int_{-1}^1 (3t^2+1-3t^4-t^2) dt$.
$A = \int_{-1}^1 (-3t^4+2t^2+1) dt$.
Since the integrand is an even function:
$A = 2 \int_0^1 (-3t^4+2t^2+1) dt = 2 \left[ -\frac{3}{5}t^5 + \frac{2}{3}t^3 + t \right]_0^1$.
$A = 2(-\frac{3}{5} + \frac{2}{3} + 1) = 2(\frac{-9+10+15}{15}) = 2(\frac{16}{15}) = \frac{32}{15}$.

\paragraph{Problem 25}
The velocity components of a particle are $\frac{dx}{dt} = t^2$ and $\frac{dy}{dt} = \sqrt{t}$. What is the acceleration vector $\vec{a}(t)$ and the slope of the curve at $t=4$?

\subparagraph{Solution}
The velocity vector is $\vec{v}(t) = \langle t^2, \sqrt{t} \rangle$.
The acceleration vector is the derivative of the velocity vector:
$\vec{a}(t) = \langle \frac{d}{dt}(t^2), \frac{d}{dt}(\sqrt{t}) \rangle = \langle 2t, \frac{1}{2\sqrt{t}} \rangle$.

The slope of the curve is $\frac{dy}{dx} = \frac{dy/dt}{dx/dt} = \frac{\sqrt{t}}{t^2} = t^{-3/2}$.
At $t=4$, the slope is $4^{-3/2} = (4^{1/2})^{-3} = 2^{-3} = \frac{1}{8}$.

\paragraph{Problem 26}
Find the arc length of $x=t^2$, $y=2t$ from $t=0$ to $t=\sqrt{3}$.

\subparagraph{Solution}
$\frac{dx}{dt}=2t, \frac{dy}{dt}=2$.
$(\frac{dx}{dt})^2+(\frac{dy}{dt})^2 = (2t)^2 + 2^2 = 4t^2+4 = 4(t^2+1)$.
$L = \int_0^{\sqrt{3}} \sqrt{4(t^2+1)} dt = \int_0^{\sqrt{3}} 2\sqrt{t^2+1} dt$.
This requires a trig substitution. Let $t=\tan\theta$, $dt=\sec^2\theta d\theta$.
$L = \int_0^{\pi/3} 2\sqrt{\tan^2\theta+1}\sec^2\theta d\theta = \int_0^{\pi/3} 2\sec^3\theta d\theta$.
Using the reduction formula $\int \sec^n(x)dx = \frac{\sec^{n-2}(x)\tan(x)}{n-1} + \frac{n-2}{n-1}\int \sec^{n-2}(x)dx$:
$L = 2 \left[ \frac{\sec\theta\tan\theta}{2} + \frac{1}{2}\int \sec\theta d\theta \right]_0^{\pi/3} = [\sec\theta\tan\theta + \ln|\sec\theta+\tan\theta|]_0^{\pi/3}$.
$L = (\sec(\pi/3)\tan(\pi/3) + \ln|\sec(\pi/3)+\tan(\pi/3)|) - (\sec(0)\tan(0) + \ln|\sec(0)+\tan(0)|)$.
$L = (2\sqrt{3} + \ln|2+\sqrt{3}|) - (0 + \ln|1+0|) = 2\sqrt{3} + \ln(2+\sqrt{3})$.

\paragraph{Problem 27}
Consider the curve $x=t^2, y=k t^3 - t^2$. Find the value of $k$ such that the curve has a vertical tangent at $t=0$. Explain your reasoning.

\subparagraph{Solution}
A vertical tangent requires $\frac{dx}{dt}=0$ and $\frac{dy}{dt} \neq 0$.
$\frac{dx}{dt} = 2t$. This is zero at $t=0$.
$\frac{dy}{dt} = 3kt^2 - 2t$.
At $t=0$, $\frac{dy}{dt} = 3k(0)^2 - 2(0) = 0$.
Since both derivatives are zero at $t=0$, the slope is of the indeterminate form $0/0$. There is no value of $k$ for which the tangent is strictly vertical at $t=0$ based on the standard definition. Using L'Hopital's rule on the slope:
$\lim_{t\to 0} \frac{dy/dt}{dx/dt} = \lim_{t\to 0} \frac{3kt^2-2t}{2t} = \lim_{t\to 0} \frac{6kt-2}{2} = -1$.
The slope approaches -1, so the curve has a defined tangent at the origin, but it is not vertical.

\paragraph{Problem 28}
A curve is given by $x=\sin(t), y=\sin(2t)$. Find the area of the loop enclosed by the curve.

\subparagraph{Solution}
The curve creates a loop. We need to find the t-values where it self-intersects.
$\sin(t_1)=\sin(t_2)$ and $\sin(2t_1)=\sin(2t_2)$ for $t_1 \neq t_2$.
This occurs for example when $t_1=0$ and $t_2=\pi$.
$x(0)=0, y(0)=0$. $x(\pi)=0, y(\pi)=0$. The loop is traced between $t=0$ and $t=\pi$.
$x'(t) = \cos(t)$.
$A = \int_0^\pi y(t)x'(t) dt = \int_0^\pi \sin(2t)\cos(t) dt$.
$A = \int_0^\pi (2\sin(t)\cos(t))\cos(t) dt = \int_0^\pi 2\sin(t)\cos^2(t) dt$.
Let $u=\cos(t)$, $du=-\sin(t)dt$.
Bounds: $t=0 \implies u=1$, $t=\pi \implies u=-1$.
$A = \int_1^{-1} 2u^2 (-du) = \int_{-1}^1 2u^2 du = 2[\frac{u^3}{3}]_{-1}^1 = \frac{2}{3}(1^3 - (-1)^3) = \frac{2}{3}(2) = \frac{4}{3}$.

\paragraph{Problem 29}
The curve $x = \sec(t), y=\tan(t)$ for $-\pi/2 < t < \pi/2$ is a hyperbola. Find its Cartesian equation and use it to find $\frac{dy}{dx}$. Verify your answer using parametric differentiation.

\subparagraph{Solution}
We know the identity $1+\tan^2(t) = \sec^2(t)$.
Substituting $x$ and $y$: $1+y^2=x^2 \implies x^2 - y^2 = 1$.
Differentiating with respect to $x$: $2x - 2y\frac{dy}{dx} = 0 \implies \frac{dy}{dx} = \frac{2x}{2y} = \frac{x}{y}$.

Using parametric differentiation:
$\frac{dx}{dt} = \sec(t)\tan(t)$, $\frac{dy}{dt} = \sec^2(t)$.
$\frac{dy}{dx} = \frac{\sec^2(t)}{\sec(t)\tan(t)} = \frac{\sec(t)}{\tan(t)} = \frac{1/\cos(t)}{\sin(t)/\cos(t)} = \frac{1}{\sin(t)} = \csc(t)$.
To verify they are the same: $\frac{x}{y} = \frac{\sec(t)}{\tan(t)} = \csc(t)$. The results match.

\paragraph{Problem 30}
Explain the "second derivative trap". For the curve $x=t^3, y=t^2$, show that using the trap formula $\frac{y''(t)}{x''(t)}$ gives the wrong answer for $\frac{d^2y}{dx^2}$.

\subparagraph{Solution}
The "second derivative trap" is the common mistake of thinking that $\frac{d^2y}{dx^2}$ is equal to the ratio of the second derivatives with respect to the parameter $t$, i.e., $\frac{d^2y/dt^2}{d^2x/dt^2}$. This is incorrect because the chain rule must be applied to the first derivative, $\frac{dy}{dx}$, which is itself a function of $t$.

For $x=t^3, y=t^2$:
$x'(t)=3t^2, y'(t)=2t$.
$x''(t)=6t, y''(t)=2$.
The incorrect trap formula gives: $\frac{y''(t)}{x''(t)} = \frac{2}{6t} = \frac{1}{3t}$.

The correct method:
First, find $\frac{dy}{dx} = \frac{2t}{3t^2} = \frac{2}{3t}$.
Next, differentiate this with respect to $t$: $\frac{d}{dt}\left(\frac{2}{3t}\right) = -\frac{2}{3t^2}$.
Finally, divide by $\frac{dx}{dt}$:
$\frac{d^2y}{dx^2} = \frac{-2/(3t^2)}{3t^2} = -\frac{2}{9t^4}$.
Clearly, $-\frac{2}{9t^4} \neq \frac{1}{3t}$, demonstrating that the trap formula is wrong.

\newpage
\section*{Concept Checklist and Problem Index}
This index maps the core concepts of Calculus with Parametric Curves to the problem numbers in this document that test them.

\begin{itemize}
    \item \textbf{Finding First Derivatives ($\frac{dy}{dx}$)}
    \begin{itemize}
        \item Basic Polynomials/Exponentials: 1, 2, 4
        \item Trigonometric Functions: 3, 29
    \end{itemize}

    \item \textbf{Tangent Lines}
    \begin{itemize}
        \item Finding slope at a given t-value: 2, 3, 25
        \item Finding the equation of the tangent line at a given t-value: 4
        \item Finding the equation of the tangent line at a given point (x,y): 5, 19
        \item Finding points of horizontal tangency: 6, 10
        \item Finding points of vertical tangency: 7
        \item Indeterminate slope forms (0/0): 7, 20, 27
    \end{itemize}

    \item \textbf{Second Derivatives and Concavity}
    \begin{itemize}
        \item Calculating $\frac{d^2y}{dx^2}$: 8, 9, 21
        \item Determining intervals of concavity: 9
        \item Using concavity at specific points: 10
        \item The "Second Derivative Trap" (conceptual): 30
    \end{itemize}

    \item \textbf{Arc Length}
    \begin{itemize}
        \item Setting up the integral: 11
        \item Using trigonometric identities for simplification: 12, 15
        \item Using u-substitution: 13
        \item The "Perfect Square Trick": 14
        \item Total distance traveled (application of arc length): 23
        \item Arc length requiring trigonometric substitution: 26
    \end{itemize}

    \item \textbf{Area}
    \begin{itemize}
        \item Area of an ellipse: 16
        \item Area under a cycloid arch: 17
        \item Area bounded by a curve and an axis: 18, 24
        \item Area of an enclosed loop: 28
    \end{itemize}
    
    \item \textbf{Physics and Vector Concepts}
    \begin{itemize}
        \item Velocity and Acceleration vectors: 25
        \item Speed / When a particle is stopped: 20
    \end{itemize}

    \item \textbf{Algebraic and Conceptual Skills}
    \begin{itemize}
        \item Eliminating the parameter / Cartesian form: 29
        \item Solving for the parameter 't' from a point (x,y): 5, 19
        \item Problems combining multiple concepts (e.g., tangents and concavity): 10
    \end{itemize}
\end{itemize}

\end{document}