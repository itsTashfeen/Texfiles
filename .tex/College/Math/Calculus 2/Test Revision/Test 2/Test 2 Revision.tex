\documentclass[12pt]{article}

% Preamble: Required packages for formatting and mathematics
\usepackage{amsmath}        % For advanced math environments like align
\usepackage{amssymb}        % For extra math symbols
\usepackage{geometry}       % To set margins for better layout
\usepackage{enumitem}       % For customized lists
\usepackage{graphicx}       % If images were needed
\usepackage{palatino}       % A more visually appealing font
\usepackage[T1]{fontenc}

% Set page geometry
\geometry{
 a4paper,
 total={170mm,257mm},
 left=20mm,
 top=20mm,
}

% Define a new command for titles to make them stand out
\newcommand{\questiontitle}[1]{\subsection*{#1}}

\begin{document}

\title{\textbf{Calculus II - Test 2 Review Solutions}}
\author{Prof. Ibrahim El Haitami - MAC 2312}
\date{}
\maketitle

\hrulefill
\vspace{1em}

\questiontitle{1. Evaluate the convergence and divergence of the following integrals}

\begin{enumerate}[label=\alph*.]
    \item \textbf{Integral: } $\displaystyle \int_{2}^{\infty} \frac{2}{x^2 - x} \,dx$
    
    \textit{Explanation:} This is an improper integral because the upper limit of integration is infinite. We first use partial fraction decomposition on the integrand.
    \begin{align*}
        \frac{2}{x^2 - x} = \frac{2}{x(x-1)} &= \frac{A}{x} + \frac{B}{x-1} \\
        2 &= A(x-1) + Bx \\
        \text{Let } x=0: 2 &= A(-1) \implies A = -2 \\
        \text{Let } x=1: 2 &= B(1) \implies B = 2 \\
        \text{So, } \frac{2}{x(x-1)} &= \frac{2}{x-1} - \frac{2}{x}
    \end{align*}
    Now we evaluate the integral as a limit:
    \begin{align*}
        \int_{2}^{\infty} \left(\frac{2}{x-1} - \frac{2}{x}\right) \,dx &= \lim_{b \to \infty} \int_{2}^{b} \left(\frac{2}{x-1} - \frac{2}{x}\right) \,dx \\
        &= \lim_{b \to \infty} \left[ 2\ln|x-1| - 2\ln|x| \right]_{2}^{b} \\
        &= \lim_{b \to \infty} \left[ 2\ln\left|\frac{x-1}{x}\right| \right]_{2}^{b} \\
        &= \lim_{b \to \infty} \left( 2\ln\left|\frac{b-1}{b}\right| - 2\ln\left|\frac{2-1}{2}\right| \right) \\
        &= 2\ln(1) - 2\ln\left(\frac{1}{2}\right) = 0 - 2(-\ln 2) = 2\ln 2
    \end{align*}
    \textbf{Result:} The integral \textbf{converges} to $2\ln 2$.

    \item \textbf{Integral: } $\displaystyle \int_{0}^{1} \frac{x+1}{\sqrt{x^2 + 2x}} \,dx$
    
    \textit{Explanation:} This integral is improper because the integrand has a discontinuity at $x=0$. We use a u-substitution. Let $u = x^2+2x$, so $du = (2x+2)dx = 2(x+1)dx$.
    \begin{align*}
        \int \frac{x+1}{\sqrt{x^2+2x}} \,dx &= \int \frac{1}{\sqrt{u}} \cdot \frac{du}{2} = \frac{1}{2} \int u^{-1/2} \,du \\
        &= \frac{1}{2} \cdot 2u^{1/2} = \sqrt{u} = \sqrt{x^2+2x}
    \end{align*}
    Now evaluate the limit:
    \begin{align*}
        \int_{0}^{1} \frac{x+1}{\sqrt{x^2 + 2x}} \,dx &= \lim_{a \to 0^+} \int_{a}^{1} \frac{x+1}{\sqrt{x^2 + 2x}} \,dx \\
        &= \lim_{a \to 0^+} \left[ \sqrt{x^2+2x} \right]_{a}^{1} \\
        &= \lim_{a \to 0^+} \left( \sqrt{1^2+2(1)} - \sqrt{a^2+2a} \right) \\
        &= \sqrt{3} - 0 = \sqrt{3}
    \end{align*}
    \textbf{Result:} The integral \textbf{converges} to $\sqrt{3}$.

    \item \textbf{Integral: } $\displaystyle \int_{0}^{\pi/2} \tan\theta \,d\theta$
    
    \textit{Explanation:} This is improper because $\tan\theta$ has a vertical asymptote at $\theta = \pi/2$.
    \begin{align*}
        \int_{0}^{\pi/2} \tan\theta \,d\theta &= \lim_{b \to (\pi/2)^-} \int_{0}^{b} \tan\theta \,d\theta \\
        &= \lim_{b \to (\pi/2)^-} \left[ -\ln|\cos\theta| \right]_{0}^{b} \\
        &= \lim_{b \to (\pi/2)^-} (-\ln|\cos b| - (-\ln|\cos 0|)) \\
        &= \lim_{b \to (\pi/2)^-} (-\ln|\cos b|) + \ln(1)
    \end{align*}
    As $b \to (\pi/2)^-$, $\cos b \to 0^+$, so $-\ln|\cos b| \to \infty$.
    \textbf{Result:} The integral \textbf{diverges}.

    \item \textbf{Integral: } $\displaystyle \int_{0}^{1} \frac{dx}{\sqrt{1-x^2}}$
    
    \textit{Explanation:} Improper due to a discontinuity at $x=1$. This is a standard integral form.
    \begin{align*}
        \int_{0}^{1} \frac{dx}{\sqrt{1-x^2}} &= \lim_{b \to 1^-} \int_{0}^{b} \frac{dx}{\sqrt{1-x^2}} \\
        &= \lim_{b \to 1^-} \left[ \arcsin(x) \right]_{0}^{b} \\
        &= \lim_{b \to 1^-} (\arcsin(b) - \arcsin(0)) \\
        &= \arcsin(1) - 0 = \frac{\pi}{2}
    \end{align*}
    \textbf{Result:} The integral \textbf{converges} to $\frac{\pi}{2}$.
    
    \item \textbf{Integral: } $\displaystyle \int_{-\infty}^{\infty} \frac{2x}{(x^2+1)^2} \,dx$
    
    \textit{Explanation:} This is improper because of the infinite limits. We split it at $x=0$.
    \begin{align*}
        \int_{-\infty}^{\infty} \frac{2x}{(x^2+1)^2} \,dx &= \int_{-\infty}^{0} \frac{2x}{(x^2+1)^2} \,dx + \int_{0}^{\infty} \frac{2x}{(x^2+1)^2} \,dx
    \end{align*}
    Let $u = x^2+1$, $du = 2x\,dx$. The antiderivative is $\int \frac{1}{u^2} \,du = -\frac{1}{u} = -\frac{1}{x^2+1}$.
    \begin{align*}
        \text{Part 1: } \lim_{a \to -\infty} \left[ -\frac{1}{x^2+1} \right]_{a}^{0} &= \left(-\frac{1}{0^2+1}\right) - \lim_{a \to -\infty}\left(-\frac{1}{a^2+1}\right) = -1 - 0 = -1 \\
        \text{Part 2: } \lim_{b \to \infty} \left[ -\frac{1}{x^2+1} \right]_{0}^{b} &= \lim_{b \to \infty}\left(-\frac{1}{b^2+1}\right) - \left(-\frac{1}{0^2+1}\right) = 0 - (-1) = 1
    \end{align*}
    The total value is $-1 + 1 = 0$.
    \textbf{Result:} The integral \textbf{converges} to $0$.
\end{enumerate}

\newpage
\questiontitle{2. Find the length of the curves}

\textit{Arc Length Formula:} $L = \displaystyle \int_{a}^{b} \sqrt{1 + (y')^2} \,dx$

\begin{enumerate}[label=\alph*.]
    \item \textbf{Curve:} $y = \frac{1}{3}(x^2+2)^{3/2}$ from $x=0$ to $x=3$
    \begin{align*}
        y' &= \frac{1}{3} \cdot \frac{3}{2}(x^2+2)^{1/2} \cdot 2x = x\sqrt{x^2+2} \\
        1+(y')^2 &= 1 + (x\sqrt{x^2+2})^2 = 1 + x^2(x^2+2) \\
        &= 1 + x^4 + 2x^2 = (x^2+1)^2 \\
        L &= \int_{0}^{3} \sqrt{(x^2+1)^2} \,dx = \int_{0}^{3} (x^2+1) \,dx \\
        &= \left[ \frac{x^3}{3} + x \right]_{0}^{3} = \left(\frac{27}{3} + 3\right) - 0 = 9+3 = 12
    \end{align*}
    \textbf{Length:} $12$.

    \item \textbf{Curve:} $y = \ln(x) - \frac{x^2}{8}$ from $x=1$ to $x=2$
    \begin{align*}
        y' &= \frac{1}{x} - \frac{2x}{8} = \frac{1}{x} - \frac{x}{4} \\
        1+(y')^2 &= 1 + \left(\frac{1}{x} - \frac{x}{4}\right)^2 = 1 + \left(\frac{1}{x^2} - \frac{1}{2} + \frac{x^2}{16}\right) \\
        &= \frac{1}{x^2} + \frac{1}{2} + \frac{x^2}{16} = \left(\frac{1}{x} + \frac{x}{4}\right)^2 \\
        L &= \int_{1}^{2} \sqrt{\left(\frac{1}{x} + \frac{x}{4}\right)^2} \,dx = \int_{1}^{2} \left(\frac{1}{x} + \frac{x}{4}\right) \,dx \\
        &= \left[ \ln|x| + \frac{x^2}{8} \right]_{1}^{2} = \left(\ln 2 + \frac{4}{8}\right) - \left(\ln 1 + \frac{1}{8}\right) = \ln 2 + \frac{1}{2} - \frac{1}{8} = \ln 2 + \frac{3}{8}
    \end{align*}
    \textbf{Length:} $\ln 2 + \frac{3}{8}$.
    
    \item \textbf{Curve:} $y = \ln(\sec x)$ from $x=0$ to $x=\pi/4$
    \begin{align*}
        y' &= \frac{1}{\sec x} \cdot (\sec x \tan x) = \tan x \\
        1+(y')^2 &= 1 + \tan^2 x = \sec^2 x \\
        L &= \int_{0}^{\pi/4} \sqrt{\sec^2 x} \,dx = \int_{0}^{\pi/4} \sec x \,dx \\
        &= \left[ \ln|\sec x + \tan x| \right]_{0}^{\pi/4} \\
        &= \ln|\sec(\pi/4) + \tan(\pi/4)| - \ln|\sec(0) + \tan(0)| \\
        &= \ln|\sqrt{2} + 1| - \ln|1+0| = \ln(\sqrt{2}+1)
    \end{align*}
    \textbf{Length:} $\ln(\sqrt{2}+1)$.
\end{enumerate}

\newpage
\questiontitle{3. Calculate the surface area generated by revolving the curves about the indicated axis}

\textit{Surface Area Formula (x-axis):} $S = \displaystyle \int_{a}^{b} 2\pi y \sqrt{1 + (y')^2} \,dx$

\begin{enumerate}[label=\alph*.]
    \item \textbf{Curve:} $y=\sqrt{2x-x^2}$, $1 \le x \le 2$; about the x-axis
    \begin{align*}
        y' &= \frac{1}{2\sqrt{2x-x^2}} \cdot (2-2x) = \frac{1-x}{\sqrt{2x-x^2}} \\
        1+(y')^2 &= 1 + \frac{(1-x)^2}{2x-x^2} = \frac{2x-x^2 + (1-2x+x^2)}{2x-x^2} = \frac{1}{2x-x^2} \\
        S &= \int_{1}^{2} 2\pi y \sqrt{1+(y')^2} \,dx \\
        &= \int_{1}^{2} 2\pi \sqrt{2x-x^2} \sqrt{\frac{1}{2x-x^2}} \,dx \\
        &= \int_{1}^{2} 2\pi \,dx = [2\pi x]_{1}^{2} = 2\pi(2) - 2\pi(1) = 2\pi
    \end{align*}
    \textbf{Surface Area:} $2\pi$.

    \item \textbf{Curve:} $y = \sqrt{1+e^x}$, $0 \le x \le 1$; about the x-axis
    \begin{align*}
        y' &= \frac{e^x}{2\sqrt{1+e^x}} \\
        1+(y')^2 &= 1 + \frac{e^{2x}}{4(1+e^x)} = \frac{4(1+e^x) + e^{2x}}{4(1+e^x)} = \frac{4+4e^x+e^{2x}}{4(1+e^x)} = \frac{(e^x+2)^2}{4(1+e^x)} \\
        S &= \int_{0}^{1} 2\pi \sqrt{1+e^x} \sqrt{\frac{(e^x+2)^2}{4(1+e^x)}} \,dx \\
        &= \int_{0}^{1} 2\pi \sqrt{1+e^x} \frac{e^x+2}{2\sqrt{1+e^x}} \,dx \\
        &= \pi \int_{0}^{1} (e^x+2) \,dx = \pi \left[ e^x+2x \right]_{0}^{1} \\
        &= \pi \left( (e^1+2) - (e^0+0) \right) = \pi(e+2-1) = \pi(e+1)
    \end{align*}
    \textbf{Surface Area:} $\pi(e+1)$.
\end{enumerate}

\newpage
\questiontitle{4. Find and graph the Cartesian equation}

\begin{enumerate}[label=\alph*.]
    \item \textbf{Equations:} $x=4\cos t, y=2\sin t, 0 \le t \le 2\pi$
    
    \textit{Eliminate the parameter:}
    From $x=4\cos t \implies \cos t = x/4$. From $y=2\sin t \implies \sin t = y/2$.
    Using the identity $\cos^2 t + \sin^2 t = 1$:
    $$ \left(\frac{x}{4}\right)^2 + \left(\frac{y}{2}\right)^2 = 1 \implies \frac{x^2}{16} + \frac{y^2}{4} = 1 $$
    \textbf{Path:} This is an ellipse centered at $(0,0)$ with a horizontal major axis of length 8 and a vertical minor axis of length 4.
    
    \textbf{Direction:} At $t=0$, $(x,y)=(4,0)$. At $t=\pi/2$, $(x,y)=(0,2)$. At $t=\pi$, $(x,y)=(-4,0)$. The particle travels \textbf{counter-clockwise} starting from $(4,0)$.

    \item \textbf{Equations:} $x=\sin t, y=\cos 2t, -\pi/2 \le t \le \pi/2$
    
    \textit{Eliminate the parameter:}
    Using the identity $\cos 2t = 1-2\sin^2 t$:
    $$ y = 1 - 2x^2 $$
    \textbf{Path:} This is a parabola opening downwards with its vertex at $(0,1)$.
    
    \textbf{Direction:} At $t=-\pi/2$, $(x,y)=(-1,-1)$. At $t=0$, $(x,y)=(0,1)$. At $t=\pi/2$, $(x,y)=(1,-1)$. The particle moves from $(-1,-1)$ to $(1,-1)$ along the parabola.
    
    \item \textbf{Equations:} $x=1+\sin t, y=\cos t-2, 0 \le t \le \pi$
    
    \textit{Eliminate the parameter:}
    From the equations, $\sin t = x-1$ and $\cos t = y+2$.
    Using $\sin^2 t + \cos^2 t = 1$:
    $$ (x-1)^2 + (y+2)^2 = 1 $$
    \textbf{Path:} This is a circle of radius 1 centered at $(1,-2)$.
    
    \textbf{Direction:} At $t=0$, $(x,y)=(1,-1)$. At $t=\pi/2$, $(x,y)=(2,-2)$. At $t=\pi$, $(x,y)=(1,-3)$. The particle traces the \textbf{top semi-circle} from right to left, starting at $(1,-1)$.
\end{enumerate}

\newpage
\questiontitle{5. Find $\frac{dy}{dx}$ and $\frac{d^2y}{dx^2}$ as a function of $t$}

\textit{Formulas:} $\frac{dy}{dx} = \frac{dy/dt}{dx/dt}$ and $\frac{d^2y}{dx^2} = \frac{\frac{d}{dt}(dy/dx)}{dx/dt}$

\begin{enumerate}[label=\alph*.]
    \item \textbf{Equations:} $x=t-t^2, y=t-t^3$
    \begin{align*}
        \frac{dx}{dt} &= 1-2t \quad \quad \frac{dy}{dt} = 1-3t^2 \\
        \frac{dy}{dx} &= \frac{1-3t^2}{1-2t} \\
        \frac{d}{dt}\left(\frac{dy}{dx}\right) &= \frac{-6t(1-2t) - (1-3t^2)(-2)}{(1-2t)^2} = \frac{-6t+12t^2+2-6t^2}{(1-2t)^2} = \frac{6t^2-6t+2}{(1-2t)^2} \\
        \frac{d^2y}{dx^2} &= \frac{(6t^2-6t+2)/(1-2t)^2}{1-2t} = \frac{2(3t^2-3t+1)}{(1-2t)^3}
    \end{align*}

    \item \textbf{Equations:} $x=\frac{1}{t+1}, y=\frac{t}{t-1}$
    \begin{align*}
        \frac{dx}{dt} &= -(t+1)^{-2} = \frac{-1}{(t+1)^2} \\
        \frac{dy}{dt} &= \frac{1(t-1)-t(1)}{(t-1)^2} = \frac{-1}{(t-1)^2} \\
        \frac{dy}{dx} &= \frac{-1/(t-1)^2}{-1/(t+1)^2} = \frac{(t+1)^2}{(t-1)^2} \\
        \frac{d}{dt}\left(\frac{dy}{dx}\right) &= \frac{2(t+1)(t-1)^2 - (t+1)^2(2(t-1))}{(t-1)^4} \\
        &= \frac{2(t+1)(t-1)[(t-1)-(t+1)]}{(t-1)^4} = \frac{2(t+1)(-2)}{(t-1)^3} = \frac{-4(t+1)}{(t-1)^3} \\
        \frac{d^2y}{dx^2} &= \frac{-4(t+1)/(t-1)^3}{-1/(t+1)^2} = \frac{4(t+1)^3}{(t-1)^3}
    \end{align*}
\end{enumerate}

\newpage
\questiontitle{6. Find an equation of the line tangent to the curve}

\begin{enumerate}[label=\alph*.]
    \item \textbf{Curve:} $x=\sec t, y=\tan t$ at $t=\pi/4$
    \begin{itemize}
        \item \textbf{Point:} $x(\pi/4)=\sec(\pi/4)=\sqrt{2}$, $y(\pi/4)=\tan(\pi/4)=1$. Point is $(\sqrt{2}, 1)$.
        \item \textbf{Slope:} $\frac{dx}{dt}=\sec t \tan t$, $\frac{dy}{dt}=\sec^2 t$.
        $\frac{dy}{dx} = \frac{\sec^2 t}{\sec t \tan t} = \frac{\sec t}{\tan t} = \csc t$.
        At $t=\pi/4$, slope $m = \csc(\pi/4)=\sqrt{2}$.
        \item \textbf{Equation:} $y-1 = \sqrt{2}(x-\sqrt{2}) \implies y-1 = \sqrt{2}x - 2 \implies y = \sqrt{2}x - 1$.
    \end{itemize}

    \item \textbf{Curve:} $x=t-\sin t, y=1-\cos t$ at $t=\pi/3$
    \begin{itemize}
        \item \textbf{Point:} $x(\pi/3)=\frac{\pi}{3}-\sin(\pi/3)=\frac{\pi}{3}-\frac{\sqrt{3}}{2}$.
        $y(\pi/3)=1-\cos(\pi/3)=1-\frac{1}{2}=\frac{1}{2}$. Point is $(\frac{\pi}{3}-\frac{\sqrt{3}}{2}, \frac{1}{2})$.
        \item \textbf{Slope:} $\frac{dx}{dt}=1-\cos t$, $\frac{dy}{dt}=\sin t$.
        $\frac{dy}{dx} = \frac{\sin t}{1-\cos t}$.
        At $t=\pi/3$, slope $m = \frac{\sin(\pi/3)}{1-\cos(\pi/3)} = \frac{\sqrt{3}/2}{1-1/2} = \frac{\sqrt{3}/2}{1/2} = \sqrt{3}$.
        \item \textbf{Equation:} $y - \frac{1}{2} = \sqrt{3}\left(x - \left(\frac{\pi}{3}-\frac{\sqrt{3}}{2}\right)\right)$.
    \end{itemize}
\end{enumerate}

\newpage
\questiontitle{7. Find the lengths of the curves}

\textit{Parametric Arc Length Formula:} $L = \displaystyle \int_{t_1}^{t_2} \sqrt{(\frac{dx}{dt})^2 + (\frac{dy}{dt})^2} \,dt$

\begin{enumerate}[label=\alph*.]
    \item \textbf{Curve:} $x = \frac{t^2}{2}, y = \frac{1}{3}(2t+1)^{3/2}, 0 \le t \le 4$
    \begin{align*}
        \frac{dx}{dt} &= t \quad \quad \frac{dy}{dt} = \frac{1}{3} \cdot \frac{3}{2}(2t+1)^{1/2} \cdot 2 = \sqrt{2t+1} \\
        \left(\frac{dx}{dt}\right)^2 + \left(\frac{dy}{dt}\right)^2 &= t^2 + (\sqrt{2t+1})^2 = t^2 + 2t + 1 = (t+1)^2 \\
        L &= \int_{0}^{4} \sqrt{(t+1)^2} \,dt = \int_{0}^{4} (t+1) \,dt \\
        &= \left[ \frac{t^2}{2} + t \right]_{0}^{4} = \left(\frac{16}{2} + 4\right) - 0 = 8+4=12
    \end{align*}
    \textbf{Length:} $12$.

    \item \textbf{Curve:} $x=t^3, y=\frac{3t^2}{2}, 0 \le t \le \sqrt{3}$
    \begin{align*}
        \frac{dx}{dt} &= 3t^2 \quad \quad \frac{dy}{dt} = 3t \\
        \left(\frac{dx}{dt}\right)^2 + \left(\frac{dy}{dt}\right)^2 &= (3t^2)^2 + (3t)^2 = 9t^4 + 9t^2 = 9t^2(t^2+1) \\
        L &= \int_{0}^{\sqrt{3}} \sqrt{9t^2(t^2+1)} \,dt = \int_{0}^{\sqrt{3}} 3t\sqrt{t^2+1} \,dt
    \end{align*}
    Let $u = t^2+1$, $du=2t\,dt \implies \frac{3}{2}du = 3t\,dt$. When $t=0, u=1$. When $t=\sqrt{3}, u=4$.
    \begin{align*}
        L &= \int_{1}^{4} \frac{3}{2}\sqrt{u} \,du = \frac{3}{2} \int_{1}^{4} u^{1/2} \,du = \frac{3}{2} \left[ \frac{2}{3}u^{3/2} \right]_{1}^{4} \\
        &= \left[ u^{3/2} \right]_{1}^{4} = 4^{3/2} - 1^{3/2} = 8-1=7
    \end{align*}
    \textbf{Length:} $7$.
\end{enumerate}

\newpage
\questiontitle{8. For what values of $t$ does the curve have a vertical tangent?}

\textbf{Equations:} $x=t^3-t^2-1, y=t^4+2t^2-8t$

\textit{Explanation:} A vertical tangent occurs when the slope $\frac{dy}{dx}$ is undefined. This happens when $\frac{dx}{dt}=0$ and $\frac{dy}{dt} \ne 0$.

\begin{enumerate}
    \item Find when $\frac{dx}{dt} = 0$:
    $$ \frac{dx}{dt} = 3t^2 - 2t = t(3t-2) $$
    Setting $\frac{dx}{dt}=0$ gives $t=0$ and $t=2/3$.
    
    \item Check if $\frac{dy}{dt} \ne 0$ at these values:
    $$ \frac{dy}{dt} = 4t^3+4t-8 $$
    \begin{itemize}
        \item At $t=0$: $\frac{dy}{dt} = 4(0)^3+4(0)-8 = -8 \ne 0$.
        \item At $t=2/3$: $\frac{dy}{dt} = 4(2/3)^3+4(2/3)-8 = 4(8/27)+8/3-8 = 32/27+72/27-216/27 = -112/27 \ne 0$.
    \end{itemize}
\end{enumerate}
\textbf{Result:} A vertical tangent exists at both $t=0$ and $t=2/3$.

\questiontitle{9. For what value of $t$ is the particle at rest?}

\textbf{Equations:} $x=t^3-3t^2, y=2t^3-3t^2-12t$

\textit{Explanation:} A particle is at rest when its velocity is zero. This means both velocity components, $\frac{dx}{dt}$ and $\frac{dy}{dt}$, must be zero simultaneously.

\begin{enumerate}
    \item Set $\frac{dx}{dt} = 0$:
    $$ \frac{dx}{dt} = 3t^2 - 6t = 3t(t-2) $$
    The solutions are $t=0$ and $t=2$.
    
    \item Set $\frac{dy}{dt} = 0$:
    $$ \frac{dy}{dt} = 6t^2 - 6t - 12 = 6(t^2 - t - 2) = 6(t-2)(t+1) $$
    The solutions are $t=2$ and $t=-1$.
    
    \item Find the common solution: The only value of $t$ that makes both derivatives zero is $t=2$.
\end{enumerate}
\textbf{Result:} The particle is at rest when $t=2$.

\end{document}