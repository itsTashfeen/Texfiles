\documentclass[12pt]{article}

% Preamble: Required packages for formatting and mathematics
\usepackage{amsmath}        % For advanced math environments like align
\usepackage{amssymb}        % For extra math symbols
\usepackage{geometry}       % To set margins for better layout
\usepackage{enumitem}       % For customized lists
\usepackage{graphicx}       % If images were needed
\usepackage{palatino}       % A more visually appealing font
\usepackage[T1]{fontenc}
\usepackage{hyperref}       % For hyperlinks in the concept check
\usepackage{amsfonts}

% Set page geometry
\geometry{
 a4paper,
 total={170mm,257mm},
 left=20mm,
 top=20mm,
}

% Define a new command for titles to make them stand out
\newcommand{\questiontitle}[1]{\subsection*{#1}}
\newcommand{\problemsettitle}[1]{\section*{#1}}

\begin{document}

\title{\textbf{Calculus II - Test 2 Additional Problems}}
\author{Based on Review by Prof. Ibrahim El Haitami - MAC 2312}
\date{}
\maketitle

\hrulefill
\vspace{1em}

\problemsettitle{1.a. Improper Integrals: Infinite Limit with Partial Fractions}

\begin{enumerate}
    \item \textbf{Integral: } $\displaystyle \int_{3}^{\infty} \frac{4}{x^2 - 4} \,dx$
    \begin{align*}
        \frac{4}{x^2 - 4} = \frac{4}{(x-2)(x+2)} &= \frac{A}{x-2} + \frac{B}{x+2} \implies A=1, B=-1 \\
        \int_{3}^{\infty} \left(\frac{1}{x-2} - \frac{1}{x+2}\right) \,dx &= \lim_{b \to \infty} \left[ \ln|x-2| - \ln|x+2| \right]_{3}^{b} \\
        &= \lim_{b \to \infty} \left[ \ln\left|\frac{x-2}{x+2}\right| \right]_{3}^{b} \\
        &= \lim_{b \to \infty} \left( \ln\left|\frac{b-2}{b+2}\right| - \ln\left|\frac{1}{5}\right| \right) = \ln(1) - \ln(1/5) = \ln 5
    \end{align*}
    \textbf{Result:} The integral \textbf{converges} to $\ln 5$.

    \item \textbf{Integral: } $\displaystyle \int_{2}^{\infty} \frac{6}{x^2 + 2x - 3} \,dx$
    \begin{align*}
        \frac{6}{(x+3)(x-1)} &= \frac{A}{x+3} + \frac{B}{x-1} \implies A=-3/2, B=3/2 \\
        \frac{3}{2} \int_{2}^{\infty} \left(\frac{1}{x-1} - \frac{1}{x+3}\right) \,dx &= \frac{3}{2} \lim_{b \to \infty} \left[ \ln\left|\frac{x-1}{x+3}\right| \right]_{2}^{b} \\
        &= \frac{3}{2} \left( \ln(1) - \ln\left|\frac{1}{5}\right| \right) = \frac{3}{2}\ln 5
    \end{align*}
    \textbf{Result:} The integral \textbf{converges} to $\frac{3}{2}\ln 5$.

    \item \textbf{Integral: } $\displaystyle \int_{0}^{\infty} \frac{1}{x^2+3x+2} \,dx$
    \begin{align*}
        \frac{1}{(x+1)(x+2)} = \frac{1}{x+1} - \frac{1}{x+2} \\
        \int_{0}^{\infty} \left(\frac{1}{x+1} - \frac{1}{x+2}\right) \,dx &= \lim_{b \to \infty} \left[ \ln\left|\frac{x+1}{x+2}\right| \right]_{0}^{b} \\
        &= \ln(1) - \ln(1/2) = \ln 2
    \end{align*}
    \textbf{Result:} The integral \textbf{converges} to $\ln 2$.

    \item \textbf{Integral: } $\displaystyle \int_{1}^{\infty} \frac{1}{x(x+1)} \,dx$
    \begin{align*}
        \frac{1}{x(x+1)} = \frac{1}{x} - \frac{1}{x+1} \\
        \int_{1}^{\infty} \left(\frac{1}{x} - \frac{1}{x+1}\right) \,dx &= \lim_{b \to \infty} \left[ \ln\left|\frac{x}{x+1}\right| \right]_{1}^{b} \\
        &= \ln(1) - \ln(1/2) = \ln 2
    \end{align*}
    \textbf{Result:} The integral \textbf{converges} to $\ln 2$.
    
    \item \textbf{Integral: } $\displaystyle \int_{4}^{\infty} \frac{3}{x^2-9} \,dx$
    \begin{align*}
        \frac{3}{(x-3)(x+3)} &= \frac{1/2}{x-3} - \frac{1/2}{x+3} \\
        \frac{1}{2} \int_{4}^{\infty} \left(\frac{1}{x-3} - \frac{1}{x+3}\right) \,dx &= \frac{1}{2} \lim_{b \to \infty} \left[ \ln\left|\frac{x-3}{x+3}\right| \right]_{4}^{b} \\
        &= \frac{1}{2} \left( \ln(1) - \ln(1/7) \right) = \frac{1}{2}\ln 7
    \end{align*}
    \textbf{Result:} The integral \textbf{converges} to $\frac{1}{2}\ln 7$.

    \item \textbf{Integral: } $\displaystyle \int_{1}^{\infty} \frac{4}{x(x+2)} \,dx$
    \begin{align*}
        \frac{4}{x(x+2)} &= \frac{2}{x} - \frac{2}{x+2} \\
        \int_{1}^{\infty} \left(\frac{2}{x} - \frac{2}{x+2}\right) \,dx &= \lim_{b \to \infty} \left[ 2\ln\left|\frac{x}{x+2}\right| \right]_{1}^{b} \\
        &= 2(\ln(1) - \ln(1/3)) = 2\ln 3
    \end{align*}
    \textbf{Result:} The integral \textbf{converges} to $2\ln 3$.

    \item \textbf{Integral: } $\displaystyle \int_{2}^{\infty} \frac{x-1}{x^3+x^2-2x} \,dx$
    \begin{align*}
        \frac{x-1}{x(x+2)(x-1)} = \frac{1}{x(x+2)} = \frac{1/2}{x} - \frac{1/2}{x+2} \\
        \frac{1}{2} \int_{2}^{\infty} \left(\frac{1}{x} - \frac{1}{x+2}\right) \,dx &= \frac{1}{2} \lim_{b \to \infty} \left[ \ln\left|\frac{x}{x+2}\right| \right]_{2}^{b} \\
        &= \frac{1}{2} (\ln(1) - \ln(2/4)) = \frac{1}{2}\ln 2
    \end{align*}
    \textbf{Result:} The integral \textbf{converges} to $\frac{1}{2}\ln 2$.
    
    \item \textbf{Integral: } $\displaystyle \int_{0}^{\infty} \frac{1}{(x+1)(x+2)(x+3)} \,dx$
    \begin{align*}
        \frac{1}{(x+1)(x+2)(x+3)} = \frac{1/2}{x+1} - \frac{1}{x+2} + \frac{1/2}{x+3} \\
        \int_{0}^{\infty} \dots \,dx &= \lim_{b \to \infty} \left[\frac{1}{2}\ln|x+1|-\ln|x+2|+\frac{1}{2}\ln|x+3|\right]_0^b \\
        &= \lim_{b \to \infty} \left[\frac{1}{2}\ln\left|\frac{(x+1)(x+3)}{(x+2)^2}\right|\right]_0^b \\
        &= \frac{1}{2} \left( \ln(1) - \ln\left(\frac{3}{4}\right) \right) = \frac{1}{2}\ln(4/3)
    \end{align*}
    \textbf{Result:} The integral \textbf{converges} to $\frac{1}{2}\ln(4/3)$.
    
    \item \textbf{Integral: } $\displaystyle \int_{3}^{\infty} \frac{2x}{x^2-1} \,dx$
    \begin{align*}
        \int_{3}^{\infty} \frac{2x}{x^2-1} \,dx &= \lim_{b \to \infty} [\ln|x^2-1|]_3^b = \lim_{b \to \infty} (\ln|b^2-1| - \ln 8) = \infty
    \end{align*}
    \textbf{Result:} The integral \textbf{diverges}.

    \item \textbf{Integral: } $\displaystyle \int_{1}^{\infty} \frac{x+1}{x^2+4x+3} \,dx$
    \begin{align*}
         \frac{x+1}{(x+1)(x+3)} = \frac{1}{x+3} \\
         \int_{1}^{\infty} \frac{1}{x+3} \,dx &= \lim_{b \to \infty} [\ln|x+3|]_{1}^{b} = \lim_{b \to \infty} (\ln|b+3| - \ln 4) = \infty
    \end{align*}
    \textbf{Result:} The integral \textbf{diverges}.

\end{enumerate}

\newpage
\problemsettitle{1.b. Improper Integrals: Discontinuity with U-Substitution}
\begin{enumerate}
    \item \textbf{Integral: } $\displaystyle \int_{0}^{2} \frac{x}{\sqrt{4-x^2}} \,dx$
    \textit{Discontinuity at $x=2$. Let $u=4-x^2$, $du=-2x\,dx$.}
    \begin{align*}
        \lim_{b \to 2^-} \int_{0}^{b} \frac{x}{\sqrt{4-x^2}} \,dx &= \lim_{b \to 2^-} [-\sqrt{4-x^2}]_0^b = 0 - (-\sqrt{4}) = 2
    \end{align*}
    \textbf{Result:} The integral \textbf{converges} to $2$.

    \item \textbf{Integral: } $\displaystyle \int_{1}^{2} \frac{1}{(x-1)^{1/3}} \,dx$
    \textit{Discontinuity at $x=1$.}
    \begin{align*}
        \lim_{a \to 1^+} \int_{a}^{2} (x-1)^{-1/3} \,dx &= \lim_{a \to 1^+} \left[\frac{3}{2}(x-1)^{2/3}\right]_a^2 = \frac{3}{2} - 0 = \frac{3}{2}
    \end{align*}
    \textbf{Result:} The integral \textbf{converges} to $3/2$.

    \item \textbf{Integral: } $\displaystyle \int_{1}^{e} \frac{1}{x\sqrt{\ln x}} \,dx$
    \textit{Discontinuity at $x=1$. Let $u=\ln x$, $du = (1/x)dx$.}
    \begin{align*}
        \lim_{a \to 1^+} \int_{a}^{e} \frac{1}{x\sqrt{\ln x}} \,dx &= \lim_{a \to 1^+} [2\sqrt{\ln x}]_a^e = 2\sqrt{1} - 0 = 2
    \end{align*}
    \textbf{Result:} The integral \textbf{converges} to $2$.

    \item \textbf{Integral: } $\displaystyle \int_{0}^{4} \frac{1}{\sqrt{x}} \,dx$
    \textit{Discontinuity at $x=0$.}
    \begin{align*}
        \lim_{a \to 0^+} \int_{a}^{4} x^{-1/2} \,dx &= \lim_{a \to 0^+} [2\sqrt{x}]_a^4 = 4 - 0 = 4
    \end{align*}
    \textbf{Result:} The integral \textbf{converges} to $4$.

    \item \textbf{Integral: } $\displaystyle \int_{0}^{\pi/2} \frac{\cos x}{\sqrt{\sin x}} \,dx$
    \textit{Discontinuity at $x=0$. Let $u=\sin x$, $du=\cos x \,dx$.}
    \begin{align*}
        \lim_{a \to 0^+} \int_{a}^{\pi/2} \frac{\cos x}{\sqrt{\sin x}} \,dx &= \lim_{a \to 0^+} [2\sqrt{\sin x}]_a^{\pi/2} = 2\sqrt{1} - 0 = 2
    \end{align*}
    \textbf{Result:} The integral \textbf{converges} to $2$.

    \item \textbf{Integral: } $\displaystyle \int_{-1}^{0} \frac{x^2}{(x^3+1)^{1/3}} \,dx$
    \textit{Discontinuity at $x=-1$. Let $u=x^3+1$, $du=3x^2 dx$.}
    \begin{align*}
        \lim_{a \to -1^+} \int_{a}^{0} \dots \,dx &= \lim_{a \to -1^+} \left[\frac{1}{2}(x^3+1)^{2/3}\right]_a^0 = \frac{1}{2} - 0 = \frac{1}{2}
    \end{align*}
    \textbf{Result:} The integral \textbf{converges} to $1/2$.

    \item \textbf{Integral: } $\displaystyle \int_{0}^{\ln 2} \frac{e^x}{e^x-1} \,dx$
    \textit{Discontinuity at $x=0$. Let $u=e^x-1$, $du=e^x dx$.}
    \begin{align*}
        \lim_{a \to 0^+} \int_{a}^{\ln 2} \frac{e^x}{e^x-1} \,dx &= \lim_{a \to 0^+} [\ln|e^x-1|]_a^{\ln 2} = \ln(1) - \lim_{a \to 0^+} \ln|e^a-1| = \infty
    \end{align*}
    \textbf{Result:} The integral \textbf{diverges}.

    \item \textbf{Integral: } $\displaystyle \int_{2}^{3} \frac{3x^2}{\sqrt{x^3-8}} \,dx$
    \textit{Discontinuity at $x=2$. Let $u=x^3-8$, $du=3x^2 dx$.}
    \begin{align*}
        \lim_{a \to 2^+} \int_{a}^{3} \dots \,dx &= \lim_{a \to 2^+} [2\sqrt{x^3-8}]_a^3 = 2\sqrt{19} - 0 = 2\sqrt{19}
    \end{align*}
    \textbf{Result:} The integral \textbf{converges} to $2\sqrt{19}$.

    \item \textbf{Integral: } $\displaystyle \int_{0}^{1} \frac{\arcsin x}{\sqrt{1-x^2}} \,dx$
    \textit{Discontinuity at $x=1$. Let $u=\arcsin x$, $du = dx/\sqrt{1-x^2}$.}
    \begin{align*}
        \lim_{b \to 1^-} \int_0^b \frac{\arcsin x}{\sqrt{1-x^2}} \,dx &= \lim_{b \to 1^-} [\frac{1}{2}(\arcsin x)^2]_0^b \\
        &= \frac{1}{2}(\arcsin 1)^2 - 0 = \frac{1}{2}(\pi/2)^2 = \frac{\pi^2}{8}
    \end{align*}
    \textbf{Result:} The integral \textbf{converges} to $\pi^2/8$.

    \item \textbf{Integral: } $\displaystyle \int_{0}^{1} \frac{1}{(1-x)^{2/3}} \,dx$
    \textit{Discontinuity at $x=1$. Let $u=1-x$, $du=-dx$.}
    \begin{align*}
        \lim_{b \to 1^-} \int_{0}^{b} (1-x)^{-2/3} \,dx &= \lim_{b \to 1^-} [-3(1-x)^{1/3}]_0^b = 0 - (-3) = 3
    \end{align*}
    \textbf{Result:} The integral \textbf{converges} to $3$.
\end{enumerate}

\newpage
\problemsettitle{1.c. Improper Integrals: Vertical Asymptote}
\begin{enumerate}
    \item \textbf{Integral: } $\displaystyle \int_{\pi/2}^{\pi} \cot\theta \,d\theta$
    \textit{Asymptote at $\theta=\pi$.}
    \begin{align*}
        \lim_{b \to \pi^-} \int_{\pi/2}^{b} \cot\theta \,d\theta &= \lim_{b \to \pi^-} [\ln|\sin\theta|]_{\pi/2}^b = (-\infty) - \ln(1) = -\infty
    \end{align*}
    \textbf{Result:} The integral \textbf{diverges}.

    \item \textbf{Integral: } $\displaystyle \int_{0}^{\pi/2} \sec\theta \,d\theta$
    \textit{Asymptote at $\theta=\pi/2$.}
    \begin{align*}
        \lim_{b \to (\pi/2)^-} \int_{0}^{b} \sec\theta \,d\theta &= \lim_{b \to (\pi/2)^-} [\ln|\sec\theta+\tan\theta|]_0^b = \infty - 0 = \infty
    \end{align*}
    \textbf{Result:} The integral \textbf{diverges}.

    \item \textbf{Integral: } $\displaystyle \int_{0}^{1} \ln x \,dx$
    \textit{Asymptote at $x=0$. Integration by parts.}
    \begin{align*}
        \lim_{a \to 0^+} \int_{a}^{1} \ln x \,dx &= \lim_{a \to 0^+} [x\ln x - x]_a^1 = (-1) - \lim_{a \to 0^+} (a\ln a - a) = -1
    \end{align*}
    \textbf{Result:} The integral \textbf{converges} to $-1$.

    \item \textbf{Integral: } $\displaystyle \int_{0}^{1} \frac{1}{x-1} \,dx$
    \textit{Asymptote at $x=1$.}
    \begin{align*}
        \lim_{b \to 1^-} \int_{0}^{b} \frac{1}{x-1} \,dx &= \lim_{b \to 1^-} [\ln|x-1|]_0^b = (-\infty) - 0 = -\infty
    \end{align*}
    \textbf{Result:} The integral \textbf{diverges}.

    \item \textbf{Integral: } $\displaystyle \int_{-\pi/2}^{0} \tan\theta \,d\theta$
    \textit{Asymptote at $\theta=-\pi/2$.}
    \begin{align*}
        \lim_{a \to (-\pi/2)^+} \int_{a}^{0} \tan\theta \,d\theta &= \lim_{a \to (-\pi/2)^+} [-\ln|\cos\theta|]_a^0 = 0 - (-\infty) = \infty
    \end{align*}
    \textbf{Result:} The integral \textbf{diverges}.
    
    \item \textbf{Integral: } $\displaystyle \int_{0}^{2} \frac{1}{(x-2)^2} \,dx$
    \textit{Asymptote at $x=2$.}
    \begin{align*}
        \lim_{b \to 2^-} \int_{0}^{b} (x-2)^{-2} \,dx &= \lim_{b \to 2^-} \left[-\frac{1}{x-2}\right]_0^b = \infty - \frac{1}{2} = \infty
    \end{align*}
    \textbf{Result:} The integral \textbf{diverges}.

    \item \textbf{Integral: } $\displaystyle \int_{1}^{2} \frac{1}{\sqrt[3]{x-1}} \,dx$
    \textit{Asymptote at $x=1$.}
    \begin{align*}
        \lim_{a \to 1^+} \int_{a}^{2} (x-1)^{-1/3} \,dx &= \lim_{a \to 1^+} \left[\frac{3}{2}(x-1)^{2/3}\right]_a^2 = \frac{3}{2}
    \end{align*}
    \textbf{Result:} The integral \textbf{converges} to $3/2$.

    \item \textbf{Integral: } $\displaystyle \int_{0}^{\pi} \csc\theta \,d\theta$
    \textit{Asymptotes at $\theta=0, \pi$. Split at $\pi/2$.}
    \begin{align*}
        \int_{0}^{\pi/2} \csc\theta \,d\theta = \lim_{a \to 0^+} [-\ln|\csc\theta+\cot\theta|]_a^{\pi/2} = \infty
    \end{align*}
    \textbf{Result:} The integral \textbf{diverges}.

    \item \textbf{Integral: } $\displaystyle \int_{0}^{1} \frac{1}{x} \,dx$
    \textit{Asymptote at $x=0$.}
    \begin{align*}
        \lim_{a \to 0^+} \int_{a}^{1} \frac{1}{x} \,dx = \lim_{a \to 0^+} [\ln|x|]_a^1 = 0 - (-\infty) = \infty
    \end{align*}
    \textbf{Result:} The integral \textbf{diverges}.

    \item \textbf{Integral: } $\displaystyle \int_{1}^{3} \frac{1}{x-3} \,dx$
    \textit{Asymptote at $x=3$.}
    \begin{align*}
        \lim_{b \to 3^-} \int_{1}^{b} \frac{1}{x-3} \,dx = \lim_{b \to 3^-} [\ln|x-3|]_1^b = -\infty
    \end{align*}
    \textbf{Result:} The integral \textbf{diverges}.
\end{enumerate}

\newpage
\problemsettitle{1.d. Improper Integrals: Discontinuity and Standard Forms}
\begin{enumerate}
    \item \textbf{Integral: } $\displaystyle \int_{-1}^{0} \frac{1}{\sqrt{1-x^2}} \,dx$
    \textit{Discontinuity at $x=-1$.}
    \begin{align*}
        \lim_{a \to -1^+} \int_{a}^{0} \frac{1}{\sqrt{1-x^2}} \,dx &= \lim_{a \to -1^+} [\arcsin(x)]_a^0 = 0 - (-\pi/2) = \frac{\pi}{2}
    \end{align*}
    \textbf{Result:} The integral \textbf{converges} to $\pi/2$.

    \item \textbf{Integral: } $\displaystyle \int_{0}^{2} \frac{1}{\sqrt{4-x^2}} \,dx$
    \textit{Discontinuity at $x=2$.}
    \begin{align*}
        \lim_{b \to 2^-} \int_{0}^{b} \frac{1}{\sqrt{4-x^2}} \,dx &= \lim_{b \to 2^-} [\arcsin(x/2)]_0^b = \pi/2 - 0 = \frac{\pi}{2}
    \end{align*}
    \textbf{Result:} The integral \textbf{converges} to $\pi/2$.

    \item \textbf{Integral: } $\displaystyle \int_{0}^{\infty} \frac{1}{1+x^2} \,dx$
    \textit{Infinite upper limit.}
    \begin{align*}
        \lim_{b \to \infty} \int_{0}^{b} \frac{1}{1+x^2} \,dx &= \lim_{b \to \infty} [\arctan(x)]_0^b = \pi/2 - 0 = \frac{\pi}{2}
    \end{align*}
    \textbf{Result:} The integral \textbf{converges} to $\pi/2$.

    \item \textbf{Integral: } $\displaystyle \int_{0}^{2} \frac{1}{x^2+4} \,dx$
    \textit{Standard integral, no discontinuity.}
    \begin{align*}
         [\frac{1}{2}\arctan(x/2)]_0^2 = \frac{1}{2}\arctan(1) - 0 = \frac{1}{2}(\pi/4) = \frac{\pi}{8}
    \end{align*}
    \textbf{Result:} This is a proper integral. The value is $\pi/8$.

    \item \textbf{Integral: } $\displaystyle \int_{0}^{3} \frac{1}{\sqrt{3-x}} \,dx$
    \textit{Discontinuity at $x=3$.}
    \begin{align*}
        \lim_{b \to 3^-} \int_{0}^{b} (3-x)^{-1/2} \,dx &= \lim_{b \to 3^-} [-2\sqrt{3-x}]_0^b = 0 - (-2\sqrt{3}) = 2\sqrt{3}
    \end{align*}
    \textbf{Result:} The integral \textbf{converges} to $2\sqrt{3}$.

    \item \textbf{Integral: } $\displaystyle \int_{-3}^{0} \frac{1}{\sqrt{9-x^2}} \,dx$
    \textit{Discontinuity at $x=-3$.}
    \begin{align*}
        \lim_{a \to -3^+} \int_{a}^{0} \frac{1}{\sqrt{9-x^2}} \,dx &= \lim_{a \to -3^+} [\arcsin(x/3)]_a^0 = 0 - (-\pi/2) = \frac{\pi}{2}
    \end{align*}
    \textbf{Result:} The integral \textbf{converges} to $\pi/2$.

    \item \textbf{Integral: } $\displaystyle \int_{-\infty}^{0} e^x \,dx$
    \textit{Infinite lower limit.}
    \begin{align*}
        \lim_{a \to -\infty} \int_{a}^{0} e^x \,dx = \lim_{a \to -\infty} [e^x]_a^0 = 1 - 0 = 1
    \end{align*}
    \textbf{Result:} The integral \textbf{converges} to $1$.

    \item \textbf{Integral: } $\displaystyle \int_{1}^{2} \frac{1}{(x-1)^{2/3}} \,dx$
    \textit{Discontinuity at $x=1$.}
    \begin{align*}
        \lim_{a \to 1^+} \int_{a}^{2} (x-1)^{-2/3} \,dx &= \lim_{a \to 1^+} [3(x-1)^{1/3}]_a^2 = 3 - 0 = 3
    \end{align*}
    \textbf{Result:} The integral \textbf{converges} to $3$.

    \item \textbf{Integral: } $\displaystyle \int_{2}^{4} \frac{1}{\sqrt{x-2}} \,dx$
    \textit{Discontinuity at $x=2$.}
    \begin{align*}
        \lim_{a \to 2^+} \int_{a}^{4} (x-2)^{-1/2} \,dx = \lim_{a \to 2^+} [2\sqrt{x-2}]_a^4 = 2\sqrt{2} - 0 = 2\sqrt{2}
    \end{align*}
    \textbf{Result:} The integral \textbf{converges} to $2\sqrt{2}$.

    \item \textbf{Integral: } $\displaystyle \int_{0}^{\infty} \frac{e^{-x}}{2} \,dx$
    \textit{Infinite upper limit.}
    \begin{align*}
        \lim_{b \to \infty} \int_{0}^{b} \frac{1}{2}e^{-x} \,dx = \lim_{b \to \infty} [-\frac{1}{2}e^{-x}]_0^b = 0 - (-\frac{1}{2}) = \frac{1}{2}
    \end{align*}
    \textbf{Result:} The integral \textbf{converges} to $1/2$.
\end{enumerate}

\newpage
\problemsettitle{1.e. Improper Integrals: Infinite Limits in Both Directions}
\begin{enumerate}
    \item \textbf{Integral: } $\displaystyle \int_{-\infty}^{\infty} x e^{-x^2} \,dx$
    \textit{Odd function over a symmetric interval.}
    \begin{align*}
        \int_{0}^{\infty} x e^{-x^2} \,dx = \lim_{b \to \infty} [-\frac{1}{2}e^{-x^2}]_0^b = 1/2. \quad \int_{-\infty}^{0} \dots = -1/2.
    \end{align*}
    \textbf{Result:} The integral \textbf{converges} to $1/2 - 1/2 = 0$.

    \item \textbf{Integral: } $\displaystyle \int_{-\infty}^{\infty} \frac{1}{x^2+9} \,dx$
    \begin{align*}
        \int_{-\infty}^{\infty} \frac{1}{x^2+9} \,dx &= [\frac{1}{3}\arctan(x/3)]_{-\infty}^{\infty} = \frac{1}{3}(\pi/2 - (-\pi/2)) = \frac{\pi}{3}
    \end{align*}
    \textbf{Result:} The integral \textbf{converges} to $\pi/3$.

    \item \textbf{Integral: } $\displaystyle \int_{-\infty}^{\infty} x^3 e^{-x^4} \,dx$
    \textit{Odd function. Converges to 0.}
    \begin{align*}
        \int_0^\infty x^3 e^{-x^4} dx = \lim_{b\to\infty} [-e^{-x^4}/4]_0^b = 1/4. \quad \int_{-\infty}^0 \dots = -1/4.
    \end{align*}
    \textbf{Result:} The integral \textbf{converges} to $0$.

    \item \textbf{Integral: } $\displaystyle \int_{-\infty}^{\infty} \frac{e^x}{1+e^{2x}} \,dx$
    \textit{Let $u=e^x$. Limits become $0$ to $\infty$.}
    \begin{align*}
        \int_{0}^{\infty} \frac{1}{1+u^2} \,du = [\arctan u]_0^\infty = \pi/2
    \end{align*}
    \textbf{Result:} The integral \textbf{converges} to $\pi/2$.

    \item \textbf{Integral: } $\displaystyle \int_{-\infty}^{\infty} \frac{1}{x^2+2x+2} \,dx$
    \textit{Complete the square: $(x+1)^2+1$.}
    \begin{align*}
        \int_{-\infty}^{\infty} \frac{1}{(x+1)^2+1} \,dx &= [\arctan(x+1)]_{-\infty}^\infty = \pi/2 - (-\pi/2) = \pi
    \end{align*}
    \textbf{Result:} The integral \textbf{converges} to $\pi$.

    \item \textbf{Integral: } $\displaystyle \int_{-\infty}^{\infty} \frac{1}{e^x+e^{-x}} \,dx$
    \textit{Multiply by $e^x/e^x$: $\int \frac{e^x}{e^{2x}+1} dx$. Same as \#4.}
    \textbf{Result:} The integral \textbf{converges} to $\pi/2$.

    \item \textbf{Integral: } $\displaystyle \int_{-\infty}^{\infty} \frac{x^2}{(x^3-1)^2} \,dx$
    \textit{Discontinuity at $x=1$. Let's check convergence near 1.}
    \begin{align*}
       \lim_{a \to 1^+} \int_{a}^{2} \frac{x^2}{(x^3-1)^2} \,dx = \lim_{a \to 1^+} [-\frac{1}{3(x^3-1)}]_a^2 = -\frac{1}{21} - (-\infty) = \infty
    \end{align*}
    \textbf{Result:} The integral \textbf{diverges}.

    \item \textbf{Integral: } $\displaystyle \int_{-\infty}^{\infty} \frac{x}{(x^2+1)^2} \,dx$
    \textit{Odd function. Let's check convergence.}
    \begin{align*}
       \int_0^\infty \frac{x}{(x^2+1)^2}dx = \lim_{b\to\infty} [-\frac{1}{2(x^2+1)}]_0^b = 0 - (-\frac{1}{2}) = \frac{1}{2}.
    \end{align*}
    \textbf{Result:} The integral \textbf{converges} to $1/2 - 1/2 = 0$.
    
    \item \textbf{Integral: } $\displaystyle \int_{-\infty}^{\infty} \frac{x}{x^4+1} \,dx$
    \textit{Odd function. Let's check convergence.}
    \begin{align*}
        \int_0^\infty \frac{x}{x^4+1}dx = \lim_{b\to\infty} [\frac{1}{2}\arctan(x^2)]_0^b = \frac{1}{2}(\pi/2 - 0) = \pi/4.
    \end{align*}
    \textbf{Result:} The integral \textbf{converges} to $\pi/4 - \pi/4 = 0$.
    
    \item \textbf{Integral: } $\displaystyle \int_{-\infty}^{\infty} x \,dx$
    \begin{align*}
        \int_0^\infty x \,dx = \lim_{b\to\infty} [x^2/2]_0^b = \infty.
    \end{align*}
    \textbf{Result:} The integral \textbf{diverges}.
\end{enumerate}

\newpage
\problemsettitle{2. Arc Length: Perfect Square Integrands}
\begin{enumerate}
    \item \textbf{Curve:} $y = \frac{x^3}{6} + \frac{1}{2x}$ from $x=1$ to $x=2$
    \begin{align*}
        y' = \frac{x^2}{2} - \frac{1}{2x^2} \implies 1+(y')^2 &= 1 + \left(\frac{x^4}{4} - \frac{1}{2} + \frac{1}{4x^4}\right) \\
        &= \frac{x^4}{4} + \frac{1}{2} + \frac{1}{4x^4} = \left(\frac{x^2}{2} + \frac{1}{2x^2}\right)^2 \\
        L = \int_1^2 \left(\frac{x^2}{2} + \frac{1}{2x^2}\right) dx &= \left[\frac{x^3}{6} - \frac{1}{2x}\right]_1^2 = (\frac{8}{6}-\frac{1}{4}) - (\frac{1}{6}-\frac{1}{2}) = \frac{17}{12}
    \end{align*}
    \textbf{Length:} $17/12$.

    \item \textbf{Curve:} $y = \frac{2}{3}(x-1)^{3/2}$ from $x=1$ to $x=5$
    \begin{align*}
        y' = (x-1)^{1/2} \implies 1+(y')^2 = 1 + (x-1) = x \\
        L = \int_1^5 \sqrt{x} \,dx = \left[\frac{2}{3}x^{3/2}\right]_1^5 = \frac{2}{3}(5\sqrt{5} - 1)
    \end{align*}
    \textbf{Length:} $\frac{2}{3}(5\sqrt{5} - 1)$.

    \item \textbf{Curve:} $y = \ln(\cos x)$ from $x=0$ to $x=\pi/4$
    \begin{align*}
        y' = \frac{-\sin x}{\cos x} = -\tan x \implies 1+(y')^2 = 1+\tan^2 x = \sec^2 x \\
        L = \int_0^{\pi/4} \sec x \,dx = [\ln|\sec x + \tan x|]_0^{\pi/4} = \ln(\sqrt{2}+1)
    \end{align*}
    \textbf{Length:} $\ln(\sqrt{2}+1)$.

    \item \textbf{Curve:} $y = \frac{x^2}{4} - \frac{\ln x}{2}$ from $x=1$ to $x=e$
    \begin{align*}
        y' = \frac{x}{2} - \frac{1}{2x} \implies 1+(y')^2 = 1 + (\frac{x^2}{4} - \frac{1}{2} + \frac{1}{4x^2}) = (\frac{x}{2} + \frac{1}{2x})^2 \\
        L = \int_1^e (\frac{x}{2} + \frac{1}{2x}) dx = \left[\frac{x^2}{4} + \frac{\ln x}{2}\right]_1^e = (\frac{e^2}{4}+\frac{1}{2}) - (\frac{1}{4}+0) = \frac{e^2+1}{4}
    \end{align*}
    \textbf{Length:} $(e^2+1)/4$.

    \item \textbf{Curve:} $y = \frac{2}{3}x^{3/2}$ from $x=0$ to $x=8$
    \begin{align*}
        y' = x^{1/2} \implies 1+(y')^2 = 1+x \\
        L = \int_0^8 \sqrt{1+x} \,dx = \left[\frac{2}{3}(1+x)^{3/2}\right]_0^8 = \frac{2}{3}(9^{3/2} - 1^{3/2}) = \frac{2}{3}(27-1) = \frac{52}{3}
    \end{align*}
    \textbf{Length:} $52/3$.

    \item \textbf{Curve:} $y = \cosh(x)$ from $x=0$ to $x=\ln 2$
    \begin{align*}
        y' = \sinh(x) \implies 1+(y')^2 = 1+\sinh^2 x = \cosh^2 x \\
        L = \int_0^{\ln 2} \cosh x \,dx = [\sinh x]_0^{\ln 2} = \sinh(\ln 2) - 0 = \frac{e^{\ln 2}-e^{-\ln 2}}{2} = \frac{2-1/2}{2} = \frac{3}{4}
    \end{align*}
    \textbf{Length:} $3/4$.

    \item \textbf{Curve:} $y = \frac{x^4}{4} + \frac{1}{8x^2}$ from $x=1$ to $x=2$
    \begin{align*}
        y' = x^3 - \frac{1}{4x^3} \implies 1+(y')^2 = 1+(x^6-\frac{1}{2}+\frac{1}{16x^6}) = (x^3+\frac{1}{4x^3})^2 \\
        L = \int_1^2 (x^3+\frac{1}{4x^3})dx = [\frac{x^4}{4}-\frac{1}{8x^2}]_1^2 = (4-\frac{1}{32}) - (\frac{1}{4}-\frac{1}{8}) = \frac{123}{32}
    \end{align*}
    \textbf{Length:} $123/32$.

    \item \textbf{Curve:} $y = \frac{1}{3}(x^2-2)^{3/2}$ from $x=\sqrt{2}$ to $x=3$
    \begin{align*}
        y' = x\sqrt{x^2-2} \implies 1+(y')^2 = 1+x^2(x^2-2) = x^4-2x^2+1=(x^2-1)^2 \\
        L = \int_{\sqrt{2}}^3 (x^2-1) dx = [\frac{x^3}{3}-x]_{\sqrt{2}}^3 = (9-3)-(\frac{2\sqrt{2}}{3}-\sqrt{2}) = 6+\frac{\sqrt{2}}{3}
    \end{align*}
    \textbf{Length:} $6 + \sqrt{2}/3$.

    \item \textbf{Curve:} $y = \ln(\csc x - \cot x)$ from $x=\pi/6$ to $x=\pi/2$
    \begin{align*}
        y' = \frac{-\csc x \cot x + \csc^2 x}{\csc x - \cot x} = \frac{\csc x(\csc x - \cot x)}{\csc x - \cot x} = \csc x \\
        1+(y')^2 = 1+\csc^2 x \quad \text{This does not simplify well. Typo in problem design.}
    \end{align*}
    \textbf{Corrected Curve:} $y = \ln(\sin x)$ from $x=\pi/4$ to $x=\pi/2$.
    \begin{align*}
        y' = \frac{\cos x}{\sin x} = \cot x \implies 1+(y')^2 = 1+\cot^2 x = \csc^2 x \\
        L = \int_{\pi/4}^{\pi/2} \csc x dx = [-\ln|\csc x+\cot x|]_{\pi/4}^{\pi/2} \\
        = (-\ln|1+0|) - (-\ln|\sqrt{2}+1|) = \ln(\sqrt{2}+1)
    \end{align*}
    \textbf{Length:} $\ln(\sqrt{2}+1)$.

    \item \textbf{Curve:} $y = \frac{x^5}{10} + \frac{1}{6x^3}$ from $x=1$ to $x=2$
    \begin{align*}
        y' = \frac{x^4}{2} - \frac{1}{2x^4} \implies 1+(y')^2 = 1+(\frac{x^8}{4}-\frac{1}{2}+\frac{1}{4x^8}) = (\frac{x^4}{2}+\frac{1}{2x^4})^2 \\
        L = \int_1^2 (\frac{x^4}{2}+\frac{1}{2x^4})dx = [\frac{x^5}{10}-\frac{1}{6x^3}]_1^2 = (\frac{32}{10}-\frac{1}{48}) - (\frac{1}{10}-\frac{1}{6}) = \frac{779}{240}
    \end{align*}
    \textbf{Length:} $779/240$.
\end{enumerate}

\newpage
\problemsettitle{3. Surface Area of Revolution}
\begin{enumerate}
    \item \textbf{Curve:} $y = \sqrt{9-x^2}$, $0 \le x \le 3$; about the x-axis.
    \begin{align*}
        y' = \frac{-x}{\sqrt{9-x^2}} \implies 1+(y')^2 = 1+\frac{x^2}{9-x^2} = \frac{9}{9-x^2} \\
        S = \int_0^3 2\pi\sqrt{9-x^2} \sqrt{\frac{9}{9-x^2}} dx = \int_0^3 2\pi(3) dx = [6\pi x]_0^3 = 18\pi
    \end{align*}
    \textbf{Surface Area:} $18\pi$. (Surface of a hemisphere).

    \item \textbf{Curve:} $y = x^3$ from $x=0$ to $x=1$; about the x-axis.
    \begin{align*}
        y' = 3x^2 \implies 1+(y')^2 = 1+9x^4 \\
        S = \int_0^1 2\pi x^3 \sqrt{1+9x^4} dx \quad (u=1+9x^4, du=36x^3 dx) \\
        = \frac{2\pi}{36}\int_1^{10} \sqrt{u} du = \frac{\pi}{18} [\frac{2}{3}u^{3/2}]_1^{10} = \frac{\pi}{27}(10\sqrt{10}-1)
    \end{align*}
    \textbf{Surface Area:} $\frac{\pi}{27}(10\sqrt{10}-1)$.

    \item \textbf{Curve:} $y=\sqrt{x}$, $1 \le x \le 4$; about the x-axis.
    \begin{align*}
        y' = \frac{1}{2\sqrt{x}} \implies 1+(y')^2 = 1+\frac{1}{4x} = \frac{4x+1}{4x} \\
        S = \int_1^4 2\pi \sqrt{x} \sqrt{\frac{4x+1}{4x}} dx = \int_1^4 \pi \sqrt{4x+1} dx \quad (u=4x+1) \\
        = \frac{\pi}{4}\int_5^{17} \sqrt{u} du = \frac{\pi}{4}[\frac{2}{3}u^{3/2}]_5^{17} = \frac{\pi}{6}(17\sqrt{17}-5\sqrt{5})
    \end{align*}
    \textbf{Surface Area:} $\frac{\pi}{6}(17\sqrt{17}-5\sqrt{5})$.

    \item \textbf{Curve:} $y = \cosh(x)$, $0 \le x \le 1$; about the x-axis.
    \begin{align*}
        y' = \sinh(x) \implies 1+(y')^2 = \cosh^2(x) \\
        S &= \int_0^1 2\pi \cosh(x)\sqrt{\cosh^2(x)} dx = 2\pi \int_0^1 \cosh^2(x) dx \\
        &= \pi \int_0^1 (1+\cosh(2x)) dx = \pi [x+\frac{1}{2}\sinh(2x)]_0^1 = \pi(1+\frac{1}{2}\sinh 2)
    \end{align*}
    \textbf{Surface Area:} $\pi(1+\frac{1}{2}\sinh 2)$.

    \item \textbf{Curve:} $y = e^{-x}$, $0 \le x \le \infty$; about the x-axis.
    \begin{align*}
        y' = -e^{-x} \implies 1+(y')^2 = 1+e^{-2x} \\
        S = \int_0^\infty 2\pi e^{-x}\sqrt{1+e^{-2x}}dx \quad (u=e^{-x}, du=-e^{-x}dx) \\
        = 2\pi\int_1^0 \sqrt{1+u^2}(-du) = 2\pi\int_0^1 \sqrt{1+u^2}du \quad (\text{Trig sub } u=\tan\theta) \\
        = 2\pi[\frac{u}{2}\sqrt{1+u^2}+\frac{1}{2}\ln|u+\sqrt{1+u^2}|]_0^1 = \pi(\sqrt{2}+\ln(1+\sqrt{2}))
    \end{align*}
    \textbf{Surface Area:} $\pi(\sqrt{2}+\ln(1+\sqrt{2}))$.

    \item \textbf{Curve:} $y = \frac{x^3}{6}+\frac{1}{2x}$, $1 \le x \le 2$; about the x-axis.
    \begin{align*}
        y' = \frac{x^2}{2}-\frac{1}{2x^2} \implies 1+(y')^2 = (\frac{x^2}{2}+\frac{1}{2x^2})^2 \\
        S &= \int_1^2 2\pi (\frac{x^3}{6}+\frac{1}{2x})(\frac{x^2}{2}+\frac{1}{2x^2}) dx \\
        &= 2\pi \int_1^2 (\frac{x^5}{12}+\frac{x}{12}+\frac{x}{4}+\frac{1}{4x^3}) dx = 2\pi[\frac{x^6}{72}+\frac{x^2}{6}-\frac{1}{8x^2}]_1^2 = \frac{47\pi}{16}
    \end{align*}
    \textbf{Surface Area:} $47\pi/16$.

    \item \textbf{Curve:} $x = \frac{1}{3}(y^2+2)^{3/2}$, $1 \le y \le 2$; about the y-axis.
    \begin{align*}
        \frac{dx}{dy} = y\sqrt{y^2+2} \implies 1+(dx/dy)^2 = (y^2+1)^2 \\
        S = \int_1^2 2\pi y \sqrt{(y^2+1)^2} dy = \int_1^2 2\pi y(y^2+1) dy \\
        = 2\pi [\frac{y^4}{4}+\frac{y^2}{2}]_1^2 = 2\pi[(4+2)-(\frac{1}{4}+\frac{1}{2})] = \frac{21\pi}{2}
    \end{align*}
    \textbf{Surface Area:} $21\pi/2$.

    \item \textbf{Curve:} $y=1-x^2$, $0 \le x \le 1$; about the y-axis.
    \begin{align*}
        S = \int_0^1 2\pi x \sqrt{1+(-2x)^2} dx = 2\pi \int_0^1 x\sqrt{1+4x^2}dx \quad (u=1+4x^2) \\
        = \frac{2\pi}{8}\int_1^5 \sqrt{u}du = \frac{\pi}{4}[\frac{2}{3}u^{3/2}]_1^5 = \frac{\pi}{6}(5\sqrt{5}-1)
    \end{align*}
    \textbf{Surface Area:} $\frac{\pi}{6}(5\sqrt{5}-1)$.

    \item \textbf{Curve:} $y=\sin x$, $0 \le x \le \pi$; about the x-axis.
    \begin{align*}
        S = \int_0^\pi 2\pi \sin x \sqrt{1+\cos^2 x} dx \quad (u=\cos x, du=-\sin x dx) \\
        = 2\pi\int_{-1}^1 \sqrt{1+u^2}du = 2\pi[\frac{u}{2}\sqrt{1+u^2}+\frac{1}{2}\ln|u+\sqrt{1+u^2}|]_{-1}^1 \\
        = 2\pi(\sqrt{2}+\ln(1+\sqrt{2}))
    \end{align*}
    \textbf{Surface Area:} $2\pi(\sqrt{2}+\ln(1+\sqrt{2}))$.

    \item \textbf{Curve:} $y = \frac{1}{4}x^4 + \frac{1}{8}x^{-2}$, $1 \le x \le 2$; about the y-axis.
    \begin{align*}
        y' = x^3 - \frac{1}{4}x^{-3} \implies 1+(y')^2 = (x^3+\frac{1}{4}x^{-3})^2 \\
        S = \int_1^2 2\pi x \sqrt{(x^3+\frac{1}{4}x^{-3})^2} dx = \int_1^2 2\pi x(x^3+\frac{1}{4}x^{-3}) dx \\
        = 2\pi \int_1^2 (x^4 + \frac{1}{4x^2})dx = 2\pi [\frac{x^5}{5}-\frac{1}{4x}]_1^2 = \frac{253\pi}{20}
    \end{align*}
    \textbf{Surface Area:} $253\pi/20$.
\end{enumerate}

\newpage
\problemsettitle{4. Parametric to Cartesian Equations}
\begin{enumerate}
    \item \textbf{Equations:} $x=3\cos t, y=5\sin t, 0 \le t \le 2\pi$
    \begin{align*}
        (\frac{x}{3})^2 + (\frac{y}{5})^2 = \cos^2 t + \sin^2 t = 1 \implies \frac{x^2}{9} + \frac{y^2}{25} = 1
    \end{align*}
    \textbf{Path:} Ellipse centered at (0,0), major vertical axis length 10, minor horizontal axis length 6.
    \textbf{Direction:} At $t=0, (3,0)$. At $t=\pi/2, (0,5)$. Counter-clockwise.

    \item \textbf{Equations:} $x=t^2, y=t^3-3t$
    \begin{align*}
        y = t(t^2-3) = \pm\sqrt{x}(x-3) \implies y^2 = x(x-3)^2
    \end{align*}
    \textbf{Path:} A self-intersecting cubic curve.
    \textbf{Direction:} For $t<0$, starts at top-left, moves to (3,0). For $t>0$, moves away from (3,0) to top-right.

    \item \textbf{Equations:} $x=3+2\sec t, y=1+4\tan t$
    \begin{align*}
        \sec t = \frac{x-3}{2}, \tan t = \frac{y-1}{4}. \quad \sec^2 t - \tan^2 t = 1 \implies (\frac{x-3}{2})^2 - (\frac{y-1}{4})^2 = 1
    \end{align*}
    \textbf{Path:} Hyperbola centered at (3,1).
    
    \item \textbf{Equations:} $x=e^t, y=e^{-2t}$
    \begin{align*}
        y = (e^t)^{-2} = x^{-2} = \frac{1}{x^2}. \text{ Since } x=e^t > 0, \text{ it's only the right branch.}
    \end{align*}
    \textbf{Path:} Part of the reciprocal square function in the first quadrant.
    
    \item \textbf{Equations:} $x=2\sin t - 1, y=3\cos t + 2, 0 \le t \le 2\pi$
    \begin{align*}
        (\frac{x+1}{2})^2 + (\frac{y-2}{3})^2 = 1
    \end{align*}
    \textbf{Path:} Ellipse centered at (-1, 2).
    \textbf{Direction:} At $t=0, (-1,5)$. At $t=\pi/2, (1,2)$. Clockwise.

    \item \textbf{Equations:} $x=\sqrt{t}, y=1-t$
    \begin{align*}
        t=x^2 \implies y=1-x^2. \text{ Since } x=\sqrt{t} \ge 0, \text{ it's the right half of the parabola.}
    \end{align*}
    \textbf{Path:} Parabola opening down, vertex at (0,1), for $x \ge 0$.
    
    \item \textbf{Equations:} $x=\sin t, y=\csc t, 0 < t < \pi$
    \begin{align*}
        y = \frac{1}{\sin t} = \frac{1}{x}
    \end{align*}
    \textbf{Path:} Reciprocal function for $x \in (0,1]$.
    
    \item \textbf{Equations:} $x=\cos(2t), y=\sin t$
    \begin{align*}
        x = 1-2\sin^2 t = 1-2y^2
    \end{align*}
    \textbf{Path:} Parabola opening left, vertex at (1,0).
    
    \item \textbf{Equations:} $x=4t^2-4, y=t, -\infty < t < \infty$
    \begin{align*}
        x = 4y^2 - 4
    \end{align*}
    \textbf{Path:} Parabola opening right, vertex at (-4,0).
    
    \item \textbf{Equations:} $x=t-1, y=t^2-2t+2$
    \begin{align*}
        t=x+1 \implies y = (x+1)^2 - 2(x+1) + 2 = x^2+2x+1-2x-2+2 = x^2+1
    \end{align*}
    \textbf{Path:} Parabola opening up, vertex at (0,1).
\end{enumerate}

\newpage
\problemsettitle{5. Parametric Derivatives}
\begin{enumerate}
    \item \textbf{Equations:} $x=t^3-3t, y=t^2-2$
    \begin{align*}
        \frac{dx}{dt} = 3t^2-3, \frac{dy}{dt}=2t \implies \frac{dy}{dx} = \frac{2t}{3(t^2-1)} \\
        \frac{d}{dt}(\frac{dy}{dx}) = \frac{2(3t^2-3)-2t(6t)}{9(t^2-1)^2} = \frac{-6t^2-6}{9(t^2-1)^2} = \frac{-2(t^2+1)}{3(t^2-1)^2} \\
        \frac{d^2y}{dx^2} = \frac{-2(t^2+1)}{3(t^2-1)^2} \cdot \frac{1}{3(t^2-1)} = \frac{-2(t^2+1)}{9(t^2-1)^3}
    \end{align*}

    \item \textbf{Equations:} $x=e^t, y=te^{-t}$
    \begin{align*}
        \frac{dx}{dt}=e^t, \frac{dy}{dt}=e^{-t}-te^{-t} \implies \frac{dy}{dx} = \frac{e^{-t}(1-t)}{e^t} = e^{-2t}(1-t) \\
        \frac{d}{dt}(\frac{dy}{dx}) = -2e^{-2t}(1-t) + e^{-2t}(-1) = e^{-2t}(2t-3) \\
        \frac{d^2y}{dx^2} = \frac{e^{-2t}(2t-3)}{e^t} = e^{-3t}(2t-3)
    \end{align*}

    \item \textbf{Equations:} $x=2\cos t, y=3\sin t$
    \begin{align*}
        \frac{dx}{dt}=-2\sin t, \frac{dy}{dt}=3\cos t \implies \frac{dy}{dx} = -\frac{3\cos t}{2\sin t} = -\frac{3}{2}\cot t \\
        \frac{d}{dt}(\frac{dy}{dx}) = \frac{3}{2}\csc^2 t \\
        \frac{d^2y}{dx^2} = \frac{(3/2)\csc^2 t}{-2\sin t} = -\frac{3}{4\sin^3 t} = -\frac{3}{4}\csc^3 t
    \end{align*}

    \item \textbf{Equations:} $x=t/(1+t), y=t^2$
    \begin{align*}
        \frac{dx}{dt} = \frac{1}{(1+t)^2}, \frac{dy}{dt}=2t \implies \frac{dy}{dx} = 2t(1+t)^2 \\
        \frac{d}{dt}(\frac{dy}{dx}) = 2(1+t)^2 + 2t(2(1+t)) = 2(1+t)(1+t+2t) = 2(1+t)(1+3t) \\
        \frac{d^2y}{dx^2} = 2(1+t)(1+3t) \cdot (1+t)^2 = 2(1+t)^3(1+3t)
    \end{align*}

    \item \textbf{Equations:} $x=\ln t, y=t^2+1$
    \begin{align*}
        \frac{dx}{dt}=1/t, \frac{dy}{dt}=2t \implies \frac{dy}{dx} = \frac{2t}{1/t} = 2t^2 \\
        \frac{d}{dt}(\frac{dy}{dx}) = 4t \implies \frac{d^2y}{dx^2} = \frac{4t}{1/t} = 4t^2
    \end{align*}

    \item \textbf{Equations:} $x=a(t-\sin t), y=a(1-\cos t)$
    \begin{align*}
        \frac{dx}{dt}=a(1-\cos t), \frac{dy}{dt}=a\sin t \implies \frac{dy}{dx} = \frac{\sin t}{1-\cos t} = \cot(t/2) \\
        \frac{d}{dt}(\frac{dy}{dx}) = -\frac{1}{2}\csc^2(t/2) \\
        \frac{d^2y}{dx^2} = \frac{-\frac{1}{2}\csc^2(t/2)}{a(1-\cos t)} = \frac{-\frac{1}{2}\csc^2(t/2)}{a(2\sin^2(t/2))} = -\frac{1}{4a}\csc^4(t/2)
    \end{align*}

    \item \textbf{Equations:} $x=t^2+1, y=t^3-1$
    \begin{align*}
        \frac{dx}{dt}=2t, \frac{dy}{dt}=3t^2 \implies \frac{dy}{dx} = \frac{3t}{2} \\
        \frac{d}{dt}(\frac{dy}{dx}) = \frac{3}{2} \implies \frac{d^2y}{dx^2} = \frac{3/2}{2t} = \frac{3}{4t}
    \end{align*}
    
    \item \textbf{Equations:} $x=\cos^3 t, y=\sin^3 t$
    \begin{align*}
        \frac{dx}{dt}=-3\cos^2 t \sin t, \frac{dy}{dt}=3\sin^2 t \cos t \implies \frac{dy}{dx} = -\tan t \\
        \frac{d}{dt}(\frac{dy}{dx}) = -\sec^2 t \implies \frac{d^2y}{dx^2} = \frac{-\sec^2 t}{-3\cos^2 t \sin t} = \frac{1}{3\cos^4 t \sin t}
    \end{align*}
    
    \item \textbf{Equations:} $x=\sqrt{t}, y=t$
    \begin{align*}
        \frac{dx}{dt}=\frac{1}{2\sqrt{t}}, \frac{dy}{dt}=1 \implies \frac{dy}{dx} = 2\sqrt{t} \\
        \frac{d}{dt}(\frac{dy}{dx}) = \frac{1}{\sqrt{t}} \implies \frac{d^2y}{dx^2} = \frac{1/\sqrt{t}}{1/(2\sqrt{t})} = 2
    \end{align*}
    (Also clear from $y=x^2 \implies y''=2$).
    
    \item \textbf{Equations:} $x=\arctan t, y=t^2$
    \begin{align*}
        \frac{dx}{dt}=\frac{1}{1+t^2}, \frac{dy}{dt}=2t \implies \frac{dy}{dx} = 2t(1+t^2) = 2t+2t^3 \\
        \frac{d}{dt}(\frac{dy}{dx}) = 2+6t^2 \implies \frac{d^2y}{dx^2} = \frac{2+6t^2}{1/(1+t^2)} = (2+6t^2)(1+t^2)
    \end{align*}

\end{enumerate}

\newpage
\problemsettitle{6. Tangent Lines to Parametric Curves}
\begin{enumerate}
    \item \textbf{Curve:} $x=t^2+1, y=t^3+t$ at $t=2$.
    \begin{itemize}
        \item \textbf{Point:} $x(2)=5, y(2)=10$. Point is $(5,10)$.
        \item \textbf{Slope:} $\frac{dx}{dt}=2t, \frac{dy}{dt}=3t^2+1 \implies \frac{dy}{dx}=\frac{3t^2+1}{2t}$. At $t=2, m=\frac{13}{4}$.
        \item \textbf{Equation:} $y-10=\frac{13}{4}(x-5)$.
    \end{itemize}

    \item \textbf{Curve:} $x=\cos t, y=\sin(2t)$ at $t=\pi/3$.
    \begin{itemize}
        \item \textbf{Point:} $x(\pi/3)=1/2, y(\pi/3)=\sqrt{3}/2$. Point is $(1/2, \sqrt{3}/2)$.
        \item \textbf{Slope:} $\frac{dx}{dt}=-\sin t, \frac{dy}{dt}=2\cos(2t) \implies \frac{dy}{dx}=\frac{2\cos(2t)}{-\sin t}$. At $t=\pi/3, m=\frac{2(-1/2)}{-\sqrt{3}/2}=\frac{2}{\sqrt{3}}$.
        \item \textbf{Equation:} $y-\frac{\sqrt{3}}{2}=\frac{2}{\sqrt{3}}(x-\frac{1}{2})$.
    \end{itemize}

    \item \textbf{Curve:} $x=e^t, y=e^{-t}$ at $t=1$.
    \begin{itemize}
        \item \textbf{Point:} $x(1)=e, y(1)=1/e$. Point is $(e, 1/e)$.
        \item \textbf{Slope:} $\frac{dy}{dx}=\frac{-e^{-t}}{e^t}=-e^{-2t}$. At $t=1, m=-e^{-2}$.
        \item \textbf{Equation:} $y-1/e = -e^{-2}(x-e)$.
    \end{itemize}
    
    \item \textbf{Curve:} $x=4\sin t, y=2\cos t$ at $t=\pi/4$.
    \begin{itemize}
        \item \textbf{Point:} $x=4(\sqrt{2}/2)=2\sqrt{2}, y=2(\sqrt{2}/2)=\sqrt{2}$. Point is $(2\sqrt{2}, \sqrt{2})$.
        \item \textbf{Slope:} $\frac{dy}{dx}=\frac{-2\sin t}{4\cos t}=-\frac{1}{2}\tan t$. At $t=\pi/4, m=-1/2$.
        \item \textbf{Equation:} $y-\sqrt{2}=-\frac{1}{2}(x-2\sqrt{2})$.
    \end{itemize}

    \item \textbf{Curve:} $x=t^3-1, y=t^2+t$ at $t=-1$.
    \begin{itemize}
        \item \textbf{Point:} $x(-1)=-2, y(-1)=0$. Point is $(-2,0)$.
        \item \textbf{Slope:} $\frac{dy}{dx}=\frac{2t+1}{3t^2}$. At $t=-1, m=\frac{-1}{3}$.
        \item \textbf{Equation:} $y-0 = -\frac{1}{3}(x+2)$.
    \end{itemize}

    \item \textbf{Curve:} $x=\sec t, y=\csc t$ at $t=\pi/3$.
    \begin{itemize}
        \item \textbf{Point:} $x=2, y=2/\sqrt{3}$. Point is $(2, 2/\sqrt{3})$.
        \item \textbf{Slope:} $\frac{dy}{dx}=\frac{-\csc t \cot t}{\sec t \tan t}=-\cot^3 t$. At $t=\pi/3, m=-(1/\sqrt{3})^3 = -1/(3\sqrt{3})$.
        \item \textbf{Equation:} $y-2/\sqrt{3} = -1/(3\sqrt{3})(x-2)$.
    \end{itemize}
    
    \item \textbf{Curve:} $x=1+\ln t, y=t^2+2$ at $t=1$.
    \begin{itemize}
        \item \textbf{Point:} $x=1, y=3$. Point is $(1,3)$.
        \item \textbf{Slope:} $\frac{dy}{dx}=\frac{2t}{1/t}=2t^2$. At $t=1, m=2$.
        \item \textbf{Equation:} $y-3=2(x-1)$.
    \end{itemize}
    
    \item \textbf{Curve:} $x=t\cos t, y=t\sin t$ at $t=\pi$.
    \begin{itemize}
        \item \textbf{Point:} $x=-\pi, y=0$. Point is $(-\pi,0)$.
        \item \textbf{Slope:} $\frac{dy}{dx}=\frac{\sin t+t\cos t}{\cos t-t\sin t}$. At $t=\pi, m=\frac{-\pi}{-1}=\pi$.
        \item \textbf{Equation:} $y-0=\pi(x+\pi)$.
    \end{itemize}

    \item \textbf{Curve:} $x=t-\sin t, y=1-\cos t$ at $t=\pi/2$.
    \begin{itemize}
        \item \textbf{Point:} $x=\pi/2-1, y=1$. Point is $(\pi/2-1, 1)$.
        \item \textbf{Slope:} $\frac{dy}{dx}=\frac{\sin t}{1-\cos t}$. At $t=\pi/2, m=1/1=1$.
        \item \textbf{Equation:} $y-1 = 1(x-(\pi/2-1))$.
    \end{itemize}
    
    \item \textbf{Curve:} $x=t^2, y=t^3-3t$ at $t=\sqrt{3}$.
    \begin{itemize}
        \item \textbf{Point:} $x=3, y=0$. Point is $(3,0)$.
        \item \textbf{Slope:} $\frac{dy}{dx}=\frac{3t^2-3}{2t}$. At $t=\sqrt{3}, m=\frac{9-3}{2\sqrt{3}}=\frac{6}{2\sqrt{3}}=\sqrt{3}$.
        \item \textbf{Equation:} $y-0=\sqrt{3}(x-3)$.
    \end{itemize}
\end{enumerate}

\newpage
\problemsettitle{7. Parametric Arc Length}
\begin{enumerate}
    \item \textbf{Curve:} $x=e^t \cos t, y=e^t \sin t, 0 \le t \le \pi$.
    \begin{align*}
        (dx/dt)^2+(dy/dt)^2 &= (e^t(\cos t-\sin t))^2 + (e^t(\sin t+\cos t))^2 \\
        &= e^{2t}(\cos^2 t-2\cos t\sin t+\sin^2 t + \sin^2 t+2\sin t\cos t+\cos^2 t) \\
        &= e^{2t}(2) \implies \sqrt{\dots} = \sqrt{2}e^t \\
        L = \int_0^\pi \sqrt{2}e^t dt = [\sqrt{2}e^t]_0^\pi = \sqrt{2}(e^\pi - 1)
    \end{align*}
    \textbf{Length:} $\sqrt{2}(e^\pi - 1)$.
    
    \item \textbf{Curve:} $x=\cos^3 t, y=\sin^3 t, 0 \le t \le \pi/2$.
    \begin{align*}
        (dx/dt)^2+(dy/dt)^2 &= (-3\cos^2t\sin t)^2+(3\sin^2t\cos t)^2 = 9\cos^4t\sin^2t+9\sin^4t\cos^2t \\
        &= 9\cos^2t\sin^2t(\cos^2t+\sin^2t) = (3\cos t\sin t)^2 \\
        L = \int_0^{\pi/2} 3\cos t\sin t dt = [\frac{3}{2}\sin^2 t]_0^{\pi/2} = \frac{3}{2}
    \end{align*}
    \textbf{Length:} $3/2$.

    \item \textbf{Curve:} $x = \frac{1}{3}t^3, y = \frac{1}{2}t^2, 0 \le t \le 1$.
    \begin{align*}
        (dx/dt)^2+(dy/dt)^2 = (t^2)^2+(t)^2 = t^4+t^2=t^2(t^2+1) \\
        L = \int_0^1 t\sqrt{t^2+1} dt \quad (u=t^2+1, du=2tdt) \\
        = \frac{1}{2}\int_1^2 \sqrt{u}du = \frac{1}{2}[\frac{2}{3}u^{3/2}]_1^2 = \frac{1}{3}(2\sqrt{2}-1)
    \end{align*}
    \textbf{Length:} $\frac{1}{3}(2\sqrt{2}-1)$.

    \item \textbf{Curve:} $x=a(\cos t+t\sin t), y=a(\sin t-t\cos t), 0 \le t \le \pi$.
    \begin{align*}
        dx/dt = a(at\cos t), dy/dt = a(at\sin t) \\
        (dx/dt)^2+(dy/dt)^2 = (at\cos t)^2+(at\sin t)^2 = a^2t^2 \\
        L = \int_0^\pi at \,dt = [a t^2/2]_0^\pi = \frac{a\pi^2}{2}
    \end{align*}
    \textbf{Length:} $a\pi^2/2$.

    \item \textbf{Curve:} $x = 2t, y = \frac{2}{3}t^{3/2}, 0 \le t \le 3$.
    \begin{align*}
        (dx/dt)^2+(dy/dt)^2 = (2)^2+(t^{1/2})^2 = 4+t \\
        L = \int_0^3 \sqrt{4+t} dt = [\frac{2}{3}(4+t)^{3/2}]_0^3 = \frac{2}{3}(7\sqrt{7}-8)
    \end{align*}
    \textbf{Length:} $\frac{2}{3}(7\sqrt{7}-8)$.

    \item \textbf{Curve:} $x=t^2, y=\frac{2}{3}(2t+1)^{3/2}, 0 \le t \le 4$. (Note: This is similar to 7a from review)
    \begin{align*}
        dx/dt = 2t, dy/dt = 2\sqrt{2t+1} \implies (dx/dt)^2+(dy/dt)^2 = 4t^2+4(2t+1) = 4t^2+8t+4 = (2t+2)^2 \\
        L = \int_0^4 (2t+2) dt = [t^2+2t]_0^4 = 16+8=24
    \end{align*}
    \textbf{Length:} 24.

    \item \textbf{Curve:} $x=1+3t^2, y=4+2t^3, 0 \le t \le 1$.
    \begin{align*}
        (dx/dt)^2+(dy/dt)^2 = (6t)^2+(6t^2)^2 = 36t^2(1+t^2) \\
        L = \int_0^1 6t\sqrt{1+t^2} dt \quad (u=1+t^2, du=2tdt) \\
        = 3\int_1^2 \sqrt{u}du = 3[\frac{2}{3}u^{3/2}]_1^2 = 2(2\sqrt{2}-1)
    \end{align*}
    \textbf{Length:} $2(2\sqrt{2}-1)$.
    
    \item \textbf{Curve:} $x=t-\sin t, y=1-\cos t, 0 \le t \le 2\pi$.
    \begin{align*}
        (dx/dt)^2+(dy/dt)^2 = (1-\cos t)^2+\sin^2 t = 1-2\cos t+\cos^2 t+\sin^2 t = 2-2\cos t \\
        = 4\sin^2(t/2) \implies \sqrt{\dots} = 2\sin(t/2) \text{ on } [0, 2\pi] \\
        L = \int_0^{2\pi} 2\sin(t/2) dt = [-4\cos(t/2)]_0^{2\pi} = -4(-1) - (-4(1)) = 8
    \end{align*}
    \textbf{Length:} 8.

    \item \textbf{Curve:} $x=3t, y=t^3, 0 \le t \le 2$.
    \begin{align*}
        (dx/dt)^2+(dy/dt)^2 = (3)^2+(3t^2)^2 = 9(1+t^4) \\
        L = \int_0^2 3\sqrt{1+t^4} dt \quad \text{ (Cannot be solved with elementary functions)}
    \end{align*}
    \textbf{Result:} Problem is not well-posed for this level.
    
    \item \textbf{Curve:} $x=e^t-t, y=4e^{t/2}, 0 \le t \le 1$.
    \begin{align*}
        dx/dt = e^t-1, dy/dt = 2e^{t/2} \\
        (dx/dt)^2+(dy/dt)^2 = (e^{2t}-2e^t+1) + 4e^t = e^{2t}+2e^t+1 = (e^t+1)^2 \\
        L = \int_0^1 (e^t+1) dt = [e^t+t]_0^1 = (e+1)-(1+0)=e
    \end{align*}
    \textbf{Length:} $e$.
\end{enumerate}

\newpage
\problemsettitle{8. Vertical Tangents}
\begin{enumerate}
    \item \textbf{Curve:} $x=t^3-3t, y=t^2-3$.
    \begin{itemize}
        \item $\frac{dx}{dt} = 3t^2-3=3(t-1)(t+1)$. $dx/dt=0$ at $t=1, t=-1$.
        \item $\frac{dy}{dt} = 2t$.
        \item At $t=1, dy/dt=2 \ne 0$. At $t=-1, dy/dt=-2 \ne 0$.
    \end{itemize}
    \textbf{Result:} Vertical tangents at $t=1$ and $t=-1$.

    \item \textbf{Curve:} $x=t^2-t, y=t^3-3t$.
    \begin{itemize}
        \item $\frac{dx}{dt} = 2t-1$. $dx/dt=0$ at $t=1/2$.
        \item $\frac{dy}{dt} = 3t^2-3$. At $t=1/2, dy/dt=3/4-3 \ne 0$.
    \end{itemize}
    \textbf{Result:} Vertical tangent at $t=1/2$.

    \item \textbf{Curve:} $x=2\cos t, y=\sin(2t)$.
    \begin{itemize}
        \item $\frac{dx}{dt}=-2\sin t$. $dx/dt=0$ at $t=k\pi$ for integer $k$.
        \item $\frac{dy}{dt}=2\cos(2t)$. At $t=k\pi, \cos(2k\pi)=1, dy/dt=2 \ne 0$.
    \end{itemize}
    \textbf{Result:} Vertical tangents at $t=k\pi$.

    \item \textbf{Curve:} $x=t-\sin t, y=1-\cos t$.
    \begin{itemize}
        \item $\frac{dx}{dt}=1-\cos t$. $dx/dt=0$ at $t=2k\pi$.
        \item $\frac{dy}{dt}=\sin t$. At $t=2k\pi, dy/dt=0$. The slope is indeterminate ($0/0$).
    \end{itemize}
    \textbf{Result:} No vertical tangents (cusps at these points).

    \item \textbf{Curve:} $x=t^4-2t^2, y=t^3-t$.
    \begin{itemize}
        \item $\frac{dx}{dt}=4t^3-4t=4t(t-1)(t+1)$. $dx/dt=0$ at $t=0,1,-1$.
        \item $\frac{dy}{dt}=3t^2-1$. At $t=0, dy/dt=-1 \ne 0$. At $t=\pm 1, dy/dt=2 \ne 0$.
    \end{itemize}
    \textbf{Result:} Vertical tangents at $t=0, 1, -1$.

    \item \textbf{Curve:} $x=t e^t, y=t^2-t$.
    \begin{itemize}
        \item $\frac{dx}{dt}=e^t+te^t=e^t(1+t)$. $dx/dt=0$ at $t=-1$.
        \item $\frac{dy}{dt}=2t-1$. At $t=-1, dy/dt=-3 \ne 0$.
    \end{itemize}
    \textbf{Result:} Vertical tangent at $t=-1$.

    \item \textbf{Curve:} $x=\sin t, y=\cos t$.
    \begin{itemize}
        \item $\frac{dx}{dt}=\cos t$. $dx/dt=0$ at $t=\pi/2+k\pi$.
        \item $\frac{dy}{dt}=-\sin t$. At these $t$, $|\sin t|=1, dy/dt \ne 0$.
    \end{itemize}
    \textbf{Result:} Vertical tangents at $t=\pi/2+k\pi$.

    \item \textbf{Curve:} $x=t^2, y=t^3-3t$.
    \begin{itemize}
        \item $\frac{dx}{dt}=2t$. $dx/dt=0$ at $t=0$.
        \item $\frac{dy}{dt}=3t^2-3$. At $t=0, dy/dt=-3 \ne 0$.
    \end{itemize}
    \textbf{Result:} Vertical tangent at $t=0$.

    \item \textbf{Curve:} $x=\ln(t^2+1), y=t-2$.
    \begin{itemize}
        \item $\frac{dx}{dt}=\frac{2t}{t^2+1}$. $dx/dt=0$ at $t=0$.
        \item $\frac{dy}{dt}=1 \ne 0$ for all $t$.
    \end{itemize}
    \textbf{Result:} Vertical tangent at $t=0$.

    \item \textbf{Curve:} $x=t^3-12t, y=t^2-1$.
    \begin{itemize}
        \item $\frac{dx}{dt}=3t^2-12=3(t-2)(t+2)$. $dx/dt=0$ at $t=\pm 2$.
        \item $\frac{dy}{dt}=2t$. At $t=\pm 2, dy/dt=\pm 4 \ne 0$.
    \end{itemize}
    \textbf{Result:} Vertical tangents at $t=2$ and $t=-2$.
\end{enumerate}

\newpage
\problemsettitle{9. Particle at Rest}
\begin{enumerate}
    \item \textbf{Curve:} $x=t^3-3t^2, y=t^3-3t$.
    \begin{itemize}
        \item $\frac{dx}{dt} = 3t^2-6t=3t(t-2)$. Roots: $t=0, 2$.
        \item $\frac{dy}{dt} = 3t^2-3=3(t-1)(t+1)$. Roots: $t=1, -1$.
    \end{itemize}
    \textbf{Result:} No common roots. Particle is never at rest.

    \item \textbf{Curve:} $x=\cos t, y=\sin(2t)$.
    \begin{itemize}
        \item $\frac{dx}{dt}=-\sin t$. Roots: $t=k\pi$.
        \item $\frac{dy}{dt}=2\cos(2t)$. At $t=k\pi, \cos(2k\pi)=1, dy/dt \ne 0$.
    \end{itemize}
    \textbf{Result:} Never at rest.

    \item \textbf{Curve:} $x=t^2-4t, y=t^3-12t$.
    \begin{itemize}
        \item $\frac{dx}{dt}=2t-4$. Root: $t=2$.
        \item $\frac{dy}{dt}=3t^2-12=3(t-2)(t+2)$. Roots: $t=2, -2$.
    \end{itemize}
    \textbf{Result:} Common root is $t=2$. Particle is at rest at $t=2$.

    \item \textbf{Curve:} $x=\sin t, y=\sin t$.
    \begin{itemize}
        \item $\frac{dx}{dt}=\cos t, \frac{dy}{dt}=\cos t$. Both are zero at $t=\pi/2+k\pi$.
    \end{itemize}
    \textbf{Result:} At rest at $t=\pi/2+k\pi$.

    \item \textbf{Curve:} $x=t^4-2t^2, y=t^3-3t^2$.
    \begin{itemize}
        \item $\frac{dx}{dt}=4t^3-4t=4t(t-1)(t+1)$. Roots: $t=0,1,-1$.
        \item $\frac{dy}{dt}=3t^2-6t=3t(t-2)$. Roots: $t=0,2$.
    \end{itemize}
    \textbf{Result:} Common root is $t=0$. At rest at $t=0$.

    \item \textbf{Curve:} $x=t^2-1, y=t^3-t$.
    \begin{itemize}
        \item $\frac{dx}{dt}=2t$. Root: $t=0$.
        \item $\frac{dy}{dt}=3t^2-1$. At $t=0, dy/dt=-1 \ne 0$.
    \end{itemize}
    \textbf{Result:} Never at rest.

    \item \textbf{Curve:} $x=\sin t - t, y=\cos t-1$.
    \begin{itemize}
        \item $\frac{dx}{dt}=\cos t - 1$. Roots: $t=2k\pi$.
        \item $\frac{dy}{dt}=-\sin t$. Roots: $t=k\pi$.
    \end{itemize}
    \textbf{Result:} Common roots are $t=2k\pi$. At rest at $t=2k\pi$.

    \item \textbf{Curve:} $x=t^3/3-t, y=t^2-1$.
    \begin{itemize}
        \item $\frac{dx}{dt}=t^2-1=(t-1)(t+1)$. Roots: $t=1, -1$.
        \item $\frac{dy}{dt}=2t$. Root: $t=0$.
    \end{itemize}
    \textbf{Result:} Never at rest.

    \item \textbf{Curve:} $x=t^3-3t, y=t^3-12t$.
    \begin{itemize}
        \item $\frac{dx}{dt}=3t^2-3=3(t-1)(t+1)$. Roots: $t=1, -1$.
        \item $\frac{dy}{dt}=3t^2-12=3(t-2)(t+2)$. Roots: $t=2, -2$.
    \end{itemize}
    \textbf{Result:} Never at rest.

    \item \textbf{Curve:} $x=t^2-2t, y=t^3-3t^2+2t$.
    \begin{itemize}
        \item $\frac{dx}{dt}=2t-2=2(t-1)$. Root: $t=1$.
        \item $\frac{dy}{dt}=3t^2-6t+2$. At $t=1, dy/dt = 3-6+2 = -1 \ne 0$.
    \end{itemize}
    \textbf{Result:} Never at rest.
\end{enumerate}

\newpage
\problemsettitle{Concept Check List}

This list summarizes the concepts tested. The numbers refer to the problem sets generated in this document.

\begin{itemize}
    \item \textbf{Improper Integrals}
    \begin{itemize}
        \item Type 1 (Infinite Limit) requiring partial fractions: \textbf{1.a} (1-10)
        \item Type 2 (Discontinuity) requiring u-substitution: \textbf{1.b} (1-10)
        \item Type 2 (Discontinuity) from a vertical asymptote: \textbf{1.c} (1-10)
        \item Type 2 (Discontinuity) solved with standard forms (arcsin, etc.): \textbf{1.d} (1-10)
        \item Type 1 (Double Infinite Limits), possibly using symmetry: \textbf{1.e} (1-10)
    \end{itemize}
    \item \textbf{Arc Length (Cartesian)}
    \begin{itemize}
        \item Integrand $\sqrt{1+(y')^2}$ simplifies to a perfect square: \textbf{2} (1-10)
    \end{itemize}
    \item \textbf{Surface Area of Revolution (Cartesian)}
    \begin{itemize}
        \item Calculating surface area for various curves, requiring algebraic simplification and u-substitution: \textbf{3} (1-10)
    \end{itemize}
    \item \textbf{Parametric Equations}
    \begin{itemize}
        \item Eliminating the parameter to find the Cartesian equation (circles, ellipses, parabolas, hyperbolas, etc.): \textbf{4} (1-10)
    \end{itemize}
    \item \textbf{Calculus with Parametric curves}
    \begin{itemize}
        \item Finding first and second derivatives ($\frac{dy}{dx}$, $\frac{d^2y}{dx^2}$): \textbf{5} (1-10)
        \item Finding the equation of a tangent line at a point: \textbf{6} (1-10)
        \item Calculating arc length, often involving perfect squares or u-substitution: \textbf{7} (1-10)
        \item Finding points of vertical tangents ($\frac{dx}{dt}=0, \frac{dy}{dt} \ne 0$): \textbf{8} (1-10)
        \item Finding when a particle is at rest ($\frac{dx}{dt}=0$ and $\frac{dy}{dt}=0$): \textbf{9} (1-10)
    \end{itemize}
\end{itemize}


\end{document}