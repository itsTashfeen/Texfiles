\documentclass[12pt]{article}
\usepackage{amsmath}
\usepackage{amsfonts}
\usepackage{amssymb}
\usepackage{graphicx}
\usepackage{geometry}
\usepackage{hyperref}

\geometry{a4paper, margin=1in}

\begin{document}

\title{8.2: Area of a Surface of Revolution - Problem Set}
\author{Tashfeen Omran}
\date{October 2025}
\maketitle

\section*{Instructions}
For each problem, find the area of the surface generated by revolving the given curve about the specified axis. For "Setup Only" problems, write down the definite integral for the surface area but do not evaluate it. For all other problems, find the exact area.

\section*{Problems}

\subsection*{Part 1: Basic Integrals and U-Substitution}

\begin{enumerate}
    \item Find the exact area of the surface obtained by rotating the curve $y = x^3$ for $0 \le x \le 2$ about the x-axis.

    \item Find the exact area of the surface obtained by rotating the curve $y = \sqrt{5x-1}$ for $1 \le x \le 2$ about the x-axis.

    \item Find the exact area of the surface obtained by rotating the curve $x = \frac{1}{3}(y^2 + 2)^{3/2}$ for $1 \le y \le 2$ about the y-axis.

    \item The curve $y = \sqrt[3]{x}$ from $(1, 1)$ to $(8, 2)$ is rotated about the y-axis. Find the surface area. (Hint: It is easier to integrate with respect to y).

    \item Find the exact area of the surface generated by revolving the curve $y = \cos(x)$ for $0 \le x \le \frac{\pi}{2}$ about the x-axis.

    \item A section of a sphere is formed by rotating the curve $y = \sqrt{9-x^2}$ for $0 \le x \le 2$ about the x-axis. Find its surface area.

    \item Find the exact area of the surface generated by rotating the curve $x=e^{2y}$ for $0 \le y \le \ln(3)$ about the y-axis.
\end{enumerate}

\subsection*{Part 2: Radical Cancellation Problems}

\begin{enumerate}
    \setcounter{enumi}{7}
    \item Find the exact area of the surface obtained by rotating the curve $y = \sqrt{12-x}$ for $3 \le x \le 8$ about the x-axis.

    \item Find the exact area of the surface obtained by rotating $y = \sqrt{25-x^2}$ for $0 \le x \le 3$ about the x-axis.

    \item Find the exact area of the surface generated by rotating the curve $y = \sqrt{2x+1}$ for $1 \le x \le 4$ about the x-axis.

    \item Find the exact area of the surface generated by rotating the curve $y=2\sqrt{x}$ from $x=3$ to $x=8$ about the x-axis.
    
    \item Find the exact area of the surface obtained by rotating the curve $y = \sqrt{x-1}$ for $2 \le x \le 5$ about the x-axis.
    
    \item Find the exact area of the surface generated by rotating the curve $y=2\sqrt{3-x}$ from $x=1$ to $x=2$ about the x-axis.
    
    \item Find the exact area of the surface generated by rotating the curve $x=\sqrt{4-y}$ from $y=0$ to $y=3$ about the y-axis.
    
    \item Find the exact area of the surface obtained by rotating the curve $y=\sqrt{x}$ for $1 \le x \le 6$ about the x-axis.
    
    \item Find the exact area of the surface obtained by rotating the curve $y=\frac{1}{2}\sqrt{x}$ for $1 \le x \le 3$ about the x-axis.

\end{enumerate}


\subsection*{Part 3: The "Perfect Square Trick"}

\begin{enumerate}
    \setcounter{enumi}{16}
    \item Find the exact area of the surface generated by rotating the curve $y = \frac{x^3}{6} + \frac{1}{2x}$ for $1 \le x \le 2$ about the x-axis.
    
    \item Find the exact area of the surface obtained by rotating the curve $y = \frac{x^2}{4} - \frac{1}{2}\ln(x)$ for $1 \le x \le e$ about the y-axis.
    
    \item Find the exact area of the surface generated by rotating the curve $x = \frac{y^4}{4} + \frac{1}{8y^2}$ for $1 \le y \le 2$ about the y-axis.

    \item Find the exact area of the surface generated by rotating the curve $y = \frac{x^5}{5} + \frac{1}{12x^3}$ for $1 \le x \le 2$ about the x-axis.
    
    \item Find the exact area of the surface generated by rotating the curve $y=\cosh(x) = \frac{e^x+e^{-x}}{2}$ for $0 \le x \le 1$ about the x-axis.
    
    \item Find the exact area of the surface generated by rotating the curve $y = \frac{x^4}{8} + \frac{1}{4x^2}$ for $1 \le x \le 2$ about the x-axis.
    
    \item Find the exact area of the surface obtained by rotating the curve $x = \frac{y^3}{3} + \frac{1}{4y}$ for $1 \le y \le 3$ about the y-axis.
    
    \item Find the exact area of the surface generated by rotating the curve $y=\frac{x^3}{3} + \frac{1}{4x}$ from $x=1$ to $x=2$ about the x-axis.
    
    \item Find the exact area of the surface obtained by rotating the curve $x = \frac{y^4}{2} + \frac{1}{16y^2}$ for $1 \le y \le 2$ about the y-axis.
\end{enumerate}

\subsection*{Part 4: Rotation About Arbitrary Lines}

\begin{enumerate}
    \setcounter{enumi}{25}
    \item Set up the integral for the surface area generated by rotating $y = x^2$ for $0 \le x \le 2$ about the line $y = -3$. (Setup Only)

    \item Set up the integral for the surface area generated by rotating $y = e^x$ for $0 \le x \le 1$ about the line $x = 2$. (Setup Only)

    \item Find the exact area of the surface generated by rotating the curve $y = x+1$ for $0 \le x \le 3$ about the line $y=1$.

    \item Find the exact area of the surface generated by rotating the line $x=2y+1$ for $0 \le y \le 2$ about the line $x=-1$.
\end{enumerate}

\subsection*{Part 5: Parametric Curves}

\begin{enumerate}
    \setcounter{enumi}{29}
    \item Find the surface area of a sphere of radius $R$ by rotating the semicircle $x = R\cos(t)$, $y = R\sin(t)$ for $0 \le t \le \pi$ about the x-axis.

    \item Find the area of the surface obtained by rotating the curve $x=t^3$, $y=t^2$ for $0 \le t \le 1$ about the x-axis.

    \item Find the area of the surface obtained by rotating the astroid $x = \cos^3(t)$, $y = \sin^3(t)$ for $0 \le t \le \frac{\pi}{2}$ about the x-axis.

    \item Set up the integral for the area of the surface generated by rotating the cycloid arc $x = t - \sin(t)$, $y = 1 - \cos(t)$ for $0 \le t \le 2\pi$ about the y-axis. (Setup Only)
\end{enumerate}

\subsection*{Part 6: Improper Integrals}

\begin{enumerate}
    \setcounter{enumi}{33}
    \item The curve $y = e^{-x}$ for $x \ge 0$ is rotated about the x-axis. Find the surface area, if it is finite.

    \item Consider the curve $y = \frac{1}{x^2}$ for $x \ge 1$. Is the surface area generated by rotating this curve about the x-axis finite or infinite? Use a comparison test.
\end{enumerate}

\subsection*{Part 7: Mixed Concepts and "Setup Only"}

\begin{enumerate}
    \setcounter{enumi}{35}
    \item Set up the integral for the surface area obtained by rotating the curve $y = \tan(x)$ for $0 \le x \le \frac{\pi}{4}$ about the x-axis. (Setup Only)

    \item Set up the integral for the surface area obtained by rotating the curve $y = \ln(\cos(x))$ for $0 \le x \le \frac{\pi}{3}$ about the y-axis. (Setup Only)

    \item Find the exact surface area generated by rotating the curve $y=\frac{2}{3}x^{3/2}$ for $0 \le x \le 3$ about the y-axis.

    \item Find the exact surface area generated by rotating the curve $9x = y^2+18$ for $2 \le x \le 6$ about the x-axis.

    \item A decorative light bulb is shaped by rotating the graph of $y = \frac{1}{3}x^{1/2} - x^{3/2}$ for $0 \le x \le \frac{1}{3}$ about the y-axis. Set up the integral for the surface area. (Setup Only)
    
    \item The circle $(x-2)^2 + y^2 = 1$ is rotated about the y-axis to form a torus. Set up the integral(s) for its surface area. (Hint: Solve for $x$ and consider the two resulting functions). (Setup Only)
    
    \item Find the surface area of the torus generated by rotating the circle $(x-R)^2 + y^2 = r^2$ (where $R>r$) about the y-axis. (Hint: Use the parametric representation $x=R+r\cos(t)$, $y=r\sin(t)$ for $0 \le t \le 2\pi$).

\end{enumerate}

\newpage
\section*{Solutions}

\subsection*{Solution to Problem 1}
$y=x^3, \frac{dy}{dx}=3x^2$. $S = \int_0^2 2\pi x^3 \sqrt{1+(3x^2)^2} dx = \int_0^2 2\pi x^3 \sqrt{1+9x^4} dx$.
Let $u=1+9x^4$, $du=36x^3 dx \Rightarrow x^3 dx = \frac{du}{36}$.
Bounds: $x=0 \Rightarrow u=1$, $x=2 \Rightarrow u=1+9(16)=145$.
$S = \int_1^{145} 2\pi \sqrt{u} \frac{du}{36} = \frac{\pi}{18} \int_1^{145} u^{1/2} du = \frac{\pi}{18} \left[\frac{2}{3}u^{3/2}\right]_1^{145} = \frac{\pi}{27}(145\sqrt{145}-1)$.

\subsection*{Solution to Problem 2}
$y=\sqrt{5x-1}, \frac{dy}{dx}=\frac{5}{2\sqrt{5x-1}}$. $1+(\frac{dy}{dx})^2 = 1+\frac{25}{4(5x-1)} = \frac{20x-4+25}{4(5x-1)} = \frac{20x+21}{4(5x-1)}$.
$S = \int_1^2 2\pi \sqrt{5x-1} \sqrt{\frac{20x+21}{4(5x-1)}} dx = \int_1^2 2\pi \sqrt{5x-1} \frac{\sqrt{20x+21}}{2\sqrt{5x-1}} dx = \pi \int_1^2 \sqrt{20x+21} dx$.
Let $u=20x+21, du=20dx$. $S = \pi \int_{41}^{61} \sqrt{u} \frac{du}{20} = \frac{\pi}{20} \left[\frac{2}{3}u^{3/2}\right]_{41}^{61} = \frac{\pi}{30}(61\sqrt{61}-41\sqrt{41})$.

\subsection*{Solution to Problem 3}
$x=\frac{1}{3}(y^2+2)^{3/2}, \frac{dx}{dy}=\frac{1}{3} \cdot \frac{3}{2}(y^2+2)^{1/2} \cdot 2y = y\sqrt{y^2+2}$.
$1+(\frac{dx}{dy})^2 = 1+y^2(y^2+2) = 1+y^4+2y^2 = (y^2+1)^2$.
$S = \int_1^2 2\pi y \sqrt{(y^2+1)^2} dy = \int_1^2 2\pi y(y^2+1) dy = 2\pi \int_1^2 (y^3+y) dy = 2\pi \left[\frac{y^4}{4}+\frac{y^2}{2}\right]_1^2 = 2\pi \left( (4+2) - (\frac{1}{4}+\frac{1}{2}) \right) = 2\pi(6-\frac{3}{4}) = \frac{21\pi}{2}$.

\subsection*{Solution to Problem 4}
$y=x^{1/3} \Rightarrow x=y^3$. $\frac{dx}{dy}=3y^2$. Bounds for y are 1 to 2.
$S = \int_1^2 2\pi x \sqrt{1+(\frac{dx}{dy})^2} dy = \int_1^2 2\pi y^3 \sqrt{1+9y^4} dy$.
Let $u=1+9y^4, du=36y^3 dy \Rightarrow y^3 dy = \frac{du}{36}$.
$S = \int_{10}^{145} 2\pi \sqrt{u} \frac{du}{36} = \frac{\pi}{18} \left[\frac{2}{3}u^{3/2}\right]_{10}^{145} = \frac{\pi}{27}(145\sqrt{145}-10\sqrt{10})$.

\subsection*{Solution to Problem 5}
$y=\cos(x), \frac{dy}{dx}=-\sin(x)$. $S = \int_0^{\pi/2} 2\pi \cos(x) \sqrt{1+\sin^2(x)} dx$.
Let $u=\sin(x), du=\cos(x) dx$. Bounds: $x=0 \Rightarrow u=0, x=\pi/2 \Rightarrow u=1$.
$S = \int_0^1 2\pi \sqrt{1+u^2} du$. This is a standard integral: $2\pi \left[\frac{u}{2}\sqrt{1+u^2} + \frac{1}{2}\ln|u+\sqrt{1+u^2}|\right]_0^1 = \pi[\sqrt{2}+\ln(1+\sqrt{2})]$.

\subsection*{Solution to Problem 6}
$y=\sqrt{9-x^2}, \frac{dy}{dx}=\frac{-2x}{2\sqrt{9-x^2}}=\frac{-x}{\sqrt{9-x^2}}$.
$1+(\frac{dy}{dx})^2 = 1+\frac{x^2}{9-x^2}=\frac{9-x^2+x^2}{9-x^2}=\frac{9}{9-x^2}$.
$S = \int_0^2 2\pi \sqrt{9-x^2} \sqrt{\frac{9}{9-x^2}} dx = \int_0^2 2\pi \sqrt{9-x^2} \frac{3}{\sqrt{9-x^2}} dx = \int_0^2 6\pi dx = 6\pi[x]_0^2 = 12\pi$.

\subsection*{Solution to Problem 7}
$x=e^{2y}, \frac{dx}{dy}=2e^{2y}$. $S = \int_0^{\ln 3} 2\pi e^{2y} \sqrt{1+4e^{4y}} dy$.
Let $u=2e^{2y}, du=4e^{2y}dy \Rightarrow e^{2y}dy = du/4$. $S = \int_{2}^{18} 2\pi \sqrt{1+u^2} \frac{du}{4} = \frac{\pi}{2} \int_2^{18} \sqrt{1+u^2} du$.
Using standard formula: $\frac{\pi}{2} \left[\frac{u}{2}\sqrt{1+u^2} + \frac{1}{2}\ln|u+\sqrt{1+u^2}|\right]_2^{18} = \frac{\pi}{4}[18\sqrt{325}+\ln(18+\sqrt{325}) - 2\sqrt{5}-\ln(2+\sqrt{5})]$.

\subsection*{Solution to Problem 8}
$y=\sqrt{12-x}, \frac{dy}{dx}=\frac{-1}{2\sqrt{12-x}}$. $1+(\frac{dy}{dx})^2 = 1+\frac{1}{4(12-x)} = \frac{48-4x+1}{4(12-x)} = \frac{49-4x}{4(12-x)}$.
$S = \int_3^8 2\pi \sqrt{12-x} \frac{\sqrt{49-4x}}{2\sqrt{12-x}} dx = \pi \int_3^8 \sqrt{49-4x} dx$.
Let $u=49-4x, du=-4dx$. $S = \pi \int_{37}^{17} \sqrt{u} \frac{du}{-4} = \frac{\pi}{4} \int_{17}^{37} u^{1/2} du = \frac{\pi}{4}[\frac{2}{3}u^{3/2}]_{17}^{37} = \frac{\pi}{6}(37\sqrt{37}-17\sqrt{17})$.

\subsection*{Solution to Problem 9}
This is the same calculation as problem 6. $y=\sqrt{R^2-x^2} \Rightarrow ds = \frac{R}{\sqrt{R^2-x^2}}dx$. Here $R=5$.
$S = \int_0^3 2\pi \sqrt{25-x^2} \frac{5}{\sqrt{25-x^2}} dx = \int_0^3 10\pi dx = 10\pi[x]_0^3 = 30\pi$.

\subsection*{Solution to Problem 10}
$y=\sqrt{2x+1}, \frac{dy}{dx}=\frac{1}{\sqrt{2x+1}}$. $1+(\frac{dy}{dx})^2 = 1+\frac{1}{2x+1} = \frac{2x+2}{2x+1}$.
$S = \int_1^4 2\pi \sqrt{2x+1} \frac{\sqrt{2x+2}}{\sqrt{2x+1}} dx = 2\pi \int_1^4 \sqrt{2x+2} dx$.
Let $u=2x+2, du=2dx$. $S=2\pi \int_4^{10} \sqrt{u}\frac{du}{2} = \pi [\frac{2}{3}u^{3/2}]_4^{10} = \frac{2\pi}{3}(10\sqrt{10}-8)$.

\subsection*{Solution to Problem 11}
$y=2\sqrt{x}, \frac{dy}{dx}=\frac{1}{\sqrt{x}}$. $1+(\frac{dy}{dx})^2 = 1+\frac{1}{x} = \frac{x+1}{x}$.
$S = \int_3^8 2\pi (2\sqrt{x}) \sqrt{\frac{x+1}{x}} dx = 4\pi \int_3^8 \sqrt{x} \frac{\sqrt{x+1}}{\sqrt{x}} dx = 4\pi \int_3^8 \sqrt{x+1} dx$.
Let $u=x+1, du=dx$. $S = 4\pi \int_4^9 u^{1/2} du = 4\pi[\frac{2}{3}u^{3/2}]_4^9 = \frac{8\pi}{3}(27-8)=\frac{152\pi}{3}$.

\subsection*{Solution to Problem 12}
$y=\sqrt{x-1}, \frac{dy}{dx}=\frac{1}{2\sqrt{x-1}}$. $1+(\frac{dy}{dx})^2 = 1+\frac{1}{4(x-1)} = \frac{4x-4+1}{4(x-1)} = \frac{4x-3}{4(x-1)}$.
$S=\int_2^5 2\pi \sqrt{x-1} \frac{\sqrt{4x-3}}{2\sqrt{x-1}} dx = \pi \int_2^5 \sqrt{4x-3} dx$.
Let $u=4x-3, du=4dx$. $S=\pi \int_5^{17} \sqrt{u} \frac{du}{4} = \frac{\pi}{4}[\frac{2}{3}u^{3/2}]_5^{17} = \frac{\pi}{6}(17\sqrt{17}-5\sqrt{5})$.

\subsection*{Solution to Problem 13}
$y=2\sqrt{3-x}, \frac{dy}{dx}=2\frac{-1}{2\sqrt{3-x}}=\frac{-1}{\sqrt{3-x}}$. $1+(\frac{dy}{dx})^2 = 1+\frac{1}{3-x} = \frac{3-x+1}{3-x} = \frac{4-x}{3-x}$.
$S=\int_1^2 2\pi (2\sqrt{3-x}) \sqrt{\frac{4-x}{3-x}} dx = 4\pi \int_1^2 \sqrt{4-x} dx$.
Let $u=4-x, du=-dx$. $S=4\pi \int_3^2 \sqrt{u}(-du) = 4\pi \int_2^3 u^{1/2}du = 4\pi[\frac{2}{3}u^{3/2}]_2^3 = \frac{8\pi}{3}(3\sqrt{3}-2\sqrt{2})$.

\subsection*{Solution to Problem 14}
$x=\sqrt{4-y}, \frac{dx}{dy}=\frac{-1}{2\sqrt{4-y}}$. $1+(\frac{dx}{dy})^2 = 1+\frac{1}{4(4-y)} = \frac{16-4y+1}{4(4-y)} = \frac{17-4y}{4(4-y)}$.
$S=\int_0^3 2\pi \sqrt{4-y} \frac{\sqrt{17-4y}}{2\sqrt{4-y}} dy = \pi \int_0^3 \sqrt{17-4y} dy$.
Let $u=17-4y, du=-4dy$. $S=\pi \int_{17}^5 \sqrt{u} \frac{du}{-4} = \frac{\pi}{4}\int_5^{17} u^{1/2}du = \frac{\pi}{4}[\frac{2}{3}u^{3/2}]_5^{17} = \frac{\pi}{6}(17\sqrt{17}-5\sqrt{5})$.

\subsection*{Solution to Problem 15}
$y=\sqrt{x}, \frac{dy}{dx}=\frac{1}{2\sqrt{x}}$. $1+(\frac{dy}{dx})^2=1+\frac{1}{4x}=\frac{4x+1}{4x}$.
$S=\int_1^6 2\pi \sqrt{x} \frac{\sqrt{4x+1}}{2\sqrt{x}} dx = \pi \int_1^6 \sqrt{4x+1} dx$.
Let $u=4x+1, du=4dx$. $S=\pi \int_5^{25} \sqrt{u}\frac{du}{4} = \frac{\pi}{4}[\frac{2}{3}u^{3/2}]_5^{25} = \frac{\pi}{6}(125-5\sqrt{5})$.

\subsection*{Solution to Problem 16}
$y=\frac{1}{2}\sqrt{x}, \frac{dy}{dx}=\frac{1}{4\sqrt{x}}$. $1+(\frac{dy}{dx})^2=1+\frac{1}{16x}=\frac{16x+1}{16x}$.
$S=\int_1^3 2\pi (\frac{1}{2}\sqrt{x}) \frac{\sqrt{16x+1}}{4\sqrt{x}} dx = \frac{\pi}{4} \int_1^3 \sqrt{16x+1} dx$.
Let $u=16x+1, du=16dx$. $S=\frac{\pi}{4}\int_{17}^{49} \sqrt{u}\frac{du}{16} = \frac{\pi}{64}[\frac{2}{3}u^{3/2}]_{17}^{49} = \frac{\pi}{96}(343-17\sqrt{17})$.

\subsection*{Solution to Problem 17}
$y=\frac{x^3}{6}+\frac{1}{2x}, \frac{dy}{dx}=\frac{x^2}{2}-\frac{1}{2x^2}$. $1+(\frac{dy}{dx})^2 = 1+(\frac{x^4}{4}-\frac{1}{2}+\frac{1}{4x^4}) = \frac{x^4}{4}+\frac{1}{2}+\frac{1}{4x^4} = (\frac{x^2}{2}+\frac{1}{2x^2})^2$.
$S=\int_1^2 2\pi (\frac{x^3}{6}+\frac{1}{2x})(\frac{x^2}{2}+\frac{1}{2x^2})dx = 2\pi\int_1^2 (\frac{x^5}{12}+\frac{x}{12}+\frac{x}{4}+\frac{1}{4x^3})dx = 2\pi\int_1^2 (\frac{x^5}{12}+\frac{x}{3}+\frac{1}{4}x^{-3})dx = 2\pi[\frac{x^6}{72}+\frac{x^2}{6}-\frac{1}{8x^2}]_1^2 = \frac{47\pi}{36}$.

\subsection*{Solution to Problem 18}
$y=\frac{x^2}{4}-\frac{1}{2}\ln x, \frac{dy}{dx}=\frac{x}{2}-\frac{1}{2x}$. $1+(\frac{dy}{dx})^2 = 1+(\frac{x^2}{4}-\frac{1}{2}+\frac{1}{4x^2}) = \frac{x^2}{4}+\frac{1}{2}+\frac{1}{4x^2} = (\frac{x}{2}+\frac{1}{2x})^2$.
$S=\int_1^e 2\pi x (\frac{x}{2}+\frac{1}{2x}) dx = \pi \int_1^e (x^2+1)dx = \pi[\frac{x^3}{3}+x]_1^e = \pi(\frac{e^3}{3}+e - \frac{4}{3})$.

\subsection*{Solution to Problem 19}
$x=\frac{y^4}{4}+\frac{1}{8y^2}, \frac{dx}{dy}=y^3-\frac{1}{4y^3}$. $1+(\frac{dx}{dy})^2=1+(y^6-\frac{1}{2}+\frac{1}{16y^6})=y^6+\frac{1}{2}+\frac{1}{16y^6}=(y^3+\frac{1}{4y^3})^2$.
$S=\int_1^2 2\pi(\frac{y^4}{4}+\frac{1}{8y^2})(y^3+\frac{1}{4y^3})dy = 2\pi \int_1^2 (\frac{y^7}{4}+\frac{y}{16}+\frac{y}{8}+\frac{1}{32y^5})dy = 2\pi \int_1^2 (\frac{y^7}{4}+\frac{3y}{16}+\frac{1}{32}y^{-5})dy = 2\pi[\frac{y^8}{32}+\frac{3y^2}{32}-\frac{1}{128y^4}]_1^2 = \frac{255\pi}{128}$.

\subsection*{Solution to Problem 20}
$y=\frac{x^5}{5}+\frac{1}{12x^3}, \frac{dy}{dx}=x^4-\frac{1}{4x^4}$. $1+(\frac{dy}{dx})^2 = 1+(x^8-\frac{1}{2}+\frac{1}{16x^8}) = x^8+\frac{1}{2}+\frac{1}{16x^8}=(x^4+\frac{1}{4x^4})^2$.
$S = \int_1^2 2\pi y \sqrt{1+(y')^2} dx = \int_1^2 2\pi (\frac{x^5}{5}+\frac{1}{12x^3})(x^4+\frac{1}{4x^4}) dx = \frac{18433\pi}{7200}$.

\subsection*{Solution to Problem 21}
$y=\cosh(x), y'=\sinh(x)$. $1+(y')^2=1+\sinh^2(x)=\cosh^2(x)$.
$S = \int_0^1 2\pi\cosh(x)\sqrt{\cosh^2(x)}dx = 2\pi\int_0^1 \cosh^2(x)dx = 2\pi\int_0^1 \frac{1+\cosh(2x)}{2}dx = \pi[x+\frac{\sinh(2x)}{2}]_0^1 = \pi(1+\frac{\sinh(2)}{2})$.

\subsection*{Solution to Problem 22}
$y=\frac{x^4}{8}+\frac{1}{4x^2}, \frac{dy}{dx}=\frac{x^3}{2}-\frac{1}{2x^3}$. $1+(\frac{dy}{dx})^2=1+(\frac{x^6}{4}-\frac{1}{2}+\frac{1}{4x^6})=\frac{x^6}{4}+\frac{1}{2}+\frac{1}{4x^6}=(\frac{x^3}{2}+\frac{1}{2x^3})^2$.
$S=\int_1^2 2\pi(\frac{x^4}{8}+\frac{1}{4x^2})(\frac{x^3}{2}+\frac{1}{2x^3})dx=2\pi\int_1^2(\frac{x^7}{16}+\frac{x}{16}+\frac{x}{8}+\frac{1}{8x^5})dx = 2\pi[\frac{x^8}{128}+\frac{3x^2}{32}-\frac{1}{32x^4}]_1^2 = \frac{303\pi}{256}$.

\subsection*{Solution to Problem 23}
$x=\frac{y^3}{3}+\frac{1}{4y}, \frac{dx}{dy}=y^2-\frac{1}{4y^2}$. $1+(\frac{dx}{dy})^2=1+y^4-\frac{1}{2}+\frac{1}{16y^4}=y^4+\frac{1}{2}+\frac{1}{16y^4}=(y^2+\frac{1}{4y^2})^2$.
$S=\int_1^3 2\pi y(y^2+\frac{1}{4y^2})dy = 2\pi\int_1^3(y^3+\frac{1}{4y})dy = 2\pi[\frac{y^4}{4}+\frac{1}{4}\ln y]_1^3 = 2\pi((\frac{81}{4}+\frac{\ln 3}{4})-(\frac{1}{4})) = \frac{\pi}{2}(80+\ln 3)$.

\subsection*{Solution to Problem 24}
$y=\frac{x^3}{3}+\frac{1}{4x}, \frac{dy}{dx}=x^2-\frac{1}{4x^2}$. $1+(\frac{dy}{dx})^2 = (x^2+\frac{1}{4x^2})^2$.
$S=\int_1^2 2\pi(\frac{x^3}{3}+\frac{1}{4x})(x^2+\frac{1}{4x^2})dx=2\pi\int_1^2(\frac{x^5}{3}+\frac{x}{12}+\frac{x}{4}+\frac{1}{16x^3})dx=2\pi[\frac{x^6}{18}+\frac{x^2}{6}-\frac{1}{32x^2}]_1^2 = \frac{589\pi}{288}$.

\subsection*{Solution to Problem 25}
$x=\frac{y^4}{2}+\frac{1}{16y^2}, \frac{dx}{dy}=2y^3-\frac{1}{8y^3}$. $1+(\frac{dx}{dy})^2=1+4y^6-\frac{1}{2}+\frac{1}{64y^6}=4y^6+\frac{1}{2}+\frac{1}{64y^6}=(2y^3+\frac{1}{8y^3})^2$.
$S=\int_1^2 2\pi y(2y^3+\frac{1}{8y^3})dy=2\pi\int_1^2(2y^4+\frac{1}{8y^2})dy=2\pi[\frac{2y^5}{5}-\frac{1}{8y}]_1^2 = \frac{2053\pi}{80}$.

\subsection*{Solution to Problem 26}
Axis $y=-3$, so radius $r=y-(-3)=x^2+3$. $y'=2x$. $ds=\sqrt{1+4x^2}dx$.
$S=\int_0^2 2\pi(x^2+3)\sqrt{1+4x^2}dx$.

\subsection*{Solution to Problem 27}
Axis $x=2$, so radius $r=2-x$. $y'=e^x$. $ds=\sqrt{1+e^{2x}}dx$.
$S=\int_0^1 2\pi(2-x)\sqrt{1+e^{2x}}dx$.

\subsection*{Solution to Problem 28}
Axis $y=1$, so radius $r=y-1=(x+1)-1=x$. $y'=1$. $ds=\sqrt{1+1^2}dx=\sqrt{2}dx$.
$S=\int_0^3 2\pi x \sqrt{2} dx = 2\pi\sqrt{2}[\frac{x^2}{2}]_0^3 = 9\pi\sqrt{2}$.

\subsection*{Solution to Problem 29}
Axis $x=-1$, so radius $r=x-(-1)=(2y+1)+1=2y+2$. $x'=2$. $ds=\sqrt{1+2^2}dy=\sqrt{5}dy$.
$S=\int_0^2 2\pi(2y+2)\sqrt{5}dy = 4\pi\sqrt{5}\int_0^2(y+1)dy = 4\pi\sqrt{5}[\frac{y^2}{2}+y]_0^2 = 4\pi\sqrt{5}(2+2)=16\pi\sqrt{5}$.

\subsection*{Solution to Problem 30}
$x'= -R\sin t, y'=R\cos t$. $ds=\sqrt{(-R\sin t)^2+(R\cos t)^2}dt = \sqrt{R^2}dt=Rdt$. $r=y(t)=R\sin t$.
$S=\int_0^\pi 2\pi (R\sin t) (R dt) = 2\pi R^2 \int_0^\pi \sin t dt = 2\pi R^2[-\cos t]_0^\pi = 2\pi R^2(1-(-1))=4\pi R^2$.

\subsection*{Solution to Problem 31}
$x=t^3, y=t^2 \Rightarrow x'=3t^2, y'=2t$. $ds=\sqrt{9t^4+4t^2}dt = t\sqrt{9t^2+4}dt$. $r=y=t^2$.
$S=\int_0^1 2\pi t^2 (t\sqrt{9t^2+4})dt = 2\pi\int_0^1 t^3\sqrt{9t^2+4}dt$.
Let $u=9t^2+4, du=18tdt, t^2=(u-4)/9$. $S=\frac{2\pi}{18}\int_4^{13}\frac{u-4}{9} \sqrt{u} du = \frac{\pi}{81}\int_4^{13}(u^{3/2}-4u^{1/2})du = \frac{\pi}{81}[\frac{2}{5}u^{5/2}-\frac{8}{3}u^{3/2}]_4^{13} = \frac{2\pi}{1215}(97\sqrt{13}-112)$.

\subsection*{Solution to Problem 32}
$x'=-3\cos^2 t \sin t, y'=3\sin^2 t \cos t$. $ds=\sqrt{9\cos^4t\sin^2t+9\sin^4t\cos^2t}dt=3|\cos t\sin t|\sqrt{\cos^2t+\sin^2t}dt=3\cos t\sin t dt$ for $t\in[0,\pi/2]$.
$r=y=\sin^3 t$. $S=\int_0^{\pi/2} 2\pi \sin^3 t (3\cos t\sin t)dt=6\pi\int_0^{\pi/2} \sin^4 t \cos t dt$.
Let $u=\sin t, du=\cos t dt$. $S=6\pi\int_0^1 u^4 du=6\pi[\frac{u^5}{5}]_0^1 = \frac{6\pi}{5}$.

\subsection*{Solution to Problem 33}
$x=t-\sin t, y=1-\cos t \Rightarrow x'=1-\cos t, y'=\sin t$. $ds=\sqrt{(1-\cos t)^2+\sin^2t}dt=\sqrt{1-2\cos t+\cos^2t+\sin^2t}dt=\sqrt{2-2\cos t}dt=\sqrt{4\sin^2(t/2)}dt=2\sin(t/2)dt$.
Radius for rotation about y-axis is $r=x=t-\sin t$.
$S=\int_0^{2\pi} 2\pi(t-\sin t)(2\sin(t/2))dt = 4\pi\int_0^{2\pi}(t-\sin t)\sin(t/2)dt$.

\subsection*{Solution to Problem 34}
$y=e^{-x}, y'=-e^{-x}$. $ds=\sqrt{1+e^{-2x}}dx$. $S=\int_0^\infty 2\pi e^{-x}\sqrt{1+e^{-2x}}dx$.
Let $u=e^{-x}, du=-e^{-x}dx$. Bounds: $x=0\Rightarrow u=1, x\to\infty\Rightarrow u\to 0$.
$S=\int_1^0 2\pi\sqrt{1+u^2}(-du) = 2\pi\int_0^1 \sqrt{1+u^2}du = \pi[\sqrt{2}+\ln(1+\sqrt{2})]$ (from problem 5). The area is finite.

\subsection*{Solution to Problem 35}
$y=1/x^2, y'=-2/x^3$. $ds=\sqrt{1+4/x^6}dx$. $S=\int_1^\infty 2\pi \frac{1}{x^2}\sqrt{1+\frac{4}{x^6}}dx = \int_1^\infty 2\pi \frac{\sqrt{x^6+4}}{x^5}dx$.
For large $x, \frac{\sqrt{x^6+4}}{x^5} \approx \frac{\sqrt{x^6}}{x^5}=\frac{x^3}{x^5}=\frac{1}{x^2}$.
We compare to $\int_1^\infty \frac{1}{x^2}dx$. This is a convergent p-integral ($p=2>1$). By limit comparison test, the surface area integral also converges. The area is finite.

\subsection*{Solution to Problem 36}
$y=\tan x, y'=\sec^2 x$. $r=y=\tan x$.
$S=\int_0^{\pi/4} 2\pi \tan x \sqrt{1+\sec^4 x} dx$.

\subsection*{Solution to Problem 37}
$y=\ln(\cos x), y'=\frac{-\sin x}{\cos x}=-\tan x$. $r=x$.
$S=\int_0^{\pi/3} 2\pi x \sqrt{1+(-\tan x)^2} dx = \int_0^{\pi/3} 2\pi x \sqrt{1+\tan^2 x} dx = \int_0^{\pi/3} 2\pi x \sec x dx$.

\subsection*{Solution to Problem 38}
$y=\frac{2}{3}x^{3/2}, y'=\sqrt{x}$. $ds=\sqrt{1+x}dx$. $r=x$.
$S=\int_0^3 2\pi x \sqrt{1+x}dx$. Let $u=1+x, x=u-1, du=dx$.
$S=2\pi\int_1^4 (u-1)\sqrt{u}du = 2\pi\int_1^4(u^{3/2}-u^{1/2})du=2\pi[\frac{2}{5}u^{5/2}-\frac{2}{3}u^{3/2}]_1^4 = 2\pi[(\frac{64}{5}-\frac{16}{3})-(\frac{2}{5}-\frac{2}{3})] = \frac{224\pi}{15}$.

\subsection*{Solution to Problem 39}
$x = y^2/9+2 \Rightarrow y=\sqrt{9x-18}=3\sqrt{x-2}$. $y'=\frac{3}{2\sqrt{x-2}}$. $1+(y')^2=1+\frac{9}{4(x-2)}=\frac{4x-8+9}{4(x-2)}=\frac{4x+1}{4(x-2)}$.
$S=\int_2^6 2\pi (3\sqrt{x-2})\frac{\sqrt{4x+1}}{2\sqrt{x-2}}dx = 3\pi\int_2^6 \sqrt{4x+1}dx$.
Let $u=4x+1, du=4dx$. $S=3\pi\int_9^{25} \sqrt{u}\frac{du}{4} = \frac{3\pi}{4}[\frac{2}{3}u^{3/2}]_9^{25} = \frac{\pi}{2}(125-27)=49\pi$.

\subsection*{Solution to Problem 40}
$y = \frac{1}{3}x^{1/2} - x^{3/2}, y'=\frac{1}{6}x^{-1/2}-\frac{3}{2}x^{1/2}$. $r=x$.
$ds=\sqrt{1+(\frac{1}{6\sqrt{x}}-\frac{3\sqrt{x}}{2})^2}dx$.
$S=\int_0^{1/3} 2\pi x \sqrt{1+(\frac{1}{6\sqrt{x}}-\frac{3\sqrt{x}}{2})^2} dx$.

\subsection*{Solution to Problem 41}
$x=2\pm\sqrt{1-y^2}$. The outer surface has radius $r_1=x=2+\sqrt{1-y^2}$ and inner surface has $r_2=x=2-\sqrt{1-y^2}$. $y$ ranges from -1 to 1.
$\frac{dx}{dy}=\pm\frac{-y}{\sqrt{1-y^2}}$. $ds=\sqrt{1+\frac{y^2}{1-y^2}}dy = \frac{1}{\sqrt{1-y^2}}dy$.
$S = \int_{-1}^1 2\pi(2+\sqrt{1-y^2})\frac{dy}{\sqrt{1-y^2}} + \int_{-1}^1 2\pi(2-\sqrt{1-y^2})\frac{dy}{\sqrt{1-y^2}}$.
$S = \int_{-1}^1 2\pi(\frac{2}{\sqrt{1-y^2}}+1)dy + \int_{-1}^1 2\pi(\frac{2}{\sqrt{1-y^2}}-1)dy = \int_{-1}^1 \frac{8\pi}{\sqrt{1-y^2}}dy$.

\subsection*{Solution to Problem 42}
$x=R+r\cos t, y=r\sin t$. $x'=-r\sin t, y'=r\cos t$. $ds=\sqrt{r^2\sin^2t+r^2\cos^2t}dt = r dt$.
Radius for rotation about y-axis is $r_{rot}=x(t)=R+r\cos t$.
$S=\int_0^{2\pi} 2\pi(R+r\cos t)(r dt) = 2\pi r \int_0^{2\pi}(R+r\cos t)dt = 2\pi r [Rt+r\sin t]_0^{2\pi} = 2\pi r(2\pi R) = 4\pi^2Rr$.

\newpage
\section*{Concept Checklist}
This checklist helps categorize the problems by the primary concept being tested.

\begin{itemize}
    \item \textbf{Setup Only Problems}: These problems test the fundamental understanding of setting up the correct integral (radius, derivative, bounds) without the need for complex integration.
    \begin{itemize}
        \item Questions: 26, 27, 33, 36, 37, 40, 41
    \end{itemize}

    \item \textbf{Direct Integration via U-Substitution}: These problems result in an integral that is solvable with a standard u-substitution after setup.
    \begin{itemize}
        \item Questions: 1, 3, 4, 5, 31, 32, 38
    \end{itemize}

    \item \textbf{Radical Cancellation Problems}: Special problems where the function $y$ is a square root, leading to a cancellation with a term in the denominator of the arc length element, simplifying the integral.
    \begin{itemize}
        \item Questions: 2, 6, 8, 9, 10, 11, 12, 13, 14, 15, 16, 39
    \end{itemize}

    \item \textbf{The "Perfect Square Trick" Problems}: Problems where the expression $1+(y')^2$ or $1+(x')^2$ simplifies into a perfect square, thus eliminating the square root from the integrand. These functions are often of the form $ax^n + bx^{-m}$.
    \begin{itemize}
        \item Questions: 17, 18, 19, 20, 21, 22, 23, 24, 25
    \end{itemize}

    \item \textbf{Rotation About Arbitrary Lines}: Problems where the axis of rotation is a horizontal or vertical line other than the x or y-axis, requiring a modification of the radius term $r$.
    \begin{itemize}
        \item Questions: 26, 27, 28, 29
    \end{itemize}

    \item \textbf{Surfaces from Parametric Curves}: Problems involving curves defined by parametric equations, requiring the use of the parametric form of the arc length element.
    \begin{itemize}
        \item Questions: 30, 31, 32, 33, 42
    \end{itemize}

    \item \textbf{Improper Integral Problems}: Problems characterized by an infinite bound of integration, which require convergence tests or direct evaluation of an improper integral to determine if the area is finite.
    \begin{itemize}
        \item Questions: 34, 35
    \end{itemize}
\end{itemize}


\end{document}