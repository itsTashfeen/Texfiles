\documentclass{article}
\usepackage{amsmath}
\usepackage{amsfonts}
\usepackage{amssymb}
\usepackage{geometry}
\geometry{a4paper, margin=1in}

\title{7.8: Improper Integrals - Problem Set}
\author{Tashfeen Omran}
\date{October 2025}

\begin{document}

\maketitle

\section*{Part I: Problems}

% Type 1: [a, inf)
\subsection*{Problem 1}
Determine whether the integral is convergent or divergent. If it is convergent, evaluate it.
\[ \int_{2}^{\infty} \frac{5}{x^3} \,dx \]

\subsection*{Problem 2}
Determine whether the integral is convergent or divergent. If it is convergent, evaluate it.
\[ \int_{1}^{\infty} \frac{1}{\sqrt[4]{x}} \,dx \]

\subsection*{Problem 3}
Determine whether the integral is convergent or divergent. If it is convergent, evaluate it.
\[ \int_{0}^{\infty} e^{-2x} \,dx \]

\subsection*{Problem 4}
Determine whether the integral is convergent or divergent. If it is convergent, evaluate it.
\[ \int_{e}^{\infty} \frac{1}{x (\ln x)^2} \,dx \]

\subsection*{Problem 5}
Determine whether the integral is convergent or divergent. If it is convergent, evaluate it.
\[ \int_{1}^{\infty} \frac{x^2 + 2}{x^3} \,dx \]

% Type 1: (-inf, b]
\subsection*{Problem 6}
Determine whether the integral is convergent or divergent. If it is convergent, evaluate it.
\[ \int_{-\infty}^{0} \frac{1}{(1-x)^{3/2}} \,dx \]

\subsection*{Problem 7}
Determine whether the integral is convergent or divergent. If it is convergent, evaluate it.
\[ \int_{-\infty}^{-1} \frac{1}{x^5} \,dx \]

\subsection*{Problem 8}
Determine whether the integral is convergent or divergent. If it is convergent, evaluate it.
\[ \int_{-\infty}^{0} \frac{x}{(x^2+1)^2} \,dx \]

% Type 1: (-inf, inf)
\subsection*{Problem 9}
Determine whether the integral is convergent or divergent. If it is convergent, evaluate it.
\[ \int_{-\infty}^{\infty} \frac{x}{1+x^2} \,dx \]

\subsection*{Problem 10}
Determine whether the integral is convergent or divergent. If it is convergent, evaluate it.
\[ \int_{-\infty}^{\infty} \frac{1}{x^2+4} \,dx \]

\subsection*{Problem 11}
Determine whether the integral is convergent or divergent. If it is convergent, evaluate it.
\[ \int_{-\infty}^{\infty} x^2 e^{-x^3} \,dx \]

% Type 2: Discontinuity
\subsection*{Problem 12}
Determine whether the integral is convergent or divergent. If it is convergent, evaluate it.
\[ \int_{0}^{1} \frac{1}{\sqrt[3]{x}} \,dx \]

\subsection*{Problem 13}
Determine whether the integral is convergent or divergent. If it is convergent, evaluate it.
\[ \int_{0}^{2} \frac{1}{(x-2)^2} \,dx \]

\subsection*{Problem 14}
Determine whether the integral is convergent or divergent. If it is convergent, evaluate it.
\[ \int_{0}^{3} \frac{1}{\sqrt{3-x}} \,dx \]

\subsection*{Problem 15}
Determine whether the integral is convergent or divergent. If it is convergent, evaluate it.
\[ \int_{-1}^{8} \frac{1}{\sqrt[3]{x}} \,dx \]

% Integration Techniques: Partial Fractions
\subsection*{Problem 16}
Determine whether the integral is convergent or divergent. If it is convergent, evaluate it.
\[ \int_{1}^{\infty} \frac{1}{x^2+x} \,dx \]

\subsection*{Problem 17}
Determine whether the integral is convergent or divergent. If it is convergent, evaluate it.
\[ \int_{2}^{\infty} \frac{4}{x^2-1} \,dx \]

% Integration Techniques: Integration by Parts
\subsection*{Problem 18}
Determine whether the integral is convergent or divergent. If it is convergent, evaluate it.
\[ \int_{0}^{\infty} x e^{-x} \,dx \]

\subsection*{Problem 19}
Determine whether the integral is convergent or divergent. If it is convergent, evaluate it.
\[ \int_{1}^{\infty} \frac{\ln x}{x^2} \,dx \]

% Oscillating Functions / Trig
\subsection*{Problem 20}
Determine whether the integral is convergent or divergent. If it is convergent, evaluate it.
\[ \int_{0}^{\infty} \cos(x) \,dx \]

\subsection*{Problem 21}
Determine whether the integral is convergent or divergent. If it is convergent, evaluate it.
\[ \int_{0}^{\infty} 2\cos^2(x) \,dx \]

% More Mixed Problems
\subsection*{Problem 22}
Determine whether the integral is convergent or divergent. If it is convergent, evaluate it.
\[ \int_{1}^{\infty} \frac{e^{-\sqrt{x}}}{\sqrt{x}} \,dx \]

\subsection*{Problem 23}
Determine whether the integral is convergent or divergent. If it is convergent, evaluate it.
\[ \int_{-\infty}^{0} xe^x \,dx \]

\subsection*{Problem 24}
Determine whether the integral is convergent or divergent. If it is convergent, evaluate it.
\[ \int_{0}^{1} \frac{1}{4y-1} \,dy \]

\subsection*{Problem 25}
Determine whether the integral is convergent or divergent. If it is convergent, evaluate it.
\[ \int_{1}^{\infty} \frac{\arctan(x)}{x^2+1} \,dx \]

\subsection*{Problem 26}
Determine whether the integral is convergent or divergent. If it is convergent, evaluate it.
\[ \int_{0}^{\pi/2} \tan(x) \,dx \]

\subsection*{Problem 27}
Determine whether the integral is convergent or divergent. If it is convergent, evaluate it.
\[ \int_{-\infty}^{\infty} \frac{e^x}{1+e^{2x}} \,dx \]

\subsection*{Problem 28}
Determine whether the integral is convergent or divergent. If it is convergent, evaluate it.
\[ \int_{0}^{1} \ln(x) \,dx \]

\subsection*{Problem 29}
Determine whether the integral is convergent or divergent. If it is convergent, evaluate it.
\[ \int_{1}^{\infty} \frac{1}{x\sqrt{x^2-1}} \,dx \]

\subsection*{Problem 30}
Determine whether the integral is convergent or divergent. If it is convergent, evaluate it.
\[ \int_{-\infty}^{1} \frac{1}{x^2-4x+5} \,dx \]

\newpage

\section*{Part II: Detailed Solutions}

\subsection*{Solution 1}
This is a Type 1 improper integral, which is a p-integral with $p=3 > 1$, so it converges.
\begin{align*}
\int_{2}^{\infty} 5x^{-3} \,dx &= \lim_{t \to \infty} \int_{2}^{t} 5x^{-3} \,dx \\
&= \lim_{t \to \infty} \left[ \frac{5x^{-2}}{-2} \right]_{2}^{t} = \lim_{t \to \infty} \left[ -\frac{5}{2x^2} \right]_{2}^{t} \\
&= \lim_{t \to \infty} \left( -\frac{5}{2t^2} - \left(-\frac{5}{2(2)^2}\right) \right) \\
&= 0 + \frac{5}{8} = \frac{5}{8}
\end{align*}
\textbf{Answer:} Convergent, value is $5/8$.

\subsection*{Solution 2}
This is a Type 1 improper integral, which is a p-integral with $p=1/4 \le 1$, so it diverges.
\begin{align*}
\int_{1}^{\infty} x^{-1/4} \,dx &= \lim_{t \to \infty} \int_{1}^{t} x^{-1/4} \,dx \\
&= \lim_{t \to \infty} \left[ \frac{x^{3/4}}{3/4} \right]_{1}^{t} = \lim_{t \to \infty} \left[ \frac{4}{3}x^{3/4} \right]_{1}^{t} \\
&= \lim_{t \to \infty} \left( \frac{4}{3}t^{3/4} - \frac{4}{3}(1)^{3/4} \right) \\
&= \infty - \frac{4}{3} = \infty
\end{align*}
\textbf{Answer:} Diverges.

\subsection*{Solution 3}
This is a Type 1 improper integral.
\begin{align*}
\int_{0}^{\infty} e^{-2x} \,dx &= \lim_{t \to \infty} \int_{0}^{t} e^{-2x} \,dx \\
&= \lim_{t \to \infty} \left[ -\frac{1}{2}e^{-2x} \right]_{0}^{t} \\
&= \lim_{t \to \infty} \left( -\frac{1}{2}e^{-2t} - \left(-\frac{1}{2}e^{0}\right) \right) \\
&= 0 + \frac{1}{2} = \frac{1}{2}
\end{align*}
\textbf{Answer:} Convergent, value is $1/2$.

\subsection*{Solution 4}
This is a Type 1 improper integral. Use u-substitution with $u=\ln x$, so $du = \frac{1}{x}dx$. When $x=e, u=1$. When $x \to \infty, u \to \infty$.
\begin{align*}
\int_{e}^{\infty} \frac{1}{x (\ln x)^2} \,dx &= \int_{1}^{\infty} \frac{1}{u^2} \,du \\
&= \lim_{t \to \infty} \int_{1}^{t} u^{-2} \,du = \lim_{t \to \infty} \left[ -u^{-1} \right]_{1}^{t} \\
&= \lim_{t \to \infty} \left( -\frac{1}{t} - (-1) \right) = 0+1=1
\end{align*}
\textbf{Answer:} Convergent, value is $1$.

\subsection*{Solution 5}
This is a Type 1 improper integral. First, simplify the integrand.
\begin{align*}
\int_{1}^{\infty} \left(\frac{x^2}{x^3} + \frac{2}{x^3}\right) \,dx &= \int_{1}^{\infty} \left(\frac{1}{x} + 2x^{-3}\right) \,dx \\
&= \lim_{t \to \infty} \int_{1}^{t} \left(\frac{1}{x} + 2x^{-3}\right) \,dx \\
&= \lim_{t \to \infty} \left[ \ln|x| - x^{-2} \right]_{1}^{t} \\
&= \lim_{t \to \infty} \left( (\ln t - \frac{1}{t^2}) - (\ln 1 - 1) \right) \\
&= (\infty - 0) - (0-1) = \infty
\end{align*}
The integral diverges because the $\int \frac{1}{x}dx$ part diverges ($p=1$).
\textbf{Answer:} Diverges.

\subsection*{Solution 6}
This is a Type 1 improper integral.
\begin{align*}
\int_{-\infty}^{0} (1-x)^{-3/2} \,dx &= \lim_{t \to -\infty} \int_{t}^{0} (1-x)^{-3/2} \,dx \\
&= \lim_{t \to -\infty} \left[ 2(1-x)^{-1/2} \right]_{t}^{0} \\
&= \lim_{t \to -\infty} \left( 2(1)^{-1/2} - 2(1-t)^{-1/2} \right) \\
&= \lim_{t \to -\infty} \left( 2 - \frac{2}{\sqrt{1-t}} \right) \\
&= 2 - 0 = 2
\end{align*}
\textbf{Answer:} Convergent, value is $2$.

\subsection*{Solution 7}
This is a Type 1 improper integral. The p-integral with $p=5 > 1$ converges on $[1, \infty)$, and similarly converges on $(-\infty, -1]$.
\begin{align*}
\int_{-\infty}^{-1} x^{-5} \,dx &= \lim_{t \to -\infty} \int_{t}^{-1} x^{-5} \,dx \\
&= \lim_{t \to -\infty} \left[ \frac{x^{-4}}{-4} \right]_{t}^{-1} \\
&= \lim_{t \to -\infty} \left( \frac{(-1)^{-4}}{-4} - \frac{t^{-4}}{-4} \right) \\
&= \lim_{t \to -\infty} \left( -\frac{1}{4} + \frac{1}{4t^4} \right) = -\frac{1}{4} + 0 = -\frac{1}{4}
\end{align*}
\textbf{Answer:} Convergent, value is $-1/4$.

\subsection*{Solution 8}
This is a Type 1 improper integral. Use u-substitution with $u=x^2+1$, $du=2x\,dx$. When $x=0, u=1$. When $x \to -\infty, u \to \infty$.
\begin{align*}
\int_{-\infty}^{0} \frac{x}{(x^2+1)^2} \,dx &= \lim_{t \to -\infty} \int_{t}^{0} \frac{x}{(x^2+1)^2} \,dx \\
&= \int_{\infty}^{1} \frac{1}{u^2} \frac{du}{2} = -\frac{1}{2} \int_{1}^{\infty} u^{-2} \,du \\
&= -\frac{1}{2} \lim_{t \to \infty} \left[ -u^{-1} \right]_{1}^{t} \\
&= -\frac{1}{2} \lim_{t \to \infty} \left( -\frac{1}{t} - (-1) \right) = -\frac{1}{2}(0+1) = -\frac{1}{2}
\end{align*}
\textbf{Answer:} Convergent, value is $-1/2$.

\subsection*{Solution 9}
This is a Type 1 integral over $(-\infty, \infty)$. We split it at $x=0$.
\[ \int_{-\infty}^{\infty} \frac{x}{1+x^2} \,dx = \int_{-\infty}^{0} \frac{x}{1+x^2} \,dx + \int_{0}^{\infty} \frac{x}{1+x^2} \,dx \]
Let's evaluate the second part. Use $u=1+x^2, du=2x\,dx$.
\begin{align*}
\int_{0}^{\infty} \frac{x}{1+x^2} \,dx &= \lim_{t \to \infty} \int_{0}^{t} \frac{x}{1+x^2} \,dx \\
&= \lim_{t \to \infty} \left[ \frac{1}{2} \ln(1+x^2) \right]_{0}^{t} \\
&= \frac{1}{2} \lim_{t \to \infty} (\ln(1+t^2) - \ln(1)) = \infty
\end{align*}
Since one part diverges, the whole integral diverges. Note: The integrand is an odd function, but for the integral to be 0, it must first converge.
\textbf{Answer:} Diverges.

\subsection*{Solution 10}
This is a Type 1 integral over $(-\infty, \infty)$. Split at $x=0$. The integrand is even.
\[ \int_{-\infty}^{\infty} \frac{1}{x^2+4} \,dx = 2 \int_{0}^{\infty} \frac{1}{x^2+4} \,dx \]
\begin{align*}
2 \lim_{t \to \infty} \int_{0}^{t} \frac{1}{x^2+2^2} \,dx &= 2 \lim_{t \to \infty} \left[ \frac{1}{2} \arctan\left(\frac{x}{2}\right) \right]_{0}^{t} \\
&= \lim_{t \to \infty} \left( \arctan\left(\frac{t}{2}\right) - \arctan(0) \right) \\
&= \frac{\pi}{2} - 0 = \frac{\pi}{2}
\end{align*}
The original integral is $2 \times (\pi/2) = \pi$.
\textbf{Answer:} Convergent, value is $\pi$.

\subsection*{Solution 11}
This is a Type 1 integral over $(-\infty, \infty)$. Split at $x=0$.
\[ \int_{-\infty}^{0} x^2 e^{-x^3} \,dx + \int_{0}^{\infty} x^2 e^{-x^3} \,dx \]
Let's evaluate the second part. Use $u=-x^3, du=-3x^2\,dx$.
\begin{align*}
\int_{0}^{\infty} x^2 e^{-x^3} \,dx &= \lim_{t \to \infty} \int_{0}^{t} x^2 e^{-x^3} \,dx \\
&= \lim_{t \to \infty} \left[ -\frac{1}{3} e^{-x^3} \right]_{0}^{t} \\
&= -\frac{1}{3} \lim_{t \to \infty} (e^{-t^3} - e^0) = -\frac{1}{3}(0-1) = \frac{1}{3}
\end{align*}
The first part diverges:
\begin{align*}
\int_{-\infty}^{0} x^2 e^{-x^3} \,dx &= \lim_{t \to -\infty} \int_{t}^{0} x^2 e^{-x^3} \,dx \\
&= \lim_{t \to -\infty} \left[ -\frac{1}{3} e^{-x^3} \right]_{t}^{0} \\
&= -\frac{1}{3} \lim_{t \to -\infty} (e^{0} - e^{-t^3}) = -\frac{1}{3}(1 - \infty) = \infty
\end{align*}
Since one part diverges, the whole integral diverges.
\textbf{Answer:} Diverges.

\subsection*{Solution 12}
This is a Type 2 improper integral with a discontinuity at $x=0$. It's a p-integral with $p=1/3 < 1$, so it converges.
\begin{align*}
\int_{0}^{1} x^{-1/3} \,dx &= \lim_{t \to 0^+} \int_{t}^{1} x^{-1/3} \,dx \\
&= \lim_{t \to 0^+} \left[ \frac{3}{2}x^{2/3} \right]_{t}^{1} \\
&= \lim_{t \to 0^+} \left( \frac{3}{2}(1)^{2/3} - \frac{3}{2}t^{2/3} \right) = \frac{3}{2} - 0 = \frac{3}{2}
\end{align*}
\textbf{Answer:} Convergent, value is $3/2$.

\subsection*{Solution 13}
This is a Type 2 improper integral with a discontinuity at $x=2$. It's a p-integral with $p=2 > 1$, so it diverges.
\begin{align*}
\int_{0}^{2} (x-2)^{-2} \,dx &= \lim_{t \to 2^-} \int_{0}^{t} (x-2)^{-2} \,dx \\
&= \lim_{t \to 2^-} \left[ -(x-2)^{-1} \right]_{0}^{t} \\
&= \lim_{t \to 2^-} \left( -\frac{1}{t-2} - \left(-\frac{1}{-2}\right) \right) \\
&= -(-\infty) - \frac{1}{2} = \infty
\end{align*}
\textbf{Answer:} Diverges.

\subsection*{Solution 14}
This is a Type 2 improper integral with a discontinuity at $x=3$.
\begin{align*}
\int_{0}^{3} (3-x)^{-1/2} \,dx &= \lim_{t \to 3^-} \int_{0}^{t} (3-x)^{-1/2} \,dx \\
&= \lim_{t \to 3^-} \left[ -2(3-x)^{1/2} \right]_{0}^{t} \\
&= \lim_{t \to 3^-} \left( -2\sqrt{3-t} - (-2\sqrt{3}) \right) \\
&= 0 + 2\sqrt{3} = 2\sqrt{3}
\end{align*}
\textbf{Answer:} Convergent, value is $2\sqrt{3}$.

\subsection*{Solution 15}
This is a Type 2 improper integral with a discontinuity at $x=0$ inside the interval. We must split it.
\[ \int_{-1}^{8} x^{-1/3} \,dx = \int_{-1}^{0} x^{-1/3} \,dx + \int_{0}^{8} x^{-1/3} \,dx \]
First part:
\begin{align*}
\lim_{t \to 0^-} \int_{-1}^{t} x^{-1/3} \,dx &= \lim_{t \to 0^-} \left[ \frac{3}{2}x^{2/3} \right]_{-1}^{t} \\
&= \lim_{t \to 0^-} \left( \frac{3}{2}t^{2/3} - \frac{3}{2}(-1)^{2/3} \right) = 0 - \frac{3}{2} = -\frac{3}{2}
\end{align*}
Second part:
\begin{align*}
\lim_{t \to 0^+} \int_{t}^{8} x^{-1/3} \,dx &= \lim_{t \to 0^+} \left[ \frac{3}{2}x^{2/3} \right]_{t}^{8} \\
&= \lim_{t \to 0^+} \left( \frac{3}{2}(8)^{2/3} - \frac{3}{2}t^{2/3} \right) = \frac{3}{2}(4) - 0 = 6
\end{align*}
Both parts converge, so the total is $-\frac{3}{2} + 6 = \frac{9}{2}$.
\textbf{Answer:} Convergent, value is $9/2$.

\subsection*{Solution 16}
This is a Type 1 integral. Use partial fractions: $\frac{1}{x(x+1)} = \frac{1}{x} - \frac{1}{x+1}$.
\begin{align*}
\int_{1}^{\infty} \left(\frac{1}{x} - \frac{1}{x+1}\right) \,dx &= \lim_{t \to \infty} \int_{1}^{t} \left(\frac{1}{x} - \frac{1}{x+1}\right) \,dx \\
&= \lim_{t \to \infty} \left[ \ln|x| - \ln|x+1| \right]_{1}^{t} \\
&= \lim_{t \to \infty} \left[ \ln\left|\frac{x}{x+1}\right| \right]_{1}^{t} \\
&= \lim_{t \to \infty} \left( \ln\left(\frac{t}{t+1}\right) - \ln\left(\frac{1}{2}\right) \right) \\
&= \ln(1) - \ln(1/2) = 0 - (-\ln 2) = \ln 2
\end{align*}
\textbf{Answer:} Convergent, value is $\ln 2$.

\subsection*{Solution 17}
This is a Type 1 integral. Use partial fractions: $\frac{4}{x^2-1} = \frac{2}{x-1} - \frac{2}{x+1}$.
\begin{align*}
\int_{2}^{\infty} \left(\frac{2}{x-1} - \frac{2}{x+1}\right) \,dx &= \lim_{t \to \infty} \left[ 2\ln|x-1| - 2\ln|x+1| \right]_{2}^{t} \\
&= 2 \lim_{t \to \infty} \left[ \ln\left|\frac{x-1}{x+1}\right| \right]_{2}^{t} \\
&= 2 \lim_{t \to \infty} \left( \ln\left(\frac{t-1}{t+1}\right) - \ln\left(\frac{1}{3}\right) \right) \\
&= 2(\ln(1) - \ln(1/3)) = 2(0 - (-\ln 3)) = 2\ln 3
\end{align*}
\textbf{Answer:} Convergent, value is $2\ln 3$.

\subsection*{Solution 18}
This is a Type 1 integral. Use integration by parts with $u=x, dv=e^{-x}dx$. Then $du=dx, v=-e^{-x}$.
\begin{align*}
\int_{0}^{\infty} x e^{-x} \,dx &= \lim_{t \to \infty} \int_{0}^{t} x e^{-x} \,dx \\
&= \lim_{t \to \infty} \left( \left[-xe^{-x}\right]_{0}^{t} - \int_{0}^{t} -e^{-x} dx \right) \\
&= \lim_{t \to \infty} \left( [-xe^{-x} - e^{-x}]_{0}^{t} \right) \\
&= \lim_{t \to \infty} \left( (-\frac{t}{e^t} - \frac{1}{e^t}) - (0 - e^0) \right) \\
&= (0 - 0) - (-1) = 1
\end{align*}
(Used L'Hôpital's Rule for $\lim_{t \to \infty} t/e^t = 0$).
\textbf{Answer:} Convergent, value is $1$.

\subsection*{Solution 19}
This is a Type 1 integral. Use integration by parts with $u=\ln x, dv=x^{-2}dx$. Then $du=1/x dx, v=-x^{-1}$.
\begin{align*}
\int_{1}^{\infty} \frac{\ln x}{x^2} \,dx &= \lim_{t \to \infty} \int_{1}^{t} (\ln x)(x^{-2}) \,dx \\
&= \lim_{t \to \infty} \left( \left[-\frac{\ln x}{x}\right]_{1}^{t} - \int_{1}^{t} -\frac{1}{x^2} dx \right) \\
&= \lim_{t \to \infty} \left( [-\frac{\ln x}{x} - \frac{1}{x}]_{1}^{t} \right) \\
&= \lim_{t \to \infty} \left( (-\frac{\ln t}{t} - \frac{1}{t}) - (-\frac{\ln 1}{1} - \frac{1}{1}) \right) \\
&= (0 - 0) - (0 - 1) = 1
\end{align*}
(Used L'Hôpital's Rule for $\lim_{t \to \infty} \ln t / t = 0$).
\textbf{Answer:} Convergent, value is $1$.

\subsection*{Solution 20}
This is a Type 1 integral with an oscillating function.
\begin{align*}
\int_{0}^{\infty} \cos(x) \,dx &= \lim_{t \to \infty} \int_{0}^{t} \cos(x) \,dx \\
&= \lim_{t \to \infty} [\sin(x)]_{0}^{t} \\
&= \lim_{t \to \infty} (\sin(t) - \sin(0)) = \lim_{t \to \infty} \sin(t)
\end{align*}
The limit does not exist as $\sin(t)$ oscillates between -1 and 1.
\textbf{Answer:} Diverges.

\subsection*{Solution 21}
This is a Type 1 integral. Use the power-reducing identity $\cos^2(x) = \frac{1+\cos(2x)}{2}$.
\begin{align*}
\int_{0}^{\infty} 2\left(\frac{1+\cos(2x)}{2}\right) \,dx &= \int_{0}^{\infty} (1+\cos(2x)) \,dx \\
&= \lim_{t \to \infty} \int_{0}^{t} (1+\cos(2x)) \,dx \\
&= \lim_{t \to \infty} \left[ x + \frac{1}{2}\sin(2x) \right]_{0}^{t} \\
&= \lim_{t \to \infty} \left( (t + \frac{1}{2}\sin(2t)) - 0 \right) = \infty
\end{align*}
The limit is infinite.
\textbf{Answer:} Diverges.

\subsection*{Solution 22}
This is a Type 1 integral. Use u-substitution with $u=-\sqrt{x}, du = -\frac{1}{2\sqrt{x}}dx$.
\begin{align*}
\int_{1}^{\infty} \frac{e^{-\sqrt{x}}}{\sqrt{x}} \,dx &= \lim_{t \to \infty} \int_{1}^{t} \frac{e^{-\sqrt{x}}}{\sqrt{x}} \,dx \\
&= \lim_{t \to \infty} \left[ -2e^{-\sqrt{x}} \right]_{1}^{t} \\
&= \lim_{t \to \infty} \left( -2e^{-\sqrt{t}} - (-2e^{-1}) \right) \\
&= 0 + \frac{2}{e} = \frac{2}{e}
\end{align*}
\textbf{Answer:} Convergent, value is $2/e$.

\subsection*{Solution 23}
This is a Type 1 integral. It is the same integral as problem 18, but over a different interval. Use integration by parts with $u=x, dv=e^x dx$.
\begin{align*}
\int_{-\infty}^{0} xe^x \,dx &= \lim_{t \to -\infty} \int_{t}^{0} xe^x \,dx \\
&= \lim_{t \to -\infty} [xe^x - e^x]_{t}^{0} \\
&= \lim_{t \to -\infty} \left( (0-e^0) - (te^t - e^t) \right) \\
&= -1 - (0-0) = -1
\end{align*}
(Used L'Hôpital's Rule for $\lim_{t \to -\infty} t e^t = \lim_{t \to -\infty} t/e^{-t} = 0$).
\textbf{Answer:} Convergent, value is $-1$.

\subsection*{Solution 24}
This is a Type 2 integral with a discontinuity at $y=1/4$, which is inside $[0,1]$. Must split.
\[ \int_{0}^{1/4} \frac{1}{4y-1} \,dy + \int_{1/4}^{1} \frac{1}{4y-1} \,dy \]
Let's evaluate the first part.
\begin{align*}
\lim_{t \to 1/4^-} \int_{0}^{t} \frac{1}{4y-1} \,dy &= \lim_{t \to 1/4^-} \left[ \frac{1}{4}\ln|4y-1| \right]_{0}^{t} \\
&= \frac{1}{4} \lim_{t \to 1/4^-} (\ln|4t-1| - \ln|-1|) \\
&= \frac{1}{4} (-\infty - 0) = -\infty
\end{align*}
Since one part diverges, the whole integral diverges.
\textbf{Answer:} Diverges.

\subsection*{Solution 25}
This is a Type 1 integral. Use u-substitution with $u=\arctan(x), du = \frac{1}{1+x^2}dx$. When $x=1, u=\pi/4$. When $x \to \infty, u \to \pi/2$.
\begin{align*}
\int_{1}^{\infty} \frac{\arctan(x)}{x^2+1} \,dx &= \int_{\pi/4}^{\pi/2} u \,du \\
&= \left[ \frac{u^2}{2} \right]_{\pi/4}^{\pi/2} \\
&= \frac{1}{2} \left( \left(\frac{\pi}{2}\right)^2 - \left(\frac{\pi}{4}\right)^2 \right) \\
&= \frac{1}{2} \left( \frac{\pi^2}{4} - \frac{\pi^2}{16} \right) = \frac{1}{2} \left( \frac{3\pi^2}{16} \right) = \frac{3\pi^2}{32}
\end{align*}
\textbf{Answer:} Convergent, value is $3\pi^2/32$.

\subsection*{Solution 26}
This is a Type 2 integral since $\tan(x)$ has a vertical asymptote at $x=\pi/2$.
\begin{align*}
\int_{0}^{\pi/2} \tan(x) \,dx &= \lim_{t \to \pi/2^-} \int_{0}^{t} \tan(x) \,dx \\
&= \lim_{t \to \pi/2^-} [-\ln|\cos(x)|]_{0}^{t} \\
&= \lim_{t \to \pi/2^-} (-\ln|\cos(t)| - (-\ln|\cos(0)|)) \\
&= -(-\infty) + \ln(1) = \infty
\end{align*}
\textbf{Answer:} Diverges.

\subsection*{Solution 27}
This is a Type 1 integral over $(-\infty, \infty)$. Let $u=e^x, du=e^x dx$. When $x \to -\infty, u \to 0$. When $x \to \infty, u \to \infty$.
\begin{align*}
\int_{-\infty}^{\infty} \frac{e^x}{1+(e^x)^2} \,dx &= \int_{0}^{\infty} \frac{1}{1+u^2} \,du \\
&= \lim_{t \to \infty} \int_{0}^{t} \frac{1}{1+u^2} \,du \\
&= \lim_{t \to \infty} [\arctan(u)]_{0}^{t} \\
&= \lim_{t \to \infty} (\arctan(t) - \arctan(0)) = \frac{\pi}{2} - 0 = \frac{\pi}{2}
\end{align*}
\textbf{Answer:} Convergent, value is $\pi/2$.

\subsection*{Solution 28}
This is a Type 2 integral with a discontinuity at $x=0$. Use integration by parts with $u=\ln x, dv=dx$. Then $du=1/x dx, v=x$.
\begin{align*}
\int_{0}^{1} \ln(x) \,dx &= \lim_{t \to 0^+} \int_{t}^{1} \ln(x) \,dx \\
&= \lim_{t \to 0^+} \left( [x\ln x]_{t}^{1} - \int_{t}^{1} 1 \,dx \right) \\
&= \lim_{t \to 0^+} [x\ln x - x]_{t}^{1} \\
&= \lim_{t \to 0^+} ((1\ln 1 - 1) - (t\ln t - t)) \\
&= (0 - 1) - (0 - 0) = -1
\end{align*}
(Used L'Hôpital's Rule for $\lim_{t \to 0^+} t \ln t = \lim_{t \to 0^+} \frac{\ln t}{1/t} = 0$).
\textbf{Answer:} Convergent, value is $-1$.

\subsection*{Solution 29}
This is a Type 2 integral with a discontinuity at $x=1$. The antiderivative of the integrand is $\text{arcsec}(x)$.
\begin{align*}
\int_{1}^{\infty} \frac{1}{x\sqrt{x^2-1}} \,dx &= \text{We must split this integral, for example at } x=2. \\
&= \int_{1}^{2} \frac{1}{x\sqrt{x^2-1}} \,dx + \int_{2}^{\infty} \frac{1}{x\sqrt{x^2-1}} \,dx \\
\text{First part: } \lim_{t \to 1^+} \int_{t}^{2} \frac{1}{x\sqrt{x^2-1}} \,dx &= \lim_{t \to 1^+} [\text{arcsec}(x)]_{t}^{2} \\
&= \text{arcsec}(2) - \lim_{t \to 1^+} \text{arcsec}(t) = \frac{\pi}{3} - 0 = \frac{\pi}{3} \\
\text{Second part: } \lim_{t \to \infty} \int_{2}^{t} \frac{1}{x\sqrt{x^2-1}} \,dx &= \lim_{t \to \infty} [\text{arcsec}(x)]_{2}^{t} \\
&= \lim_{t \to \infty} \text{arcsec}(t) - \text{arcsec}(2) = \frac{\pi}{2} - \frac{\pi}{3} = \frac{\pi}{6}
\end{align*}
Both parts converge. Total value is $\frac{\pi}{3} + \frac{\pi}{6} = \frac{\pi}{2}$.
\textbf{Answer:} Convergent, value is $\pi/2$.

\subsection*{Solution 30}
This is a Type 1 integral. Complete the square for the denominator: $x^2-4x+5 = (x^2-4x+4)+1 = (x-2)^2+1$.
\begin{align*}
\int_{-\infty}^{1} \frac{1}{(x-2)^2+1} \,dx &= \lim_{t \to -\infty} \int_{t}^{1} \frac{1}{(x-2)^2+1} \,dx \\
&= \lim_{t \to -\infty} [\arctan(x-2)]_{t}^{1} \\
&= \lim_{t \to -\infty} (\arctan(1-2) - \arctan(t-2)) \\
&= \arctan(-1) - (-\pi/2) = -\frac{\pi}{4} + \frac{\pi}{2} = \frac{\pi}{4}
\end{align*}
\textbf{Answer:} Convergent, value is $\pi/4$.

\newpage

\section*{Concept and Problem Number Index}
Here is a list of the concepts tested and the corresponding problem numbers.

\begin{itemize}
    \item \textbf{Type 1 Integrals, upper limit $\infty$}: 1, 2, 3, 4, 5, 16, 17, 18, 19, 21, 22, 25
    \item \textbf{Type 1 Integrals, lower limit $-\infty$}: 6, 7, 8, 23, 30
    \item \textbf{Type 1 Integrals, on $(-\infty, \infty)$}: 9, 10, 11, 27
    \item \textbf{Type 2 Integrals, discontinuity at endpoint}: 12, 13, 14, 26, 28
    \item \textbf{Type 2 Integrals, discontinuity inside interval}: 15, 24
    \item \textbf{Mixed Type 1 and Type 2}: 29
    \item \textbf{p-Test (Direct or after substitution)}:
    	\begin{itemize}
    		\item Convergent ($p > 1$): 1, 4, 7, 8
    		\item Divergent ($p \le 1$): 2, 5, 9
    	\end{itemize}
    \item \textbf{u-Substitution}: 4, 6, 8, 11, 22, 25, 27, 30
    \item \textbf{Integration by Parts}: 18, 19, 23, 28
    \item \textbf{Partial Fraction Decomposition}: 16, 17
    \item \textbf{Trigonometric Functions/Identities}: 21 (Power-reducing), 26 (tan(x))
    \item \textbf{Oscillating Functions (leading to divergence)}: 20, 21
    \item \textbf{Symmetry (Odd/Even Functions)}: 9 (Odd, but diverges), 10 (Even)
    \item \textbf{Algebraic Simplification}: 5
    \item \textbf{Logarithm Properties for Limits}: 16, 17
    \item \textbf{Integrals involving $\ln(x)$}: 4, 19, 28
    \item \textbf{Integrals involving $e^x$}: 3, 11, 18, 22, 23, 27
    \item \textbf{Integrals involving inverse trig functions}: 10, 25, 29, 30
\end{itemize}

\end{document}