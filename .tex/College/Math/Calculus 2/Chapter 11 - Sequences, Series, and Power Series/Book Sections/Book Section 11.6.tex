\documentclass{article}
\usepackage{amsmath}
\usepackage{amssymb}
\usepackage[margin=1in]{geometry}

\title{Homework 11.6: The Ratio and Root Tests}
\author{Tashfeen Omran}
\date{\today}

\begin{document}

\maketitle

\part{Comprehensive Introduction, Context, and Prerequisites}

\section{Core Concepts}
The Ratio and Root Tests are powerful tools for determining the convergence of an infinite series, specifically for determining \textit{absolute convergence}. Unlike the Comparison Tests or the Integral Test, they do not require an external series or function for comparison; they test the series against itself by examining the behavior of its own terms.

The central idea behind both tests is to compare the given series $\sum a_n$ to a geometric series $\sum r^n$. A geometric series converges if its ratio $|r|$ is less than 1. The Ratio and Root Tests essentially check if the terms of $\sum a_n$ are eventually growing or shrinking at a rate similar to that of a geometric series.

\subsection{The Ratio Test}
This test focuses on the ratio of consecutive terms. Let $\sum a_n$ be an infinite series. We compute the limit:
\[ L = \lim_{n \to \infty} \left| \frac{a_{n+1}}{a_n} \right| \]
There are three possible outcomes:
\begin{enumerate}
    \item If $L < 1$, the series $\sum a_n$ is \textbf{absolutely convergent} (and therefore convergent).
    \item If $L > 1$ or the limit is $\infty$, the series $\sum a_n$ is \textbf{divergent}.
    \item If $L = 1$, the Ratio Test is \textbf{inconclusive}. The series could be absolutely convergent, conditionally convergent, or divergent. Another test must be used.
\end{enumerate}

\subsection{The Root Test}
This test focuses on the $n$-th root of the magnitude of the $n$-th term. Let $\sum a_n$ be an infinite series. We compute the limit:
\[ L = \lim_{n \to \infty} \sqrt[n]{|a_n|} = \lim_{n \to \infty} |a_n|^{1/n} \]
The outcomes are identical to the Ratio Test:
\begin{enumerate}
    \item If $L < 1$, the series $\sum a_n$ is \textbf{absolutely convergent}.
    \item If $L > 1$ or the limit is $\infty$, the series $\sum a_n$ is \textbf{divergent}.
    \item If $L = 1$, the Root Test is \textbf{inconclusive}.
\end{enumerate}

\section{Intuition and Derivation}
The proofs for both tests formalize the comparison to a geometric series.
\begin{itemize}
    \item \textbf{Ratio Test Intuition:} If $\lim_{n \to \infty} |a_{n+1}/a_n| = L < 1$, it means that for very large $n$, $|a_{n+1}| \approx L \cdot |a_n|$. This is the defining property of a geometric series with ratio $L$. Since $L<1$, the terms are shrinking fast enough for the series to converge. The formal proof (as outlined in the textbook) picks a number $r$ such that $L < r < 1$ and shows that for $n$ beyond some point $N$, we have $|a_{n+1}| < r|a_n|$. By induction, this shows $|a_{N+k}| < |a_N|r^k$. The tail end of the series $\sum |a_n|$ is therefore smaller than a convergent geometric series $|a_N|\sum r^k$, so it converges by the Direct Comparison Test.
    \item \textbf{Root Test Intuition:} If $\lim_{n \to \infty} \sqrt[n]{|a_n|} = L < 1$, then for very large $n$, we have $\sqrt[n]{|a_n|} \approx L$, which implies $|a_n| \approx L^n$. This means the series $\sum |a_n|$ behaves like the convergent geometric series $\sum L^n$. Therefore, $\sum |a_n|$ converges by comparison.
\end{itemize}

\section{Historical Context and Motivation}
The rigorous study of infinite series blossomed in the 19th century as mathematicians sought to place calculus on a firm logical foundation. Before this, manipulations of series were often done formally, without a full understanding of when they were valid, sometimes leading to paradoxes.

The Ratio Test is often attributed to the French mathematician Jean le Rond d'Alembert and is sometimes called d'Alembert's test. The Root Test was developed by the French mathematician Augustin-Louis Cauchy. These tests were a major advancement because they were more "algorithmic" than previous tests. Tests like the Comparison Test or Integral Test require the ingenuity of finding a suitable series or function to compare against. The Ratio and Root tests, however, only require calculating a limit involving the terms of the series itself. This was a crucial step towards developing systematic methods for analyzing series, especially power series, which would become the cornerstone of complex analysis and the series-based solutions to differential equations.

\section{Key Formulas}
\begin{itemize}
    \item \textbf{Ratio Test Limit:} $L = \lim_{n \to \infty} \left| \frac{a_{n+1}}{a_n} \right|$
    \item \textbf{Root Test Limit:} $L = \lim_{n \to \infty} \sqrt[n]{|a_n|}$
    \item \textbf{Conditions for both tests:}
        \begin{itemize}
            \item $L < 1 \implies$ Absolutely Convergent
            \item $L > 1 \implies$ Divergent
            \item $L = 1 \implies$ Inconclusive
        \end{itemize}
\end{itemize}

\section{Prerequisites}
To master the Ratio and Root Tests, you must be proficient in the following areas:
\begin{itemize}
    \item \textbf{Algebra:} Manipulation of fractions, exponent rules, and especially factorials (e.g., simplifying $\frac{(n+1)!}{n!}$ or $\frac{(2n)!}{(2(n+1))!}$).
    \item \textbf{Calculus I (Limits):} Evaluating limits at infinity, especially of rational functions. Knowledge of the special limit $\lim_{n \to \infty} (1 + \frac{k}{n})^n = e^k$ is essential.
    \item \textbf{Series Fundamentals:} A clear understanding of what a series is, the difference between convergence and divergence, and the distinction between absolute and conditional convergence. You should also be familiar with the Test for Divergence, p-series, and geometric series.
\end{itemize}

\part{Detailed Homework Solutions}
\section{Solutions for Exercises (Pages 778-779)}

\subsection*{1. What can you say about the series $\sum a_n$ in each of the following cases?}
\begin{description}
    \item[(a) $\lim_{n \to \infty} |\frac{a_{n+1}}{a_n}| = 8$]
    Since the limit $L = 8$ is greater than 1, the series $\sum a_n$ is \textbf{divergent} by the Ratio Test.

    \item[(b) $\lim_{n \to \infty} |\frac{a_{n+1}}{a_n}| = 0.8$]
    Since the limit $L = 0.8$ is less than 1, the series $\sum a_n$ is \textbf{absolutely convergent} (and therefore convergent) by the Ratio Test.

    \item[(c) $\lim_{n \to \infty} |\frac{a_{n+1}}{a_n}| = 1$]
    Since the limit $L = 1$, the Ratio Test is \textbf{inconclusive}. The series could converge or diverge.
\end{description}

\subsection*{2. Suppose that for the series $\sum a_n$ we have $\lim_{n \to \infty} |\frac{a_n}{a_{n+1}}| = 2$.}
First, we find the limit needed for the Ratio Test:
\[ \lim_{n \to \infty} \left| \frac{a_{n+1}}{a_n} \right| = \lim_{n \to \infty} \frac{1}{|a_n/a_{n+1}|} = \frac{1}{\lim_{n \to \infty} |a_n/a_{n+1}|} = \frac{1}{2} \]
Since the limit $L = 1/2$ is less than 1, the series $\sum a_n$ is \textbf{absolutely convergent} by the Ratio Test.

\subsection*{3. $\sum_{n=1}^{\infty} \frac{n}{5^n}$}
Let $a_n = \frac{n}{5^n}$. We apply the Ratio Test:
\[ L = \lim_{n \to \infty} \left| \frac{a_{n+1}}{a_n} \right| = \lim_{n \to \infty} \left| \frac{(n+1)/5^{n+1}}{n/5^n} \right| = \lim_{n \to \infty} \frac{n+1}{5^{n+1}} \cdot \frac{5^n}{n} = \lim_{n \to \infty} \frac{n+1}{n} \cdot \frac{5^n}{5^{n+1}} = \lim_{n \to \infty} \left(1 + \frac{1}{n}\right) \cdot \frac{1}{5} \]
\[ L = (1) \cdot \frac{1}{5} = \frac{1}{5} \]
Since $L = 1/5 < 1$, the series is \textbf{absolutely convergent}.

\subsection*{4. $\sum_{n=1}^{\infty} \frac{(-2)^n}{n^2}$}
Let $a_n = \frac{(-2)^n}{n^2}$. We apply the Ratio Test:
\[ L = \lim_{n \to \infty} \left| \frac{a_{n+1}}{a_n} \right| = \lim_{n \to \infty} \left| \frac{(-2)^{n+1}/(n+1)^2}{(-2)^n/n^2} \right| = \lim_{n \to \infty} \frac{2^{n+1}}{(n+1)^2} \cdot \frac{n^2}{2^n} = \lim_{n \to \infty} 2 \cdot \left(\frac{n}{n+1}\right)^2 \]
\[ L = 2 \cdot \left(\lim_{n \to \infty} \frac{n}{n+1}\right)^2 = 2 \cdot (1)^2 = 2 \]
Since $L = 2 > 1$, the series is \textbf{divergent}.

\subsection*{5. $\sum_{n=1}^{\infty} (-1)^{n-1} \frac{3^n}{2^n n^3}$}
Let $a_n = (-1)^{n-1} \frac{3^n}{2^n n^3}$. We apply the Ratio Test:
\[ L = \lim_{n \to \infty} \left| \frac{a_{n+1}}{a_n} \right| = \lim_{n \to \infty} \left| \frac{3^{n+1}}{2^{n+1}(n+1)^3} \cdot \frac{2^n n^3}{3^n} \right| = \lim_{n \to \infty} \frac{3}{2} \cdot \left(\frac{n}{n+1}\right)^3 = \frac{3}{2} \cdot (1)^3 = \frac{3}{2} \]
Since $L = 3/2 > 1$, the series is \textbf{divergent}.

\subsection*{6. $\sum_{n=0}^{\infty} \frac{(-3)^n}{(2n+1)!}$}
Let $a_n = \frac{(-3)^n}{(2n+1)!}$. We apply the Ratio Test. Note $a_{n+1}$ has $(2(n+1)+1)! = (2n+3)!$.
\begin{align*}
L &= \lim_{n \to \infty} \left| \frac{a_{n+1}}{a_n} \right| = \lim_{n \to \infty} \left| \frac{(-3)^{n+1}/(2n+3)!}{(-3)^n/(2n+1)!} \right| = \lim_{n \to \infty} \frac{3^{n+1}}{(2n+3)!} \cdot \frac{(2n+1)!}{3^n} \\
&= \lim_{n \to \infty} 3 \cdot \frac{(2n+1)!}{(2n+3)(2n+2)(2n+1)!} = \lim_{n \to \infty} \frac{3}{(2n+3)(2n+2)} = 0
\end{align*}
Since $L = 0 < 1$, the series is \textbf{absolutely convergent}.

\subsection*{7. $\sum_{k=1}^{\infty} \frac{1}{k!}$}
Let $a_k = \frac{1}{k!}$. We apply the Ratio Test:
\[ L = \lim_{k \to \infty} \left| \frac{a_{k+1}}{a_k} \right| = \lim_{k \to \infty} \left| \frac{1/(k+1)!}{1/k!} \right| = \lim_{k \to \infty} \frac{k!}{(k+1)!} = \lim_{k \to \infty} \frac{k!}{(k+1)k!} = \lim_{k \to \infty} \frac{1}{k+1} = 0 \]
Since $L = 0 < 1$, the series is \textbf{convergent}.

\subsection*{8. $\sum_{k=1}^{\infty} ke^{-k}$}
Let $a_k = \frac{k}{e^k}$. We apply the Ratio Test:
\[ L = \lim_{k \to \infty} \left| \frac{a_{k+1}}{a_k} \right| = \lim_{k \to \infty} \left| \frac{(k+1)/e^{k+1}}{k/e^k} \right| = \lim_{k \to \infty} \frac{k+1}{e^{k+1}} \cdot \frac{e^k}{k} = \lim_{k \to \infty} \frac{k+1}{k} \cdot \frac{1}{e} = 1 \cdot \frac{1}{e} = \frac{1}{e} \]
Since $L = 1/e < 1$, the series is \textbf{convergent}.

\subsection*{9. $\sum_{n=1}^{\infty} \frac{10^n}{(n+1)4^{2n+1}}$}
Let $a_n = \frac{10^n}{(n+1)4^{2n+1}}$. We apply the Ratio Test:
\begin{align*}
L &= \lim_{n \to \infty} \left| \frac{a_{n+1}}{a_n} \right| = \lim_{n \to \infty} \left| \frac{10^{n+1}}{(n+2)4^{2(n+1)+1}} \cdot \frac{(n+1)4^{2n+1}}{10^n} \right| \\
&= \lim_{n \to \infty} \frac{10}{1} \cdot \frac{n+1}{n+2} \cdot \frac{4^{2n+1}}{4^{2n+3}} = \lim_{n \to \infty} 10 \cdot \frac{n+1}{n+2} \cdot \frac{1}{4^2} \\
&= \frac{10}{16} \lim_{n \to \infty} \frac{n+1}{n+2} = \frac{10}{16} \cdot 1 = \frac{5}{8}
\end{align*}
Since $L = 5/8 < 1$, the series is \textbf{convergent}.

\subsection*{10. $\sum_{n=1}^{\infty} \frac{n!}{100^n}$}
Let $a_n = \frac{n!}{100^n}$. We apply the Ratio Test:
\[ L = \lim_{n \to \infty} \left| \frac{a_{n+1}}{a_n} \right| = \lim_{n \to \infty} \left| \frac{(n+1)!/100^{n+1}}{n!/100^n} \right| = \lim_{n \to \infty} \frac{(n+1)!}{100^{n+1}} \cdot \frac{100^n}{n!} = \lim_{n \to \infty} \frac{n+1}{100} = \infty \]
Since $L = \infty > 1$, the series is \textbf{divergent}.

\subsection*{11. $\sum_{n=1}^{\infty} \frac{n\pi^n}{(-3)^{n-1}}$}
Let $a_n = \frac{n\pi^n}{(-3)^{n-1}}$. We test for absolute convergence.
\[ L = \lim_{n \to \infty} \left| \frac{a_{n+1}}{a_n} \right| = \lim_{n \to \infty} \left| \frac{(n+1)\pi^{n+1}}{(-3)^n} \cdot \frac{(-3)^{n-1}}{n\pi^n} \right| = \lim_{n \to \infty} \frac{n+1}{n} \cdot \frac{\pi}{|-3|} = 1 \cdot \frac{\pi}{3} = \frac{\pi}{3} \]
Since $\pi \approx 3.14$, $L = \pi/3 > 1$. The series is \textbf{divergent}.

\subsection*{12. $\sum_{n=1}^{\infty} \frac{n^{10}}{(-10)^{n+1}}$}
Let $a_n = \frac{n^{10}}{(-10)^{n+1}}$. We apply the Ratio Test:
\[ L = \lim_{n \to \infty} \left| \frac{a_{n+1}}{a_n} \right| = \lim_{n \to \infty} \left| \frac{(n+1)^{10}}{(-10)^{n+2}} \cdot \frac{(-10)^{n+1}}{n^{10}} \right| = \lim_{n \to \infty} \left(\frac{n+1}{n}\right)^{10} \cdot \frac{1}{|-10|} = 1^{10} \cdot \frac{1}{10} = \frac{1}{10} \]
Since $L = 1/10 < 1$, the series is \textbf{absolutely convergent}.

\subsection*{13. $\sum_{n=1}^{\infty} \frac{\cos(n\pi/3)}{n!}$}
Let $a_n = \frac{\cos(n\pi/3)}{n!}$. We test for absolute convergence with $\sum |a_n| = \sum \frac{|\cos(n\pi/3)|}{n!}$.
We can use the Direct Comparison Test. Since $|\cos(x)| \le 1$ for all $x$, we have:
\[ \frac{|\cos(n\pi/3)|}{n!} \le \frac{1}{n!} \]
We know from problem 7 that $\sum_{n=1}^{\infty} \frac{1}{n!}$ converges. Therefore, by the Direct Comparison Test, $\sum \frac{|\cos(n\pi/3)|}{n!}$ converges.
Thus, the original series is \textbf{absolutely convergent}.

\subsection*{14. $\sum_{n=1}^{\infty} \frac{n!}{n^n}$}
Let $a_n = \frac{n!}{n^n}$. We apply the Ratio Test:
\begin{align*}
L &= \lim_{n \to \infty} \left| \frac{a_{n+1}}{a_n} \right| = \lim_{n \to \infty} \frac{(n+1)!/(n+1)^{n+1}}{n!/n^n} = \lim_{n \to \infty} \frac{(n+1)!}{(n+1)^{n+1}} \cdot \frac{n^n}{n!} \\
&= \lim_{n \to \infty} \frac{n+1}{(n+1)^{n+1}} \cdot n^n = \lim_{n \to \infty} \frac{n^n}{(n+1)^n} = \lim_{n \to \infty} \left(\frac{n}{n+1}\right)^n \\
&= \lim_{n \to \infty} \left(\frac{1}{1+1/n}\right)^n = \frac{1}{\lim_{n \to \infty}(1+1/n)^n} = \frac{1}{e}
\end{align*}
Since $L = 1/e < 1$, the series is \textbf{convergent}.

\subsection*{15. $\sum_{n=1}^{\infty} \frac{n100^n}{n!}$}
Let $a_n = \frac{n100^n}{n!}$. We apply the Ratio Test:
\begin{align*}
L &= \lim_{n \to \infty} \left| \frac{a_{n+1}}{a_n} \right| = \lim_{n \to \infty} \left| \frac{(n+1)100^{n+1}/(n+1)!}{n100^n/n!} \right| = \lim_{n \to \infty} \frac{(n+1)100^{n+1}}{(n+1)!} \cdot \frac{n!}{n100^n} \\
&= \lim_{n \to \infty} \frac{n+1}{n} \cdot 100 \cdot \frac{n!}{(n+1)!} = \lim_{n \to \infty} \frac{n+1}{n} \cdot \frac{100}{n+1} = \lim_{n \to \infty} \frac{100}{n} = 0
\end{align*}
Since $L = 0 < 1$, the series is \textbf{convergent}.

\subsection*{16. $\sum_{n=1}^{\infty} \frac{(2n)!}{(n!)^2}$}
Let $a_n = \frac{(2n)!}{(n!)^2}$. We apply the Ratio Test:
\begin{align*}
L &= \lim_{n \to \infty} \left| \frac{a_{n+1}}{a_n} \right| = \lim_{n \to \infty} \frac{(2(n+1))!/((n+1)!)^2}{(2n)!/(n!)^2} = \lim_{n \to \infty} \frac{(2n+2)!}{((n+1)!)^2} \cdot \frac{(n!)^2}{(2n)!} \\
&= \lim_{n \to \infty} \frac{(2n+2)(2n+1)(2n)!}{(n+1)^2(n!)^2} \cdot \frac{(n!)^2}{(2n)!} = \lim_{n \to \infty} \frac{(2n+2)(2n+1)}{(n+1)^2} \\
&= \lim_{n \to \infty} \frac{2(n+1)(2n+1)}{(n+1)^2} = \lim_{n \to \infty} \frac{2(2n+1)}{n+1} = \lim_{n \to \infty} \frac{4n+2}{n+1} = 4
\end{align*}
Since $L = 4 > 1$, the series is \textbf{divergent}.

\subsection*{17. $1 + \frac{2!}{1 \cdot 3} + \frac{3!}{1 \cdot 3 \cdot 5} + \frac{4!}{1 \cdot 3 \cdot 5 \cdot 7} + \dots$}
The $n$-th term is $a_n = \frac{n!}{1 \cdot 3 \cdot 5 \cdots (2n-1)}$ (for $n \ge 1$).
So, $a_{n+1} = \frac{(n+1)!}{1 \cdot 3 \cdot 5 \cdots (2n-1)(2n+1)}$.
Apply the Ratio Test:
\begin{align*}
L &= \lim_{n \to \infty} \left| \frac{a_{n+1}}{a_n} \right| = \lim_{n \to \infty} \frac{(n+1)!/(1 \cdot 3 \cdots (2n+1))}{n!/(1 \cdot 3 \cdots (2n-1))} \\
&= \lim_{n \to \infty} \frac{(n+1)!}{n!} \cdot \frac{1 \cdot 3 \cdots (2n-1)}{1 \cdot 3 \cdots (2n-1)(2n+1)} = \lim_{n \to \infty} (n+1) \cdot \frac{1}{2n+1} = \lim_{n \to \infty} \frac{n+1}{2n+1} = \frac{1}{2}
\end{align*}
Since $L = 1/2 < 1$, the series is \textbf{convergent}.

\subsection*{18. $\frac{2}{3} + \frac{2 \cdot 5}{3 \cdot 5} + \frac{2 \cdot 5 \cdot 8}{3 \cdot 5 \cdot 7} + \dots$}
The $n$-th term is $a_n = \frac{2 \cdot 5 \cdot 8 \cdots (3n-1)}{3 \cdot 5 \cdot 7 \cdots (2n+1)}$.
So, $a_{n+1} = a_n \cdot \frac{3(n+1)-1}{2(n+1)+1} = a_n \cdot \frac{3n+2}{2n+3}$.
Apply the Ratio Test:
\[ L = \lim_{n \to \infty} \left| \frac{a_{n+1}}{a_n} \right| = \lim_{n \to \infty} \frac{3n+2}{2n+3} = \frac{3}{2} \]
Since $L = 3/2 > 1$, the series is \textbf{divergent}.

\subsection*{19. $\sum_{n=1}^{\infty} \frac{2 \cdot 4 \cdot 6 \cdots (2n)}{n!}$}
The numerator is $(2 \cdot 1) \cdot (2 \cdot 2) \cdot (2 \cdot 3) \cdots (2 \cdot n) = 2^n(1 \cdot 2 \cdot 3 \cdots n) = 2^n n!$.
So, $a_n = \frac{2^n n!}{n!} = 2^n$.
The series is $\sum_{n=1}^{\infty} 2^n$. By the Test for Divergence, $\lim_{n \to \infty} 2^n = \infty \neq 0$.
The series is \textbf{divergent}.

\subsection*{20. $\sum_{n=1}^{\infty} (-1)^n \frac{2^n n!}{5 \cdot 8 \cdot 11 \cdots (3n+2)}$}
Let $a_n = (-1)^n \frac{2^n n!}{5 \cdot 8 \cdot 11 \cdots (3n+2)}$. We apply the Ratio Test for absolute convergence.
The denominator of $a_{n+1}$ will have an extra factor of $3(n+1)+2 = 3n+5$.
\begin{align*}
L &= \lim_{n \to \infty} \left| \frac{a_{n+1}}{a_n} \right| = \lim_{n \to \infty} \left| \frac{2^{n+1}(n+1)!/(5 \cdots (3n+2)(3n+5))}{2^n n!/(5 \cdots (3n+2))} \right| \\
&= \lim_{n \to \infty} \frac{2^{n+1}(n+1)!}{2^n n!} \cdot \frac{1}{3n+5} = \lim_{n \to \infty} 2(n+1) \cdot \frac{1}{3n+5} = \lim_{n \to \infty} \frac{2n+2}{3n+5} = \frac{2}{3}
\end{align*}
Since $L = 2/3 < 1$, the series is \textbf{absolutely convergent}.

\subsection*{21. $\sum_{n=1}^{\infty} \left(\frac{n^2+1}{2n^2+1}\right)^n$}
Let $a_n = \left(\frac{n^2+1}{2n^2+1}\right)^n$. We apply the Root Test:
\[ L = \lim_{n \to \infty} \sqrt[n]{|a_n|} = \lim_{n \to \infty} \left( \left(\frac{n^2+1}{2n^2+1}\right)^n \right)^{1/n} = \lim_{n \to \infty} \frac{n^2+1}{2n^2+1} = \frac{1}{2} \]
Since $L = 1/2 < 1$, the series is \textbf{absolutely convergent}.

\subsection*{22. $\sum_{n=1}^{\infty} \frac{(-2)^n}{n^n}$}
Let $a_n = \frac{(-2)^n}{n^n} = \left(\frac{-2}{n}\right)^n$. We apply the Root Test:
\[ L = \lim_{n \to \infty} \sqrt[n]{|a_n|} = \lim_{n \to \infty} \left( \left|\frac{-2}{n}\right|^n \right)^{1/n} = \lim_{n \to \infty} \left|\frac{-2}{n}\right| = \lim_{n \to \infty} \frac{2}{n} = 0 \]
Since $L = 0 < 1$, the series is \textbf{absolutely convergent}.

\subsection*{23. $\sum_{n=2}^{\infty} \frac{(-1)^{n-1}}{(\ln n)^n}$}
Let $a_n = \frac{(-1)^{n-1}}{(\ln n)^n}$. We apply the Root Test:
\[ L = \lim_{n \to \infty} \sqrt[n]{|a_n|} = \lim_{n \to \infty} \left( \frac{1}{(\ln n)^n} \right)^{1/n} = \lim_{n \to \infty} \frac{1}{\ln n} = 0 \]
Since $L = 0 < 1$, the series is \textbf{absolutely convergent}.

\subsection*{24. $\sum_{n=1}^{\infty} \left(\frac{-2n}{n+1}\right)^{5n}$}
Let $a_n = \left(\frac{-2n}{n+1}\right)^{5n}$. We apply the Root Test:
\begin{align*}
L &= \lim_{n \to \infty} \sqrt[n]{|a_n|} = \lim_{n \to \infty} \left( \left|\left(\frac{-2n}{n+1}\right)^{5n}\right| \right)^{1/n} = \lim_{n \to \infty} \left( \left(\frac{2n}{n+1}\right)^{5n} \right)^{1/n} \\
&= \lim_{n \to \infty} \left(\frac{2n}{n+1}\right)^5 = \left(\lim_{n \to \infty} \frac{2n}{n+1}\right)^5 = 2^5 = 32
\end{align*}
Since $L = 32 > 1$, the series is \textbf{divergent}.

\subsection*{25. $\sum_{n=1}^{\infty} \left(1 + \frac{1}{n}\right)^{n^2}$}
(Assuming the sum starts at $n=1$). Let $a_n = \left(1 + \frac{1}{n}\right)^{n^2}$. We apply the Root Test:
\[ L = \lim_{n \to \infty} \sqrt[n]{|a_n|} = \lim_{n \to \infty} \left( \left(1 + \frac{1}{n}\right)^{n^2} \right)^{1/n} = \lim_{n \to \infty} \left(1 + \frac{1}{n}\right)^n = e \]
Since $L = e > 1$, the series is \textbf{divergent}.

\subsection*{26. $\sum_{n=0}^{\infty} (\arctan n)^n$}
Let $a_n = (\arctan n)^n$. We apply the Root Test:
\[ L = \lim_{n \to \infty} \sqrt[n]{|a_n|} = \lim_{n \to \infty} \arctan n = \frac{\pi}{2} \]
Since $L = \pi/2 \approx 1.57 > 1$, the series is \textbf{divergent}.

\subsection*{27. $\sum_{n=2}^{\infty} \frac{(-1)^n \ln n}{n}$}
This is an alternating series. First, check for absolute convergence with $\sum_{n=2}^{\infty} \frac{\ln n}{n}$. Using the Integral Test, $\int_2^\infty \frac{\ln x}{x} dx = [\frac{1}{2}(\ln x)^2]_2^\infty = \infty$. The series does not converge absolutely.
Now, check for conditional convergence with the Alternating Series Test. Let $b_n = \frac{\ln n}{n}$.
1. $\lim_{n \to \infty} b_n = \lim_{n \to \infty} \frac{\ln n}{n} \overset{L'H}{=} \lim_{n \to \infty} \frac{1/n}{1} = 0$.
2. $f(x) = \frac{\ln x}{x} \implies f'(x) = \frac{1-\ln x}{x^2} < 0$ for $x > e$. So $b_n$ is eventually decreasing.
Both conditions are met, so the series is \textbf{conditionally convergent}.

\subsection*{28. $\sum_{n=1}^{\infty} \left(\frac{1-n}{2+3n}\right)^n$}
Let $a_n = \left(\frac{1-n}{2+3n}\right)^n$. Use the Root Test.
\[ L = \lim_{n \to \infty} \sqrt[n]{|a_n|} = \lim_{n \to \infty} \left| \frac{1-n}{2+3n} \right| = \lim_{n \to \infty} \frac{n-1}{3n+2} = \frac{1}{3} \]
Since $L = 1/3 < 1$, the series is \textbf{absolutely convergent}.

\subsection*{29. $\sum_{n=1}^{\infty} \frac{(-9)^n}{n 10^{n+1}}$}
Let $a_n = \frac{(-9)^n}{n 10^{n+1}}$. Use the Ratio Test.
\begin{align*}
L &= \lim_{n \to \infty} \left| \frac{a_{n+1}}{a_n} \right| = \lim_{n \to \infty} \left| \frac{(-9)^{n+1}}{(n+1)10^{n+2}} \cdot \frac{n 10^{n+1}}{(-9)^n} \right| \\
&= \lim_{n \to \infty} \frac{9}{1} \cdot \frac{n}{n+1} \cdot \frac{1}{10} = \frac{9}{10}
\end{align*}
Since $L = 9/10 < 1$, the series is \textbf{absolutely convergent}.

\subsection*{30. $\sum_{n=1}^{\infty} \frac{n^5 2^n}{10^{n+1}}$}
Let $a_n = \frac{n^5 2^n}{10^{n+1}} = \frac{n^5}{10} \left(\frac{1}{5}\right)^n$. Use the Ratio Test.
\[ L = \lim_{n \to \infty} \left| \frac{(n+1)^5/ (10 \cdot 5^{n+1})}{n^5 / (10 \cdot 5^n)} \right| = \lim_{n \to \infty} \left(\frac{n+1}{n}\right)^5 \cdot \frac{1}{5} = 1^5 \cdot \frac{1}{5} = \frac{1}{5} \]
Since $L = 1/5 < 1$, the series is \textbf{absolutely convergent}.

\subsection*{31. $\sum_{n=2}^{\infty} \frac{n}{\ln n}$}
Use the Test for Divergence:
\[ \lim_{n \to \infty} a_n = \lim_{n \to \infty} \frac{n}{\ln n} \overset{L'H}{=} \lim_{n \to \infty} \frac{1}{1/n} = \lim_{n \to \infty} n = \infty \]
Since the limit of the terms is not zero, the series is \textbf{divergent}.

\subsection*{32. $\sum_{n=1}^{\infty} \frac{\sin(n\pi/6)}{1+n\sqrt{n}}$}
We test for absolute convergence: $\sum_{n=1}^{\infty} \frac{|\sin(n\pi/6)|}{1+n\sqrt{n}}$.
Since $|\sin(x)| \le 1$, we have $\frac{|\sin(n\pi/6)|}{1+n^{3/2}} \le \frac{1}{1+n^{3/2}} < \frac{1}{n^{3/2}}$.
The series $\sum \frac{1}{n^{3/2}}$ is a convergent p-series ($p=3/2 > 1$).
By the Direct Comparison Test, the series of absolute values converges.
Therefore, the original series is \textbf{absolutely convergent}.

\subsection*{33. $\sum_{n=1}^{\infty} \frac{(-1)^n \arctan n}{n^2}$}
We test for absolute convergence: $\sum_{n=1}^{\infty} \frac{|\arctan n|}{n^2}$.
As $n \to \infty$, $\arctan n \to \pi/2$. So $|\arctan n| \le \pi/2$ for all $n \ge 1$.
\[ \frac{|\arctan n|}{n^2} \le \frac{\pi/2}{n^2} \]
The series $\frac{\pi}{2} \sum \frac{1}{n^2}$ is a constant multiple of a convergent p-series ($p=2>1$).
By the Direct Comparison Test, the series of absolute values converges.
Therefore, the original series is \textbf{absolutely convergent}.

\subsection*{34. $\sum_{n=2}^{\infty} \frac{(-1)^n}{\sqrt{n}\ln n}$}
This is an alternating series. Test absolute convergence: $\sum_{n=2}^{\infty} \frac{1}{\sqrt{n}\ln n}$.
Compare to the divergent p-series $\sum \frac{1}{n}$. We know $\ln n < \sqrt{n}$ for large $n$. Thus $\sqrt{n}\ln n < \sqrt{n}\sqrt{n} = n$. This implies $\frac{1}{\sqrt{n}\ln n} > \frac{1}{n}$.
Since $\sum \frac{1}{n}$ diverges, by Direct Comparison, $\sum \frac{1}{\sqrt{n}\ln n}$ diverges. The series is not absolutely convergent.
Check for conditional convergence with AST. Let $b_n = \frac{1}{\sqrt{n}\ln n}$.
1. $\lim_{n \to \infty} \frac{1}{\sqrt{n}\ln n} = 0$.
2. The denominator $\sqrt{n}\ln n$ is increasing for $n \ge 1$, so $b_n$ is decreasing.
Both conditions are met, so the series is \textbf{conditionally convergent}.

\subsection*{35. $a_1=2, a_{n+1}=\frac{5n+1}{4n+3}a_n$. Determine whether $\sum a_n$ converges or diverges.}
The recurrence relation gives the ratio of consecutive terms. All terms are positive.
\[ \frac{a_{n+1}}{a_n} = \frac{5n+1}{4n+3} \]
Apply the Ratio Test:
\[ L = \lim_{n \to \infty} \frac{a_{n+1}}{a_n} = \lim_{n \to \infty} \frac{5n+1}{4n+3} = \frac{5}{4} \]
Since $L = 5/4 > 1$, the series is \textbf{divergent}.

\subsection*{36. $a_1=1, a_{n+1}=\frac{2+\cos n}{\sqrt{n}}a_n$. Determine whether $\sum a_n$ converges or diverges.}
The ratio is $\frac{a_{n+1}}{a_n} = \frac{2+\cos n}{\sqrt{n}}$. All terms are positive.
Since $-1 \le \cos n \le 1$, we have $1 \le 2+\cos n \le 3$.
So, $\frac{1}{\sqrt{n}} \le \frac{a_{n+1}}{a_n} \le \frac{3}{\sqrt{n}}$.
By the Squeeze Theorem, since both $\lim \frac{1}{\sqrt{n}} = 0$ and $\lim \frac{3}{\sqrt{n}} = 0$, we have:
\[ L = \lim_{n \to \infty} \frac{a_{n+1}}{a_n} = 0 \]
Since $L = 0 < 1$, the series is \textbf{convergent} by the Ratio Test.

\subsection*{37. $\sum_{n=1}^\infty \frac{b_n^n \cos(n\pi)}{n}$, where $\{b_n\} \to 1/2$ and $b_n > 0$.}
Let $a_n = \frac{b_n^n (-1)^n}{n}$. Use the Root Test for absolute convergence.
\[ L = \lim_{n \to \infty} \sqrt[n]{|a_n|} = \lim_{n \to \infty} \left( \frac{b_n^n}{n} \right)^{1/n} = \lim_{n \to \infty} \frac{b_n}{n^{1/n}} \]
Since $\lim_{n \to \infty} b_n = 1/2$ and $\lim_{n \to \infty} n^{1/n} = 1$, we have:
\[ L = \frac{1/2}{1} = \frac{1}{2} \]
Since $L = 1/2 < 1$, the series is \textbf{absolutely convergent}.

\subsection*{38. $\sum_{n=1}^\infty \frac{(-1)^n n!}{n^n b_1 b_2 \cdots b_n}$, where $\{b_n\} \to 1/2$ and $b_n > 0$.}
The problem text is hard to read, assuming the term is $a_n = \frac{(-1)^n n!}{n! b_1 b_2 \cdots b_n}$. If so, $a_n = \frac{(-1)^n}{b_1 \cdots b_n}$. The term $n!/n^n$ from Q14 seems more likely. Assuming the term is $a_n = \frac{(-1)^n n!}{n^n b_1 b_2 \cdots b_n}$, let's use the Ratio Test.
\[ \left|\frac{a_{n+1}}{a_n}\right| = \frac{(n+1)!}{(n+1)^{n+1} b_1 \cdots b_{n+1}} \cdot \frac{n^n b_1 \cdots b_n}{n!} = \frac{n+1}{(n+1)^{n+1} b_{n+1}} \cdot n^n = \left(\frac{n}{n+1}\right)^n \frac{1}{b_{n+1}} \]
\[ L = \lim_{n \to \infty} \left[ \left(\frac{n}{n+1}\right)^n \frac{1}{b_{n+1}} \right] = \left(\lim_{n \to \infty} \left(\frac{1}{1+1/n}\right)^n\right) \left(\lim_{n \to \infty} \frac{1}{b_{n+1}}\right) = \frac{1}{e} \cdot \frac{1}{1/2} = \frac{2}{e} \]
Since $L = 2/e < 1$, the series is \textbf{absolutely convergent}.

\subsection*{39. For which series is the Ratio Test inconclusive?}
The test is inconclusive if the limit of the ratio is 1.
\begin{description}
    \item[(a) $\sum_{n=1}^\infty \frac{1}{n^3}$:] $L = \lim_{n \to \infty} \left(\frac{n}{n+1}\right)^3 = 1$. \textbf{Inconclusive}.
    \item[(b) $\sum_{n=1}^\infty \frac{n}{2^n}$:] $L = \lim_{n \to \infty} \frac{n+1}{n} \cdot \frac{1}{2} = 1/2$. Conclusive.
    \item[(c) $\sum_{n=1}^\infty \frac{(-3)^{n-1}}{\sqrt{n}}$:] $L = \lim_{n \to \infty} 3 \frac{\sqrt{n}}{\sqrt{n+1}} = 3$. Conclusive.
    \item[(d) $\sum_{n=1}^\infty \frac{\sqrt{n}}{1+n^2}$:] $L = \lim_{n \to \infty} \sqrt{\frac{n+1}{n}} \cdot \frac{1+n^2}{1+(n+1)^2} = 1 \cdot 1 = 1$. \textbf{Inconclusive}.
\end{description}
The test is inconclusive for \textbf{(a)} and \textbf{(d)}.

\subsection*{40. For which positive integers $k$ is $\sum_{n=1}^\infty \frac{(n!)^2}{(kn)!}$ convergent?}
Let $a_n = \frac{(n!)^2}{(kn)!}$. Use the Ratio Test.
\[ \frac{a_{n+1}}{a_n} = \frac{((n+1)!)^2}{(k(n+1))!} \cdot \frac{(kn)!}{(n!)^2} = \frac{(n+1)^2 (n!)^2}{(kn+k)!} \cdot \frac{(kn)!}{(n!)^2} = \frac{(n+1)^2}{(kn+k)(kn+k-1)\cdots(kn+1)} \]
We analyze the limit of this ratio as $n \to \infty$.
\begin{itemize}
    \item If $k=1$: $\frac{a_{n+1}}{a_n} = \frac{(n+1)^2}{n+1} = n+1$. The limit is $\infty$, so the series diverges.
    \item If $k=2$: $\frac{a_{n+1}}{a_n} = \frac{(n+1)^2}{(2n+2)(2n+1)}$. The limit is $\lim_{n \to \infty} \frac{n^2+2n+1}{4n^2+6n+2} = \frac{1}{4}$. Since $L<1$, it converges.
    \item If $k > 2$: The numerator is a polynomial in $n$ of degree 2. The denominator is a product of $k$ terms, each of degree 1 in $n$, so the denominator is a polynomial of degree $k$. Since $k>2$, the degree of the denominator is greater than the degree of the numerator. The limit is 0. Since $L<1$, it converges.
\end{itemize}
The series converges for positive integers $k \ge 2$.

\subsection*{41. (a) Show $\sum_{n=0}^\infty x^n/n!$ converges for all $x$. (b) Deduce $\lim_{n \to \infty} x^n/n! = 0$.}
\begin{description}
    \item[(a)] Let $a_n = x^n/n!$. We use the Ratio Test.
    \[ L = \lim_{n \to \infty} \left| \frac{a_{n+1}}{a_n} \right| = \lim_{n \to \infty} \left| \frac{x^{n+1}/(n+1)!}{x^n/n!} \right| = \lim_{n \to \infty} \left| \frac{x^{n+1}}{x^n} \cdot \frac{n!}{(n+1)!} \right| = \lim_{n \to \infty} \left| \frac{x}{n+1} \right| \]
    For any fixed real number $x$, $|x|$ is a constant. As $n \to \infty$, $n+1 \to \infty$.
    \[ L = |x| \lim_{n \to \infty} \frac{1}{n+1} = |x| \cdot 0 = 0 \]
    Since $L=0 < 1$ for any value of $x$, the series converges for all $x$.
    \item[(b)] The Test for Divergence states that if a series $\sum a_n$ converges, then its terms must approach zero, i.e., $\lim_{n \to \infty} a_n = 0$.
    Since we proved in part (a) that the series $\sum_{n=0}^\infty x^n/n!$ converges for all $x$, it must be true that the terms of the series go to zero.
    Therefore, we deduce that $\lim_{n \to \infty} \frac{x^n}{n!} = 0$ for all $x$.
\end{description}

\subsection*{42. Let $\sum a_n$ be a series with positive terms and let $r_n = a_{n+1}/a_n$. Suppose $\lim_{n\to\infty} r_n = L < 1$.}
Let $R_n = a_{n+1} + a_{n+2} + \dots$ be the remainder.
\begin{description}
    \item[(a) If $\{r_n\}$ is a decreasing sequence and $r_{n+1} < 1$, show $R_n \le \frac{a_{n+1}}{1-r_{n+1}}$.]
    We have $a_{n+2} = r_{n+1} a_{n+1}$, $a_{n+3} = r_{n+2} a_{n+2} = r_{n+2} r_{n+1} a_{n+1}$, and so on.
    \[ R_n = a_{n+1} + a_{n+2} + a_{n+3} + \dots = a_{n+1} + r_{n+1}a_{n+1} + r_{n+2}r_{n+1}a_{n+1} + \dots \]
    Since $\{r_n\}$ is a decreasing sequence, for any $k \ge 1$, we have $r_{n+k} \le r_{n+1}$.
    Therefore, we can establish an inequality by replacing each $r_{n+k}$ with the larger or equal value $r_{n+1}$:
    \begin{align*}
        R_n &= a_{n+1} + r_{n+1}a_{n+1} + r_{n+2}r_{n+1}a_{n+1} + r_{n+3}r_{n+2}r_{n+1}a_{n+1} + \dots \\
        &\le a_{n+1} + r_{n+1}a_{n+1} + (r_{n+1})(r_{n+1})a_{n+1} + (r_{n+1})(r_{n+1})(r_{n+1})a_{n+1} + \dots \\
        &= a_{n+1} (1 + r_{n+1} + r_{n+1}^2 + r_{n+1}^3 + \dots)
    \end{align*}
    The expression in the parenthesis is a geometric series with ratio $r_{n+1}$. Since $r_{n+1} < 1$, the series converges to $\frac{1}{1-r_{n+1}}$.
    Thus, $R_n \le \frac{a_{n+1}}{1-r_{n+1}}$.
    \item[(b) If $\{r_n\}$ is an increasing sequence, show $R_n \le \frac{a_{n+1}}{1-L}$.]
    Since $\{r_n\}$ is increasing and converges to $L$, we have $r_n \le L$ for all $n$.
    Following the same logic as part (a):
    \[ R_n = a_{n+1} + r_{n+1}a_{n+1} + r_{n+2}r_{n+1}a_{n+1} + \dots \]
    We replace each $r_{n+k}$ with the larger or equal value $L$:
    \begin{align*}
        R_n &\le a_{n+1} + L a_{n+1} + L \cdot L a_{n+1} + L \cdot L \cdot L a_{n+1} + \dots \\
        &= a_{n+1} (1 + L + L^2 + L^3 + \dots)
    \end{align*}
    The expression in the parenthesis is a geometric series with ratio $L$. Since $L < 1$, the series converges to $\frac{1}{1-L}$.
    Thus, $R_n \le \frac{a_{n+1}}{1-L}$.
\end{description}

\subsection*{43. (a) Find $s_5$ of $\sum_{n=1}^\infty \frac{1}{n 2^n}$. (b) Estimate the error.}
\begin{description}
    \item[(a)] The partial sum $s_5$ is:
    \[ s_5 = \frac{1}{1 \cdot 2^1} + \frac{1}{2 \cdot 2^2} + \frac{1}{3 \cdot 2^3} + \frac{1}{4 \cdot 2^4} + \frac{1}{5 \cdot 2^5} \]
    \[ s_5 = \frac{1}{2} + \frac{1}{8} + \frac{1}{24} + \frac{1}{64} + \frac{1}{160} = 0.5 + 0.125 + 0.04166\dots + 0.015625 + 0.00625 \approx 0.68854 \]
    \item[(b)] The error is $R_5$. We use the result from Exercise 42. First, find $r_n$.
    $a_n = \frac{1}{n2^n}$.
    \[ r_n = \frac{a_{n+1}}{a_n} = \frac{1/((n+1)2^{n+1})}{1/(n2^n)} = \frac{n2^n}{(n+1)2^{n+1}} = \frac{n}{2(n+1)} \]
    To determine if $\{r_n\}$ is increasing or decreasing, consider $f(x) = \frac{x}{2(x+1)}$.
    $f'(x) = \frac{1 \cdot 2(x+1) - x \cdot 2}{(2(x+1))^2} = \frac{2x+2-2x}{4(x+1)^2} = \frac{2}{4(x+1)^2} > 0$.
    Since the derivative is positive, the sequence $\{r_n\}$ is increasing.
    We use part (b) of Exercise 42. The limit is $L = \lim_{n \to \infty} \frac{n}{2(n+1)} = \frac{1}{2}$.
    The error estimate is:
    \[ R_5 \le \frac{a_6}{1-L} = \frac{1/(6 \cdot 2^6)}{1-1/2} = \frac{1/(6 \cdot 64)}{1/2} = \frac{2}{384} = \frac{1}{192} \approx 0.0052 \]
\end{description}

\subsection*{44. Use $s_{10}$ to approximate $\sum_{n=1}^\infty \frac{n}{2^n}$ and estimate the error.}
The sum $s_{10}$ is $\sum_{n=1}^{10} \frac{n}{2^n}$.
To estimate the error $R_{10}$, we use Exercise 42. Let $a_n = n/2^n$.
\[ r_n = \frac{a_{n+1}}{a_n} = \frac{(n+1)/2^{n+1}}{n/2^n} = \frac{n+1}{2n} \]
Consider $f(x) = \frac{x+1}{2x}$. $f'(x) = \frac{1 \cdot 2x - (x+1) \cdot 2}{(2x)^2} = \frac{2x-2x-2}{4x^2} = \frac{-2}{4x^2} < 0$.
Since the derivative is negative, the sequence $\{r_n\}$ is decreasing.
We use part (a) of Exercise 42. We need $r_{n+1}$ for $n=10$, which is $r_{11}$.
\[ r_{11} = \frac{11+1}{2(11)} = \frac{12}{22} = \frac{6}{11} \]
The error estimate is:
\[ R_{10} \le \frac{a_{11}}{1-r_{11}} = \frac{11/2^{11}}{1-6/11} = \frac{11/2048}{5/11} = \frac{11}{2048} \cdot \frac{11}{5} = \frac{121}{10240} \approx 0.0118 \]

\subsection*{45. Prove the Root Test.}
We are given $\lim_{n \to \infty} \sqrt[n]{|a_n|} = L$.
\textbf{Part (i):} Assume $L < 1$. We must show $\sum a_n$ is absolutely convergent.
The idea is to compare $\sum |a_n|$ to a convergent geometric series.
Let's choose a real number $r$ such that $L < r < 1$.
Since $\lim_{n \to \infty} \sqrt[n]{|a_n|} = L$, by the definition of a limit, there must exist an integer $N$ such that for all $n \ge N$, the terms $\sqrt[n]{|a_n|}$ are closer to $L$ than they are to $r$. Specifically, $\sqrt[n]{|a_n|} < r$ for all $n \ge N$.
Raising both sides to the $n$-th power, we get $|a_n| < r^n$ for all $n \ge N$.
Now we can compare the tail of our series, $\sum_{n=N}^{\infty} |a_n|$, with the geometric series $\sum_{n=N}^{\infty} r^n$.
Since we chose $0 < L < r < 1$, the geometric series $\sum r^n$ converges.
Because $|a_n| < r^n$ for all $n \ge N$, the series $\sum_{n=N}^{\infty} |a_n|$ converges by the Direct Comparison Test.
Since the tail of the series $\sum |a_n|$ converges, the full series $\sum |a_n|$ converges.
Therefore, $\sum a_n$ is absolutely convergent.

\subsection*{46. Ramanujan's formula for $1/\pi$.}
The formula is $\frac{1}{\pi} = \frac{2\sqrt{2}}{9801} \sum_{n=0}^{\infty} \frac{(4n)!(1103+26390n)}{(n!)^4 396^{4n}}$.
\begin{description}
    \item[(a) Verify that the series is convergent.]
    Let $a_n = \frac{(4n)!(1103+26390n)}{(n!)^4 396^{4n}}$. We use the Ratio Test.
    \begin{align*}
        \frac{a_{n+1}}{a_n} &= \frac{(4(n+1))!(1103+26390(n+1))}{((n+1)!)^4 396^{4(n+1)}} \cdot \frac{(n!)^4 396^{4n}}{(4n)!(1103+26390n)} \\
        &= \frac{(4n+4)!}{(4n)!} \cdot \frac{(n!)^4}{((n+1)!)^4} \cdot \frac{396^{4n}}{396^{4n+4}} \cdot \frac{1103+26390n+26390}{1103+26390n} \\
        &= \frac{(4n+4)(4n+3)(4n+2)(4n+1)}{(n+1)^4} \cdot \frac{1}{396^4} \cdot \frac{27493+26390n}{1103+26390n}
    \end{align*}
    Now we take the limit as $n \to \infty$.
    \begin{align*}
        L = \lim_{n \to \infty} &\left[ \left(\frac{4n+4}{n+1} \cdot \frac{4n+3}{n+1} \cdot \frac{4n+2}{n+1} \cdot \frac{4n+1}{n+1}\right) \cdot \frac{1}{396^4} \cdot \frac{27493+26390n}{1103+26390n} \right] \\
        &= \left(4 \cdot 4 \cdot 4 \cdot 4\right) \cdot \frac{1}{396^4} \cdot 1 = \frac{4^4}{396^4} = \left(\frac{4}{396}\right)^4 = \left(\frac{1}{99}\right)^4
    \end{align*}
    Since $L = (1/99)^4 < 1$, the series is \textbf{convergent} by the Ratio Test.
    \item[(b) How many correct decimal places of $\pi$ with one term? Two terms?]
    \textbf{One Term ($n=0$):}
    \[ \frac{1}{\pi} \approx \frac{2\sqrt{2}}{9801} \left( \frac{(0)!(1103+0)}{(0!)^4 396^0} \right) = \frac{2\sqrt{2} \cdot 1103}{9801} \]
    \[ \pi \approx \frac{9801}{2206\sqrt{2}} \approx \frac{9801}{2206 \cdot 1.41421356} \approx \frac{9801}{3119.8336} \approx 3.141592730 \]
    Comparing to the actual value $\pi \approx 3.141592653\dots$, the approximation is correct to \textbf{6 decimal places}.
    \textbf{Two Terms ($n=0, 1$):}
    The term for $n=1$ is:
    \[ a_1 = \frac{4!(1103+26390)}{(1!)^4 396^4} = \frac{24(27493)}{396^4} \approx \frac{659832}{2.476 \times 10^{10}} \approx 2.66 \times 10^{-5} \]
    The sum is $a_0+a_1 = 1103 + a_1$.
    \[ \frac{1}{\pi} \approx \frac{2\sqrt{2}}{9801}(1103 + 2.66 \times 10^{-5}) \]
    This shows the second term is extremely small. The convergence is incredibly rapid. Each additional term in this series adds approximately 8 correct digits to the approximation of $\pi$.
\end{description}

\part{In-Depth Analysis of Problems and Techniques}
\section{Problem Types and General Approach}
\begin{description}
    \item[Type 1: Factorials and Exponentials (Problems 3-12, 14-20, 29, 30, 35, 36)]
    These problems involve terms like $n!$, $k^n$, or $(2n)!$.
    \textbf{General Approach:} The Ratio Test is almost always the most effective strategy. The ratio $|a_{n+1}/a_n|$ creates cancellations which dramatically simplify the limit calculation.

    \item[Type 2: Terms Raised to the $n$-th Power (Problems 21-26, 28, 37)]
    These problems have the form $a_n = (f(n))^n$.
    \textbf{General Approach:} The Root Test is tailor-made for this structure. Applying the $n$-th root, $\sqrt[n]{|a_n|} = |f(n)|$, immediately simplifies the expression.

    \item[Type 3: Rational and Algebraic Functions (p-series like) (Problem 39a, 39d)]
    These are series where $a_n$ is a ratio of polynomials or roots of polynomials in $n$.
    \textbf{General Approach:} The Ratio Test will yield $L=1$, making it inconclusive. The best approach is the Limit Comparison Test, typically comparing the series to a known p-series $\sum 1/n^p$.

    \item[Type 4: Hybrid/Complex Series (Problems 13, 27, 31-34)]
    These require choosing the most appropriate test from the entire toolkit.
    \textbf{General Approach:} First, check Test for Divergence (Problem 31). For alternating series, test for absolute convergence first (often with a comparison test), then use AST for conditional convergence (Problems 27, 34). For bounded terms like $\sin n$, use Direct Comparison (Problem 32).

    \item[Type 5: Theoretical Proofs and Derivations (Problems 41, 42, 45)]
    These problems ask to prove the tests themselves or derive related properties.
    \textbf{General Approach:} Go back to the formal definitions of limits. The key is often to compare the series to a simpler, known series (usually geometric) and use a comparison test, as shown in the proof of the Root Test (Q45) and the derivation of the remainder bounds (Q42).

    \item[Type 6: Parameter-Based Convergence (Problem 40)]
    These problems include an unknown constant, and the goal is to find the values of that constant for which the series converges.
    \textbf{General Approach:} Apply the Ratio Test as usual. The resulting limit $L$ will be an expression involving the parameter (in this case, $k$). Solve the inequality $L<1$ to find the range of the parameter for convergence.

    \item[Type 7: Numerical Application and Estimation (Problems 43, 44, 46)]
    These problems involve calculating partial sums and estimating the error or using a formula for a real-world calculation.
    \textbf{General Approach:} Use the formulas derived in theoretical problems (like Q42 for Q43 and Q44) to bound the remainder term. For problems like Q46, it's a matter of careful calculation.
\end{description}

\section{Key Algebraic and Calculus Manipulations}
\begin{itemize}
    \item \textbf{Factorial Simplification:} The key to solving problems with factorials (e.g., 6, 16, 40, 46) is the property $k! = k \cdot (k-1)!$. This allows for cancellations like:
    \[ \frac{(n+1)!}{n!} = n+1 \quad \text{and} \quad \frac{(2n)!}{(2n+2)!} = \frac{1}{(2n+2)(2n+1)} \]
    \item \textbf{Exponent Rules for Ratios:} Simplifying ratios of powers like $k^n$ is fundamental.
    \[ \frac{a^{n+1}}{a^n} = a \quad \text{and} \quad \frac{4^{2n+1}}{4^{2(n+1)+1}} = \frac{1}{4^2} \quad (\text{from Problem 9}) \]
    \item \textbf{The Limit for $e$:} The limit $\lim_{n \to \infty} (1 + \frac{k}{n})^n = e^k$ is crucial and appeared in Problem 14. The reciprocal form is just as important:
    \[ \lim_{n \to \infty} \left(\frac{n}{n+1}\right)^n = \lim_{n \to \infty} \left(\frac{1}{1 + 1/n}\right)^n = \frac{1}{e} \]
    \item \textbf{Limit of a Rational Function:} For the limit of a ratio of polynomials as $n \to \infty$, one can simply take the ratio of the leading terms.
    \[ \lim_{n \to \infty} \frac{2n^2+n}{n^2+2n+1} = \lim_{n \to \infty} \frac{2n^2}{n^2} = 2 \quad (\text{Used in Q40 analysis, Q46}) \]
    \item \textbf{Squeeze Theorem:} Used in Problem 36 to handle the oscillating $\cos n$ term and prove the limit of the ratio was 0.
    \item \textbf{Geometric Series Summation:} The formula $\sum_{k=0}^\infty ar^k = \frac{a}{1-r}$ was essential for the proofs in Exercise 42.
\end{itemize}

\part{"Cheatsheet" and Tips for Success}
\section{Summary of Formulas}
\begin{itemize}
    \item \textbf{Ratio Test:} For $\sum a_n$, compute $L = \lim_{n \to \infty} |\frac{a_{n+1}}{a_n}|$.
    \item \textbf{Root Test:} For $\sum a_n$, compute $L = \lim_{n \to \infty} \sqrt[n]{|a_n|}$.
    \item \textbf{Interpretation:}
        \begin{itemize}
            \item $L < 1 \implies$ Absolutely Convergent.
            \item $L > 1 \implies$ Divergent.
            \item $L = 1 \implies$ Inconclusive (Try another test!).
        \end{itemize}
    \item \textbf{Remainder Estimate for Ratio Test ($R_n = \sum_{k=n+1}^\infty a_k$):}
        \begin{itemize}
            \item If ratio $r_n$ is decreasing: $R_n \le \frac{a_{n+1}}{1-r_{n+1}}$.
            \item If ratio $r_n$ is increasing: $R_n \le \frac{a_{n+1}}{1-L}$.
        \end{itemize}
\end{itemize}

\section{Tips, Tricks, and Shortcuts}
\begin{itemize}
    \item \textbf{Problem Recognition "Cheat Sheet":}
        \begin{itemize}
            \item If you see \textbf{factorials ($n!$)} or \textbf{simple exponentials ($k^n$)}, immediately think \textbf{Ratio Test}.
            \item If you see the entire term raised to the \textbf{$n$-th power ($(\dots)^n$)}, immediately think \textbf{Root Test}.
            \item If the term is a \textbf{rational function of $n$}, the Ratio/Root Test will give $L=1$. Use the \textbf{Limit Comparison Test} with a p-series instead.
        \end{itemize}
    \item \textbf{Quick Limit Evaluation:} Remember $\lim_{n \to \infty} n^{1/n} = 1$ and $\lim_{n \to \infty} c^{1/n} = 1$ for any constant $c>0$.
\end{itemize}

\section{Common Pitfalls and How to Avoid Them}
\begin{itemize}
    \item \textbf{The $L=1$ Trap:} Never conclude that a series converges or diverges when the Ratio/Root Test yields $L=1$. It means the test failed and you \textbf{must} use a different test.
    \item \textbf{Forgetting Absolute Value:} Always use absolute values when setting up the limit. Forgetting this can lead to incorrect limits for alternating series.
    \item \textbf{Algebraic Errors with Factorials:} A common mistake is simplifying $(2n)!$ as $2 \cdot n!$. Always write it out: $(2(n+1))! = (2n+2)! = (2n+2)(2n+1)(2n)!$.
\end{itemize}

\part{Conceptual Synthesis and The "Big Picture"}
\section{Thematic Connections}
The core theme of the Ratio and Root Tests is **quantifying the rate of decay of a series' terms**. For a series $\sum a_n$ to converge, its terms $a_n$ must approach zero. The central question is, "do they approach zero \textit{fast enough}?"

These tests answer this question by comparing the series' decay rate to that of a **geometric series**, which serves as a benchmark for rapid convergence.
\begin{itemize}
    \item A limit $L < 1$ signifies that, in the long run, the series' terms shrink at least as fast as a convergent geometric series, guaranteeing convergence.
    \item A limit $L > 1$ signifies that the terms are eventually growing, causing divergence.
\end{itemize}
This theme of "rates of change/growth" is fundamental to all of calculus. It connects directly to the derivative, L'Hôpital's Rule, and the convergence of improper integrals.

\section{Forward and Backward Links}
\begin{description}
    \item[Backward Links (Foundations):]
    \begin{itemize}
        \item \textbf{Geometric Series:} The entire intuition and proof of the Ratio and Root Tests are based on a comparison to the behavior of a geometric series $\sum r^n$.
        \item \textbf{Limits at Infinity:} The tests are, at their core, applications of limit evaluation.
    \end{itemize}
    \item[Forward Links (Future Applications):]
    \begin{itemize}
        \item \textbf{Power Series and Radius of Convergence:} The single most important application of the Ratio and Root tests is finding the \textbf{radius of convergence} for a power series of the form $\sum c_n (x-a)^n$. Problem 41 is a direct example of this, showing the series for $e^x$ converges for all $x$.
        \item \textbf{Taylor and Maclaurin Series:} When we represent functions like $e^x$, $\sin(x)$, and $\cos(x)$ as infinite series, the Ratio Test is used to prove that these representations are valid for all real numbers $x$.
    \end{itemize}
\end{description}

\part{Real-World Application and Modeling}
\section{Concrete Scenarios in Finance and Economics}
\begin{description}
    \item[Scenario 1: Derivative Pricing and Stochastic Calculus]
    In quantitative finance, models like the Cox-Ross-Rubinstein binomial option pricing model are used to determine the fair price of stock options. The mathematical series that describes the value of certain exotic derivatives can be complex. The Ratio Test is a vital tool to prove that these pricing series converge to a finite value, ensuring the model gives a non-infinite, meaningful price. For example, the probability mass function of a Poisson process, $P(k) = \frac{\lambda^k e^{-\lambda}}{k!}$, used to model random events like stock price jumps, must sum to 1. The Ratio Test is the quickest way to prove $\sum_{k=0}^\infty \frac{\lambda^k}{k!}$ converges (to $e^\lambda$).

    \item[Scenario 2: Economic Modeling of Present Value]
    Economists and financial analysts constantly evaluate streams of future cash flows. An infinite stream of payments forms an infinite series. The Ratio Test can be used to determine if a company's projected, non-constant dividend growth is sustainable (i.e., if the present value of all future dividends is finite). If the ratio of projected dividend growth exceeds the discount rate, the model would predict an infinite stock price, signaling a flawed projection.

    \item[Scenario 3: Actuarial Science and Insurance]
    Actuaries model lifespans and calculate insurance premiums. The present value of a life annuity is calculated as a series where the $n$-th term is the payment, discounted to the present, multiplied by the probability that the person is still alive to receive it. The series must converge for the insurance company to set a finite premium.
\end{description}

\section{Model Problem Setup: Valuing a Growing Perpetuity}
Let's model Scenario 2: A financial asset is expected to generate cash flows forever.
\begin{itemize}
    \item \textbf{Problem:} A stock pays a dividend of \$100 this year. The company's dividend policy is to increase the dividend by 3\% each year. An investor requires a 7\% annual return on their investment (this is the discount rate). What is the fair price, or Present Value (PV), of this stock?
    \item \textbf{Variables:} $D_1 = \$100$, $g = 0.03$ (growth rate), $r = 0.07$ (discount rate).
    \item \textbf{Model Formulation:} The present value of the $n$-th dividend is $\frac{D_1 (1+g)^{n-1}}{(1+r)^n}$. The total present value is the sum of all discounted future dividends:
        \[ PV = \sum_{n=1}^{\infty} \frac{D_1 (1+g)^{n-1}}{(1+r)^n} \]
    \item \textbf{Equation to Solve:} The series is $\sum_{n=1}^{\infty} 100 \frac{(1.03)^{n-1}}{(1.07)^n}$. We can use the Ratio Test to confirm it converges.
    \[ L = \lim_{n \to \infty} \left| \frac{a_{n+1}}{a_n} \right| = \lim_{n \to \infty} \left| \frac{100(1.03)^n / (1.07)^{n+1}}{100(1.03)^{n-1} / (1.07)^n} \right| = \frac{1.03}{1.07} \]
    Since $L < 1$, the series converges, confirming the stock has a finite value. (This is a geometric series that can be summed to get the Gordon Growth Model formula: $PV = \frac{D_1}{r-g} = \frac{100}{0.07-0.03} = \$2500$).
\end{itemize}

\part{Common Variations and Untested Concepts}
\section{Untested Concept 1: Raabe's Test}
When the Ratio Test yields $L=1$, Raabe's Test can sometimes provide a conclusion.
\textbf{Theorem (Raabe's Test):} If $\sum a_n$ is a series of positive terms and
\[ L = \lim_{n \to \infty} n \left( 1 - \frac{a_{n+1}}{a_n} \right) \]
then if $L > 1$ the series converges, and if $L < 1$ the series diverges.
\textbf{Worked Example:} Test the series $\sum_{n=1}^{\infty} \frac{1}{n^2}$.
\begin{enumerate}
    \item \textbf{Ratio Test:} $\lim_{n \to \infty} (\frac{n}{n+1})^2 = 1$. Inconclusive.
    \item \textbf{Raabe's Test:}
    \begin{align*}
    L &= \lim_{n \to \infty} n \left( 1 - \frac{n^2}{(n+1)^2} \right) = \lim_{n \to \infty} n \left( \frac{(n+1)^2 - n^2}{(n+1)^2} \right) \\
    &= \lim_{n \to \infty} n \left( \frac{2n+1}{n^2+2n+1} \right) = \lim_{n \to \infty} \frac{2n^2+n}{n^2+2n+1} = 2
    \end{align*}
    \item \textbf{Conclusion:} Since $L=2 > 1$, Raabe's Test shows the series converges.
\end{enumerate}

\section{Untested Concept 2: Limit Superior}
The more general forms of the Ratio and Root tests use the \textit{limit superior} ($\limsup$), which is the largest possible limit point of a sequence.
\textbf{General Root Test:} For a series $\sum a_n$, let $L = \limsup_{n \to \infty} \sqrt[n]{|a_n|}$. If $L<1$, it converges absolutely. If $L>1$, it diverges.

\part{Advanced Diagnostic Testing: "Find the Flaw"}

\subsection*{Problem 1}
Test the convergence of the series $\sum_{n=1}^{\infty} \frac{(n!)^2}{(2n)!}$.
\textbf{Flawed Solution:}
We use the Ratio Test. Let $a_n = \frac{(n!)^2}{(2n)!}$.
\begin{align*}
\lim_{n \to \infty} \left| \frac{a_{n+1}}{a_n} \right| &= \lim_{n \to \infty} \frac{((n+1)!)^2}{(2(n+1))!} \cdot \frac{(2n)!}{(n!)^2} = \lim_{n \to \infty} \frac{(n+1)^2}{(2n+2)(2n+1)} = \frac{1}{4}
\end{align*}
Since $L = 1/4 < 1$, the series diverges.
\textbf{Flaw:} The conclusion is incorrect. \textbf{Why it's an error:} The Ratio Test states that if $L<1$, the series converges absolutely. \textbf{Correct step and solution:} The calculation $L=1/4$ is correct. Since $L=1/4 < 1$, the series is absolutely convergent.

\subsection*{Problem 2}
Test the convergence of the series $\sum_{n=1}^{\infty} \frac{(-3)^n}{n^3}$.
\textbf{Flawed Solution:}
We use the Ratio Test. Let $a_n = \frac{(-3)^n}{n^3}$.
\[ L = \lim_{n \to \infty} \frac{a_{n+1}}{a_n} = \lim_{n \to \infty} \frac{(-3)^{n+1}/(n+1)^3}{(-3)^n/n^3} = \lim_{n \to \infty} -3 \left(\frac{n}{n+1}\right)^3 = -3 \]
Since $L = -3 < 1$, the series converges.
\textbf{Flaw:} The setup for the Ratio Test limit is missing absolute value bars. \textbf{Why it's an error:} The Ratio Test is defined for $L = \lim_{n \to \infty} |\frac{a_{n+1}}{a_n}|$, so the limit $L$ must be non-negative. \textbf{Correct step and solution:} $L = \lim_{n \to \infty} \left| -3 \left(\frac{n}{n+1}\right)^3 \right| = |-3| \cdot (1)^3 = 3$. Since $L=3>1$, the series diverges.

\subsection*{Problem 3}
Test the convergence of the series $\sum_{n=2}^\infty \frac{1}{n \ln n}$.
\textbf{Flawed Solution:}
We use the Ratio Test. Let $a_n = \frac{1}{n \ln n}$.
\[ L = \lim_{n \to \infty} \frac{n \ln n}{(n+1)\ln(n+1)} = \left(\lim_{n \to \infty} \frac{n}{n+1}\right) \left(\lim_{n \to \infty} \frac{\ln n}{\ln(n+1)}\right) = 1 \cdot 1 = 1 \]
Since the limit is 1, the series diverges.
\textbf{Flaw:} The conclusion drawn from $L=1$ is invalid. \textbf{Why it's an error:} A limit of 1 from the Ratio Test is inconclusive; it does not imply divergence. \textbf{Correct step and solution:} The calculation $L=1$ is correct, but another test is needed. Using the Integral Test: $\int_2^\infty \frac{1}{x \ln x} dx = [\ln(\ln x)]_2^\infty = \infty$. The integral diverges, so the series diverges.

\subsection*{Problem 4}
Test the convergence of the series $\sum_{n=1}^\infty \left( \frac{n}{n+1} \right)^{n^2}$.
\textbf{Flawed Solution:}
We use the Root Test. Let $a_n = \left( \frac{n}{n+1} \right)^{n^2}$.
\[ L = \lim_{n \to \infty} \sqrt[n]{|a_n|} = \lim_{n \to \infty} \left( \left( \frac{n}{n+1} \right)^{n^2} \right)^{1/n} = \lim_{n \to \infty} \left( \frac{n}{n+1} \right)^n = \frac{1}{e} \]
Since $L=1/e \approx 0.368$, the series is less than 1 and therefore converges.
\textbf{Flaw:} The text "the series is less than 1" is imprecise and conceptually muddled. \textbf{Why it's an error:} The series itself is a sum, not a number that can be compared to 1 in this way; it is the *limit L* that is less than 1. \textbf{Correct step and solution:} The calculation $L=1/e$ is correct. Since the limit $L=1/e < 1$, the series converges by the Root Test.

\subsection*{Problem 5}
Test the convergence of the series $\sum_{n=1}^{\infty} \frac{e^n}{n!}$.
\textbf{Flawed Solution:}
We use the Ratio Test. Let $a_n = \frac{e^n}{n!}$.
\[ L = \lim_{n \to \infty} \left| \frac{e^{n+1}/(n+1)!}{e^n/n!} \right| = \lim_{n \to \infty} \frac{e}{n+1} \]
As $n \to \infty$, the denominator $n+1$ goes to infinity. When the denominator of a fraction goes to infinity, the fraction itself goes to infinity. So $L=\infty$. Since $L > 1$, the series diverges.
\textbf{Flaw:} The evaluation of the limit is incorrect. \textbf{Why it's an error:} When a constant is divided by a quantity that goes to infinity, the limit is zero, not infinity. \textbf{Correct step and solution:} The setup is correct. $L = \lim_{n \to \infty} \frac{e}{n+1} = 0$. Since $L=0<1$, the series converges.

\end{document}