\documentclass{article}
\usepackage{amsmath}
\usepackage{amssymb}
\usepackage{geometry}
\geometry{a4paper, margin=1in}

\title{Homework 11.5: Alternating Series and Absolute Convergence}
\author{Tashfeen Omran}
\date{\today}

\begin{document}

\maketitle

\part{Comprehensive Introduction, Context, and Prerequisites}

\section{Core Concepts}

This section delves into series whose terms are not all positive. We introduce specific tests for series whose terms alternate in sign, and we develop the crucial concepts of absolute and conditional convergence to understand the behavior of series with arbitrarily mixed signs.

\subsection{Alternating Series}
An \textbf{alternating series} is an infinite series whose terms alternate between positive and negative. They can be written in one of two forms:
\begin{itemize}
    \item $\sum_{n=1}^{\infty} (-1)^{n-1}b_n = b_1 - b_2 + b_3 - b_4 + \dots$
    \item $\sum_{n=1}^{\infty} (-1)^{n}b_n = -b_1 + b_2 - b_3 + b_4 - \dots$
\end{itemize}
where $b_n$ is a sequence of positive numbers for all $n$. In fact, $b_n = |a_n|$.

\subsection{The Alternating Series Test (AST)}
This is the primary tool for determining the convergence of an alternating series. The alternating series $\sum (-1)^{n-1}b_n$ (or $\sum (-1)^{n}b_n$) converges if it satisfies two simple conditions:
\begin{enumerate}
    \item \textbf{Decreasing Magnitude:} The terms $b_n$ must be decreasing for all $n$ past a certain point. That is, $b_{n+1} \le b_n$ for all $n \ge N$ for some integer $N$.
    \item \textbf{Limit is Zero:} The limit of the terms $b_n$ must be zero. That is, $\lim_{n \to \infty} b_n = 0$.
\end{enumerate}
If \textit{both} conditions are met, the series converges. If the second condition fails (i.e., $\lim_{n \to \infty} b_n \neq 0$), then the limit of the series terms $a_n = (-1)^{n-1}b_n$ is also not zero, and the series diverges by the Test for Divergence.

\subsection{Estimating Sums: The Alternating Series Estimation Theorem}
For a convergent alternating series that satisfies the conditions of the AST, we can estimate its sum $s$ using a partial sum $s_n$. The error, or remainder $R_n = s - s_n$, is bounded by the magnitude of the first neglected term:
\[ |R_n| = |s - s_n| \le b_{n+1} \]
This is an incredibly useful theorem because it gives us a simple way to guarantee the accuracy of our approximation. It also tells us that the true sum $s$ is always located between any two consecutive partial sums, $s_n$ and $s_{n+1}$.

\subsection{Absolute and Conditional Convergence}
For any series $\sum a_n$ (not necessarily alternating), we can analyze its convergence in a more robust way.
\begin{itemize}
    \item \textbf{Absolute Convergence:} A series $\sum a_n$ is called \textbf{absolutely convergent} if the series of its absolute values, $\sum |a_n|$, converges.
    \item \textbf{Conditional Convergence:} A series $\sum a_n$ is called \textbf{conditionally convergent} if it converges, but the series of its absolute values, $\sum |a_n|$, diverges.
\end{itemize}
A foundational theorem states: \textbf{If a series converges absolutely, then it converges.}
This means absolute convergence is a stronger condition than regular convergence.

\subsection{Rearrangements}
The distinction between absolute and conditional convergence is critical when considering rearrangements of terms.
\begin{itemize}
    \item If a series is \textbf{absolutely convergent}, any rearrangement of its terms will converge to the \textit{same sum}.
    \item If a series is \textbf{conditionally convergent}, its terms can be rearranged to make the new series converge to \textit{any real number}, or even diverge. This is the content of the Riemann Rearrangement Theorem.
\end{itemize}

\section{Intuition and Derivation}
The intuition behind the Alternating Series Test is best visualized on a number line.
\begin{enumerate}
    \item Start at 0. The first partial sum is $s_1 = b_1$.
    \item To get $s_2$, we subtract $b_2$, moving to the left. Since $b_2 \le b_1$, we have $s_2 \ge 0$.
    \item To get $s_3$, we add $b_3$, moving to the right. Since $b_3 \le b_2$, we don't move past $s_1$. So $s_2 \le s_3 \le s_1$.
    \item This process continues. The even partial sums ($s_2, s_4, s_6, \dots$) are always increasing, while the odd partial sums ($s_1, s_3, s_5, \dots$) are always decreasing.
\end{enumerate}
The sequence of even partial sums is increasing and bounded above by $b_1$. The sequence of odd partial sums is decreasing and bounded below by $s_2$. Therefore, both sequences must converge. Since the distance between them, $s_{2n+1} - s_{2n} = b_{2n+1}$, goes to 0 as $n \to \infty$, they must converge to the same limit, $s$. The sum $s$ is "trapped" within these oscillating partial sums. The error estimation theorem follows directly from this picture: the distance from $s_n$ to the final sum $s$ can't be more than the length of the next "jump," which is $b_{n+1}$.

\section{Historical Context and Motivation}
The study of infinite series blossomed in the 17th and 18th centuries with mathematicians like Newton and Euler, who often manipulated them with astounding creativity but a lack of formal rigor. They treated infinite sums much like finite ones, which sometimes led to correct results and other times to paradoxes. For example, the alternating harmonic series $1 - 1/2 + 1/3 - \dots$ was known to converge to $\ln(2)$, but rearranging it could produce different sums, a deeply troubling discovery.

This motivated mathematicians of the 19th century, such as Cauchy, Abel, and Weierstrass, to build a rigorous foundation for calculus and analysis. They formally defined concepts like limits and convergence. It was in this context that Bernhard Riemann proved his famous Rearrangement Theorem around 1854, demonstrating the bizarre and beautiful nature of conditionally convergent series. This work solidified the critical distinction between absolute and conditional convergence, showing that only absolutely convergent series obey the familiar commutative law of addition that holds for finite sums. This was a crucial step in taming the infinite and ensuring that the methods of calculus were built on a sound logical footing.

\section{Key Formulas}
\begin{itemize}
    \item \textbf{Alternating Series Test:} For $\sum (-1)^{n-1}b_n$ with $b_n>0$, if
    \begin{enumerate}
        \item $b_{n+1} \le b_n$ for all $n \ge N$
        \item $\lim_{n \to \infty} b_n = 0$
    \end{enumerate}
    then the series converges.
    
    \item \textbf{Alternating Series Estimation Theorem:} If $s = \sum (-1)^{n-1}b_n$ is the sum of a convergent alternating series, then the remainder $R_n = s-s_n$ satisfies:
    \[ |R_n| = |s - s_n| \le b_{n+1} \]
    
    \item \textbf{Absolute Convergence Implies Convergence:}
    \[ \text{If } \sum_{n=1}^{\infty} |a_n| \text{ converges, then } \sum_{n=1}^{\infty} a_n \text{ converges.} \]
\end{itemize}

\section{Prerequisites}
To master this topic, you must be proficient in the following:
\begin{itemize}
    \item \textbf{Limits of Sequences:} All convergence tests ultimately rely on evaluating limits as $n \to \infty$.
    \item \textbf{The Test for Divergence:} If $\lim_{n \to \infty} a_n \neq 0$, the series $\sum a_n$ diverges. This is the first check for any series.
    \item \textbf{Convergence Tests for Positive Series:} You must know how to use the Integral Test, Comparison Test, Limit Comparison Test, and Ratio/Root Tests to analyze $\sum|a_n|$.
    \item \textbf{Benchmark Series:} Instant recognition of the convergence/divergence of p-series ($\sum 1/n^p$) and geometric series ($\sum ar^n$) is essential.
    \item \textbf{Derivatives:} Finding the derivative of a function $f(x)$ corresponding to $b_n$ is a key method for proving that $b_n$ is decreasing.
\end{itemize}

\part{Detailed Homework Solutions}

\section*{Exercises 11.5}

\subsection*{Problem 1}
\textbf{(a) What is an alternating series?}
\textbf{Solution:} An alternating series is a series whose terms are alternately positive and negative. It can be written in the form $\sum_{n=1}^{\infty} (-1)^{n-1} b_n$ or $\sum_{n=1}^{\infty} (-1)^{n} b_n$, where $b_n$ is a positive number for all $n$.

\textbf{(b) Under what conditions does an alternating series converge?}
\textbf{Solution:} An alternating series converges if it satisfies the two conditions of the Alternating Series Test (AST):
\begin{enumerate}
    \item The sequence of magnitudes $\{b_n\}$ is eventually decreasing, i.e., $b_{n+1} \le b_n$ for all $n$ greater than some integer $N$.
    \item The limit of the magnitudes is zero, i.e., $\lim_{n \to \infty} b_n = 0$.
\end{enumerate}

\textbf{(c) If these conditions are satisfied, what can you say about the remainder after n terms?}
\textbf{Solution:} If the conditions of the AST are satisfied, the remainder $R_n = s - s_n$ (where $s$ is the true sum and $s_n$ is the $n$-th partial sum) is bounded by the magnitude of the first neglected term. Specifically, $|R_n| = |s - s_n| \le b_{n+1}$.

\subsection*{Problem 2}
Test the series $\frac{2}{3} - \frac{2}{5} + \frac{2}{7} - \frac{2}{9} + \dots$ for convergence or divergence.
\textbf{Solution:} The series can be written as $\sum_{n=1}^{\infty} (-1)^{n-1} \frac{2}{2n+1}$. This is an alternating series with $b_n = \frac{2}{2n+1}$. We apply the AST.
\begin{enumerate}
    \item Check if $b_n$ is decreasing: $b_{n+1} = \frac{2}{2(n+1)+1} = \frac{2}{2n+3}$. Since $2n+3 > 2n+1$, the denominator is increasing, so the fraction is decreasing. Thus, $b_{n+1} < b_n$.
    \item Check the limit: $\lim_{n \to \infty} b_n = \lim_{n \to \infty} \frac{2}{2n+1} = 0$.
\end{enumerate}
Both conditions of the AST are satisfied.
\textbf{Final Answer:} The series is convergent.

\subsection*{Problem 3}
Test the series $-\frac{2}{5} + \frac{4}{6} - \frac{6}{7} + \frac{8}{8} - \frac{10}{9} + \dots$ for convergence or divergence.
\textbf{Solution:} The series can be written as $\sum_{n=1}^{\infty} (-1)^{n} \frac{2n}{n+4}$. This is an alternating series with $b_n = \frac{2n}{n+4}$. Let's check the limit of $b_n$.
\[ \lim_{n \to \infty} b_n = \lim_{n \to \infty} \frac{2n}{n+4} = \lim_{n \to \infty} \frac{2}{1+4/n} = 2 \]
Since $\lim_{n \to \infty} b_n = 2 \neq 0$, the limit of the general term of the series, $\lim_{n \to \infty} a_n = \lim_{n \to \infty} (-1)^n \frac{2n}{n+4}$, does not exist. By the Test for Divergence, the series diverges.
\textbf{Final Answer:} The series is divergent.

\subsection*{Problem 4}
Test the series $\frac{1}{\ln 3} - \frac{1}{\ln 4} + \frac{1}{\ln 5} - \frac{1}{\ln 6} + \frac{1}{\ln 7} - \dots$ for convergence or divergence.
\textbf{Solution:} The series can be written as $\sum_{n=3}^{\infty} (-1)^{n-1} \frac{1}{\ln n}$. This is an alternating series with $b_n = \frac{1}{\ln n}$. We apply the AST.
\begin{enumerate}
    \item Check if $b_n$ is decreasing: The function $\ln(x)$ is increasing for $x>0$. Therefore, $\frac{1}{\ln x}$ is decreasing. So, $b_{n+1} = \frac{1}{\ln(n+1)} < \frac{1}{\ln n} = b_n$.
    \item Check the limit: $\lim_{n \to \infty} b_n = \lim_{n \to \infty} \frac{1}{\ln n} = 0$, since $\ln n \to \infty$.
\end{enumerate}
Both conditions of the AST are satisfied.
\textbf{Final Answer:} The series is convergent.

\subsection*{Problem 5}
Test the series $\sum_{n=1}^{\infty} \frac{(-1)^{n-1}}{3+5n}$ for convergence or divergence.
\textbf{Solution:} This is an alternating series with $b_n = \frac{1}{3+5n}$. We apply the AST.
\begin{enumerate}
    \item Check if $b_n$ is decreasing: The denominator $3+5n$ is an increasing function of $n$. Thus, the fraction $\frac{1}{3+5n}$ is decreasing. So $b_{n+1} < b_n$.
    \item Check the limit: $\lim_{n \to \infty} b_n = \lim_{n \to \infty} \frac{1}{3+5n} = 0$.
\end{enumerate}
Both conditions of the AST are satisfied.
\textbf{Final Answer:} The series is convergent.

\subsection*{Problem 6}
Test the series $\sum_{n=0}^{\infty} \frac{(-1)^{n+1}}{\sqrt{n+1}}$ for convergence or divergence.
\textbf{Solution:} The series can be written as $\sum_{k=1}^{\infty} \frac{(-1)^{k}}{\sqrt{k}}$ by letting $k=n+1$. This is an alternating series with $b_k = \frac{1}{\sqrt{k}}$. We apply the AST.
\begin{enumerate}
    \item Check if $b_k$ is decreasing: The denominator $\sqrt{k}$ is increasing for $k \ge 1$. Thus, the fraction $\frac{1}{\sqrt{k}}$ is decreasing. So $b_{k+1} < b_k$.
    \item Check the limit: $\lim_{k \to \infty} b_k = \lim_{k \to \infty} \frac{1}{\sqrt{k}} = 0$.
\end{enumerate}
Both conditions of the AST are satisfied.
\textbf{Final Answer:} The series is convergent.

\subsection*{Problem 7}
Test the series $\sum_{n=1}^{\infty} (-1)^n \frac{3n-1}{2n+1}$ for convergence or divergence.
\textbf{Solution:} This is an alternating series with $b_n = \frac{3n-1}{2n+1}$. We check the limit of $b_n$.
\[ \lim_{n \to \infty} b_n = \lim_{n \to \infty} \frac{3n-1}{2n+1} = \lim_{n \to \infty} \frac{3-1/n}{2+1/n} = \frac{3}{2} \]
Since $\lim_{n \to \infty} b_n \neq 0$, the Test for Divergence applies. The limit $\lim_{n \to \infty} a_n$ does not exist.
\textbf{Final Answer:} The series is divergent.

\subsection*{Problem 8}
Test the series $\sum_{n=1}^{\infty} (-1)^n \frac{n^2}{n^2+n+1}$ for convergence or divergence.
\textbf{Solution:} This is an alternating series with $b_n = \frac{n^2}{n^2+n+1}$. We check the limit of $b_n$.
\[ \lim_{n \to \infty} b_n = \lim_{n \to \infty} \frac{n^2}{n^2+n+1} = \lim_{n \to \infty} \frac{1}{1+1/n+1/n^2} = 1 \]
Since $\lim_{n \to \infty} b_n \neq 0$, the Test for Divergence applies.
\textbf{Final Answer:} The series is divergent.

\subsection*{Problem 9}
Test the series $\sum_{n=1}^{\infty} (-1)^n e^{-n}$ for convergence or divergence.
\textbf{Solution:} The series is $\sum_{n=1}^{\infty} (-1)^n \frac{1}{e^n}$. This is an alternating series with $b_n = \frac{1}{e^n} = (1/e)^n$. This is an alternating geometric series with ratio $r = -1/e$. Since $|r| = 1/e < 1$, the series converges.
Alternatively, using the AST:
\begin{enumerate}
    \item $b_n = \frac{1}{e^n}$. Since $e^n$ is an increasing function, $1/e^n$ is decreasing.
    \item $\lim_{n \to \infty} b_n = \lim_{n \to \infty} \frac{1}{e^n} = 0$.
\end{enumerate}
Both conditions are met.
\textbf{Final Answer:} The series is convergent.

\subsection*{Problem 10}
Test the series $\sum_{n=1}^{\infty} (-1)^n \frac{\sqrt{n}}{2n+3}$ for convergence or divergence.
\textbf{Solution:} This is an alternating series with $b_n = \frac{\sqrt{n}}{2n+3}$. We apply the AST.
\begin{enumerate}
    \item Check the limit: $\lim_{n \to \infty} b_n = \lim_{n \to \infty} \frac{\sqrt{n}}{2n+3} = \lim_{n \to \infty} \frac{1/\sqrt{n}}{2+3/n} = 0$.
    \item Check if decreasing: Let $f(x) = \frac{\sqrt{x}}{2x+3}$.
    \[ f'(x) = \frac{\frac{1}{2\sqrt{x}}(2x+3) - \sqrt{x}(2)}{(2x+3)^2} = \frac{\frac{2x+3 - 4x}{2\sqrt{x}}}{(2x+3)^2} = \frac{3-2x}{2\sqrt{x}(2x+3)^2} \]
    The derivative $f'(x)$ is negative for $3-2x < 0$, which means $x > 3/2$. Thus, the sequence $\{b_n\}$ is decreasing for $n \ge 2$.
\end{enumerate}
Both conditions of the AST are satisfied.
\textbf{Final Answer:} The series is convergent.

\subsection*{Problem 11}
Test the series $\sum_{n=1}^{\infty} (-1)^{n+1} \frac{n^2}{n^3+4}$ for convergence or divergence.
\textbf{Solution:} This is an alternating series with $b_n = \frac{n^2}{n^3+4}$. We apply the AST.
\begin{enumerate}
    \item Check the limit: $\lim_{n \to \infty} b_n = \lim_{n \to \infty} \frac{n^2}{n^3+4} = \lim_{n \to \infty} \frac{1/n}{1+4/n^3} = 0$.
    \item Check if decreasing: Let $f(x) = \frac{x^2}{x^3+4}$.
    \[ f'(x) = \frac{2x(x^3+4) - x^2(3x^2)}{(x^3+4)^2} = \frac{2x^4+8x-3x^4}{(x^3+4)^2} = \frac{8x-x^4}{(x^3+4)^2} = \frac{x(8-x^3)}{(x^3+4)^2} \]
    The derivative $f'(x)$ is negative for $8-x^3 < 0$, which means $x^3 > 8$ or $x > 2$. The sequence $\{b_n\}$ is decreasing for $n \ge 3$.
\end{enumerate}
Both conditions of the AST are satisfied.
\textbf{Final Answer:} The series is convergent.

\subsection*{Problem 12}
Test the series $\sum_{n=1}^{\infty} (-1)^n \frac{n}{2^n}$ for convergence or divergence.
\textbf{Solution:} This is an alternating series. We can test for absolute convergence using the Ratio Test.
Consider $\sum_{n=1}^{\infty} \left| (-1)^n \frac{n}{2^n} \right| = \sum_{n=1}^{\infty} \frac{n}{2^n}$.
\begin{align*}
    L = \lim_{n \to \infty} \left| \frac{a_{n+1}}{a_n} \right| &= \lim_{n \to \infty} \frac{(n+1)/2^{n+1}}{n/2^n} \\
    &= \lim_{n \to \infty} \frac{n+1}{n} \cdot \frac{2^n}{2^{n+1}} \\
    &= \lim_{n \to \infty} \left(1 + \frac{1}{n}\right) \cdot \frac{1}{2} = 1 \cdot \frac{1}{2} = \frac{1}{2}
\end{align*}
Since $L=1/2 < 1$, the series of absolute values converges. Therefore, the original series converges absolutely.
\textbf{Final Answer:} The series is convergent.

\subsection*{Problem 13}
Test the series $\sum_{n=1}^{\infty} (-1)^{n-1} e^{2/n}$ for convergence or divergence.
\textbf{Solution:} This is an alternating series with $b_n = e^{2/n}$. We check the limit of $b_n$.
As $n \to \infty$, $2/n \to 0$. Therefore,
\[ \lim_{n \to \infty} b_n = \lim_{n \to \infty} e^{2/n} = e^0 = 1 \]
Since the limit of $b_n$ is not 0, the series diverges by the Test for Divergence.
\textbf{Final Answer:} The series is divergent.

\subsection*{Problem 14}
Test the series $\sum_{n=1}^{\infty} (-1)^{n-1} \arctan n$ for convergence or divergence.
\textbf{Solution:} This is an alternating series with $b_n = \arctan n$. We check the limit of $b_n$.
\[ \lim_{n \to \infty} b_n = \lim_{n \to \infty} \arctan n = \frac{\pi}{2} \]
Since the limit of $b_n$ is not 0, the series diverges by the Test for Divergence.
\textbf{Final Answer:} The series is divergent.

\subsection*{Problem 15}
Test the series $\sum_{n=0}^{\infty} \frac{\sin(n + 1/2)\pi}{1+\sqrt{n}}$ for convergence or divergence.
\textbf{Solution:} Let's analyze the term $\sin(n + 1/2)\pi$.
For $n=0$: $\sin(\pi/2)=1$. For $n=1$: $\sin(3\pi/2)=-1$. For $n=2$: $\sin(5\pi/2)=1$.
In general, $\sin(n\pi + \pi/2) = \sin(n\pi)\cos(\pi/2) + \cos(n\pi)\sin(\pi/2) = \cos(n\pi) = (-1)^n$.
So the series is $\sum_{n=0}^{\infty} \frac{(-1)^n}{1+\sqrt{n}}$. This is an alternating series with $b_n = \frac{1}{1+\sqrt{n}}$. We apply the AST.
\begin{enumerate}
    \item Check if decreasing: The denominator $1+\sqrt{n}$ is an increasing function of $n$, so the fraction is decreasing.
    \item Check the limit: $\lim_{n \to \infty} b_n = \lim_{n \to \infty} \frac{1}{1+\sqrt{n}} = 0$.
\end{enumerate}
Both conditions are met.
\textbf{Final Answer:} The series is convergent.

\subsection*{Problem 16}
Test the series $\sum_{n=1}^{\infty} \frac{n \cos(n\pi)}{2^n}$ for convergence or divergence.
\textbf{Solution:} The term $\cos(n\pi)$ is equal to $(-1)^n$. So the series is $\sum_{n=1}^{\infty} \frac{n (-1)^n}{2^n}$. This is the same series as in Problem 12, which we showed converges absolutely by the Ratio Test.
\textbf{Final Answer:} The series is convergent.

\subsection*{Problem 17}
Test the series $\sum_{n=1}^{\infty} (-1)^n \sin(\frac{\pi}{n})$ for convergence or divergence.
\textbf{Solution:} This is an alternating series with $b_n = \sin(\frac{\pi}{n})$. We apply the AST.
\begin{enumerate}
    \item Check the limit: As $n \to \infty$, $\frac{\pi}{n} \to 0$. Since $\sin(x) \to 0$ as $x \to 0$, we have $\lim_{n \to \infty} b_n = \lim_{n \to \infty} \sin(\frac{\pi}{n}) = 0$.
    \item Check if decreasing: For $n \ge 2$, $0 < \frac{\pi}{n} \le \frac{\pi}{2}$. In this interval, $\sin(x)$ is an increasing function. Since $\frac{\pi}{n}$ is a decreasing sequence, $b_n = \sin(\frac{\pi}{n})$ is also a decreasing sequence. That is, $\frac{\pi}{n+1} < \frac{\pi}{n} \implies \sin(\frac{\pi}{n+1}) < \sin(\frac{\pi}{n})$.
\end{enumerate}
Both conditions are met.
\textbf{Final Answer:} The series is convergent.

\subsection*{Problem 18}
Test the series $\sum_{n=1}^{\infty} (-1)^n \cos(\frac{\pi}{n})$ for convergence or divergence.
\textbf{Solution:} This is an alternating series with $b_n = \cos(\frac{\pi}{n})$. We check the limit of $b_n$.
As $n \to \infty$, $\frac{\pi}{n} \to 0$. Therefore,
\[ \lim_{n \to \infty} b_n = \lim_{n \to \infty} \cos(\frac{\pi}{n}) = \cos(0) = 1 \]
Since the limit of $b_n$ is not 0, the series diverges by the Test for Divergence.
\textbf{Final Answer:} The series is divergent.

\subsection*{Problem 19}
Test the series $\sum_{n=1}^{\infty} (-1)^n \frac{n^2}{5^n}$ for convergence or divergence.
\textbf{Solution:} We test for absolute convergence using the Ratio Test. Consider $\sum_{n=1}^{\infty} \frac{n^2}{5^n}$.
\begin{align*}
    L = \lim_{n \to \infty} \left| \frac{a_{n+1}}{a_n} \right| &= \lim_{n \to \infty} \frac{(n+1)^2/5^{n+1}}{n^2/5^n} \\
    &= \lim_{n \to \infty} \frac{(n+1)^2}{n^2} \cdot \frac{5^n}{5^{n+1}} \\
    &= \lim_{n \to \infty} \left(1 + \frac{1}{n}\right)^2 \cdot \frac{1}{5} = 1^2 \cdot \frac{1}{5} = \frac{1}{5}
\end{align*}
Since $L=1/5 < 1$, the series converges absolutely.
\textbf{Final Answer:} The series is convergent.

\subsection*{Problem 20}
Test the series $\sum_{n=1}^{\infty} (-1)^n (\sqrt{n+1} - \sqrt{n})$ for convergence or divergence.
\textbf{Solution:} This is an alternating series with $b_n = \sqrt{n+1} - \sqrt{n}$. Let's simplify $b_n$ by multiplying by the conjugate.
\[ b_n = (\sqrt{n+1} - \sqrt{n}) \frac{\sqrt{n+1} + \sqrt{n}}{\sqrt{n+1} + \sqrt{n}} = \frac{(n+1)-n}{\sqrt{n+1}+\sqrt{n}} = \frac{1}{\sqrt{n+1}+\sqrt{n}} \]
Now we apply the AST.
\begin{enumerate}
    \item Check the limit: $\lim_{n \to \infty} b_n = \lim_{n \to \infty} \frac{1}{\sqrt{n+1}+\sqrt{n}} = 0$.
    \item Check if decreasing: The denominator $\sqrt{n+1}+\sqrt{n}$ is an increasing function of $n$, so the fraction is decreasing.
\end{enumerate}
Both conditions are met.
\textbf{Final Answer:} The series is convergent.

\subsection*{Problem 21}
\textbf{(a) What does it mean for a series to be absolutely convergent?}
\textbf{Solution:} A series $\sum a_n$ is absolutely convergent if the series of the absolute values of its terms, $\sum |a_n|$, converges.

\textbf{(b) What does it mean for a series to be conditionally convergent?}
\textbf{Solution:} A series $\sum a_n$ is conditionally convergent if it is convergent, but it is not absolutely convergent (i.e., $\sum |a_n|$ diverges).

\textbf{(c) If the series of positive terms $\sum b_n$ converges, then what can you say about the series $\sum (-1)^n b_n$?}
\textbf{Solution:} If $\sum b_n$ converges (where $b_n > 0$), then the series $\sum (-1)^n b_n$ is absolutely convergent because $\sum |(-1)^n b_n| = \sum b_n$, which converges by the premise. Since absolute convergence implies convergence, $\sum (-1)^n b_n$ is convergent.

\subsection*{Problem 22}
Determine whether $\sum_{n=1}^{\infty} \frac{(-1)^n}{n^4}$ is AC, CC, or D.
\textbf{Solution:}
\begin{enumerate}
    \item \textbf{Test Absolute Convergence:} Consider $\sum_{n=1}^{\infty} \left| \frac{(-1)^n}{n^4} \right| = \sum_{n=1}^{\infty} \frac{1}{n^4}$. This is a p-series with $p=4$. Since $p > 1$, the series converges.
\end{enumerate}
Since the series of absolute values converges, the original series is absolutely convergent.
\textbf{Final Answer:} Absolutely convergent.

\subsection*{Problem 23}
Determine whether $\sum_{n=1}^{\infty} \frac{(-1)^{n-1}}{\sqrt{n^2}}$ is AC, CC, or D.
\textbf{Solution:} The series is $\sum_{n=1}^{\infty} \frac{(-1)^{n-1}}{n}$, since $\sqrt{n^2}=n$ for $n>0$. This is the alternating harmonic series.
\begin{enumerate}
    \item \textbf{Test Absolute Convergence:} Consider $\sum_{n=1}^{\infty} \left| \frac{(-1)^{n-1}}{n} \right| = \sum_{n=1}^{\infty} \frac{1}{n}$. This is the harmonic series (p-series with $p=1$), which diverges. So, the series is not absolutely convergent.
    \item \textbf{Test Conditional Convergence (AST):} Let $b_n = 1/n$.
    (i) $b_{n+1} = \frac{1}{n+1} < \frac{1}{n} = b_n$, so it's decreasing.
    (ii) $\lim_{n \to \infty} b_n = \lim_{n \to \infty} \frac{1}{n} = 0$.
    By the AST, the original series converges.
\end{enumerate}
Since the series converges but not absolutely, it is conditionally convergent.
\textbf{Final Answer:} Conditionally convergent.

\subsection*{Problem 24}
Determine whether $\sum_{n=0}^{\infty} \frac{(-1)^{n+1}}{n^2+1}$ is AC, CC, or D.
\textbf{Solution:}
\begin{enumerate}
    \item \textbf{Test Absolute Convergence:} Consider $\sum_{n=0}^{\infty} \left| \frac{(-1)^{n+1}}{n^2+1} \right| = \sum_{n=0}^{\infty} \frac{1}{n^2+1}$. We use the Limit Comparison Test with the convergent p-series $\sum \frac{1}{n^2}$ ($p=2$).
    \[ L = \lim_{n \to \infty} \frac{1/(n^2+1)}{1/n^2} = \lim_{n \to \infty} \frac{n^2}{n^2+1} = 1 \]
    Since $0 < L < \infty$ and $\sum \frac{1}{n^2}$ converges, the series $\sum \frac{1}{n^2+1}$ also converges.
\end{enumerate}
The series is absolutely convergent.
\textbf{Final Answer:} Absolutely convergent.

\subsection*{Problem 25}
Determine whether $\sum_{n=1}^{\infty} \frac{(-1)^n}{5n+1}$ is AC, CC, or D.
\textbf{Solution:}
\begin{enumerate}
    \item \textbf{Test Absolute Convergence:} Consider $\sum_{n=1}^{\infty} \frac{1}{5n+1}$. We use the Limit Comparison Test with the divergent harmonic series $\sum \frac{1}{n}$.
    \[ L = \lim_{n \to \infty} \frac{1/(5n+1)}{1/n} = \lim_{n \to \infty} \frac{n}{5n+1} = \frac{1}{5} \]
    Since $0 < L < \infty$ and $\sum \frac{1}{n}$ diverges, the series $\sum \frac{1}{5n+1}$ also diverges. The original series is not absolutely convergent.
    \item \textbf{Test Conditional Convergence (AST):} Let $b_n = \frac{1}{5n+1}$.
    (i) The denominator is increasing, so $b_n$ is decreasing.
    (ii) $\lim_{n \to \infty} b_n = 0$.
    The original series converges by the AST.
\end{enumerate}
The series is conditionally convergent.
\textbf{Final Answer:} Conditionally convergent.

\subsection*{Problem 26}
Determine whether $\sum_{n=1}^{\infty} \frac{-n}{n^2+1}$ is AC, CC, or D.
\textbf{Solution:} The series is $\sum_{n=1}^{\infty} - \frac{n}{n^2+1}$. This is not an alternating series; all terms are negative. Its convergence is equivalent to the convergence of the positive series $\sum_{n=1}^{\infty} \frac{n}{n^2+1}$. We use the Limit Comparison Test with the divergent harmonic series $\sum \frac{1}{n}$.
\[ L = \lim_{n \to \infty} \frac{n/(n^2+1)}{1/n} = \lim_{n \to \infty} \frac{n^2}{n^2+1} = 1 \]
Since $0 < L < \infty$ and $\sum \frac{1}{n}$ diverges, the series $\sum \frac{n}{n^2+1}$ diverges. Thus the original series also diverges.
\textbf{Final Answer:} Divergent.

\subsection*{Problem 27}
Determine whether $\sum_{n=1}^{\infty} \frac{(-1)^n}{n^2+1}$ is AC, CC, or D.
\textbf{Solution:} This is an alternating series.
\begin{enumerate}
    \item \textbf{Test Absolute Convergence:} Consider $\sum_{n=1}^{\infty} \frac{1}{n^2+1}$. By limit comparison with the convergent p-series $\sum \frac{1}{n^2}$, this series converges (as shown in Problem 24).
\end{enumerate}
Since the series of absolute values converges, the original series is absolutely convergent.
\textbf{Final Answer:} Absolutely convergent.

\subsection*{Problem 28}
Determine whether $\sum_{n=1}^{\infty} \frac{\sin n}{2^n}$ is AC, CC, or D.
\textbf{Solution:} This series has mixed signs but is not strictly alternating. We test for absolute convergence.
Consider $\sum_{n=1}^{\infty} \left| \frac{\sin n}{2^n} \right| = \sum_{n=1}^{\infty} \frac{|\sin n|}{2^n}$.
We know that $0 \le |\sin n| \le 1$ for all $n$. Therefore, $\frac{|\sin n|}{2^n} \le \frac{1}{2^n}$.
The series $\sum_{n=1}^{\infty} \frac{1}{2^n}$ is a convergent geometric series with $|r|=1/2 < 1$.
By the Direct Comparison Test, since $\sum \frac{1}{2^n}$ converges, the series $\sum \frac{|\sin n|}{2^n}$ also converges.
Thus, the original series is absolutely convergent.
\textbf{Final Answer:} Absolutely convergent.

\subsection*{Problem 29}
Determine whether $\sum_{n=1}^{\infty} \frac{1+2\sin n}{n^3}$ is AC, CC, or D.
\textbf{Solution:} This is not an alternating series. We test for absolute convergence. Consider $\sum_{n=1}^{\infty} \left| \frac{1+2\sin n}{n^3} \right| = \sum_{n=1}^{\infty} \frac{|1+2\sin n|}{n^3}$.
Since $-1 \le \sin n \le 1$, we have $-2 \le 2\sin n \le 2$, which implies $-1 \le 1+2\sin n \le 3$.
So, $|1+2\sin n| \le 3$.
Therefore, $\frac{|1+2\sin n|}{n^3} \le \frac{3}{n^3}$.
The series $\sum_{n=1}^{\infty} \frac{3}{n^3} = 3 \sum_{n=1}^{\infty} \frac{1}{n^3}$ is a convergent p-series ($p=3$).
By the Direct Comparison Test, $\sum \frac{|1+2\sin n|}{n^3}$ converges.
The original series is absolutely convergent.
\textbf{Final Answer:} Absolutely convergent.

\subsection*{Problem 30}
Determine whether $\sum_{n=1}^{\infty} (-1)^{n-1} \frac{n}{n^2+4}$ is AC, CC, or D.
\textbf{Solution:}
\begin{enumerate}
    \item \textbf{Test Absolute Convergence:} Consider $\sum_{n=1}^{\infty} \frac{n}{n^2+4}$. Limit compare with the divergent harmonic series $\sum \frac{1}{n}$.
    \[ L = \lim_{n \to \infty} \frac{n/(n^2+4)}{1/n} = \lim_{n \to \infty} \frac{n^2}{n^2+4} = 1 \]
    Since $L=1$ and $\sum \frac{1}{n}$ diverges, the series does not converge absolutely.
    \item \textbf{Test Conditional Convergence (AST):} Let $b_n = \frac{n}{n^2+4}$.
    (i) Limit: $\lim_{n \to \infty} \frac{n}{n^2+4} = 0$.
    (ii) Decreasing: Let $f(x) = \frac{x}{x^2+4}$. $f'(x) = \frac{4-x^2}{(x^2+4)^2}$, which is negative for $x>2$. So $b_n$ is decreasing for $n \ge 3$.
    The original series converges by the AST.
\end{enumerate}
The series is conditionally convergent.
\textbf{Final Answer:} Conditionally convergent.

\subsection*{Problem 31}
Determine whether $\sum_{n=2}^{\infty} \frac{(-1)^n}{\ln n}$ is AC, CC, or D.
\textbf{Solution:}
\begin{enumerate}
    \item \textbf{Test Absolute Convergence:} Consider $\sum_{n=2}^{\infty} \frac{1}{\ln n}$. For $n \ge 2$, we know $\ln n < n$. This implies $\frac{1}{\ln n} > \frac{1}{n}$. Since the harmonic series $\sum \frac{1}{n}$ diverges, by the Direct Comparison Test, $\sum \frac{1}{\ln n}$ also diverges. The series is not absolutely convergent.
    \item \textbf{Test Conditional Convergence (AST):} Let $b_n = \frac{1}{\ln n}$.
    (i) Decreasing: $\ln x$ is increasing, so $1/\ln x$ is decreasing.
    (ii) Limit: $\lim_{n \to \infty} \frac{1}{\ln n} = 0$.
    The original series converges by the AST.
\end{enumerate}
The series is conditionally convergent.
\textbf{Final Answer:} Conditionally convergent.

\subsection*{Problem 32}
Determine whether $\sum_{n=1}^{\infty} (-1)^n \frac{n}{\sqrt{n^3+2}}$ is AC, CC, or D.
\textbf{Solution:}
\begin{enumerate}
    \item \textbf{Test Absolute Convergence:} Consider $\sum_{n=1}^{\infty} \frac{n}{\sqrt{n^3+2}}$. For large $n$, this behaves like $\frac{n}{\sqrt{n^3}} = \frac{n}{n^{3/2}} = \frac{1}{n^{1/2}}$. We use Limit Comparison with the divergent p-series $\sum \frac{1}{\sqrt{n}}$.
    \[ L = \lim_{n \to \infty} \frac{n/\sqrt{n^3+2}}{1/\sqrt{n}} = \lim_{n \to \infty} \frac{n^{3/2}}{\sqrt{n^3+2}} = \lim_{n \to \infty} \sqrt{\frac{n^3}{n^3+2}} = 1 \]
    Since $L=1$ and $\sum \frac{1}{\sqrt{n}}$ diverges, the series does not converge absolutely.
    \item \textbf{Test Conditional Convergence (AST):} Let $b_n = \frac{n}{\sqrt{n^3+2}}$.
    (i) Limit: $\lim_{n \to \infty} \frac{n}{\sqrt{n^3+2}} = \lim_{n \to \infty} \frac{1}{\sqrt{n+2/n^2}} = 0$.
    (ii) Decreasing: Let $f(x) = \frac{x}{\sqrt{x^3+2}}$. The derivative is complicated, but considering the dominant terms, the denominator's power ($3/2$) grows faster than the numerator's power (1), so the terms will eventually decrease.
\end{enumerate}
The series is conditionally convergent.
\textbf{Final Answer:} Conditionally convergent.

\subsection*{Problem 33}
Determine whether $\sum_{n=1}^{\infty} \frac{\cos(n\pi)}{3n+2}$ is AC, CC, or D.
\textbf{Solution:} Since $\cos(n\pi) = (-1)^n$, the series is $\sum_{n=1}^{\infty} \frac{(-1)^n}{3n+2}$.
\begin{enumerate}
    \item \textbf{Test Absolute Convergence:} Consider $\sum_{n=1}^{\infty} \frac{1}{3n+2}$. Limit compare with the divergent harmonic series $\sum \frac{1}{n}$.
    \[ L = \lim_{n \to \infty} \frac{1/(3n+2)}{1/n} = \lim_{n \to \infty} \frac{n}{3n+2} = \frac{1}{3} \]
    Since $L=1/3$ and $\sum \frac{1}{n}$ diverges, the series is not absolutely convergent.
    \item \textbf{Test Conditional Convergence (AST):} Let $b_n = \frac{1}{3n+2}$.
    (i) Decreasing: The denominator is increasing.
    (ii) Limit: $\lim_{n \to \infty} \frac{1}{3n+2} = 0$.
    The original series converges by the AST.
\end{enumerate}
The series is conditionally convergent.
\textbf{Final Answer:} Conditionally convergent.

\subsection*{Problem 34}
Determine whether $\sum_{n=2}^{\infty} \frac{(-1)^n}{n \ln n}$ is AC, CC, or D.
\textbf{Solution:}
\begin{enumerate}
    \item \textbf{Test Absolute Convergence:} Consider $\sum_{n=2}^{\infty} \frac{1}{n \ln n}$. We use the Integral Test. Let $f(x) = \frac{1}{x \ln x}$. It is positive, continuous, and decreasing for $x \ge 2$.
    \[ \int_2^\infty \frac{1}{x \ln x} dx = \lim_{t \to \infty} [\ln(\ln x)]_2^t = \lim_{t \to \infty} (\ln(\ln t) - \ln(\ln 2)) = \infty \]
    Since the integral diverges, the series of absolute values diverges.
    \item \textbf{Test Conditional Convergence (AST):} Let $b_n = \frac{1}{n \ln n}$.
    (i) Decreasing: Since $n$ and $\ln n$ are increasing, their product is increasing, so the reciprocal is decreasing.
    (ii) Limit: $\lim_{n \to \infty} \frac{1}{n \ln n} = 0$.
    The original series converges by the AST.
\end{enumerate}
The series is conditionally convergent.
\textbf{Final Answer:} Conditionally convergent.

\subsection*{Problem 35}
For $\sum_{n=1}^{\infty} \frac{(-0.8)^n}{n!}$, graph terms and partial sums. Estimate the sum correct to four decimal places.
\textbf{Solution:} The series is $\sum_{n=1}^{\infty} (-1)^n \frac{(0.8)^n}{n!}$.
The error is bounded by $b_{n+1}$. We want $|R_n| \le b_{n+1} < 0.00005$.
$b_1 = 0.8/1! = 0.8$
$b_2 = (0.8)^2/2! = 0.32$
$b_3 = (0.8)^3/3! \approx 0.08533$
$b_4 = (0.8)^4/4! \approx 0.01707$
$b_5 = (0.8)^5/5! \approx 0.00273$
$b_6 = (0.8)^6/6! \approx 0.00036$
$b_7 = (0.8)^7/7! \approx 0.0000416 < 0.00005$.
We need to sum the first 6 terms ($s_6$).
$s_6 = -b_1 + b_2 - b_3 + b_4 - b_5 + b_6 \approx -0.8 + 0.32 - 0.085333 + 0.017067 - 0.002731 + 0.000364 \approx -0.55063$.
\textbf{Final Answer:} The sum is approximately -0.5506. (Note: $\sum_{n=0}^\infty \frac{x^n}{n!} = e^x$, so this sum is $e^{-0.8}-1 \approx 0.4493-1 = -0.5507$).

\subsection*{Problem 36}
For $\sum_{n=1}^{\infty} \frac{(-1)^{n-1}}{n 8^n}$, estimate the sum correct to four decimal places.
\textbf{Solution:} We want $|R_n| \le b_{n+1} < 0.00005$.
$b_n = \frac{1}{n 8^n}$.
$b_1 = 1/8 = 0.125$
$b_2 = 1/(2 \cdot 8^2) = 1/128 \approx 0.00781$
$b_3 = 1/(3 \cdot 8^3) = 1/1536 \approx 0.00065$
$b_4 = 1/(4 \cdot 8^4) = 1/16384 \approx 0.000061 > 0.00005$
$b_5 = 1/(5 \cdot 8^5) = 1/163840 \approx 0.000006 < 0.00005$.
We need to sum the first 4 terms ($s_4$).
$s_4 = b_1 - b_2 + b_3 - b_4 \approx 0.125 - 0.0078125 + 0.000651 - 0.000061 \approx 0.11778$.
\textbf{Final Answer:} The sum is approximately 0.1178.

\subsection*{Problem 37}
How many terms of $\sum_{n=1}^{\infty} \frac{(-1)^{n+1}}{n^4}$ are needed for error $< 0.00005$?
\textbf{Solution:} We need $|R_n| \le b_{n+1} < 0.00005$.
$b_{n+1} = \frac{1}{(n+1)^4}$.
$\frac{1}{(n+1)^4} < 0.00005 \implies (n+1)^4 > \frac{1}{0.00005} = 20000$.
$n+1 > (20000)^{1/4}$. Since $10^4=10000$ and $12^4=20736$, $(20000)^{1/4}$ is just under 12. Let's check $n+1=12$.
$12^4 = 20736 > 20000$. So we can take $n+1=12$, which means $n=11$.
\textbf{Final Answer:} 11 terms.

\subsection*{Problem 38}
How many terms of $\sum_{n=1}^{\infty} \frac{(-1)^n}{n^2}$ are needed for error $< 0.0005$?
\textbf{Solution:} We need $|R_n| \le b_{n+1} < 0.0005$.
$b_{n+1} = \frac{1}{(n+1)^2}$.
$\frac{1}{(n+1)^2} < 0.0005 \implies (n+1)^2 > \frac{1}{0.0005} = 2000$.
$n+1 > \sqrt{2000}$. Since $40^2=1600$ and $45^2=2025$, $\sqrt{2000} \approx 44.7$.
We need $n+1 \ge 45$, so $n=44$.
\textbf{Final Answer:} 44 terms.

\subsection*{Problem 39}
How many terms of $\sum_{n=1}^{\infty} \frac{(-1)^{n-1}}{n^2 2^n}$ are needed for error $< 0.0005$?
\textbf{Solution:} We need $|R_n| \le b_{n+1} < 0.0005$.
$b_{n+1} = \frac{1}{(n+1)^2 2^{n+1}}$.
$n=1: b_2 = 1/(2^2 \cdot 2^2) = 1/16 = 0.0625$
$n=2: b_3 = 1/(3^2 \cdot 2^3) = 1/72 \approx 0.0138$
$n=3: b_4 = 1/(4^2 \cdot 2^4) = 1/256 \approx 0.0039$
$n=4: b_5 = 1/(5^2 \cdot 2^5) = 1/800 = 0.00125$
$n=5: b_6 = 1/(6^2 \cdot 2^6) = 1/2304 \approx 0.00043 < 0.0005$.
We need $b_{n+1}$ to be the first term smaller than the error. This is $b_6$, so we need to sum up to $n=5$.
\textbf{Final Answer:} 5 terms.

\subsection*{Problem 40}
How many terms of $\sum_{n=1}^{\infty} \frac{(-1)^n n}{4^n}$ are needed for error $< 0.00005$?
\textbf{Solution:} We need $|R_n| \le b_{n+1} < 0.00005$.
$b_{n+1} = \frac{n+1}{4^{n+1}}$.
$n=1: b_2 = 2/16 = 0.125$
$n=2: b_3 = 3/64 \approx 0.0468$
$n=3: b_4 = 4/256 \approx 0.0156$
$n=4: b_5 = 5/1024 \approx 0.00488$
$n=5: b_6 = 6/4096 \approx 0.00146$
$n=6: b_7 = 7/16384 \approx 0.000427$
$n=7: b_8 = 8/65536 \approx 0.000122$
$n=8: b_9 = 9/262144 \approx 0.000034 < 0.00005$.
We need $b_{n+1}$ to be the first term smaller than the error. This is $b_9$, so we need to sum up to $n=8$.
\textbf{Final Answer:} 8 terms.

\subsection*{Problem 41}
Approximate $\sum_{n=1}^{\infty} \frac{(-1)^n}{(2n)!}$ to four decimal places (error $< 0.00005$).
\textbf{Solution:} We need $b_{n+1} = \frac{1}{(2(n+1))!} < 0.00005$.
$b_1 = 1/2! = 0.5$
$b_2 = 1/4! = 1/24 \approx 0.04167$
$b_3 = 1/6! = 1/720 \approx 0.00139$
$b_4 = 1/8! = 1/40320 \approx 0.0000248 < 0.00005$.
We need to sum the first 3 terms ($s_3$).
$s_3 = a_1 + a_2 + a_3 = -b_1 + b_2 - b_3 = -1/2 + 1/24 - 1/720 \approx -0.5 + 0.041667 - 0.001389 = -0.459722$.
\textbf{Final Answer:} -0.4597.

\subsection*{Problem 42}
Approximate $\sum_{n=1}^{\infty} \frac{(-1)^{n+1}}{n^6}$ to four decimal places.
\textbf{Solution:} We need $b_{n+1} = \frac{1}{(n+1)^6} < 0.00005$.
$(n+1)^6 > 20000$.
$n+1 > (20000)^{1/6}$. $4^6=4096$, $5^6=15625$, $6^6=46656$. So $n+1$ must be 6.
This means $n=5$. We need to sum the first 5 terms.
$s_5 = 1/1^6 - 1/2^6 + 1/3^6 - 1/4^6 + 1/5^6 = 1 - 1/64 + 1/729 - 1/4096 + 1/15625 \approx 1 - 0.015625 + 0.001372 - 0.000244 + 0.000064 = 0.985567$.
\textbf{Final Answer:} 0.9856.

\subsection*{Problem 43}
Approximate $\sum_{n=1}^{\infty} (-1)^n n e^{-2n}$ to four decimal places.
\textbf{Solution:} $b_n = n/e^{2n}$. We need $b_{n+1} = \frac{n+1}{e^{2(n+1)}} < 0.00005$.
$b_1 = 1/e^2 \approx 0.1353$
$b_2 = 2/e^4 \approx 0.0366$
$b_3 = 3/e^6 \approx 0.0074$
$b_4 = 4/e^8 \approx 0.0013$
$b_5 = 5/e^{10} \approx 0.000227$
$b_6 = 6/e^{12} \approx 0.000037 < 0.00005$.
We need to sum the first 5 terms ($s_5$).
$s_5 = -b_1+b_2-b_3+b_4-b_5 \approx -0.135335 + 0.036631 - 0.007447 + 0.001342 - 0.000227 = -0.105036$.
\textbf{Final Answer:} -0.1050.

\subsection*{Problem 44}
Approximate $\sum_{n=1}^{\infty} \frac{(-1)^{n-1}}{n 4^n}$ to four decimal places.
\textbf{Solution:} $b_n = \frac{1}{n 4^n}$. We need $b_{n+1} < 0.00005$.
$b_1 = 1/4 = 0.25$
$b_2 = 1/(2 \cdot 16) = 1/32 = 0.03125$
$b_3 = 1/(3 \cdot 64) = 1/192 \approx 0.0052$
$b_4 = 1/(4 \cdot 256) = 1/1024 \approx 0.00097$
$b_5 = 1/(5 \cdot 1024) = 1/5120 \approx 0.000195$
$b_6 = 1/(6 \cdot 4096) = 1/24576 \approx 0.0000407 < 0.00005$.
We need to sum the first 5 terms ($s_5$).
$s_5 = b_1 - b_2 + b_3 - b_4 + b_5 \approx 0.25 - 0.03125 + 0.005208 - 0.000977 + 0.000195 = 0.223176$.
\textbf{Final Answer:} 0.2232.

\subsection*{Problem 45}
Is $s_{50}$ of $\sum_{n=1}^\infty (-1)^{n-1}/n$ an overestimate or an underestimate?
\textbf{Solution:} The sum is $s = s_{50} + R_{50} = s_{50} + a_{51} + a_{52} + \dots$.
The true sum $s$ lies between any two consecutive partial sums, $s_n$ and $s_{n+1}$.
Here, $s_{50} = 1 - 1/2 + \dots - 1/50$. The next term is $a_{51} = +1/51$. So $s_{51} = s_{50} + 1/51 > s_{50}$.
The odd partial sums are decreasing and the even partial sums are increasing. The true sum $s$ is between them.
$s_{50} < s < s_{51}$.
Therefore, $s_{50}$ is an underestimate of the total sum.
\textbf{Final Answer:} Underestimate, because the first neglected term, $a_{51}$, is positive, so the sum must increase from $s_{50}$ to get closer to $s$.

\subsection*{Problem 46}
For what values of $p$ is $\sum_{n=1}^{\infty} \frac{(-1)^{n-1}}{n^p}$ convergent?
\textbf{Solution:} This is an alternating series with $b_n = 1/n^p$.
For the AST to apply, we need $\lim_{n \to \infty} b_n = 0$. This is true if $p>0$.
If $p>0$, $b_n = 1/n^p$ is a decreasing sequence.
So, by the AST, the series converges for all $p>0$.
We can also consider absolute convergence. $\sum |a_n| = \sum 1/n^p$ is a p-series, which converges for $p>1$.
So, it converges absolutely for $p>1$ and conditionally for $0 < p \le 1$.
\textbf{Final Answer:} Convergent for $p>0$.

\subsection*{Problem 47}
For what values of $p$ is $\sum_{n=1}^{\infty} \frac{(-1)^n}{n+p}$ convergent?
\textbf{Solution:} This is an alternating series with $b_n = \frac{1}{n+p}$. We need $n+p>0$ for all $n$, so $p$ can be any real number as long as we start $n$ large enough. Assuming $n+p \neq 0$.
\begin{enumerate}
    \item Limit: $\lim_{n \to \infty} b_n = \lim_{n \to \infty} \frac{1}{n+p} = 0$ for any $p$.
    \item Decreasing: $b_{n+1} = \frac{1}{n+1+p}$. Since $n+1+p > n+p$, we have $b_{n+1} < b_n$.
\end{enumerate}
By the AST, the series converges for all values of $p$ (assuming $n+p$ is never zero for any $n$).
\textbf{Final Answer:} All real numbers $p$, provided $p$ is not a negative integer.

\subsection*{Problem 48}
For what values of $p$ is $\sum_{n=2}^{\infty} (-1)^{n-1} \frac{(\ln n)^p}{n}$ convergent?
\textbf{Solution:} This is an alternating series with $b_n = \frac{(\ln n)^p}{n}$.
\begin{enumerate}
    \item Limit: We need $\lim_{n \to \infty} b_n = 0$. Let $f(x) = \frac{(\ln x)^p}{x}$. Using L'Hopital's rule (repeatedly if $p>1$), we can show that this limit is 0 for any real number $p$.
    \item Decreasing: Let's check the derivative of $f(x)$.
    \[ f'(x) = \frac{p(\ln x)^{p-1}(1/x) \cdot x - (\ln x)^p \cdot 1}{x^2} = \frac{(\ln x)^{p-1}(p-\ln x)}{x^2} \]
    For the terms to be decreasing, we need $f'(x) < 0$. This happens when $p - \ln x < 0$, or $\ln x > p$. This means $x > e^p$. So, the sequence is eventually decreasing for any value of $p$.
\end{enumerate}
By the AST, the series converges for all real numbers $p$.
\textbf{Final Answer:} All real numbers $p$.

\subsection*{Problem 49}
Show that the series $\sum (-1)^{n-1}b_n$ with $b_n = 1/n$ if $n$ is odd and $b_n=1/n^2$ if $n$ is even, is divergent. Why does the AST not apply?
\textbf{Solution:} The terms are $1, -1/4, 1/3, -1/16, 1/5, -1/36, \dots$.
The condition $\lim_{n \to \infty} b_n = 0$ is satisfied, since both $1/n \to 0$ and $1/n^2 \to 0$.
However, the decreasing condition $b_{n+1} \le b_n$ is not satisfied for all $n$.
Let's check for $n=2$ (even): $b_2 = 1/2^2 = 1/4$. $b_3 = 1/3$. Is $b_3 \le b_2$? Yes, $1/3 \le 1/4$ is false.
Let's check for $n=3$ (odd): $b_3 = 1/3$. $b_4 = 1/4^2 = 1/16$. Is $b_4 \le b_3$? Yes, $1/16 \le 1/3$.
The condition fails whenever we go from an even $n$ to an odd $n+1$. For instance, $b_{2k+1} = \frac{1}{2k+1}$ and $b_{2k} = \frac{1}{(2k)^2}$. For large $k$, $\frac{1}{2k+1} > \frac{1}{4k^2}$, so $b_{2k+1} > b_{2k}$. The sequence is not decreasing. Thus the AST does not apply.
To show divergence, consider the even partial sums:
$s_{2N} = \sum_{n=1}^{2N} a_n = (1-\frac{1}{4}) + (\frac{1}{3}-\frac{1}{16}) + \dots + (\frac{1}{2N-1} - \frac{1}{(2N)^2})$.
This is a sum of positive terms. The series $\sum (\frac{1}{2n-1} - \frac{1}{4n^2})$ diverges by limit comparison with $\sum 1/n$. Thus, the sequence of even partial sums diverges, and so the series diverges.
\textbf{Final Answer:} The series diverges. The AST does not apply because the sequence $b_n$ is not decreasing.

\subsection*{Problem 50}
Use the steps to show $\sum_{n=1}^\infty \frac{(-1)^{n-1}}{n} = \ln 2$.
\textbf{Solution:}
(a) Show $s_{2n} = h_{2n} - h_n$.
$s_{2n} = (1 - 1/2) + (1/3 - 1/4) + \dots + (1/(2n-1) - 1/(2n))$
$s_{2n} = (1+1/3+\dots+1/(2n-1)) - (1/2+1/4+\dots+1/(2n))$
$h_{2n} = 1+1/2+1/3+1/4+\dots+1/(2n-1)+1/(2n)$
$h_{2n} = (1+1/3+\dots+1/(2n-1)) + (1/2+1/4+\dots+1/(2n))$
$h_n = 1/1+1/2+\dots+1/n$. Then $h_n/2$ is not quite right. Let's do it directly.
$s_{2n} = \sum_{k=1}^{2n} \frac{(-1)^{k-1}}{k} = 1 - \frac{1}{2} + \frac{1}{3} - \frac{1}{4} + \dots + \frac{1}{2n-1} - \frac{1}{2n}$
$h_{2n} = 1 + \frac{1}{2} + \frac{1}{3} + \dots + \frac{1}{2n}$
$h_n = 1 + \frac{1}{2} + \dots + \frac{1}{n}$
$h_{2n} - h_n = (\frac{1}{n+1} + \frac{1}{n+2} + \dots + \frac{1}{2n})$. This is not $s_{2n}$.
Let's try another way for (a).
$s_{2n} = (1+\frac{1}{2}+\frac{1}{3}+\dots+\frac{1}{2n}) - 2(\frac{1}{2}+\frac{1}{4}+\dots+\frac{1}{2n}) = h_{2n} - (1+\frac{1}{2}+\dots+\frac{1}{n}) = h_{2n} - h_n$. This is correct.

(b) Given $h_n - \ln n \to \gamma$ as $n \to \infty$. So $h_{2n} - \ln(2n) \to \gamma$.
We have $s_{2n} = h_{2n} - h_n$.
$s_{2n} = (h_{2n} - \ln(2n)) - (h_n - \ln n) + \ln(2n) - \ln n$.
As $n \to \infty$, $h_{2n}-\ln(2n) \to \gamma$ and $h_n - \ln n \to \gamma$.
So, $\lim_{n \to \infty} s_{2n} = (\gamma - \gamma) + \lim_{n \to \infty} (\ln(2n)-\ln n) = \lim_{n \to \infty} \ln(\frac{2n}{n}) = \ln 2$.
Since the odd partial sums must converge to the same limit, $\lim s_n = \ln 2$.
\textbf{Final Answer:} The steps show that the sum of the alternating harmonic series is $\ln 2$.

\subsection*{Problem 51}
Given $\sum a_n$, let $a_n^+ = (a_n+|a_n|)/2$ and $a_n^- = (a_n-|a_n|)/2$.
\textbf{(a) If $\sum a_n$ is AC, show $\sum a_n^+$ and $\sum a_n^-$ are convergent.}
\textbf{Solution:} If $\sum a_n$ is AC, then $\sum |a_n|$ converges.
We know $a_n^+ = (a_n+|a_n|)/2$ and $a_n^- = (a_n-|a_n|)/2$.
So $\sum a_n^+ = \frac{1}{2} \sum (a_n + |a_n|)$ and $\sum a_n^- = \frac{1}{2} \sum (a_n - |a_n|)$.
Since $\sum a_n$ converges (because it's AC) and $\sum |a_n|$ converges, their sum and difference must also converge.
$\sum(a_n + |a_n|)$ converges. Therefore $\sum a_n^+$ converges.
$\sum(a_n - |a_n|)$ converges. Therefore $\sum a_n^-$ converges.

\textbf{(b) If $\sum a_n$ is CC, show $\sum a_n^+$ and $\sum a_n^-$ are divergent.}
\textbf{Solution:} If $\sum a_n$ is CC, then $\sum a_n$ converges but $\sum |a_n|$ diverges.
Assume, for contradiction, that $\sum a_n^+$ converges.
Then the series $\sum (a_n + |a_n|) = 2 \sum a_n^+$ must also converge.
We know $\sum a_n$ converges. If we subtract the convergent series $\sum a_n$ from the convergent series $\sum(a_n+|a_n|)$, the result must converge.
$\sum(a_n+|a_n|) - \sum a_n = \sum |a_n|$.
This would imply that $\sum |a_n|$ converges, which contradicts the premise that the series is CC. So the assumption must be false. $\sum a_n^+$ diverges.
A similar argument holds for $\sum a_n^-$. Assume it converges.
Then $\sum(a_n-|a_n|)$ converges.
The sum of two convergent series must converge: $\sum a_n + \sum(a_n-|a_n|) = \sum(2a_n - |a_n|)$. This doesn't help.
Let's try this: $|a_n| = a_n^+ - a_n^-$. $\sum |a_n| = \sum (a_n^+ - a_n^-)$.
If both $\sum a_n^+$ and $\sum a_n^-$ were convergent, then their difference $\sum |a_n|$ would have to converge, a contradiction.
Since we already showed $\sum a_n^+$ diverges, we cannot conclude anything about $\sum a_n^-$ from this.
Let's use: $a_n = a_n^+ + a_n^-$. $\sum a_n = \sum a_n^+ + \sum a_n^-$.
If $\sum a_n^+$ diverges to $\infty$ (it is a series of positive terms), and $\sum a_n$ converges to a finite number $L$, then $\sum a_n^-$ must diverge to $-\infty$ to compensate.
Therefore, both series must diverge.

\textbf{Final Answer:} The proofs show that for an AC series, the positive and negative parts both converge. For a CC series, they both diverge.

\part{In-Depth Analysis of Problems and Techniques}

\section{Problem Types and General Approach}
\begin{description}
    \item[Type 1: Direct Application of the Alternating Series Test (AST)] (Problems 2, 4, 5, 6, 9-11, 15, 17, 20). These problems present a clear alternating series. The approach is to identify $b_n$ and verify the two conditions: $b_n$ is decreasing and $\lim_{n \to \infty} b_n = 0$.
    
    \item[Type 2: Divergence by the Test for Divergence] (Problems 3, 7, 8, 13, 14, 18). The series is alternating, but $\lim_{n \to \infty} b_n \neq 0$. This is the quickest way to establish divergence for an alternating series.
    
    \item[Type 3: Absolute vs. Conditional Convergence Analysis] (Problems 22-25, 27, 30-34). This requires a two-stage process. First, test $\sum |a_n|$ for convergence (using p-series, comparison, LCT, integral test, etc.). If it converges, the series is AC. If not, apply the AST to the original series $\sum a_n$ to check for conditional convergence.
    
    \item[Type 4: Absolute Convergence via Ratio Test] (Problems 12, 16, 19). For series involving factorials or $k^n$ terms, the Ratio Test applied to $\sum|a_n|$ is the most efficient method to establish absolute convergence.
    
    \item[Type 5: Non-Alternating Series with Mixed Signs] (Problems 26, 28, 29). These are not strictly alternating. The primary strategy is to test for absolute convergence using the Comparison Test, often by bounding terms like $|\sin n| \le 1$. Problem 26 was a series of all negative terms, whose convergence is equivalent to its positive counterpart.
        
    \item[Type 6: Error Estimation and Approximation] (Problems 35-44). These use the Alternating Series Estimation Theorem, $|R_n| \le b_{n+1}$. The task is to either find $n$ such that $b_{n+1}$ is less than a given error, or to calculate the partial sum $s_n$ once $n$ is found.
    
    \item[Type 7: Theoretical and Parameter-Based Problems] (Problems 1, 21, 45-51). These questions test definitions, proofs, and the influence of parameters. They require a solid grasp of the underlying theorems and concepts.
\end{description}

\section{Key Algebraic and Calculus Manipulations}
\begin{description}
    \item[Proving $b_n$ is Decreasing:]
    \begin{itemize}
        \item \textbf{Function Derivative:} The most robust method, used for terms like $\frac{\sqrt{x}}{2x+3}$ (Problem 10) and $\frac{x^2}{x^3+4}$ (Problem 11). This often requires the quotient rule.
    \end{itemize}
    
    \item[Trigonometric Identities:]
    \begin{itemize}
        \item Recognizing that $\cos(n\pi) = (-1)^n$ (Problems 16, 33) and $\sin(n\pi + \pi/2) = (-1)^n$ (Problem 15) is essential to identify the series as alternating.
    \end{itemize}
    
    \item[Multiplying by the Conjugate:] A key step for simplifying terms involving differences of square roots, as seen in Problem 20 with $b_n = \sqrt{n+1} - \sqrt{n}$.
    
    \item[Limit Comparison Test for Absolute Convergence:] The workhorse for Stage 1 analysis. Used in Problems 24, 25, 30, 32, 33 to compare with p-series.
    
    \item[Direct Comparison Test for Absolute Convergence:] Crucial for terms with bounded functions like $|\sin n| \le 1$, as in Problems 28 and 29. Also used to show divergence in Problem 31 ($\frac{1}{\ln n} > \frac{1}{n}$).
    
    \item[The Integral Test for Absolute Convergence:] A necessary tool for terms like $\frac{1}{n \ln n}$ (Problem 34), where other comparison tests are less obvious.

    \item[Solving Error Inequalities:] Often requires testing values of $n$ (Problems 35, 36, 39, 40, 41, 43, 44) or solving polynomial inequalities like $(n+1)^4 > 20000$ (Problem 37).
\end{description}

\part{"Cheatsheet" and Tips for Success}

\section{Summary of Formulas and Definitions}
\begin{itemize}
    \item \textbf{Alternating Series Test (AST):} Converges if (1) $b_{n+1} \le b_n$ (decreasing) and (2) $\lim_{n \to \infty} b_n = 0$.
    \item \textbf{Error Bound:} $|s - s_n| \le b_{n+1}$. The error is smaller than the first unused term.
    \item \textbf{Absolutely Convergent (AC):} $\sum |a_n|$ converges.
    \item \textbf{Conditionally Convergent (CC):} $\sum a_n$ converges, but $\sum |a_n|$ diverges.
    \item \textbf{Key Theorem:} Absolute Convergence $\implies$ Convergence.
\end{itemize}

\section{Cheats, Tricks, and Shortcuts}
\begin{itemize}
    \item \textbf{First Check Rule:} Always check the Test for Divergence first! For any series, alternating or not, if $\lim_{n \to \infty} a_n \neq 0$, you are done. It diverges. This is the fastest way to solve problems like $\sum (-1)^n \frac{n}{2n+1}$.
    \item \textbf{AC/CC Workflow:} When asked to classify a series, always test for absolute convergence first. It's often a more straightforward positive-series test. If it's AC, you don't need to run the AST.
    \item \textbf{Ratio Test is Your Friend:} For anything with factorials ($n!$) or exponentials ($k^n$), immediately try the Ratio Test for absolute convergence. It's almost always the intended and easiest path.
    \item \textbf{Over/Underestimate:} The sign of the first neglected term tells you if the approximation $s_n$ is an overestimate or an underestimate. If the first neglected term, $a_{n+1}$, is positive, then $s_n < s$ (underestimate). If $a_{n+1}$ is negative, then $s_n > s$ (overestimate). Problem 45 is a classic example.
\end{itemize}

\section{Common Pitfalls and How to Avoid Them}
\begin{itemize}
    \item \textbf{Forgetting the Decreasing Condition:} A common error is to only check that $\lim b_n = 0$ and conclude convergence. You MUST also verify that the terms are decreasing. If not, the AST does not apply (see Problem 49).
    \item \textbf{Stopping After Absolute Test Fails:} If $\sum |a_n|$ diverges, you are NOT done. You have only ruled out absolute convergence. You must still test the original series $\sum a_n$ (usually with the AST) to check for conditional convergence.
    \item \textbf{Applying Error Bound to Non-Alternating Series:} The remainder estimate $|R_n| \le b_{n+1}$ is a special property of alternating series that meet the AST conditions. It cannot be used for other types of series.
    \item \textbf{Confusing $b_n$ and $b_{n+1}$ in Error Problems:} The error after summing $n$ terms (i.e., calculating $s_n$) is bounded by $b_{n+1}$, the \textit{next} term.
\end{itemize}

\part{Conceptual Synthesis and The "Big Picture"}

\section{Thematic Connections}
The core theme of this topic is \textbf{the surprisingly complex and delicate nature of summing infinitely many numbers.} While our intuition, built on finite addition, suggests that order shouldn't matter, this chapter shows that for infinite sums, it is paramount. The distinction between absolute and conditional convergence is the key. Absolutely convergent series are robust; they behave like finite sums in that they obey a form of the commutative property (rearrangements don't change the sum). Conditionally convergent series are fragile; their sum is conditioned on the specific order in which the terms are presented. This theme connects directly to our study of improper integrals, where the convergence of $\int_1^\infty |f(x)| dx$ is a stronger condition than the convergence of $\int_1^\infty f(x) dx$. It's a fundamental lesson in mathematical analysis: infinity requires a more careful and rigorous approach than the finite world.

\section{Forward and Backward Links}
\begin{description}
    \item[Backward Links:] This chapter is the culmination of our study of sequences and series. It is impossible to analyze absolute convergence without a firm grasp of the convergence tests for positive-term series (Comparison, Limit Comparison, Integral, Ratio Tests). The very foundation of the AST relies on the Monotonic Sequence Theorem (a bounded, monotonic sequence converges), which links back to the fundamental properties of sequences.
    
    \item[Forward Links:] Understanding absolute convergence is the \textbf{absolute prerequisite} for the next major topic: \textbf{Power Series and Taylor Series}. A power series is a function defined by an infinite series, like $f(x) = \sum_{n=0}^\infty c_n x^n$. The central question is, "For which values of $x$ does this series converge?" We answer this by testing for \textit{absolute convergence} using the Ratio Test. This yields an "interval of convergence." The behavior at the endpoints of this interval often requires checking a specific numeric series for conditional convergence, frequently using the Alternating Series Test. Thus, the skills from this section are not just a separate topic; they are the essential tools for building and understanding approximations of functions, which is a cornerstone of modern science, engineering, and numerical analysis.
\end{description}

\part{Real-World Application and Modeling}

\section{Concrete Scenarios in Finance and Economics}
\begin{description}
    \item[1. Net Present Value (NPV) of Volatile Projects:] In corporate finance, a project's value is its NPV, the sum of all future discounted cash flows. Some projects, especially in R\&D or resource exploration, involve years of heavy investment (negative cash flow) followed by years of revenue (positive cash flow), which may then be interspersed with further costs for upgrades or environmental remediation. The total cash flow stream can be an alternating series. Determining if the project has a finite, positive value requires summing this series. Convergence implies a stable, predictable long-term value; divergence could imply untenable risk.
    
    \item[2. Valuing Derivatives with Oscillating Payoffs:] In quantitative finance, complex derivatives can have payoffs that depend on an asset price crossing certain barriers. For example, a "binary range option" might pay out \$1 if the asset stays within a range and cost \$1 if it moves out. If the probability of being in or out of the range oscillates over time (e.g., in a mean-reverting market), the expected value of the derivative could be modeled as an alternating series of expected payoffs over many time steps. The sum of this series gives the fair price of the derivative today.
    
    \item[3. Economic Shock Propagation (Impulse-Response):] In econometrics, an impulse-response function models how a one-time shock to the economy (like a sudden interest rate hike) propagates over time. Often, the economy overcorrects and oscillates. The effect in quarter 1 might be -0.5\%, in quarter 2 +0.2\%, in quarter 3 -0.1\%, and so on. This is an alternating series. If the series converges absolutely, the shock is stable and its total long-term impact is finite and independent of other shocks. If it converges conditionally, the total impact might depend on how it interacts with other economic events, suggesting a less stable system.
\end{description}

\section{Model Problem Setup: NPV of an Alternating Project}
\begin{description}
    \item[Scenario:] A tech company is developing a new AI platform. The project has an initial revenue burst, followed by alternating years of server upgrade costs and new subscription revenues.
    \item[Problem:] In year 1, the platform generates \$1 million. In year 2, it requires a \$0.5 million server upgrade. In year 3, it brings in \$0.333 million in new revenue. In year 4, it costs \$0.25 million for another upgrade. This pattern continues, with cash flow in year $n$ being $(-1)^{n-1} \frac{\$1,000,000}{n}$. The firm uses a discount rate of 10\% to value future cash flows. What is the Net Present Value (NPV) of the entire project over its infinite lifetime?
    \item[Model Setup:]
    \begin{itemize}
        \item \textbf{Variables:}
        $C_n$: Cash flow in year $n$, $C_n = (-1)^{n-1} \frac{1,000,000}{n}$.
        $r$: Discount rate, $r = 0.10$.
        $PV(C_n)$: Present value of the cash flow in year $n$.
        
        \item \textbf{Function:} The present value of a future cash flow is given by $PV(C_n) = \frac{C_n}{(1+r)^n}$.
        
        \item \textbf{Equation to Solve:} The NPV is the sum of all discounted cash flows.
        \[ NPV = \sum_{n=1}^{\infty} PV(C_n) = \sum_{n=1}^{\infty} \frac{(-1)^{n-1} \frac{1,000,000}{n}}{(1+0.10)^n} \]
        \[ NPV = 1,000,000 \sum_{n=1}^{\infty} \frac{(-1)^{n-1}}{n(1.1)^n} \]
        To find the value, we must determine if this series converges. It is an alternating series with $b_n = \frac{1}{n(1.1)^n}$. It converges by the AST since $b_n$ is clearly decreasing and its limit is 0. Furthermore, $\sum|a_n| = \sum \frac{1}{n(1.1)^n}$ converges by comparison with the convergent geometric series $\sum \frac{1}{(1.1)^n}$. Therefore, the project has a finite, stable value (it converges absolutely).
    \end{itemize}
\end{description}

\part{Common Variations and Untested Concepts}

The provided homework set is comprehensive, but it focuses heavily on comparison tests and the AST. Here are two powerful tests, common in standard curricula, that were not explicitly required but are useful for many of the problems.

\section{The Ratio Test for Absolute Convergence}
The Ratio Test is extremely useful for series involving factorials ($n!$) or $n$-th powers ($k^n$). For a series $\sum a_n$, we compute:
\[ L = \lim_{n \to \infty} \left| \frac{a_{n+1}}{a_n} \right| \]
\begin{itemize}
    \item If $L < 1$, the series is absolutely convergent.
    \item If $L > 1$ (or $L=\infty$), the series is divergent.
    \item If $L = 1$, the test is inconclusive.
\end{itemize}

\textbf{Worked-out Example (Alternative for Problem 19):} Test the series $\sum_{n=1}^{\infty} (-1)^n \frac{n^2}{5^n}$ for convergence.
\textbf{Solution:} We test for absolute convergence using the Ratio Test.
Let $a_n = (-1)^n \frac{n^2}{5^n}$.
\begin{align*}
    L &= \lim_{n \to \infty} \left| \frac{a_{n+1}}{a_n} \right| = \lim_{n \to \infty} \left| \frac{(-1)^{n+1} (n+1)^2 / 5^{n+1}}{(-1)^n n^2 / 5^n} \right| \\
    &= \lim_{n \to \infty} \left( \frac{(n+1)^2}{n^2} \cdot \frac{5^n}{5^{n+1}} \right) \\
    &= \lim_{n \to \infty} \left( \left( 1 + \frac{1}{n} \right)^2 \cdot \frac{1}{5} \right) = (1)^2 \cdot \frac{1}{5} = \frac{1}{5}
\end{align*}
Since $L = 1/5 < 1$, the series is absolutely convergent.

\section{The Root Test for Absolute Convergence}
The Root Test is most effective for series involving terms raised to the $n$-th power. For a series $\sum a_n$, we compute:
\[ L = \lim_{n \to \infty} \sqrt[n]{|a_n|} \]
The conclusions for $L$ are the same as for the Ratio Test.

\textbf{Worked-out Example:} Test the series $\sum_{n=1}^{\infty} \left( \frac{-2n}{3n+1} \right)^n$ for convergence.
\textbf{Solution:} The $n$-th power form is a perfect signal for the Root Test. We test for absolute convergence.
Let $a_n = \left( \frac{-2n}{3n+1} \right)^n$.
\begin{align*}
    L &= \lim_{n \to \infty} \sqrt[n]{|a_n|} = \lim_{n \to \infty} \sqrt[n]{\left| \left( \frac{-2n}{3n+1} \right)^n \right|} \\
    &= \lim_{n \to \infty} \sqrt[n]{\left( \frac{2n}{3n+1} \right)^n} \\
    &= \lim_{n \to \infty} \frac{2n}{3n+1} = \lim_{n \to \infty} \frac{2}{3 + 1/n} = \frac{2}{3}
\end{align*}
Since $L = 2/3 < 1$, the series is absolutely convergent.

\part{Advanced Diagnostic Testing: "Find the Flaw"}

For each problem below, a complete step-by-step solution is provided. However, each solution contains a single, subtle, critical flaw. Your task is to find the specific error, explain in one sentence why it is an error, and then provide the correct step and final solution.

\subsection*{Problem 1}
\textbf{Task:} Determine if $\sum_{n=1}^{\infty} (-1)^{n+1} \frac{\sqrt{n}}{n+2}$ is AC, CC, or D.

\textbf{Flawed Solution:}
\begin{enumerate}
    \item \textbf{Test Absolute Convergence:} We examine $\sum_{n=1}^{\infty} \frac{\sqrt{n}}{n+2}$. We use the Limit Comparison Test with the divergent p-series $\sum \frac{1}{\sqrt{n}}$ ($p=1/2$).
    \[ L = \lim_{n \to \infty} \frac{\sqrt{n}/(n+2)}{1/\sqrt{n}} = \lim_{n \to \infty} \frac{n}{n+2} = 1 \]
    Since $L=1$ and $\sum \frac{1}{\sqrt{n}}$ diverges, $\sum \frac{\sqrt{n}}{n+2}$ also diverges. The series is not absolutely convergent.
    \item \textbf{Test Conditional Convergence (AST):} Let $b_n = \frac{\sqrt{n}}{n+2}$.
    \[ \lim_{n \to \infty} \frac{\sqrt{n}}{n+2} = \lim_{n \to \infty} \frac{1/\sqrt{n}}{1+2/n} = 0 \]
    Since the limit is 0, the series converges by the Alternating Series Test.
\end{enumerate}
\textbf{Conclusion:} The series is conditionally convergent.

\subsection*{Problem 2}
\textbf{Task:} Determine if $\sum_{n=1}^{\infty} (-1)^n \frac{n!}{10^n}$ is AC, CC, or D.

\textbf{Flawed Solution:}
\begin{enumerate}
    \item \textbf{Test Absolute Convergence:} We use the Ratio Test on $\sum_{n=1}^{\infty} \frac{n!}{10^n}$.
    \begin{align*}
        L &= \lim_{n \to \infty} \left| \frac{a_{n+1}}{a_n} \right| = \lim_{n \to \infty} \frac{(n+1)!/10^{n+1}}{n!/10^n} \\
        &= \lim_{n \to \infty} \frac{(n+1)!}{n!} \cdot \frac{10^n}{10^{n+1}} \\
        &= \lim_{n \to \infty} (n+1) \cdot 10 = \infty
    \end{align*}
    Since $L > 1$, the series is not absolutely convergent.
    \item \textbf{Test Conditional Convergence (AST):} Let $b_n = \frac{n!}{10^n}$. We check the limit. Since the terms grow very fast, the limit $\lim_{n \to \infty} b_n = \infty \neq 0$. Thus the series diverges by the Test for Divergence.
\end{enumerate}
\textbf{Conclusion:} The series is divergent.

\subsection*{Problem 3}
\textbf{Task:} Find how many terms are needed to approximate the sum of $\sum_{n=1}^{\infty} \frac{(-1)^{n-1}}{n^2}$ with an error less than 0.005.

\textbf{Flawed Solution:}
We need to find $n$ such that the error $|R_n|$ is less than 0.005. By the Alternating Series Estimation Theorem, $|R_n| \le b_n$. So we need to solve:
\[ b_n < 0.005 \implies \frac{1}{n^2} < 0.005 \implies n^2 > \frac{1}{0.005} \implies n^2 > 200 \]
Since $14^2 = 196$ and $15^2 = 225$, we need $n > \sqrt{200} \approx 14.14$. So, we must choose $n=15$.
\textbf{Conclusion:} 15 terms are needed.

\subsection*{Problem 4}
\textbf{Task:} Determine if $\sum_{n=2}^{\infty} \frac{(-1)^n}{n \ln n}$ is AC, CC, or D.

\textbf{Flawed Solution:}
\begin{enumerate}
    \item \textbf{Test Absolute Convergence:} We examine $\sum_{n=2}^{\infty} \frac{1}{n \ln n}$. We use the Integral Test. Let $f(x) = \frac{1}{x \ln x}$. This function is positive, continuous, and decreasing for $x \ge 2$.
    \begin{align*}
        \int_2^\infty \frac{1}{x \ln x} dx &= \lim_{t \to \infty} \int_2^t \frac{1}{x \ln x} dx \quad (\text{Let } u = \ln x, du = \frac{1}{x} dx) \\
        &= \lim_{t \to \infty} \int_{\ln 2}^{\ln t} \frac{1}{u} du = \lim_{t \to \infty} [\ln|u|]_{\ln 2}^{\ln t} \\
        &= \lim_{t \to \infty} (\ln(\ln t) - \ln(\ln 2)) = \infty
    \end{align*}
    Since the integral diverges, the series $\sum \frac{1}{n \ln n}$ diverges.
\end{enumerate}
\textbf{Conclusion:} The series is divergent.

\subsection*{Problem 5}
\textbf{Task:} Determine if $\sum_{n=1}^{\infty} (-1)^n (\sqrt{n+1} - \sqrt{n})$ is AC, CC, or D.

\textbf{Flawed Solution:}
Let $b_n = \sqrt{n+1} - \sqrt{n}$. We check the limit for the Test for Divergence.
\[ \lim_{n \to \infty} (\sqrt{n+1} - \sqrt{n}) = \lim_{n \to \infty} \sqrt{n} (\sqrt{1+1/n} - 1) = \infty \cdot 0 \]
This is an indeterminate form. Let's rationalize the expression:
\[ b_n = (\sqrt{n+1} - \sqrt{n}) \cdot \frac{\sqrt{n+1} + \sqrt{n}}{\sqrt{n+1} + \sqrt{n}} = \frac{(n+1) - n}{\sqrt{n+1} + \sqrt{n}} = \frac{1}{\sqrt{n+1} + \sqrt{n}} \]
Now, let's test for absolute convergence. We examine $\sum_{n=1}^{\infty} \frac{1}{\sqrt{n+1} + \sqrt{n}}$.
We use the Direct Comparison Test. For large $n$, $\sqrt{n+1}+\sqrt{n}$ is close to $2\sqrt{n}$.
So we compare to $\sum \frac{1}{\sqrt{n}}$.
\[ \frac{1}{\sqrt{n+1} + \sqrt{n}} < \frac{1}{\sqrt{n}} \]
Since $\sum \frac{1}{\sqrt{n}}$ is a divergent p-series ($p=1/2$), this comparison is not helpful.
Let's try the Limit Comparison Test with $\sum \frac{1}{\sqrt{n}}$.
\[ L = \lim_{n \to \infty} \frac{1/(\sqrt{n+1}+\sqrt{n})}{1/\sqrt{n}} = \lim_{n \to \infty} \frac{\sqrt{n}}{\sqrt{n+1}+\sqrt{n}} = \lim_{n \to \infty} \frac{1}{\sqrt{1+1/n}+1} = \frac{1}{2} \]
Since $L=1/2$ and $\sum \frac{1}{\sqrt{n}}$ diverges, our series diverges absolutely.
\textbf{Conclusion:} The series is divergent.

\end{document}