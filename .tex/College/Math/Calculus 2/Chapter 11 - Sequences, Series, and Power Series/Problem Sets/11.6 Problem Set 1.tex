\documentclass{article}
\usepackage{amsmath}
\usepackage{amssymb}
\usepackage{geometry}
\geometry{a4paper, margin=1in}

\title{Practice Problems for Section 11.6: The Ratio and Root Tests}
\author{Generated by Gemini}
\date{\today}

\begin{document}

\maketitle

\section*{Practice Problems}
Use the Ratio Test, Root Test, or any other appropriate test to determine whether the following series are absolutely convergent, conditionally convergent, or divergent. Show all your work.

\begin{enumerate}
    % C1a
    \item $\sum_{n=1}^{\infty} \frac{n^3}{5^n}$
    % C4a
    \item $\sum_{n=1}^{\infty} \left(\frac{2n^2 + 1}{3n^2 - 4}\right)^n$
    % C2c
    \item $\sum_{n=1}^{\infty} \frac{(-10)^n}{n!}$
    % C5b
    \item $\sum_{n=1}^{\infty} \frac{1}{n\sqrt{n+1}}$
    % C2b
    \item $\sum_{n=1}^{\infty} \frac{n^{10} + n^5}{n!}$
    % C1c
    \item $\sum_{n=1}^{\infty} \frac{(-1)^n n^2}{e^n}$
    % C4b
    \item $\sum_{n=1}^{\infty} \left(1 + \frac{3}{n}\right)^{n^2}$
    % C2d
    \item $\sum_{n=1}^{\infty} \frac{(n!)^2}{(2n)!}$
    % C6a
    \item $\sum_{n=2}^{\infty} \frac{(-1)^n}{\ln n}$
    % C5a
    \item $\sum_{n=1}^{\infty} \frac{n^n}{n!}$ (Hint: Use Ratio Test and know the limit for $e$.)
    % C1b
    \item $\sum_{n=1}^{\infty} \frac{3^n}{2^{n+1}}$
    % C3a
    \item $\sum_{n=1}^{\infty} \frac{1 \cdot 3 \cdot 5 \cdots (2n-1)}{5^n n!}$
    % C4a
    \item $\sum_{n=1}^{\infty} \left(\frac{-2n}{3n+1}\right)^{5n}$
    % C2a
    \item $\sum_{k=1}^{\infty} \frac{k!}{(k+2)!}$
    % C5b
    \item $\sum_{n=1}^{\infty} \frac{n+5}{n^3+2n}$
    % C1a
    \item $\sum_{n=1}^{\infty} n^2 \left(\frac{2}{3}\right)^n$
    % C4c
    \item $\sum_{n=1}^{\infty} \left(\frac{n}{\ln(n+1)}\right)^n$
    % C2d
    \item $\sum_{n=1}^{\infty} \frac{(3n)!}{100^n (n!)^3}$
    % C7a
    \item A series $\sum a_n$ is tested with the Ratio Test, yielding $\lim_{n \to \infty} |\frac{a_{n+1}}{a_n}| = \frac{e}{2}$. What can you conclude about the series?
    % C5a
    \item $\sum_{n=1}^{\infty} \left(\frac{n}{n+1}\right)^n$
    % C1c
    \item $\sum_{n=1}^{\infty} \frac{n \cdot (-2)^n}{3^n}$
    % C2c
    \item $\sum_{n=1}^{\infty} \frac{n! e^n}{n^n}$ (This is a more challenging version of Problem 10)
    % C4b
    \item $\sum_{n=1}^{\infty} 2 \left(1-\frac{1}{n}\right)^{n^2}$
    % C6a
    \item $\sum_{n=1}^{\infty} \frac{(-1)^{n-1}}{n^2+1}$ (Determine if it's absolutely or conditionally convergent).
    % C3a
    \item $\sum_{n=1}^{\infty} \frac{2 \cdot 4 \cdot 6 \cdots (2n)}{1 \cdot 4 \cdot 7 \cdots (3n-2)}$
    % C2b
    \item $\sum_{n=1}^{\infty} \frac{1000n^{100}}{n!}$
    % C5b
    \item $\sum_{n=1}^{\infty} \frac{\arctan(n)}{n^2}$
    % C4a
    \item $\sum_{n=1}^{\infty} \left(\arctan(n)\right)^{-n}$
    % C7b
    \item Without calculating the limit, predict whether $\sum \frac{n^{10} 10^n}{n!}$ converges or diverges, and justify your prediction based on the hierarchy of growth.
    % C1a
    \item $\sum_{n=1}^{\infty} \frac{n \pi^n}{4^n}$
\end{enumerate}

\newpage
\section*{Solutions to Practice Problems}

\begin{enumerate}
    \item $\sum \frac{n^3}{5^n}$. Use Ratio Test. $L = \lim_{n \to \infty} \left|\frac{(n+1)^3/5^{n+1}}{n^3/5^n}\right| = \lim_{n \to \infty} \frac{(n+1)^3}{n^3} \frac{5^n}{5^{n+1}} = \lim_{n \to \infty} (\frac{n+1}{n})^3 \frac{1}{5} = 1^3 \cdot \frac{1}{5} = \frac{1}{5}$. Since $L < 1$, the series is \textbf{absolutely convergent}.

    \item $\sum (\frac{2n^2 + 1}{3n^2 - 4})^n$. Use Root Test. $L = \lim_{n \to \infty} \left|\frac{2n^2 + 1}{3n^2 - 4}\right| = \lim_{n \to \infty} \frac{2+1/n^2}{3-4/n^2} = \frac{2}{3}$. Since $L < 1$, the series is \textbf{absolutely convergent}.

    \item $\sum \frac{(-10)^n}{n!}$. Use Ratio Test. $L = \lim_{n \to \infty} \left|\frac{(-10)^{n+1}/(n+1)!}{(-10)^n/n!}\right| = \lim_{n \to \infty} \frac{10^{n+1}}{10^n} \frac{n!}{(n+1)!} = \lim_{n \to \infty} 10 \cdot \frac{1}{n+1} = 0$. Since $L < 1$, the series is \textbf{absolutely convergent}.

    \item $\sum \frac{1}{n\sqrt{n+1}}$. Ratio test gives $L=1$. Use Limit Comparison Test with $b_n = \frac{1}{n\sqrt{n}} = \frac{1}{n^{3/2}}$, a convergent p-series ($p=3/2 > 1$). $\lim_{n \to \infty} \frac{a_n}{b_n} = \lim_{n \to \infty} \frac{1/(n\sqrt{n+1})}{1/(n\sqrt{n})} = \lim_{n \to \infty} \frac{\sqrt{n}}{\sqrt{n+1}} = 1$. Since the limit is finite and positive, the series \textbf{converges}.

    \item $\sum \frac{n^{10} + n^5}{n!}$. Use Ratio Test. $L = \lim_{n \to \infty} \left|\frac{((n+1)^{10} + (n+1)^5)/(n+1)!}{(n^{10} + n^5)/n!}\right| = \lim_{n \to \infty} \frac{(n+1)^{10}(1+...)}{n^{10}(1+...)} \frac{1}{n+1} = 0$. Since $L < 1$, the series is \textbf{absolutely convergent}.

    \item $\sum \frac{(-1)^n n^2}{e^n}$. Use Ratio Test. $L = \lim_{n \to \infty} \left|\frac{(-1)^{n+1}(n+1)^2/e^{n+1}}{(-1)^n n^2/e^n}\right| = \lim_{n \to \infty} \frac{(n+1)^2}{n^2} \frac{1}{e} = \frac{1}{e}$. Since $L < 1$, the series is \textbf{absolutely convergent}.

    \item $\sum (1 + \frac{3}{n})^{n^2}$. Use Root Test. $L = \lim_{n \to \infty} \left( (1 + \frac{3}{n})^{n^2} \right)^{1/n} = \lim_{n \to \infty} (1 + \frac{3}{n})^n = e^3$. Since $L > 1$, the series \textbf{diverges}.

    \item $\sum \frac{(n!)^2}{(2n)!}$. Use Ratio Test. $L = \lim_{n \to \infty} \left|\frac{((n+1)!)^2/(2(n+1))!}{(n!)^2/(2n)!}\right| = \lim_{n \to \infty} \frac{((n+1)n!)^2}{(n!)^2} \frac{(2n)!}{(2n+2)!} = \lim_{n \to \infty} (n+1)^2 \frac{1}{(2n+2)(2n+1)} = \lim_{n \to \infty} \frac{n^2+2n+1}{4n^2+6n+2} = \frac{1}{4}$. Since $L < 1$, the series is \textbf{absolutely convergent}.

    \item $\sum \frac{(-1)^n}{\ln n}$. Ratio test on $|\frac{(-1)^n}{\ln n}|$ gives $L=1$. Use Alternating Series Test. $b_n = \frac{1}{\ln n}$. $b_n$ is positive and decreasing, and $\lim_{n \to \infty} b_n = 0$. The series converges. However, $\sum \frac{1}{\ln n}$ diverges by comparison with $\sum \frac{1}{n}$. Thus, the series is \textbf{conditionally convergent}.

    \item $\sum \frac{n^n}{n!}$. Use Ratio Test. $L = \lim_{n \to \infty} \left|\frac{(n+1)^{n+1}/(n+1)!}{n^n/n!}\right| = \lim_{n \to \infty} \frac{(n+1)^{n+1}}{n^n} \frac{n!}{(n+1)!} = \lim_{n \to \infty} \frac{(n+1)^n (n+1)}{n^n} \frac{1}{n+1} = \lim_{n \to \infty} (\frac{n+1}{n})^n = e$. Since $L > 1$, the series \textbf{diverges}.

    \item $\sum \frac{3^n}{2^{n+1}}$. This is a geometric series: $\frac{1}{2} \sum (\frac{3}{2})^n$. Since the ratio $|r| = 3/2 \ge 1$, the series \textbf{diverges}.

    \item $\sum \frac{1 \cdot 3 \cdots (2n-1)}{5^n n!}$. Use Ratio Test. $L = \lim_{n \to \infty} \left|\frac{1 \cdot 3 \cdots (2n-1)(2n+1)}{5^{n+1}(n+1)!} \cdot \frac{5^n n!}{1 \cdot 3 \cdots (2n-1)}\right| = \lim_{n \to \infty} \frac{2n+1}{5(n+1)} = \frac{2}{5}$. Since $L < 1$, the series is \textbf{absolutely convergent}.

    \item $\sum (\frac{-2n}{3n+1})^{5n}$. Use Root Test. $L = \lim_{n \to \infty} \left| \left(\frac{-2n}{3n+1}\right)^5 \right| = \left( \lim_{n \to \infty} \frac{2n}{3n+1} \right)^5 = (\frac{2}{3})^5 = \frac{32}{243}$. Since $L < 1$, the series is \textbf{absolutely convergent}.

    \item $\sum \frac{k!}{(k+2)!}$. Simplify $a_k = \frac{k!}{ (k+2)(k+1)k!} = \frac{1}{(k+1)(k+2)}$. This is a telescoping series or can be shown to converge by Limit Comparison to $\sum \frac{1}{k^2}$. The series \textbf{converges}.

    \item $\sum \frac{n+5}{n^3+2n}$. Ratio test gives $L=1$. Use Limit Comparison with $b_n = \frac{n}{n^3} = \frac{1}{n^2}$, a convergent p-series ($p=2>1$). $\lim_{n \to \infty} \frac{a_n}{b_n} = \lim_{n \to \infty} \frac{n+5}{n^3+2n} \cdot n^2 = \lim_{n \to \infty} \frac{n^3+5n^2}{n^3+2n} = 1$. The series \textbf{converges}.

    \item $\sum n^2 (\frac{2}{3})^n$. Use Ratio Test. $L = \lim_{n \to \infty} \left|\frac{(n+1)^2(2/3)^{n+1}}{n^2(2/3)^n}\right| = \lim_{n \to \infty} (\frac{n+1}{n})^2 \frac{2}{3} = \frac{2}{3}$. Since $L < 1$, the series is \textbf{absolutely convergent}.

    \item $\sum (\frac{n}{\ln(n+1)})^n$. Use Root Test. $L = \lim_{n \to \infty} \frac{n}{\ln(n+1)}$. Using L'Hôpital's Rule: $\lim_{n \to \infty} \frac{1}{1/(n+1)} = \lim_{n \to \infty} n+1 = \infty$. Since $L > 1$, the series \textbf{diverges}.

    \item $\sum \frac{(3n)!}{100^n (n!)^3}$. Use Ratio Test. $L = \lim_{n \to \infty} \frac{(3n+3)!}{100^{n+1}((n+1)!)^3} \cdot \frac{100^n (n!)^3}{(3n)!} = \lim_{n \to \infty} \frac{(3n+3)(3n+2)(3n+1)}{100(n+1)^3} = \frac{27}{100}$. Since $L < 1$, the series is \textbf{absolutely convergent}.

    \item Since $L = e/2 \approx 2.718/2 \approx 1.359 > 1$, the series \textbf{diverges} by the Ratio Test.

    \item $\sum (\frac{n}{n+1})^n$. Root Test gives $L=1$. Use Test for Divergence. $\lim_{n \to \infty} a_n = \lim_{n \to \infty} (\frac{n}{n+1})^n = \lim_{n \to \infty} \frac{1}{((n+1)/n)^n} = \frac{1}{(1+1/n)^n} = \frac{1}{e}$. Since the limit is not 0, the series \textbf{diverges}.

    \item $\sum \frac{n \cdot (-2)^n}{3^n}$. Use Ratio Test. $L = \lim_{n \to \infty} \left|\frac{(n+1)(-2)^{n+1}/3^{n+1}}{n(-2)^n/3^n}\right| = \lim_{n \to \infty} \frac{n+1}{n} \frac{2}{3} = \frac{2}{3}$. Since $L < 1$, the series is \textbf{absolutely convergent}.

    \item $\sum \frac{n! e^n}{n^n}$. Use Ratio Test. $L = \lim_{n \to \infty} \frac{(n+1)!e^{n+1}/(n+1)^{n+1}}{n!e^n/n^n} = \lim_{n \to \infty} \frac{(n+1)!}{n!} \frac{e^{n+1}}{e^n} \frac{n^n}{(n+1)^{n+1}} = \lim_{n \to \infty} (n+1) e \frac{n^n}{(n+1)^n(n+1)} = \lim_{n \to \infty} e (\frac{n}{n+1})^n = e \cdot \frac{1}{e} = 1$. The test is inconclusive. (Note: This is related to Stirling's approximation; the series diverges).

    \item $\sum 2 (1-\frac{1}{n})^{n^2}$. Use Root Test. $L = \lim_{n \to \infty} (2^{1/n}) (1-\frac{1}{n})^n = 1 \cdot e^{-1} = \frac{1}{e}$. Since $L < 1$, the series \textbf{converges}.

    \item $\sum \frac{(-1)^{n-1}}{n^2+1}$. Test for absolute convergence: $\sum \frac{1}{n^2+1}$ converges by limit comparison with $\sum \frac{1}{n^2}$. Since the series of absolute values converges, the original series is \textbf{absolutely convergent}.

    \item $\sum \frac{2 \cdot 4 \cdots (2n)}{1 \cdot 4 \cdots (3n-2)}$. Use Ratio Test. $L = \lim_{n \to \infty} \left|\frac{a_{n+1}}{a_n}\right| = \lim_{n \to \infty} \frac{2n+2}{3(n+1)-2} = \lim_{n \to \infty} \frac{2n+2}{3n+1} = \frac{2}{3}$. Since $L < 1$, the series is \textbf{absolutely convergent}.

    \item $\sum \frac{1000n^{100}}{n!}$. Use Ratio Test. $L = \lim_{n \to \infty} \left|\frac{1000(n+1)^{100}/(n+1)!}{1000n^{100}/n!}\right| = \lim_{n \to \infty} (\frac{n+1}{n})^{100} \frac{1}{n+1} = 1 \cdot 0 = 0$. Since $L < 1$, the series is \textbf{absolutely convergent}.

    \item $\sum \frac{\arctan(n)}{n^2}$. Ratio test is inconclusive. Use Limit Comparison with $b_n = \frac{1}{n^2}$. $\lim_{n \to \infty} \frac{a_n}{b_n} = \lim_{n \to \infty} \arctan(n) = \frac{\pi}{2}$. Since the limit is finite and positive, and $\sum b_n$ converges, the series \textbf{converges}.

    \item $\sum (\arctan(n))^{-n}$. Use Root Test. $L = \lim_{n \to \infty} \left| (\arctan(n))^{-1} \right| = \lim_{n \to \infty} \frac{1}{\arctan(n)} = \frac{1}{\pi/2} = \frac{2}{\pi}$. Since $L < 1$, the series is \textbf{absolutely convergent}.

    \item Prediction: The series \textbf{converges}. Justification: According to the hierarchy of growth, factorial functions ($n!$) grow much faster than any exponential function ($10^n$) or any polynomial function ($n^{10}$). Since the dominant term $n!$ is in the denominator, the terms of the series will go to zero very rapidly, suggesting convergence.

    \item $\sum \frac{n \pi^n}{4^n}$. This can be written as $\sum n (\frac{\pi}{4})^n$. Since $\pi/4 < 1$, we can use the Ratio Test. $L = \lim_{n \to \infty} \frac{(n+1)(\pi/4)^{n+1}}{n(\pi/4)^n} = \lim_{n \to \infty} \frac{n+1}{n} \frac{\pi}{4} = \frac{\pi}{4}$. Since $L < 1$, the series is \textbf{absolutely convergent}.
\end{enumerate}

\newpage
\section*{Concept Checklist and Problem Cross-Reference}

\begin{itemize}
    \item \textbf{C1: Ratio Test - Basic Application (L < 1 or L > 1)}
    \begin{itemize}
        \item C1a: Polynomial / Exponential: Problems 1, 16, 30
        \item C1b: Exponential / Exponential: Problem 11
        \item C1c: Alternating Polynomial / Exponential: Problems 6, 21
    \end{itemize}
    \item \textbf{C2: Ratio Test - Factorials}
    \begin{itemize}
        \item C2a: Simple Factorial: Problem 14
        \item C2b: Factorial with Polynomials: Problems 5, 26
        \item C2c: Factorial with Exponentials: Problems 3, 22
        \item C2d: Complex Factorial Expressions: Problems 8, 18
    \end{itemize}
    \item \textbf{C3: Ratio Test - Product Series}
    \begin{itemize}
        \item C3a: Products in numerator/denominator: Problems 12, 25
    \end{itemize}
    \item \textbf{C4: Root Test - Basic Application (L < 1 or L > 1)}
    \begin{itemize}
        \item C4a: Simple form $(f(n))^n$: Problems 2, 13, 28
        \item C4b: Complex form leading to the limit 'e': Problems 7, 23
        \item C4c: Form requiring L'Hôpital's Rule: Problem 17
    \end{itemize}
    \item \textbf{C5: Inconclusive Case (L = 1) and Follow-up Tests}
    \begin{itemize}
        \item C5a: Divergence by Test for Divergence: Problems 10, 20
        \item C5b: Requires Limit Comparison Test: Problems 4, 15, 27
    \end{itemize}
    \item \textbf{C6: Absolute vs. Conditional Convergence}
    \begin{itemize}
        \item C6a: Requires Alternating Series Test after Ratio Test fails for absolute convergence: Problems 9, 24
    \end{itemize}
    \item \textbf{C7: Conceptual Understanding}
    \begin{itemize}
        \item C7a: State conclusion from a given limit L: Problem 19
        \item C7b: Hierarchy of Growth prediction: Problem 29
    \end{itemize}
\end{itemize}

\end{document}