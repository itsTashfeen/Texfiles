\documentclass{article}
\usepackage{amsmath}
\usepackage{amsfonts}
\usepackage{geometry}
\geometry{a4paper, margin=1in}

\title{Homework 11.1 Sequences: Problem Set}
\author{Generated by Gemini}
\date{\today}

\begin{document}

\maketitle

\section*{Concept Checklist}
This problem set is designed to test the following concepts related to sequences:
\begin{itemize}
    \item \textbf{Direct Calculation \& Basic Properties:}
    \begin{itemize}
        \item Writing the first few terms of a sequence from an explicit formula.
        \item Writing the first few terms of a recursively defined sequence.
        \item Understanding the definitions of convergence and divergence.
    \end{itemize}
    \item \textbf{Pattern Recognition:}
    \begin{itemize}
        \item Finding an explicit formula for an arithmetic sequence.
        \item Finding an explicit formula for a geometric sequence.
        \item Finding an explicit formula for sequences involving alternating signs, factorials, and powers.
    \end{itemize}
    \item \textbf{Limit Evaluation via Algebraic Manipulation:}
    \begin{itemize}
        \item Limits of rational functions of $n$ (dividing by the highest power).
        \item Limits involving radicals (dividing by highest power, using conjugates).
        \item Limits of geometric sequences ($\lim_{n \to \infty} r^n$).
        \item Limits involving exponential functions (dividing by the fastest-growing base).
        \item Limits involving factorials (simplification).
        \item Limits involving logarithmic properties.
    \end{itemize}
    \item \textbf{Limit Evaluation using Calculus Theorems:}
    \begin{itemize}
        \item \textbf{L'H\^opital's Rule:} For indeterminate forms $\frac{\infty}{\infty}$ or $\frac{0}{0}$ involving functions of $n$ (like logarithms, exponentials, and powers).
        \item \textbf{Squeeze Theorem:} For sequences involving bounded, oscillating terms like $\sin(n)$, $\cos(n)$, or $(-1)^n$.
        \item \textbf{Continuity:} Evaluating limits by passing the limit inside a continuous function (e.g., $\lim_{n \to \infty} e^{a_n} = e^{\lim a_n}$).
        \item \textbf{Indeterminate Forms:} Evaluating limits of the form $1^\infty$, $0^0$, or $\infty^0$ using logarithms and L'H\^opital's Rule.
    \end{itemize}
    \item \textbf{Monotonic Sequence Theorem:}
    \begin{itemize}
        \item Proving a sequence is monotonic (increasing or decreasing).
        \item Proving a sequence is bounded (above and/or below).
        \item Using the Monotonic Sequence Theorem to conclude convergence and find the limit of a recursive sequence.
    \end{itemize}
\end{itemize}

\section*{Problem Set}
\subsection*{Part 1: Direct Calculation and Pattern Recognition}

\begin{enumerate}
    \item List the first five terms of the sequence $a_n = \frac{n^2 - 1}{n^2 + 1}$.

    \item List the first five terms of the sequence defined by $a_1 = 2$ and $a_{n+1} = \frac{a_n}{a_n - 1}$.

    \item Find a formula for the general term $a_n$ of the sequence, assuming the pattern continues:
    \[ \left\{ \frac{3}{4}, -\frac{4}{8}, \frac{5}{16}, -\frac{6}{32}, \dots \right\} \]

    \item Find a formula for the general term $a_n$ of the arithmetic sequence $\{11, 8, 5, 2, \dots \}$.

    \item Find a formula for the general term $a_n$ of the geometric sequence $\{ 5, -10/3, 20/9, -40/27, \dots \}$.
\end{enumerate}


\subsection*{Part 2: Determining Convergence or Divergence}
For each of the following sequences, determine whether it converges or diverges. If it converges, find the limit.

\begin{enumerate}
    \setcounter{enumi}{5}
    % Rational Functions
    \item $a_n = \frac{3n^2 - 5n + 2}{8n^2 + 4n - 1}$
    \item $a_n = \frac{n}{n^3 + 1}$
    \item $a_n = \frac{n^4 - n^2}{n^3 + n}$

    % Radicals
    \item $a_n = \frac{\sqrt{4n^2 + 1}}{3n - 2}$
    \item $a_n = \sqrt{n^2 + n} - n$

    % Exponentials and Geometric
    \item $a_n = \frac{5^n + 3^n}{5^n - 2^n}$
    \item $a_n = (-1.01)^n$
    \item $a_n = \frac{3^{n+2}}{7^n}$

    % Squeeze Theorem
    \item $a_n = \frac{\cos(n)}{n^2}$
    \item $a_n = \frac{(-1)^n n!}{n^n}$ (Hint: Write out the terms of $\frac{n!}{n^n}$ and bound it.)
    \item $a_n = \frac{5n^2 - \sin(2n)}{n^2 + n}$

    % L'Hôpital's Rule
    \item $a_n = \frac{\ln(n)}{\sqrt{n}}$
    \item $a_n = n^2 e^{-n}$
    \item $a_n = \frac{(\ln n)^3}{n}$

    % Continuity
    \item $a_n = \arctan(2n)$
    \item $a_n = \cos\left(\frac{n\pi}{n+1}\right)$

    % Factorials & Combinations
    \item $a_n = \frac{(n+1)! - n!}{(n+1)!}$
    \item $a_n = \frac{(2n-1)!}{(2n+1)!}$
    
    % Indeterminate Forms
    \item $a_n = \left(1 + \frac{3}{n}\right)^n$
    \item $a_n = n^{1/n}$
    
    % Mixed Forms
    \item $a_n = n \sin\left(\frac{1}{n}\right)$
    \item $a_n = \frac{\ln(n^2+1)}{\ln(3n+1)}$
    \item $a_n = \frac{2^n}{n!}$
    \item $a_n = \frac{n\sin(n\pi)}{2n+1}$
\end{enumerate}


\subsection*{Part 3: Monotonic Sequence Theorem}

\begin{enumerate}
    \setcounter{enumi}{29}
    \item Consider the sequence defined by $a_1 = \sqrt{5}$ and $a_{n+1} = \sqrt{5 + a_n}$.
    \begin{itemize}
        \item[a.] Show that $\{a_n\}$ is increasing.
        \item[b.] Show that $\{a_n\}$ is bounded above by 3.
        \item[c.] Explain why the sequence converges and find its limit.
    \end{itemize}

    \item Consider the sequence defined by $a_1 = 3$ and $a_{n+1} = \frac{1}{4}(a_n + 6)$.
    \begin{itemize}
        \item[a.] Show that $\{a_n\}$ is decreasing and bounded below.
        \item[b.] Find the limit of the sequence.
    \end{itemize}
\end{enumerate}


\newpage
\section*{Solutions}

\begin{enumerate}
    \item \textbf{Solution:} $a_1 = \frac{1-1}{1+1} = 0$, $a_2 = \frac{4-1}{4+1} = \frac{3}{5}$, $a_3 = \frac{9-1}{9+1} = \frac{8}{10} = \frac{4}{5}$, $a_4 = \frac{16-1}{16+1} = \frac{15}{17}$, $a_5 = \frac{25-1}{25+1} = \frac{24}{26} = \frac{12}{13}$.
    The terms are $\left\{ 0, \frac{3}{5}, \frac{4}{5}, \frac{15}{17}, \frac{12}{13}, \dots \right\}$.

    \item \textbf{Solution:} $a_1 = 2$, $a_2 = \frac{2}{2-1} = 2$, $a_3 = \frac{2}{2-1} = 2$, $a_4 = 2$, $a_5 = 2$.
    The sequence is a constant sequence $\{2, 2, 2, 2, 2, \dots\}$.

    \item \textbf{Solution:} The signs are alternating, starting negative if we consider $n=1$ for the first term $-4/8$. Let's re-index to start at $n=1$ for $3/4$. The sign is $(-1)^{n+1}$. The numerator starts at 3 and increases by 1, so it is $n+2$. The denominator is a power of 2, starting with $4 = 2^2$, then $8=2^3$, etc. The denominator is $2^{n+1}$.
    Thus, $a_n = (-1)^{n+1} \frac{n+2}{2^{n+1}}$.

    \item \textbf{Solution:} This is an arithmetic sequence with first term $a_1 = 11$ and common difference $d = 8 - 11 = -3$.
    The formula is $a_n = a_1 + (n-1)d = 11 + (n-1)(-3) = 11 - 3n + 3 = 14 - 3n$.

    \item \textbf{Solution:} This is a geometric sequence with first term $a_1 = 5$. The common ratio is $r = \frac{-10/3}{5} = -\frac{10}{15} = -\frac{2}{3}$.
    The formula is $a_n = a_1 r^{n-1} = 5 \left(-\frac{2}{3}\right)^{n-1}$.

    \item \textbf{Solution:} Divide by the highest power of $n$ in the denominator, which is $n^2$.
    \[ \lim_{n \to \infty} \frac{3n^2/n^2 - 5n/n^2 + 2/n^2}{8n^2/n^2 + 4n/n^2 - 1/n^2} = \lim_{n \to \infty} \frac{3 - 5/n + 2/n^2}{8 + 4/n - 1/n^2} = \frac{3-0+0}{8+0-0} = \frac{3}{8}. \]
    Converges to $\frac{3}{8}$.

    \item \textbf{Solution:} Divide by $n^3$.
    \[ \lim_{n \to \infty} \frac{n/n^3}{n^3/n^3 + 1/n^3} = \lim_{n \to \infty} \frac{1/n^2}{1 + 1/n^3} = \frac{0}{1+0} = 0. \]
    Converges to $0$.

    \item \textbf{Solution:} The degree of the numerator (4) is greater than the degree of the denominator (3).
    \[ \lim_{n \to \infty} \frac{n^4 - n^2}{n^3 + n} = \lim_{n \to \infty} \frac{n(1 - 1/n^2)}{1 + 1/n^2} = \infty. \]
    The sequence diverges.

    \item \textbf{Solution:} Divide by $n$ (which is $\sqrt{n^2}$).
    \[ \lim_{n \to \infty} \frac{\sqrt{4n^2/n^2 + 1/n^2}}{3n/n - 2/n} = \lim_{n \to \infty} \frac{\sqrt{4 + 1/n^2}}{3 - 2/n} = \frac{\sqrt{4+0}}{3-0} = \frac{2}{3}. \]
    Converges to $\frac{2}{3}$.

    \item \textbf{Solution:} This is an indeterminate form $\infty - \infty$. Multiply by the conjugate.
    \[ \lim_{n \to \infty} \frac{(\sqrt{n^2 + n} - n)(\sqrt{n^2 + n} + n)}{\sqrt{n^2 + n} + n} = \lim_{n \to \infty} \frac{n^2 + n - n^2}{\sqrt{n^2 + n} + n} = \lim_{n \to \infty} \frac{n}{\sqrt{n^2(1 + 1/n)} + n} \]
    \[ = \lim_{n \to \infty} \frac{n}{n\sqrt{1 + 1/n} + n} = \lim_{n \to \infty} \frac{1}{\sqrt{1 + 1/n} + 1} = \frac{1}{\sqrt{1+0}+1} = \frac{1}{2}. \]
    Converges to $\frac{1}{2}$.

    \item \textbf{Solution:} Divide by the fastest-growing term, $5^n$.
    \[ \lim_{n \to \infty} \frac{5^n/5^n + 3^n/5^n}{5^n/5^n - 2^n/5^n} = \lim_{n \to \infty} \frac{1 + (3/5)^n}{1 - (2/5)^n} = \frac{1+0}{1-0} = 1. \]
    Converges to 1.

    \item \textbf{Solution:} This is a geometric sequence with ratio $r = -1.01$. Since $|r| > 1$, the sequence diverges.

    \item \textbf{Solution:} Rewrite as $a_n = 9 \cdot \left(\frac{3}{7}\right)^n$. This is a geometric sequence with $|r| = 3/7 < 1$.
    \[ \lim_{n \to \infty} 9 \left(\frac{3}{7}\right)^n = 9 \cdot 0 = 0. \]
    Converges to 0.

    \item \textbf{Solution:} Use the Squeeze Theorem. We know $-1 \le \cos(n) \le 1$.
    \[ -\frac{1}{n^2} \le \frac{\cos(n)}{n^2} \le \frac{1}{n^2} \]
    Since $\lim_{n \to \infty} -\frac{1}{n^2} = 0$ and $\lim_{n \to \infty} \frac{1}{n^2} = 0$, by the Squeeze Theorem, $\lim_{n \to \infty} \frac{\cos(n)}{n^2} = 0$. Converges to 0.

    \item \textbf{Solution:} Use the Squeeze Theorem.
    $a_n = (-1)^n \frac{n!}{n^n} = (-1)^n \left(\frac{1}{n} \cdot \frac{2}{n} \cdot \frac{3}{n} \cdots \frac{n}{n}\right)$.
    We know that $0 \le \frac{n!}{n^n} \le \frac{1}{n}$.
    So, $-\frac{1}{n} \le (-1)^n \frac{n!}{n^n} \le \frac{1}{n}$.
    Since $\lim_{n \to \infty} \pm \frac{1}{n} = 0$, the limit of $a_n$ is 0. Converges to 0.

    \item \textbf{Solution:} Divide by $n^2$.
    $a_n = \frac{5 - \sin(2n)/n^2}{1 + 1/n}$. For the term $\frac{\sin(2n)}{n^2}$, we can use the Squeeze Theorem.
    $-\frac{1}{n^2} \le \frac{\sin(2n)}{n^2} \le \frac{1}{n^2}$, so its limit is 0.
    \[ \lim_{n \to \infty} a_n = \frac{5 - 0}{1 + 0} = 5. \]
    Converges to 5.

    \item \textbf{Solution:} Use L'H\^opital's Rule for the indeterminate form $\frac{\infty}{\infty}$.
    \[ \lim_{x \to \infty} \frac{\ln(x)}{x^{1/2}} \overset{L'H}{=} \lim_{x \to \infty} \frac{1/x}{(1/2)x^{-1/2}} = \lim_{x \to \infty} \frac{2}{\sqrt{x}} = 0. \]
    Converges to 0.

    \item \textbf{Solution:} Use L'H\^opital's Rule for $\frac{\infty}{\infty}$. Let $f(x) = x^2/e^x$.
    \[ \lim_{x \to \infty} \frac{x^2}{e^x} \overset{L'H}{=} \lim_{x \to \infty} \frac{2x}{e^x} \overset{L'H}{=} \lim_{x \to \infty} \frac{2}{e^x} = 0. \]
    Converges to 0.
    
    \item \textbf{Solution:} Use L'H\^opital's Rule repeatedly. Let $f(x) = (\ln x)^3 / x$.
    \[ \lim_{x \to \infty} \frac{(\ln x)^3}{x} \overset{L'H}{=} \lim_{x \to \infty} \frac{3(\ln x)^2 \cdot (1/x)}{1} = \lim_{x \to \infty} \frac{3(\ln x)^2}{x} \overset{L'H}{=} \lim_{x \to \infty} \frac{6(\ln x) \cdot (1/x)}{1} = \lim_{x \to \infty} \frac{6 \ln x}{x} \overset{L'H}{=} \lim_{x \to \infty} \frac{6/x}{1} = 0. \]
    Converges to 0.

    \item \textbf{Solution:} As $n \to \infty$, $2n \to \infty$. The range of $\arctan(x)$ is $(-\pi/2, \pi/2)$.
    \[ \lim_{n \to \infty} \arctan(2n) = \frac{\pi}{2}. \]
    Converges to $\pi/2$.

    \item \textbf{Solution:} The function $\cos(x)$ is continuous.
    \[ \lim_{n \to \infty} \cos\left(\frac{n\pi}{n+1}\right) = \cos\left(\lim_{n \to \infty} \frac{n\pi}{n+1}\right) = \cos\left(\lim_{n \to \infty} \frac{\pi}{1+1/n}\right) = \cos(\pi) = -1. \]
    Converges to -1.

    \item \textbf{Solution:} Simplify the expression.
    \[ a_n = \frac{n!(n+1) - n!}{n!(n+1)} = \frac{n!(n+1-1)}{n!(n+1)} = \frac{n}{n+1}. \]
    \[ \lim_{n \to \infty} \frac{n}{n+1} = 1. \]
    Converges to 1.

    \item \textbf{Solution:} Simplify the expression using $(n+1)! = (n+1)n!$.
    \[ a_n = \frac{(2n-1)!}{(2n+1)(2n)(2n-1)!} = \frac{1}{(2n+1)(2n)}. \]
    \[ \lim_{n \to \infty} \frac{1}{4n^2+2n} = 0. \]
    Converges to 0.

    \item \textbf{Solution:} This is the indeterminate form $1^\infty$. Let $L = \lim_{n \to \infty} \left(1 + \frac{3}{n}\right)^n$.
    Take the natural log: $\ln(L) = \lim_{n \to \infty} n \ln\left(1 + \frac{3}{n}\right) = \lim_{n \to \infty} \frac{\ln(1+3/n)}{1/n}$.
    This is $\frac{0}{0}$, so use L'H\^opital's Rule.
    \[ \ln(L) = \lim_{n \to \infty} \frac{\frac{1}{1+3/n} \cdot (-3/n^2)}{-1/n^2} = \lim_{n \to \infty} \frac{3}{1+3/n} = 3. \]
    So, $\ln(L) = 3$, which means $L = e^3$. Converges to $e^3$.

    \item \textbf{Solution:} This is the indeterminate form $\infty^0$. Let $L = \lim_{n \to \infty} n^{1/n}$.
    Take the natural log: $\ln(L) = \lim_{n \to \infty} \frac{1}{n} \ln(n)$.
    This is $\frac{\infty}{\infty}$, so use L'H\^opital's Rule.
    \[ \ln(L) = \lim_{n \to \infty} \frac{1/n}{1} = 0. \]
    So, $\ln(L) = 0$, which means $L = e^0 = 1$. Converges to 1.

    \item \textbf{Solution:} Indeterminate form $\infty \cdot 0$. Rewrite and use L'H\^opital's Rule. Let $x = 1/n$. As $n \to \infty$, $x \to 0^+$.
    \[ \lim_{n \to \infty} \frac{\sin(1/n)}{1/n} = \lim_{x \to 0^+} \frac{\sin(x)}{x} \overset{L'H}{=} \lim_{x \to 0^+} \frac{\cos(x)}{1} = 1. \]
    Converges to 1.

    \item \textbf{Solution:} Use L'H\^opital's Rule for $\frac{\infty}{\infty}$.
    \[ \lim_{x \to \infty} \frac{\ln(x^2+1)}{\ln(3x+1)} \overset{L'H}{=} \lim_{x \to \infty} \frac{\frac{2x}{x^2+1}}{\frac{3}{3x+1}} = \lim_{x \to \infty} \frac{2x(3x+1)}{3(x^2+1)} = \lim_{x \to \infty} \frac{6x^2+2x}{3x^2+3}. \]
    This is still $\frac{\infty}{\infty}$. The degrees are equal, so the limit is the ratio of leading coefficients, $6/3 = 2$. Converges to 2.

    \item \textbf{Solution:} For $n > 2$, $n!$ grows much faster than $2^n$. Let's look at the terms:
    $a_1=2, a_2=2, a_3=8/6, a_4=16/24, \dots$
    We can see $0 < a_n = \frac{2 \cdot 2 \cdots 2}{1 \cdot 2 \cdots n} = \frac{2}{1} \cdot \frac{2}{2} \cdot \frac{2}{3} \cdots \frac{2}{n} \le 2 \cdot 1 \cdot \left(\frac{2}{3}\right)^{n-2}$. As $n \to \infty$, this goes to 0.
    The limit is 0. Converges to 0.

    \item \textbf{Solution:} $\sin(n\pi) = 0$ for any integer $n$. So, $a_n = \frac{n \cdot 0}{2n+1} = 0$ for all $n$.
    The sequence is $\{0, 0, 0, \dots\}$. Converges to 0.
    
    \item \textbf{Solution:}
    \begin{itemize}
        \item[a.] \textbf{Increasing:} Use induction. Base Case: $a_1 = \sqrt{5} \approx 2.23$, $a_2 = \sqrt{5+\sqrt{5}} \approx 2.69$. So $a_2 > a_1$.
        Inductive Step: Assume $a_{k+1} > a_k$ for some $k \ge 1$. Then $5+a_{k+1} > 5+a_k$. Since square root is an increasing function, $\sqrt{5+a_{k+1}} > \sqrt{5+a_k}$, which means $a_{k+2} > a_{k+1}$. By induction, the sequence is increasing.
        \item[b.] \textbf{Bounded:} Use induction. Base Case: $a_1 = \sqrt{5} < 3$.
        Inductive Step: Assume $a_k < 3$. Then $a_{k+1} = \sqrt{5+a_k} < \sqrt{5+3} = \sqrt{8} < 3$. By induction, $a_n < 3$ for all $n$.
        \item[c.] \textbf{Limit:} Since the sequence is increasing and bounded above, it must converge by the Monotonic Sequence Theorem. Let $L = \lim_{n \to \infty} a_n$.
        Then $L = \lim_{n \to \infty} a_{n+1} = \lim_{n \to \infty} \sqrt{5+a_n} = \sqrt{5+L}$.
        $L^2 = 5+L \implies L^2 - L - 5 = 0$. Using the quadratic formula, $L = \frac{1 \pm \sqrt{1 - 4(1)(-5)}}{2} = \frac{1 \pm \sqrt{21}}{2}$.
        Since the terms are all positive, the limit must be positive. So, $L = \frac{1 + \sqrt{21}}{2}$.
    \end{itemize}

    \item \textbf{Solution:}
    \begin{itemize}
        \item[a.] \textbf{Monotonic and Bounded:} $a_1=3$, $a_2 = \frac{1}{4}(3+6) = \frac{9}{4}=2.25$, $a_3 = \frac{1}{4}(2.25+6) = \frac{8.25}{4} = 2.0625$. The sequence appears to be decreasing. It is bounded below by 2.
        Proof (Decreasing by induction): Base case $a_2 < a_1$ is true. Assume $a_k < a_{k-1}$. Then $a_k+6 < a_{k-1}+6$, so $\frac{1}{4}(a_k+6) < \frac{1}{4}(a_{k-1}+6)$, which is $a_{k+1} < a_k$.
        Proof (Bounded below by 2): Base case $a_1 > 2$. Assume $a_k > 2$. Then $a_{k+1} = \frac{1}{4}(a_k+6) > \frac{1}{4}(2+6) = 2$.
        \item[b.] \textbf{Limit:} Since the sequence is decreasing and bounded below, it converges. Let the limit be $L$.
        $L = \frac{1}{4}(L+6) \implies 4L = L+6 \implies 3L = 6 \implies L=2$.
    \end{itemize}

\end{enumerate}


\newpage
\section*{Problem Cross-Reference by Concept}
\begin{itemize}
    \item \textbf{Direct Calculation \& Basic Properties:}
    \begin{itemize}
        \item Writing first terms (explicit): 1
        \item Writing first terms (recursive): 2
        \item Definitions are implicitly tested in all limit problems.
    \end{itemize}
    \item \textbf{Pattern Recognition:}
    \begin{itemize}
        \item Arithmetic: 4
        \item Geometric: 5
        \item Mixed (alternating, powers, etc.): 3
    \end{itemize}
    \item \textbf{Limit Evaluation via Algebraic Manipulation:}
    \begin{itemize}
        \item Rational functions of $n$: 6, 7, 8
        \item Radicals: 9, 10
        \item Geometric sequences / Exponentials: 11, 12, 13
        \item Factorials: 22, 23
        \item Logarithmic properties: 27 (can also be solved with L'H\^opital's)
    \end{itemize}
    \item \textbf{Limit Evaluation using Calculus Theorems:}
    \begin{itemize}
        \item L'H\^opital's Rule: 17, 18, 19, 26, 27
        \item Squeeze Theorem: 14, 15, 16, 28 (denominator dominates)
        \item Continuity: 20, 21
        \item Indeterminate Forms ($1^\infty, \infty^0$): 24, 25
    \end{itemize}
    \item \textbf{Mixed Forms and Special Cases:}
    \begin{itemize}
        \item Rewriting for L'H\^opital's Rule: 26
        \item Recognizing zero terms: 29
    \end{itemize}
    \item \textbf{Monotonic Sequence Theorem:}
    \begin{itemize}
        \item Proving properties and finding the limit: 30, 31
    \end{itemize}
\end{itemize}


\end{document}