\documentclass[11pt]{article}
\usepackage{amsmath}
\usepackage{amssymb}
\usepackage[margin=1in]{geometry}

\title{Problem Set: 11.4 Comparison Tests for Series}
\author{Generated by Gemini for Tashfeen Omran}
\date{\today}

\begin{document}

\maketitle

\section*{Instructions}
For each of the following series, determine whether it converges or diverges. State which test you are using (Direct Comparison Test or Limit Comparison Test), show the comparison series you chose, and provide a clear, step-by-step justification for your conclusion.

\section*{Problems}

\begin{enumerate}
    % --- Conceptual Questions ---
    \item \textbf{(Conceptual)} Suppose $\sum a_n$ is a series with positive terms. If you use the Direct Comparison Test with a known divergent series $\sum b_n$ and find that $a_n \ge b_n$, what can you conclude? If you find $a_n \le b_n$, what can you conclude? Explain your reasoning for both cases.

    \item \textbf{(Conceptual)} Suppose $\sum a_n$ is a series with positive terms. You apply the Limit Comparison Test with a known convergent series $\sum b_n$ and find that $\lim_{n \to \infty} \frac{a_n}{b_n} = 0$. What can you conclude about $\sum a_n$? What if the limit was $\infty$?

    % --- Error Analysis ---
    \item \textbf{(Error Analysis)} A student wants to determine if $\sum_{n=1}^{\infty} \frac{1}{n^2 - n + 1}$ converges. They reason as follows: "For the Direct Comparison Test, I know that $n^2 - n + 1 < n^2$. Taking the reciprocal reverses the inequality, so $\frac{1}{n^2 - n + 1} > \frac{1}{n^2}$. Since $\sum \frac{1}{n^2}$ is a convergent p-series, and my series is greater than a convergent series, my series must diverge." Identify the flaw in the student's reasoning and provide the correct analysis.

    \item \textbf{(Error Analysis)} A student tests the series $\sum_{n=1}^{\infty} \frac{3^n - 2}{5^n}$. They use the Limit Comparison Test with $b_n = (\frac{3}{5})^n$. They calculate $\lim_{n \to \infty} \frac{(3^n-2)/5^n}{(3/5)^n} = \lim_{n \to \infty} \frac{3^n-2}{3^n} = \lim_{n \to \infty} (1 - \frac{2}{3^n}) = 1$. They conclude that since the limit is 1 and $\sum b_n$ is a divergent geometric series, the original series must also diverge. Find the error and correct it.

    % --- p-Series Comparisons ---
    \item Determine if the series $\sum_{n=1}^{\infty} \frac{n-1}{n^3+1}$ converges or diverges.

    \item Determine if the series $\sum_{n=1}^{\infty} \frac{5n^2+3n}{2n^4+n^2+1}$ converges or diverges.

    \item Determine if the series $\sum_{n=1}^{\infty} \frac{1}{\sqrt{n^2+4}}$ converges or diverges.

    \item Determine if the series $\sum_{n=1}^{\infty} \frac{\sqrt{n+1}}{n^2+n}$ converges or diverges.

    \item Determine if the series $\sum_{n=2}^{\infty} \frac{n}{\sqrt[3]{n^5-10}}$ converges or diverges.

    \item Determine if the series $\sum_{n=1}^{\infty} \frac{(2n+1)^2}{(n^2+1)^3}$ converges or diverges.

    \item Determine if the series $\sum_{n=1}^{\infty} \frac{n^2}{n^3 + \sqrt{n}}$ converges or diverges.

    \item Determine if the series $\sum_{n=1}^{\infty} \left( \frac{1}{n} + \frac{1}{n^2} \right)$ converges or diverges.

    \item Determine if the series $\sum_{n=1}^{\infty} \frac{5 + \cos(n)}{\sqrt{n}}$ converges or diverges.

    \item Determine if the series $\sum_{n=1}^{\infty} \frac{\arctan(n)}{n^2}$ converges or diverges.
    
    \item Determine if the series $\sum_{n=1}^{\infty} \frac{n}{(n+1)(n+2)(n+3)}$ converges or diverges.
    
    \item Determine if the series $\sum_{n=1}^{\infty} \frac{\sqrt[3]{n^2+1}}{\sqrt{n^3+n}}$ converges or diverges.

    % --- Geometric Series Comparisons ---
    \item Determine if the series $\sum_{n=1}^{\infty} \frac{4^n}{5^n-20}$ converges or diverges.

    \item Determine if the series $\sum_{n=1}^{\infty} \frac{1}{2^n+n}$ converges or diverges.

    \item Determine if the series $\sum_{n=1}^{\infty} \frac{n^3}{3^n}$ converges or diverges. (Hint: Use LCT with a convergent p-series to show the limit is 0).

    \item Determine if the series $\sum_{n=1}^{\infty} \frac{50}{n!}$ converges or diverges.

    \item Determine if the series $\sum_{n=1}^{\infty} \frac{2^n + 3^n}{4^n}$ converges or diverges.

    \item Determine if the series $\sum_{n=1}^{\infty} \frac{(n!)^2}{(2n)!}$ converges or diverges.
    
    \item Determine if the series $\sum_{n=1}^{\infty} n e^{-n^2}$ converges or diverges.

    % --- Logarithmic / Harmonic Comparisons ---
    \item Determine if the series $\sum_{n=2}^{\infty} \frac{1}{n \ln(n)}$ converges or diverges. (Hint: The Integral Test is best, but try comparing to $\frac{1}{n}$.)

    \item Determine if the series $\sum_{n=2}^{\infty} \frac{\ln(n)}{n^2}$ converges or diverges.
    
    \item Determine if the series $\sum_{n=1}^{\infty} \frac{1}{n^{1+1/n}}$ converges or diverges.
    
    \item Determine if the series $\sum_{n=2}^{\infty} \frac{1}{(\ln n)^2}$ converges or diverges.
    
    \item Determine if the series $\sum_{n=2}^{\infty} \frac{\ln(n)}{\sqrt{n^3}}$ converges or diverges.

    % --- Advanced / Mixed Techniques ---
    \item Determine if the series $\sum_{n=1}^{\infty} \sin\left(\frac{1}{n^2}\right)$ converges or diverges. (Hint: For small $x$, $\sin(x) \approx x$.)

    \item Determine if the series $\sum_{n=1}^{\infty} \frac{1}{n} \left(1 - \cos\left(\frac{1}{n}\right)\right)$ converges or diverges. (Hint: For small $x$, $1-\cos(x) \approx \frac{x^2}{2}$.)
    
    \item Determine if the series $\sum_{n=1}^{\infty} (\sqrt{n^2+1}-n)$ converges or diverges. (Hint: Use the conjugate.)
    
    \item Determine if the series $\sum_{n=1}^{\infty} \frac{2^n}{3^n+n^3}$ converges or diverges.

\end{enumerate}

\newpage
\section*{Solutions}
\begin{enumerate}
    \item \textbf{Solution:} 
    \begin{itemize}
        \item If $a_n \ge b_n$ and $\sum b_n$ diverges, the Direct Comparison Test states that the "larger" series $\sum a_n$ must also \textbf{diverge}. Its terms are accumulating at least as fast as a series that goes to infinity.
        \item If $a_n \le b_n$ and $\sum b_n$ diverges, the test is \textbf{inconclusive}. Being smaller than a divergent series doesn't tell us anything. For example, $\sum \frac{1}{n^2} \le \sum \frac{1}{n}$, but the first converges while the second diverges.
    \end{itemize}

    \item \textbf{Solution:} 
    \begin{itemize}
        \item If $\lim_{n \to \infty} \frac{a_n}{b_n} = 0$ and $\sum b_n$ converges, this is a special case of the LCT. It means $a_n$ is significantly smaller than $b_n$ for large $n$. Therefore, we can conclude that $\sum a_n$ also \textbf{converges}.
        \item If the limit was $\infty$, it would mean $a_n$ is significantly larger than $b_n$. Since $\sum b_n$ converges, being larger than a convergent series is \textbf{inconclusive}.
    \end{itemize}

    \item \textbf{Solution:} The flaw is in the student's conclusion. The logic "my series is greater than a convergent series" is an \textbf{inconclusive} case for the Direct Comparison Test.
    \textbf{Correct Analysis:} Use the Limit Comparison Test with $b_n = \frac{1}{n^2}$.
    \[ L = \lim_{n \to \infty} \frac{1/(n^2-n+1)}{1/n^2} = \lim_{n \to \infty} \frac{n^2}{n^2-n+1} = 1 \]
    Since $0 < L < \infty$ and the comparison series $\sum \frac{1}{n^2}$ converges (p-series, $p=2>1$), the original series also \textbf{converges}.

    \item \textbf{Solution:} The student correctly calculated the limit. The error is the statement that $\sum (\frac{3}{5})^n$ is a \textbf{divergent} geometric series. In fact, for a geometric series to converge, the absolute value of the ratio must be less than 1. Here, $|r| = 3/5 < 1$, so the comparison series \textbf{converges}. Since the limit is 1 and the comparison series converges, the original series must also \textbf{converge}.

    \item \textbf{Solution (LCT):} Compare with $b_n = \frac{n}{n^3} = \frac{1}{n^2}$.
    \[ L = \lim_{n \to \infty} \frac{(n-1)/(n^3+1)}{1/n^2} = \lim_{n \to \infty} \frac{n^2(n-1)}{n^3+1} = \lim_{n \to \infty} \frac{n^3-n^2}{n^3+1} = 1 \]
    Since $\sum \frac{1}{n^2}$ converges (p-series, $p=2>1$) and $0<L<\infty$, the series \textbf{converges}.

    \item \textbf{Solution (LCT):} Compare with $b_n = \frac{n^2}{n^4} = \frac{1}{n^2}$.
    \[ L = \lim_{n \to \infty} \frac{(5n^2+3n)/(2n^4+n^2+1)}{1/n^2} = \lim_{n \to \infty} \frac{n^2(5n^2+3n)}{2n^4+n^2+1} = \lim_{n \to \infty} \frac{5n^4+3n^3}{2n^4+n^2+1} = \frac{5}{2} \]
    Since $\sum \frac{1}{n^2}$ converges and $0<L<\infty$, the series \textbf{converges}.

    \item \textbf{Solution (LCT):} Compare with $b_n = \frac{1}{\sqrt{n^2}} = \frac{1}{n}$.
    \[ L = \lim_{n \to \infty} \frac{1/\sqrt{n^2+4}}{1/n} = \lim_{n \to \infty} \frac{n}{\sqrt{n^2+4}} = \lim_{n \to \infty} \frac{n}{n\sqrt{1+4/n^2}} = 1 \]
    Since $\sum \frac{1}{n}$ diverges (harmonic series, $p=1$) and $0<L<\infty$, the series \textbf{diverges}.

    \item \textbf{Solution (LCT):} Compare with $b_n = \frac{\sqrt{n}}{n^2} = \frac{1}{n^{3/2}}$.
    \[ L = \lim_{n \to \infty} \frac{\sqrt{n+1}/(n^2+n)}{1/n^{3/2}} = \lim_{n \to \infty} \frac{n^{3/2}\sqrt{n+1}}{n^2+n} = \lim_{n \to \infty} \frac{n^2\sqrt{1+1/n}}{n^2(1+1/n)} = 1 \]
    Since $\sum \frac{1}{n^{3/2}}$ converges (p-series, $p=3/2>1$) and $0<L<\infty$, the series \textbf{converges}.

    \item \textbf{Solution (LCT):} Compare with $b_n = \frac{n}{\sqrt[3]{n^5}} = \frac{n}{n^{5/3}} = \frac{1}{n^{2/3}}$.
    \[ L = \lim_{n \to \infty} \frac{n/\sqrt[3]{n^5-10}}{1/n^{2/3}} = \lim_{n \to \infty} \frac{n^{5/3}}{\sqrt[3]{n^5-10}} = 1 \]
    Since $\sum \frac{1}{n^{2/3}}$ diverges (p-series, $p=2/3 \le 1$) and $0<L<\infty$, the series \textbf{diverges}.

    \item \textbf{Solution (LCT):} The dominant term in the numerator is $(2n)^2=4n^2$. The dominant term in the denominator is $(n^2)^3=n^6$. Compare with $b_n = \frac{n^2}{n^6} = \frac{1}{n^4}$.
    \[ L = \lim_{n \to \infty} \frac{(2n+1)^2/(n^2+1)^3}{1/n^4} = \lim_{n \to \infty} \frac{n^4(4n^2+4n+1)}{n^6+3n^4+3n^2+1} = \lim_{n \to \infty} \frac{4n^6+\dots}{n^6+\dots} = 4 \]
    Since $\sum \frac{1}{n^4}$ converges (p-series, $p=4>1$) and $0<L<\infty$, the series \textbf{converges}.

    \item \textbf{Solution (LCT):} Compare with $b_n = \frac{n^2}{n^3} = \frac{1}{n}$.
    \[ L = \lim_{n \to \infty} \frac{n^2/(n^3+\sqrt{n})}{1/n} = \lim_{n \to \infty} \frac{n^3}{n^3+\sqrt{n}} = 1 \]
    Since $\sum \frac{1}{n}$ diverges and $0<L<\infty$, the series \textbf{diverges}.

    \item \textbf{Solution (DCT):} Note that $a_n = \frac{1}{n} + \frac{1}{n^2}$. For all $n \ge 1$, $\frac{1}{n^2} > 0$, so $\frac{1}{n} + \frac{1}{n^2} > \frac{1}{n}$. Let $b_n = \frac{1}{n}$. Since $\sum \frac{1}{n}$ diverges (harmonic series) and our series is term-by-term larger, our series \textbf{diverges}.

    \item \textbf{Solution (DCT):} We know $-1 \le \cos(n) \le 1$. Thus, $4 \le 5+\cos(n) \le 6$.
    So, $\frac{4}{\sqrt{n}} \le \frac{5+\cos(n)}{\sqrt{n}}$. Let $b_n = \frac{4}{\sqrt{n}}$. The series $\sum b_n = 4\sum\frac{1}{n^{1/2}}$ diverges (p-series, $p=1/2 \le 1$). Since our series is term-by-term larger than a divergent series, it \textbf{diverges}.
    
    \item \textbf{Solution (DCT):} For $n \ge 1$, we know $0 < \arctan(n) < \pi/2$.
    Therefore, $0 < \frac{\arctan(n)}{n^2} < \frac{\pi/2}{n^2}$. Let $b_n = \frac{\pi/2}{n^2}$. The series $\sum b_n = \frac{\pi}{2}\sum\frac{1}{n^2}$ converges (p-series, $p=2>1$). Since our series is term-by-term smaller than a convergent series, it \textbf{converges}.

    \item \textbf{Solution (LCT):} For large $n$, the denominator behaves like $n^3$. Compare with $b_n = \frac{n}{n^3} = \frac{1}{n^2}$.
    \[ L = \lim_{n \to \infty} \frac{n/((n+1)(n+2)(n+3))}{1/n^2} = \lim_{n \to \infty} \frac{n^3}{n^3+6n^2+11n+6} = 1 \]
    Since $\sum \frac{1}{n^2}$ converges (p-series, $p=2>1$) and $0<L<\infty$, the series \textbf{converges}.

    \item \textbf{Solution (LCT):} Dominant term of numerator is $\sqrt[3]{n^2}=n^{2/3}$. Dominant term of denominator is $\sqrt{n^3}=n^{3/2}$. Compare with $b_n = \frac{n^{2/3}}{n^{3/2}} = \frac{1}{n^{3/2-2/3}} = \frac{1}{n^{5/6}}$.
    \[ L = \lim_{n \to \infty} \frac{\sqrt[3]{n^2+1}/\sqrt{n^3+n}}{1/n^{5/6}} = 1 \]
    Since $\sum \frac{1}{n^{5/6}}$ diverges (p-series, $p=5/6 \le 1$) and $0<L<\infty$, the series \textbf{diverges}.

    \item \textbf{Solution (LCT):} Compare with $b_n = \frac{4^n}{5^n} = (\frac{4}{5})^n$.
    \[ L = \lim_{n \to \infty} \frac{4^n/(5^n-20)}{(4/5)^n} = \lim_{n \to \infty} \frac{5^n}{5^n-20} = 1 \]
    Since $\sum(\frac{4}{5})^n$ is a convergent geometric series ($|r|=4/5<1$) and $0<L<\infty$, the series \textbf{converges}.

    \item \textbf{Solution (DCT):} For $n \ge 1$, $2^n+n > 2^n$. Thus, $\frac{1}{2^n+n} < \frac{1}{2^n}$. Let $b_n = (\frac{1}{2})^n$. The series $\sum b_n$ is a convergent geometric series ($|r|=1/2<1$). Since our series is term-by-term smaller, it \textbf{converges}.

    \item \textbf{Solution (LCT):} We know exponentials grow faster than polynomials, so we expect convergence. Let's compare with $b_n = \frac{1}{n^2}$ to use the special LCT case.
    \[ L = \lim_{n \to \infty} \frac{n^3/3^n}{1/n^2} = \lim_{n \to \infty} \frac{n^5}{3^n} = 0 \quad (\text{by L'Hôpital's Rule repeatedly, or knowing exponentials dominate}) \]
    Since $L=0$ and the comparison series $\sum \frac{1}{n^2}$ converges, the original series \textbf{converges}.

    \item \textbf{Solution (DCT):} For $n \ge 4$, we know $n! > 2^n$. Thus, $\frac{1}{n!} < \frac{1}{2^n}$.
    So we compare $\sum \frac{50}{n!}$ to $\sum \frac{50}{2^n}$. The series $\sum 50(\frac{1}{2})^n$ is a convergent geometric series. Since our series is smaller (for $n \ge 4$), it \textbf{converges}.

    \item \textbf{Solution:} Split the series: $\sum \frac{2^n+3^n}{4^n} = \sum \frac{2^n}{4^n} + \sum \frac{3^n}{4^n} = \sum(\frac{1}{2})^n + \sum(\frac{3}{4})^n$. Both are convergent geometric series. The sum of two convergent series is convergent. Thus, the series \textbf{converges}.

    \item \textbf{Solution (DCT):} Let $a_n = \frac{(n!)^2}{(2n)!} = \frac{n! \cdot n!}{(2n)(2n-1)\dots(n+1)n!} = \frac{n!}{(2n)(2n-1)\dots(n+1)}$.
    There are $n$ terms in the denominator. Each term is greater than $n$. So, the denominator is greater than $n^n$.
    $a_n = \frac{n \cdot (n-1) \dots 1}{(2n)(2n-1)\dots(n+1)}$. The term $\frac{n}{2n} = \frac{1}{2}$, $\frac{n-1}{2n-1} < \frac{1}{2}$, etc.
    Let's try a simpler approach. $a_n = \frac{n}{2n} \cdot \frac{n-1}{2n-1} \cdots \frac{1}{n+1} < (\frac{1}{2})^n$. Since $\sum(\frac{1}{2})^n$ converges, the series \textbf{converges}.

    \item \textbf{Solution (DCT):} For $n \ge 1$, $n^2 > n$. Since $e^x$ is an increasing function, $e^{n^2} > e^n$. Thus, $\frac{n}{e^{n^2}} < \frac{n}{e^n}$. It is a known result (from integral or ratio test) that $\sum \frac{n}{e^n}$ converges. A simpler comparison: For $n \ge 2$, $n^2 \ge 2n$, so $e^{n^2} \ge e^{2n} = (e^2)^n$. $\frac{n}{e^{n^2}} \le \frac{n}{(e^2)^n}$. Now compare $\sum \frac{n}{(e^2)^n}$ with a geometric series $(\frac{1}{e})^n$. The limit of their ratio is 0, so the series converges. A better comparison is $b_n = \frac{1}{n^2}$. $\lim_{n \to \infty} \frac{ne^{-n^2}}{1/n^2} = \lim_{n \to \infty} \frac{n^3}{e^{n^2}} = 0$. Since $L=0$ and $\sum \frac{1}{n^2}$ converges, the series \textbf{converges}.

    \item \textbf{Solution (DCT):} For $n \ge 3$, $\ln(n) > 1$. Therefore, $n\ln(n) > n$, which means $\frac{1}{n\ln(n)} < \frac{1}{n}$. This is an inconclusive comparison. The integral test is required for a definitive proof, showing it diverges. If forced to compare, for any $k>1$, $n\ln(n) < n^k$ for large $n$. $\frac{1}{n\ln n} > \frac{1}{n^k}$ (if $k$ close to 1). This series \textbf{diverges}.

    \item \textbf{Solution (LCT):} We know $\ln(n)$ grows slower than any power of $n$. Compare with $b_n = \frac{1}{n^{3/2}}$.
    \[ L = \lim_{n \to \infty} \frac{\ln(n)/n^2}{1/n^{3/2}} = \lim_{n \to \infty} \frac{\ln(n)}{\sqrt{n}} = 0 \quad (\text{by L'Hôpital's Rule}) \]
    Since $L=0$ and $\sum \frac{1}{n^{3/2}}$ converges (p-series, $p=3/2>1$), the original series \textbf{converges}.

    \item \textbf{Solution (LCT):} $a_n = \frac{1}{n \cdot n^{1/n}}$. As $n \to \infty$, $\lim_{n \to \infty} n^{1/n} = 1$. So the term behaves like $1/n$. Compare with $b_n = \frac{1}{n}$.
    \[ L = \lim_{n \to \infty} \frac{1/(n \cdot n^{1/n})}{1/n} = \lim_{n \to \infty} \frac{1}{n^{1/n}} = \frac{1}{1} = 1 \]
    Since $\sum \frac{1}{n}$ diverges and $0<L<\infty$, the series \textbf{diverges}.

    \item \textbf{Solution (DCT):} For large $n$, we know $\ln n < \sqrt{n}$. So $(\ln n)^2 < n$.
    This gives $\frac{1}{(\ln n)^2} > \frac{1}{n}$. Let $b_n = \frac{1}{n}$. Since $\sum b_n$ diverges and our series is term-by-term larger, the series \textbf{diverges}.

    \item \textbf{Solution (LCT):} As in problem 26, compare with a p-series between 1 and 3/2. Let's use $b_n = \frac{1}{n^{5/4}}$.
    \[ L = \lim_{n \to \infty} \frac{\ln(n)/n^{3/2}}{1/n^{5/4}} = \lim_{n \to \infty} \frac{\ln(n)}{n^{1/4}} = 0 \quad (\text{by L'Hôpital's Rule}) \]
    Since $L=0$ and $\sum \frac{1}{n^{5/4}}$ converges (p-series, $p=5/4>1$), the original series \textbf{converges}.

    \item \textbf{Solution (LCT):} Use the hint that for small angles, $\sin(\theta) \approx \theta$. As $n \to \infty$, $1/n^2 \to 0$. So we compare $a_n = \sin(1/n^2)$ with $b_n = \frac{1}{n^2}$.
    \[ L = \lim_{n \to \infty} \frac{\sin(1/n^2)}{1/n^2} = 1 \quad (\text{This is the fundamental limit} \lim_{\theta \to 0} \frac{\sin\theta}{\theta}=1) \]
    Since $\sum \frac{1}{n^2}$ converges (p-series, $p=2>1$) and $0<L<\infty$, the series \textbf{converges}.

    \item \textbf{Solution (LCT):} Use the hint $1-\cos(x) \approx x^2/2$. As $n \to \infty$, $1/n \to 0$.
    $a_n = \frac{1}{n} (1-\cos(1/n))$. We compare with $b_n = \frac{1}{n} (\frac{(1/n)^2}{2}) = \frac{1}{2n^3}$.
    \[ L = \lim_{n \to \infty} \frac{\frac{1}{n}(1-\cos(1/n))}{1/(2n^3)} = \lim_{n \to \infty} \frac{2n^2(1-\cos(1/n))}{1} \]
    Let $x=1/n$. As $n \to \infty, x \to 0$. Limit becomes $\lim_{x \to 0} \frac{2(1-\cos x)}{x^2} = 2(\frac{1}{2})=1$.
    Since $\sum \frac{1}{2n^3}$ converges (p-series, $p=3>1$) and $0<L<\infty$, the series \textbf{converges}.

    \item \textbf{Solution:} First, simplify the term by multiplying by the conjugate:
    \[ a_n = (\sqrt{n^2+1}-n) \frac{\sqrt{n^2+1}+n}{\sqrt{n^2+1}+n} = \frac{n^2+1-n^2}{\sqrt{n^2+1}+n} = \frac{1}{\sqrt{n^2+1}+n} \]
    Now use LCT. Compare with $b_n = \frac{1}{n}$.
    \[ L = \lim_{n \to \infty} \frac{1/(\sqrt{n^2+1}+n)}{1/n} = \lim_{n \to \infty} \frac{n}{\sqrt{n^2+1}+n} = \lim_{n \to \infty} \frac{1}{\sqrt{1+1/n^2}+1} = \frac{1}{2} \]
    Since $\sum \frac{1}{n}$ diverges and $0<L<\infty$, the series \textbf{diverges}.

    \item \textbf{Solution (LCT):} Compare with $b_n = \frac{2^n}{3^n} = (\frac{2}{3})^n$.
    \[ L = \lim_{n \to \infty} \frac{2^n/(3^n+n^3)}{(2/3)^n} = \lim_{n \to \infty} \frac{3^n}{3^n+n^3} = \lim_{n \to \infty} \frac{1}{1+n^3/3^n} = \frac{1}{1+0} = 1 \]
    Since $\sum (\frac{2}{3})^n$ is a convergent geometric series ($|r|=2/3<1$) and $0<L<\infty$, the series \textbf{converges}.

\end{enumerate}

\newpage
\section*{Concept Checklist and Problem Mapping}
This checklist outlines the key concepts and problem types related to the Comparison Tests. The numbers refer to the problems in this document that test each concept.

\subsection*{1. Foundational Concepts}
\begin{itemize}
    \item \textbf{Understanding Direct Comparison Test (DCT) Logic:} Problems 1, 3, 12, 13, 14, 18, 20, 22, 28.
    \item \textbf{Understanding Limit Comparison Test (LCT) Logic:} Problems 2, 4.
    \item \textbf{Identifying Inconclusive Cases:} Problems 1, 3.
    \item \textbf{LCT Special Case (L=0):} Problems 2, 19, 23, 26, 29.
    \item \textbf{LCT Special Case (L=$\infty$):} Problem 2.
\end{itemize}

\subsection*{2. Problem Types and Techniques}
\begin{itemize}
    \item \textbf{Error Analysis:} Problems 3, 4.
    \item \textbf{Comparison with p-Series (Rational \& Algebraic Functions):} Problems 5, 6, 7, 8, 9, 10, 11, 15, 16, 32.
    \item \textbf{Comparison with Geometric Series (Exponential \& Factorial Functions):} Problems 17, 18, 19, 20, 21, 22, 33.
    \item \textbf{Comparison with Harmonic Series:} Problems 7, 11, 12, 24, 27, 28, 32.
    \item \textbf{Handling Logarithmic Terms:} Problems 25, 26, 27, 28, 29.
    \item \textbf{Handling Trigonometric Terms:} Problems 13, 14, 30, 31.
\end{itemize}

\subsection*{3. Advanced Skills}
\begin{itemize}
    \item \textbf{Algebraic Manipulation (e.g., using conjugates):} Problem 32.
    \item \textbf{Using Taylor Series / Small Angle Approximations:} Problems 30, 31.
    \item \textbf{Choosing an appropriate, less obvious comparison series:} Problems 19, 26, 29.
    \item \textbf{Breaking a series into multiple, simpler series:} Problem 21.
    \item \textbf{Identifying behavior from non-dominant terms:} Problem 27.
\end{itemize}


\end{document}