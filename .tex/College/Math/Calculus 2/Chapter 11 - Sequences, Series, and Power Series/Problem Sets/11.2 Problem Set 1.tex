\documentclass[12pt]{article}
\usepackage{amsmath}
\usepackage{amssymb}
\usepackage{geometry}
\geometry{a4paper, margin=1in}

\title{Homework 11.2: Infinite Series Practice Problems}
\author{Generated by Gemini}
\date{\today}

\begin{document}

\maketitle

\section*{Concept Checklist}
This problem set is designed to test the following concepts related to infinite series, as detailed in the source material.
\begin{itemize}
    \item \textbf{Core Concepts:} Understanding the difference between a sequence and a series; definition of convergence and divergence based on the limit of partial sums ($s_n$).
    \item \textbf{Calculating from Partial Sums:} Determining the sum of a series given the formula for its $n$-th partial sum, $s_n$.
    \item \textbf{Analyzing Partial Sums:} Calculating the first few terms of the sequence of partial sums to form a conjecture about the convergence or divergence of a series.
    \item \textbf{Geometric Series:} Identifying a geometric series, its first term ($a$), and its common ratio ($r$). Determining convergence ($|r| < 1$) and calculating its sum using the formula $S = \frac{a}{1-r}$.
    \item \textbf{Test for Divergence:} Applying the test by finding the limit of the $n$-th term, $\lim_{n\to\infty} a_n$. Understanding that if the limit is not zero, the series diverges, and if the limit is zero, the test is inconclusive.
    \item \textbf{Telescoping Series:} Identifying and finding the sum of a telescoping series, often requiring algebraic manipulation like partial fraction decomposition.
    \item \textbf{Problem Analysis ("Find the Flaw"):} Critically evaluating a given solution to identify logical or computational errors in the application of series tests and concepts.
\end{itemize}

\section{Problems}

\subsection{Core Concepts and Partial Sums}

% Problems 1-8
\begin{enumerate}
    \item Explain the fundamental difference between the sequence $\{ \frac{n}{n+1} \}_{n=1}^\infty$ and the series $\sum_{n=1}^\infty \frac{n}{n+1}$.

    \item A series $\sum a_n$ has a sequence of partial sums $\{s_n\}$ where $s_n = \frac{5n-2}{2n+8}$. Does the series converge, and if so, to what sum?

    \item The partial sums of a series $\sum a_n$ are given by $s_n = 4 - e^{-n}$.
    \begin{enumerate}
        \item Find the sum of the series.
        \item Find the first term of the series, $a_1$.
        \item Find a formula for the $n$-th term, $a_n$, for $n \ge 2$.
    \end{enumerate}

    \item Calculate the first eight terms of the sequence of partial sums for the series $\sum_{n=1}^\infty \frac{1}{n^2}$. Based on your calculations, does the series appear to be convergent or divergent? (Note: This is a p-series with $p=2 > 1$ and is known to converge).

    \item Calculate the first eight terms of the sequence of partial sums for the series $\sum_{n=1}^\infty \frac{n}{2n+1}$. Does the series appear to be convergent or divergent? Use the Test for Divergence to confirm your suspicion.

    \item Explain what it means to say that $\sum_{n=1}^\infty a_n = 10$.

    \item The partial sums of a series $\sum a_n$ are given by $s_n = \frac{n^2 - \cos(n)}{3n^2+1}$. Find the sum of the series.
    
    \item Calculate the first eight terms of the sequence of partial sums for $\sum_{n=1}^\infty \cos(\pi n)$. Does it appear the series is convergent or divergent?

\end{enumerate}

\subsection{Geometric Series}
For each series, determine if it is a geometric series. If it is, determine whether it converges or diverges. If it converges, find its sum.

% Problems 9-18
\begin{enumerate}
    \setcounter{enumi}{8}
    \item $\sum_{n=1}^\infty 5 \left(\frac{2}{3}\right)^{n-1}$
    
    \item $\sum_{n=1}^\infty \frac{(-3)^{n-1}}{4^n}$
    
    \item $\sum_{n=0}^\infty \frac{2^{2n}}{5^{n+1}}$

    \item $10 - 2 + 0.4 - 0.08 + \dots$

    \item $\sum_{n=1}^\infty \frac{e^n}{3^{n-1}}$

    \item $\sum_{n=1}^\infty \frac{4^n + 3^n}{5^n}$
    
    \item $\sum_{n=2}^\infty 3 \left(-\frac{1}{2}\right)^n$
    
    \item $\sum_{n=1}^\infty \frac{100}{101^n}$
    
    \item Express the repeating decimal $0.\overline{47} = 0.474747\dots$ as a geometric series and find its sum (as a fraction).
    
    \item A ball is dropped from a height of 10 meters. Each time it strikes the ground, it bounces to 70\% of its previous height. What is the total distance the ball travels?

\end{enumerate}

\subsection{Test for Divergence}
Use the Test for Divergence to determine whether the series diverges. If the test is inconclusive, state so.

% Problems 19-24
\begin{enumerate}
    \setcounter{enumi}{18}
    \item $\sum_{n=1}^\infty \frac{n^2 - 1}{3n^2 + n}$

    \item $\sum_{n=1}^\infty \arctan(n)$

    \item $\sum_{n=1}^\infty \left(1 + \frac{1}{n}\right)^n$
    
    \item $\sum_{n=1}^\infty \frac{n}{\sqrt{n^2+4}}$
    
    \item $\sum_{n=1}^\infty \frac{1}{n}$ (The Harmonic Series)
    
    \item $\sum_{k=1}^\infty \frac{k!}{(k+2)!}$

\end{enumerate}

\subsection{Telescoping Series}
Determine whether the series converges or diverges by finding the $n$-th partial sum and taking the limit. If it converges, find the sum.

% Problems 25-30
\begin{enumerate}
    \setcounter{enumi}{24}
    \item $\sum_{n=1}^\infty \left(\frac{1}{n} - \frac{1}{n+1}\right)$
    
    \item $\sum_{n=1}^\infty \frac{1}{n(n+2)}$
    
    \item $\sum_{n=2}^\infty \frac{2}{n^2 - 1}$
    
    \item $\sum_{n=1}^\infty \ln\left(\frac{n+1}{n}\right)$
    
    \item $\sum_{n=1}^\infty (\cos(\frac{1}{n+1}) - \cos(\frac{1}{n}))$
    
    \item $\sum_{n=3}^\infty \left(\frac{1}{\sqrt{n}} - \frac{1}{\sqrt{n+1}}\right)$
\end{enumerate}

\subsection{Find the Flaw}
In each of the following problems, a flawed solution is presented. Identify the error in reasoning and provide a correct solution.

% Problems 31-33
\begin{enumerate}
    \setcounter{enumi}{30}
    \item \textbf{Problem:} Determine if the series $\sum_{n=1}^\infty \frac{5}{n}$ converges or diverges.
    \newline
    \textbf{Flawed Solution:}
    \begin{enumerate}
        \item We use the Test for Divergence.
        \item We compute the limit of the terms: $\lim_{n\to\infty} a_n = \lim_{n\to\infty} \frac{5}{n} = 0$.
        \item Since the limit of the terms is 0, the series converges.
    \end{enumerate}

    \item \textbf{Problem:} Find the sum of the series $\sum_{n=0}^\infty 3\left(\frac{4}{3}\right)^n$.
    \newline
    \textbf{Flawed Solution:}
    \begin{enumerate}
        \item This is a geometric series with first term $a = 3(4/3)^0 = 3$.
        \item The common ratio is $r = 4/3$.
        \item The sum is $S = \frac{a}{1-r} = \frac{3}{1 - 4/3} = \frac{3}{-1/3} = -9$.
    \end{enumerate}

    \item \textbf{Problem:} Find the sum of the telescoping series $\sum_{n=1}^\infty (\frac{1}{n+1} - \frac{1}{n+3})$.
    \newline
    \textbf{Flawed Solution:}
    \begin{enumerate}
        \item Let's write out the $n$-th partial sum, $s_n$.
        \item $s_n = (\frac{1}{2} - \frac{1}{4}) + (\frac{1}{3} - \frac{1}{5}) + \dots + (\frac{1}{n+1} - \frac{1}{n+3})$.
        \item The terms cancel out, leaving the first term and the last term.
        \item $s_n = \frac{1}{2} - \frac{1}{n+3}$.
        \item The sum is $S = \lim_{n\to\infty} s_n = \lim_{n\to\infty} (\frac{1}{2} - \frac{1}{n+3}) = \frac{1}{2} - 0 = \frac{1}{2}$.
    \end{enumerate}

\end{enumerate}

\newpage
\section{Solutions}

\begin{enumerate}
    \item \textbf{Solution:} The sequence is an ordered list of numbers: $\{\frac{1}{2}, \frac{2}{3}, \frac{3}{4}, \dots\}$. The series is the sum of these numbers: $\frac{1}{2} + \frac{2}{3} + \frac{3}{4} + \dots$.

    \item \textbf{Solution:} The sum of a series is the limit of its sequence of partial sums. $S = \lim_{n\to\infty} s_n = \lim_{n\to\infty} \frac{5n-2}{2n+8} = \lim_{n\to\infty} \frac{5 - 2/n}{2 + 8/n} = \frac{5}{2}$. The series converges to $\frac{5}{2}$.

    \item \textbf{Solution:} 
    \begin{enumerate}
        \item $S = \lim_{n\to\infty} s_n = \lim_{n\to\infty} (4 - e^{-n}) = 4 - 0 = 4$.
        \item $a_1 = s_1 = 4 - e^{-1}$.
        \item $a_n = s_n - s_{n-1} = (4 - e^{-n}) - (4 - e^{-(n-1)}) = e^{-n+1} - e^{-n}$.
    \end{enumerate}

    \item \textbf{Solution:} $s_1 = 1$, $s_2 = 1.25$, $s_3 \approx 1.361$, $s_4 \approx 1.424$, $s_5 = 1.464$, $s_6 \approx 1.491$, $s_7 \approx 1.512$, $s_8 \approx 1.527$. The partial sums are increasing but the amount of increase is getting smaller. It appears to be convergent.

    \item \textbf{Solution:} $s_1 \approx 0.333$, $s_2 \approx 0.733$, $s_3 \approx 1.162$,... The partial sums are increasing and do not appear to be leveling off. It appears to be divergent.
    \textbf{Test for Divergence:} $\lim_{n\to\infty} a_n = \lim_{n\to\infty} \frac{n}{2n+1} = \frac{1}{2}$. Since the limit is not 0, the series diverges.
    
    \item \textbf{Solution:} This means the series is convergent, and its sum is 10. The limit of its sequence of partial sums is 10, i.e., $\lim_{n\to\infty} s_n = 10$.

    \item \textbf{Solution:} $S = \lim_{n\to\infty} s_n = \lim_{n\to\infty} \frac{n^2 - \cos(n)}{3n^2+1} = \lim_{n\to\infty} \frac{1 - \cos(n)/n^2}{3 + 1/n^2} = \frac{1 - 0}{3 + 0} = \frac{1}{3}$. The series converges to $\frac{1}{3}$.

    \item \textbf{Solution:} $a_n = \cos(\pi n) = (-1)^n$. The terms are $-1, 1, -1, 1, \dots$. The sequence of partial sums is $\{-1, 0, -1, 0, -1, \dots\}$. This sequence oscillates and does not approach a single value. The series appears to be divergent.

    \item \textbf{Solution:} This is a geometric series with $a=5$ and $r=2/3$. Since $|r| < 1$, it converges. $S = \frac{a}{1-r} = \frac{5}{1-2/3} = \frac{5}{1/3} = 15$.

    \item \textbf{Solution:} $a_n = \frac{(-3)^{n-1}}{4^n} = \frac{(-3)^{n-1}}{4 \cdot 4^{n-1}} = \frac{1}{4} \left(-\frac{3}{4}\right)^{n-1}$. This is a geometric series with $a=1/4$ and $r=-3/4$. Since $|r| < 1$, it converges. $S = \frac{1/4}{1 - (-3/4)} = \frac{1/4}{7/4} = \frac{1}{7}$.

    \item \textbf{Solution:} $a_n = \frac{2^{2n}}{5^{n+1}} = \frac{(2^2)^n}{5 \cdot 5^n} = \frac{4^n}{5 \cdot 5^n} = \frac{1}{5}\left(\frac{4}{5}\right)^n$. This is a geometric series. The first term (at $n=0$) is $a=1/5$. The ratio is $r=4/5$. Since $|r| < 1$, it converges. $S = \frac{1/5}{1 - 4/5} = \frac{1/5}{1/5} = 1$.

    \item \textbf{Solution:} This is a geometric series with first term $a=10$ and common ratio $r = -2/10 = -1/5$. Since $|r| = 1/5 < 1$, the series converges. $S = \frac{10}{1 - (-1/5)} = \frac{10}{6/5} = \frac{50}{6} = \frac{25}{3}$.

    \item \textbf{Solution:} $a_n = \frac{e^n}{3^{n-1}} = \frac{e \cdot e^{n-1}}{3^{n-1}} = e \left(\frac{e}{3}\right)^{n-1}$. This is a geometric series with $a=e$ and $r=e/3$. Since $e \approx 2.718$, $|r| = e/3 < 1$. It converges. $S = \frac{e}{1-e/3} = \frac{3e}{3-e}$.

    \item \textbf{Solution:} We can split this into two series: $\sum_{n=1}^\infty \left(\frac{4}{5}\right)^n + \sum_{n=1}^\infty \left(\frac{3}{5}\right)^n$. Both are convergent geometric series.
    For the first, $a = 4/5, r=4/5$. Sum is $\frac{4/5}{1-4/5} = 4$.
    For the second, $a = 3/5, r=3/5$. Sum is $\frac{3/5}{1-3/5} = \frac{3}{2}$.
    Total sum is $4 + \frac{3}{2} = \frac{11}{2}$.

    \item \textbf{Solution:} This is a geometric series with $r=-1/2$. The first term is for $n=2$, so $a = 3(-1/2)^2 = 3/4$. Since $|r|<1$, it converges. $S = \frac{a}{1-r} = \frac{3/4}{1 - (-1/2)} = \frac{3/4}{3/2} = \frac{1}{2}$.
    
    \item \textbf{Solution:} This is a geometric series, which can be written as $\sum_{n=1}^\infty 100 \left(\frac{1}{101}\right)^n$. The first term is $a = 100/101$ and the ratio is $r=1/101$. Since $|r|<1$, it converges. $S = \frac{100/101}{1-1/101} = \frac{100/101}{100/101} = 1$.

    \item \textbf{Solution:} $0.\overline{47} = \frac{47}{100} + \frac{47}{10000} + \frac{47}{100^3} + \dots$. This is a geometric series with $a = 47/100$ and $r=1/100$. It converges to $S = \frac{47/100}{1-1/100} = \frac{47/100}{99/100} = \frac{47}{99}$.

    \item \textbf{Solution:} Total distance = $10 + 2(10 \cdot 0.7) + 2(10 \cdot 0.7^2) + \dots = 10 + \sum_{n=1}^\infty 20(0.7)^n$. The sum is a geometric series with $a=20(0.7)=14$ and $r=0.7$. Sum = $\frac{14}{1-0.7} = \frac{14}{0.3} = \frac{140}{3}$. Total distance = $10 + \frac{140}{3} = \frac{170}{3}$ meters.

    \item \textbf{Solution:} $\lim_{n\to\infty} a_n = \lim_{n\to\infty} \frac{n^2 - 1}{3n^2 + n} = \frac{1}{3}$. Since the limit is not 0, the series diverges by the Test for Divergence.

    \item \textbf{Solution:} $\lim_{n\to\infty} a_n = \lim_{n\to\infty} \arctan(n) = \frac{\pi}{2}$. Since the limit is not 0, the series diverges by the Test for Divergence.
    
    \item \textbf{Solution:} $\lim_{n\to\infty} a_n = \lim_{n\to\infty} \left(1 + \frac{1}{n}\right)^n = e$. Since the limit is not 0, the series diverges by the Test for Divergence.

    \item \textbf{Solution:} $\lim_{n\to\infty} a_n = \lim_{n\to\infty} \frac{n}{\sqrt{n^2+4}} = \lim_{n\to\infty} \frac{1}{\sqrt{1+4/n^2}} = 1$. Since the limit is not 0, the series diverges by the Test for Divergence.

    \item \textbf{Solution:} $\lim_{n\to\infty} a_n = \lim_{n\to\infty} \frac{1}{n} = 0$. The Test for Divergence is inconclusive. (Note: The harmonic series is known to diverge).

    \item \textbf{Solution:} $a_k = \frac{k!}{(k+2)!} = \frac{k!}{k!(k+1)(k+2)} = \frac{1}{(k+1)(k+2)}$. $\lim_{k\to\infty} a_k = 0$. The Test for Divergence is inconclusive.

    \item \textbf{Solution:} $s_n = (1 - \frac{1}{2}) + (\frac{1}{2} - \frac{1}{3}) + \dots + (\frac{1}{n} - \frac{1}{n+1}) = 1 - \frac{1}{n+1}$.
    $S = \lim_{n\to\infty} s_n = \lim_{n\to\infty} (1 - \frac{1}{n+1}) = 1$. The series converges to 1.
    
    \item \textbf{Solution:} Use partial fractions: $\frac{1}{n(n+2)} = \frac{A}{n} + \frac{B}{n+2} \Rightarrow 1 = A(n+2) + Bn$. Let $n=0 \Rightarrow A=1/2$. Let $n=-2 \Rightarrow B=-1/2$.
    $a_n = \frac{1}{2}\left(\frac{1}{n} - \frac{1}{n+2}\right)$.
    $s_n = \frac{1}{2}\left[ (1 - \frac{1}{3}) + (\frac{1}{2} - \frac{1}{4}) + (\frac{1}{3} - \frac{1}{5}) + \dots + (\frac{1}{n} - \frac{1}{n+2}) \right]$.
    The terms that don't cancel are $1$ and $1/2$ at the beginning, and $-\frac{1}{n+1}$ and $-\frac{1}{n+2}$ at the end.
    $s_n = \frac{1}{2}\left(1 + \frac{1}{2} - \frac{1}{n+1} - \frac{1}{n+2}\right)$.
    $S = \lim_{n\to\infty} s_n = \frac{1}{2}(1 + \frac{1}{2} - 0 - 0) = \frac{3}{4}$.

    \item \textbf{Solution:} Use partial fractions: $\frac{2}{n^2 - 1} = \frac{2}{(n-1)(n+1)} = \frac{1}{n-1} - \frac{1}{n+1}$.
    $s_n = \sum_{k=2}^n (\frac{1}{k-1} - \frac{1}{k+1}) = (1 - \frac{1}{3}) + (\frac{1}{2} - \frac{1}{4}) + \dots + (\frac{1}{n-1} - \frac{1}{n+1})$.
    The terms that remain are $1$ and $1/2$ at the beginning, and $-\frac{1}{n}$ and $-\frac{1}{n+1}$ at the end.
    $s_n = 1 + \frac{1}{2} - \frac{1}{n} - \frac{1}{n+1}$.
    $S = \lim_{n\to\infty} s_n = 1 + \frac{1}{2} = \frac{3}{2}$.

    \item \textbf{Solution:} Using logarithm properties, $\ln(\frac{n+1}{n}) = \ln(n+1) - \ln(n)$.
    $s_n = (\ln(2)-\ln(1)) + (\ln(3)-\ln(2)) + \dots + (\ln(n+1) - \ln(n)) = \ln(n+1) - \ln(1) = \ln(n+1)$.
    $S = \lim_{n\to\infty} s_n = \lim_{n\to\infty} \ln(n+1) = \infty$. The series diverges.
    
    \item \textbf{Solution:} This is a telescoping series. Let $b_n = \cos(1/n)$. The series is $\sum (b_{n+1}-b_n)$.
    $s_n = (\cos(1/2) - \cos(1)) + (\cos(1/3) - \cos(1/2)) + \dots + (\cos(\frac{1}{n+1}) - \cos(\frac{1}{n})) = \cos(\frac{1}{n+1}) - \cos(1)$.
    $S = \lim_{n\to\infty} s_n = \lim_{n\to\infty} (\cos(\frac{1}{n+1}) - \cos(1)) = \cos(0) - \cos(1) = 1 - \cos(1)$.

    \item \textbf{Solution:} This is a telescoping series starting at $n=3$.
    $s_N = \sum_{n=3}^N (\frac{1}{\sqrt{n}} - \frac{1}{\sqrt{n+1}}) = (\frac{1}{\sqrt{3}} - \frac{1}{\sqrt{4}}) + (\frac{1}{\sqrt{4}} - \frac{1}{\sqrt{5}}) + \dots + (\frac{1}{\sqrt{N}} - \frac{1}{\sqrt{N+1}}) = \frac{1}{\sqrt{3}} - \frac{1}{\sqrt{N+1}}$.
    $S = \lim_{N\to\infty} s_N = \frac{1}{\sqrt{3}} - 0 = \frac{1}{\sqrt{3}}$.
    
    \item \textbf{Solution:}
    \textbf{Flaw:} The conclusion in step 3 is incorrect. The Test for Divergence can only prove divergence, never convergence. If the limit of the terms is 0, the test is inconclusive.
    \textbf{Correct Solution:} This is the harmonic series multiplied by a constant (5). The harmonic series $\sum \frac{1}{n}$ is a p-series with $p=1$, which is known to diverge. Therefore, the series $\sum \frac{5}{n}$ also diverges.

    \item \textbf{Solution:}
    \textbf{Flaw:} Step 3 misuses the geometric series sum formula. The formula $S = a/(1-r)$ is only valid when the series converges, which requires $|r| < 1$.
    \textbf{Correct Solution:} This is a geometric series with ratio $r = 4/3$. Since $|r| = 4/3 \ge 1$, the series diverges. There is no finite sum.

    \item \textbf{Solution:}
    \textbf{Flaw:} The cancellation pattern in step 3 is identified incorrectly. The term $-1/4$ from the first parentheses does not cancel with $1/3$ from the second. More terms must be written out to see the correct pattern.
    \textbf{Correct Solution:}
    $s_n = (\frac{1}{2}-\frac{1}{4}) + (\frac{1}{3}-\frac{1}{5}) + (\frac{1}{4}-\frac{1}{6}) + \dots + (\frac{1}{n}-\frac{1}{n+2}) + (\frac{1}{n+1}-\frac{1}{n+3})$.
    The $-1/4$ cancels with the $1/4$. The terms that do not cancel are $1/2$ and $1/3$ from the beginning, and $-1/(n+2)$ and $-1/(n+3)$ from the end.
    $s_n = \frac{1}{2} + \frac{1}{3} - \frac{1}{n+2} - \frac{1}{n+3}$.
    $S = \lim_{n\to\infty} s_n = \frac{1}{2} + \frac{1}{3} = \frac{5}{6}$.

\end{enumerate}

\newpage
\section*{Concept Checklist with Problem Mapping}

\begin{itemize}
    \item \textbf{Core Concepts:} 1, 6, 8
    \item \textbf{Calculating from Partial Sums:} 2, 3, 7
    \item \textbf{Analyzing Partial Sums:} 4, 5, 8
    \item \textbf{Geometric Series:} 9, 10, 11, 12, 13, 14, 15, 16, 17, 18
    \item \textbf{Test for Divergence:} 5, 19, 20, 21, 22, 23, 24
    \item \textbf{Telescoping Series:} 25, 26, 27, 28, 29, 30
    \item \textbf{Problem Analysis ("Find the Flaw"):} 31, 32, 33
\end{itemize}

\end{document}