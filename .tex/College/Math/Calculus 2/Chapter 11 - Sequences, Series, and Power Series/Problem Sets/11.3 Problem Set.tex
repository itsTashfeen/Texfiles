\documentclass[12pt]{article}
\usepackage{amsmath}
\usepackage{amssymb}
\usepackage{geometry}
\geometry{a4paper, margin=1in}

\title{Problem Set: The Integral Test and Estimates of Sums}
\author{Calculus II Practice}
\date{\today}

\begin{document}

\maketitle

\section*{Instructions}
Use the Integral Test, p-series test, or the Test for Divergence to determine whether the following series are convergent or divergent. For problems that ask for a full evaluation, provide the value of the corresponding integral.

\section*{Problems}
\begin{enumerate}

    %-- Concept C2d, C3c (Rational function, u-sub -> log)
    \item Use the Integral Test to determine if the series $\sum_{n=1}^{\infty} \frac{6}{3n+2}$ converges or diverges. Evaluate the corresponding integral.
    \item Use the Integral Test to determine if the series $\sum_{n=1}^{\infty} \frac{n}{n^2+9}$ converges or diverges. Evaluate the corresponding integral.
    \item Use the Integral Test to determine if the series $\sum_{n=2}^{\infty} \frac{n^2}{n^3 - 4}$ converges or diverges. Evaluate the corresponding integral.

    %-- Concept C2c, C3c (Exponential, u-sub)
    \item Use the Integral Test to determine if the series $\sum_{n=1}^{\infty} n e^{-n^2}$ converges or diverges. Evaluate the corresponding integral.
    \item Use the Integral Test to determine if the series $\sum_{n=1}^{\infty} \frac{e^{1/n}}{n^2}$ converges or diverges. Evaluate the corresponding integral.

    %-- Concept C2e, C3e (Inverse Trig)
    \item Use the Integral Test to determine if the series $\sum_{n=0}^{\infty} \frac{1}{n^2+1}$ converges or diverges. Evaluate the corresponding integral.
    \item Use the Integral Test to determine if the series $\sum_{n=1}^{\infty} \frac{5}{n^2+25}$ converges or diverges. Evaluate the corresponding integral.

    %-- Concept C2b, C3c (Logarithmic)
    \item Use the Integral Test to determine if the series $\sum_{n=2}^{\infty} \frac{1}{n(\ln n)^3}$ converges or diverges. Evaluate the corresponding integral.
    \item Use the Integral Test to determine if the series $\sum_{n=3}^{\infty} \frac{1}{n\sqrt{\ln n}}$ converges or diverges. Evaluate the corresponding integral.
    
    %-- Concept C3d (Integration by Parts)
    \item Use the Integral Test to determine if the series $\sum_{n=1}^{\infty} \frac{\ln n}{n^2}$ converges or diverges. Evaluate the corresponding integral.

    %-- Concept C2a (p-series)
    \item Determine whether the series $\sum_{n=1}^{\infty} \frac{1}{\sqrt[5]{n}}$ is convergent or divergent.
    \item Determine whether the series $\sum_{n=1}^{\infty} n^{-1.0001}$ is convergent or divergent.
    \item Determine whether the series $\sum_{n=1}^{\infty} \frac{\sqrt{n}}{n}$ is convergent or divergent.
    \item Determine whether the series $\sum_{n=1}^{\infty} \frac{1}{n^{\pi/2}}$ is convergent or divergent.
    
    %-- Concept C4 (Test for Divergence)
    \item Determine whether the series $\sum_{n=1}^{\infty} \frac{3n-2}{5n+1}$ is convergent or divergent.
    \item Determine whether the series $\sum_{n=1}^{\infty} \arctan(n)$ is convergent or divergent.
    \item Determine whether the series $\sum_{n=1}^{\infty} \left(1 + \frac{1}{n}\right)^n$ is convergent or divergent.

    %-- Concept C5 (Pattern Recognition)
    \item Determine if the following series is convergent or divergent: $1 + \frac{1}{8} + \frac{1}{27} + \frac{1}{64} + \dots$
    \item Determine if the following series is convergent or divergent: $\frac{1}{5} + \frac{1}{7} + \frac{1}{9} + \frac{1}{11} + \dots$
    \item Determine if the following series is convergent or divergent: $1 + \frac{1}{2\sqrt{2}} + \frac{1}{3\sqrt{3}} + \frac{1}{4\sqrt{4}} + \dots$
    \item Determine if the following series is convergent or divergent: $\frac{\ln(2)}{2} + \frac{\ln(3)}{3} + \frac{\ln(4)}{4} + \dots$

    %-- Concept C1 (Checking Conditions)
    \item Does the function $f(x) = \frac{x}{x^2-1}$ satisfy the conditions of the Integral Test for the series $\sum_{n=2}^{\infty} \frac{n}{n^2-1}$? Explain why or why not.
    \item Does the function $f(x) = \frac{2+\cos(x)}{x^2}$ satisfy the conditions for the Integral Test on $[1, \infty)$? Explain why or why not.
    \item Explain why the Integral Test cannot be used for the series $\sum_{n=1}^{\infty} \frac{\sin^2(n)}{n^2}$.

    %--- Mixed Practice
    \item Determine if the series $\sum_{n=1}^{\infty} \frac{n+4}{n^2+1}$ is convergent or divergent.
    \item Determine if the series $\sum_{n=1}^{\infty} \frac{1}{n^2+3n+2}$ is convergent or divergent.
    \item Determine if the series $\sum_{n=2}^{\infty} \frac{1}{n \ln(n^2)}$ is convergent or divergent.
    \item Determine if the series $\sum_{n=1}^{\infty} 5n^{-2/3}$ is convergent or divergent.
    \item Determine if the series $\sum_{n=1}^{\infty} \frac{e^{-\sqrt{n}}}{\sqrt{n}}$ is convergent or divergent.
    \item Determine if the series $\sum_{n=1}^{\infty} \frac{n}{e^n}$ is convergent or divergent.
    \item Determine if the series $\sum_{n=1}^{\infty} n \sin(1/n)$ is convergent or divergent.
    \item Determine if the series $\sum_{n=5}^{\infty} \frac{1}{(n-4)^3}$ is convergent or divergent.
    \item Determine if the series $\sum_{n=2}^{\infty} \frac{\ln(n)}{n}$ is convergent or divergent.
\end{enumerate}

\newpage
\section*{Solutions}
\begin{enumerate}
    \item \textbf{Divergent}. The function $f(x) = \frac{6}{3x+2}$ is continuous, positive, and decreasing for $x \ge 1$.
    $\int_{1}^{\infty} \frac{6}{3x+2} dx = \lim_{t \to \infty} [2\ln(3x+2)]_{1}^{t} = \lim_{t \to \infty} (2\ln(3t+2) - 2\ln(5)) = \infty$. Since the integral diverges, the series diverges.

    \item \textbf{Divergent}. $f(x) = \frac{x}{x^2+9}$ is continuous, positive, and decreasing for $x \ge 3$.
    $\int_{1}^{\infty} \frac{x}{x^2+9} dx = \lim_{t \to \infty} [\frac{1}{2}\ln(x^2+9)]_{1}^{t} = \lim_{t \to \infty} (\frac{1}{2}\ln(t^2+9) - \frac{1}{2}\ln(10)) = \infty$. The integral diverges, so the series diverges.

    \item \textbf{Divergent}. $f(x) = \frac{x^2}{x^3-4}$ is continuous, positive, and decreasing for $x \ge 2$.
    $\int_{2}^{\infty} \frac{x^2}{x^3-4} dx = \lim_{t \to \infty} [\frac{1}{3}\ln(x^3-4)]_{2}^{t} = \lim_{t \to \infty} (\frac{1}{3}\ln(t^3-4) - \frac{1}{3}\ln(4)) = \infty$. The integral diverges, so the series diverges.

    \item \textbf{Convergent}. $f(x) = xe^{-x^2}$ is positive, continuous, and decreasing for $x \ge 1$. Let $u = -x^2, du = -2x dx$.
    $\int_{1}^{\infty} xe^{-x^2} dx = \lim_{t \to \infty} [-\frac{1}{2}e^{-x^2}]_{1}^{t} = \lim_{t \to \infty} (-\frac{1}{2}e^{-t^2} - (-\frac{1}{2}e^{-1})) = 0 + \frac{1}{2e} = \frac{1}{2e}$. The integral converges, so the series converges.
    
    \item \textbf{Convergent}. $f(x) = \frac{e^{1/x}}{x^2}$ is positive, continuous, and decreasing for $x \ge 1$. Let $u = 1/x, du = -1/x^2 dx$.
    $\int_{1}^{\infty} \frac{e^{1/x}}{x^2} dx = \lim_{t \to \infty} [-e^{1/x}]_{1}^{t} = \lim_{t \to \infty} (-e^{1/t} - (-e^{1})) = -e^0 + e = e-1$. The integral converges, so the series converges.
    
    \item \textbf{Convergent}. $f(x) = \frac{1}{x^2+1}$ is positive, continuous, and decreasing for $x \ge 0$.
    $\int_{0}^{\infty} \frac{1}{x^2+1} dx = \lim_{t \to \infty} [\arctan(x)]_{0}^{t} = \lim_{t \to \infty} (\arctan(t) - \arctan(0)) = \frac{\pi}{2} - 0 = \frac{\pi}{2}$. The integral converges, so the series converges.

    \item \textbf{Convergent}. $f(x) = \frac{5}{x^2+25}$ is positive, continuous, and decreasing for $x \ge 1$.
    $\int_{1}^{\infty} \frac{5}{x^2+25} dx = 5 \lim_{t \to \infty} [\frac{1}{5}\arctan(\frac{x}{5})]_{1}^{t} = \lim_{t \to \infty} (\arctan(\frac{t}{5}) - \arctan(\frac{1}{5})) = \frac{\pi}{2} - \arctan(\frac{1}{5})$. The integral converges, so the series converges.
    
    \item \textbf{Convergent}. $f(x) = \frac{1}{x(\ln x)^3}$ is continuous, positive, and decreasing for $x \ge 2$. Let $u = \ln x, du = \frac{1}{x} dx$.
    $\int_{2}^{\infty} \frac{1}{x(\ln x)^3} dx = \lim_{t \to \infty} \int_{\ln 2}^{\ln t} u^{-3} du = \lim_{t \to \infty} [-\frac{1}{2u^2}]_{\ln 2}^{\ln t} = \lim_{t \to \infty} (-\frac{1}{2(\ln t)^2} + \frac{1}{2(\ln 2)^2}) = \frac{1}{2(\ln 2)^2}$. The integral converges, so the series converges.

    \item \textbf{Divergent}. $f(x) = \frac{1}{x\sqrt{\ln x}}$ is continuous, positive, and decreasing for $x \ge 3$. Let $u = \ln x, du = \frac{1}{x} dx$.
    $\int_{3}^{\infty} \frac{1}{x\sqrt{\ln x}} dx = \lim_{t \to \infty} \int_{\ln 3}^{\ln t} u^{-1/2} du = \lim_{t \to \infty} [2\sqrt{u}]_{\ln 3}^{\ln t} = \lim_{t \to \infty} (2\sqrt{\ln t} - 2\sqrt{\ln 3}) = \infty$. The integral diverges, so the series diverges.

    \item \textbf{Convergent}. $f(x) = \frac{\ln x}{x^2}$ is positive, continuous, and decreasing for $x \ge 2$. Use Integration by Parts: $u = \ln x, dv = x^{-2}dx$.
    $\int_{1}^{\infty} \frac{\ln x}{x^2} dx = \lim_{t \to \infty} [-\frac{\ln x}{x} - \frac{1}{x}]_{1}^{t} = \lim_{t \to \infty} (-\frac{\ln t}{t} - \frac{1}{t}) - (0-1) = 0 - 0 + 1 = 1$. (Note: $\lim_{t\to\infty} \frac{\ln t}{t} = 0$ by L'Hôpital's Rule). The integral converges, so the series converges.

    \item \textbf{Divergent}. This is a p-series $\sum \frac{1}{n^p}$ with $p = 1/5$. Since $p \le 1$, the series diverges.

    \item \textbf{Convergent}. This is a p-series with $p = 1.0001$. Since $p > 1$, the series converges.
    
    \item \textbf{Divergent}. The series is $\sum \frac{n^{1/2}}{n} = \sum \frac{1}{n^{1/2}}$. This is a p-series with $p = 1/2$. Since $p \le 1$, the series diverges.
    
    \item \textbf{Convergent}. This is a p-series with $p=\pi/2 \approx 1.57$. Since $p > 1$, the series converges.
    
    \item \textbf{Divergent}. Use the Test for Divergence: $\lim_{n \to \infty} \frac{3n-2}{5n+1} = \frac{3}{5} \neq 0$. The series diverges.
    
    \item \textbf{Divergent}. Use the Test for Divergence: $\lim_{n \to \infty} \arctan(n) = \frac{\pi}{2} \neq 0$. The series diverges.
    
    \item \textbf{Divergent}. Use the Test for Divergence: $\lim_{n \to \infty} \left(1 + \frac{1}{n}\right)^n = e \neq 0$. The series diverges.

    \item \textbf{Convergent}. The series can be written as $\sum_{n=1}^{\infty} \frac{1}{n^3}$. This is a p-series with $p=3$. Since $p > 1$, the series converges.

    \item \textbf{Divergent}. The series is $\sum_{n=1}^{\infty} \frac{1}{2n+3}$. Let $f(x) = \frac{1}{2x+3}$. The integral $\int_1^\infty \frac{1}{2x+3} dx = \lim_{t\to\infty} [\frac{1}{2}\ln(2x+3)]_1^t = \infty$. The series diverges. (Can also use Limit Comparison Test with $\sum 1/n$).

    \item \textbf{Convergent}. The series can be written as $\sum_{n=1}^{\infty} \frac{1}{n\sqrt{n}} = \sum_{n=1}^{\infty} \frac{1}{n^{3/2}}$. This is a p-series with $p=3/2$. Since $p > 1$, the series converges.

    \item \textbf{Divergent}. The series is $\sum_{n=2}^{\infty} \frac{\ln(n)}{n}$. The function $f(x) = \frac{\ln x}{x}$ is positive, continuous, and decreasing for $x \ge 3$. $\int_2^\infty \frac{\ln x}{x} dx = \lim_{t\to\infty} [\frac{1}{2}(\ln x)^2]_2^t = \infty$. The series diverges.
    
    \item \textbf{Yes}. For $x \ge 2$:
        \begin{itemize}
            \item \textbf{Positive:} For $x \ge 2$, $x > 0$ and $x^2-1 > 0$, so $f(x)$ is positive.
            \item \textbf{Continuous:} $f(x)$ is a rational function, continuous wherever the denominator is not zero ($x \neq \pm 1$), so it is continuous on $[2, \infty)$.
            \item \textbf{Decreasing:} $f'(x) = \frac{(x^2-1)(1) - x(2x)}{(x^2-1)^2} = \frac{-x^2-1}{(x^2-1)^2}$. Since the numerator is always negative and the denominator is always positive for $x \ge 2$, $f'(x) < 0$, so $f(x)$ is decreasing.
        \end{itemize}
    
    \item \textbf{Yes}. For $x \ge 1$:
        \begin{itemize}
            \item \textbf{Positive:} Since $-1 \le \cos(x) \le 1$, the numerator $2+\cos(x)$ is always between 1 and 3. The denominator $x^2$ is positive. So $f(x)$ is positive.
            \item \textbf{Continuous:} The numerator and denominator are continuous, and the denominator is never zero on $[1, \infty)$, so $f(x)$ is continuous.
            \item \textbf{Decreasing:} $f'(x) = \frac{-x\sin(x) - 4 - 2\cos(x)}{x^3}$. For large $x$, the numerator is dominated by the $-4$ term, making $f'(x)$ negative. The function is eventually decreasing.
        \end{itemize}
        
    \item The Integral Test requires the function $f(x)$ to be \textbf{decreasing}. The function $f(x) = \frac{\sin^2(x)}{x^2}$ is not decreasing on $[1, \infty)$ because the $\sin^2(x)$ term oscillates between 0 and 1, causing the function to have many local maxima and minima.

    \item \textbf{Divergent}. Use the Limit Comparison Test with the harmonic series $\sum \frac{1}{n}$. $\lim_{n\to\infty} \frac{(n+4)/(n^2+1)}{1/n} = \lim_{n\to\infty} \frac{n(n+4)}{n^2+1} = 1$. Since the limit is a finite positive number and $\sum \frac{1}{n}$ diverges, the series diverges. The Integral test could also be used.

    \item \textbf{Convergent}. The function $f(x) = \frac{1}{x^2+3x+2}$ is positive, continuous, and decreasing for $x \ge 1$. This can be compared to $\sum \frac{1}{n^2}$, which converges. Using the Integral Test, $\int_1^\infty \frac{1}{(x+1)(x+2)} dx$ can be solved with partial fractions and is convergent.

    \item \textbf{Divergent}. The series is $\sum_{n=2}^{\infty} \frac{1}{2n \ln(n)}$. This is a constant multiple ($1/2$) of the series $\sum_{n=2}^{\infty} \frac{1}{n \ln(n)}$, which diverges by the Integral Test (p-series test for logarithms with p=1).
    
    \item \textbf{Divergent}. This is $5 \sum n^{-2/3}$. It is a constant multiple of a p-series with $p=2/3$. Since $p \le 1$, the series diverges.
    
    \item \textbf{Convergent}. Use the Integral Test with $f(x) = \frac{e^{-\sqrt{x}}}{\sqrt{x}}$. Let $u = -\sqrt{x}$, $du = -\frac{1}{2\sqrt{x}}dx$.
    $\int_1^\infty \frac{e^{-\sqrt{x}}}{\sqrt{x}} dx = \lim_{t\to\infty} [-2e^{-\sqrt{x}}]_1^t = \lim_{t\to\infty} (-2e^{-\sqrt{t}} + 2e^{-1}) = \frac{2}{e}$. The integral converges, so the series converges.
    
    \item \textbf{Convergent}. Use the Integral Test. $f(x)=xe^{-x}$ is positive, continuous, and decreasing for $x \ge 1$. Integrate by parts: $\int_1^\infty xe^{-x} dx = \lim_{t\to\infty} [-xe^{-x} - e^{-x}]_1^t = (0-0) - (-e^{-1}-e^{-1}) = \frac{2}{e}$. The integral converges, so the series converges.
    
    \item \textbf{Divergent}. Use the Test for Divergence. $\lim_{n\to\infty} n \sin(1/n) = \lim_{n\to\infty} \frac{\sin(1/n)}{1/n}$. Let $x=1/n$. As $n\to\infty$, $x\to 0$. The limit becomes $\lim_{x\to 0} \frac{\sin(x)}{x} = 1 \neq 0$. The series diverges.
    
    \item \textbf{Convergent}. This is a shifted p-series. Let $k = n-4$. When $n=5, k=1$. The series is $\sum_{k=1}^\infty \frac{1}{k^3}$. This is a p-series with $p=3$. Since $p>1$, it converges.
    
    \item \textbf{Divergent}. Use the Integral Test with $f(x)=\frac{\ln x}{x}$. Let $u=\ln x, du = \frac{1}{x}dx$. $\int_2^\infty \frac{\ln x}{x} dx = \lim_{t\to\infty} [\frac{1}{2}(\ln x)^2]_2^t = \infty$. The integral diverges, so the series diverges.
\end{enumerate}

\newpage
\section*{Concept Checklist and Problem Mapping}
This checklist outlines the key concepts tested in this problem set. The numbers refer to the problems that primarily test each concept.

\begin{itemize}
    \item[\textbf{C1:}] \textbf{The Integral Test Conditions:} Verifying if a function is continuous, positive, and decreasing.
        \begin{itemize}
            \item Problems: 22, 23, 24
        \end{itemize}

    \item[\textbf{C2:}] \textbf{Applying the Integral Test for Convergence/Divergence:}
        \begin{itemize}
            \item[\textbf{C2a:}] \textbf{p-Series:} Directly applying the p-series test ($p>1$ converges, $p \le 1$ diverges).
                \begin{itemize}
                    \item Problems: 11, 12, 13, 14, 18, 20, 28, 32
                \end{itemize}
            \item[\textbf{C2b:}] \textbf{Logarithmic Functions:} Series of the form $\sum \frac{1}{n(\ln n)^p}$.
                \begin{itemize}
                    \item Problems: 8, 9, 21, 27, 33
                \end{itemize}
            \item[\textbf{C2c:}] \textbf{Exponential Functions:} Series involving exponential terms.
                \begin{itemize}
                    \item Problems: 4, 5, 29, 30
                \end{itemize}
            \item[\textbf{C2d:}] \textbf{Rational Functions:} Series where the corresponding integral is of a rational function.
                \begin{itemize}
                    \item Problems: 1, 2, 3, 25, 26
                \end{itemize}
            \item[\textbf{C2e:}] \textbf{Inverse Trig Functions:} Series where the integral leads to an arctan function.
                \begin{itemize}
                    \item Problems: 6, 7
                \end{itemize}
        \end{itemize}

    \item[\textbf{C3:}] \textbf{Prerequisite Skill - Evaluating Improper Integrals:}
        \begin{itemize}
            \item[\textbf{C3a, C3b:}] Basic Power and Logarithmic Rules are implicit in many problems.
            \item[\textbf{C3c:}] \textbf{U-Substitution:} Required for integral evaluation.
                \begin{itemize}
                    \item Problems: 1, 2, 3, 4, 5, 8, 9, 29, 33
                \end{itemize}
            \item[\textbf{C3d:}] \textbf{Integration by Parts:} Required for more complex integrals.
                \begin{itemize}
                    \item Problems: 10, 30
                \end{itemize}
            \item[\textbf{C3e:}] \textbf{Inverse Trig Integrals:} Integrals of the form $\int \frac{1}{x^2+a^2} dx$.
                \begin{itemize}
                    \item Problems: 6, 7
                \end{itemize}
        \end{itemize}

    \item[\textbf{C4:}] \textbf{Test for Divergence:} Applying the test where $\lim_{n \to \infty} a_n \neq 0$.
        \begin{itemize}
            \item Problems: 15, 16, 17, 31
        \end{itemize}
    
    \item[\textbf{C5:}] \textbf{Pattern Recognition:} Deducing the general term $a_n$ from the first few terms of a series.
        \begin{itemize}
            \item Problems: 18, 19, 20, 21
        \end{itemize}
\end{itemize}

\end{document}