\documentclass{article}
\usepackage{amsmath}
\usepackage{amssymb}
\usepackage[margin=1in]{geometry}

\title{Homework 11.1 Sequences}
\author{Tashfeen Omran}
\date{\today}

\begin{document}

\maketitle

\section{Comprehensive Introduction, Context, and Prerequisites}

\subsection{Core Concepts}
An **infinite sequence** is a list of numbers written in a specific, definite order. We can think of it as a function whose domain is the set of positive integers. Instead of using function notation like $f(n)$, we typically denote the terms of a sequence by $a_n$. The entire sequence is represented by notations such as $\{a_1, a_2, a_3, \dots\}$, $\{a_n\}$, or $\{a_n\}_{n=1}^{\infty}$.

\subsubsection{Convergence and Divergence}
The central question we ask about a sequence is: What happens to the terms $a_n$ as the index $n$ gets very large?

\begin{itemize}
    \item \textbf{Convergence:} A sequence $\{a_n\}$ is said to \textbf{converge} if its terms approach a single, finite number $L$ as $n$ approaches infinity. We call $L$ the \textbf{limit} of the sequence and write:
    \[ \lim_{n \to \infty} a_n = L \quad \text{or} \quad a_n \to L \text{ as } n \to \infty \]
    Intuitively, this means we can make the terms $a_n$ as close to $L$ as we like, simply by choosing a large enough $n$.

    \item \textbf{Divergence:} If a sequence does not converge, it is said to \textbf{diverge}. Divergence can happen in a few ways:
    \begin{enumerate}
        \item The terms approach infinity ($\infty$) or negative infinity ($-\infty$). For example, $a_n = n^2$ diverges to $\infty$.
        \item The terms oscillate between two or more values without settling on one. For example, $a_n = (-1)^n$ oscillates between -1 and 1 and thus diverges.
    \end{enumerate}
\end{itemize}

\subsubsection{Monotonic and Bounded Sequences}
\begin{itemize}
    \item A sequence is \textbf{monotonic} if it is either always increasing ($a_n \le a_{n+1}$ for all $n$) or always decreasing ($a_n \ge a_{n+1}$ for all $n$).
    \item A sequence is \textbf{bounded} if there are numbers $m$ and $M$ such that $m \le a_n \le M$ for all $n$.
\end{itemize}
These concepts are important because of the \textbf{Monotonic Sequence Theorem}, which states that every bounded, monotonic sequence converges.

\subsection{Intuition and Derivation}
The concept of a limit for a sequence is a discrete analogue of the limit of a function $f(x)$ as $x \to \infty$. Imagine plotting the terms of a sequence $(n, a_n)$ on a graph. If the sequence converges to $L$, the points will get closer and closer to the horizontal line $y=L$ as you move to the right (as $n$ increases).

Many of the rules for finding limits of functions apply directly to sequences. This is formalized by a crucial theorem: If $\lim_{x \to \infty} f(x) = L$ and $f(n) = a_n$ for all integers $n$, then $\lim_{n \to \infty} a_n = L$. This allows us to use powerful tools like L'Hôpital's Rule to find the limits of sequences by analyzing their continuous counterparts.

\subsection{Historical Context and Motivation}
The idea of infinite processes has intrigued mathematicians for millennia. The ancient Greek philosopher Zeno of Elea posed his famous paradoxes in the 5th century BC. One paradox argues that to walk across a room, you must first walk half the distance, then half the remaining distance, and so on, ad infinitum. This process creates an infinite sequence of distances: $\frac{1}{2}, \frac{1}{4}, \frac{1}{8}, \dots$. Does the sum of these infinite terms equal the finite length of the room?

While Zeno used this to question the nature of space and time, mathematicians later used it to motivate the study of infinite sequences and series. It wasn't until the 19th century that mathematicians like Augustin-Louis Cauchy and Karl Weierstrass established a rigorous, formal foundation for calculus, including precise definitions of limits and convergence. This rigor was necessary to confidently work with the infinite processes that are central to calculus, from the definition of the derivative to the powerful tool of infinite series for representing functions.

\subsection{Key Formulas and Theorems}
Let $\{a_n\}$ and $\{b_n\}$ be convergent sequences.
\begin{enumerate}
    \item \textbf{Limit Laws for Sequences:}
    \begin{itemize}
        \item $\lim_{n \to \infty} (a_n \pm b_n) = \lim_{n \to \infty} a_n \pm \lim_{n \to \infty} b_n$
        \item $\lim_{n \to \infty} (c \cdot a_n) = c \cdot \lim_{n \to \infty} a_n$
        \item $\lim_{n \to \infty} (a_n \cdot b_n) = (\lim_{n \to \infty} a_n) \cdot (\lim_{n \to \infty} b_n)$
        \item $\lim_{n \to \infty} \frac{a_n}{b_n} = \frac{\lim_{n \to \infty} a_n}{\lim_{n \to \infty} b_n}$ (if $\lim_{n \to \infty} b_n \neq 0$)
    \end{itemize}
    \item \textbf{Squeeze Theorem:} If $a_n \le b_n \le c_n$ for $n \ge n_0$ and $\lim_{n \to \infty} a_n = \lim_{n \to \infty} c_n = L$, then $\lim_{n \to \infty} b_n = L$.
    \item \textbf{Theorem for Absolute Values:} If $\lim_{n \to \infty} |a_n| = 0$, then $\lim_{n \to \infty} a_n = 0$.
    \item \textbf{Limit of a Geometric Sequence:} The sequence $\{r^n\}$ converges if $-1 < r \le 1$ and diverges for all other values of $r$.
    \[ \lim_{n \to \infty} r^n = \begin{cases} 0 & \text{if } -1 < r < 1 \\ 1 & \text{if } r = 1 \end{cases} \]
\end{enumerate}

\subsection{Prerequisites}
\begin{itemize}
    \item \textbf{Algebra:} Mastery of manipulating fractions, exponents (especially negative and fractional), radicals, and logarithms.
    \item \textbf{Functions:} A solid understanding of function notation and behavior, including domains.
    \item \textbf{Calculus I Limits:} Evaluating limits at infinity for functions. The "dominant term" method (dividing by the highest power of the variable) is crucial.
    \item \textbf{L'Hôpital's Rule:} Essential for evaluating indeterminate forms like $\frac{\infty}{\infty}$ or $\frac{0}{0}$ when finding limits of sequences via their continuous function counterparts.
\end{itemize}

\section{Detailed Homework Solutions}

\subsection{Problem 1}
\begin{itemize}
    \item[\textbf{(a)}] \textbf{What is a sequence?}
    
    \textbf{Solution:} A sequence is an ordered list of numbers. (Correct choice: A sequence is an ordered list of numbers.)
    
    \item[\textbf{(b)}] \textbf{What does it mean to say that $\lim_{n \to \infty} a_n = 8$?}
    
    \textbf{Solution:} It means that the terms of the sequence, $a_n$, get arbitrarily close to the value 8 as the index $n$ becomes very large. (Correct choice: The terms $a_n$ approach 8 as $n$ becomes large.)
    
    \item[\textbf{(c)}] \textbf{What does it mean to say that $\lim_{n \to \infty} a_n = \infty$?}
    
    \textbf{Solution:} It means that the terms of the sequence, $a_n$, can be made arbitrarily large by taking a sufficiently large index $n$. The sequence is divergent. (Correct choice: The terms $a_n$ become large as $n$ becomes large.)
\end{itemize}

\subsection{Problem 2}
List the first five terms of the sequence $a_n = n^3 - 3$.
\begin{itemize}
    \item $a_1 = (1)^3 - 3 = 1 - 3 = -2$
    \item $a_2 = (2)^3 - 3 = 8 - 3 = 5$
    \item $a_3 = (3)^3 - 3 = 27 - 3 = 24$
    \item $a_4 = (4)^3 - 3 = 64 - 3 = 61$
    \item $a_5 = (5)^3 - 3 = 125 - 3 = 122$
\end{itemize}
\textbf{Final Answer:} $a_1 = -2, a_2 = 5, a_3 = 24, a_4 = 61, a_5 = 122$.

\subsection{Problem 3}
List the first five terms of the sequence $\{3^n + n\}_{n=2}^{\infty}$.
\begin{itemize}
    \item $a_2 = 3^2 + 2 = 9 + 2 = 11$
    \item $a_3 = 3^3 + 3 = 27 + 3 = 30$
    \item $a_4 = 3^4 + 4 = 81 + 4 = 85$
    \item $a_5 = 3^5 + 5 = 243 + 5 = 248$
    \item $a_6 = 3^6 + 6 = 729 + 6 = 735$
\end{itemize}
\textbf{Final Answer:} $a_2 = 11, a_3 = 30, a_4 = 85, a_5 = 248, a_6 = 735$.

\subsection{Problem 4}
List the first five terms of the sequence $a_n = \frac{(-1)^n - 1}{n^2}$.
\begin{itemize}
    \item $a_1 = \frac{(-1)^1 - 1}{1^2} = \frac{-1 - 1}{1} = -2$
    \item $a_2 = \frac{(-1)^2 - 1}{2^2} = \frac{1 - 1}{4} = 0$
    \item $a_3 = \frac{(-1)^3 - 1}{3^2} = \frac{-1 - 1}{9} = -\frac{2}{9}$
    \item $a_4 = \frac{(-1)^4 - 1}{4^2} = \frac{1 - 1}{16} = 0$
    \item $a_5 = \frac{(-1)^5 - 1}{5^2} = \frac{-1 - 1}{25} = -\frac{2}{25}$
\end{itemize}
\textbf{Final Answer:} $a_1 = -2, a_2 = 0, a_3 = -2/9, a_4 = 0, a_5 = -2/25$.

\subsection{Problem 5}
List the first five terms of the sequence $a_n = \frac{(-1)^n}{3^n}$.
\begin{itemize}
    \item $a_1 = \frac{(-1)^1}{3^1} = -\frac{1}{3}$
    \item $a_2 = \frac{(-1)^2}{3^2} = \frac{1}{9}$
    \item $a_3 = \frac{(-1)^3}{3^3} = -\frac{1}{27}$
    \item $a_4 = \frac{(-1)^4}{3^4} = \frac{1}{81}$
    \item $a_5 = \frac{(-1)^5}{3^5} = -\frac{1}{243}$
\end{itemize}
\textbf{Final Answer:} $a_1 = -1/3, a_2 = 1/9, a_3 = -1/27, a_4 = 1/81, a_5 = -1/243$.

\subsection{Problem 6}
List the first five terms of the sequence $a_n = \frac{(-2)^n}{(n+2)!}$.
\begin{itemize}
    \item $a_1 = \frac{(-2)^1}{(1+2)!} = \frac{-2}{3!} = \frac{-2}{6} = -\frac{1}{3}$
    \item $a_2 = \frac{(-2)^2}{(2+2)!} = \frac{4}{4!} = \frac{4}{24} = \frac{1}{6}$
    \item $a_3 = \frac{(-2)^3}{(3+2)!} = \frac{-8}{5!} = \frac{-8}{120} = -\frac{1}{15}$
    \item $a_4 = \frac{(-2)^4}{(4+2)!} = \frac{16}{6!} = \frac{16}{720} = \frac{1}{45}$
    \item $a_5 = \frac{(-2)^5}{(5+2)!} = \frac{-32}{7!} = \frac{-32}{5040} = -\frac{2}{315}$
\end{itemize}
\textbf{Final Answer:} $a_1 = -1/3, a_2 = 1/6, a_3 = -1/15, a_4 = 1/45, a_5 = -2/315$.

\subsection{Problem 7}
List the first five terms of the sequence $a_1 = 18, a_{n+1} = \frac{a_n}{n}$.
\begin{itemize}
    \item $a_1 = 18$
    \item $a_2 = a_{1+1} = \frac{a_1}{1} = \frac{18}{1} = 18$
    \item $a_3 = a_{2+1} = \frac{a_2}{2} = \frac{18}{2} = 9$
    \item $a_4 = a_{3+1} = \frac{a_3}{3} = \frac{9}{3} = 3$
    \item $a_5 = a_{4+1} = \frac{a_4}{4} = \frac{3}{4}$
\end{itemize}
\textbf{Final Answer:} $a_1 = 18, a_2 = 18, a_3 = 9, a_4 = 3, a_5 = 3/4$.

\subsection{Problem 8}
Find a formula for the general term $a_n$ of the sequence $\{\frac{1}{4}, \frac{1}{8}, \frac{1}{12}, \frac{1}{16}, \dots\}$.
The numerators are all 1. The denominators are 4, 8, 12, 16, ... which are multiples of 4.
The denominator of the $n$-th term is $4n$.
\textbf{Final Answer:} $a_n = \frac{1}{4n}$.

\subsection{Problem 9}
Find a formula for the general term $a_n$ of the sequence $\{-4, \frac{8}{3}, -\frac{16}{9}, \frac{32}{27}, -\frac{64}{81}, \dots\}$.
This is a geometric sequence. Let's find the common ratio $r$:
$r = \frac{a_2}{a_1} = \frac{8/3}{-4} = \frac{8}{3} \cdot (-\frac{1}{4}) = -\frac{2}{3}$.
Let's check: $a_3 = a_2 \cdot r = \frac{8}{3} \cdot (-\frac{2}{3}) = -\frac{16}{9}$. This is correct.
The formula for a geometric sequence is $a_n = a_1 \cdot r^{n-1}$.
\textbf{Final Answer:} $a_n = -4 \left(-\frac{2}{3}\right)^{n-1}$.

\subsection{Problem 10}
Find a formula for the general term $a_n$ of the sequence $\{1, 4, 7, 10, 13, \dots\}$.
This is an arithmetic sequence. The common difference is $d = 4-1 = 3$.
The formula for an arithmetic sequence is $a_n = a_1 + (n-1)d$.
$a_n = 1 + (n-1)3 = 1 + 3n - 3 = 3n - 2$.
\textbf{Final Answer:} $a_n = 3n - 2$.

\subsection{Problem 11}
Find a formula for the general term $a_n$ of the sequence $\{\frac{1}{5}, -\frac{4}{6}, \frac{9}{7}, -\frac{16}{8}, \frac{25}{9}, \dots\}$.
The signs are alternating, starting with positive, so we use $(-1)^{n+1}$ or $(-1)^{n-1}$.
The numerators are $1, 4, 9, 16, 25, \dots$, which are the squares of the term numbers, $n^2$.
The denominators are $5, 6, 7, 8, 9, \dots$. The denominator of the $n$-th term is $n+4$.
Combining these parts:
\textbf{Final Answer:} $a_n = (-1)^{n+1} \frac{n^2}{n+4}$.

\subsection{Problem 12}
Consider $a_n = \frac{9n}{1+18n}$.
First ten terms (to four decimal places):
$a_1 = \frac{9}{19} \approx 0.4737$, $a_2 = \frac{18}{37} \approx 0.4865$, $a_3 = \frac{27}{55} \approx 0.4909$, $a_4 = \frac{36}{73} \approx 0.4932$, $a_5 = \frac{45}{91} \approx 0.4945$, $a_6 = \frac{54}{109} \approx 0.4954$, $a_7 = \frac{63}{127} \approx 0.4961$, $a_8 = \frac{72}{145} \approx 0.4966$, $a_9 = \frac{81}{163} \approx 0.4969$, $a_{10} = \frac{90}{181} \approx 0.4972$.

The graph that shows points starting below 0.5 and increasing towards it is the correct one.

The sequence appears to have a limit. To calculate it, we divide the numerator and denominator by the highest power of $n$, which is $n$.
\[ \lim_{n \to \infty} a_n = \lim_{n \to \infty} \frac{9n}{1+18n} = \lim_{n \to \infty} \frac{\frac{9n}{n}}{\frac{1}{n}+\frac{18n}{n}} = \lim_{n \to \infty} \frac{9}{\frac{1}{n}+18} \]
As $n \to \infty$, $\frac{1}{n} \to 0$.
\[ = \frac{9}{0+18} = \frac{9}{18} = \frac{1}{2} = 0.5 \]
\textbf{Final Answer:} Yes, the sequence has a limit. The limit is $0.5$.

\subsection{Problem 13}
$a_n = \frac{9}{n+2}$. Determine convergence.
\[ \lim_{n \to \infty} a_n = \lim_{n \to \infty} \frac{9}{n+2} \]
As $n \to \infty$, the denominator $n+2 \to \infty$. A constant divided by an infinitely large number approaches 0.
\textbf{Final Answer:} The sequence converges to 0.

\subsection{Problem 14}
$a_n = \frac{6n^2 - 7n}{2n^2 + 1}$. Determine convergence.
Divide by the highest power of $n$ in the denominator, which is $n^2$.
\[ \lim_{n \to \infty} \frac{6n^2 - 7n}{2n^2 + 1} = \lim_{n \to \infty} \frac{\frac{6n^2}{n^2} - \frac{7n}{n^2}}{\frac{2n^2}{n^2} + \frac{1}{n^2}} = \lim_{n \to \infty} \frac{6 - \frac{7}{n}}{2 + \frac{1}{n^2}} \]
As $n \to \infty$, $\frac{7}{n} \to 0$ and $\frac{1}{n^2} \to 0$.
\[ = \frac{6 - 0}{2 + 0} = \frac{6}{2} = 3 \]
\textbf{Final Answer:} The sequence converges to 3.

\subsection{Problem 15}
$a_n = \frac{n^4}{n^3 - 6n}$. Determine convergence.
Divide by the highest power of $n$ in the denominator, which is $n^3$.
\[ \lim_{n \to \infty} \frac{n^4}{n^3 - 6n} = \lim_{n \to \infty} \frac{\frac{n^4}{n^3}}{\frac{n^3}{n^3} - \frac{6n}{n^3}} = \lim_{n \to \infty} \frac{n}{1 - \frac{6}{n^2}} \]
As $n \to \infty$, the numerator $n \to \infty$ and the denominator approaches $1-0=1$.
The limit is $\infty$.
\textbf{Final Answer:} The sequence diverges (DIVERGES).

\subsection{Problem 16}
$a_n = \frac{6\sqrt{n}}{\sqrt{n}+2}$. Determine convergence.
Divide by the highest power of $n$ in the denominator, which is $\sqrt{n}$.
\[ \lim_{n \to \infty} \frac{6\sqrt{n}}{\sqrt{n}+2} = \lim_{n \to \infty} \frac{\frac{6\sqrt{n}}{\sqrt{n}}}{\frac{\sqrt{n}}{\sqrt{n}}+\frac{2}{\sqrt{n}}} = \lim_{n \to \infty} \frac{6}{1+\frac{2}{\sqrt{n}}} \]
As $n \to \infty$, $\frac{2}{\sqrt{n}} \to 0$.
\[ = \frac{6}{1+0} = 6 \]
\textbf{Final Answer:} The sequence converges to 6.

\subsection{Problem 17}
\begin{itemize}
    \item[\textbf{(a)}] $a_n = -\frac{3}{\sqrt{n}}$.
    \[ \lim_{n \to \infty} -\frac{3}{\sqrt{n}} = -3 \cdot \lim_{n \to \infty} \frac{1}{\sqrt{n}} = -3 \cdot 0 = 0 \]
    \textbf{Final Answer (a):} Converges to 0.
    
    \item[\textbf{(b)}] $b_n = e^{-3/\sqrt{n}}$.
    Let's first find the limit of the exponent: $\lim_{n \to \infty} -\frac{3}{\sqrt{n}} = 0$.
    Since the exponential function $f(x)=e^x$ is continuous everywhere, we can bring the limit inside:
    \[ \lim_{n \to \infty} e^{-3/\sqrt{n}} = e^{\lim_{n \to \infty} -3/\sqrt{n}} = e^0 = 1 \]
    \textbf{Final Answer (b):} Converges to 1.
\end{itemize}

\subsection{Problem 18}
$a_n = \frac{3^n}{1+4^n}$. Determine convergence.
Divide by the fastest-growing term in the denominator, which is $4^n$.
\[ \lim_{n \to \infty} \frac{3^n}{1+4^n} = \lim_{n \to \infty} \frac{\frac{3^n}{4^n}}{\frac{1}{4^n}+\frac{4^n}{4^n}} = \lim_{n \to \infty} \frac{(\frac{3}{4})^n}{\frac{1}{4^n}+1} \]
This is a geometric sequence with $r=3/4$. Since $|r|<1$, $\lim_{n \to \infty} (\frac{3}{4})^n = 0$. Also, $\lim_{n \to \infty} \frac{1}{4^n} = 0$.
\[ = \frac{0}{0+1} = 0 \]
\textbf{Final Answer:} The sequence converges to 0.

\subsection{Problem 19}
$a_n = \frac{n^2}{\sqrt{n^3 + 8n}}$. Determine convergence.
The dominant power in the numerator is $n^2$. The dominant power in the denominator is $\sqrt{n^3} = n^{3/2}$. Since the power of the numerator is greater than the power of the denominator ($2 > 3/2$), the sequence diverges to $\infty$.
To show this formally, divide numerator and denominator by $n^{3/2}$:
\[ \lim_{n \to \infty} \frac{n^2}{\sqrt{n^3+8n}} = \lim_{n \to \infty} \frac{n^2/n^{3/2}}{\sqrt{n^3+8n}/n^{3/2}} = \lim_{n \to \infty} \frac{n^{1/2}}{\sqrt{(n^3+8n)/n^3}} = \lim_{n \to \infty} \frac{\sqrt{n}}{\sqrt{1+\frac{8}{n^2}}} = \frac{\infty}{\sqrt{1+0}} = \infty \]
\textbf{Final Answer:} The sequence diverges (DIVERGES).

\subsection{Problem 20}
$a_n = \frac{\ln(n)}{\ln(3n)}$. Determine convergence.
Using log properties: $\ln(3n) = \ln(3) + \ln(n)$.
\[ \lim_{n \to \infty} \frac{\ln(n)}{\ln(3)+\ln(n)} \]
Divide by $\ln(n)$:
\[ \lim_{n \to \infty} \frac{\frac{\ln(n)}{\ln(n)}}{\frac{\ln(3)}{\ln(n)}+\frac{\ln(n)}{\ln(n)}} = \lim_{n \to \infty} \frac{1}{\frac{\ln(3)}{\ln(n)}+1} \]
As $n \to \infty$, $\ln(n) \to \infty$, so $\frac{\ln(3)}{\ln(n)} \to 0$.
\[ = \frac{1}{0+1} = 1 \]
\textbf{Final Answer:} The sequence converges to 1.

\subsection{Problem 21}
$a_n = \frac{\cos^2(n)}{3^n}$. Determine convergence.
We use the Squeeze Theorem. The cosine function is bounded: $-1 \le \cos(n) \le 1$. Squaring it gives $0 \le \cos^2(n) \le 1$.
Now, divide the inequality by $3^n$ (which is always positive):
\[ 0 \le \frac{\cos^2(n)}{3^n} \le \frac{1}{3^n} \]
We take the limit of the bounding sequences:
\[ \lim_{n \to \infty} 0 = 0 \quad \text{and} \quad \lim_{n \to \infty} \frac{1}{3^n} = 0 \]
Since both bounding sequences converge to 0, by the Squeeze Theorem, the sequence in the middle must also converge to 0.
\textbf{Final Answer:} The sequence converges to 0.

\subsection{Problem 22}
$a_n = \frac{(\ln(n))^2}{n}$. Determine convergence.
This is an indeterminate form $\frac{\infty}{\infty}$. We consider the related function $f(x) = \frac{(\ln x)^2}{x}$ and use L'Hôpital's Rule.
\[ \lim_{x \to \infty} \frac{(\ln x)^2}{x} \overset{L'H}{=} \lim_{x \to \infty} \frac{2(\ln x) \cdot \frac{1}{x}}{1} = \lim_{x \to \infty} \frac{2 \ln x}{x} \]
This is still $\frac{\infty}{\infty}$, so we apply L'Hôpital's Rule again.
\[ \overset{L'H}{=} \lim_{x \to \infty} \frac{2 \cdot \frac{1}{x}}{1} = \lim_{x \to \infty} \frac{2}{x} = 0 \]
Since the limit of the function is 0, the limit of the sequence is also 0.
\textbf{Final Answer:} The sequence converges to 0.

\section{In-Depth Analysis of Problems and Techniques}

\subsection{Problem Types and General Approach}
\begin{itemize}
    \item \textbf{Direct Calculation (Problems 2-7):} These problems test the basic understanding of sequence notation. For explicit formulas like $a_n = f(n)$, simply substitute the integer values for $n$. For recursive formulas where $a_{n+1}$ depends on $a_n$, use the previously calculated term to find the next.
    
    \item \textbf{Pattern Recognition (Problems 8-11):} These problems require you to act as a detective and find the rule governing the sequence. The general approach is to analyze numerators and denominators separately, look for alternating signs (which implies a factor of $(-1)^n$ or similar), and identify common patterns like arithmetic (common difference) or geometric (common ratio) progressions.

    \item \textbf{Limit Evaluation via Algebraic Manipulation (Problems 12, 14-16, 18-20):} This is the most common type. The key strategy for rational functions of $n$ or similar expressions (like those with radicals) is to identify the dominant term (highest power of $n$) in the denominator and divide both the numerator and denominator by it. This transforms the expression into a form where many terms go to zero.

    \item \textbf{Limit Evaluation using Calculus Theorems (Problems 17, 21, 22):} When algebraic manipulation isn't enough, we turn to more powerful tools.
        \begin{itemize}
            \item \textbf{Continuity (Problem 17b):} If you have a continuous function of a convergent sequence, you can pass the limit inside the function, i.e., $\lim f(a_n) = f(\lim a_n)$.
            \item \textbf{Squeeze Theorem (Problem 21):} This is the go-to method for sequences involving bounded oscillating terms like $\sin(n)$ or $\cos(n)$.
            \item \textbf{L'Hôpital's Rule (Problems 20, 22):} For indeterminate forms like $\frac{\infty}{\infty}$ involving functions like logarithms or exponentials, creating a continuous version $f(x)$ and applying L'Hôpital's rule is a standard and effective technique.
        \end{itemize}
\end{itemize}

\subsection{Key Algebraic and Calculus Manipulations}
\begin{itemize}
    \item \textbf{Dividing by the Highest Power of n:} In Problem 14, for $(6n^2 - 7n) / (2n^2 + 1)$, dividing all four terms by $n^2$ was the crucial step. It turned the indeterminate form into $(6 - 7/n) / (2 + 1/n^2)$, where the fractional parts clearly go to zero, revealing the limit of $6/2 = 3$.

    \item \textbf{Handling Radicals with Highest Powers:} In Problem 16, for $6\sqrt{n} / (\sqrt{n} + 2)$, the dominant term is $\sqrt{n}$. Dividing by it simplifies the expression to $6 / (1 + 2/\sqrt{n})$, making the limit obvious. It's important to be consistent; dividing by $n^2$ outside a square root is equivalent to dividing by $n^4$ inside it.

    \item \textbf{Exploiting Logarithm Properties:} In Problem 20, recognizing that $\ln(3n) = \ln(3) + \ln(n)$ provided an elegant alternative to L'Hôpital's Rule. This algebraic simplification immediately led to the form $1 / (\frac{\ln(3)}{\ln(n)} + 1)$, whose limit is easy to evaluate.

    \item \textbf{Setting up the Squeeze Theorem:} In Problem 21, the key manipulation was not of the sequence itself, but the creation of its bounds. Starting with the fundamental inequality $0 \le \cos^2(n) \le 1$ and then dividing all parts by $3^n$ constructed the "squeeze" necessary to prove the limit.

    \item \textbf{Proper Application of L'Hôpital's Rule:} In Problem 22, the technique required forming a real-valued function $f(x) = (\ln x)^2 / x$. The calculus manipulation was then to correctly differentiate the numerator (using the chain rule) and the denominator separately, not using the quotient rule. This process was repeated because the first application still resulted in an indeterminate form.
\end{itemize}

\section{"Cheatsheet" and Tips for Success}

\subsection{Summary of Formulas and Tests}
\begin{itemize}
    \item \textbf{Rational Function Test:} For $a_n = \frac{P(n)}{Q(n)}$, compare degrees. If $\deg(P) < \deg(Q)$, the limit is 0. If $\deg(P) = \deg(Q)$, the limit is the ratio of leading coefficients. If $\deg(P) > \deg(Q)$, the limit is $\pm\infty$.
    \item \textbf{Geometric Sequence:} $\lim_{n \to \infty} r^n = 0$ if $|r|<1$.
    \item \textbf{Squeeze Theorem:} Ideal for trig functions like $\sin(n), \cos(n)$. Remember $-1 \le \sin(n) \le 1$ and $-1 \le \cos(n) \le 1$.
    \item \textbf{Hierarchy of Growth Rates:} Exponentials grow faster than polynomials, which grow faster than logarithms.
    \[ \lim_{n \to \infty} \frac{\ln n}{n^p} = 0 \quad (p>0) \qquad \lim_{n \to \infty} \frac{n^p}{b^n} = 0 \quad (b>1) \]
\end{itemize}

\subsection{Tricks and Shortcuts}
\begin{itemize}
    \item For rational functions, you can often just "eyeball" the limit by looking at the ratio of the terms with the highest power (see Rational Function Test above).
    \item If you see $(-1)^n$ or $(-1)^{n+1}$, immediately check if the rest of the expression goes to 0. If it does (like in $a_n = \frac{(-1)^n}{n}$), the limit is 0. If it doesn't (like in $a_n = (-1)^n \frac{n}{n+1}$), the sequence diverges by oscillation.
    \item If you have different exponential bases, divide by the largest base. For $a_n = \frac{3^n - 2^n}{4^n+1}$, divide everything by $4^n$.
\end{itemize}

\subsection{Common Pitfalls and Mistakes}
\begin{itemize}
    \item \textbf{Misapplying L'Hôpital's Rule:} Do not use the quotient rule when applying L'Hôpital's. Differentiate the top and bottom separately. Also, ensure you have an indeterminate form first!
    \item \textbf{Algebraic Errors:} Be very careful when dividing by $n^k$ inside radicals. Remember that $\frac{\sqrt{f(n)}}{n^k} = \sqrt{\frac{f(n)}{n^{2k}}}$.
    \item \textbf{Assuming Convergence:} Don't assume a sequence converges. For example, $a_n = \cos(n\pi) = (-1)^n$ diverges, even though it's bounded.
    \item \textbf{Confusing Sequences and Series:} This chapter is about the behavior of the terms themselves. Later, in the study of series, we will be concerned with the sum of the terms, which is a very different concept.
\end{itemize}

\section{Conceptual Synthesis and The "Big Picture"}

\subsection{Thematic Connections}
The core theme of this topic is **analyzing the limiting behavior of infinite processes**. This is one of the foundational pillars of calculus. We've seen this theme before in several contexts:
\begin{itemize}
    \item \textbf{Limits of Functions:} The limit of a sequence is a specific case of the limit of a function, where the domain is restricted to integers. The core idea of "approaching" a value is the same.
    \item \textbf{Improper Integrals:} When we evaluate an integral like $\int_1^\infty f(x) dx$, we are asking about the behavior of the accumulated area as the boundary goes to infinity. This is conceptually similar to asking about the behavior of a sequence's terms as its index goes to infinity.
    \item \textbf{Definition of the Derivative:} The derivative, $f'(x) = \lim_{h \to 0} \frac{f(x+h)-f(x)}{h}$, is defined by a limit. We can think of this as a sequence of difference quotients for $h = 1, \frac{1}{2}, \frac{1}{3}, \dots$.
\end{itemize}

\subsection{Forward and Backward Links}
\begin{itemize}
    \item \textbf{Backward Link (Foundation):} This entire topic rests on the concept of limits at infinity, which you studied in Calculus I. The techniques for evaluating limits of functions (dividing by the highest power, L'Hôpital's rule) are the primary tools we use for evaluating limits of sequences.

    \item \textbf{Forward Link (Gateway):} Understanding sequences is an absolute prerequisite for the study of \textbf{Infinite Series}. An infinite series is the sum of the terms of an infinite sequence. The single most important question about a series is whether it converges to a finite sum. To answer this, we will not only analyze the sequence of terms ($a_n$) but also a new, more important sequence: the sequence of partial sums ($S_n = a_1 + a_2 + \dots + a_n$). The concepts of convergence, divergence, and limits that we are mastering here are the fundamental building blocks for all of Chapters 11 and beyond, including Power Series and Taylor Series, which are used to approximate virtually any function in science and engineering.
\end{itemize}

\section{Real-World Application and Modeling}

\subsection{Concrete Scenarios in Finance and Economics}
\begin{itemize}
    \item \textbf{1. Future Value of an Annuity:} An annuity is a financial product that involves a series of regular payments. If you deposit a fixed amount $d$ every month into an account with a monthly interest rate $i$, the total value after $n$ months is a sequence. This is the cornerstone of retirement planning (like a 401k) and loan amortization calculations. Financial analysts use these sequence and series formulas to determine how much one needs to save to reach a retirement goal.

    \item \textbf{2. Option Pricing with Binomial Trees:} In quantitative finance, a common way to price a financial option (like the right to buy a stock at a future date for a set price) is the binomial tree model. The model assumes that in each small time step, the stock price can only go up or down by a certain amount. By creating a tree of all possible price paths over many time steps (a sequence of possible states), one can calculate the expected payoff of the option at expiration and work backward to find its fair value today. As the number of time steps $n$ goes to infinity, this discrete model's limit becomes the famous continuous-time Black-Scholes model.

    \item \textbf{3. Economic Multiplier Effect:} In macroeconomics, if the government injects money into the economy (e.g., through spending), that money is spent by its recipients. The people who receive that money then save a portion and spend the rest. This creates a chain reaction. If people spend, for example, 80\% of any new income, an initial \$1 billion injection leads to a sequence of spending: \$1B, \$0.8B, \$0.64B, etc. This is a geometric sequence. The limit of the sum of this sequence (the series) tells economists the total impact, or "multiplier effect," of the initial stimulus.
\end{itemize}

\subsection{Model Problem Setup: Retirement Savings}
Let's model the growth of a retirement account.

\begin{itemize}
    \item \textbf{Scenario:} A person starts with \$10,000 in a retirement account and contributes \$500 at the beginning of each month. The account earns an annual interest rate of 7.2\%, compounded monthly.
    \item \textbf{Variables:}
        \begin{itemize}
            \item Let $P_n$ be the balance in the account at the end of month $n$.
            \item Initial Principal: $P_0 = \$10,000$.
            \item Monthly Contribution: $d = \$500$.
            \item Annual Interest Rate: $r = 0.072$.
            \item Monthly Interest Rate: $i = r/12 = 0.006$.
        \end{itemize}
    \item \textbf{Formulation:} The balance at the end of a month is the balance from the start of the month (previous end balance + new contribution) multiplied by the interest factor.
    \[ P_{n} = (P_{n-1} + d) \cdot (1+i) \]
    This defines the sequence $\{P_n\}$ recursively.
    \item \textbf{The Equation to Solve:} To find the account balance after 30 years (360 months), one would need to calculate the 360th term of this sequence, $P_{360}$. This recursive relationship can be solved to find an explicit formula:
    \[ P_n = P_0(1+i)^n + d \left[ \frac{(1+i)^n - 1}{i} \right] (1+i) \]
    A financial analyst would plug in $n=360$ to find the final amount. The sequence $\{P_n\}$ itself models the account's growth trajectory over time.
\end{itemize}

\section{Common Variations and Untested Concepts}
The provided homework focuses heavily on evaluating limits. A key concept not explicitly tested is proving sequences are \textbf{monotonic} and \textbf{bounded} to guarantee convergence via the Monotonic Sequence Theorem.

\subsubsection{Concept: The Monotonic Sequence Theorem}
The theorem states: \textbf{Every bounded, monotonic sequence is convergent.}
This is powerful because it allows us to prove that a limit exists without having to calculate what it is. This is especially useful for recursively defined sequences.

\subsubsection{Worked Example}
\textbf{Problem:} Consider the sequence defined by $a_1 = 2$ and $a_{n+1} = \frac{1}{2}\left(a_n + 6\right)$. Show that $\{a_n\}$ is monotonic and bounded, and therefore converges. Then find its limit.

\textbf{Solution:}
\begin{enumerate}
    \item \textbf{Step 1: Check first few terms.}
    $a_1 = 2$
    $a_2 = \frac{1}{2}(2+6) = 4$
    $a_3 = \frac{1}{2}(4+6) = 5$
    The sequence appears to be increasing (monotonic) and approaching some value.

    \item \textbf{Step 2: Prove it is bounded.}
    We claim the sequence is bounded above by 6.
    \begin{itemize}
        \item Base Case: $a_1 = 2 < 6$. True.
        \item Inductive Step: Assume $a_k < 6$ for some integer $k$. We need to show $a_{k+1} < 6$.
        \[ a_{k+1} = \frac{1}{2}(a_k + 6) < \frac{1}{2}(6 + 6) = 6 \]
        So $a_{k+1} < 6$. By induction, $a_n < 6$ for all $n$. It is also bounded below by 2 since it's increasing from 2. Thus, the sequence is bounded.
    \end{itemize}
    
    \item \textbf{Step 3: Prove it is monotonic (increasing).}
    We want to show $a_{n+1} > a_n$ for all $n$.
    \[ \frac{1}{2}(a_n + 6) > a_n \iff a_n + 6 > 2a_n \iff 6 > a_n \]
    From Step 2, we know $a_n < 6$ is true for all $n$. Therefore, the sequence is increasing.

    \item \textbf{Step 4: Conclude and find the limit.}
    Since $\{a_n\}$ is both bounded and monotonic, by the Monotonic Sequence Theorem, it must converge. Let's say $\lim_{n \to \infty} a_n = L$.
    Since $a_n \to L$, it must be that $a_{n+1}$ also approaches $L$. We can take the limit of both sides of the recursive formula:
    \begin{align*}
        \lim_{n \to \infty} a_{n+1} &= \lim_{n \to \infty} \frac{1}{2}(a_n + 6) \\
        L &= \frac{1}{2}(L+6) \\
        2L &= L+6 \\
        L &= 6
    \end{align*}
    The sequence converges to 6.
\end{enumerate}

\section{Advanced Diagnostic Testing: "Find the Flaw"}

This section presents five problems with flawed solutions. The goal is to identify the error, explain it, and provide the correct solution.

\subsection{Problem 1}
Determine if the sequence $a_n = \frac{5n^3 - 2n}{10n^3 + n^2 + 1}$ converges or diverges.

\textbf{Flawed Solution:}
\begin{align*}
    \lim_{n \to \infty} \frac{5n^3 - 2n}{10n^3 + n^2 + 1} &= \lim_{n \to \infty} \frac{\frac{5n^3}{n^3} - \frac{2n}{n^3}}{\frac{10n^3}{n^3} + \frac{n^2}{n^3} + 1} \\
    &= \lim_{n \to \infty} \frac{5 - \frac{2}{n^2}}{10 + \frac{1}{n} + 1} \\
    &= \frac{5 - 0}{10 + 0 + 1} = \frac{5}{11}
\end{align*}

\textbf{Corrected Analysis:}
\begin{itemize}
    \item \textbf{The Flaw Is:} In the first step, when dividing the denominator by $n^3$, the term '1' was not divided by $n^3$.
    \item \textbf{Explanation of Error:} Every term in both the numerator and denominator must be divided by the same value for the expression to remain equivalent.
    \item \textbf{Correction:}
    \begin{align*}
        \lim_{n \to \infty} \frac{5n^3 - 2n}{10n^3 + n^2 + 1} &= \lim_{n \to \infty} \frac{\frac{5n^3}{n^3} - \frac{2n}{n^3}}{\frac{10n^3}{n^3} + \frac{n^2}{n^3} + \frac{1}{n^3}} \\
        &= \lim_{n \to \infty} \frac{5 - \frac{2}{n^2}}{10 + \frac{1}{n} + \frac{1}{n^3}} \\
        &= \frac{5 - 0}{10 + 0 + 0} = \frac{5}{10} = \frac{1}{2}
    \end{align*}
\end{itemize}


\subsection{Problem 2}
Find the limit of the sequence $a_n = \frac{n!}{ (n+1)! }$.

\textbf{Flawed Solution:}
This looks like an $\frac{\infty}{\infty}$ form, so we use L'Hôpital's Rule. Let $f(x) = \frac{x!}{(x+1)!}$. The derivative of $x!$ is complex, but this seems like it should work. The limit should approach 1 because the factorials are almost the same for large $n$. The limit is 1.

\textbf{Corrected Analysis:}
\begin{itemize}
    \item \textbf{The Flaw Is:} The solution attempts to use L'Hôpital's Rule and then incorrectly guesses the limit.
    \item \textbf{Explanation of Error:} L'Hôpital's Rule cannot be applied because the factorial function is only defined for integers and is not a differentiable function of a real variable. The conclusion that the limit is 1 is also incorrect.
    \item \textbf{Correction:} Simplify the expression algebraically using the definition of factorial, $(n+1)! = (n+1) \cdot n!$.
    \[ a_n = \frac{n!}{(n+1) \cdot n!} = \frac{1}{n+1} \]
    Now take the limit:
    \[ \lim_{n \to \infty} a_n = \lim_{n \to \infty} \frac{1}{n+1} = 0 \]
\end{itemize}

\subsection{Problem 3}
Find the limit of $a_n = \frac{3^n - 2^n}{3^{n+1}}$.

\textbf{Flawed Solution:}
Divide by the highest power, which is $3^{n+1}$.
\begin{align*}
    \lim_{n \to \infty} \frac{3^n - 2^n}{3^{n+1}} &= \lim_{n \to \infty} \frac{\frac{3^n}{3^{n+1}} - \frac{2^n}{3^{n+1}}}{1} \\
    &= \lim_{n \to \infty} \left( \frac{1}{3} - \frac{1}{3} \left(\frac{2}{3}\right)^n \right) \\
    &= \frac{1}{3} - \frac{1}{3} \cdot 1 = 0
\end{align*}

\textbf{Corrected Analysis:}
\begin{itemize}
    \item \textbf{The Flaw Is:} The limit of $(\frac{2}{3})^n$ as $n \to \infty$ was incorrectly evaluated as 1.
    \item \textbf{Explanation of Error:} This is a geometric sequence with ratio $r = 2/3$. Since $|r| < 1$, the limit as $n \to \infty$ is 0, not 1.
    \item \textbf{Correction:}
    \begin{align*}
        \lim_{n \to \infty} \left( \frac{1}{3} - \frac{1}{3} \left(\frac{2}{3}\right)^n \right) &= \frac{1}{3} - \frac{1}{3} \cdot \left(\lim_{n \to \infty} \left(\frac{2}{3}\right)^n \right) \\
        &= \frac{1}{3} - \frac{1}{3} \cdot 0 = \frac{1}{3}
    \end{align*}
\end{itemize}

\subsection{Problem 4}
Find the limit of $a_n = n \sin(\frac{1}{n})$.

\textbf{Flawed Solution:}
This sequence can be bounded using the Squeeze Theorem. We know that $-1 \le \sin(\frac{1}{n}) \le 1$.
Multiplying by $n$ gives $-n \le n\sin(\frac{1}{n}) \le n$.
Since $\lim_{n \to \infty} -n = -\infty$ and $\lim_{n \to \infty} n = \infty$, the sequence is unbounded and must diverge.

\textbf{Corrected Analysis:}
\begin{itemize}
    \item \textbf{The Flaw Is:} The application of the Squeeze Theorem is incorrect.
    \item \textbf{Explanation of Error:} The Squeeze Theorem is only conclusive if the upper and lower bounds converge to the *same finite limit*. Since the bounds go to $-\infty$ and $+\infty$, the theorem provides no information.
    \item \textbf{Correction:} This is an indeterminate form $ \infty \cdot 0 $. We should rewrite it as a fraction and use L'Hôpital's Rule. Let $x = 1/n$. As $n \to \infty$, $x \to 0^+$.
    \[ \lim_{n \to \infty} n \sin\left(\frac{1}{n}\right) = \lim_{x \to 0^+} \frac{\sin(x)}{x} \]
    This is a standard limit known to be 1. Alternatively, using L'Hôpital's Rule:
    \[ \lim_{x \to 0^+} \frac{\sin(x)}{x} \overset{L'H}{=} \lim_{x \to 0^+} \frac{\cos(x)}{1} = \frac{\cos(0)}{1} = 1 \]
\end{itemize}

\subsection{Problem 5}
Find the limit of the sequence $a_n = \left(1 - \frac{1}{n}\right)^n$.

\textbf{Flawed Solution:}
We can evaluate the limit of the base first.
\[ \lim_{n \to \infty} \left(1 - \frac{1}{n}\right) = 1 - 0 = 1 \]
Since the base goes to 1, the limit of the whole expression is just $1^\infty$, which is 1.
\[ \lim_{n \to \infty} a_n = 1 \]

\textbf{Corrected Analysis:}
\begin{itemize}
    \item \textbf{The Flaw Is:} The solution incorrectly assumes that $1^\infty$ is a determinate form equal to 1.
    \item \textbf{Explanation of Error:} The form $1^\infty$ is an indeterminate form. The base is approaching 1 while the exponent is approaching infinity, and their interaction determines the true limit.
    \item \textbf{Correction:} This limit is related to the definition of $e$. Let $L = \lim_{n \to \infty} \left(1 - \frac{1}{n}\right)^n$. Take the natural log of both sides.
    \begin{align*}
      \ln(L) &= \lim_{n \to \infty} \ln\left( \left(1 - \frac{1}{n}\right)^n \right) \\
      &= \lim_{n \to \infty} n \ln\left(1 - \frac{1}{n}\right) \quad (\text{form } \infty \cdot 0) \\
      &= \lim_{n \to \infty} \frac{\ln(1 - 1/n)}{1/n} \quad (\text{form } 0/0)
    \end{align*}
    Apply L'Hôpital's Rule with respect to $n$:
    \begin{align*}
      \ln(L) &\overset{L'H}{=} \lim_{n \to \infty} \frac{\frac{1}{1-1/n} \cdot (1/n^2)}{-1/n^2} \\
      &= \lim_{n \to \infty} \frac{-1}{1-1/n} = \frac{-1}{1-0} = -1
    \end{align*}
    Since $\ln(L) = -1$, we have $L = e^{-1} = \frac{1}{e}$.
\end{itemize}

\end{document}