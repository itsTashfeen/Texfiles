\documentclass{article}
\usepackage{amsmath}
\usepackage{amssymb}
\usepackage[margin=1in]{geometry}

\title{Homework 11.3: The Integral Test and Estimates of Sums}
\author{Tashfeen Omran}
\date{October 2025}

\begin{document}

\maketitle

\section{Comprehensive Introduction, Context, and Prerequisites}

\subsection{Core Concepts}
The Integral Test is a powerful method for determining whether an infinite series converges or diverges. It works by comparing the sum of the discrete terms of a series to the value of a continuous improper integral.

\textbf{The Main Idea:} Consider a series $\sum_{n=N}^{\infty} a_n$ where the terms $a_n$ are positive. We can think of these terms as the values of a function, $f(n) = a_n$. If we extend this function to all real numbers $x \ge N$, we can investigate the area under the curve of $f(x)$ using the improper integral $\int_N^\infty f(x) dx$. The Integral Test establishes a direct link: the infinite series and the improper integral either both converge (have a finite value) or both diverge (go to infinity).

\textbf{Conditions for the Integral Test:}
For the test to be applicable, the function $f(x)$ corresponding to the series terms $a_n$ must satisfy three critical conditions for all $x \ge N$ (where $N$ is some integer, often 1):
\begin{enumerate}
    \item \textbf{Continuous:} The function $f(x)$ must be continuous on the interval $[N, \infty)$.
    \item \textbf{Positive:} The function $f(x)$ must be positive, meaning $f(x) > 0$, on the interval $[N, \infty)$.
    \item \textbf{Decreasing:} The function $f(x)$ must be decreasing on the interval $[N, \infty)$. This is often verified by showing that its derivative, $f'(x)$, is negative.
\end{enumerate}

\textbf{Conclusion of the Integral Test:}
If the three conditions are met, then:
\begin{itemize}
    \item If the integral $\int_N^\infty f(x) dx$ is \textbf{convergent} (evaluates to a finite number), then the series $\sum_{n=N}^{\infty} a_n$ is also \textbf{convergent}.
    \item If the integral $\int_N^\infty f(x) dx$ is \textbf{divergent} (evaluates to $\infty$), then the series $\sum_{n=N}^{\infty} a_n$ is also \textbf{divergent}.
\end{itemize}
\textbf{Important Note:} The value of the integral is NOT equal to the sum of the series. The test only determines the behavior (convergence or divergence).

\subsection{Intuition and Derivation}
The intuition behind the Integral Test is best understood visually. Imagine the graph of a continuous, positive, decreasing function $f(x)$. The value of the integral $\int_1^\infty f(x) dx$ represents the total area under the curve from $x=1$ to infinity.

Now, consider the series $\sum_{n=1}^{\infty} a_n$, where $a_n = f(n)$. We can represent the terms of this series as the areas of rectangles, each with a width of 1.

\begin{itemize}
    \item \textbf{Lower Sum:} We can draw rectangles with height $a_{n+1} = f(n+1)$ on the interval $[n, n+1]$. The total area of these rectangles, $\sum_{n=2}^{\infty} a_n$, is clearly less than the area under the curve, $\int_1^\infty f(x) dx$. If the area under the curve is finite (integral converges), the smaller sum of rectangle areas must also be finite (series converges).
    \item \textbf{Upper Sum:} We can draw rectangles with height $a_n = f(n)$ on the interval $[n, n+1]$. The total area of these rectangles, $\sum_{n=1}^{\infty} a_n$, is greater than the area under the curve, $\int_1^\infty f(x) dx$. If the area under the curve is infinite (integral diverges), the even larger sum of rectangle areas must also be infinite (series diverges).
\end{itemize}
These two relationships together show that the series and the integral must have the same convergence behavior.

\subsection{Historical Context and Motivation}
The rigorous study of infinite series blossomed in the 18th and 19th centuries, driven by the need to understand complex physical phenomena and to place calculus on a solid logical foundation. Mathematicians like Leonhard Euler, the Bernoulli family, and later Augustin-Louis Cauchy were central figures. The Integral Test, often attributed to both Colin Maclaurin and Cauchy, emerged as a way to handle series for which simpler tests (like the geometric series test) did not apply.

The primary motivation was to create a reliable tool to determine the convergence of series arising from calculus, such as those used in celestial mechanics or fluid dynamics. A classic and powerful application of this test is the definitive analysis of the \textbf{p-series} ($\sum 1/n^p$), which settled the question of convergence for a vast and important class of series, including the famous divergent harmonic series ($p=1$). The test provided a beautiful bridge between the discrete world of sums and the continuous world of integrals.

\subsection{Key Formulas}
\begin{itemize}
    \item \textbf{The p-Series Rule:} A series of the form $\sum_{n=1}^{\infty} \frac{1}{n^p}$ is called a p-series.
    \begin{itemize}
        \item It \textbf{converges} if $p > 1$.
        \item It \textbf{diverges} if $p \le 1$.
    \end{itemize}
    \item \textbf{Improper Integral Definition:}
    \[ \int_a^\infty f(x) dx = \lim_{t \to \infty} \int_a^t f(x) dx \]
    \item \textbf{Remainder Estimate for the Integral Test:} If a series $\sum a_n$ converges to a sum $S$, we can approximate $S$ with a partial sum $S_n = \sum_{i=1}^{n} a_i$. The error in this approximation is the remainder, $R_n = S - S_n = a_{n+1} + a_{n+2} + \dots$. The integral test gives us bounds on this error:
    \[ \int_{n+1}^\infty f(x) dx \le R_n \le \int_n^\infty f(x) dx \]
\end{itemize}

\subsection{Prerequisites}
To master the Integral Test, a solid foundation in the following areas is essential:
\begin{itemize}
    \item \textbf{Infinite Series Basics:} Understand the concepts of convergence and divergence.
    \item \textbf{Derivatives (Calculus I):} Ability to find the derivative of a function to test if it is decreasing ($f'(x) < 0$).
    \item \textbf{Integration Techniques (Calculus II):} Proficiency in various integration methods, especially u-substitution, integration by parts, and handling special forms like those involving arctangent.
    \item \textbf{Improper Integrals (Calculus II):} Mastery of setting up and evaluating integrals with infinite bounds, including the use of limits.
    \item \textbf{Limit Evaluation (Calculus I):} Skill in evaluating limits as a variable approaches infinity, including knowledge of L'Hôpital's Rule.
\end{itemize}

\section{Detailed Homework Solutions}

\subsection*{Problem 1}
Use the Integral Test to determine whether the series $\sum_{n=1}^{\infty} n^{-8}$ is convergent or divergent.

\textbf{Solution:}
The series is $\sum_{n=1}^{\infty} \frac{1}{n^8}$. Let $f(x) = \frac{1}{x^8}$.
\begin{enumerate}
    \item For $x \ge 1$, $f(x)$ is continuous since $x^8 \ne 0$.
    \item For $x \ge 1$, $f(x)$ is positive.
    \item For $x \ge 1$, as $x$ increases, $x^8$ increases, so $1/x^8$ decreases. Thus, $f(x)$ is decreasing.
\end{enumerate}
The conditions for the Integral Test are met. Now, we evaluate the integral:
\begin{align*}
    \int_1^\infty \frac{1}{x^8} dx &= \int_1^\infty x^{-8} dx \\
    &= \lim_{t \to \infty} \int_1^t x^{-8} dx \\
    &= \lim_{t \to \infty} \left[ \frac{x^{-7}}{-7} \right]_1^t \\
    &= \lim_{t \to \infty} \left[ -\frac{1}{7x^7} \right]_1^t \\
    &= \lim_{t \to \infty} \left( -\frac{1}{7t^7} - \left(-\frac{1}{7(1)^7}\right) \right) \\
    &= 0 + \frac{1}{7} = \frac{1}{7}
\end{align*}
\textbf{Final Answer:} The integral is finite, evaluating to $\frac{1}{7}$. Since the integral converges, the series is \textbf{convergent}.

\subsection*{Problem 2}
Use the Integral Test to determine whether the series $\sum_{n=1}^{\infty} \frac{4}{3n-1}$ is convergent or divergent.

\textbf{Solution:}
Let $f(x) = \frac{4}{3x-1}$.
\begin{enumerate}
    \item For $x \ge 1$, $f(x)$ is continuous.
    \item For $x \ge 1$, $f(x)$ is positive.
    \item As $x$ increases, $3x-1$ increases, so $4/(3x-1)$ decreases. $f(x)$ is decreasing.
\end{enumerate}
The conditions are met. We evaluate the integral:
\begin{align*}
    \int_1^\infty \frac{4}{3x-1} dx &= \lim_{t \to \infty} \int_1^t \frac{4}{3x-1} dx
\end{align*}
Let $u = 3x-1$, so $du = 3dx$ and $dx = du/3$.
\begin{align*}
    \lim_{t \to \infty} \int_{x=1}^{x=t} \frac{4}{u} \frac{du}{3} &= \frac{4}{3} \lim_{t \to \infty} \left[ \ln|u| \right]_{x=1}^{x=t} \\
    &= \frac{4}{3} \lim_{t \to \infty} \left[ \ln(3x-1) \right]_1^t \\
    &= \frac{4}{3} \lim_{t \to \infty} (\ln(3t-1) - \ln(3(1)-1)) \\
    &= \frac{4}{3} (\infty - \ln(2)) = \infty
\end{align*}
\textbf{Final Answer:} The integral evaluates to $\infty$. Since the integral diverges, the series is \textbf{divergent}.

\subsection*{Problem 3}
Use the integral test to determine whether the series $\sum_{n=2}^{\infty} \frac{n^2}{n^3+1}$ is convergent or divergent.

\textbf{Solution:}
Let $f(x) = \frac{x^2}{x^3+1}$.
\begin{enumerate}
    \item For $x \ge 2$, $f(x)$ is continuous and positive.
    \item To check if it is decreasing, we find the derivative:
    \[ f'(x) = \frac{2x(x^3+1) - x^2(3x^2)}{(x^3+1)^2} = \frac{2x^4+2x-3x^4}{(x^3+1)^2} = \frac{2x-x^4}{(x^3+1)^2} = \frac{x(2-x^3)}{(x^3+1)^2} \]
    For $x \ge 2$, $(2-x^3)$ is negative, so $f'(x) < 0$. The function is decreasing.
\end{enumerate}
The conditions are met. Evaluate the integral:
\begin{align*}
    \int_2^\infty \frac{x^2}{x^3+1} dx &= \lim_{t \to \infty} \int_2^t \frac{x^2}{x^3+1} dx
\end{align*}
Let $u = x^3+1$, so $du = 3x^2dx$ and $x^2dx = du/3$.
\begin{align*}
    \lim_{t \to \infty} \int_{x=2}^{x=t} \frac{1}{u} \frac{du}{3} &= \frac{1}{3} \lim_{t \to \infty} \left[ \ln|u| \right]_{x=2}^{x=t} \\
    &= \frac{1}{3} \lim_{t \to \infty} \left[ \ln(x^3+1) \right]_2^t \\
    &= \frac{1}{3} \lim_{t \to \infty} (\ln(t^3+1) - \ln(2^3+1)) \\
    &= \frac{1}{3} (\infty - \ln(9)) = \infty
\end{align*}
\textbf{Final Answer:} The integral is infinite. Since the integral diverges, the series is \textbf{divergent}.

\subsection*{Problem 4}
Use the integral test to determine whether the series $\sum_{n=1}^{\infty} n^2 e^{-n^3}$ is convergent or divergent.

\textbf{Solution:}
Let $f(x) = x^2 e^{-x^3}$.
\begin{enumerate}
    \item For $x \ge 1$, $f(x)$ is continuous and positive.
    \item To check if it is decreasing, $f'(x) = 2x e^{-x^3} + x^2 e^{-x^3}(-3x^2) = x e^{-x^3}(2-3x^3)$. For $x \ge 1$, $(2-3x^3)$ is negative, so $f'(x) < 0$. The function is decreasing.
\end{enumerate}
The conditions are met. Evaluate the integral:
\begin{align*}
    \int_1^\infty x^2 e^{-x^3} dx &= \lim_{t \to \infty} \int_1^t x^2 e^{-x^3} dx
\end{align*}
Let $u = -x^3$, so $du = -3x^2dx$ and $x^2dx = -du/3$.
\begin{align*}
    \lim_{t \to \infty} \int_{x=1}^{x=t} e^u \left(-\frac{du}{3}\right) &= -\frac{1}{3} \lim_{t \to \infty} \left[ e^u \right]_{x=1}^{x=t} \\
    &= -\frac{1}{3} \lim_{t \to \infty} \left[ e^{-x^3} \right]_1^t \\
    &= -\frac{1}{3} \lim_{t \to \infty} (e^{-t^3} - e^{-1^3}) \\
    &= -\frac{1}{3} (0 - e^{-1}) = \frac{1}{3e}
\end{align*}
\textbf{Final Answer:} The integral is finite, evaluating to $\frac{1}{3e}$. Since the integral converges, the series is \textbf{convergent}.

\subsection*{Problem 5}
Determine whether the series $\sum_{n=1}^{\infty} \frac{1}{n\sqrt[3]{n}}$ is convergent or divergent.

\textbf{Solution:}
First, simplify the general term: $a_n = \frac{1}{n \cdot n^{1/3}} = \frac{1}{n^{1+1/3}} = \frac{1}{n^{4/3}}$.
This series is a p-series of the form $\sum \frac{1}{n^p}$ with $p = 4/3$.
According to the p-series rule, the series converges if $p > 1$.
Since $p = 4/3 \approx 1.33$, which is greater than 1, the series converges.
\textbf{Final Answer:} The series is \textbf{convergent}.

\subsection*{Problem 6}
Determine whether the series $\sum_{n=3}^{\infty} n^{-0.9999}$ is convergent or divergent.

\textbf{Solution:}
The series is $\sum_{n=3}^{\infty} \frac{1}{n^{0.9999}}$.
This is a p-series with $p = 0.9999$.
The p-series rule states that the series diverges if $p \le 1$.
Since $p=0.9999 \le 1$, the series diverges. The starting index $n=3$ does not affect the convergence behavior.
\textbf{Final Answer:} The series is \textbf{divergent}.

\subsection*{Problem 7}
Determine whether the series $1 + \frac{1}{4} + \frac{1}{9} + \frac{1}{16} + \frac{1}{25} + \dots$ is convergent or divergent.

\textbf{Solution:}
The terms of the series are $1/1^2, 1/2^2, 1/3^2, 1/4^2, 1/5^2, \dots$.
The general term is $a_n = \frac{1}{n^2}$. The series is $\sum_{n=1}^{\infty} \frac{1}{n^2}$.
This is a p-series with $p=2$. Since $p=2>1$, the series converges.
\textbf{Final Answer:} The series is \textbf{convergent}.

\subsection*{Problem 8}
Determine whether the series $\frac{1}{8} + \frac{1}{17} + \frac{1}{26} + \frac{1}{35} + \dots$ is convergent or divergent.

\textbf{Solution:}
Let's find the general term. The denominators are 8, 17, 26, 35, ...
The difference between consecutive denominators is $17-8=9$, $26-17=9$, etc. This suggests an arithmetic progression, of the form $9n+c$. For $n=1$, $9(1)+c=8 \implies c=-1$.
The terms seem to be $a_n = \frac{1}{9n-1}$, starting from $n=1$.
We can use the Limit Comparison Test with the harmonic series $b_n = \frac{1}{n}$, which is divergent.
\[ L = \lim_{n\to\infty} \frac{a_n}{b_n} = \lim_{n\to\infty} \frac{1/(9n-1)}{1/n} = \lim_{n\to\infty} \frac{n}{9n-1} = \frac{1}{9} \]
Since $L$ is a finite positive number ($0 < 1/9 < \infty$) and the comparison series $\sum 1/n$ diverges, the original series also diverges.
Alternatively, using the Integral Test from Problem 2 as a guide, $\int_1^\infty \frac{1}{9x-1} dx = [\frac{1}{9}\ln(9x-1)]_1^\infty = \infty$.
\textbf{Final Answer:} The series is \textbf{divergent}.

\subsection*{Problem 9}
Determine whether the series $1 + \frac{1}{2\sqrt{2}} + \frac{1}{3\sqrt{3}} + \frac{1}{4\sqrt{4}} + \dots$ is convergent or divergent.

\textbf{Solution:}
The general term is $a_n = \frac{1}{n\sqrt{n}} = \frac{1}{n \cdot n^{1/2}} = \frac{1}{n^{3/2}}$.
The series is $\sum_{n=1}^{\infty} \frac{1}{n^{3/2}}$.
This is a p-series with $p=3/2$. Since $p=1.5>1$, the series converges.
\textbf{Final Answer:} The series is \textbf{convergent}.

\subsection*{Problem 10}
Determine whether the series $\sum_{n=1}^{\infty} \frac{\sqrt{n}+6}{n^2}$ is convergent or divergent.

\textbf{Solution:}
We can split the series: $\sum_{n=1}^{\infty} \left( \frac{\sqrt{n}}{n^2} + \frac{6}{n^2} \right) = \sum_{n=1}^{\infty} \frac{1}{n^{3/2}} + \sum_{n=1}^{\infty} \frac{6}{n^2}$.
The first series, $\sum 1/n^{3/2}$, is a convergent p-series ($p=3/2>1$).
The second series, $\sum 6/n^2 = 6 \sum 1/n^2$, is a constant multiple of a convergent p-series ($p=2>1$), so it also converges.
The sum of two convergent series is convergent.
\textbf{Final Answer:} The series is \textbf{convergent}.

\subsection*{Problem 11}
Determine whether the series $\sum_{n=1}^{\infty} \frac{3\sqrt{n}}{1+3n^{3/2}}$ is convergent or divergent.

\textbf{Solution:}
We use the Limit Comparison Test. The dominant term in the numerator is $\sqrt{n} = n^{1/2}$ and in the denominator is $n^{3/2}$.
We compare with the series $b_n = \frac{n^{1/2}}{n^{3/2}} = \frac{1}{n}$. The series $\sum b_n = \sum 1/n$ is the harmonic series, which diverges.
\[ L = \lim_{n\to\infty} \frac{a_n}{b_n} = \lim_{n\to\infty} \frac{3\sqrt{n}/(1+3n^{3/2})}{1/n} = \lim_{n\to\infty} \frac{3n\sqrt{n}}{1+3n^{3/2}} = \lim_{n\to\infty} \frac{3n^{3/2}}{1+3n^{3/2}} \]
Divide the numerator and denominator by $n^{3/2}$:
\[ L = \lim_{n\to\infty} \frac{3}{1/n^{3/2}+3} = \frac{3}{0+3} = 1 \]
Since $L$ is a finite positive number and $\sum 1/n$ diverges, the original series also diverges.
\textbf{Final Answer:} The series is \textbf{divergent}.

\subsection*{Problem 12}
Consider the series $\sum_{n=1}^{\infty} \frac{1}{n^2+49}$. Evaluate the corresponding integral to determine convergence.

\textbf{Solution:}
Let $f(x) = \frac{1}{x^2+49}$. For $x \ge 1$, the function is continuous, positive, and decreasing. The Integral Test applies.
\begin{align*}
    \int_1^\infty \frac{1}{x^2+49} dx &= \int_1^\infty \frac{1}{x^2+7^2} dx \\
    &= \lim_{t \to \infty} \int_1^t \frac{1}{x^2+7^2} dx \\
    &= \lim_{t \to \infty} \left[ \frac{1}{7}\arctan\left(\frac{x}{7}\right) \right]_1^t \\
    &= \lim_{t \to \infty} \left( \frac{1}{7}\arctan\left(\frac{t}{7}\right) - \frac{1}{7}\arctan\left(\frac{1}{7}\right) \right) \\
    &= \frac{1}{7} \left( \frac{\pi}{2} \right) - \frac{1}{7}\arctan\left(\frac{1}{7}\right) = \frac{\pi}{14} - \frac{1}{7}\arctan\left(\frac{1}{7}\right)
\end{align*}
\textbf{Final Answer:} The integral is finite. Since the integral converges, the series is \textbf{convergent}.

\subsection*{Problem 13}
Consider the series $\sum_{n=2}^{\infty} \frac{4}{n \ln(n)}$. Evaluate the corresponding integral to determine convergence.

\textbf{Solution:}
Let $f(x) = \frac{4}{x \ln(x)}$. For $x \ge 2$, the function is continuous, positive, and decreasing. The Integral Test applies.
\begin{align*}
    \int_2^\infty \frac{4}{x \ln(x)} dx &= \lim_{t \to \infty} \int_2^t \frac{4}{x \ln(x)} dx
\end{align*}
Let $u = \ln(x)$, so $du = \frac{1}{x}dx$.
\begin{align*}
    4 \lim_{t \to \infty} \int_{x=2}^{x=t} \frac{1}{u} du &= 4 \lim_{t \to \infty} \left[ \ln|u| \right]_{x=2}^{x=t} \\
    &= 4 \lim_{t \to \infty} \left[ \ln(\ln(x)) \right]_2^t \\
    &= 4 \lim_{t \to \infty} (\ln(\ln(t)) - \ln(\ln(2))) \\
    &= 4 (\infty - \ln(\ln(2))) = \infty
\end{align*}
\textbf{Final Answer:} The integral is infinite. Since the integral diverges, the series is \textbf{divergent}.

\subsection*{Problem 14}
Determine whether the series $\sum_{n=1}^{\infty} \frac{2}{n^2+n^3}$ is convergent or divergent.

\textbf{Solution:}
We use the Direct Comparison Test. For $n \ge 1$, $n^2+n^3 > n^3$.
Therefore, $\frac{1}{n^2+n^3} < \frac{1}{n^3}$, which means $\frac{2}{n^2+n^3} < \frac{2}{n^3}$.
The series $\sum_{n=1}^{\infty} \frac{2}{n^3} = 2 \sum_{n=1}^{\infty} \frac{1}{n^3}$ is a constant multiple of a convergent p-series ($p=3>1$), so it converges.
Since the terms of the original series are smaller than the terms of a convergent series, the original series must also converge.
\textbf{Final Answer:} The series is \textbf{convergent}.

\subsection*{Problem 15}
Consider the series $\sum_{n=1}^{\infty} \frac{3}{n^2+n^3}$. Evaluate the corresponding integral to determine convergence.

\textbf{Solution:}
Let $f(x) = \frac{3}{x^2+x^3} = \frac{3}{x^2(1+x)}$. For $x \ge 1$, $f(x)$ is continuous, positive, and decreasing. We evaluate the integral using partial fractions.
\[ \frac{3}{x^2(1+x)} = \frac{A}{x} + \frac{B}{x^2} + \frac{C}{1+x} \]
$3 = Ax(1+x) + B(1+x) + Cx^2$.
If $x=0$, $3=B$. If $x=-1$, $3=C$. If $x=1$, $3=2A+2B+C \implies 3=2A+6+3 \implies 2A=-6 \implies A=-3$.
So, we integrate:
\begin{align*}
    \int_1^\infty \left( -\frac{3}{x} + \frac{3}{x^2} + \frac{3}{1+x} \right) dx &= 3 \lim_{t \to \infty} \left[ -\ln|x| - \frac{1}{x} + \ln|1+x| \right]_1^t \\
    &= 3 \lim_{t \to \infty} \left[ \ln\left(\frac{1+x}{x}\right) - \frac{1}{x} \right]_1^t \\
    &= 3 \lim_{t \to \infty} \left[ \left(\ln\left(1+\frac{1}{t}\right) - \frac{1}{t}\right) - \left(\ln(2) - 1\right) \right] \\
    &= 3 [ (\ln(1) - 0) - (\ln(2) - 1) ] \\
    &= 3 (0 - \ln(2) + 1) = 3(1-\ln(2))
\end{align*}
\textbf{Final Answer:} The integral is finite. Since the integral converges, the series is \textbf{convergent}.

\subsection*{Problem 16}
Fill in the blanks for the p-series test.

\textbf{Solution:}
If $p < 0$, then $\lim_{n \to \infty} (1/n^p) = \infty$. If $p=0$, then $\lim_{n \to \infty} (1/n^0) = \lim_{n \to \infty} 1 = \textbf{1}$. In either case, the limit is not 0, so the series diverges by the Test for Divergence.
If $p>0$, the function $f(x) = 1/x^p$ is continuous, positive, and decreasing on $[1, \infty)$.
We found that $\int_1^\infty \frac{1}{x^p} dx$ converges if $p > \textbf{1}$ and diverges if $p \le \textbf{1}$.
It follows from the Integral Test that the series $\sum \frac{1}{n^p}$ converges if $p > \textbf{1}$ and diverges if $0 < p \le \textbf{1}$.

\subsection*{Problem 17}
The series $\sum_{n=0}^{\infty} \frac{1}{n^p}$ converges when $p \ge 1$. True or False?

\textbf{Solution:}
The series as written starts at $n=0$. The first term would be $1/0^p$, which is undefined for any value of $p$. The series cannot be evaluated.
Even if we assume a typo and the series starts at $n=1$, $\sum_{n=1}^{\infty} \frac{1}{n^p}$, it converges for $p>1$. At $p=1$, we get the harmonic series, which diverges. Therefore, the statement $p \ge 1$ is incorrect.
\textbf{Final Answer:} \textbf{False}.

\subsection*{Problem 18}
Determine whether the series $\sum_{n=1}^{\infty} \frac{1}{n^2+9}$ is convergent or divergent.

\textbf{Solution:}
This is similar to problem 12. Let $f(x) = \frac{1}{x^2+9}$. This function satisfies the conditions for the Integral Test for $x \ge 1$.
\begin{align*}
    \int_1^\infty \frac{1}{x^2+9} dx &= \int_1^\infty \frac{1}{x^2+3^2} dx \\
    &= \lim_{t \to \infty} \left[ \frac{1}{3}\arctan\left(\frac{x}{3}\right) \right]_1^t \\
    &= \lim_{t \to \infty} \left( \frac{1}{3}\arctan\left(\frac{t}{3}\right) - \frac{1}{3}\arctan\left(\frac{1}{3}\right) \right) \\
    &= \frac{1}{3} \left( \frac{\pi}{2} \right) - \frac{1}{3}\arctan\left(\frac{1}{3}\right)
\end{align*}
The integral is finite. Thus, the series converges.
\textbf{Final Answer:} The series is \textbf{convergent}.

\section{In-Depth Analysis of Problems and Techniques}

\subsection{Problem Types and General Approach}
\begin{enumerate}
    \item \textbf{Direct Integral Test Application (Problems 1, 2, 3, 4):} These problems feature series where the corresponding function $f(x)$ is relatively simple to integrate. The strategy is to verify the test's conditions (positive, continuous, decreasing) and then evaluate the improper integral $\int_N^\infty f(x) dx$. Convergence of the integral implies convergence of the series, and vice versa.

    \item \textbf{p-Series Recognition (Problems 5, 6, 7, 9):} These problems are either explicitly p-series or can be simplified into the form $\sum 1/n^p$. The approach is to identify the exponent $p$ and apply the rule: converge if $p>1$, diverge if $p \le 1$.

    \item \textbf{Comparison Test Candidates (Problems 8, 10, 11, 14, 15, 18):} For these series, which often involve rational functions or sums of terms, comparison tests are typically more efficient than the Integral Test. The general strategy is to compare the given series to a simpler known series (usually a p-series) using either the Direct Comparison Test or the Limit Comparison Test.

    \item \textbf{Integral Test with Special Antiderivatives (Problems 12, 13, 18):} These require recognizing specific integral forms. Problems 12 and 18 lead to an arctangent function ($\int dx/(x^2+a^2)$), while Problem 13 leads to a natural logarithm of a natural logarithm ($\int dx/(x \ln x)$).

    \item \textbf{Conceptual Understanding (Problems 16, 17):} These test the theory behind the rules. Problem 16 walks through the derivation of the p-series rule, while Problem 17 tests close reading of the conditions (starting index and the inequality for p).
\end{enumerate}

\subsection{Key Algebraic and Calculus Manipulations}
\begin{itemize}
    \item \textbf{U-Substitution:} This was the most frequent integration technique.
    \begin{itemize}
        \item In Problem 3, setting $u = x^3+1$ was key, as its derivative, $du=3x^2dx$, matched the remaining integrand.
        \item In Problem 4, $u=-x^3$ simplified the exponential term, making the integral trivial.
        \item In Problem 13, the substitution $u=\ln(x)$ transformed the complex fraction into a simple $\int 4/u \, du$.
    \end{itemize}
    \item \textbf{Partial Fraction Decomposition:} In Problem 15, this algebraic technique was necessary to integrate the rational function $\frac{3}{x^2(1+x)}$ by breaking it into simpler parts.
    \item \textbf{Exponent Simplification:} In Problems 5 and 9, combining terms like $n \cdot n^{1/3}$ into $n^{4/3}$ was the crucial first step to identify them as p-series.
    \item \textbf{Verifying the Decreasing Condition:} For non-obvious functions like in Problem 3 ($f(x) = x^2/(x^3+1)$), formally taking the derivative $f'(x)$ and proving it is negative for $x \ge N$ is the rigorous way to satisfy the conditions of the Integral Test.
    \item \textbf{Recognizing Antiderivative Patterns:} Success in Problems 12 and 18 relied on knowing the formula $\int \frac{1}{x^2+a^2} dx = \frac{1}{a}\arctan(\frac{x}{a})$.
    \item \textbf{Evaluating Limits at Infinity:} Every integral test problem required correctly evaluating limits as $t \to \infty$. This involved knowing the behavior of functions like $\ln(t)$, $e^{-t}$, and $\arctan(t)$.
\end{itemize}

\section{"Cheatsheet" and Tips for Success}

\subsection{Core Formulas \& Rules}
\begin{itemize}
    \item \textbf{Integral Test Conditions:} For $\sum a_n$, let $f(n)=a_n$. $f(x)$ must be \textbf{continuous}, \textbf{positive}, and \textbf{decreasing} for $x \ge N$.
    \item \textbf{Integral Test Conclusion:} $\sum a_n$ and $\int_N^\infty f(x) dx$ either both converge or both diverge.
    \item \textbf{p-Series Test:} $\sum_{n=1}^\infty \frac{1}{n^p}$ \textbf{converges} for $p > 1$ and \textbf{diverges} for $p \le 1$.
\end{itemize}

\subsection{Tricks and Shortcuts}
\begin{itemize}
    \item \textbf{Look for p-Series First:} Before attempting a complex integral, always check if the series is a p-series in disguise.
    \item \textbf{Dominant Term Heuristic:} For complicated fractions, look at the highest power of $n$ in the numerator and denominator to guess the behavior. The series $\sum \frac{n^2+1}{n^4+5}$ behaves like $\sum n^2/n^4 = \sum 1/n^2$, so it likely converges. Use the Limit Comparison Test to confirm.
    \item \textbf{Remember Key Anchors for p-Series:}
        \begin{itemize}
            \item $\sum 1/n$ (Harmonic, $p=1$) \textbf{Diverges}.
            \item $\sum 1/n^2$ (Basel Problem, $p=2$) \textbf{Converges}.
        \end{itemize}
        Use these to remember the $p>1$ rule.
\end{itemize}

\subsection{Common Pitfalls and How to Avoid Them}
\begin{itemize}
    \item \textbf{Forgetting to Check Conditions:} Do not apply the Integral Test without first stating that the function is continuous, positive, AND decreasing. You might lose points for an incomplete justification.
    \item \textbf{Equating the Integral and the Sum:} The value of a convergent integral is NOT the sum of the series. The test only determines the behavior.
    \item \textbf{Mistakes with Improper Integrals:} Be careful with your limit notation and evaluation. A common error is plugging in $\infty$ directly instead of using a limit variable.
    \item \textbf{Starting Index Errors:} The starting index of the series becomes the lower limit of the integral. An incorrect lower limit will lead to a wrong value for the integral (though it usually won't change the convergence conclusion).
\end{itemize}

\section{Conceptual Synthesis and The "Big Picture"}

\subsection{Thematic Connections}
The central theme of this topic is \textbf{bridging the discrete and the continuous}. An infinite series is a discrete sum, adding distinct values at integer points. An improper integral is a continuous sum, accumulating area over an interval. The Integral Test provides a formal link, showing that under the right conditions, the behavior of one perfectly mirrors the behavior of the other. This powerful idea of approximating or understanding a discrete process with a continuous one (and vice versa) is a recurring theme in mathematics, appearing in Riemann sums (defining integrals via discrete sums), numerical integration methods, and the finite difference methods used to solve differential equations.

\subsection{Forward and Backward Links}
\begin{itemize}
    \item \textbf{Backward Link (Foundation):} This topic is a direct and necessary application of the theory of \textbf{Improper Integrals}. The ability to set up, compute, and interpret an integral with an infinite limit of integration is the mechanical core of the Integral Test. Without mastering improper integrals, this test is impossible.

    \item \textbf{Forward Link (Building Block):} Understanding convergence via the Integral Test is a crucial stepping stone towards more advanced series concepts. It builds the intuition needed for \textbf{Taylor and Maclaurin Series}, where we represent functions like $e^x$ or $\sin(x)$ as infinite polynomial series. The skills learned here are essential for determining the \textit{interval of convergence} for these power series, which defines the set of $x$-values for which the series representation is valid. This forms the basis of approximation theory and many methods in physics and engineering.
\end{itemize}

\section{Real-World Application and Modeling}

\subsection{Concrete Scenarios in Economics and Finance}
\begin{enumerate}
    \item \textbf{Perpetuity Valuation with Declining Payments:} A perpetuity is an asset that pays a stream of cash flows forever. While a simple model assumes constant payments, a more realistic model for a company in a declining industry might have payments that decrease over time. If the dividend paid in year $n$ is modeled as $D_n = \frac{C}{n^{1.2}}$, a financial analyst must determine if the total present value of all future dividends is finite. The Integral Test on the series $\sum \frac{C}{n^{1.2}(1+r)^n}$ (or more simply, a comparison to $\sum 1/n^{1.2}$) shows that the sum converges, meaning the company has a finite, calculable valuation.

    \item \textbf{Aggregate Economic Shock Decay:} Economists model the lingering effects of a major economic shock (e.g., a financial crisis or a policy change). If the impact on GDP in year $n$ after the event is $I_n = \frac{A}{n \ln(n+1)}$, a key question is whether the total cumulative effect over time is finite or infinite. The series is $\sum I_n$. Using the Integral Test on the corresponding function $f(x) = \frac{A}{x \ln(x+1)}$ reveals that the integral diverges. This implies the shock has an infinite total effect, meaning the economy never fully returns to its original trend line.
\end{enumerate}

\subsection{Model Problem Setup}
\begin{itemize}
    \item \textbf{Scenario:} Valuing a long-term infrastructure bond whose coupon payments are tied to a declining resource output.
    \item \textbf{Problem:} A government issues a perpetual bond to finance a mining project. The annual coupon payment in year $n$ is projected to be $C_n = \frac{\$1,000,000}{n^{1.5}}$ as the mine's output dwindles. An investor wants to determine if the total nominal value of all future payments is finite.
    \item \textbf{Model Setup:}
    \begin{itemize}
        \item \textbf{Variables:} $C_n$ is the coupon payment in year $n$. $V$ is the total nominal value of all future payments.
        \item \textbf{Formulation:} The total value $V$ is the sum of all future payments:
        \[ V = \sum_{n=1}^{\infty} C_n = \sum_{n=1}^{\infty} \frac{1,000,000}{n^{1.5}} \]
        \item \textbf{Equation to be Analyzed:} To find if the valuation is finite, we must determine if the series converges.
        \[ \sum_{n=1}^{\infty} \frac{1}{n^{1.5}} \]
        This is a p-series with $p=1.5$. Since $p>1$, the series converges. This means that even though the payments continue forever, their total sum is a finite amount of money, and the bond can be priced. The Integral Test on $\int_1^\infty \frac{1}{x^{1.5}}dx$ would confirm this, as the integral converges to 2.
    \end{itemize}
\end{itemize}

\section{Common Variations and Untested Concepts}
The homework focused on determining convergence. A crucial application not tested is using the integral to estimate the sum.

\subsubsection{Concept 1: Remainder Estimation}
\textbf{Explanation:} When a series converges, we can approximate its sum $S$ with a partial sum $S_n$. The error, or remainder, is $R_n = S - S_n$. The Integral Test provides tight bounds on this error: $\int_{n+1}^\infty f(x) dx \le R_n \le \int_n^\infty f(x) dx$.

\textbf{Worked Example:} Approximate the sum of $\sum_{n=1}^\infty \frac{1}{n^4}$ using the first 5 terms, and estimate the error.
\begin{itemize}
    \item \textbf{Approximation:} $S_5 = \frac{1}{1^4} + \frac{1}{2^4} + \frac{1}{3^4} + \frac{1}{4^4} + \frac{1}{5^4} = 1 + \frac{1}{16} + \frac{1}{81} + \frac{1}{256} + \frac{1}{625} \approx 1.08035$.
    \item \textbf{Error Estimate:} The remainder $R_5$ is bounded by:
    \[ \int_6^\infty \frac{1}{x^4} dx \le R_5 \le \int_5^\infty \frac{1}{x^4} dx \]
    The integral is $\int x^{-4} dx = -\frac{1}{3x^3}$.
    \[ \int_n^\infty \frac{1}{x^4} dx = \lim_{t\to\infty} [-\frac{1}{3x^3}]_n^t = 0 - (-\frac{1}{3n^3}) = \frac{1}{3n^3} \]
    So, the bounds are:
    \[ \frac{1}{3(6)^3} \le R_5 \le \frac{1}{3(5)^3} \]
    \[ \frac{1}{648} \le R_5 \le \frac{1}{375} \]
    \[ 0.00154 \le R_5 \le 0.00267 \]
    The error in our approximation $S_5 \approx 1.08035$ is between 0.00154 and 0.00267. The true sum (which is $\pi^4/90 \approx 1.08232$) is between $S_5+0.00154 = 1.08189$ and $S_5+0.00267=1.08302$.
\end{itemize}

\subsubsection{Concept 2: Finding $n$ for a Desired Accuracy}
\textbf{Explanation:} We can use the remainder estimate to determine how many terms $n$ are required to approximate the sum $S$ to within a certain tolerance. We use the upper bound, $R_n \le \int_n^\infty f(x) dx$, and solve for $n$.

\textbf{Worked Example:} How many terms are needed to ensure the sum of $\sum \frac{1}{n^4}$ is accurate to within 0.0001?
\begin{itemize}
    \item We need the error $R_n$ to be less than 0.0001.
    \[ R_n \le \int_n^\infty \frac{1}{x^4} dx < 0.0001 \]
    From the previous example, we know $\int_n^\infty \frac{1}{x^4} dx = \frac{1}{3n^3}$.
    \[ \frac{1}{3n^3} < 0.0001 \]
    \[ 1 < 0.0003 n^3 \]
    \[ n^3 > \frac{1}{0.0003} \approx 3333.33 \]
    \[ n > \sqrt[3]{3333.33} \approx 14.93 \]
    Since $n$ must be an integer, we must sum at least \textbf{15 terms} to guarantee the desired accuracy.
\end{itemize}

\section{Advanced Diagnostic Testing: "Find the Flaw"}
For each problem below, a flawed solution is presented. Your task is to find the single critical error, explain it, and provide the correct solution.

\subsection*{Problem 1}
Determine if $\sum_{n=1}^\infty \frac{n}{n^2+1}$ converges or diverges.
\begin{itemize}
    \item \textbf{Flawed Solution:} Let $f(x) = \frac{x}{x^2+1}$. We use the Integral Test. Let $u = x^2+1$, so $du = 2x dx$.
    \begin{align*}
        \int_1^\infty \frac{x}{x^2+1} dx &= \int_2^\infty \frac{1}{u} \frac{du}{2} = \frac{1}{2} [\ln|u|]_2^\infty = \frac{1}{2} [\ln(x^2+1)]_1^\infty \\
        &= \lim_{t\to\infty} \frac{1}{2} (\ln(t^2+1) - \ln(1^2+1)) = \lim_{t\to\infty} \frac{1}{2} (\ln(t^2+1) - \ln(2)) = \infty
    \end{align*}
    Since the integral diverges, the series converges.
\end{itemize}
\textbf{The Flaw:} The conclusion is incorrect; the Integral Test states that if the integral diverges, the series must also diverge. \\
\textbf{Correct Solution:} The calculation showing the integral diverges to $\infty$ is correct. Therefore, by the Integral Test, the series \textbf{diverges}.

\subsection*{Problem 2}
Determine if $\sum_{n=1}^\infty \frac{1}{n^2}$ converges or diverges.
\begin{itemize}
    \item \textbf{Flawed Solution:} This is a p-series with $p=2$. According to the p-series rule, a series of the form $\sum 1/n^p$ diverges if $p>1$. Since $p=2>1$, the series diverges.
\end{itemize}
\textbf{The Flaw:} The p-series rule has been stated incorrectly; the series converges if $p>1$. \\
\textbf{Correct Solution:} This is a p-series with $p=2$. Since $p>1$, the series \textbf{converges}.

\subsection*{Problem 3}
Determine if $\sum_{n=2}^\infty \frac{1}{n \ln^2(n)}$ converges or diverges.
\begin{itemize}
    \item \textbf{Flawed Solution:} We use the Integral Test with $f(x) = \frac{1}{x \ln^2(x)}$. Let $u = \ln(x)$, so $du = (1/x) dx$.
    \begin{align*}
        \int_2^\infty \frac{1}{x \ln^2(x)} dx &= \int_{\ln(2)}^\infty \frac{1}{u^2} du = [-\frac{1}{u}]_{\ln(2)}^\infty \\
        &= \lim_{t\to\infty} (-\frac{1}{t}) - (-\frac{1}{\ln(2)}) = 0 + \frac{1}{\ln(2)} = \frac{1}{\ln(2)}
    \end{align*}
    The integral is finite, so the series diverges.
\end{itemize}
\textbf{The Flaw:} The conclusion is incorrect; if the integral is finite (converges), the series must also converge. \\
\textbf{Correct Solution:} The calculation showing the integral converges to $\frac{1}{\ln(2)}$ is correct. Therefore, by the Integral Test, the series \textbf{converges}.

\subsection*{Problem 4}
Find the smallest integer $n$ such that the remainder $R_n$ for the series $\sum_{k=1}^\infty \frac{1}{k^4}$ is less than $\frac{1}{3000}$.
\begin{itemize}
    \item \textbf{Flawed Solution:} We need $R_n < 1/3000$. We use the remainder estimate $R_n \le \int_n^\infty \frac{1}{x^4} dx$.
    \[ \int_n^\infty \frac{1}{x^4} dx = [-\frac{1}{3x^3}]_n^\infty = 0 - (-\frac{1}{3n^3}) = \frac{1}{3n^3} \]
    We set up the inequality $\frac{1}{3n^3} < \frac{1}{3000}$, which simplifies to $3n^3 > 3000$, then $n^3 > 1000$, so $n>10$. The smallest integer is $n=10$.
\end{itemize}
\textbf{The Flaw:} The conclusion is incorrect; if $n>10$, the smallest integer value for $n$ is 11, not 10. \\
\textbf{Correct Solution:} The inequality $n>10$ is correct. The smallest integer $n$ that satisfies this condition is $n=\textbf{11}$.

\subsection*{Problem 5}
Determine if $\sum_{n=1}^\infty e^{-n}$ converges or diverges.
\begin{itemize}
    \item \textbf{Flawed Solution:} We use the Integral Test with $f(x) = e^{-x}$. The conditions are met.
    \begin{align*}
    \int_1^\infty e^{-x} dx = [e^{-x}]_1^\infty = \lim_{t\to\infty} (e^{-t} - e^{-1}) = 0 - e^{-1} = -1/e
    \end{align*}
    Since the integral converges to a finite value, the series converges.
\end{itemize}
\textbf{The Flaw:} There is an algebraic error in the evaluation of the integral's antiderivative. The integral of $e^{-x}$ is $-e^{-x}$. \\
\textbf{Correct Solution:}
\begin{align*}
    \int_1^\infty e^{-x} dx &= [-e^{-x}]_1^\infty \\
    &= \lim_{t\to\infty} (-e^{-t} - (-e^{-1})) \\
    &= 0 + e^{-1} = 1/e
\end{align*}
The integral converges to the positive value $1/e$. Since the integral converges, the series \textbf{converges}. (Note: This is also a geometric series with $r=1/e < 1$, which confirms convergence).

\end{document}