\documentclass{article}
\usepackage{amsmath}
\usepackage{amssymb}
\usepackage{geometry}
\geometry{a4paper, margin=1in}

\title{Homework 11.5 Alternating Series and Absolute Convergence}
\author{Tashfeen Omran}
\date{November 2025}

\begin{document}

\maketitle

\part{Comprehensive Introduction, Context, and Prerequisites}

\section{Core Concepts}
This chapter introduces a special category of series called \textbf{alternating series} and provides a powerful test for their convergence. It also distinguishes between two important types of convergence: \textbf{absolute} and \textbf{conditional}.

\subsection{Alternating Series}
An alternating series is a series whose terms alternate between positive and negative. They can be written in one of two forms:
\[ \sum_{n=1}^{\infty} (-1)^{n-1} b_n = b_1 - b_2 + b_3 - b_4 + \dots \]
or
\[ \sum_{n=1}^{\infty} (-1)^{n} b_n = -b_1 + b_2 - b_3 + b_4 - \dots \]
where $b_n$ is a sequence of positive numbers, i.e., $b_n > 0$. The term $b_n$ represents the magnitude, or absolute value, of the $n$-th term.

\subsection{The Alternating Series Test (AST)}
The Alternating Series Test (also known as Leibniz's Test) gives us conditions to determine if an alternating series converges. The series $\sum (-1)^{n-1} b_n$ converges if it satisfies two conditions:
\begin{enumerate}
    \item \textbf{Decreasing Magnitude:} The sequence of magnitudes $b_n$ is eventually decreasing. That is, there exists an integer $N$ such that for all $n \ge N$, we have $b_{n+1} \le b_n$.
    \item \textbf{Limit is Zero:} The limit of the magnitudes is zero.
    \[ \lim_{n \to \infty} b_n = 0 \]
\end{enumerate}
\textbf{Note:} Both conditions must be met. If $\lim_{n \to \infty} b_n \neq 0$, the series diverges by the Test for Divergence. If the limit is zero but the terms are not decreasing, the test is inconclusive.

\subsection{Alternating Series Estimation Theorem}
If a convergent alternating series satisfies the conditions of the AST, we can estimate the total sum $s$ by using a partial sum $s_n$. The error, or remainder $R_n = s - s_n$, is bounded by the magnitude of the \emph{first unused term}.
\[ |R_n| = |s - s_n| \le b_{n+1} \]
This is extremely useful because it gives us a firm grasp on the accuracy of our approximation.

\subsection{Absolute and Conditional Convergence}
This is a crucial concept that classifies \emph{how} a series converges.
\begin{itemize}
    \item \textbf{Absolute Convergence:} A series $\sum a_n$ is \textbf{absolutely convergent} if the series of its absolute values, $\sum |a_n|$, converges.
    \item \textbf{Conditional Convergence:} A series $\sum a_n$ is \textbf{conditionally convergent} if the series itself converges, but the series of its absolute values, $\sum |a_n|$, diverges.
\end{itemize}

A foundational theorem connects these ideas:
\textbf{Theorem:} If a series $\sum a_n$ is absolutely convergent, then it is convergent.
This means that absolute convergence is a stronger condition. To classify an alternating series, we follow a clear procedure:
\begin{enumerate}
    \item Test the series of absolute values, $\sum |a_n| = \sum b_n$, for convergence using any of our known tests (Integral, Comparison, LCT, Ratio, etc.).
    \item If $\sum b_n$ converges, we are done. The original series is \textbf{absolutely convergent}.
    \item If $\sum b_n$ diverges, we proceed to step 2. We must now test the original alternating series $\sum a_n$ with the Alternating Series Test.
    \item If $\sum a_n$ converges by AST, then the series is \textbf{conditionally convergent}.
    \item If $\sum a_n$ diverges (which, for an alternating series, usually happens because $\lim b_n \neq 0$), then the series is simply \textbf{divergent}.
\end{enumerate}

\section{Intuition and Derivation}
\textbf{Alternating Series Test:} Imagine the partial sums plotted on a number line. $s_1 = b_1$. To get $s_2$, we subtract $b_2$, moving to the left. To get $s_3$, we add $b_3$, moving to the right, but by a smaller amount than we moved left. The partial sums $s_n$ bounce back and forth. The condition $b_{n+1} \le b_n$ ensures the jumps get smaller each time, and the condition $\lim_{n \to \infty} b_n = 0$ ensures these jumps shrink to nothing. The sequence of partial sums gets squeezed between its upper and lower bounds and must converge to a single point.

\textbf{Absolute vs. Conditional Convergence:} Think of absolute convergence as having a "convergence safety net." The series $\sum |a_n|$ has all positive terms. If this series converges, it means the terms go to zero so quickly that even without any helpful cancellation from negative signs, the sum is finite. The original series $\sum a_n$, with its cancellations, must therefore also converge. Conditional convergence is a more delicate balancing act. The terms don't go to zero fast enough for $\sum |a_n|$ to converge (e.g., the harmonic series $\sum 1/n$). It is only through the precise, structured cancellation of positive and negative terms that the overall sum manages to converge.

\section{Historical Context and Motivation}
The study of infinite series blossomed in the 17th and 18th centuries, driven by the development of calculus. Mathematicians like Newton and Leibniz discovered that many functions, such as $\ln(1+x)$ or $\arctan(x)$, could be represented by infinite polynomial series (Taylor series). A prime example is the Gregory-Leibniz series for $\pi$:
\[ \frac{\pi}{4} = 1 - \frac{1}{3} + \frac{1}{5} - \frac{1}{7} + \dots \]
This is a classic alternating series. A critical question arose: when does such a series actually converge to the value of the function it represents? This practical need motivated Gottfried Wilhelm Leibniz to formalize the conditions for the convergence of alternating series around 1682, leading to what we now call the Alternating Series Test. The deeper distinction between absolute and conditional convergence was explored later by mathematicians like Cauchy and Riemann in the 19th century as they built a more rigorous foundation for calculus.

\section{Key Formulas}
\begin{itemize}
    \item \textbf{Alternating Series Form:} $\sum (-1)^{n-1} b_n$ or $\sum (-1)^{n} b_n$ with $b_n > 0$.
    \item \textbf{Alternating Series Test Conditions:}
    \begin{enumerate}
        \item $b_{n+1} \le b_n$ (for $n \ge N$)
        \item $\lim_{n \to \infty} b_n = 0$
    \end{enumerate}
    \item \textbf{Alternating Series Remainder Estimate:} $|R_n| = |s - s_n| \le b_{n+1}$.
    \item \textbf{Absolute Convergence:} $\sum |a_n|$ converges.
    \item \textbf{Conditional Convergence:} $\sum a_n$ converges AND $\sum |a_n|$ diverges.
\end{itemize}

\section{Prerequisites}
To master this topic, you must be proficient in the following:
\begin{itemize}
    \item \textbf{Limits:} Calculating limits at infinity is essential for the AST and the Test for Divergence.
    \item \textbf{Derivatives:} The most reliable way to prove a sequence is decreasing is often to take the derivative of the corresponding function $f(x)$ and show it's negative. This requires knowing differentiation rules (especially the quotient rule).
    \item \textbf{Tests for Positive Series:} All previous convergence tests are now part of the toolkit for testing absolute convergence. You must be able to recognize when and how to apply:
    \begin{itemize}
        \item The Test for Divergence
        \item The p-Series Test
        \item The Limit Comparison Test (very common)
        \item The Direct Comparison Test
        \item The Integral Test
    \end{itemize}
\end{itemize}

\part{Detailed Homework Solutions}

\section*{Problem 1}
\textbf{(a) What is an alternating series?}
An alternating series is a \textbf{series} whose terms are \textbf{alternately positive and negative}.

\textbf{(b) Under what conditions does an alternating series converge?}
An alternating series $\sum_{n=1}^{\infty} (-1)^{n-1} b_n$, where $b_n = |a_n|$, converges if $0 < b_{n+1} \le b_n$ for all $n$, and $\lim_{n \to \infty} b_n = \mathbf{0}$.

\textbf{(c) If these conditions are satisfied, what can you say about the remainder after n terms?}
The error involved in using the partial sum $s_n$ as an approximation to the total sum $s$ is the \textbf{remainder} $R_n = s - s_n$ and the size of the error is \textbf{less than or equal to} $b_{n+1}$.

\section*{Problem 2}
Test the series for convergence using the Alternating Series Test.
\[ \frac{2}{3} - \frac{2}{5} + \frac{2}{7} - \frac{2}{9} + \frac{2}{11} - \dots = \sum_{n=1}^{\infty} (-1)^{n-1} \frac{2}{2n+1} \]
\textbf{Step 1: Identify $b_n$.}
Here, $b_n = \frac{2}{2n+1}$.

\textbf{Step 2: Check if $\lim_{n \to \infty} b_n = 0$.}
\[ \lim_{n \to \infty} \frac{2}{2n+1} = 0 \]
The first condition is satisfied.

\textbf{Step 3: Check if $b_n$ is decreasing.}
We need to check if $b_{n+1} \le b_n$.
\[ b_{n+1} = \frac{2}{2(n+1)+1} = \frac{2}{2n+3} \]
Since $2n+3 > 2n+1$, the denominator of $b_{n+1}$ is larger than the denominator of $b_n$. For fractions with the same numerator, the one with the larger denominator is smaller. Thus,
\[ \frac{2}{2n+3} \le \frac{2}{2n+1} \implies b_{n+1} \le b_n \]
The second condition is satisfied.

\textbf{Conclusion:} Since both conditions of the Alternating Series Test are met, \textbf{the series converges}.

\section*{Problem 3}
Test the series for convergence using the Alternating Series Test.
\[ \frac{8}{9} - \frac{8}{11} + \frac{8}{13} - \frac{8}{15} + \frac{8}{17} - \dots = \sum_{n=1}^{\infty} (-1)^{n-1} \frac{8}{2n+7} \]
\textbf{Step 1: Identify $b_n$.}
Here, $b_n = \frac{8}{2n+7}$.

\textbf{Step 2: Check if $\lim_{n \to \infty} b_n = 0$.}
\[ \lim_{n \to \infty} \frac{8}{2n+7} = 0 \]
The first condition is satisfied.

\textbf{Step 3: Check if $b_n$ is decreasing.}
\[ b_{n+1} = \frac{8}{2(n+1)+7} = \frac{8}{2n+9} \]
Since $2n+9 > 2n+7$, we have $b_{n+1} \le b_n$. The second condition is satisfied.

\textbf{Conclusion:} Both conditions are met. The series converges by the AST.

\section*{Problem 4}
Test the series for convergence using the Alternating Series Test.
\[ \frac{2}{8} - \frac{4}{9} + \frac{6}{10} - \frac{8}{11} + \dots = \sum_{n=1}^{\infty} (-1)^{n+1} \frac{2n}{n+7} \]
\textbf{Step 1: Identify $b_n$.}
$b_n = \frac{2n}{n+7}$.

\textbf{Step 2: Check the limit of $b_n$.}
We apply the Test for Divergence to the alternating series by checking the limit of the terms.
\[ \lim_{n \to \infty} b_n = \lim_{n \to \infty} \frac{2n}{n+7} = \lim_{n \to \infty} \frac{2}{1+7/n} = 2 \]
Since the limit is 2 (and not 0), the terms of the series, $(-1)^{n+1}b_n$, do not approach 0. They approach alternating between 2 and -2.

\textbf{Conclusion:} By the Test for Divergence, the series \textbf{diverges}.

\section*{Problem 5}
Test the series for convergence using the Alternating Series Test.
\[ \frac{1}{\ln(4)} - \frac{1}{\ln(5)} + \frac{1}{\ln(6)} - \dots = \sum_{n=1}^{\infty} (-1)^{n-1} \frac{1}{\ln(n+3)} \]
\textbf{Step 1: Identify $b_n$.}
$b_n = \frac{1}{\ln(n+3)}$.

\textbf{Step 2: Check if $\lim_{n \to \infty} b_n = 0$.}
As $n \to \infty$, $\ln(n+3) \to \infty$, so:
\[ \lim_{n \to \infty} \frac{1}{\ln(n+3)} = 0 \]
The first condition is satisfied.

\textbf{Step 3: Check if $b_n$ is decreasing.}
The natural logarithm function, $\ln(x)$, is an increasing function. Therefore, as $n$ increases, $n+3$ increases, and $\ln(n+3)$ increases. This means the denominator of $b_n$ is always increasing. For a fraction with a constant numerator, an increasing denominator means the fraction value is decreasing. Thus, $b_{n+1} \le b_n$. The second condition is satisfied.

\textbf{Conclusion:} Both conditions of the AST are met. The series \textbf{converges}.

\section*{Problem 6}
Test the series for convergence using the Alternating Series Test.
\[ \sum_{n=1}^{\infty} \frac{(-1)^{n-1} n}{2+7n} \]
\textbf{Step 1: Identify $b_n$.}
$b_n = \frac{n}{2+7n}$.

\textbf{Step 2: Check the limit of $b_n$.}
\[ \lim_{n \to \infty} b_n = \lim_{n \to \infty} \frac{n}{2+7n} = \lim_{n \to \infty} \frac{1}{2/n+7} = \frac{1}{7} \]
The limit of the magnitude of the terms is not 0.

\textbf{Conclusion:} By the Test for Divergence, the series \textbf{diverges}.

\section*{Problem 7}
Test the series for convergence using the Alternating Series Test.
\[ \sum_{n=0}^{\infty} \frac{(-1)^{n+1}}{\sqrt{n+8}} \]
\textbf{Step 1: Identify $b_n$.}
$b_n = \frac{1}{\sqrt{n+8}}$.

\textbf{Step 2: Check if $\lim_{n \to \infty} b_n = 0$.}
As $n \to \infty$, $\sqrt{n+8} \to \infty$, so:
\[ \lim_{n \to \infty} \frac{1}{\sqrt{n+8}} = 0 \]
The first condition is satisfied.

\textbf{Step 3: Check if $b_n$ is decreasing.}
As $n$ increases, $n+8$ increases, and $\sqrt{n+8}$ increases. An increasing denominator means the fraction is decreasing. Thus, $b_{n+1} \le b_n$. The second condition is satisfied.

\textbf{Conclusion:} The series \textbf{converges} by the AST.

\section*{Problem 8}
Test the series for convergence using the Alternating Series Test.
\[ \sum_{n=1}^{\infty} (-1)^n \frac{9n-1}{8n+1} \]
\textbf{Step 1: Identify $b_n$.}
$b_n = \frac{9n-1}{8n+1}$.

\textbf{Step 2: Check the limit of $b_n$.}
\[ \lim_{n \to \infty} b_n = \lim_{n \to \infty} \frac{9n-1}{8n+1} = \lim_{n \to \infty} \frac{9-1/n}{8+1/n} = \frac{9}{8} \]
Since the limit is not 0, the series diverges.

\textbf{Conclusion:} The series \textbf{diverges} by the Test for Divergence.

\section*{Problem 9}
Test the series for convergence or divergence using the Alternating Series Test.
\[ \sum_{n=1}^{\infty} (-1)^n \frac{4n-1}{7n+1} \]
This problem is structurally identical to the previous one.
\textbf{Step 1: Identify $b_n$.}
$b_n = \frac{4n-1}{7n+1}$.

\textbf{Step 2: Check the limit of $b_n$.}
\[ \lim_{n \to \infty} b_n = \lim_{n \to \infty} \frac{4n-1}{7n+1} = \lim_{n \to \infty} \frac{4-1/n}{7+1/n} = \frac{4}{7} \]
\textbf{Conclusion:} Since $\lim_{n \to \infty} b_n \neq 0$, the series \textbf{diverges} by the Test for Divergence.

\section*{Problem 10}
Test the series for convergence using the Alternating Series Test.
\[ \sum_{n=1}^{\infty} (-1)^n \frac{n^4}{n^4+n^2+1} \]
\textbf{Step 1: Identify $b_n$.}
$b_n = \frac{n^4}{n^4+n^2+1}$.

\textbf{Step 2: Check the limit of $b_n$.}
Divide numerator and denominator by $n^4$:
\[ \lim_{n \to \infty} b_n = \lim_{n \to \infty} \frac{1}{1+1/n^2+1/n^4} = \frac{1}{1+0+0} = 1 \]
\textbf{Conclusion:} Since the limit is not 0, the series \textbf{diverges} by the Test for Divergence.

\section*{Problem 11}
Test the series for convergence using the Alternating Series Test.
\[ \sum_{n=1}^{\infty} 5(-1)^n e^{-n} = \sum_{n=1}^{\infty} 5(-1)^n \frac{1}{e^n} \]
\textbf{Step 1: Identify $b_n$.}
$b_n = \frac{5}{e^n}$.

\textbf{Step 2: Check if $\lim_{n \to \infty} b_n = 0$.}
As $n \to \infty$, $e^n \to \infty$, so:
\[ \lim_{n \to \infty} \frac{5}{e^n} = 0 \]
The first condition is satisfied.

\textbf{Step 3: Check if $b_n$ is decreasing.}
The function $e^x$ is an increasing function. Therefore, the denominator $e^n$ is increasing, which means the fraction $b_n = 5/e^n$ is decreasing. The second condition is satisfied.

\textbf{Conclusion:} The series \textbf{converges} by the AST.

\section*{Problem 12}
Test the series for convergence using the Alternating Series Test.
\[ \sum_{n=1}^{\infty} (-1)^n \frac{\sqrt{n}}{3n+5} \]
\textbf{Step 1: Identify $b_n$.}
$b_n = \frac{\sqrt{n}}{3n+5}$.

\textbf{Step 2: Check if $\lim_{n \to \infty} b_n = 0$.}
Divide numerator and denominator by $n$:
\[ \lim_{n \to \infty} \frac{\sqrt{n}}{3n+5} = \lim_{n \to \infty} \frac{1/\sqrt{n}}{3+5/n} = \frac{0}{3+0} = 0 \]
The first condition is satisfied.

\textbf{Step 3: Check if $b_n$ is decreasing.}
This is not immediately obvious, so let's use the derivative test. Let $f(x) = \frac{\sqrt{x}}{3x+5}$.
Using the quotient rule:
\begin{align*}
f'(x) &= \frac{(3x+5)\cdot \frac{1}{2\sqrt{x}} - \sqrt{x} \cdot 3}{(3x+5)^2} \\
&= \frac{\frac{3x+5}{2\sqrt{x}} - 3\sqrt{x}}{(3x+5)^2} \\
&= \frac{3x+5 - 3\sqrt{x}(2\sqrt{x})}{2\sqrt{x}(3x+5)^2} \\
&= \frac{3x+5 - 6x}{2\sqrt{x}(3x+5)^2} = \frac{5 - 3x}{2\sqrt{x}(3x+5)^2}
\end{align*}
The denominator is always positive for $x > 0$. The numerator, $5-3x$, is negative for $x > 5/3$. Therefore, for $n \ge 2$, $f'(n)$ is negative, which means the sequence $b_n$ is decreasing for $n \ge 2$. The second condition is satisfied.

\textbf{Conclusion:} Both conditions are met. The series \textbf{converges} by the AST.

\section*{Problem 13}
\textbf{(a) What does it mean for a series to be absolutely convergent?}
$\sum a_n$ is absolutely convergent if $\mathbf{\sum a_n \text{ converges and } \sum |a_n| \text{ converges}}$. (Note: The best definition is just that $\sum |a_n|$ converges, which implies $\sum a_n$ converges). The second option is the correct choice from the screenshot.

\textbf{(b) What does it mean for a series to be conditionally convergent?}
$\sum a_n$ is conditionally convergent if $\mathbf{\sum a_n \text{ converges but } \sum |a_n| \text{ diverges}}$.

\textbf{(c) If the series of positive terms $\sum_{n=1}^{\infty} b_n$ converges, then what can you say about the series $\sum_{n=1}^{\infty} (-1)^n b_n$?}
If $\sum b_n$ converges, then by definition, the series $\sum (-1)^n b_n$ is absolutely convergent. The series of absolute values is $\sum |(-1)^n b_n| = \sum b_n$, which is given to converge.
The correct option is: $\mathbf{\sum_{n=1}^{\infty} (-1)^n b_n \text{ is absolutely convergent}}$.

\section*{Problem 14}
Determine whether the series is absolutely convergent, conditionally convergent, or divergent.
\[ \sum_{n=1}^{\infty} \frac{(-1)^{n-1}}{n^{5/6}} \]
\textbf{Step 1: Test for absolute convergence.}
Consider the series of absolute values:
\[ \sum_{n=1}^{\infty} \left| \frac{(-1)^{n-1}}{n^{5/6}} \right| = \sum_{n=1}^{\infty} \frac{1}{n^{5/6}} \]
This is a p-series with $p = 5/6$. Since $p = 5/6 \le 1$, the series of absolute values \textbf{diverges}. This means the original series is NOT absolutely convergent.

\textbf{Step 2: Test the original series for convergence using AST.}
The original series is an alternating series with $b_n = \frac{1}{n^{5/6}}$.
\begin{enumerate}
    \item $\lim_{n \to \infty} b_n = \lim_{n \to \infty} \frac{1}{n^{5/6}} = 0$.
    \item $b_{n+1} = \frac{1}{(n+1)^{5/6}} \le \frac{1}{n^{5/6}} = b_n$ because the denominator is increasing.
\end{enumerate}
Both conditions of the AST are met, so the original series converges.

\textbf{Conclusion:} The series converges, but not absolutely. Therefore, the series is \textbf{conditionally convergent}.

\section*{Problem 15}
Determine whether the series is absolutely convergent, conditionally convergent, or divergent.
\[ \sum_{n=1}^{\infty} \frac{(-1)^n}{9n+1} \]
\textbf{Step 1: Test for absolute convergence.}
Consider the series of absolute values:
\[ \sum_{n=1}^{\infty} \frac{1}{9n+1} \]
We use the Limit Comparison Test (LCT) with the harmonic series $\sum \frac{1}{n}$, which is a divergent p-series ($p=1$).
\[ L = \lim_{n \to \infty} \frac{\frac{1}{9n+1}}{\frac{1}{n}} = \lim_{n \to \infty} \frac{n}{9n+1} = \frac{1}{9} \]
Since $L=1/9$ is a finite positive number, both series share the same fate. Since $\sum \frac{1}{n}$ diverges, the series $\sum \frac{1}{9n+1}$ also \textbf{diverges}. The original series is NOT absolutely convergent.

\textbf{Step 2: Test the original series with AST.}
The original series is alternating with $b_n = \frac{1}{9n+1}$.
\begin{enumerate}
    \item $\lim_{n \to \infty} b_n = \lim_{n \to \infty} \frac{1}{9n+1} = 0$.
    \item $b_{n+1} = \frac{1}{9(n+1)+1} = \frac{1}{9n+10} \le \frac{1}{9n+1} = b_n$ because the denominator is increasing.
\end{enumerate}
Both conditions are met, so the original series converges.

\textbf{Conclusion:} The series converges, but not absolutely. Therefore, the series is \textbf{conditionally convergent}.

\section*{Problem 16}
This problem breaks down Problem 15 into parts.
\[ \sum_{n=1}^{\infty} \frac{(-1)^n}{\sqrt{7n+1}} \]
This appears to be a typo in the OCR vs the problem number, which references 11.5.025.EP. Problem 15 is 11.5.025. I will solve the problem shown in the image for Q16.
\[ \sum_{n=1}^{\infty} \frac{(-1)^n}{\sqrt{7n+1}} \]
\textbf{Step 1: Identify $b_n$ and test AST conditions.}
$b_n = \frac{1}{\sqrt{7n+1}}$.
$\lim_{n \to \infty} \frac{1}{\sqrt{7n+1}} = 0$.
The denominator $\sqrt{7n+1}$ is increasing, so $b_n$ is decreasing.
The series converges by AST.

\textbf{Step 2: Test $\sum b_n$ for convergence.}
We test $\sum_{n=1}^{\infty} \frac{1}{\sqrt{7n+1}} = \sum_{n=1}^{\infty} \frac{1}{(7n+1)^{1/2}}$ for convergence.
Use LCT with the divergent p-series $\sum \frac{1}{\sqrt{n}} = \sum \frac{1}{n^{1/2}}$ ($p=1/2 \le 1$).
\[ L = \lim_{n \to \infty} \frac{\frac{1}{\sqrt{7n+1}}}{\frac{1}{\sqrt{n}}} = \lim_{n \to \infty} \frac{\sqrt{n}}{\sqrt{7n+1}} = \lim_{n \to \infty} \sqrt{\frac{n}{7n+1}} = \sqrt{\frac{1}{7}} \]
Since $L$ is finite and positive, $\sum b_n$ diverges.

\textbf{Conclusion:} The series converges, but the series of absolute values diverges. Therefore, the series is \textbf{conditionally convergent}. The correct choice for testing $\sum b_n$ would be: "The series diverges by the Limit Comparison Test with a divergent p-series."

\section*{Problem 17}
Determine whether the series is absolutely convergent, conditionally convergent, or divergent.
\[ \sum_{n=1}^{\infty} \frac{(-1)^n}{n^5+6} \]
\textbf{Step 1: Test for absolute convergence.}
Consider the series of absolute values:
\[ \sum_{n=1}^{\infty} \frac{1}{n^5+6} \]
We use the Limit Comparison Test with the convergent p-series $\sum \frac{1}{n^5}$ ($p=5 > 1$).
\[ L = \lim_{n \to \infty} \frac{\frac{1}{n^5+6}}{\frac{1}{n^5}} = \lim_{n \to \infty} \frac{n^5}{n^5+6} = 1 \]
Since $L=1$ is a finite positive number, and $\sum \frac{1}{n^5}$ converges, the series $\sum \frac{1}{n^5+6}$ also converges.

\textbf{Conclusion:} Since the series of absolute values converges, the original series is \textbf{absolutely convergent}.

\section*{Problem 18}
Consider the series $\sum_{n=2}^{\infty} \frac{(-1)^n}{\ln(8n)}$.
\textbf{Step 1: Test for absolute convergence.}
Consider the series $\sum_{n=2}^{\infty} \frac{1}{\ln(8n)}$.
We can use the Direct Comparison Test. For $n \ge 1$, we know that $n > \ln(8n)$. (This is true for $n \ge 1$, as $e^n$ grows much faster than $8n$). Therefore,
\[ \frac{1}{n} < \frac{1}{\ln(8n)} \]
Since $\sum_{n=2}^{\infty} \frac{1}{n}$ is the harmonic series (divergent p-series), and its terms are smaller than the terms of our series, the series $\sum \frac{1}{\ln(8n)}$ must also \textbf{diverge} by the Direct Comparison Test.
The original series is NOT absolutely convergent.

\textbf{Step 2: Test the original series with AST.}
The series is alternating with $b_n = \frac{1}{\ln(8n)}$.
\begin{enumerate}
    \item $\lim_{n \to \infty} b_n = \lim_{n \to \infty} \frac{1}{\ln(8n)} = 0$ since $\ln(8n) \to \infty$.
    \item As $n$ increases, $8n$ increases, and $\ln(8n)$ increases. Therefore, $b_n = \frac{1}{\ln(8n)}$ is decreasing.
\end{enumerate}
Both conditions of the AST are met. The original series converges.

\textbf{Conclusion:} The series converges, but not absolutely. It is \textbf{conditionally convergent}. The test for $\sum b_n$ is: "The series diverges by the Direct Comparison Test. Each term is greater than that of a comparable harmonic series."

\part{In-Depth Analysis of Problems and Techniques}
\subsection{Problem Types and General Approach}
The homework problems can be categorized into three main types:
\begin{enumerate}
    \item \textbf{Conceptual Problems (1, 13):} These problems test the direct knowledge of definitions. The approach is to have the formal definitions of an alternating series, the AST, absolute convergence, and conditional convergence memorized.
    \item \textbf{Divergence by the n-th Term Test (4, 6, 8, 9, 10):} These problems are designed to look like standard alternating series, but they fail the most basic requirement for convergence.
        \textbf{General Approach:} For any alternating series $\sum (-1)^n b_n$, the very first thing you should do is mentally (or quickly on paper) compute $\lim_{n \to \infty} b_n$. If this limit is anything other than 0, you are done. The series diverges by the Test for Divergence. No other steps are needed.
    \item \textbf{Classification Problems (Absolute vs. Conditional) (2, 3, 5, 7, 11, 12, 14, 15, 16, 17, 18):} These are the core problems of the section, requiring a full analysis.
        \textbf{General Approach:}
        \begin{enumerate}
            \item \textbf{Check Absolute Convergence First:} Look at the series $\sum |a_n| = \sum b_n$. Use your toolbox of tests for positive-termed series (p-Series, LCT, DCT) to determine if $\sum b_n$ converges or diverges.
            \item \textbf{Conclude if Possible:} If $\sum b_n$ converges, the series is \textbf{absolutely convergent}. Stop.
            \item \textbf{Apply AST:} If $\sum b_n$ diverges, you must proceed. Now test the original alternating series $\sum a_n$ using the two conditions of the AST ($\lim b_n = 0$ and $b_n$ decreasing).
            \item \textbf{Final Classification:} If the AST shows convergence, the series is \textbf{conditionally convergent}. If the AST fails, the series is \textbf{divergent}.
        \end{enumerate}
\end{enumerate}

\subsection{Key Algebraic and Calculus Manipulations}
\begin{itemize}
    \item \textbf{Limit Evaluation at Infinity:} This was used in every single problem. The key technique for rational functions (e.g., problems 6, 8, 9, 10) is to divide the numerator and denominator by the highest power of $n$ in the denominator.
    \item \textbf{Derivative for Decreasing Test:} In Problem 12, the term $b_n = \frac{\sqrt{n}}{3n+5}$ was not obviously decreasing. The standard procedure is to create a function $f(x) = \frac{\sqrt{x}}{3x+5}$ and use the \textbf{Quotient Rule} to find $f'(x)$. Showing that $f'(x) < 0$ for all $x$ beyond a certain value proves the sequence is eventually decreasing. This is a crucial technique.
    \item \textbf{P-Series Identification:} This was the essential tool in the absolute convergence test for Problem 14 ($\sum 1/n^{5/6}$). Recognizing the form $\sum 1/n^p$ and knowing it converges only for $p>1$ is fundamental.
    \item \textbf{Limit Comparison Test (LCT):} This was the workhorse for testing absolute convergence.
    \begin{itemize}
        \item In Problem 15/16, $\sum 1/(9n+1)$ was compared to the divergent harmonic series $\sum 1/n$.
        \item In Problem 17, $\sum 1/(n^5+6)$ was compared to the convergent p-series $\sum 1/n^5$.
        \item \textbf{Why it was necessary:} The LCT allows us to ignore less significant parts of the terms (like the `+1` or `+6`) and focus on the dominant behavior of the function as $n \to \infty$.
    \end{itemize}
    \item \textbf{Direct Comparison Test (DCT):} This was used in Problem 18 to show that $\sum 1/\ln(8n)$ diverges.
    \begin{itemize}
        \item \textbf{Why it was necessary:} The key was knowing the fundamental inequality $\ln(x) < x$ for $x \ge 1$. This allowed us to establish that $\frac{1}{\ln(8n)} > \frac{1}{8n}$ (or simply $\frac{1}{n}$). By showing our terms were larger than the terms of a known divergent series (the harmonic series), we could conclude divergence.
    \end{itemize}
\end{itemize}

\part{"Cheatsheet" and Tips for Success}
\section{Summary of Formulas}
\begin{itemize}
    \item \textbf{Alternating Series Test:} For $\sum (-1)^n b_n$ with $b_n > 0$, if
    \begin{enumerate}
        \item $b_{n+1} \le b_n$ (eventually)
        \item $\lim_{n \to \infty} b_n = 0$
    \end{enumerate}
    ...then the series converges.
    \item \textbf{Remainder Estimate:} $|s - s_n| \le b_{n+1}$.
    \item \textbf{Classification Flowchart:}
    \begin{enumerate}
        \item Does $\sum |a_n|$ converge? $\rightarrow$ YES: Absolutely Convergent.
        \item Does $\sum |a_n|$ converge? $\rightarrow$ NO: Go to next step.
        \item Does $\sum a_n$ converge by AST? $\rightarrow$ YES: Conditionally Convergent.
        \item Does $\sum a_n$ converge by AST? $\rightarrow$ NO: Divergent.
    \end{enumerate}
\end{itemize}

\section{Tricks and Shortcuts}
\begin{itemize}
    \item \textbf{The Divergence Test is Your First Check.} Always compute $\lim_{n \to \infty} b_n$. If it's not zero, you're done. This instantly solves Problems 4, 6, 8, 9, and 10.
    \item \textbf{Test Absolute Convergence First.} When classifying, always start with $\sum |a_n|$. If it converges, you save yourself the work of checking the AST conditions.
    \item \textbf{LCT is Your Go-To for Polynomials/Rationals.} When $|a_n|$ looks like a fraction of polynomials (e.g., $1/(9n+1)$), use LCT with the appropriate p-series (e.g., $1/n$).
    \item \textbf{Know Your Inequalities for DCT.} For terms involving logarithms or other functions, remember key inequalities like $\ln(n) < n$ and $\sin(n) \le 1$.
\end{itemize}

\section{Common Pitfalls and Mistakes}
\begin{itemize}
    \item \textbf{Forgetting to check both AST conditions.} A common mistake is to see $\lim b_n = 0$ and immediately conclude convergence without showing the terms are decreasing.
    \item \textbf{Confusing "Divergent" with "Conditionally Convergent".} If $\sum |a_n|$ diverges, you cannot stop. You must still test $\sum a_n$.
    \item \textbf{Algebraic errors with the derivative.} The quotient rule can be tricky. Be careful with your algebra when trying to show $f'(x) < 0$.
    \item \textbf{Improper use of LCT.} Remember that the LCT only works for series with positive terms. You use it on $\sum |a_n|$, not the original alternating series.
\end{itemize}

\part{Conceptual Synthesis and The "Big Picture"}
\section{Thematic Connections}
The central theme of this topic is \textbf{convergence through cancellation}. Previously, we studied series with positive terms, where convergence was a race: the terms had to shrink to zero "fast enough" to produce a finite sum. If they decreased too slowly (like the harmonic series $\sum 1/n$), the sum would diverge.

Alternating series introduce a new mechanism for convergence. The alternating signs provide a "braking" force. Even if the magnitudes of the terms decrease very slowly, the constant back-and-forth cancellation can be enough to force the partial sums to settle on a finite value. The convergence of the alternating harmonic series $\sum (-1)^{n-1}/n$ is the quintessential example of this principle. It shows that the \emph{structure} and \emph{arrangement} of terms can be just as important as their magnitude.

\section{Forward and Backward Links}
\begin{itemize}
    \item \textbf{Backward Link:} This chapter is the culmination of everything we have learned about series so far. It forces us to synthesize our knowledge. The tests for positive-termed series (p-series, LCT, DCT, etc.) are no longer isolated topics; they are now the essential tools used in the first step of analyzing any alternating series—the test for absolute convergence. Without a firm grasp of the previous chapter, this one is impossible.
    \item \textbf{Forward Link:} The concepts of absolute convergence and the AST are foundational for the study of \textbf{Power Series} and \textbf{Taylor Series}. When we represent a function like $f(x) = \sin(x)$ as an infinite series, we must determine its \textbf{interval of convergence}—the set of $x$-values for which the series converges. We typically use the Ratio Test to find this interval, but the Ratio Test is inconclusive at the interval's endpoints. To determine convergence at the endpoints, we often must use other tests, and frequently the series at an endpoint turns out to be an alternating series, requiring the AST. Understanding conditional convergence is key to fully characterizing the behavior of these powerful function representations.
\end{itemize}

\part{Real-World Application and Modeling}
\section{Concrete Scenarios in Finance and Economics}
\begin{enumerate}
    \item \textbf{Derivative Pricing and Hedging:} In financial engineering, complex derivatives are often priced using models that involve summing expected future payoffs. In some advanced models (like certain stochastic volatility models), the terms used to calculate the price or a hedging parameter (a "Greek") can form an alternating series. The convergence of this series is paramount; a divergent series implies the price is infinite or undefined, meaning the model is unstable and cannot be used for pricing or risk management.
    \item \textbf{Economic Shock Persistence (Time Series Analysis):} In econometrics, an autoregressive model might describe a country's GDP as a function of its past values. An external shock (like a policy change or natural disaster) creates an effect that ripples through time. The formula for the total long-term impact of this shock can be an infinite series. If the model has coefficients that cause this series to alternate, the AST can determine if the shock's effect eventually dies out (convergence) or if it causes ever-widening oscillations that destabilize the economy (divergence).
    \item \textbf{Net Present Value (NPV) of Complex Projects:} Standard NPV calculations sum discounted future cash flows. Consider a large-scale infrastructure project with initial investment costs (negative flow), followed by years of revenue (positive flows), but then requiring a massive, expensive decommissioning at the end of its life (a large negative flow). The stream of cash flows may not be strictly alternating, but the concept of balancing positive and negative terms to see if the overall value converges to a profitable number is the same. Conditional convergence could be likened to a project that is profitable, but so sensitive that a minor change in the timing or magnitude of cash flows could render it unprofitable.
\end{enumerate}

\section{Model Problem Setup}
\textbf{Scenario:} An analyst is valuing a financial instrument that pays out based on the performance of an asset that follows a mean-reverting process. The model for the instrument's value simplifies to an infinite series representing expected payments at future time steps.

\textbf{Problem:} The value, $V$, of the instrument is given by the series:
\[ V = \sum_{n=1}^{\infty} (-1)^{n+1} \frac{100 \cdot \ln(n+1)}{n^2} \]
The terms alternate because the underlying asset is expected to oscillate around a long-term mean. The analyst needs to determine if the instrument has a finite, well-defined value.

\textbf{Model Setup:}
\begin{itemize}
    \item \textbf{Variables:} $n$ is the time step (e.g., year). The term $a_n = (-1)^{n+1} \frac{100 \ln(n+1)}{n^2}$ is the discounted expected payoff at time $n$.
    \item \textbf{Function/Equation:} The total value is the sum of the series $V = \sum a_n$.
    \item \textbf{Analysis to be Performed:} To determine if $V$ is finite, we must test the series for convergence. We would test for absolute convergence first:
    \[ \sum_{n=1}^{\infty} |a_n| = \sum_{n=1}^{\infty} \frac{100 \ln(n+1)}{n^2} \]
    This series can be tested using the Integral Test or by comparing it to a convergent p-series like $\sum 1/n^{1.5}$. Since $\ln(n+1)$ grows slower than any power of $n$, this series of absolute values will converge. Therefore, the original alternating series is absolutely convergent, and the financial instrument has a stable, finite value.
\end{itemize}

\part{Common Variations and Untested Concepts}
The homework set provides a solid foundation but omits two of the most powerful tools for testing absolute convergence, and a key theoretical concept.

\section{The Ratio Test}
This is the most important test not featured in the homework. It is especially useful for series involving factorials ($n!$) or exponentials ($k^n$).
\textbf{Theorem:} For a series $\sum a_n$, let $L = \lim_{n \to \infty} \left| \frac{a_{n+1}}{a_n} \right|$.
\begin{itemize}
    \item If $L < 1$, the series is absolutely convergent.
    \item If $L > 1$ or $L=\infty$, the series is divergent.
    \item If $L = 1$, the test is inconclusive.
\end{itemize}
\textbf{Worked Example:} Determine if $\sum_{n=1}^{\infty} \frac{(-1)^n n^3}{3^n}$ is absolutely convergent.
\begin{align*}
L &= \lim_{n \to \infty} \left| \frac{(-1)^{n+1}(n+1)^3}{3^{n+1}} \cdot \frac{3^n}{(-1)^n n^3} \right| \\
&= \lim_{n \to \infty} \left| \frac{(n+1)^3}{n^3} \cdot \frac{3^n}{3^{n+1}} \right| \\
&= \lim_{n \to \infty} \left( \frac{n+1}{n} \right)^3 \cdot \frac{1}{3} \\
&= \left( \lim_{n \to \infty} 1 + \frac{1}{n} \right)^3 \cdot \frac{1}{3} = (1)^3 \cdot \frac{1}{3} = \frac{1}{3}
\end{align*}
Since $L=1/3 < 1$, the series is \textbf{absolutely convergent}.

\section{The Root Test}
The Root Test is useful for series where the entire term is raised to the $n$-th power.
\textbf{Theorem:} For a series $\sum a_n$, let $L = \lim_{n \to \infty} \sqrt[n]{|a_n|}$. The conclusions are the same as for the Ratio Test.
\textbf{Worked Example:} Determine if $\sum_{n=1}^{\infty} \left( \frac{-5n}{2n+3} \right)^n$ is absolutely convergent.
\begin{align*}
L &= \lim_{n \to \infty} \sqrt[n]{\left| \left( \frac{-5n}{2n+3} \right)^n \right|} \\
&= \lim_{n \to \infty} \left| \frac{-5n}{2n+3} \right| = \lim_{n \to \infty} \frac{5n}{2n+3} \\
&= \lim_{n \to \infty} \frac{5}{2+3/n} = \frac{5}{2}
\end{align*}
Since $L=5/2 > 1$, the series is \textbf{divergent}.

\section{Rearrangement of Series}
A profound concept not tested is the effect of rearranging terms.
\begin{itemize}
    \item \textbf{Absolutely Convergent Series:} If a series is absolutely convergent, its terms can be rearranged in any order, and the new series will still converge to the \emph{same sum}.
    \item \textbf{Conditionally Convergent Series (Riemann Series Theorem):} If a series is conditionally convergent, its terms can be rearranged to make the new series converge to \textbf{any real number you desire}, or even to make it diverge. This highlights the delicate "balancing act" of conditional convergence.
\end{itemize}

\part{Advanced Diagnostic Testing: "Find the Flaw"}
The following five problems have solutions that contain a single, subtle error. Your task is to find the flaw, explain it, and provide the correct solution.

\section*{Problem A}
\textbf{Problem:} Test the series $\sum_{n=1}^{\infty} \frac{(-10)^n}{n!}$ for convergence.

\textbf{Flawed Solution:}
We use the Ratio Test. Let $a_n = \frac{(-10)^n}{n!}$.
\begin{align*}
L &= \lim_{n \to \infty} \frac{a_{n+1}}{a_n} = \lim_{n \to \infty} \frac{(-10)^{n+1}}{(n+1)!} \cdot \frac{n!}{(-10)^n} \\
&= \lim_{n \to \infty} \frac{-10 \cdot (-10)^n}{(n+1) \cdot n!} \cdot \frac{n!}{(-10)^n} \\
&= \lim_{n \to \infty} \frac{-10}{n+1} = 0
\end{align*}
Since $L=0 < 1$, the series converges.

---

\section*{Problem B}
\textbf{Problem:} Determine if $\sum_{n=1}^{\infty} \frac{(-1)^n}{\sqrt{n}+1}$ is absolutely or conditionally convergent.

\textbf{Flawed Solution:}
First, we test for absolute convergence by examining $\sum_{n=1}^{\infty} \frac{1}{\sqrt{n}+1}$. We use the Limit Comparison Test with the convergent p-series $\sum \frac{1}{n^2}$.
\[ L = \lim_{n \to \infty} \frac{1/(\sqrt{n}+1)}{1/n^2} = \lim_{n \to \infty} \frac{n^2}{\sqrt{n}+1} = \infty \]
Since the limit is infinity and $\sum \frac{1}{n^2}$ converges, our series $\sum \frac{1}{\sqrt{n}+1}$ must also converge. Therefore, the original series is absolutely convergent.

---

\section*{Problem C}
\textbf{Problem:} Test the series $\sum_{n=1}^{\infty} (-1)^n \sin\left(\frac{1}{n}\right)$ for convergence.

\textbf{Flawed Solution:}
This is an alternating series with $b_n = \sin(1/n)$. We check the two conditions of the AST.
\begin{enumerate}
    \item As $n \to \infty$, $1/n \to 0$, so $\lim_{n \to \infty} \sin(1/n) = \sin(0) = 0$.
\end{enumerate}
Since the first condition is met, the series converges by the Alternating Series Test.

---

\section*{Problem D}
\textbf{Problem:} Test the series $\sum_{n=1}^{\infty} \frac{(-1)^n n}{e^n}$ for convergence.

\textbf{Flawed Solution:}
This is an alternating series with $b_n = n/e^n$.
The limit $\lim_{n \to \infty} n/e^n = 0$ by L'Hopital's Rule.
To check if $b_n$ is decreasing, let $f(x) = x/e^x$. Then $f'(x) = \frac{e^x(1) - x(e^x)}{(e^x)^2} = \frac{e^x(1-x)}{(e^x)^2} = \frac{x-1}{e^x}$.
For $x>1$, $f'(x) > 0$, so the function is increasing. Since the decreasing condition of the AST fails, the series diverges.

---

\section*{Problem E}
\textbf{Problem:} Determine if $\sum_{n=1}^{\infty} \frac{(-1)^n (n+1)}{n^2+1}$ is absolutely or conditionally convergent.

\textbf{Flawed Solution:}
We test for absolute convergence with the Ratio Test. $|a_n| = \frac{n+1}{n^2+1}$.
\begin{align*}
L &= \lim_{n \to \infty} \frac{(n+1)+1}{(n+1)^2+1} \cdot \frac{n^2+1}{n+1} \\
&= \lim_{n \to \infty} \frac{n+2}{n^2+2n+2} \cdot \frac{n^2+1}{n+1} \\
&= \lim_{n \to \infty} \frac{n^3+2n^2+n+2}{n^3+3n^2+4n+2} = 1
\end{align*}
Since $L=1$, the Ratio Test indicates that the series diverges. Therefore, the series is divergent.

\newpage
\subsection*{Solutions and Explanations for "Find the Flaw"}

\textbf{Problem A Flaw:}
\begin{itemize}
    \item \textbf{The Flaw:} The Ratio Test requires taking the limit of the absolute value of the ratio, $|a_{n+1}/a_n|$. The flawed solution calculated $\lim (a_{n+1}/a_n)$ instead.
    \item \textbf{Correction:} $L = \lim_{n \to \infty} \left| \frac{(-10)^{n+1}}{(n+1)!} \cdot \frac{n!}{(-10)^n} \right| = \lim_{n \to \infty} \left| \frac{-10}{n+1} \right| = \lim_{n \to \infty} \frac{10}{n+1} = 0$.
    \item \textbf{Final Answer:} Since $L=0<1$, the series is absolutely convergent.
\end{itemize}

\textbf{Problem B Flaw:}
\begin{itemize}
    \item \textbf{The Flaw:} The conclusion drawn from the Limit Comparison Test is incorrect. When $L=\infty$, the LCT states that if the comparison series (here $\sum 1/n^2$) converges, then our series *diverges*. The logic used was backward.
    \item \textbf{Correction:} A better comparison series is the divergent p-series $\sum 1/\sqrt{n}$.
    \[ L = \lim_{n \to \infty} \frac{1/(\sqrt{n}+1)}{1/\sqrt{n}} = \lim_{n \to \infty} \frac{\sqrt{n}}{\sqrt{n}+1} = 1 \]
    Since $L=1$ and $\sum 1/\sqrt{n}$ diverges, $\sum 1/(\sqrt{n}+1)$ also diverges. So the series is not absolutely convergent. Now check AST: $\lim 1/(\sqrt{n}+1) = 0$ and the terms are decreasing. The AST passes.
    \item \textbf{Final Answer:} The series is conditionally convergent.
\end{itemize}

\textbf{Problem C Flaw:}
\begin{itemize}
    \item \textbf{The Flaw:} The solution only checked one of the two required conditions for the Alternating Series Test. It found that $\lim b_n = 0$ but never established that $b_n$ is a decreasing sequence.
    \item \textbf{Correction:} We must also show $b_n = \sin(1/n)$ is decreasing. For $n \ge 1$, $1/n$ is in the interval $(0, 1]$. In this interval, $\sin(x)$ is an increasing function. Since $1/n$ is a decreasing sequence, applying an increasing function to it preserves the decreasing nature. Thus $b_n = \sin(1/n)$ is decreasing. Since both AST conditions hold, the series converges.
    \item \textbf{Final Answer:} The series converges.
\end{itemize}

\textbf{Problem D Flaw:}
\begin{itemize}
    \item \textbf{The Flaw:} There is a simple but critical algebraic error in the derivative calculation.
    \item \textbf{Correction:} $f'(x) = \frac{e^x(1) - x(e^x)}{(e^x)^2} = \frac{e^x(1-x)}{(e^x)^2} = \frac{1-x}{e^x}$. The numerator of the final fraction was mistakenly written as $x-1$. For $x>1$, the correct numerator $(1-x)$ is negative. Thus, $f'(x) < 0$ for $x>1$, and the sequence is decreasing.
    \item \textbf{Final Answer:} Since both conditions of the AST are met (limit is 0 and terms are decreasing), the series converges.
\end{itemize}

\textbf{Problem E Flaw:}
\begin{itemize}
    \item \textbf{The Flaw:} The conclusion "Since $L=1$, the Ratio Test indicates that the series diverges" is incorrect. When $L=1$, the Ratio Test is \textbf{inconclusive}. Another test must be used.
    \item \textbf{Correction:} Since the Ratio Test failed for absolute convergence, we must use a different test. Let's use LCT on $\sum |a_n|$ with the divergent harmonic series $\sum 1/n$.
    \[ L_{LCT} = \lim_{n \to \infty} \frac{(n+1)/(n^2+1)}{1/n} = \lim_{n \to \infty} \frac{n(n+1)}{n^2+1} = \lim_{n \to \infty} \frac{n^2+n}{n^2+1} = 1 \]
    Since the limit is 1 and $\sum 1/n$ diverges, $\sum |a_n|$ diverges. The series is not absolutely convergent. Now, we apply the AST to the original series. The limit of $b_n = (n+1)/(n^2+1)$ is 0, and a derivative test would show it is decreasing. Thus, the original series converges.
    \item \textbf{Final Answer:} The series is conditionally convergent.
\end{itemize}

\end{document}