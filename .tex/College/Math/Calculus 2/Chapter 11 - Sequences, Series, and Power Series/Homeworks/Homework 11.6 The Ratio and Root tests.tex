\documentclass{article}
\usepackage{amsmath}
\usepackage{amssymb}
\usepackage{geometry}
\geometry{a4paper, margin=1in}

\title{Homework 11.6 The Ratio and Root tests}
\author{Tashfeen Omran}
\date{November 2025}

\begin{document}

\maketitle

\part{Comprehensive Introduction, Context, and Prerequisites}

\section{Core Concepts}
The Ratio and Root tests are powerful tools in calculus used to determine the convergence or divergence of an infinite series, particularly its \textit{absolute convergence}. They are especially effective for series involving factorials (like $n!$) or $n$-th powers.

The core idea behind both tests is to compare the given series $\sum a_n$ to a geometric series. We examine the behavior of the terms of the series as $n$ approaches infinity.

\begin{itemize}
    \item \textbf{The Ratio Test} looks at the limit of the ratio of consecutive terms, $L = \lim_{n \to \infty} \left| \frac{a_{n+1}}{a_n} \right|$. This limit, $L$, can be thought of as the "eventual" common ratio of the series.
    \item \textbf{The Root Test} looks at the limit of the $n$-th root of the absolute value of the terms, $L = \lim_{n \to \infty} \sqrt[n]{|a_n|}$. This limit, $L$, also acts like an "eventual" common ratio.
\end{itemize}

For both tests, the conclusion is drawn based on the value of $L$:
\begin{enumerate}
    \item If $L < 1$, the series is \textbf{absolutely convergent}. This means the series of absolute values, $\sum |a_n|$, converges. Any absolutely convergent series is also convergent.
    \item If $L > 1$ (or $L = \infty$), the series is \textbf{divergent}. The terms of the series are growing too quickly for the sum to be finite.
    \item If $L = 1$, the test is \textbf{inconclusive}. The series could be convergent, conditionally convergent, or divergent. Another test (like the Integral Test, Comparison Test, or Alternating Series Test) must be used.
\end{enumerate}

\section{Intuition and Derivation}
The power of these tests comes from their direct connection to the geometric series, $\sum_{n=0}^{\infty} ar^n$, which converges if $|r| < 1$ and diverges if $|r| \ge 1$.

\textbf{Ratio Test Intuition:} If the limit $\lim_{n \to \infty} \left| \frac{a_{n+1}}{a_n} \right| = L$ exists, then for very large $n$, we have $|a_{n+1}| \approx L|a_n|$. This implies that the tail end of the series behaves like a geometric series with a common ratio of $L$.
\begin{itemize}
    \item If $L < 1$, the terms are decreasing fast enough (like in a convergent geometric series) for the sum to be finite.
    \item If $L > 1$, the terms are eventually increasing, so the sum must be infinite. The Test for Divergence also confirms this, as the terms don't go to zero.
\end{itemize}

\textbf{Root Test Intuition:} If the limit $\lim_{n \to \infty} \sqrt[n]{|a_n|} = L$ exists, then for very large $n$, we have $\sqrt[n]{|a_n|} \approx L$, which means $|a_n| \approx L^n$. So, the series $\sum |a_n|$ behaves like the geometric series $\sum L^n$.
\begin{itemize}
    \item If $L < 1$, the series behaves like a convergent geometric series.
    \item If $L > 1$, the series behaves like a divergent geometric series.
\end{itemize}

\section{Historical Context and Motivation}
The formal study of infinite series blossomed in the 18th and 19th centuries as mathematicians sought to put calculus on a more rigorous footing and to represent functions as infinite polynomial series (power series). Jean le Rond d'Alembert first published the Ratio Test in 1768. Later, Augustin-Louis Cauchy, a central figure in the rigorous development of analysis, formalized the Root Test in his influential 1821 textbook \textit{Cours d'Analyse}.

The motivation was to create more general convergence tests that did not require clever comparisons to known series or the evaluation of difficult integrals. The Ratio and Root tests were developed to handle a wider class of series, especially those arising from Taylor series expansions of functions like $e^x = \sum \frac{x^n}{n!}$ or $\sin(x) = \sum (-1)^n \frac{x^{2n+1}}{(2n+1)!}$. These series involve factorials and powers, for which the Ratio and Root tests are perfectly suited and are essential for determining the \textit{radius of convergence} of such power series.

\section{Key Formulas}
Let $\sum a_n$ be an infinite series.

\subsubsection{The Ratio Test}
Calculate the limit $L$:
\[ L = \lim_{n \to \infty} \left| \frac{a_{n+1}}{a_n} \right| \]
\begin{itemize}
    \item If $L < 1$, the series is absolutely convergent.
    \item If $L > 1$ or $L = \infty$, the series is divergent.
    \item If $L = 1$, the test is inconclusive.
\end{itemize}

\subsubsection{The Root Test}
Calculate the limit $L$:
\[ L = \lim_{n \to \infty} \sqrt[n]{|a_n|} = \lim_{n \to \infty} |a_n|^{1/n} \]
\begin{itemize}
    \item If $L < 1$, the series is absolutely convergent.
    \item If $L > 1$ or $L = \infty$, the series is divergent.
    \item If $L = 1$, the test is inconclusive.
\end{itemize}

\section{Prerequisites}
To master the Ratio and Root tests, you must be proficient in the following:
\begin{itemize}
    \item \textbf{Algebra:} Simplifying complex fractions, properties of exponents and absolute values, and especially simplification of expressions involving factorials (e.g., $\frac{k!}{(k+1)!} = \frac{1}{k+1}$).
    \item \textbf{Limits:} Evaluating limits at infinity, especially of rational functions. Knowledge of L'Hôpital's Rule is often necessary. You must also know key limits like $\lim_{n \to \infty} (1 + \frac{1}{n})^n = e$.
    \item \textbf{Calculus Concepts:} A firm grasp of what an infinite series is and the meaning of convergence, divergence, and absolute convergence. Familiarity with other convergence tests (like for p-series and geometric series) is crucial for cases where the Ratio/Root test is inconclusive.
\end{itemize}

\part{Detailed Homework Solutions}

\section{Problem 1}
What can you say about the series $\sum a_n$ in each of the following cases?

\subsection*{(a) $\lim_{n \to \infty} \left|\frac{a_{n+1}}{a_n}\right| = 2$}
\textbf{Solution:}
The limit is $L=2$. According to the Ratio Test, if $L > 1$, the series is divergent.
\begin{itemize}
    \item \textbf{Answer:} divergent
\end{itemize}

\subsection*{(b) $\lim_{n \to \infty} \left|\frac{a_{n+1}}{a_n}\right| = 0.4$}
\textbf{Solution:}
The limit is $L=0.4$. According to the Ratio Test, if $L < 1$, the series is absolutely convergent.
\begin{itemize}
    \item \textbf{Answer:} absolutely convergent
\end{itemize}

\subsection*{(c) $\lim_{n \to \infty} \left|\frac{a_{n+1}}{a_n}\right| = 1$}
\textbf{Solution:}
The limit is $L=1$. According to the Ratio Test, if $L=1$, the test is inconclusive. We cannot determine if the series converges or diverges from this information alone.
\begin{itemize}
    \item \textbf{Answer:} cannot be determined
\end{itemize}

\section{Problem 2}
Use the Ratio Test to determine whether the series $\sum_{n=1}^{\infty} \frac{n}{6^n}$ is convergent or divergent.

\textbf{Solution:}
\begin{enumerate}
    \item \textbf{Identify $a_n$:} $a_n = \frac{n}{6^n}$.
    \item \textbf{Set up the Ratio Test limit:} We need to evaluate $L = \lim_{n \to \infty} \left| \frac{a_{n+1}}{a_n} \right|$.
        \[ a_{n+1} = \frac{n+1}{6^{n+1}} \]
        \[ L = \lim_{n \to \infty} \left| \frac{\frac{n+1}{6^{n+1}}}{\frac{n}{6^n}} \right| \]
    \item \textbf{Evaluate the limit:}
        \begin{align*}
            L &= \lim_{n \to \infty} \left| \frac{n+1}{6^{n+1}} \cdot \frac{6^n}{n} \right| \\
            &= \lim_{n \to \infty} \left| \frac{n+1}{n} \cdot \frac{6^n}{6^{n+1}} \right| \\
            &= \lim_{n \to \infty} \left| \left(1 + \frac{1}{n}\right) \cdot \frac{1}{6} \right| \\
            &= \left| (1 + 0) \cdot \frac{1}{6} \right| = \frac{1}{6}
        \end{align*}
    \item \textbf{Conclusion:} Since $L = \frac{1}{6} < 1$, the series is absolutely convergent.
\end{enumerate}
\begin{itemize}
    \item \textbf{Final Answer:} The series is convergent.
\end{itemize}

\section{Problem 3}
Use the Ratio Test to determine whether the series $\sum_{n=1}^{\infty} \frac{(-3)^n}{n^2}$ is convergent or divergent.

\textbf{Solution:}
\begin{enumerate}
    \item \textbf{Identify $a_n$:} $a_n = \frac{(-3)^n}{n^2}$.
    \item \textbf{Set up the Ratio Test limit:}
        \[ a_{n+1} = \frac{(-3)^{n+1}}{(n+1)^2} \]
        \[ L = \lim_{n \to \infty} \left| \frac{\frac{(-3)^{n+1}}{(n+1)^2}}{\frac{(-3)^n}{n^2}} \right| \]
    \item \textbf{Evaluate the limit:}
        \begin{align*}
            L &= \lim_{n \to \infty} \left| \frac{(-3)^{n+1}}{(n+1)^2} \cdot \frac{n^2}{(-3)^n} \right| \\
            &= \lim_{n \to \infty} \left| \frac{(-3)}{1} \cdot \frac{n^2}{(n+1)^2} \right| \\
            &= \lim_{n \to \infty} \left| -3 \left(\frac{n}{n+1}\right)^2 \right| \\
            &= 3 \cdot \left(\lim_{n \to \infty} \frac{n}{n+1}\right)^2 \\
            &= 3 \cdot \left(\lim_{n \to \infty} \frac{1}{1+1/n}\right)^2 = 3 \cdot (1)^2 = 3
        \end{align*}
    \item \textbf{Conclusion:} Since $L=3 > 1$, the series is divergent.
\end{enumerate}
\begin{itemize}
    \item \textbf{Final Answer:} The series is divergent.
\end{itemize}

\section{Problem 4}
Use the Ratio Test to determine whether the series $\sum_{n=1}^{\infty} (-1)^{n-1} \frac{7^n}{5^n n^3}$ is convergent or divergent.

\textbf{Solution:}
\begin{enumerate}
    \item \textbf{Identify $a_n$:} $a_n = (-1)^{n-1} \frac{7^n}{5^n n^3}$.
    \item \textbf{Set up the Ratio Test limit:}
        \[ a_{n+1} = (-1)^{n} \frac{7^{n+1}}{5^{n+1} (n+1)^3} \]
        \[ L = \lim_{n \to \infty} \left| \frac{(-1)^{n} \frac{7^{n+1}}{5^{n+1} (n+1)^3}}{(-1)^{n-1} \frac{7^n}{5^n n^3}} \right| \]
    \item \textbf{Evaluate the limit:}
        \begin{align*}
            L &= \lim_{n \to \infty} \left| \frac{7^{n+1}}{5^{n+1}(n+1)^3} \cdot \frac{5^n n^3}{7^n} \right| \quad (\text{absolute value removes } (-1)^n) \\
            &= \lim_{n \to \infty} \left| \frac{7}{5} \cdot \frac{n^3}{(n+1)^3} \right| \\
            &= \frac{7}{5} \lim_{n \to \infty} \left(\frac{n}{n+1}\right)^3 \\
            &= \frac{7}{5} \left(\lim_{n \to \infty} \frac{1}{1+1/n}\right)^3 = \frac{7}{5} \cdot (1)^3 = \frac{7}{5}
        \end{align*}
    \item \textbf{Conclusion:} Since $L=\frac{7}{5} > 1$, the series is divergent.
\end{enumerate}
\begin{itemize}
    \item \textbf{Final Answer:} The series is divergent.
\end{itemize}

\section{Problem 5}
Use the Ratio Test to determine whether the series $\sum_{k=1}^{\infty} \frac{4}{k!}$ is convergent or divergent.

\textbf{Solution:}
\begin{enumerate}
    \item \textbf{Identify $a_k$:} $a_k = \frac{4}{k!}$.
    \item \textbf{Set up the Ratio Test limit:}
        \[ a_{k+1} = \frac{4}{(k+1)!} \]
        \[ L = \lim_{k \to \infty} \left| \frac{\frac{4}{(k+1)!}}{\frac{4}{k!}} \right| \]
    \item \textbf{Evaluate the limit:}
        \begin{align*}
            L &= \lim_{k \to \infty} \left| \frac{4}{(k+1)!} \cdot \frac{k!}{4} \right| \\
            &= \lim_{k \to \infty} \left| \frac{k!}{(k+1) \cdot k!} \right| \\
            &= \lim_{k \to \infty} \frac{1}{k+1} = 0
        \end{align*}
    \item \textbf{Conclusion:} Since $L=0 < 1$, the series is absolutely convergent.
\end{enumerate}
\begin{itemize}
    \item \textbf{Final Answer:} The series is convergent.
\end{itemize}

\section{Problem 6}
Use the Ratio Test to determine whether the series $\sum_{k=1}^{\infty} 5ke^{-k}$ is convergent or divergent.

\textbf{Solution:}
\begin{enumerate}
    \item \textbf{Identify $a_k$:} $a_k = 5ke^{-k} = \frac{5k}{e^k}$.
    \item \textbf{Set up the Ratio Test limit:}
        \[ a_{k+1} = 5(k+1)e^{-(k+1)} = \frac{5(k+1)}{e^{k+1}} \]
        \[ L = \lim_{k \to \infty} \left| \frac{\frac{5(k+1)}{e^{k+1}}}{\frac{5k}{e^k}} \right| \]
    \item \textbf{Evaluate the limit:}
        \begin{align*}
            L &= \lim_{k \to \infty} \left| \frac{5(k+1)}{e^{k+1}} \cdot \frac{e^k}{5k} \right| \\
            &= \lim_{k \to \infty} \left| \frac{k+1}{k} \cdot \frac{e^k}{e^{k+1}} \right| \\
            &= \lim_{k \to \infty} \left(1 + \frac{1}{k}\right) \cdot \frac{1}{e} \\
            &= (1+0) \cdot \frac{1}{e} = \frac{1}{e}
        \end{align*}
    \item \textbf{Conclusion:} Since $L=\frac{1}{e} \approx \frac{1}{2.718} < 1$, the series is absolutely convergent.
\end{enumerate}
\begin{itemize}
    \item \textbf{Final Answer:} The series is convergent.
\end{itemize}

\section{Problem 7}
Use the Ratio Test to determine whether the series $\sum_{n=1}^{\infty} \frac{n\pi^n}{(-6)^n - 1}$ is convergent or divergent.

\textbf{Solution:}
\begin{enumerate}
    \item \textbf{Identify $a_n$:} $a_n = \frac{n\pi^n}{(-6)^n - 1}$.
    \item \textbf{Set up the Ratio Test limit:}
        \[ a_{n+1} = \frac{(n+1)\pi^{n+1}}{(-6)^{n+1} - 1} \]
        \[ L = \lim_{n \to \infty} \left| \frac{\frac{(n+1)\pi^{n+1}}{(-6)^{n+1} - 1}}{\frac{n\pi^n}{(-6)^n - 1}} \right| \]
    \item \textbf{Evaluate the limit:}
        \begin{align*}
            L &= \lim_{n \to \infty} \left| \frac{(n+1)\pi^{n+1}}{(-6)^{n+1} - 1} \cdot \frac{(-6)^n - 1}{n\pi^n} \right| \\
            &= \lim_{n \to \infty} \left| \frac{n+1}{n} \cdot \frac{\pi^{n+1}}{\pi^n} \cdot \frac{(-6)^n - 1}{(-6)^{n+1} - 1} \right| \\
            &= \pi \lim_{n \to \infty} \left| \left(1+\frac{1}{n}\right) \cdot \frac{(-6)^n - 1}{-6(-6)^n - 1} \right| \\
            &= \pi \lim_{n \to \infty} \left| (1+0) \cdot \frac{(-6)^n(1 - 1/(-6)^n)}{(-6)^n(-6 - 1/(-6)^n)} \right| \quad (\text{Divide by } (-6)^n) \\
            &= \pi \lim_{n \to \infty} \left| \frac{1 - 0}{-6 - 0} \right| = \pi \left|-\frac{1}{6}\right| = \frac{\pi}{6}
        \end{align*}
    \item \textbf{Conclusion:} Since $L=\frac{\pi}{6} \approx \frac{3.14159}{6} < 1$, the series is absolutely convergent.
\end{enumerate}
\begin{itemize}
    \item \textbf{Final Answer:} The series is convergent.
\end{itemize}

\section{Problem 8}
Use the Ratio Test to determine whether the series $\sum_{n=1}^{\infty} \frac{\cos(n\pi/7)}{n!}$ is convergent or divergent.

\textbf{Solution:}
\begin{enumerate}
    \item \textbf{Identify $a_n$:} $a_n = \frac{\cos(n\pi/7)}{n!}$.
    \item \textbf{Set up the Ratio Test limit:}
        \[ a_{n+1} = \frac{\cos((n+1)\pi/7)}{(n+1)!} \]
        \[ L = \lim_{n \to \infty} \left| \frac{\frac{\cos((n+1)\pi/7)}{(n+1)!}}{\frac{\cos(n\pi/7)}{n!}} \right| \]
    \item \textbf{Evaluate the limit:}
        \begin{align*}
            L &= \lim_{n \to \infty} \left| \frac{\cos((n+1)\pi/7)}{(n+1)!} \cdot \frac{n!}{\cos(n\pi/7)} \right| \\
            &= \lim_{n \to \infty} \left| \frac{\cos((n+1)\pi/7)}{\cos(n\pi/7)} \cdot \frac{n!}{(n+1)n!} \right| \\
            &= \lim_{n \to \infty} \left| \frac{\cos((n+1)\pi/7)}{\cos(n\pi/7)} \cdot \frac{1}{n+1} \right|
        \end{align*}
        The term $|\cos((n+1)\pi/7)/\cos(n\pi/7)|$ is bounded (it oscillates but doesn't grow to infinity). However, the term $\frac{1}{n+1}$ goes to 0. By the Squeeze Theorem, a bounded value times zero is zero.
        \[ L = 0 \]
    \item \textbf{Conclusion:} Since $L=0 < 1$, the series is absolutely convergent.
\end{enumerate}
\begin{itemize}
    \item \textbf{Final Answer:} The series is convergent.
\end{itemize}

\section{Problem 9}
Use the Ratio Test to determine whether the series $\sum_{n=1}^{\infty} \frac{n^{500} 500^n}{n!}$ is convergent or divergent.

\textbf{Solution:}
\begin{enumerate}
    \item \textbf{Identify $a_n$:} $a_n = \frac{n^{500} 500^n}{n!}$.
    \item \textbf{Set up the Ratio Test limit:}
        \[ a_{n+1} = \frac{(n+1)^{500} 500^{n+1}}{(n+1)!} \]
        \[ L = \lim_{n \to \infty} \left| \frac{\frac{(n+1)^{500} 500^{n+1}}{(n+1)!}}{\frac{n^{500} 500^n}{n!}} \right| \]
    \item \textbf{Evaluate the limit:}
        \begin{align*}
            L &= \lim_{n \to \infty} \left| \frac{(n+1)^{500} 500^{n+1}}{(n+1)!} \cdot \frac{n!}{n^{500} 500^n} \right| \\
            &= \lim_{n \to \infty} \left| \frac{(n+1)^{500}}{n^{500}} \cdot \frac{500^{n+1}}{500^n} \cdot \frac{n!}{(n+1)!} \right| \\
            &= \lim_{n \to \infty} \left| \left(\frac{n+1}{n}\right)^{500} \cdot 500 \cdot \frac{1}{n+1} \right| \\
            &= \lim_{n \to \infty} \left(1 + \frac{1}{n}\right)^{500} \frac{500}{n+1} \\
            &= (1)^{500} \cdot \lim_{n \to \infty} \frac{500}{n+1} = 1 \cdot 0 = 0
        \end{align*}
    \item \textbf{Conclusion:} Since $L=0 < 1$, the series is absolutely convergent. This shows that factorials grow much faster than any exponential or polynomial term.
\end{enumerate}
\begin{itemize}
    \item \textbf{Final Answer:} The series is convergent.
\end{itemize}

\section{Problem 10}
Use the Ratio Test to determine whether the series $1 - \frac{2!}{1 \cdot 3} + \frac{3!}{1 \cdot 3 \cdot 5} - \frac{4!}{1 \cdot 3 \cdot 5 \cdot 7} + \dots$ is convergent or divergent.

\textbf{Solution:}
\begin{enumerate}
    \item \textbf{Identify $a_n$:} The general term is $a_n = (-1)^{n-1} \frac{n!}{1 \cdot 3 \cdot 5 \cdot \dots \cdot (2n-1)}$.
    \item \textbf{Set up the Ratio Test limit:}
        \[ a_{n+1} = (-1)^{n} \frac{(n+1)!}{1 \cdot 3 \cdot 5 \cdot \dots \cdot (2n-1) \cdot (2(n+1)-1)} = (-1)^{n} \frac{(n+1)!}{1 \cdot 3 \cdot \dots \cdot (2n-1)(2n+1)} \]
        \[ L = \lim_{n \to \infty} \left| \frac{a_{n+1}}{a_n} \right| \]
    \item \textbf{Evaluate the limit:}
        \begin{align*}
            L &= \lim_{n \to \infty} \left| \frac{(n+1)!}{1 \cdot 3 \cdot \dots \cdot (2n+1)} \cdot \frac{1 \cdot 3 \cdot \dots \cdot (2n-1)}{n!} \right| \\
            &= \lim_{n \to \infty} \left| \frac{(n+1)!}{n!} \cdot \frac{1 \cdot 3 \cdot \dots \cdot (2n-1)}{1 \cdot 3 \cdot \dots \cdot (2n-1)(2n+1)} \right| \\
            &= \lim_{n \to \infty} \left| (n+1) \cdot \frac{1}{2n+1} \right| \\
            &= \lim_{n \to \infty} \frac{n+1}{2n+1} = \lim_{n \to \infty} \frac{1+1/n}{2+1/n} = \frac{1}{2}
        \end{align*}
    \item \textbf{Conclusion:} Since $L=\frac{1}{2} < 1$, the series is absolutely convergent.
\end{enumerate}
\begin{itemize}
    \item \textbf{Final Answer:} The series is convergent.
\end{itemize}

\section{Problem 11}
Use the Ratio Test to determine whether the series $\sum_{n=1}^{\infty} \frac{2 \cdot 4 \cdot 6 \cdot \dots \cdot (2n)}{n!}$ is convergent or divergent.

\textbf{Solution:}
\begin{enumerate}
    \item \textbf{Identify $a_n$:} $a_n = \frac{2 \cdot 4 \cdot 6 \cdot \dots \cdot (2n)}{n!}$. We can rewrite the numerator as $2^n(1 \cdot 2 \cdot 3 \cdot \dots \cdot n) = 2^n n!$.
        \[ a_n = \frac{2^n n!}{n!} = 2^n \]
        The series is $\sum_{n=1}^{\infty} 2^n$, which is a geometric series with ratio $r=2$. Since $|r| \ge 1$, it diverges. Let's confirm with the Ratio Test.
    \item \textbf{Set up the Ratio Test limit:}
        \[ a_{n+1} = \frac{2 \cdot 4 \cdot \dots \cdot (2n) \cdot (2(n+1))}{(n+1)!} = \frac{2 \cdot 4 \cdot \dots \cdot (2n) \cdot (2n+2)}{(n+1)!} \]
        \[ L = \lim_{n \to \infty} \left| \frac{a_{n+1}}{a_n} \right| \]
    \item \textbf{Evaluate the limit:}
        \begin{align*}
            L &= \lim_{n \to \infty} \left| \frac{2 \cdot 4 \cdot \dots \cdot (2n) \cdot (2n+2)}{(n+1)!} \cdot \frac{n!}{2 \cdot 4 \cdot \dots \cdot (2n)} \right| \\
            &= \lim_{n \to \infty} \left| \frac{2n+2}{1} \cdot \frac{n!}{(n+1)!} \right| \\
            &= \lim_{n \to \infty} \left| (2n+2) \cdot \frac{1}{n+1} \right| \\
            &= \lim_{n \to \infty} \frac{2(n+1)}{n+1} = 2
        \end{align*}
    \item \textbf{Conclusion:} Since $L=2 > 1$, the series is divergent.
\end{enumerate}
\begin{itemize}
    \item \textbf{Final Answer:} The series is divergent.
\end{itemize}

\section{Problem 12}
Use the Root Test to determine whether the series $\sum_{n=1}^{\infty} \left(\frac{n^2+3}{8n^2+9}\right)^n$ is convergent or divergent.

\textbf{Solution:}
\begin{enumerate}
    \item \textbf{Identify $a_n$:} $a_n = \left(\frac{n^2+3}{8n^2+9}\right)^n$. This form is a perfect candidate for the Root Test.
    \item \textbf{Set up the Root Test limit:}
        \[ L = \lim_{n \to \infty} \sqrt[n]{|a_n|} = \lim_{n \to \infty} \left| \left(\frac{n^2+3}{8n^2+9}\right)^n \right|^{1/n} \]
    \item \textbf{Evaluate the limit:}
        \begin{align*}
            L &= \lim_{n \to \infty} \frac{n^2+3}{8n^2+9} \\
            &= \lim_{n \to \infty} \frac{n^2(1+3/n^2)}{n^2(8+9/n^2)} \\
            &= \lim_{n \to \infty} \frac{1+3/n^2}{8+9/n^2} = \frac{1+0}{8+0} = \frac{1}{8}
        \end{align*}
    \item \textbf{Conclusion:} Since $L=\frac{1}{8} < 1$, the series is absolutely convergent.
\end{enumerate}
\begin{itemize}
    \item \textbf{Final Answer:} The series is convergent.
\end{itemize}

\section{Problem 13}
Use the Root Test to determine whether the series $\sum_{n=1}^{\infty} \left(\frac{-3n}{n+1}\right)^{3n}$ is convergent or divergent.

\textbf{Solution:}
\begin{enumerate}
    \item \textbf{Identify $a_n$:} $a_n = \left(\frac{-3n}{n+1}\right)^{3n}$.
    \item \textbf{Set up the Root Test limit:}
        \[ L = \lim_{n \to \infty} \sqrt[n]{|a_n|} = \lim_{n \to \infty} \left| \left(\frac{-3n}{n+1}\right)^{3n} \right|^{1/n} \]
    \item \textbf{Evaluate the limit:}
        \begin{align*}
            L &= \lim_{n \to \infty} \left| \left(\frac{-3n}{n+1}\right)^3 \right| \\
            &= \lim_{n \to \infty} \left| \left(\frac{-3}{1+1/n}\right)^3 \right| \\
            &= \left| (-3)^3 \right| = |-27| = 27
        \end{align*}
    \item \textbf{Conclusion:} Since $L=27 > 1$, the series is divergent.
\end{enumerate}
\begin{itemize}
    \item \textbf{Final Answer:} The series is divergent.
\end{itemize}

\section{Problem 14}
Use the Root Test to determine whether the series $\sum_{n=3}^{\infty} 6\left(1+\frac{1}{n}\right)^{n^2}$ is convergent or divergent.

\textbf{Solution:}
\begin{enumerate}
    \item \textbf{Identify $a_n$:} $a_n = 6\left(1+\frac{1}{n}\right)^{n^2}$.
    \item \textbf{Set up the Root Test limit:}
        \[ L = \lim_{n \to \infty} \sqrt[n]{|a_n|} = \lim_{n \to \infty} \left| 6\left(1+\frac{1}{n}\right)^{n^2} \right|^{1/n} \]
    \item \textbf{Evaluate the limit:}
        \begin{align*}
            L &= \lim_{n \to \infty} 6^{1/n} \left(\left(1+\frac{1}{n}\right)^{n^2}\right)^{1/n} \\
            &= \lim_{n \to \infty} 6^{1/n} \left(1+\frac{1}{n}\right)^{n^2/n} \\
            &= \lim_{n \to \infty} 6^{1/n} \left(1+\frac{1}{n}\right)^{n}
        \end{align*}
        We evaluate the limits of the two parts separately: $\lim_{n \to \infty} 6^{1/n} = 6^0 = 1$ and $\lim_{n \to \infty} (1+\frac{1}{n})^n = e$.
        \[ L = 1 \cdot e = e \]
    \item \textbf{Conclusion:} Since $L=e \approx 2.718 > 1$, the series is divergent.
\end{enumerate}
\begin{itemize}
    \item \textbf{Final Answer:} The series is divergent.
\end{itemize}

\section{Problem 15}
Use any test to determine whether the series $\sum_{n=2}^{\infty} \left(\frac{5n}{\ln(n)}\right)^n$ is absolutely convergent, conditionally convergent, or divergent.

\textbf{Solution:}
\begin{enumerate}
    \item \textbf{Choose a test:} The form of $a_n = \left(\frac{5n}{\ln(n)}\right)^n$ strongly suggests the Root Test.
    \item \textbf{Set up the Root Test limit:}
        \[ L = \lim_{n \to \infty} \sqrt[n]{|a_n|} = \lim_{n \to \infty} \left| \left(\frac{5n}{\ln(n)}\right)^n \right|^{1/n} \]
    \item \textbf{Evaluate the limit:}
        \[ L = \lim_{n \to \infty} \frac{5n}{\ln(n)} \]
        This is an indeterminate form $\frac{\infty}{\infty}$, so we can use L'Hôpital's Rule.
        \begin{align*}
            L &= 5 \lim_{n \to \infty} \frac{n}{\ln(n)} \\
            &\overset{L'H}{=} 5 \lim_{n \to \infty} \frac{\frac{d}{dn}(n)}{\frac{d}{dn}(\ln n)} \\
            &= 5 \lim_{n \to \infty} \frac{1}{1/n} = 5 \lim_{n \to \infty} n = \infty
        \end{align*}
    \item \textbf{Conclusion:} Since $L=\infty > 1$, the series is divergent.
\end{enumerate}
\begin{itemize}
    \item \textbf{Final Answer:} divergent
\end{itemize}

\part{In-Depth Analysis of Problems and Techniques}
\section{Problem Types and General Approach}
\begin{itemize}
    \item \textbf{Type 1: Conceptual Understanding (Problem 1)}
        \begin{itemize}
            \item \textbf{Description:} These problems directly test the theoretical conclusions of the Ratio Test based on the value of the limit $L$.
            \item \textbf{General Approach:} Simply state the conclusion based on whether $L<1$, $L>1$, or $L=1$. No calculation is needed.
        \end{itemize}
    \item \textbf{Type 2: Polynomials and Exponentials (Problems 2, 3, 4, 6, 7)}
        \begin{itemize}
            \item \textbf{Description:} The terms $a_n$ are composed of powers of $n$ (polynomials) and constant bases raised to the $n$-th power (exponentials).
            \item \textbf{General Approach:} The Ratio Test is ideal. The exponential terms simplify to constants (e.g., $b^{n+1}/b^n = b$), and the ratios of polynomials will have a limit of 1. The final limit $L$ is typically the ratio of the bases of the exponential terms.
        \end{itemize}
    \item \textbf{Type 3: Factorial-Based Series (Problems 5, 8, 9, 10, 11)}
        \begin{itemize}
            \item \textbf{Description:} The terms $a_n$ contain factorials ($n!$) or similar products ($1 \cdot 3 \cdot 5 \dots$).
            \item \textbf{General Approach:} The Ratio Test is almost always the best choice. The key simplification $\frac{n!}{(n+1)!} = \frac{1}{n+1}$ (or a similar cancellation) often drives the limit to 0, proving convergence, unless other terms grow even faster (as in Problem 11).
        \end{itemize}
    \item \textbf{Type 4: Expressions to the $n$-th Power (Problems 12, 13, 14, 15)}
        \begin{itemize}
            \item \textbf{Description:} The term $a_n$ has the structure of an entire expression raised to a power involving $n$, like $(f(n))^n$ or $(f(n))^{n^2}$.
            \item \textbf{General Approach:} The Root Test is tailor-made for these problems. The application of $(\cdot)^{1/n}$ dramatically simplifies the expression, often leaving a much easier limit to evaluate.
        \end{itemize}
\end{itemize}

\section{Key Algebraic and Calculus Manipulations}
\begin{itemize}
    \item \textbf{Ratio Test Setup (Invert and Multiply):} The most fundamental step. For $L = \lim |a_{n+1}/a_n|$, always write it as $\lim |a_{n+1} \cdot (1/a_n)|$. This avoids errors with complex fractions.
        \begin{itemize}
            \item \textbf{Example (Problem 3):} $\left| \frac{\frac{(-3)^{n+1}}{(n+1)^2}}{\frac{(-3)^n}{n^2}} \right|$ becomes the much cleaner $\left| \frac{(-3)^{n+1}}{(n+1)^2} \cdot \frac{n^2}{(-3)^n} \right|$.
        \end{itemize}
    \item \textbf{Factorial Simplification:} This is the most crucial trick for factorial problems.
        \begin{itemize}
            \item \textbf{Example (Problem 5):} In the ratio $\frac{k!}{(k+1)!}$, we write $(k+1)!$ as $(k+1) \cdot k!$, allowing cancellation to $\frac{1}{k+1}$. This was essential for finding the limit was 0.
        \end{itemize}
    \item \textbf{Exponential Simplification:} Used in almost every Ratio Test problem involving exponentials.
        \begin{itemize}
            \item \textbf{Example (Problem 2):} The ratio $\frac{6^n}{6^{n+1}}$ simplifies directly to $\frac{1}{6}$.
        \end{itemize}
    \item \textbf{Root Test Simplification:} The core of the Root Test is simplifying powers.
        \begin{itemize}
            \item \textbf{Example (Problem 14):} The term $((1+1/n)^{n^2})^{1/n}$ simplifies to $(1+1/n)^n$. This was necessary to reveal the famous limit that evaluates to $e$.
        \end{itemize}
    \item \textbf{Limits of Rational Functions:} To find the limit of a ratio of polynomials, divide the numerator and denominator by the highest power of $n$.
        \begin{itemize}
            \item \textbf{Example (Problem 12):} $\lim_{n \to \infty} \frac{n^2+3}{8n^2+9}$ was solved by dividing everything by $n^2$ to get $\lim_{n \to \infty} \frac{1+3/n^2}{8+9/n^2} = \frac{1}{8}$.
        \end{itemize}
    \item \textbf{Recognizing the Limit for $e$:} The limit $\lim_{n \to \infty} (1 + k/n)^n = e^k$ is a special form that must be memorized.
        \begin{itemize}
            \item \textbf{Example (Problem 14):} Identifying $\lim_{n \to \infty} (1+1/n)^n = e$ was the key step after applying the Root Test.
        \end{itemize}
    \item \textbf{L'Hôpital's Rule:} Used for indeterminate forms like $\infty/\infty$ or $0/0$.
        \begin{itemize}
            \item \textbf{Example (Problem 15):} The limit $\lim_{n \to \infty} \frac{n}{\ln(n)}$ was of the form $\infty/\infty$. Taking the derivative of the top and bottom gave $\lim_{n \to \infty} \frac{1}{1/n} = \lim_{n \to \infty} n = \infty$.
        \end{itemize}
\end{itemize}

\part{"Cheatsheet" and Tips for Success}
\section{Summary of Formulas}
\begin{itemize}
    \item \textbf{Ratio Test:} For $\sum a_n$, compute $L = \lim_{n \to \infty} \left| \frac{a_{n+1}}{a_n} \right|$.
    \item \textbf{Root Test:} For $\sum a_n$, compute $L = \lim_{n \to \infty} \sqrt[n]{|a_n|}$.
    \item \textbf{Conclusions:} If $L<1 \Rightarrow$ Absolutely Convergent. If $L>1 \Rightarrow$ Divergent. If $L=1 \Rightarrow$ Test Fails.
\end{itemize}

\section{Tricks and Shortcuts}
\begin{itemize}
    \item \textbf{Problem Recognition Heuristic:}
        \begin{itemize}
            \item If you see \textbf{factorials} ($n!$) or products of consecutive terms, use the \textbf{Ratio Test}.
            \item If you see the entire expression for $a_n$ is raised to the \textbf{$n$-th power}, use the \textbf{Root Test}.
            \item If you have simple polynomials or p-series (like $\sum 1/n^2$), the Ratio/Root tests will likely yield $L=1$. Use a different test.
        \end{itemize}
    \item \textbf{Hierarchy of Growth:} For large $n$, factorials grow faster than exponentials, which grow faster than polynomials, which grow faster than logarithms.
        \[ n! \gg b^n \gg n^p \gg \ln(n) \]
        This helps predict limits. For example, in $\sum \frac{n^{500} 500^n}{n!}$ (Problem 9), the $n!$ in the denominator dominates everything, so the limit of the ratio goes to 0, implying convergence.
\end{itemize}

\section{Common Pitfalls and Mistakes}
\begin{itemize}
    \item \textbf{Forgetting Absolute Value:} Always use absolute values when setting up the limit, especially for alternating series. A common mistake is getting a negative limit and concluding convergence (e.g., Problem 5 in the "Find the Flaw" section).
    \item \textbf{Mistake at $L=1$:} Do not conclude that the series diverges if $L=1$. The test is \textit{inconclusive}. You must state this and use another test.
    \item \textbf{Algebraic Errors:} Be very careful when simplifying the ratio $a_{n+1}/a_n$. Write out the terms clearly. A common error is simplifying $(2(n+1))!$ to $2(n+1)!$. The correct form is $(2n+2)!$.
    \item \textbf{Incorrectly Applying Root Test:} The Root Test is for $\lim (\dots)^{1/n}$. Do not confuse this with the Test for Divergence, which is $\lim a_n$. For example, for $\sum (1+1/n)^n$, the terms approach $e \neq 0$, so it diverges. The Root test would give $L=1$, which is inconclusive.
\end{itemize}

\part{Conceptual Synthesis and The "Big Picture"}
\section{Thematic Connections}
The central theme of the Ratio and Root tests is **determining convergence by comparing a series to a geometric series based on its asymptotic behavior**. "Asymptotic behavior" simply refers to how the terms of the series behave as $n$ gets very large. The limit $L$ that we calculate in both tests is effectively the "eventual" common ratio of our series. If this eventual ratio is less than 1, the series "acts like" a convergent geometric series and thus converges.

This theme of comparing a complicated object to a simpler, well-understood one is a cornerstone of mathematical analysis. We see it in:
\begin{itemize}
    \item \textbf{Linear Approximation:} We approximate a complex function near a point with a simple straight line.
    \item \textbf{Taylor Series:} We approximate a complex function with a polynomial, which is an infinite sum. The Ratio Test is the primary tool used to find for which $x$-values this approximation is valid (the interval of convergence).
    \item \textbf{Limit Comparison Test:} We formally compare an unknown series to a known p-series or geometric series to determine its convergence.
\end{itemize}

\section{Forward and Backward Links}
\begin{itemize}
    \item \textbf{Backward Links (Foundations):}
        \begin{enumerate}
            \item \textbf{Geometric Series:} The entire logic of the Ratio and Root tests is a generalization of the convergence condition for a geometric series, $\sum r^n$, which converges only when $|r|<1$. The limit $L$ is our calculated, generalized version of $|r|$.
            \item \textbf{Limits at Infinity:} The tests are meaningless without a robust understanding of how to evaluate limits as $n \to \infty$. Skills in handling indeterminate forms are essential.
        \end{enumerate}
    \item \textbf{Forward Links (Applications):}
        \begin{enumerate}
            \item \textbf{Power Series and Taylor Series:} This is the single most important application of the Ratio Test. When analyzing a power series $\sum c_n(x-a)^n$, we use the Ratio Test to find the values of $x$ for which the series converges. The test directly gives us the \textbf{radius of convergence}, a fundamental property of any power series representation of a function. The skills you learn here are not just for abstract series; they are for defining the domain of functions represented as infinite polynomials.
        \end{enumerate}
\end{itemize}

\part{Real-World Application and Modeling}
\section{Concrete Scenarios in Finance and Economics}
\begin{itemize}
    \item \textbf{Scenario 1: Derivative Pricing Models.} In quantitative finance, the price of options and other derivatives can be modeled using series. For example, the price of certain exotic options can be represented as an infinite sum of discounted expected payoffs. A trader or risk manager needs to know if this sum converges to a finite price. If the series diverges, the model is misspecified or implies an arbitrage opportunity (infinite profit), signaling a fundamental flaw. The Ratio Test is used to check the stability and validity of these pricing models, especially when the payoffs involve factorials or powers (e.g., in path-dependent options).
    \item \textbf{Scenario 2: Economic Growth Models (Solow-Swan).} In macroeconomics, models of long-term economic growth involve capital accumulation over time. The total accumulated capital stock can be seen as a series where each term is the depreciated capital from the previous period plus new investment. If the parameters of the model (savings rate, depreciation rate, population growth) are represented in a series, convergence tests can determine if the economy reaches a stable "steady-state" level of capital per worker (the series converges) or if it grows indefinitely (the series diverges).
    \item \textbf{Scenario 3: Perpetuity Valuation with Non-standard Growth.} A standard perpetuity (a bond that pays coupons forever) has a finite present value if the discount rate is greater than the constant growth rate of the coupon. However, a financial engineer might design a security where the payment in year $n$ is not constant but grows in a more complex way, for instance, $C_n = \frac{\$100 \cdot n^2}{2^n}$. To price this security, one must calculate the present value, which is an infinite series: $PV = \sum_{n=1}^\infty \frac{C_n}{(1+r)^n}$. The Ratio Test is the perfect tool to determine if this security has a finite price or if the payment stream grows too fast to be valued.
\end{itemize}

\section{Model Problem Setup: Valuing a "Super-Growth" Security}
\begin{itemize}
    \item \textbf{Scenario:} A biotech startup issues a novel financial security. It promises to pay a dividend in year $n$ equal to $D_n = \frac{1000 \cdot n!}{10^n}$. The idea is that initial R\&D costs are high, but if successful, profits will grow at a factorial rate. You are a financial analyst, and your client wants to know if this security has a finite theoretical value, assuming a discount rate of 8\% per year.
    \item \textbf{Model Setup:}
        \begin{itemize}
            \item \textbf{Variables:}
                \begin{itemize}
                    \item $D_n = \frac{1000 \cdot n!}{10^n}$: Dividend payment in year $n$.
                    \item $r = 0.08$: Annual discount rate.
                \end{itemize}
            \item \textbf{Formulation:} The Present Value (PV) of the security is the infinite sum of all future dividends, each discounted back to the present. The discount factor for a payment in year $n$ is $\frac{1}{(1+r)^n}$.
            \[ PV = \sum_{n=1}^\infty \frac{D_n}{(1+r)^n} = \sum_{n=1}^\infty \frac{1000 \cdot n! / 10^n}{(1.08)^n} = \sum_{n=1}^\infty \frac{1000 \cdot n!}{(10 \cdot 1.08)^n} = \sum_{n=1}^\infty \frac{1000 \cdot n!}{(10.8)^n} \]
            \item \textbf{Equation to Solve:} To determine if the PV is finite, we must test the convergence of the series with terms $a_n = \frac{1000 \cdot n!}{(10.8)^n}$. We would apply the Ratio Test:
            \begin{align*}
                L &= \lim_{n \to \infty} \left| \frac{a_{n+1}}{a_n} \right| = \lim_{n \to \infty} \left| \frac{1000 \cdot (n+1)!}{(10.8)^{n+1}} \cdot \frac{(10.8)^n}{1000 \cdot n!} \right| \\
                &= \lim_{n \to \infty} \left| \frac{(n+1)!}{n!} \cdot \frac{(10.8)^n}{(10.8)^{n+1}} \right| = \lim_{n \to \infty} \left| (n+1) \cdot \frac{1}{10.8} \right| = \infty
            \end{align*}
            Since $L = \infty > 1$, the series diverges. The theoretical price is infinite, meaning the security is impossible to price and likely based on an unsustainable promise.
        \end{itemize}
\end{itemize}

\part{Common Variations and Untested Concepts}
My homework set was excellent but missed a critical scenario: what to do when the Ratio or Root Test results in $L=1$. This is a major component of a full curriculum on series.

\section{Untested Concept 1: Handling the Inconclusive Case ($L=1$)}
When $L=1$, the test provides no information. You must revert to a different test. The most common follow-ups are the Integral Test, the Limit Comparison Test, or the Alternating Series Test.

\subsection*{Example 1: $\sum_{n=1}^\infty \frac{1}{n^2+1}$}
\begin{enumerate}
    \item \textbf{Ratio Test (yields $L=1$):}
    \[ \lim_{n \to \infty} \left| \frac{1/((n+1)^2+1)}{1/(n^2+1)} \right| = \lim_{n \to \infty} \frac{n^2+1}{n^2+2n+2} = 1 \]
    The Ratio Test is inconclusive.
    \item \textbf{Correct Follow-up (Limit Comparison Test):}
    We compare our series to a known p-series. The terms $a_n = \frac{1}{n^2+1}$ behave like $b_n = \frac{1}{n^2}$ for large $n$. We know that $\sum \frac{1}{n^2}$ converges because it is a p-series with $p=2>1$. Let's check the limit of the ratio of the terms:
    \[ \lim_{n \to \infty} \frac{a_n}{b_n} = \lim_{n \to \infty} \frac{1/(n^2+1)}{1/n^2} = \lim_{n \to \infty} \frac{n^2}{n^2+1} = 1 \]
    Since the limit is a finite, positive number (1), and the series we compared to ($\sum b_n$) converges, our original series $\sum a_n$ must also converge.
\end{enumerate}

\section{Untested Concept 2: Conditional Convergence}
A series is conditionally convergent if it converges as written, but diverges when you take the absolute value of its terms. This often happens with alternating series where the Ratio Test on $|a_n|$ gives $L=1$.

\subsection*{Example 2: $\sum_{n=1}^\infty \frac{(-1)^{n-1}}{n}$}
\begin{enumerate}
    \item \textbf{Ratio Test (for absolute convergence):}
    We test the series of absolute values, $\sum |a_n| = \sum \frac{1}{n}$.
    \[ L = \lim_{n \to \infty} \left| \frac{1/(n+1)}{1/n} \right| = \lim_{n \to \infty} \frac{n}{n+1} = 1 \]
    The Ratio Test is inconclusive about absolute convergence. (We know from the p-series test that $\sum 1/n$ diverges, so it is not absolutely convergent).
    \item \textbf{Correct Follow-up (Alternating Series Test):}
    We test the original series $\sum \frac{(-1)^{n-1}}{n}$. Let $b_n = 1/n$.
    \begin{enumerate}
        \item $b_n = 1/n > 0$ for $n \ge 1$. (Condition 1 is met).
        \item $b_{n+1} = \frac{1}{n+1} \le \frac{1}{n} = b_n$. (Terms are decreasing. Condition 2 is met).
        \item $\lim_{n \to \infty} b_n = \lim_{n \to \infty} \frac{1}{n} = 0$. (Condition 3 is met).
    \end{enumerate}
    Since all three conditions of the Alternating Series Test are met, the series converges. Because the original series converges but the series of absolute values diverges, the series is \textbf{conditionally convergent}.
\end{enumerate}

\part{Advanced Diagnostic Testing: "Find the Flaw"}
For each problem below, a flawed step-by-step solution is provided. Your task is to identify the specific error, explain in one sentence why it's wrong, and then provide the correct step and final solution.

\section{Problem 1}
Test the convergence of $\sum_{n=1}^\infty \frac{(-4)^n}{n^2}$.

\subsection*{Flawed Solution}
\begin{enumerate}
    \item Let $a_n = \frac{(-4)^n}{n^2}$. We apply the Ratio Test.
    \item $L = \lim_{n \to \infty} \frac{a_{n+1}}{a_n} = \lim_{n \to \infty} \frac{(-4)^{n+1}/(n+1)^2}{(-4)^n/n^2}$.
    \item $L = \lim_{n \to \infty} \frac{(-4)^{n+1}}{(n+1)^2} \cdot \frac{n^2}{(-4)^n} = \lim_{n \to \infty} -4 \left(\frac{n}{n+1}\right)^2$.
    \item The limit of $(\frac{n}{n+1})^2$ is 1. So, $L = -4 \cdot 1 = -4$.
    \item Since $L=-4 < 1$, the series converges.
\end{enumerate}
\hrule
\textbf{Find the Flaw:}
\begin{itemize}
    \item \textbf{The Flaw is in Step 2/5:} The Ratio Test requires the limit of the \textit{absolute value} of the ratio.
    \item \textbf{Explanation:} The conclusion was based on a negative limit, but the Ratio Test limit $L$ must be non-negative.
    \item \textbf{Correction:}
    The setup should be $L = \lim_{n \to \infty} \left| \frac{a_{n+1}}{a_n} \right|$.
    \[ L = \lim_{n \to \infty} \left| -4 \left(\frac{n}{n+1}\right)^2 \right| = \lim_{n \to \infty} 4 \left(\frac{n}{n+1}\right)^2 = 4(1)^2 = 4 \]
    Since $L=4 > 1$, the series \textbf{diverges}.
\end{itemize}

\section{Problem 2}
Test the convergence of $\sum_{n=1}^\infty \frac{n!}{100^n}$.

\subsection*{Flawed Solution}
\begin{enumerate}
    \item Let $a_n = \frac{n!}{100^n}$. We apply the Ratio Test.
    \item $L = \lim_{n \to \infty} \left| \frac{(n+1)!/100^{n+1}}{n!/100^n} \right|$.
    \item $L = \lim_{n \to \infty} \left| \frac{(n+1)!}{n!} \cdot \frac{100^n}{100^{n+1}} \right| = \lim_{n \to \infty} \left| (n+1) \cdot \frac{1}{100} \right|$.
    \item As $n \to \infty$, $n+1$ goes to infinity.
    \item So, $L = \infty$. Since $L > 1$, the series converges.
\end{enumerate}
\hrule
\textbf{Find the Flaw:}
\begin{itemize}
    \item \textbf{The Flaw is in Step 5:} The conclusion drawn from the limit value is incorrect.
    \item \textbf{Explanation:} According to the Ratio Test, if the limit $L$ is greater than 1 (including infinity), the series diverges.
    \item \textbf{Correction:}
    The calculation that $L = \infty$ is correct.
    Since $L = \infty > 1$, the series \textbf{diverges}.
\end{itemize}

\section{Problem 3}
Test the convergence of $\sum_{n=1}^\infty \left( \frac{n+1}{n} \right)^{n}$.

\subsection*{Flawed Solution}
\begin{enumerate}
    \item Let $a_n = \left( \frac{n+1}{n} \right)^{n}$. The $n$-th power suggests the Root Test.
    \item $L = \lim_{n \to \infty} \sqrt[n]{|a_n|} = \lim_{n \to \infty} \left( \left( \frac{n+1}{n} \right)^{n} \right)^{1/n}$.
    \item $L = \lim_{n \to \infty} \frac{n+1}{n} = \lim_{n \to \infty} \left(1 + \frac{1}{n}\right)$.
    \item $L = 1+0=1$.
    \item Since $L=1$, the series converges.
\end{enumerate}
\hrule
\textbf{Find the Flaw:}
\begin{itemize}
    \item \textbf{The Flaw is in Step 5:} The conclusion that the series converges when $L=1$ is incorrect.
    \item \textbf{Explanation:} When the Root Test (or Ratio Test) yields $L=1$, the test is inconclusive and cannot be used to determine convergence.
    \item \textbf{Correction:}
    The calculation that $L=1$ is correct, but this means the Root Test fails. We must use a different test. Let's use the Test for Divergence:
    \[ \lim_{n \to \infty} a_n = \lim_{n \to \infty} \left( \frac{n+1}{n} \right)^{n} = \lim_{n \to \infty} \left( 1 + \frac{1}{n} \right)^{n} = e \]
    Since the limit of the terms is $e \neq 0$, the series \textbf{diverges} by the Test for Divergence.
\end{itemize}

\section{Problem 4}
Test the convergence of $\sum_{n=1}^\infty \frac{(2n)!}{n! n!}$.

\subsection*{Flawed Solution}
\begin{enumerate}
    \item Let $a_n = \frac{(2n)!}{n! n!}$. We apply the Ratio Test.
    \item $a_{n+1} = \frac{(2n+1)!}{(n+1)!(n+1)!}$.
    \item $L = \lim_{n \to \infty} \left| \frac{(2n+1)!}{(n+1)!(n+1)!} \cdot \frac{n! n!}{(2n)!} \right|$.
    \item $L = \lim_{n \to \infty} \left| \frac{(2n+1)!}{(2n)!} \cdot \frac{n!}{(n+1)!} \cdot \frac{n!}{(n+1)!} \right| = \lim_{n \to \infty} \left| (2n+1) \cdot \frac{1}{n+1} \cdot \frac{1}{n+1} \right|$.
    \item $L = \lim_{n \to \infty} \frac{2n+1}{n^2+2n+1} = \lim_{n \to \infty} \frac{2/n+1/n^2}{1+2/n+1/n^2} = \frac{0}{1}=0$.
    \item Since $L=0<1$, the series converges.
\end{enumerate}
\hrule
\textbf{Find the Flaw:}
\begin{itemize}
    \item \textbf{The Flaw is in Step 2:} The expression for $a_{n+1}$ is incorrect due to a common mistake with factorials.
    \item \textbf{Explanation:} The term $(2(n+1))!$ was incorrectly written as $(2n+1)!$, but it should be $(2n+2)!$.
    \item \textbf{Correction:}
    The correct $a_{n+1}$ is $\frac{(2(n+1))!}{(n+1)!(n+1)!} = \frac{(2n+2)!}{(n+1)!(n+1)!}$.
    The ratio calculation should be:
    \begin{align*}
        L &= \lim_{n \to \infty} \left| \frac{(2n+2)!}{(n+1)!(n+1)!} \cdot \frac{n! n!}{(2n)!} \right| \\
        &= \lim_{n \to \infty} \left| \frac{(2n+2)(2n+1)(2n)!}{(2n)!} \cdot \left(\frac{n!}{(n+1)!}\right)^2 \right| \\
        &= \lim_{n \to \infty} \left| (2n+2)(2n+1) \cdot \left(\frac{1}{n+1}\right)^2 \right| \\
        &= \lim_{n \to \infty} \frac{4n^2+6n+2}{n^2+2n+1} = 4
    \end{align*}
    Since $L=4>1$, the series \textbf{diverges}.
\end{itemize}

\section{Problem 5}
Test the convergence of $\sum_{k=1}^\infty \frac{k^2+1}{k^3+1}$.

\subsection*{Flawed Solution}
\begin{enumerate}
    \item Let $a_k = \frac{k^2+1}{k^3+1}$. We apply the Ratio Test.
    \item $L = \lim_{k \to \infty} \left| \frac{(k+1)^2+1}{(k+1)^3+1} \cdot \frac{k^3+1}{k^2+1} \right|$.
    \item Expanding the polynomials: $L = \lim_{k \to \infty} \left| \frac{k^2+2k+2}{k^3+3k^2+3k+2} \cdot \frac{k^3+1}{k^2+1} \right|$.
    \item The dominant terms are $\frac{k^2}{k^3} \cdot \frac{k^3}{k^2} = 1$.
    \item The limit is $L=1$.
    \item Since $L=1$, the series diverges.
\end{enumerate}
\hrule
\textbf{Find the Flaw:}
\begin{itemize}
    \item \textbf{The Flaw is in Step 6:} The conclusion drawn from $L=1$ is incorrect.
    \item \textbf{Explanation:} The Ratio Test is inconclusive when $L=1$; it does not imply divergence.
    \item \textbf{Correction:}
    The calculation that $L=1$ is correct, meaning the Ratio Test fails. We must use another test. The Limit Comparison Test is appropriate. Compare $a_k = \frac{k^2+1}{k^3+1}$ to $b_k = \frac{k^2}{k^3} = \frac{1}{k}$.
    \[ \lim_{k \to \infty} \frac{a_k}{b_k} = \lim_{k \to \infty} \frac{(k^2+1)/(k^3+1)}{1/k} = \lim_{k \to \infty} \frac{k(k^2+1)}{k^3+1} = \lim_{k \to \infty} \frac{k^3+k}{k^3+1} = 1 \]
    Since the limit is a finite positive number, and the comparison series $\sum b_k = \sum \frac{1}{k}$ (the harmonic series) diverges, the original series also \textbf{diverges}. (The final answer was correct by coincidence, but the reasoning was flawed).
\end{itemize}

\end{document}