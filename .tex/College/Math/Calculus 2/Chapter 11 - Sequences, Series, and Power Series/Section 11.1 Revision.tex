\documentclass{article}
\usepackage{amsmath}
\usepackage{amssymb}
\usepackage{geometry}
\geometry{a4paper, margin=1in}

\title{Homework 11.1 Sequences: Additional Practice Problems}
\author{Generated by Gemini}
\date{\today}
\begin{document}
\maketitle

\section*{Problems}

\begin{enumerate}
    % Direct Calculation
    \item List the first five terms of the sequence $a_n = \frac{(-1)^n(n+1)}{n^2}$.
    \item List the first five terms of the sequence given by $a_1 = 2$ and $a_{n+1} = \frac{a_n}{a_n + 1}$.
    \item List the first five terms of the sequence $\{ \cos(n\pi) \}_{n=1}^{\infty}$.
    \item List the first five terms of the sequence $a_n = \frac{(n+2)!}{n!}$.

    % Pattern Recognition
    \item Find a formula for the general term $a_n$ of the sequence, assuming the pattern continues: 
    $\{ \frac{2}{3}, \frac{3}{5}, \frac{4}{7}, \frac{5}{9}, \dots \}$.
    \item Find a formula for the general term $a_n$ of the sequence, assuming the pattern continues:
    $\{ -5, 10, -15, 20, -25, \dots \}$.
    \item Find a formula for the general term $a_n$ of the sequence, assuming the pattern continues:
    $\{ 1, -\frac{1}{4}, \frac{1}{9}, -\frac{1}{16}, \frac{1}{25}, \dots \}$.
    \item Find a formula for the general term $a_n$ of the sequence, which is an arithmetic sequence:
    $\{ 11, 7, 3, -1, -5, \dots \}$.

    % Limit Evaluation: Algebraic Manipulation
    \item Determine if the sequence $a_n = \frac{5n^2 - 3n + 1}{2n^2 + n}$ converges or diverges. If it converges, find the limit.
    \item Determine if the sequence $a_n = \frac{\sqrt{4n^2 + n}}{3n - 1}$ converges or diverges. If it converges, find the limit.
    \item Determine if the sequence $a_n = \frac{n^3}{n^2 + 8}$ converges or diverges. If it converges, find the limit.
    \item Determine if the sequence $a_n = n - \sqrt{n^2 - 4n}$ converges or diverges. If it converges, find the limit.
    \item Determine if the sequence $a_n = \frac{8^n + 1}{8^n - 1}$ converges or diverges. If it converges, find the limit.

    % Limit Evaluation: L'Hôpital's Rule
    \item Determine if the sequence $a_n = \frac{\ln(n^2)}{n}$ converges or diverges. If it converges, find the limit.
    \item Determine if the sequence $a_n = n^2 e^{-n}$ converges or diverges. If it converges, find the limit.
    \item Determine if the sequence $a_n = (1 + \frac{2}{n})^n$ converges or diverges. If it converges, find the limit. (Hint: Consider the limit of $\ln(a_n)$).
    \item Determine if the sequence $a_n = \frac{\arctan(n)}{n}$ converges or diverges. If it converges, find the limit.
    \item Determine if the sequence $a_n = n \sin(\frac{1}{n})$ converges or diverges. If it converges, find the limit.

    % Limit Evaluation: Squeeze Theorem
    \item Determine if the sequence $a_n = \frac{\sin(n)}{n^2}$ converges or diverges. If it converges, find the limit.
    \item Determine if the sequence $a_n = \frac{(-1)^n}{n!}$ converges or diverges. If it converges, find the limit.
    \item Determine if the sequence $a_n = \frac{5^n}{n!}$ converges or diverges. If it converges, find the limit. (Hint: For $n > 5$, $\frac{5^n}{n!} = \frac{5^5}{5!} \cdot \frac{5}{6} \cdot \frac{5}{7} \cdots \frac{5}{n}$).

    % Hierarchy of Growth & Special Cases
    \item Determine if the sequence $a_n = \frac{100^n}{n!}$ converges or diverges. If it converges, find the limit.
    \item Determine if the sequence $a_n = \frac{(\ln n)^2}{\sqrt{n}}$ converges or diverges. If it converges, find the limit.
    \item Determine if the sequence with terms $\{0.9, 0.99, 0.999, 0.9999, \dots\}$ converges or diverges. If it converges, find the limit.
    \item Determine if the sequence $a_n = (\frac{\pi}{e})^n$ converges or diverges. If it converges, find the limit.

    % Monotonicity, Boundedness, and Recursive Sequences
    \item Consider the sequence defined by $a_1 = 1$ and $a_{n+1} = \sqrt{6 + a_n}$. Show that the sequence is increasing and bounded above by 3. Then find its limit.
    \item Determine if the sequence $a_n = \frac{n}{n+1}$ is monotonic and bounded.
    \item Determine if the sequence $a_n = n + \frac{1}{n}$ is monotonic and bounded.
    \item A sequence is defined by $a_1 = 4$ and $a_{n+1} = \frac{1}{2}(a_n + \frac{9}{a_n})$. Assuming the sequence converges, find its limit.

    % "Find the Flaw" Problems
    \item \textbf{Problem:} Determine if $a_n = (\frac{n-1}{n})^n$ converges.
    \textbf{Flawed Solution:} The base is $\frac{n-1}{n} = 1 - \frac{1}{n}$. As $n \to \infty$, the base goes to 1. Since $1$ raised to any power is $1$, the limit is $1$.
    \textbf{Your Task:} Identify the flaw, explain the error, and provide the correct solution.
    
    \item \textbf{Problem:} Find the limit of the sequence $a_n = \frac{n^2}{n^3 + 1}$.
    \textbf{Flawed Solution:} This is an indeterminate form $\frac{\infty}{\infty}$, so we use L'Hôpital's Rule. 
    $\lim_{n\to\infty} \frac{n^2}{n^3 + 1} \overset{L'H}{=} \lim_{n\to\infty} \frac{2n}{3n^2} \overset{L'H}{=} \lim_{n\to\infty} \frac{2}{6n} = \infty$. The sequence diverges.
    \textbf{Your Task:} Identify the flaw, explain the error, and provide the correct solution.

    \item \textbf{Problem:} Determine if $a_n = \frac{(-1)^n n}{n+1}$ converges.
    \textbf{Flawed Solution:} Let's consider the absolute value: $|a_n| = \frac{n}{n+1}$. The limit is $\lim_{n\to\infty} \frac{n}{n+1} = 1$. Since the limit of the absolute value is 1, the sequence converges to 1.
    \textbf{Your Task:} Identify the flaw, explain the error, and provide the correct solution.

    % Conceptual & Application
    \item A patient receives a 200 mg dose of a drug at the same time every day. Between doses, 25\% of the drug in the body is eliminated. Let $Q_n$ be the quantity of the drug in the body immediately after the $n$-th dose. Write a recursive formula for $Q_n$. Assuming it converges, find the long-term quantity of the drug in the body (the limit).
    \item Does every bounded sequence converge? Explain your reasoning and provide an example if necessary.
\end{enumerate}

\newpage

\section*{Solutions}
\begin{enumerate}
    \item $a_1 = \frac{-2}{1} = -2$, $a_2 = \frac{3}{4}$, $a_3 = \frac{-4}{9}$, $a_4 = \frac{5}{16}$, $a_5 = \frac{-6}{25}$.
    \item $a_1 = 2$, $a_2 = \frac{2}{3}$, $a_3 = \frac{2/3}{2/3+1} = \frac{2/3}{5/3} = \frac{2}{5}$, $a_4 = \frac{2/5}{2/5+1} = \frac{2/5}{7/5} = \frac{2}{7}$, $a_5 = \frac{2/7}{2/7+1} = \frac{2/7}{9/7} = \frac{2}{9}$.
    \item $\cos(\pi)=-1$, $\cos(2\pi)=1$, $\cos(3\pi)=-1$, $\cos(4\pi)=1$, $\cos(5\pi)=-1$. The terms are $\{-1, 1, -1, 1, -1, \dots\}$.
    \item $a_1 = \frac{3!}{1!} = 6$, $a_2 = \frac{4!}{2!} = 12$, $a_3 = \frac{5!}{3!} = 20$, $a_4 = \frac{6!}{4!} = 30$, $a_5 = \frac{7!}{5!} = 42$. (Note: $a_n = (n+2)(n+1)$).
    \item Numerator is $n+1$. Denominator is an arithmetic sequence $3,5,7,\dots$ which is $2n+1$. So, $a_n = \frac{n+1}{2n+1}$.
    \item The terms are alternating multiples of 5. $a_n = (-1)^n (5n)$.
    \item Alternating signs and denominators are perfect squares. $a_n = (-1)^{n+1} \frac{1}{n^2}$.
    \item Common difference is $d = 7-11 = -4$. $a_n = a_1 + (n-1)d = 11 + (n-1)(-4) = 11 - 4n + 4 = 15 - 4n$.
    \item Divide numerator and denominator by $n^2$: $\lim_{n\to\infty} \frac{5 - 3/n + 1/n^2}{2 + 1/n} = \frac{5-0+0}{2+0} = \frac{5}{2}$. Converges to $5/2$.
    \item Divide by $n$ (which is $\sqrt{n^2}$ inside the radical): $\lim_{n\to\infty} \frac{\sqrt{4 + 1/n}}{3 - 1/n} = \frac{\sqrt{4+0}}{3-0} = \frac{2}{3}$. Converges to $2/3$.
    \item The degree of the numerator is greater than the degree of the denominator. The limit is $\infty$. The sequence diverges.
    \item Multiply by the conjugate: $\lim_{n\to\infty} (n - \sqrt{n^2-4n}) \frac{n + \sqrt{n^2-4n}}{n + \sqrt{n^2-4n}} = \lim_{n\to\infty} \frac{n^2 - (n^2-4n)}{n + \sqrt{n^2-4n}} = \lim_{n\to\infty} \frac{4n}{n + \sqrt{n^2(1-4/n)}} = \lim_{n\to\infty} \frac{4n}{n + n\sqrt{1-4/n}} = \lim_{n\to\infty} \frac{4}{1 + \sqrt{1-4/n}} = \frac{4}{1+\sqrt{1}} = 2$. Converges to 2.
    \item Divide by the fastest-growing term, $8^n$: $\lim_{n\to\infty} \frac{1 + 1/8^n}{1 - 1/8^n} = \frac{1+0}{1-0} = 1$. Converges to 1.
    \item Use L'Hôpital's Rule on $f(x) = \frac{2\ln(x)}{x}$. $\lim_{x\to\infty} \frac{2\ln(x)}{x} \overset{L'H}{=} \lim_{x\to\infty} \frac{2/x}{1} = 0$. Converges to 0.
    \item Use L'Hôpital's Rule on $f(x) = \frac{x^2}{e^x}$. $\lim_{x\to\infty} \frac{x^2}{e^x} \overset{L'H}{=} \lim_{x\to\infty} \frac{2x}{e^x} \overset{L'H}{=} \lim_{x\to\infty} \frac{2}{e^x} = 0$. Converges to 0.
    \item Let $L = \lim_{n\to\infty} (1 + \frac{2}{n})^n$. Then $\ln L = \lim_{n\to\infty} n \ln(1 + \frac{2}{n}) = \lim_{n\to\infty} \frac{\ln(1+2/n)}{1/n}$. Using L'Hôpital's Rule for $x \to \infty$: $\lim_{x\to\infty} \frac{\frac{1}{1+2/x}(-2/x^2)}{-1/x^2} = \lim_{x\to\infty} \frac{2}{1+2/x} = 2$. So $\ln L = 2$, which means $L=e^2$. Converges to $e^2$.
    \item As $n \to \infty$, $\arctan(n) \to \frac{\pi}{2}$ and the denominator $n \to \infty$. A constant divided by infinity is 0. Converges to 0.
    \item Rewrite as $\lim_{n\to\infty} \frac{\sin(1/n)}{1/n}$. Let $x = 1/n$. As $n\to\infty$, $x\to 0$. The limit becomes $\lim_{x\to 0} \frac{\sin(x)}{x} = 1$. Converges to 1.
    \item We know $-1 \le \sin(n) \le 1$. Thus, $-\frac{1}{n^2} \le \frac{\sin(n)}{n^2} \le \frac{1}{n^2}$. Since $\lim_{n\to\infty} -\frac{1}{n^2} = 0$ and $\lim_{n\to\infty} \frac{1}{n^2} = 0$, by the Squeeze Theorem, the sequence converges to 0.
    \item We know $-1 \le (-1)^n \le 1$. Thus, $-\frac{1}{n!} \le \frac{(-1)^n}{n!} \le \frac{1}{n!}$. Since $\lim_{n\to\infty} \frac{1}{n!} = 0$, by the Squeeze Theorem, the sequence converges to 0.
    \item For $n>5$, we have $0 \le a_n = \frac{5^n}{n!} = \frac{5^5}{5!} \left(\frac{5}{6}\right) \left(\frac{5}{7}\right) \cdots \left(\frac{5}{n}\right) \le \frac{5^5}{5!} \left(\frac{5}{6}\right)^{n-5}$. The right side is a geometric sequence with $|r| = 5/6 < 1$, so it converges to 0. By the Squeeze Theorem, the sequence converges to 0.
    \item For $n>100$, $a_n = \frac{100^n}{n!} = \frac{100^{100}}{100!} \cdot \frac{100}{101} \cdots \frac{100}{n} \to 0$. Converges to 0 as factorials grow faster than exponentials.
    \item Polynomials (even roots) grow faster than logarithms. The limit is 0. Using L'Hôpital's twice on $f(x) = (\ln x)^2 / \sqrt{x}$ confirms this. Converges to 0.
    \item The terms are $1-0.1, 1-0.01, 1-0.001, \dots$. The sequence can be written as $a_n = 1 - 10^{-n}$. As $n \to \infty$, $10^{-n} \to 0$, so the limit is 1. Converges to 1.
    \item This is a geometric sequence with ratio $r = \pi/e$. Since $\pi \approx 3.14159$ and $e \approx 2.71828$, $r > 1$. The sequence diverges to $\infty$.
    \item \textbf{Monotonic:} $a_1=1, a_2=\sqrt{7}\approx 2.64$. Assume $a_k > a_{k-1}$. Then $a_k+6 > a_{k-1}+6$, so $\sqrt{a_k+6} > \sqrt{a_{k-1}+6}$, which means $a_{k+1} > a_k$. Sequence is increasing. \textbf{Bounded:} $a_1 < 3$. Assume $a_k < 3$. Then $a_k+6 < 9$, so $a_{k+1}=\sqrt{a_k+6} < \sqrt{9}=3$. Bounded above by 3. Since it is monotonic and bounded, it converges. Let $L = \lim a_n$. Then $L = \sqrt{6+L} \implies L^2 = 6+L \implies L^2-L-6=0 \implies (L-3)(L+2)=0$. Since terms are positive, $L=3$.
    \item $a_{n+1}-a_n = \frac{n+1}{n+2} - \frac{n}{n+1} = \frac{(n+1)^2 - n(n+2)}{(n+2)(n+1)} = \frac{n^2+2n+1 - n^2-2n}{(n+2)(n+1)} = \frac{1}{(n+2)(n+1)} > 0$. It is increasing (monotonic). It is bounded below by $a_1=1/2$ and above by its limit, 1.
    \item $a_1=2, a_2=2.5, a_3 \approx 3.33$. It appears to be increasing. $f(x)=x+1/x \implies f'(x)=1-1/x^2 > 0$ for $x>1$. So it's increasing. It is monotonic. It is bounded below by $a_1=2$, but it is not bounded above since $\lim_{n\to\infty} n+1/n = \infty$.
    \item Let $L$ be the limit. Then $L = \frac{1}{2}(L + \frac{9}{L}) \implies 2L = L + \frac{9}{L} \implies L = \frac{9}{L} \implies L^2 = 9$. Since all terms are positive, the limit must be positive. $L=3$.
    \item \textbf{Flaw:} The expression is of the indeterminate form $1^\infty$, which is not necessarily 1. \textbf{Explanation:} The base approaches 1 while the exponent approaches infinity. Their interaction determines the limit. \textbf{Correction:} This is a famous limit. Let $L = \lim_{n\to\infty} (1 - \frac{1}{n})^n$. Use the fact that $\lim_{n\to\infty} (1+\frac{x}{n})^n = e^x$. Here $x=-1$, so the limit is $e^{-1} = \frac{1}{e}$. Converges to $1/e$.
    \item \textbf{Flaw:} The final step in the calculation is wrong. \textbf{Explanation:} $\lim_{n\to\infty} \frac{2}{6n}$ is of the form "constant over infinity", which is 0, not $\infty$. \textbf{Correction:} The limit is 0. Alternatively, dividing by the highest power in the denominator ($n^3$) gives $\lim_{n\to\infty} \frac{1/n}{1 + 1/n^3} = \frac{0}{1+0} = 0$. Converges to 0.
    \item \textbf{Flaw:} The Theorem for Absolute Values only states that if $\lim |a_n| = 0$, then $\lim a_n=0$. It does not apply if the limit of the absolute value is non-zero. \textbf{Explanation:} The sequence terms are approximately $\{-1, 1, -1, 1, \dots\}$ for large $n$. The sequence oscillates and does not approach a single value. \textbf{Correction:} The sequence oscillates between values close to -1 and 1. It does not converge. The sequence diverges.
    \item After a dose, the amount remaining from the previous day is $0.75 Q_{n-1}$. A new 200 mg dose is added. So, $Q_n = 0.75 Q_{n-1} + 200$. Let $L$ be the limit. Then $L = 0.75L + 200 \implies 0.25L = 200 \implies L = 800$. The long-term quantity is 800 mg.
    \item No. A bounded sequence does not necessarily converge. For a sequence to be guaranteed to converge, it must be both bounded and monotonic (Monotonic Sequence Theorem). \textbf{Example:} The sequence $a_n = (-1)^n$ is bounded (between -1 and 1) but it diverges because it oscillates and never settles on a single limit.
\end{enumerate}

\newpage

\section*{Concept Checklist and Problem Index}

\begin{itemize}
    \item \textbf{Core Concepts \& Direct Calculation}
        \begin{itemize}
            \item Listing terms of an explicit sequence: 1, 3, 4
            \item Listing terms of a recursive sequence: 2
            \item Definition of convergence/divergence: 34, 35 and all limit problems
        \end{itemize}
    \item \textbf{Pattern Recognition}
        \begin{itemize}
            \item Finding a formula for an algebraic sequence: 5
            \item Finding a formula for an alternating sequence: 6, 7
            \item Finding a formula for an arithmetic sequence: 8
        \end{itemize}
    \item \textbf{Limit Evaluation: Algebraic Manipulation}
        \begin{itemize}
            \item Rational functions (dividing by highest power): 9
            \item Functions with radicals: 10
            \item Comparing degrees of numerator/denominator: 11
            \item Using the conjugate method: 12
            \item Exponential functions (dividing by fastest-growing term): 13
        \end{itemize}
    \item \textbf{Limit Evaluation: Calculus Theorems}
        \begin{itemize}
            \item L'Hôpital's Rule: 14, 15, 18
            \item L'Hôpital's Rule for indeterminate powers ($1^\infty$, etc.): 16
            \item Squeeze Theorem: 17, 19, 20, 21
        \end{itemize}
    \item \textbf{Hierarchy of Growth \& Special Cases}
        \begin{itemize}
            \item Factorials vs. Exponentials: 21, 22
            \item Polynomials vs. Logarithms: 23
            \item Geometric sequences: 25
            \item Recognizing a limit pattern: 24
        \end{itemize}
    \item \textbf{Monotonic \& Bounded Sequences}
        \begin{itemize}
            \item Determining if a sequence is monotonic and bounded: 27, 28
            \item Proving monotonicity and boundedness (by induction): 26
            \item Applying the Monotonic Sequence Theorem: 26, 35
        \end{itemize}
    \item \textbf{Recursive Sequences}
        \begin{itemize}
            \item Finding the limit of a convergent recursive sequence: 26, 29, 34
        \end{itemize}
    \item \textbf{Advanced Diagnostic / "Find the Flaw"}
        \begin{itemize}
            \item Error with indeterminate form $1^\infty$: 30
            \item Error in applying L'Hôpital's Rule: 31
            \item Error with the Absolute Value Theorem for limits: 32
        \end{itemize}
    \item \textbf{Applications \& Conceptual Understanding}
        \begin{itemize}
            \item Real-world modeling (pharmacokinetics): 34
            \item Relationship between boundedness and convergence: 35
        \end{itemize}
\end{itemize}

\end{document}