\documentclass{article}

\usepackage{amsmath}
\usepackage{amssymb}
\usepackage[margin=1in]{geometry}
\usepackage{chemformula}

\title{Homework 9: Chemical Reactions in Aqueous Solutions}
\author{Tashfeen Omran}
\date{October 2025}

\begin{document}

\maketitle

\section*{Quiz Details}
\begin{itemize}
    \item \textbf{When:} Monday, October 27, 2025, during class.
    \item \textbf{Format:} Closed notes.
    \item \textbf{Allowed Materials:} A scientific calculator. Phones are not permitted.
\end{itemize}

\section{Topic 1: Distinguishing Electrolytes}
\subsection{Key Principles}
\begin{itemize}
    \item \textbf{Electrolytes} are substances that dissolve in water to produce ions, creating a solution that can conduct electricity. \textbf{Nonelectrolytes} dissolve but do not produce ions and do not conduct electricity.
    \item \textbf{Strong Electrolytes:} Dissociate completely (or nearly 100\%) into ions in water. The light bulb in a conductivity test shines brightly.
        \begin{itemize}
            \item \textbf{Memorize This Group:} Soluble ionic compounds, strong acids, and strong bases.
        \end{itemize}
    \item \textbf{Weak Electrolytes:} Dissociate only partially into ions. Most of the substance remains as neutral molecules. The light bulb shines dimly.
        \begin{itemize}
            \item \textbf{Memorize This Group:} Insoluble ionic compounds (they dissolve very slightly), weak acids, and weak bases.
        \end{itemize}
    \item \textbf{Nonelectrolytes:} Do not dissociate into ions at all. The light bulb does not light up.
        \begin{itemize}
            \item \textbf{Memorize This Group:} Most molecular compounds (that are not acids or bases), such as sugars (\ch{C6H12O6}), alcohols (\ch{CH3OH}), etc.
        \end{itemize}
\end{itemize}

\subsection{Worked Examples}
\subsubsection{Example 1}
\textbf{Problem:} Is \ch{PbSO4} a strong, weak, or nonelectrolyte when dissolved in water?

\textbf{Source:} Electrolytes and molecular/ionic equations.pdf, Problem 1a.

\textbf{Analysis:} First, identify the type of compound. \ch{PbSO4} is an ionic compound (metal + polyatomic ion). Second, check its solubility using the solubility rules. According to the rules, sulfates (\ch{SO4^2-}) are generally soluble, but \ch{Pb^2+} is a key exception, making \ch{PbSO4} insoluble. Insoluble ionic compounds are considered weak electrolytes because a very tiny amount does dissolve and produce ions.

\textbf{Step-by-Step Solution:}
\begin{enumerate}
    \item Identify the compound type: \ch{PbSO4} is ionic.
    \item Check solubility rules: Sulfates are soluble, EXCEPT with \ch{Sr^2+}, \ch{Ba^2+}, \ch{Pb^2+}, and \ch{Ag+}.
    \item Conclude solubility: \ch{PbSO4} is insoluble.
    \item Classify the electrolyte: Insoluble ionic compounds are classified as weak electrolytes.
\end{enumerate}

\textbf{Answer:} \textbf{Weak electrolyte}.

\subsubsection{Example 2}
\textbf{Problem:} Is \ch{K2CrO4} a strong, weak, or nonelectrolyte when dissolved in water?

\textbf{Source:} Electrolytes and molecular/ionic equations.pdf, Problem 1d.

\textbf{Analysis:} \ch{K2CrO4} is an ionic compound. According to the solubility rules, compounds containing Group 1 cations (like \ch{K+}) are always soluble, with no exceptions. Soluble ionic compounds are strong electrolytes.

\textbf{Step-by-Step Solution:}
\begin{enumerate}
    \item Identify the compound type: \ch{K2CrO4} is ionic.
    \item Check solubility rules: Compounds with Group 1 alkali metal ions (\ch{Li+}, \ch{Na+}, \ch{K+}, etc.) are soluble.
    \item Conclude solubility: \ch{K2CrO4} is soluble.
    \item Classify the electrolyte: Soluble ionic compounds are classified as strong electrolytes.
\end{enumerate}

\textbf{Answer:} \textbf{Strong electrolyte}.

\subsubsection{Example 3}
\textbf{Problem:} Is \ch{C6H12O6} a strong, weak, or nonelectrolyte when dissolved in water?

\textbf{Source:} Electrolytes and molecular/ionic equations.pdf, Problem 1e.

\textbf{Analysis:} \ch{C6H12O6} (glucose) is a molecular compound made of nonmetals. It is not an acid or a base. Molecular compounds that are not acids or bases are nonelectrolytes. They dissolve in water but do not form ions.

\textbf{Step-by-Step Solution:}
\begin{enumerate}
    \item Identify the compound type: \ch{C6H12O6} is molecular (contains only nonmetals).
    \item Check if it's an acid or base: It is not on the list of common acids or bases.
    \item Classify the electrolyte: Molecular compounds that are not acids/bases are nonelectrolytes.
\end{enumerate}

\textbf{Answer:} \textbf{Nonelectrolyte}.

\subsubsection{Example 4}
\textbf{Problem:} Is \ch{HNO2} a strong, weak, or nonelectrolyte when dissolved in water?

\textbf{Source:} Electrolytes and molecular/ionic equations.pdf, Problem 1l.

\textbf{Analysis:} The formula \ch{HNO2} starts with H, identifying it as an acid. We must determine if it is a strong or weak acid. The list of seven strong acids must be memorized. \ch{HNO2} (nitrous acid) is NOT on that list, therefore it is a weak acid. All weak acids are weak electrolytes.

\textbf{Step-by-Step Solution:}
\begin{enumerate}
    \item Identify the compound type: Starts with H, so it's an acid.
    \item Check the strong acid list: Is \ch{HNO2} one of the seven strong acids? No. (\ch{HNO3} is strong, but \ch{HNO2} is not).
    \item Conclude acid strength: \ch{HNO2} is a weak acid.
    \item Classify the electrolyte: Weak acids are weak electrolytes.
\end{enumerate}

\textbf{Answer:} \textbf{Weak electrolyte}.

\section{Topic 2: Using Solubility Rules}
\subsection{Key Principles}
\begin{itemize}
    \item Solubility rules are a set of guidelines used to predict whether an ionic compound will dissolve in water (soluble, aq) or form a solid (insoluble, s).
    \item These rules are essential for predicting the products of precipitation reactions and for writing ionic equations.
    \item \textbf{Memorization Required:} You must memorize the solubility rules. A common version is found in \textit{lecture\_slides.pdf, slide 34}.
    \item \textbf{Key Rules to Prioritize:}
        \begin{itemize}
            \item Compounds with Group 1 cations (\ch{Li+}, \ch{Na+}, \ch{K+}), ammonium (\ch{NH4+}), and nitrate (\ch{NO3-}) are \textbf{ALWAYS soluble}.
            \item Halides (\ch{Cl-}, \ch{Br-}, \ch{I-}) are soluble, \textbf{EXCEPT} with \ch{Ag+}, \ch{Pb^2+}, and \ch{Hg2^2+}.
            \item Sulfates (\ch{SO4^2-}) are soluble, \textbf{EXCEPT} with \ch{Sr^2+}, \ch{Ba^2+}, \ch{Pb^2+}, and \ch{Ag+}.
            \item Hydroxides (\ch{OH-}) and Sulfides (\ch{S^2-}) are \textbf{insoluble}, \textbf{EXCEPT} with Group 1 cations, \ch{NH4+}, and the heavy Group 2 cations (\ch{Ca^2+}, \ch{Sr^2+}, \ch{Ba^2+}).
        \end{itemize}
\end{itemize}

\subsection{Worked Examples}
\subsubsection{Example 1}
\textbf{Problem:} Classify \ch{CaSO4} as soluble or insoluble in water.

\textbf{Source:} lecture\_slides.pdf, slide 36, Problem b.

\textbf{Analysis:} We need to apply the solubility rules for sulfate ions (\ch{SO4^2-}). The general rule is that sulfates are soluble, but there are important exceptions. We must check if calcium (\ch{Ca^2+}) is one of those exceptions.

\textbf{Step-by-Step Solution:}
\begin{enumerate}
    \item Identify the ions: \ch{Ca^2+} and \ch{SO4^2-}.
    \item Recall the sulfate rule: \ch{SO4^2-} compounds are generally soluble.
    \item Recall the exceptions: The exceptions are \ch{Sr^2+}, \ch{Ba^2+}, \ch{Pb^2+}, \ch{Ag+}, and \ch{Ca^2+}.
    \item Apply the rule: Since \ch{Ca^2+} is on the exception list, \ch{CaSO4} is insoluble.
\end{enumerate}

\textbf{Answer:} \textbf{Insoluble (s)}.

\subsubsection{Example 2}
\textbf{Problem:} Classify \ch{K2CO3} as soluble or insoluble in water.

\textbf{Source:} lecture\_slides.pdf, slide 36, Problem c.

\textbf{Analysis:} This compound contains the potassium ion (\ch{K+}) and the carbonate ion (\ch{CO3^2-}). We can use the rule for either ion, but the rule for Group 1 cations is the most straightforward as it has no exceptions.

\textbf{Step-by-Step Solution:}
\begin{enumerate}
    \item Identify the ions: \ch{K+} and \ch{CO3^2-}.
    \item Recall the Group 1 cation rule: All compounds containing Group 1 cations are soluble.
    \item Apply the rule: Since the compound contains \ch{K+}, it must be soluble.
    \item (Alternative) Recall the carbonate rule: \ch{CO3^2-} compounds are generally insoluble, EXCEPT with Group 1 cations and \ch{NH4+}. This also leads to the conclusion that \ch{K2CO3} is soluble.
\end{enumerate}

\textbf{Answer:} \textbf{Soluble (aq)}.

\subsubsection{Example 3}
\textbf{Problem:} Characterize the compound \ch{Hg(NO3)2} as soluble or insoluble in water.

\textbf{Source:} Questions and Problems (textbook).pdf, Chapter 9, Problem 9.22c.

\textbf{Analysis:} This compound contains the mercury(II) ion (\ch{Hg^2+}) and the nitrate ion (\ch{NO3-}). The rule for nitrate is one of the "always soluble" rules.

\textbf{Step-by-Step Solution:}
\begin{enumerate}
    \item Identify the ions: \ch{Hg^2+} and \ch{NO3-}.
    \item Recall the nitrate rule: All nitrate compounds are soluble.
    \item Apply the rule: Since the compound contains \ch{NO3-}, it is soluble.
\end{enumerate}

\textbf{Answer:} \textbf{Soluble (aq)}.

\subsubsection{Example 4}
\textbf{Problem:} Characterize the compound \ch{Mn(OH)2} as soluble or insoluble in water.

\textbf{Source:} Questions and Problems (textbook).pdf, Chapter 9, Problem 9.23b.

\textbf{Analysis:} This compound contains the manganese(II) ion (\ch{Mn^2+}) and the hydroxide ion (\ch{OH-}). We must apply the rule for hydroxides.

\textbf{Step-by-Step Solution:}
\begin{enumerate}
    \item Identify the ions: \ch{Mn^2+} and \ch{OH-}.
    \item Recall the hydroxide rule: Most hydroxide compounds are insoluble.
    \item Recall the exceptions: The main exceptions are Group 1 cations, \ch{NH4+}, and the heavy Group 2 cations (\ch{Ca^2+}, \ch{Sr^2+}, \ch{Ba^2+}).
    \item Apply the rule: Manganese (\ch{Mn^2+}) is not on the list of exceptions. Therefore, \ch{Mn(OH)2} is insoluble.
\end{enumerate}

\textbf{Answer:} \textbf{Insoluble (s)}.

\section{Topic 3: Using Expected Charges for Ionic Equations}
\subsection{Key Principles}
\begin{itemize}
    \item To write ionic equations, you must break soluble ionic compounds into their constituent ions with the correct charges and subscripts.
    \item The periodic table is a guide to the expected charges of main-group elements (\textit{lecture\_slides.pdf, slide 39}).
        \begin{itemize}
            \item Group 1 metals form +1 ions (\ch{Na+}).
            \item Group 2 metals form +2 ions (\ch{Ca^2+}).
            \item Group 17 halogens form -1 ions (\ch{Cl-}).
            \item Aluminum (Group 13) forms a +3 ion (\ch{Al^3+}).
        \end{itemize}
    \item \textbf{Deducing Polyatomic Ion Charges:} If you know the charge of one ion in a neutral compound, you can deduce the charge of the other. For example, in \ch{Fe(NO3)3}, there are three nitrate ions. Since the overall charge is 0, and we know each nitrate is -1 (total of -3), the single Fe ion must be +3 to balance it.
    \item Subscripts in the chemical formula become coefficients for the ions when dissociated (e.g., \ch{CaCl2(aq)} becomes \ch{Ca^2+(aq) + 2Cl-(aq)}).
\end{itemize}

\subsection{Worked Examples}
\subsubsection{Example 1}
\textbf{Problem:} How does \ch{Al(NO3)3(aq)} dissociate into ions for a complete ionic equation?

\textbf{Source:} Electrolytes and molecular/ionic equations.pdf, Problem 2a.

\textbf{Analysis:} Aluminum is in Group 13 and typically forms a +3 ion. The subscript '3' outside the parenthesis for nitrate applies to the nitrate ion. This subscript becomes the coefficient for the nitrate ion.

\textbf{Step-by-Step Solution:}
\begin{enumerate}
    \item Identify the cation: \ch{Al}. Based on its position in the periodic table, its expected charge is +3.
    \item Identify the anion: \ch{NO3}. This is the nitrate polyatomic ion.
    \item Balance the charges to confirm: One \ch{Al^3+} ion has a charge of +3. The subscript '3' indicates there are three \ch{NO3} ions. If each nitrate has a -1 charge, the total negative charge is $3 \times (-1) = -3$. The total charge of $(+3) + (-3) = 0$ confirms the ion charges are correct.
    \item Write the dissociation: The subscript for nitrate becomes a coefficient.
    \[ \ch{Al(NO3)3(aq) -> Al^3+(aq) + 3NO3-(aq)} \]
\end{enumerate}

\textbf{Answer:} \textbf{\ch{Al^3+(aq) + 3NO3-(aq)}}.

\subsubsection{Example 2}
\textbf{Problem:} How does \ch{Li3PO4(aq)} dissociate into ions?

\textbf{Source:} Electrolytes and molecular/ionic equations.pdf, Problem 2d.

\textbf{Analysis:} Lithium (\ch{Li}) is a Group 1 alkali metal, so its ion will have a +1 charge. There are three Li atoms, so this will produce three \ch{Li+} ions. We can use this information to deduce the charge of the phosphate ion.

\textbf{Step-by-Step Solution:}
\begin{enumerate}
    \item Identify the cation: \ch{Li}. As a Group 1 metal, its charge is +1.
    \item Determine the total positive charge: The subscript '3' on Li means there are three lithium ions, for a total positive charge of $3 \times (+1) = +3$.
    \item Deduce the anion's charge: The anion is the phosphate ion, \ch{PO4}. Since there is only one phosphate ion and the compound is neutral, its charge must be -3 to balance the +3 from the lithium ions.
    \item Write the dissociation: The subscript '3' for Li becomes its coefficient.
    \[ \ch{Li3PO4(aq) -> 3Li+(aq) + PO4^3-(aq)} \]
\end{enumerate}

\textbf{Answer:} \textbf{\ch{3Li+(aq) + PO4^3-(aq)}}.

\subsubsection{Example 3}
\textbf{Problem:} How does \ch{Ca(OH)2(aq)} dissociate into ions?

\textbf{Source:} Electrolytes and molecular/ionic equations.pdf, Problem 2f.

\textbf{Analysis:} Calcium (\ch{Ca}) is a Group 2 alkaline earth metal, so its ion will have a +2 charge. The subscript '2' applies to the hydroxide polyatomic ion.

\textbf{Step-by-Step Solution:}
\begin{enumerate}
    \item Identify the cation: \ch{Ca}. As a Group 2 metal, its charge is +2.
    \item Identify the anion: \ch{OH}. This is the hydroxide ion.
    \item Check the balance and write the dissociation: There is one \ch{Ca^2+} ion. The subscript '2' means there are two \ch{OH} ions. The total charge is $(+2) + 2 \times (-1) = 0$. The subscript '2' for hydroxide becomes its coefficient.
    \[ \ch{Ca(OH)2(aq) -> Ca^2+(aq) + 2OH-(aq)} \]
\end{enumerate}

\textbf{Answer:} \textbf{\ch{Ca^2+(aq) + 2OH-(aq)}}.

\subsubsection{Example 4}
\textbf{Problem:} How does \ch{CoBr3(aq)} dissociate into ions?

\textbf{Source:} Electrolytes and molecular/ionic equations.pdf, Problem 2e.

\textbf{Analysis:} Cobalt (\ch{Co}) is a transition metal, so its charge can vary. Bromine (\ch{Br}) is a Group 17 halogen, so its ion (\ch{Br-}) has a -1 charge. We can use the charge of the bromide ion to deduce the charge of the cobalt ion.

\textbf{Step-by-Step Solution:}
\begin{enumerate}
    \item Identify the anion: \ch{Br}. As a Group 17 halogen, its charge is -1.
    \item Determine the total negative charge: The subscript '3' on Br means there are three bromide ions, for a total negative charge of $3 \times (-1) = -3$.
    \item Deduce the cation's charge: The cation is \ch{Co}. Since there is only one cobalt ion and the compound is neutral, its charge must be +3 to balance the -3 from the bromide ions.
    \item Write the dissociation: The subscript '3' for Br becomes its coefficient.
    \[ \ch{CoBr3(aq) -> Co^3+(aq) + 3Br-(aq)} \]
\end{enumerate}

\textbf{Answer:} \textbf{\ch{Co^3+(aq) + 3Br-(aq)}}.

\section{Topic 4: Writing Complete and Net Ionic Equations}
\subsection{Key Principles}
\begin{itemize}
    \item This is a three-step process to reveal the actual chemical change in an aqueous reaction.
    \item \textbf{Step 1: Molecular Equation.} Write the balanced chemical equation, predicting products and using solubility rules to assign states (aq, s).
    \item \textbf{Step 2: Complete Ionic Equation.} Dissociate (break apart) ONLY the strong electrolytes that are aqueous (aq). Leave everything else (solids, liquids, gases, weak electrolytes) in its molecular form.
    \item \textbf{Step 3: Net Ionic Equation.} Identify spectator ions (ions that are identical on both sides of the complete ionic equation) and cancel them out. Write the remaining species.
    \item \textbf{Common Pitfall:} Do NOT dissociate solids (precipitates), even if they are ionic compounds. The (s) symbol is your instruction to leave it as a single unit.
\end{itemize}

\subsection{Worked Examples}
\subsubsection{Example 1}
\textbf{Problem:} Write the complete and net ionic equations for the reaction: \ch{2 NaI(aq) + Pb(NO3)2(aq) -> PbI2(s) + 2 NaNO3(aq)}

\textbf{Source:} Electrolytes and molecular/ionic equations.pdf, Problem 2b.

\textbf{Analysis:} The molecular equation is already balanced. We will identify all aqueous strong electrolytes and break them into their constituent ions to write the complete ionic equation. Then we will cancel spectator ions to find the net ionic equation.

\textbf{Step-by-Step Solution:}
\begin{enumerate}
    \item \textbf{Identify species to dissociate:} \ch{NaI}, \ch{Pb(NO3)2}, and \ch{NaNO3} are all aqueous strong electrolytes. \ch{PbI2} is a solid (s), so it will not be dissociated.
    \item \textbf{Write the Complete Ionic Equation:}
    \[ \ch{2Na+(aq) + 2I-(aq) + Pb^2+(aq) + 2NO3-(aq) -> PbI2(s) + 2Na+(aq) + 2NO3-(aq)} \]
    \item \textbf{Identify and Cancel Spectator Ions:} The spectator ions are \ch{Na+(aq)} and \ch{NO3-(aq)} because they appear on both sides.
    \[ \ch{\cancel{2Na+(aq)} + 2I-(aq) + Pb^2+(aq) + \cancel{2NO3-(aq)} -> PbI2(s) + \cancel{2Na+(aq)} + \cancel{2NO3-(aq)}} \]
    \item \textbf{Write the Net Ionic Equation:}
    \[ \ch{Pb^2+(aq) + 2I-(aq) -> PbI2(s)} \]
\end{enumerate}

\textbf{Answer:} Net Ionic Equation: \textbf{\ch{Pb^2+(aq) + 2I-(aq) -> PbI2(s)}}.

\subsubsection{Example 2}
\textbf{Problem:} Write the complete and net ionic equations for the reaction: \ch{3 Ba(C2H3O2)2(aq) + 2 Li3PO4(aq) -> 6 LiC2H3O2(aq) + Ba3(PO4)2(s)}

\textbf{Source:} Electrolytes and molecular/ionic equations.pdf, Problem 2d.

\textbf{Analysis:} This is another precipitation reaction. The process is the same: identify aqueous strong electrolytes, dissociate them, then cancel spectators.

\textbf{Step-by-Step Solution:}
\begin{enumerate}
    \item \textbf{Identify species to dissociate:} \ch{Ba(C2H3O2)2}, \ch{Li3PO4}, and \ch{LiC2H3O2} are aqueous strong electrolytes. \ch{Ba3(PO4)2} is a solid.
    \item \textbf{Write the Complete Ionic Equation:}
    \[ \ch{3Ba^2+(aq) + 6C2H3O2-(aq) + 6Li+(aq) + 2PO4^3-(aq) -> 6Li+(aq) + 6C2H3O2-(aq) + Ba3(PO4)2(s)} \]
    \item \textbf{Identify and Cancel Spectator Ions:} The spectator ions are \ch{Li+(aq)} and \ch{C2H3O2-(aq)}.
    \[ \ch{3Ba^2+(aq) + \cancel{6C2H3O2-(aq)} + \cancel{6Li+(aq)} + 2PO4^3-(aq) -> \cancel{6Li+(aq)} + \cancel{6C2H3O2-(aq)} + Ba3(PO4)2(s)} \]
    \item \textbf{Write the Net Ionic Equation:}
    \[ \ch{3Ba^2+(aq) + 2PO4^3-(aq) -> Ba3(PO4)2(s)} \]
\end{enumerate}

\textbf{Answer:} Net Ionic Equation: \textbf{\ch{3Ba^2+(aq) + 2PO4^3-(aq) -> Ba3(PO4)2(s)}}.

\subsubsection{Example 3}
\textbf{Problem:} Write the complete and net ionic equations for the reaction: \ch{3 Ca(OH)2(aq) + 2 FeCl3(aq) -> 3 CaCl2(aq) + 2 Fe(OH)3(s)}

\textbf{Source:} Electrolytes and molecular/ionic equations.pdf, Problem 2f.

\textbf{Analysis:} Following the same procedure. Note that \ch{Ca(OH)2} is a strong base and therefore a strong electrolyte.

\textbf{Step-by-Step Solution:}
\begin{enumerate}
    \item \textbf{Identify species to dissociate:} \ch{Ca(OH)2}, \ch{FeCl3}, and \ch{CaCl2} are aqueous strong electrolytes. \ch{Fe(OH)3} is a solid.
    \item \textbf{Write the Complete Ionic Equation:}
    \[ \ch{3Ca^2+(aq) + 6OH-(aq) + 2Fe^3+(aq) + 6Cl-(aq) -> 3Ca^2+(aq) + 6Cl-(aq) + 2Fe(OH)3(s)} \]
    \item \textbf{Identify and Cancel Spectator Ions:} The spectator ions are \ch{Ca^2+(aq)} and \ch{Cl-(aq)}.
    \[ \ch{\cancel{3Ca^2+(aq)} + 6OH-(aq) + 2Fe^3+(aq) + \cancel{6Cl-(aq)} -> \cancel{3Ca^2+(aq)} + \cancel{6Cl-(aq)} + 2Fe(OH)3(s)} \]
    \item \textbf{Write the Net Ionic Equation and reduce coefficients:}
    \[ \ch{2Fe^3+(aq) + 6OH-(aq) -> 2Fe(OH)3(s)} \]
    All coefficients are divisible by 2, so we reduce them to the simplest whole-number ratio.
    \[ \ch{Fe^3+(aq) + 3OH-(aq) -> Fe(OH)3(s)} \]
\end{enumerate}

\textbf{Answer:} Net Ionic Equation: \textbf{\ch{Fe^3+(aq) + 3OH-(aq) -> Fe(OH)3(s)}}.

\subsubsection{Example 4}
\textbf{Problem:} Write the ionic and net ionic equations for the combination of aqueous solutions of \ch{LiNO3} and \ch{NaC2H3O2}.

\textbf{Source:} Questions and Problems (textbook).pdf, Chapter 9, Problem 9.20.

\textbf{Analysis:} First, we must write the molecular equation by predicting the products of the double displacement reaction. Then, we check the solubility of the products. If both products are soluble, no reaction occurs.

\textbf{Step-by-Step Solution:}
\begin{enumerate}
    \item \textbf{Predict Products:} The cations (\ch{Li+} and \ch{Na+}) swap anions (\ch{NO3-} and \ch{C2H3O2-}). The products are \ch{LiC2H3O2} and \ch{NaNO3}.
    \item \textbf{Check Solubility:} According to the rules, compounds with Group 1 cations (\ch{Li+}, \ch{Na+}) are always soluble. Compounds with nitrate (\ch{NO3-}) are always soluble. Therefore, both products are soluble (aq).
    \item \textbf{Write Molecular Equation:}
    \[ \ch{LiNO3(aq) + NaC2H3O2(aq) -> LiC2H3O2(aq) + NaNO3(aq)} \]
    \item \textbf{Write Complete Ionic Equation:} Since all reactants and products are aqueous strong electrolytes, everything dissociates.
    \[ \ch{Li+(aq) + NO3-(aq) + Na+(aq) + C2H3O2-(aq) -> Li+(aq) + C2H3O2-(aq) + Na+(aq) + NO3-(aq)} \]
    \item \textbf{Identify Spectator Ions:} Every ion is a spectator.
\end{enumerate}

\textbf{Answer:} \textbf{No net ionic equation. No reaction occurs.}

\section{Topic 5: Strong Acids and Bases}
\subsection{Key Principles}
\begin{itemize}
    \item Strong acids and strong bases are strong electrolytes because they dissociate 100\% in water.
    \item An acid is considered "weak" if it is not one of the seven strong acids. A base is considered "weak" if it is not one of the strong bases.
    \item \textbf{Memorization Required:} You must know the lists of strong acids and strong bases by heart to classify compounds and to correctly write ionic equations.
    \item \textbf{The 7 Strong Acids:} (\textit{lecture\_slides.pdf, slide 33})
        \begin{itemize}
            \item \ch{HCl} (Hydrochloric acid)
            \item \ch{HBr} (Hydrobromic acid)
            \item \ch{HI} (Hydroiodic acid)
            \item \ch{HNO3} (Nitric acid)
            \item \ch{H2SO4} (Sulfuric acid)
            \item \ch{HClO4} (Perchloric acid)
            \item \ch{HClO3} (Chloric acid)
        \end{itemize}
    \item \textbf{The Strong Bases:} (\textit{lecture\_slides.pdf, slide 33})
        \begin{itemize}
            \item Group 1 Hydroxides: \ch{LiOH}, \ch{NaOH}, \ch{KOH}, \ch{RbOH}, \ch{CsOH}
            \item Heavy Group 2 Hydroxides: \ch{Ca(OH)2}, \ch{Sr(OH)2}, \ch{Ba(OH)2}
        \end{itemize}
\end{itemize}

\subsection{Worked Examples}
\subsubsection{Example 1}
\textbf{Problem:} Identify \ch{NH3} as a weak or strong acid or base.

\textbf{Source:} Questions and Problems (textbook).pdf, Chapter 9, Problem 9.33a.

\textbf{Analysis:} \ch{NH3} (ammonia) is a common base. We need to check if it is on the list of strong bases.

\textbf{Step-by-Step Solution:}
\begin{enumerate}
    \item Identify compound type: \ch{NH3} is a base.
    \item Check strong base list: The strong bases are the hydroxides of Group 1 and heavy Group 2 metals.
    \item Classify: \ch{NH3} is not on this list. Therefore, it is a weak base.
\end{enumerate}

\textbf{Answer:} \textbf{Weak base}.

\subsubsection{Example 2}
\textbf{Problem:} Identify \ch{H3PO4} as a weak or strong acid or base.

\textbf{Source:} Questions and Problems (textbook).pdf, Chapter 9, Problem 9.33b.

\textbf{Analysis:} The formula starts with H, so \ch{H3PO4} (phosphoric acid) is an acid. We must check if it is on the list of seven strong acids.

\textbf{Step-by-Step Solution:}
\begin{enumerate}
    \item Identify compound type: Starts with H, so it's an acid.
    \item Check strong acid list: \ch{HCl}, \ch{HBr}, \ch{HI}, \ch{HNO3}, \ch{H2SO4}, \ch{HClO4}, \ch{HClO3}.
    \item Classify: \ch{H3PO4} is not on the list. Therefore, it is a weak acid.
\end{enumerate}

\textbf{Answer:} \textbf{Weak acid}.

\subsubsection{Example 3}
\textbf{Problem:} Identify \ch{LiOH} as a weak or strong acid or base.

\textbf{Source:} Questions and Problems (textbook).pdf, Chapter 9, Problem 9.33c.

\textbf{Analysis:} This compound contains the hydroxide ion (\ch{OH-}), so it is a base. Lithium (\ch{Li}) is a Group 1 metal. We check the strong base list.

\textbf{Step-by-Step Solution:}
\begin{enumerate}
    \item Identify compound type: Contains \ch{OH-}, so it's a base.
    \item Check strong base list: The hydroxides of Group 1 metals are strong bases.
    \item Classify: \ch{LiOH} is on the list. Therefore, it is a strong base.
\end{enumerate}

\textbf{Answer:} \textbf{Strong base}.

\subsubsection{Example 4}
\textbf{Problem:} Identify \ch{H2SO4} as a weak or strong acid or base.

\textbf{Source:} Questions and Problems (textbook).pdf, Chapter 9, Problem 9.33e.

\textbf{Analysis:} The formula starts with H, so \ch{H2SO4} (sulfuric acid) is an acid. We check the list of seven strong acids.

\textbf{Step-by-Step Solution:}
\begin{enumerate}
    \item Identify compound type: Starts with H, so it's an acid.
    \item Check strong acid list: \ch{HCl}, \ch{HBr}, \ch{HI}, \ch{HNO3}, \ch{H2SO4}, \ch{HClO4}, \ch{HClO3}.
    \item Classify: \ch{H2SO4} is on the list. Therefore, it is a strong acid.
\end{enumerate}

\textbf{Answer:} \textbf{Strong acid}.

\section{Topic 6: Balancing Redox Reactions \& Identifying Agents}
\subsection{Key Principles}
\begin{itemize}
    \item Redox reactions must be balanced for both mass (atoms) and charge.
    \item \textbf{OIL RIG:} Oxidation Is Loss of electrons (oxidation number goes up); Reduction Is Gain of electrons (oxidation number goes down).
    \item \textbf{Oxidizing Agent:} The reactant that \textit{causes} oxidation. It does so by being reduced itself.
    \item \textbf{Reducing Agent:} The reactant that \textit{causes} reduction. It does so by being oxidized itself.
    \item \textbf{Important Note:} Oxidizing and reducing agents are \textbf{ALWAYS REACTANTS}. Never choose a product.
    \item \textbf{Half-Reaction Method:}
        \begin{enumerate}
            \item Assign oxidation numbers to identify what is oxidized and what is reduced.
            \item Write two half-reactions (oxidation and reduction).
            \item Balance all elements except O and H.
            \item Balance charge by adding electrons (\ch{e-}) to the more positive side.
            \item Multiply the half-reactions by integers to make the number of electrons in each equal.
            \item Add the balanced half-reactions and cancel the electrons.
        \end{enumerate}
\end{itemize}

\subsection{Worked Examples}
\subsubsection{Example 1}
\textbf{Problem:} Balance the reaction using the half-reaction method: \ch{Cr(s) + Ni^2+(aq) -> Cr^3+(aq) + Ni(s)}

\textbf{Source:} Oxidation states and redox reactions.pdf, Problem 3a.

\textbf{Analysis:} We will separate this into two half-reactions. For each, we balance the charge by adding electrons. Then we will scale the reactions so the electrons cancel when the reactions are combined.

\textbf{Step-by-Step Solution:}
\begin{enumerate}
    \item \textbf{Separate into half-reactions:}
        \begin{itemize}
            \item Oxidation: \ch{Cr(s) -> Cr^3+(aq)} (Oxidation number from 0 to +3)
            \item Reduction: \ch{Ni^2+(aq) -> Ni(s)} (Oxidation number from +2 to 0)
        \end{itemize}
    \item \textbf{Balance charge with electrons:}
        \begin{itemize}
            \item Oxidation: \ch{Cr(s) -> Cr^3+(aq) + 3e-} (Add 3e- to the product side to balance the +3 charge)
            \item Reduction: \ch{Ni^2+(aq) + 2e- -> Ni(s)} (Add 2e- to the reactant side to balance the +2 charge)
        \end{itemize}
    \item \textbf{Equalize electrons:} The least common multiple of 3 and 2 is 6. Multiply the oxidation reaction by 2 and the reduction reaction by 3.
        \begin{itemize}
            \item $2 \times (\ch{Cr(s) -> Cr^3+(aq) + 3e-}) \implies \ch{2Cr(s) -> 2Cr^3+(aq) + 6e-}$
            \item $3 \times (\ch{Ni^2+(aq) + 2e- -> Ni(s)}) \implies \ch{3Ni^2+(aq) + 6e- -> 3Ni(s)}$
        \end{itemize}
    \item \textbf{Combine and cancel:} Add the two new half-reactions. The \ch{6e-} on each side cancel.
    \[ \ch{2Cr(s) + 3Ni^2+(aq) + \cancel{6e-} -> 2Cr^3+(aq) + 3Ni(s) + \cancel{6e-}} \]
\end{enumerate}

\textbf{Answer:} \textbf{\ch{2Cr(s) + 3Ni^2+(aq) -> 2Cr^3+(aq) + 3Ni(s)}}.

\subsubsection{Example 2}
\textbf{Problem:} For the reaction \ch{2Fe + V2O3 -> Fe2O3 + 2V}, identify the species oxidized, reduced, oxidizing agent, and reducing agent.

\textbf{Source:} Oxidation states and redox reactions.pdf, Problem 2b.

\textbf{Analysis:} We must assign oxidation numbers to all atoms, identify the changes, and then assign the roles and agents based on those changes.

\textbf{Step-by-Step Solution:}
\begin{enumerate}
    \item \textbf{Assign oxidation numbers:}
        \begin{itemize}
            \item Reactants: \ch{Fe} is 0. In \ch{V2O3}, O is -2, so $2x + 3(-2) = 0 \implies x = +3$ for V.
            \item Products: In \ch{Fe2O3}, O is -2, so $2x + 3(-2) = 0 \implies x = +3$ for Fe. \ch{V} is 0.
        \end{itemize}
    \item \textbf{Identify changes:}
        \begin{itemize}
            \item Iron: Fe(0) $\rightarrow$ Fe(+3). Increase in oxidation number $\implies$ \textbf{Fe is oxidized}.
            \item Vanadium: V(+3) $\rightarrow$ V(0). Decrease in oxidation number $\implies$ \textbf{\ch{V^3+} is reduced}.
        \end{itemize}
    \item \textbf{Identify agents:}
        \begin{itemize}
            \item The reducing agent is what was oxidized: \textbf{Fe is the reducing agent}.
            \item The oxidizing agent is what was reduced: \textbf{\ch{V^3+} (in \ch{V2O3}) is the oxidizing agent}.
        \end{itemize}
\end{enumerate}

\textbf{Answer:} \textbf{Fe is oxidized (reducing agent); \ch{V^3+} is reduced (oxidizing agent).}

\subsubsection{Example 3}
\textbf{Problem:} Balance the reaction using the half-reaction method: \ch{Sn(s) + Cu^2+(aq) -> Sn^4+(aq) + Cu(s)}

\textbf{Source:} Oxidation states and redox reactions.pdf, Problem 3b.

\textbf{Analysis:} Separate into half-reactions, balance charge with electrons, find the lowest common multiple for the electrons, and recombine.

\textbf{Step-by-Step Solution:}
\begin{enumerate}
    \item \textbf{Separate into half-reactions:}
        \begin{itemize}
            \item Oxidation: \ch{Sn(s) -> Sn^4+(aq)} (0 to +4)
            \item Reduction: \ch{Cu^2+(aq) -> Cu(s)} (+2 to 0)
        \end{itemize}
    \item \textbf{Balance charge with electrons:}
        \begin{itemize}
            \item \ch{Sn(s) -> Sn^4+(aq) + 4e-}
            \item \ch{Cu^2+(aq) + 2e- -> Cu(s)}
        \end{itemize}
    \item \textbf{Equalize electrons:} The LCM of 4 and 2 is 4. Multiply the reduction reaction by 2.
        \begin{itemize}
            \item \ch{Sn(s) -> Sn^4+(aq) + 4e-}
            \item $2 \times (\ch{Cu^2+(aq) + 2e- -> Cu(s)}) \implies \ch{2Cu^2+(aq) + 4e- -> 2Cu(s)}$
        \end{itemize}
    \item \textbf{Combine and cancel electrons:}
     \[ \ch{Sn(s) + 2Cu^2+(aq) + \cancel{4e-} -> Sn^4+(aq) + 2Cu(s) + \cancel{4e-}} \]
\end{enumerate}

\textbf{Answer:} \textbf{\ch{Sn(s) + 2Cu^2+(aq) -> Sn^4+(aq) + 2Cu(s)}}.

\subsubsection{Example 4}
\textbf{Problem:} For the reaction \ch{8Ba + S8 -> 8BaS}, identify the species oxidized, reduced, oxidizing agent, and reducing agent.

\textbf{Source:} Oxidation states and redox reactions.pdf, Problem 2c.

\textbf{Analysis:} Assign oxidation numbers to the elemental forms and the ionic product to determine what is oxidized and reduced.

\textbf{Step-by-Step Solution:}
\begin{enumerate}
    \item \textbf{Assign oxidation numbers:}
        \begin{itemize}
            \item Reactants: \ch{Ba} and \ch{S8} are both elements in their standard state, so their oxidation numbers are 0.
            \item Product: \ch{BaS} is an ionic compound. Ba is a Group 2 metal, so it is +2. Sulfur must then be -2.
        \end{itemize}
    \item \textbf{Identify changes:}
        \begin{itemize}
            \item Barium: Ba(0) $\rightarrow$ Ba(+2). Increase in oxidation number $\implies$ \textbf{Ba is oxidized}.
            \item Sulfur: S(0) $\rightarrow$ S(-2). Decrease in oxidation number $\implies$ \textbf{\ch{S8} is reduced}.
        \end{itemize}
    \item \textbf{Identify agents:}
        \begin{itemize}
            \item The reducing agent is what was oxidized: \textbf{Ba is the reducing agent}.
            \item The oxidizing agent is what was reduced: \textbf{\ch{S8} is the oxidizing agent}.
        \end{itemize}
\end{enumerate}

\textbf{Answer:} \textbf{Ba is oxidized (reducing agent); \ch{S8} is reduced (oxidizing agent).}

\section{Topic 7: Assigning Oxidation Numbers}
\subsection{Key Principles}
\begin{itemize}
    \item Oxidation numbers (or states) track electron distribution in compounds and are essential for identifying redox reactions.
    \item \textbf{Memorization Required:} The rules must be applied in hierarchical order.
    \item \textbf{Summary of Rules} (\textit{lecture\_slides.pdf, slides 48-49}):
        \begin{enumerate}
            \item An atom in its elemental form has an oxidation number of 0 (e.g., \ch{Fe}, \ch{O2}, \ch{S8}).
            \item The oxidation number of a monatomic ion is equal to its charge (e.g., \ch{Na+} is +1, \ch{S^2-} is -2).
            \item The oxidation number of oxygen in most compounds is -2. (Exception: -1 in peroxides like \ch{H2O2}).
            \item The oxidation number of hydrogen is +1 with nonmetals and -1 with metals.
            \item The oxidation number of fluorine is always -1. Other halogens are usually -1, unless bonded to a more electronegative atom (like oxygen).
            \item The sum of oxidation numbers for a neutral compound is 0.
            \item The sum of oxidation numbers for a polyatomic ion equals the ion's charge.
        \end{enumerate}
\end{itemize}

\subsection{Worked Examples}
\subsubsection{Example 1}
\textbf{Problem:} State the oxidation state for S in \ch{HSO4-}.

\textbf{Source:} Oxidation states and redox reactions.pdf, Problem 1j.

\textbf{Analysis:} This is a polyatomic ion with an overall charge of -1. We use the known rules for H and O, and solve for S, setting the sum of all oxidation numbers equal to -1.

\textbf{Step-by-Step Solution:}
\begin{enumerate}
    \item The overall charge of the ion is -1.
    \item Assign known oxidation numbers: H is +1, O is -2.
    \item Let the oxidation number of S be $x$.
    \item Set up the equation: $(+1) + (x) + 4 \times (-2) = -1$.
    \item Solve for $x$: $1 + x - 8 = -1 \implies x - 7 = -1 \implies x = +6$.
\end{enumerate}

\textbf{Answer:} \textbf{+6}.

\subsubsection{Example 2}
\textbf{Problem:} State the oxidation state for C in \ch{C2O4^2-}.

\textbf{Source:} Oxidation states and redox reactions.pdf, Problem 1u.

\textbf{Analysis:} This is the oxalate polyatomic ion with a -2 charge. We will use the rule for oxygen and the overall charge to solve for the oxidation state of carbon.

\textbf{Step-by-Step Solution:}
\begin{enumerate}
    \item The overall charge of the ion is -2.
    \item Assign known oxidation numbers: O is -2.
    \item Let the oxidation number of a single C atom be $x$. Since there are two C atoms, their total contribution is $2x$.
    \item Set up the equation: $2x + 4 \times (-2) = -2$.
    \item Solve for $x$: $2x - 8 = -2 \implies 2x = +6 \implies x = +3$.
\end{enumerate}

\textbf{Answer:} \textbf{+3}.

\subsubsection{Example 3}
\textbf{Problem:} State the oxidation state for Fe in \ch{Fe(ClO2)3}.

\textbf{Source:} Oxidation states and redox reactions.pdf, Problem 1k and 1x.

\textbf{Analysis:} This is a neutral compound. We need to find the charge of the chlorite polyatomic ion (\ch{ClO2}) first, then use that to find the oxidation state of Fe. Inside the chlorite ion, O is -2, allowing us to find Cl.

\textbf{Step-by-Step Solution:}
\begin{enumerate}
    \item \textbf{Find the oxidation state of Cl in \ch{ClO2-}:} The chlorite ion has a -1 charge. Let Cl be $y$. Then $y + 2(-2) = -1 \implies y - 4 = -1 \implies y = +3$. The oxidation state of Cl is +3.
    \item \textbf{Find the oxidation state of Fe:} The compound is \ch{Fe(ClO2)3}. It is neutral, so the sum of oxidation numbers is 0.
    \item We know the charge of the entire chlorite ion is -1. There are three of them, for a total of -3.
    \item Let the oxidation number of Fe be $x$.
    \item Set up the equation for the overall compound: $x + 3 \times (-1) = 0$.
    \item Solve for $x$: $x - 3 = 0 \implies x = +3$.
\end{enumerate}

\textbf{Answer:} Cl in \ch{Fe(ClO2)3} is \textbf{+3}; Fe in \ch{Fe(ClO2)3} is \textbf{+3}.

\subsubsection{Example 4}
\textbf{Problem:} State the oxidation state for C in \ch{CN-}.

\textbf{Source:} Oxidation states and redox reactions.pdf, Problem 1r.

\textbf{Analysis:} This is the cyanide ion with a -1 charge. This case is ambiguous without more rules. However, in most conventions, nitrogen is more electronegative than carbon, so it is assigned its typical ionic charge of -3. We can then solve for carbon.

\textbf{Step-by-Step Solution:}
\begin{enumerate}
    \item The overall charge is -1.
    \item Assign the more electronegative element its charge: Nitrogen is typically -3.
    \item Let the oxidation number of C be $x$.
    \item Set up the equation: $x + (-3) = -1$.
    \item Solve for $x$: $x = +2$.
\end{enumerate}

\textbf{Answer:} \textbf{+2}.

\section{Topic 8: Using the Reactivity Series}
\subsection{Key Principles}
\begin{itemize}
    \item The \textbf{Activity Series} is a list of metals (and hydrogen) arranged in order of decreasing ease of oxidation. Metals at the top are most reactive (easiest to oxidize); metals at the bottom are least reactive.
    \item \textbf{Memorization Required:} You need to know the relative positions of common metals in the activity series to predict reactions. The full series is on \textit{lecture\_slides.pdf, slide 55}:
    \[ \ch{Li > K > Ba > Ca > Na > Mg > Al > Mn > Zn > Fe > Co > Ni > Sn > Pb > (H) > Cu > Ag > Hg > Pt > Au} \]
    \item \textbf{Prediction Rule:} A metal higher on the list that is in its elemental form (solid) will displace a metal ion from a compound if that ion is from a metal lower on the list.
    \item In short: \textbf{Elemental Metal (Higher) + Ionic Compound (Lower) $\rightarrow$ Reaction}
    \item \textbf{Elemental Metal (Lower) + Ionic Compound (Higher) $\rightarrow$ No Reaction}
\end{itemize}

\subsection{Worked Examples}
\subsubsection{Example 1}
\textbf{Problem:} Which of the following reactions will occur? \ch{Co(s) + BaI2(aq) -> ?}

\textbf{Source:} lecture\_slides.pdf, slide 56, Problem a.

\textbf{Analysis:} We must compare the positions of the elemental metal (Cobalt, Co) and the metal in the ionic compound (Barium, Ba) in the activity series.

\textbf{Step-by-Step Solution:}
\begin{enumerate}
    \item Identify the elemental metal: \ch{Co}.
    \item Identify the metal in the compound: \ch{Ba}.
    \item Compare their positions in the activity series: Barium (Ba) is very high on the list. Cobalt (Co) is much lower, below iron.
    \item Apply the rule: The elemental metal (\ch{Co}) is lower than the metal in the compound (\ch{Ba}). Therefore, no reaction will occur.
\end{enumerate}

\textbf{Answer:} \textbf{No reaction}.

\subsubsection{Example 2}
\textbf{Problem:} Which of the following reactions will occur? \ch{Sn(s) + CuBr2(aq) -> ?}

\textbf{Source:} lecture\_slides.pdf, slide 56, Problem b.

\textbf{Analysis:} Compare the positions of the elemental metal (Tin, Sn) and the metal in the ionic compound (Copper, Cu) in the activity series.

\textbf{Step-by-Step Solution:}
\begin{enumerate}
    \item Identify the elemental metal: \ch{Sn}.
    \item Identify the metal in the compound: \ch{Cu}.
    \item Compare their positions in the activity series: Tin (Sn) is above Hydrogen. Copper (Cu) is below Hydrogen. Sn is higher than Cu.
    \item Apply the rule: The elemental metal (\ch{Sn}) is higher than the metal in the compound (\ch{Cu}). Therefore, a reaction will occur. Tin will be oxidized to \ch{Sn^2+} and copper will be reduced to \ch{Cu(s)}. The products will be \ch{SnBr2(aq)} and \ch{Cu(s)}.
\end{enumerate}

\textbf{Answer:} \textbf{Reaction occurs to form \ch{Cu(s) + SnBr2(aq)}}.

\subsubsection{Example 3}
\textbf{Problem:} Predict the outcome of the reaction: \ch{Mg(s) + CuSO4(aq) ->}

\textbf{Source:} Questions and Problems (textbook).pdf, Chapter 9, Problem 9.54c.

\textbf{Analysis:} Compare the positions of elemental Magnesium (\ch{Mg}) and the metal ion in the compound, Copper (\ch{Cu^2+}).

\textbf{Step-by-Step Solution:}
\begin{enumerate}
    \item Identify the elemental metal: \ch{Mg}.
    \item Identify the metal in the compound: \ch{Cu}.
    \item Compare their positions in the activity series: Mg is very high on the list. Cu is low on the list (below Hydrogen).
    \item Apply the rule: The elemental metal (\ch{Mg}) is higher than the metal in the compound (\ch{Cu}). A reaction will occur.
\end{enumerate}

\textbf{Answer:} \textbf{Reaction occurs. Products are \ch{MgSO4(aq) + Cu(s)}}.

\subsubsection{Example 4}
\textbf{Problem:} Predict the outcome of the reaction: \ch{Cu(s) + HCl(aq) ->}

\textbf{Source:} Questions and Problems (textbook).pdf, Chapter 9, Problem 9.54a.

\textbf{Analysis:} This involves a metal reacting with an acid. The acid provides \ch{H+} ions. We compare the position of the elemental metal (Copper, Cu) with Hydrogen (H) in the activity series.

\textbf{Step-by-Step Solution:}
\begin{enumerate}
    \item Identify the elemental metal: \ch{Cu}.
    \item Identify the "metal" in the compound (acid): Hydrogen (\ch{H}).
    \item Compare their positions in the activity series: Copper (Cu) is below Hydrogen.
    \item Apply the rule: The elemental metal (\ch{Cu}) is lower than Hydrogen. Therefore, copper cannot displace hydrogen from an acid. No reaction will occur.
\end{enumerate}

\textbf{Answer:} \textbf{No reaction}.

\end{document}