\documentclass{article}

\usepackage{amsmath}
\usepackage{amssymb}
\usepackage[margin=1in]{geometry}
\usepackage{chemformula}

\title{Homework 10: Energy Changes in Chemical Reactions}
\author{Tashfeen Omran}
\date{October 2025}

\begin{document}

\maketitle

\section*{Quiz Details}
\begin{itemize}
    \item \textbf{When:} Monday, November 3, 2025, during class.
    \item \textbf{Format:} Closed notes.
    \item \textbf{Allowed Materials:} A scientific calculator. Phones are not permitted.
\end{itemize}

\section{Topic 1: The First Law of Thermodynamics}
\subsection{Key Principles}
\begin{itemize}
    \item The First Law of Thermodynamics is the law of conservation of energy. It states that the total energy of the universe is constant. Energy cannot be created or destroyed, only transferred between a \textbf{system} (the part of the universe we are studying) and its \textbf{surroundings} (everything else).
    \item The \textbf{internal energy ($U$)} of a system is the sum of all kinetic and potential energies of its components. We can't measure the absolute value of $U$, but we can measure the change in internal energy, $\Delta U$.
    \item The change in internal energy is the sum of heat ($q$) transferred and work ($w$) done:
    \[ \Delta U = q + w \]
    \item Any energy lost by the system must be gained by the surroundings, and vice versa. This is a fundamental concept for all energy transfer calculations.
    \[ \Delta U_{sys} = -\Delta U_{surr} \]
\end{itemize}

\subsection{Worked Examples}
\subsubsection{Example 1}
\textbf{Problem:} The internal energy of a system increased by 982 J when it absorbed 492 J of heat. How much work was done? Was work done by or on the system?
\textbf{Source:} Internal energy, enthalpy, thermochemical equations.pdf, Problem 1.
\textbf{Analysis:} This problem directly applies the First Law of Thermodynamics, $\Delta U = q + w$. We are given the change in internal energy ($\Delta U$) and the heat transferred ($q$). We must use the correct signs for these values and then solve for work ($w$).
\textbf{Step-by-Step Solution:}
\begin{enumerate}
    \item Identify the given values with their correct signs based on the problem description. "Internal energy... increased by 982 J" means $\Delta U = +982$ J. "it absorbed 492 J of heat" means $q = +492$ J.
    \item Rearrange the First Law equation to solve for work: $w = \Delta U - q$.
    \item Substitute the values: $w = (+982 \text{ J}) - (+492 \text{ J}) = +490$ J.
    \item The sign of $w$ is positive, which means work was done \textbf{on} the system.
\end{enumerate}
\textbf{Answer:} \textbf{+490 J; Work was done on the system}.

\subsubsection{Example 2}
\textbf{Problem:} A gas expands and does PV work on the surroundings equal to 325 J. At the same time, it absorbs 127 J of heat from the surroundings. Calculate the change in energy of the gas.
\textbf{Source:} Questions and Problems (textbook).pdf, Chapter 10, Problem 10.22.
\textbf{Analysis:} We need to calculate $\Delta U$ using the First Law. This requires assigning the correct signs to the work and heat based on the description.
\textbf{Step-by-Step Solution:}
\begin{enumerate}
    \item "does PV work on the surroundings equal to 325 J" means the system is doing the work, so $w = -325$ J.
    \item "it absorbs 127 J of heat" means heat flows into the system, so $q = +127$ J.
    \item Apply the First Law equation: $\Delta U = q + w$.
    \item Substitute the values: $\Delta U = (+127 \text{ J}) + (-325 \text{ J}) = -198$ J.
\end{enumerate}
\textbf{Answer:} \textbf{-198 J}.

\subsubsection{Example 3}
\textbf{Problem:} In a reaction, 157 kJ of energy flows into a system. Calculate the work for the system if it absorbs 260 kJ of heat. Is the work done on or by the system?
\textbf{Source:} Internal energy, enthalpy, thermochemical equations.pdf, Problem 3.
\textbf{Analysis:} This problem provides the total change in internal energy and the heat, asking for the work. The wording "energy flows into a system" refers to the overall change, $\Delta U$.
\textbf{Step-by-Step Solution:}
\begin{enumerate}
    \item "157 kJ of energy flows into a system" means the total internal energy increased, so $\Delta U = +157$ kJ.
    \item "it absorbs 260 kJ of heat" means $q = +260$ kJ.
    \item Rearrange the First Law equation: $w = \Delta U - q$.
    \item Substitute the values: $w = (+157 \text{ kJ}) - (+260 \text{ kJ}) = -103$ kJ.
    \item The sign of $w$ is negative, which means work was done \textbf{by} the system.
\end{enumerate}
\textbf{Answer:} \textbf{-103 kJ; Work was done by the system}.

\subsubsection{Example 4}
\textbf{Problem:} The work done to compress a gas is 112 J. As a result, 51 J of heat is given off to the surroundings. Calculate the change in energy of the gas.
\textbf{Source:} Questions and Problems (textbook).pdf, Chapter 10, Problem 10.11.
\textbf{Analysis:} A direct calculation of $\Delta U$ after assigning the correct signs to $q$ and $w$ from the problem statement.
\textbf{Step-by-Step Solution:}
\begin{enumerate}
    \item "work done to compress a gas is 112 J" means work is done on the system, so $w = +112$ J.
    \item "51 J of heat is given off" means heat flows out of the system, so $q = -51$ J.
    \item Apply the First Law equation: $\Delta U = q + w$.
    \item Substitute the values: $\Delta U = (-51 \text{ J}) + (+112 \text{ J}) = +61$ J.
\end{enumerate}
\textbf{Answer:} \textbf{+61 J}.

\section{Topic 2: Sign Conventions for Work (w) and Heat (q)}
\subsection{Key Principles}
\begin{itemize}
    \item \textbf{Memorization Required:} All sign conventions in thermodynamics are defined from the perspective of the \textbf{system}.
    \item \textbf{Heat (q):}
        \begin{itemize}
            \item \textbf{Positive (+):} Heat is absorbed BY the system FROM the surroundings. Keywords: "absorbs heat," "heat added to," "endothermic." The surroundings get colder.
            \item \textbf{Negative (-):} Heat is released FROM the system TO the surroundings. Keywords: "releases heat," "evolves heat," "gives off heat," "exothermic." The surroundings get warmer.
        \end{itemize}
    \item \textbf{Work (w):}
        \begin{itemize}
            \item \textbf{Positive (+):} Work is done ON the system BY the surroundings. This typically involves a decrease in the system's volume (compression).
            \item \textbf{Negative (-):} Work is done BY the system ON the surroundings. This typically involves an increase in the system's volume (expansion). Keywords: "expansion work," "does work."
        \end{itemize}
    \item A table summary is available in \textit{lecture\_slides.pdf, slide 6}.
\end{itemize}

\subsection{Worked Examples}
\subsubsection{Example 1}
\textbf{Problem:} Calculate $q$ and determine whether heat is absorbed or released when a system does work on the surroundings equal to 64 J and $\Delta U = -23$ J.
\textbf{Source:} Questions and Problems (textbook).pdf, Chapter 10, Problem 10.12.
\textbf{Analysis:} We are given $\Delta U$ and enough information to determine $w$ and its sign. We need to solve for $q$ and then interpret its sign.
\textbf{Step-by-Step Solution:}
\begin{enumerate}
    \item The system "does work on the surroundings," so work is negative: $w = -64$ J.
    \item The change in internal energy is given: $\Delta U = -23$ J.
    \item Rearrange the First Law equation: $q = \Delta U - w$.
    \item Substitute values: $q = (-23 \text{ J}) - (-64 \text{ J}) = -23 \text{ J} + 64 \text{ J} = +41$ J.
    \item The sign of $q$ is positive, so heat was \textbf{absorbed} by the system.
\end{enumerate}
\textbf{Answer:} \textbf{q = +41 J; heat is absorbed}.

\subsubsection{Example 2}
\textbf{Problem:} Calculate $w$ and determine whether work is done by the system or on the system when 98 J of heat is released and $\Delta U = -215$ J.
\textbf{Source:} Questions and Problems (textbook).pdf, Chapter 10, Problem 10.13.
\textbf{Analysis:} Similar to the previous example, but solving for $w$ and interpreting its sign.
\textbf{Step-by-Step Solution:}
\begin{enumerate}
    \item "98 J of heat is released" means heat leaves the system, so $q = -98$ J.
    \item The change in internal energy is given: $\Delta U = -215$ J.
    \item Rearrange the First Law equation: $w = \Delta U - q$.
    \item Substitute values: $w = (-215 \text{ J}) - (-98 \text{ J}) = -215 \text{ J} + 98 \text{ J} = -117$ J.
    \item The sign of $w$ is negative, so work was done \textbf{by the system}.
\end{enumerate}
\textbf{Answer:} \textbf{w = -117 J; work is done by the system}.

\subsubsection{Example 3}
\textbf{Problem:} In a gas expansion, 36 J of heat is absorbed from the surroundings and the energy of the system decreases by 182 J. Calculate the work done.
\textbf{Source:} Questions and Problems (textbook).pdf, Chapter 10, Problem 10.10.
\textbf{Analysis:} Identify the signs for $\Delta U$ and $q$ from the wording and solve for $w$.
\textbf{Step-by-Step Solution:}
\begin{enumerate}
    \item "heat is absorbed" means $q = +36$ J.
    \item "energy of the system decreases by 182 J" means $\Delta U = -182$ J.
    \item Rearrange the First Law equation: $w = \Delta U - q$.
    \item Substitute values: $w = (-182 \text{ J}) - (+36 \text{ J}) = -218$ J.
\end{enumerate}
\textbf{Answer:} \textbf{-218 J}.

\subsubsection{Example 4}
\textbf{Problem:} If a system does 30 J of work, and the internal energy is 20 J less than before, what is the heat exchanged by the system? Is the heat flowing into or out of the system?
\textbf{Source:} Internal energy, enthalpy, thermochemical equations.pdf, Problem 5.
\textbf{Analysis:} Interpret the wording to find the signs and values of $w$ and $\Delta U$, then solve for $q$.
\textbf{Step-by-Step Solution:}
\begin{enumerate}
    \item "system does 30 J of work" means work is done by the system, so $w = -30$ J.
    \item "internal energy is 20 J less than before" means the internal energy has decreased, so $\Delta U = -20$ J.
    \item Rearrange the First Law equation: $q = \Delta U - w$.
    \item Substitute values: $q = (-20 \text{ J}) - (-30 \text{ J}) = -20 \text{ J} + 30 \text{ J} = +10$ J.
    \item The sign of $q$ is positive, so heat is flowing \textbf{into} the system.
\end{enumerate}
\textbf{Answer:} \textbf{+10 J; heat is flowing into the system}.

\section{Topic 3: Calculating with the Internal Energy Equation}
\subsection{Key Principles}
\begin{itemize}
    \item The core equation is $\Delta U = q + w$. It can be algebraically rearranged to solve for any of the three variables.
    \item It is critical to ensure all units are consistent before adding or subtracting. Most often, this means converting everything to Joules (J) or kilojoules (kJ).
    \item Remember that $\Delta U$ is a \textbf{state function}, meaning its value depends only on the initial and final states of the system, not on the path taken (how the heat was transferred or the work was done). Heat ($q$) and work ($w$) are \textbf{not} state functions.
\end{itemize}

\subsection{Worked Examples}
\subsubsection{Example 1 (Solving for $\Delta U$)}
\textbf{Problem:} Calculate the change in internal energy (in J) for a system that releases 188 J of heat and has 141 J of work done on it by the surroundings.
\textbf{Source:} lecture\_slides.pdf, slide 8, Problem c.
\textbf{Analysis:} This is a direct calculation of $\Delta U$ using the provided values for $q$ and $w$ with their correct signs.
\textbf{Step-by-Step Solution:}
\begin{enumerate}
    \item "releases 188 J of heat" $\implies q = -188$ J.
    \item "has 141 J of work done on it" $\implies w = +141$ J.
    \item Use the equation $\Delta U = q + w$.
    \item Substitute values: $\Delta U = (-188 \text{ J}) + (+141 \text{ J}) = -47$ J.
\end{enumerate}
\textbf{Answer:} \textbf{-47 J}.

\subsubsection{Example 2 (Solving for $w$)}
\textbf{Problem:} Calculate the work for the system if it absorbs 260 kJ of heat and 157 kJ of energy flows into the system.
\textbf{Source:} Internal energy, enthalpy, thermochemical equations.pdf, Problem 3.
\textbf{Analysis:} Here we are given $\Delta U$ and $q$, and need to rearrange the equation to find $w$.
\textbf{Step-by-Step Solution:}
\begin{enumerate}
    \item "absorbs 260 kJ of heat" $\implies q = +260$ kJ.
    \item "157 kJ of energy flows into a system" $\implies \Delta U = +157$ kJ.
    \item Rearrange the formula: $w = \Delta U - q$.
    \item Substitute values: $w = (+157 \text{ kJ}) - (+260 \text{ kJ}) = -103$ kJ.
\end{enumerate}
\textbf{Answer:} \textbf{-103 kJ}.

\subsubsection{Example 3 (Solving for $q$)}
\textbf{Problem:} A system does 30 J of work and its internal energy decreases by 20 J. What is the heat exchanged?
\textbf{Source:} Internal energy, enthalpy, thermochemical equations.pdf, Problem 5.
\textbf{Analysis:} We are given $w$ and $\Delta U$ and must solve for $q$.
\textbf{Step-by-Step Solution:}
\begin{enumerate}
    \item "system does 30 J of work" $\implies w = -30$ J.
    \item "internal energy decreases by 20 J" $\implies \Delta U = -20$ J.
    \item Rearrange the formula: $q = \Delta U - w$.
    \item Substitute values: $q = (-20 \text{ J}) - (-30 \text{ J}) = +10$ J.
\end{enumerate}
\textbf{Answer:} \textbf{+10 J}.

\subsubsection{Example 4 (Solving for $\Delta U$)}
\textbf{Problem:} What is the change in internal energy of a system (in J) if it releases 44.0 J of heat and does 10.0 J of expansion work?
\textbf{Source:} lecture\_slides.pdf, slide 8, Problem b.
\textbf{Analysis:} Another direct calculation of $\Delta U$.
\textbf{Step-by-Step Solution:}
\begin{enumerate}
    \item "releases 44.0 J of heat" $\implies q = -44.0$ J.
    \item "does 10.0 J of expansion work" $\implies w = -10.0$ J.
    \item Use the equation $\Delta U = q + w$.
    \item Substitute values: $\Delta U = (-44.0 \text{ J}) + (-10.0 \text{ J}) = -54.0$ J.
\end{enumerate}
\textbf{Answer:} \textbf{-54.0 J}.

\section{Topic 4: Endothermic vs. Exothermic Processes}
\subsection{Key Principles}
\begin{itemize}
    \item These terms describe the direction of heat flow ($q$) during a chemical or physical process.
    \item \textbf{Exothermic:} The system \textbf{releases} heat into the surroundings.
        \begin{itemize}
            \item $q$ is negative, $\Delta H$ is negative.
            \item The surroundings (e.g., the beaker, the air) get warmer.
            \item Examples: Combustion (burning), condensation (gas to liquid), freezing (liquid to solid).
        \end{itemize}
    \item \textbf{Endothermic:} The system \textbf{absorbs} heat from the surroundings.
        \begin{itemize}
            \item $q$ is positive, $\Delta H$ is positive.
            \item The surroundings get colder.
            \item Examples: Melting (solid to liquid), boiling/evaporation (liquid to gas), photosynthesis.
        \end{itemize}
\end{itemize}

\subsection{Worked Examples}
\subsubsection{Example 1}
\textbf{Problem:} Is warm milk being placed in a refrigerator an endothermic or exothermic process? (Consider the milk as the system).
\textbf{Source:} Internal energy, enthalpy, thermochemical equations.pdf, Problem 6b.
\textbf{Analysis:} The system is the warm milk. The refrigerator is the surroundings. We need to determine the direction of heat flow. Heat naturally flows from a warmer object to a colder object.
\textbf{Step-by-Step Solution:}
\begin{enumerate}
    \item The milk is warm and the refrigerator is cold.
    \item Heat will flow from the milk (system) to the refrigerator (surroundings).
    \item Since the system is releasing heat, the process is exothermic.
\end{enumerate}
\textbf{Answer:} \textbf{Exothermic}.

\subsubsection{Example 2}
\textbf{Problem:} Is an ice cube melting on a countertop an endothermic or exothermic process? (Consider the ice cube as the system).
\textbf{Source:} Internal energy, enthalpy, thermochemical equations.pdf, Problem 6i.
\textbf{Analysis:} The system is the ice cube. For the ice to melt (solid to liquid), it must gain energy to break the bonds holding the water molecules in a fixed lattice. This energy comes from the surroundings (the countertop and the air).
\textbf{Step-by-Step Solution:}
\begin{enumerate}
    \item The process is melting: \ch{H2O(s) -> H2O(l)}.
    \item This process requires an input of energy to break intermolecular forces.
    \item The system (ice) absorbs this heat from the surroundings.
    \item Since the system absorbs heat, the process is endothermic.
\end{enumerate}
\textbf{Answer:} \textbf{Endothermic}.

\subsubsection{Example 3}
\textbf{Problem:} Is the burning of wood an endothermic or exothermic process?
\textbf{Source:} Internal energy, enthalpy, thermochemical equations.pdf, Problem 6g.
\textbf{Analysis:} "Burning" is the common term for combustion. Combustion reactions are a classic example of a process that releases energy.
\textbf{Step-by-Step Solution:}
\begin{enumerate}
    \item The process is combustion.
    \item We observe that burning wood releases large amounts of heat and light to the surroundings.
    \item Since the system (the wood) is releasing heat, the process is exothermic.
\end{enumerate}
\textbf{Answer:} \textbf{Exothermic}.

\subsubsection{Example 4}
\textbf{Problem:} Is photosynthesis an endothermic or exothermic process?
\textbf{Source:} Internal energy, enthalpy, thermochemical equations.pdf, Problem 6e.
\textbf{Analysis:} Photosynthesis is the process plants use to convert light energy, water, and carbon dioxide into chemical energy (glucose). The problem notes "plants absorbing energy".
\textbf{Step-by-Step Solution:}
\begin{enumerate}
    \item The description explicitly states that plants (the system) are absorbing energy from the sun (surroundings).
    \item Since the system is absorbing energy (in the form of light, which is converted to heat), the process is endothermic.
\end{enumerate}
\textbf{Answer:} \textbf{Endothermic}.

\section{Topic 5: Thermochemical Stoichiometry}
\subsection{Key Principles}
\begin{itemize}
    \item A thermochemical equation provides the molar enthalpy change ($\Delta H_{rxn}$) for the specific stoichiometric coefficients in the balanced reaction.
    \item The $\Delta H_{rxn}$ value is an extensive property, meaning it is proportional to the amount of substance reacting.
    \item The relationship between moles and enthalpy from the balanced equation can be used as a conversion factor.
    \item The general calculation pathway is:
    \[ \text{Grams of A} \xrightarrow{\text{Molar Mass}} \text{Moles of A} \xrightarrow{\text{Mole Ratio}} \text{Moles of B} \]
    Or, using enthalpy:
    \[ \text{Grams of A} \xrightarrow{\text{Molar Mass}} \text{Moles of A} \xrightarrow{\frac{\Delta H_{rxn}}{\text{moles of A}}} \text{kJ of heat} \]
    And the reverse:
    \[ \text{kJ of heat} \xrightarrow{\frac{\text{moles of A}}{\Delta H_{rxn}}} \text{Moles of A} \xrightarrow{\text{Molar Mass}} \text{Grams of A} \]
\end{itemize}

\subsection{Worked Examples}
\subsubsection{Example 1}
\textbf{Problem:} For the reaction \ch{CH4(g) + 2O2(g) -> CO2(g) + 2H2O(l)}, $\Delta H_{rxn} = -890$ kJ. How much heat is produced if 15.0 g of \ch{O2} reacts with excess \ch{CH4}?
\textbf{Source:} Internal energy, enthalpy, thermochemical equations.pdf, Problem 7a.
\textbf{Analysis:} We will convert the mass of \ch{O2} to moles, then use the thermochemical equation to convert moles of \ch{O2} to kJ of heat.
\textbf{Step-by-Step Solution:}
\begin{enumerate}
    \item Find the molar mass of \ch{O2}: $2 \times 16.00 = 32.00$ g/mol.
    \item Set up the dimensional analysis. The thermochemical equation tells us that -890 kJ of heat are released for every 2 moles of \ch{O2} reacted.
    \[ 15.0 \text{ g } \ch{O2} \times \frac{1 \text{ mol } \ch{O2}}{32.00 \text{ g } \ch{O2}} \times \frac{-890 \text{ kJ}}{2 \text{ mol } \ch{O2}} \]
    \item Calculate the result:
    \[ = 0.46875 \text{ mol } \ch{O2} \times \frac{-890 \text{ kJ}}{2 \text{ mol } \ch{O2}} = -208.6 \text{ kJ} \]
\end{enumerate}
\textbf{Answer:} \textbf{-209 kJ}.

\subsubsection{Example 2}
\textbf{Problem:} How many grams of iron (III) oxide will be produced if 4500 kJ of heat energy is released in the reaction \ch{4Fe(s) + 3O2(g) -> 2Fe2O3(s)}, $\Delta H_{rxn} = -1652$ kJ?
\textbf{Source:} Internal energy, enthalpy, thermochemical equations.pdf, Problem 8a.
\textbf{Analysis:} We will convert from kJ of heat to moles of product (\ch{Fe2O3}) using the thermochemical equation, and then convert moles to grams using molar mass.
\textbf{Step-by-Step Solution:}
\begin{enumerate}
    \item Find the molar mass of \ch{Fe2O3}: $2 \times 55.85 + 3 \times 16.00 = 159.70$ g/mol.
    \item Set up the dimensional analysis. "Energy is released" implies a negative sign for the heat. The equation shows that 2 moles of \ch{Fe2O3} are produced for every -1652 kJ of heat.
    \[ -4500 \text{ kJ} \times \frac{2 \text{ mol } \ch{Fe2O3}}{-1652 \text{ kJ}} \times \frac{159.70 \text{ g } \ch{Fe2O3}}{1 \text{ mol } \ch{Fe2O3}} \]
    \item Calculate the result:
    \[ = 5.448 \text{ mol } \ch{Fe2O3} \times 159.70 \frac{\text{g}}{\text{mol}} = 869.9 \text{ g} \]
\end{enumerate}
\textbf{Answer:} \textbf{870 g \ch{Fe2O3}}.

\subsubsection{Example 3}
\textbf{Problem:} You burn 15.0 g sulfur (\ch{S8}) in air. How much heat evolves from this amount of sulfur? The molar mass of \ch{S8} is 256.52 g/mole and the balanced equation is: \ch{S8(s) + 8O2(g) -> 8SO2(g)}, $\Delta H_{rxn} = -2.39 \times 10^3$ kJ.
\textbf{Source:} lecture\_slides.pdf, slide 18.
\textbf{Analysis:} Convert grams of reactant (\ch{S8}) to moles, then use the molar enthalpy from the thermochemical equation to find the heat evolved.
\textbf{Step-by-Step Solution:}
\begin{enumerate}
    \item The thermochemical equation tells us that $-2.39 \times 10^3$ kJ of heat are released for every 1 mole of \ch{S8}.
    \item Set up the dimensional analysis:
    \[ 15.0 \text{ g } \ch{S8} \times \frac{1 \text{ mol } \ch{S8}}{256.52 \text{ g } \ch{S8}} \times \frac{-2.39 \times 10^3 \text{ kJ}}{1 \text{ mol } \ch{S8}} \]
    \item Calculate the result:
    \[ = 0.05847 \text{ mol } \ch{S8} \times (-2390 \frac{\text{kJ}}{\text{mol}}) = -139.7 \text{ kJ} \]
\end{enumerate}
\textbf{Answer:} \textbf{-1.40 x 10$^2$ kJ}.

\subsubsection{Example 4}
\textbf{Problem:} Determine the amount of heat (in kJ) given off when 1.26 x 10$^4$ g of \ch{NO2} are produced according to the equation: \ch{2NO(g) + O2(g) -> 2NO2(g)}, $\Delta H = -114.6$ kJ.
\textbf{Source:} Questions and Problems (textbook).pdf, Chapter 10, Problem 10.24.
\textbf{Analysis:} Convert mass of product (\ch{NO2}) to moles, then use the thermochemical ratio to find the heat.
\textbf{Step-by-Step Solution:}
\begin{enumerate}
    \item The molar mass of \ch{NO2} is $14.01 + 2 \times 16.00 = 46.01$ g/mol.
    \item The equation shows -114.6 kJ of heat are released for every 2 moles of \ch{NO2} formed.
    \item Set up the dimensional analysis:
    \[ 1.26 \times 10^4 \text{ g } \ch{NO2} \times \frac{1 \text{ mol } \ch{NO2}}{46.01 \text{ g } \ch{NO2}} \times \frac{-114.6 \text{ kJ}}{2 \text{ mol } \ch{NO2}} \]
    \item Calculate the result:
    \[ = 273.85 \text{ mol } \ch{NO2} \times \frac{-114.6 \text{ kJ}}{2 \text{ mol } \ch{NO2}} = -15690 \text{ kJ} \]
\end{enumerate}
\textbf{Answer:} \textbf{-1.57 x 10$^4$ kJ}.

\section{Topic 6: Heat Calculations (q = mCsΔT and q = CΔT)}
\subsection{Key Principles}
\begin{itemize}
    \item These equations relate the amount of heat transferred ($q$) to the temperature change ($\Delta T$) of a substance or object.
    \item \textbf{$q = mC_s\Delta T$}: Used when you know the mass ($m$) and the \textbf{specific heat capacity ($C_s$)} of the substance. $C_s$ is an intensive property (e.g., J/g$\cdot^\circ$C).
    \item \textbf{$q = C\Delta T$}: Used when you know the \textbf{heat capacity ($C$)} of the entire object (like a calorimeter). $C$ is an extensive property (e.g., J/$^\circ$C).
    \item Remember that $\Delta T = T_{final} - T_{initial}$. The sign of $\Delta T$ determines the sign of $q$. If temperature increases, $\Delta T$ is positive and $q$ is positive (heat is absorbed). If temperature decreases, $\Delta T$ is negative and $q$ is negative (heat is released).
\end{itemize}

\subsection{Worked Examples}
\subsubsection{Example 1 (Solving for q)}
\textbf{Problem:} If 200.0 mL of water is heated from 24.0°C to 100.0°C to make a cup of tea, how much heat must be added? The specific heat capacity of water is 4.18 J/g·°C.
\textbf{Source:} Specific heat capacity and calorimetry.pdf, Problem 2.
\textbf{Analysis:} We will use $q = mC_s\Delta T$. We are given $C_s$ and the temperatures. We can find the mass from the volume, assuming the density of water is 1.00 g/mL.
\textbf{Step-by-Step Solution:}
\begin{enumerate}
    \item Calculate the mass of water: $m = 200.0 \text{ mL} \times \frac{1.00 \text{ g}}{1 \text{ mL}} = 200.0$ g.
    \item Calculate the temperature change: $\Delta T = 100.0^\circ\text{C} - 24.0^\circ\text{C} = 76.0^\circ\text{C}$.
    \item Use the heat equation: $q = mC_s\Delta T$.
    \item Substitute values: $q = (200.0 \text{ g}) \times (4.18 \frac{\text{J}}{\text{g}\cdot^\circ\text{C}}) \times (76.0^\circ\text{C}) = 63536$ J.
\end{enumerate}
\textbf{Answer:} \textbf{6.35 x 10$^4$ J} or \textbf{63.5 kJ}.

\subsubsection{Example 2 (Solving for C$_s$)}
\textbf{Problem:} What is the specific heat capacity of a substance (in J/g·°C) that absorbs 2.500 kJ of heat when a 100. g sample of the substance increases in temperature from 11°C to 71°C?
\textbf{Source:} Specific heat capacity and calorimetry.pdf, Problem 1.
\textbf{Analysis:} We need to rearrange the equation $q=mC_s\Delta T$ to solve for $C_s$. We must ensure our units are consistent (convert kJ to J).
\textbf{Step-by-Step Solution:}
\begin{enumerate}
    \item Convert heat to Joules: $q = 2.500 \text{ kJ} \times \frac{1000 \text{ J}}{1 \text{ kJ}} = 2500$ J.
    \item Calculate the temperature change: $\Delta T = 71^\circ\text{C} - 11^\circ\text{C} = 60.^\circ\text{C}$.
    \item Rearrange the formula: $C_s = \frac{q}{m\Delta T}$.
    \item Substitute values: $C_s = \frac{2500 \text{ J}}{(100. \text{ g})(60.^\circ\text{C})} = 0.4166... \frac{\text{J}}{\text{g}\cdot^\circ\text{C}}$.
\end{enumerate}
\textbf{Answer:} \textbf{0.417 J/g·°C}.

\subsubsection{Example 3 (Solving for Mass)}
\textbf{Problem:} How many grams of water would require 2200 joules of heat to raise its temperature from 34°C to 100.°C? The specific heat capacity of water is 4.18 J/g·°C.
\textbf{Source:} Specific heat capacity and calorimetry.pdf, Problem 3.
\textbf{Analysis:} Rearrange $q=mC_s\Delta T$ to solve for mass ($m$).
\textbf{Step-by-Step Solution:}
\begin{enumerate}
    \item Calculate the temperature change: $\Delta T = 100.^\circ\text{C} - 34^\circ\text{C} = 66^\circ\text{C}$.
    \item Rearrange the formula: $m = \frac{q}{C_s\Delta T}$.
    \item Substitute values: $m = \frac{2200 \text{ J}}{(4.18 \frac{\text{J}}{\text{g}\cdot^\circ\text{C}})(66^\circ\text{C})} = 7.96$ g.
\end{enumerate}
\textbf{Answer:} \textbf{8.0 g}.

\subsubsection{Example 4 (Calorimetry)}
\textbf{Problem:} A 21.8 g sample of ethanol (\ch{C2H5OH}) is burned in a bomb calorimeter. If the temperature rises from 25.0 to 62.3°C, determine the heat capacity ($C_{cal}$) of the calorimeter. The molar mass of ethanol is 46.07 g/mole, and its molar enthalpy of combustion is $\Delta H_{rxn} = -1235$ kJ/mol.
\textbf{Source:} Specific heat capacity and calorimetry.pdf, Problem 6.
\textbf{Analysis:} First, we calculate the total heat released by the combustion of the 21.8 g sample ($q_{rxn}$). Then, since $q_{cal} = -q_{rxn}$, we can use the equation $q_{cal} = C_{cal}\Delta T$ to solve for $C_{cal}$.
\textbf{Step-by-Step Solution:}
\begin{enumerate}
    \item Calculate the heat released by the reaction ($q_{rxn}$):
    \[ 21.8 \text{ g } \ch{C2H5OH} \times \frac{1 \text{ mol } \ch{C2H5OH}}{46.07 \text{ g } \ch{C2H5OH}} \times \frac{-1235 \text{ kJ}}{1 \text{ mol } \ch{C2H5OH}} = -584.4 \text{ kJ} \]
    So, $q_{rxn} = -584.4$ kJ.
    \item Find the heat absorbed by the calorimeter: $q_{cal} = -q_{rxn} = -(-584.4 \text{ kJ}) = +584.4$ kJ.
    \item Calculate the temperature change: $\Delta T = 62.3^\circ\text{C} - 25.0^\circ\text{C} = 37.3^\circ\text{C}$.
    \item Rearrange the calorimeter equation: $C_{cal} = \frac{q_{cal}}{\Delta T}$.
    \item Substitute values: $C_{cal} = \frac{584.4 \text{ kJ}}{37.3^\circ\text{C}} = 15.66$ kJ/$^\circ$C.
\end{enumerate}
\textbf{Answer:} \textbf{15.7 kJ/$^\circ$C}.

\section{Topic 7: Manipulating Thermochemical Equations}
\subsection{Key Principles}
\begin{itemize}
    \item The enthalpy change ($\Delta H$) of a reaction is directly proportional to the amounts of reactants and products. Manipulating the equation requires a corresponding manipulation of the $\Delta H$ value.
    \item \textbf{Memorization Required: The Two Rules}
        \begin{enumerate}
            \item \textbf{Reversing a reaction:} If you flip the reactants and products, you must change the sign of $\Delta H$.
            \[ A \rightarrow B, \Delta H = X \implies B \rightarrow A, \Delta H = -X \]
            \item \textbf{Multiplying a reaction:} If you multiply all stoichiometric coefficients by a factor $n$, you must multiply the $\Delta H$ by the same factor $n$.
            \[ A \rightarrow B, \Delta H = X \implies nA \rightarrow nB, \Delta H = nX \]
        \end{enumerate}
    \item These two rules can be combined. If a reaction is reversed and doubled, you must change the sign of $\Delta H$ AND multiply it by 2.
\end{itemize}

\subsection{Worked Examples}
\textbf{Source for all examples:} lecture\_slides.pdf, slides 32-33.
Given the base equation: \ch{CH4(g) + 2O2(g) -> CO2(g) + 2H2O(l)}, $\Delta H_{rxn} = -890.4$ kJ/mol.

\subsubsection{Example 1}
\textbf{Problem:} Determine the $\Delta H_{rxn}$ for: \ch{3CH4(g) + 6O2(g) -> 3CO2(g) + 6H2O(l)}.
\textbf{Analysis:} Compare the target equation to the base equation. All coefficients have been multiplied by 3. Therefore, we must multiply the base $\Delta H$ by 3.
\textbf{Step-by-Step Solution:}
\begin{enumerate}
    \item Identify the manipulation: The entire reaction has been multiplied by 3.
    \item Apply the rule: Multiply the $\Delta H$ value by 3.
    \[ \Delta H_{rxn} = 3 \times (-890.4 \text{ kJ/mol}) = -2671.2 \text{ kJ/mol} \]
\end{enumerate}
\textbf{Answer:} \textbf{-2671.2 kJ/mol}.

\subsubsection{Example 2}
\textbf{Problem:} Determine the $\Delta H_{rxn}$ for: \ch{CO2(g) + 2H2O(l) -> CH4(g) + 2O2(g)}.
\textbf{Analysis:} The target equation is the exact reverse of the base equation. The coefficients are the same. Therefore, we only need to change the sign of the base $\Delta H$.
\textbf{Step-by-Step Solution:}
\begin{enumerate}
    \item Identify the manipulation: The reaction has been reversed.
    \item Apply the rule: Change the sign of the $\Delta H$ value.
    \[ \Delta H_{rxn} = -(-890.4 \text{ kJ/mol}) = +890.4 \text{ kJ/mol} \]
\end{enumerate}
\textbf{Answer:} \textbf{+890.4 kJ/mol}.

\subsubsection{Example 3}
\textbf{Problem:} Determine the $\Delta H_{rxn}$ for: \ch{2CO2(g) + 4H2O(l) -> 2CH4(g) + 4O2(g)}.
\textbf{Analysis:} This combines both rules. The target equation is the reverse of the base equation, AND all of its coefficients are multiplied by 2.
\textbf{Step-by-Step Solution:}
\begin{enumerate}
    \item Identify the manipulations: The reaction has been reversed and multiplied by 2.
    \item Apply both rules: Change the sign of the $\Delta H$ value AND multiply it by 2.
    \[ \Delta H_{rxn} = -(-890.4 \text{ kJ/mol}) \times 2 \]
    \[ \Delta H_{rxn} = (+890.4 \text{ kJ/mol}) \times 2 = +1780.8 \text{ kJ/mol} \]
\end{enumerate}
\textbf{Answer:} \textbf{+1780.8 kJ/mol}.

\subsubsection{Example 4}
\textbf{Problem:} Consider the reaction: \ch{2CH3OH(l) + 3O2(g) -> 2CO2(g) + 4H2O(l)}, $\Delta H = -1452.8$ kJ/mol. What is the value of $\Delta H$ if the equation is multiplied throughout by 1/2?
\textbf{Source:} Adapted from Questions and Problems (textbook).pdf, Chapter 10, Problem 10.19.
\textbf{Analysis:} The reaction is multiplied by a factor of 1/2 (or divided by 2). We must do the same to the $\Delta H$ value.
\textbf{Step-by-Step Solution:}
\begin{enumerate}
    \item The new equation is: \ch{CH3OH(l) + 3/2 O2(g) -> CO2(g) + 2H2O(l)}.
    \item Apply the rule: Multiply the $\Delta H$ value by 1/2.
    \[ \Delta H_{rxn} = \frac{1}{2} \times (-1452.8 \text{ kJ/mol}) = -726.4 \text{ kJ/mol} \]
\end{enumerate}
\textbf{Answer:} \textbf{-726.4 kJ/mol}.

\section{Topic 8: Hess's Law}
\subsection{Key Principles}
\begin{itemize}
    \item Hess's Law states that if a reaction can be expressed as the sum of two or more other reactions, the $\Delta H$ for the overall reaction is the sum of the $\Delta H$ values for the individual reactions.
    \item This allows us to calculate the enthalpy change for a reaction that is difficult to measure directly.
    \item \textbf{Strategy:}
        \begin{enumerate}
            \item Identify the target equation you want to find the $\Delta H$ for.
            \item Look at the given "step" equations. Find one reactant or product from your target equation in each step equation.
            \item Manipulate each step equation (reverse it, multiply it, or leave it as is) to get the target reactants/products on the correct side with the correct coefficient.
            \item Apply the corresponding changes to the $\Delta H$ value for each manipulated step.
            \item Add the manipulated step equations and their $\Delta H$ values. Intermediates (species that appear on both the product and reactant side) should cancel out, leaving you with the target equation.
        \end{enumerate}
\end{itemize}

\subsection{Worked Examples}
\subsubsection{Example 1}
\textbf{Problem:} Using the following data, find the $\Delta H_{rxn}$ for the coal gasification process: \ch{2C(s) + 2H2O(g) -> CH4(g) + CO2(g)}.
Given:
1) \ch{C(s) + H2O(g) -> CO(g) + H2(g)}, $\Delta H = +131.3$ kJ
2) \ch{CO(g) + H2O(g) -> CO2(g) + H2(g)}, $\Delta H = -41.2$ kJ
3) \ch{CO(g) + 3H2(g) -> CH4(g) + H2O(g)}, $\Delta H = -206.1$ kJ
\textbf{Source:} Hess's law, bond dissociation, standard enthalpy.pdf, Problem 1.
\textbf{Analysis:} We need to manipulate the three given equations to sum to the target equation.
\textbf{Step-by-Step Solution:}
\begin{enumerate}
    \item \textbf{Target \ch{C(s)}:} The target needs 2 \ch{C(s)} as a reactant. Equation 1 has 1 \ch{C(s)} as a reactant. Multiply Equation 1 by 2.
    \begin{itemize}
        \item \ch{2C(s) + 2H2O(g) -> 2CO(g) + 2H2(g)}, $\Delta H_1' = 2 \times (+131.3) = +262.6$ kJ
    \end{itemize}
    \item \textbf{Target \ch{CO2(g)}:} The target needs 1 \ch{CO2(g)} as a product. Equation 2 has 1 \ch{CO2(g)} as a product. Use Equation 2 as is.
    \begin{itemize}
        \item \ch{CO(g) + H2O(g) -> CO2(g) + H2(g)}, $\Delta H_2' = -41.2$ kJ
    \end{itemize}
    \item \textbf{Target \ch{CH4(g)}:} The target needs 1 \ch{CH4(g)} as a product. Equation 3 has 1 \ch{CH4(g)} as a product. Use Equation 3 as is.
    \begin{itemize}
        \item \ch{CO(g) + 3H2(g) -> CH4(g) + H2O(g)}, $\Delta H_3' = -206.1$ kJ
    \end{itemize}
    \item \textbf{Combine and Sum:} Add the three manipulated equations and their $\Delta H$ values.
    \begin{align*}
        \ch{2C(s) + 2H2O(g)} &\rightarrow \ch{2CO(g) + 2H2(g)} \\
        \ch{CO(g) + H2O(g)} &\rightarrow \ch{CO2(g) + H2(g)} \\
        \ch{CO(g) + 3H2(g)} &\rightarrow \ch{CH4(g) + H2O(g)} \\
        \hline
        % Simplify by combining like terms before cancelling
        \ch{2C(s) + 3H2O(g) + 2CO(g) + 3H2(g)} &\rightarrow \ch{CH4(g) + CO2(g) + 2CO(g) + 3H2(g) + H2O(g)} \\
        % Cancel intermediates: 2CO, 3H2, and 1 H2O cancel from each side.
        \ch{2C(s) + 2H2O(g)} &\rightarrow \ch{CH4(g) + CO2(g)} \\
        \hline
        \Delta H_{rxn} &= (+262.6) + (-41.2) + (-206.1) = +15.3 \text{ kJ}
    \end{align*}
\end{enumerate}
\textbf{Answer:} \textbf{+15.3 kJ}.

\subsubsection{Example 2}
\textbf{Problem:} Find the $\Delta H_{rxn}$ for the formation of nitrogen monoxide gas: \ch{4NH3(g) + 5O2(g) -> 4NO(g) + 6H2O(g)}.
Given:
1) \ch{N2(g) + O2(g) -> 2NO(g)}, $\Delta H = +180.5$ kJ
2) \ch{N2(g) + 3H2(g) -> 2NH3(g)}, $\Delta H = -91.8$ kJ
3) \ch{2H2(g) + O2(g) -> 2H2O(g)}, $\Delta H = -483.6$ kJ
\textbf{Source:} Hess's law, bond dissociation, standard enthalpy.pdf, Problem 3.
\textbf{Analysis:} Manipulate the three given equations to match the target.
\textbf{Step-by-Step Solution:}
\begin{enumerate}
    \item \textbf{Target \ch{NO(g)}:} We need 4 \ch{NO} as a product. Equation 1 has 2 \ch{NO} as a product. Multiply Equation 1 by 2.
    \begin{itemize}
        \item \ch{2N2(g) + 2O2(g) -> 4NO(g)}, $\Delta H_1' = 2 \times (+180.5) = +361.0$ kJ
    \end{itemize}
    \item \textbf{Target \ch{NH3(g)}:} We need 4 \ch{NH3} as a reactant. Equation 2 has 2 \ch{NH3} as a product. Reverse and multiply Equation 2 by 2.
    \begin{itemize}
        \item \ch{4NH3(g) -> 2N2(g) + 6H2(g)}, $\Delta H_2' = -2 \times (-91.8) = +183.6$ kJ
    \end{itemize}
    \item \textbf{Target \ch{H2O(g)}:} We need 6 \ch{H2O} as a product. Equation 3 has 2 \ch{H2O} as a product. Multiply Equation 3 by 3.
    \begin{itemize}
        \item \ch{6H2(g) + 3O2(g) -> 6H2O(g)}, $\Delta H_3' = 3 \times (-483.6) = -1450.8$ kJ
    \end{itemize}
    \item \textbf{Add the manipulated equations:} \ch{2N2} and \ch{6H2} will cancel. The \ch{2O2} and \ch{3O2} on the reactant side combine to \ch{5O2}. The final equation matches the target.
    \item \textbf{Sum the enthalpies:}
    \[ \Delta H_{rxn} = (+361.0) + (+183.6) + (-1450.8) = -906.2 \text{ kJ} \]
\end{enumerate}
\textbf{Answer:} \textbf{-906.2 kJ}. (Solution key has an error in the final addition.)

\subsubsection{Example 3}
\textbf{Problem:} From the data in Problem 10.49, calculate the enthalpy change for the transformation S(rhombic) $\rightarrow$ S(monoclinic).
Given:
1) \ch{S(rhombic) + O2(g) -> SO2(g)}, $\Delta H = -296.06$ kJ
2) \ch{S(monoclinic) + O2(g) -> SO2(g)}, $\Delta H = -296.36$ kJ
\textbf{Source:} Questions and Problems (textbook).pdf, Chapter 10, Problem 10.49.
\textbf{Analysis:} We need S(rhombic) as a reactant and S(monoclinic) as a product. We will use equation 1 as is, and reverse equation 2.
\textbf{Step-by-Step Solution:}
\begin{enumerate}
    \item \textbf{Equation 1:} Keep as is to have S(rhombic) as a reactant.
    \begin{itemize}
        \item \ch{S(rhombic) + O2(g) -> SO2(g)}, $\Delta H_1' = -296.06$ kJ
    \end{itemize}
    \item \textbf{Equation 2:} Reverse to have S(monoclinic) as a product. Change the sign of $\Delta H$.
    \begin{itemize}
        \item \ch{SO2(g) -> S(monoclinic) + O2(g)}, $\Delta H_2' = +296.36$ kJ
    \end{itemize}
    \item \textbf{Add the equations:} \ch{O2(g)} and \ch{SO2(g)} cancel.
    \[ \ch{S(rhombic)} \rightarrow \ch{S(monoclinic)} \]
    \item \textbf{Sum the enthalpies:}
    \[ \Delta H_{rxn} = -296.06 \text{ kJ} + 296.36 \text{ kJ} = +0.30 \text{ kJ} \]
\end{enumerate}
\textbf{Answer:} \textbf{+0.30 kJ}.

\subsubsection{Example 4}
\textbf{Problem:} Calculate the standard enthalpy change for the reaction \ch{2Al(s) + Fe2O3(s) -> 2Fe(s) + Al2O3(s)}.
Given:
1) \ch{2Al(s) + 3/2 O2(g) -> Al2O3(s)}, $\Delta H = -1601$ kJ
2) \ch{2Fe(s) + 3/2 O2(g) -> Fe2O3(s)}, $\Delta H = -821$ kJ
\textbf{Source:} Questions and Problems (textbook).pdf, Chapter 10, Problem 10.48.
\textbf{Analysis:} We need \ch{Al(s)} and \ch{Fe2O3(s)} as reactants. We will keep equation 1 as is, and reverse equation 2.
\textbf{Step-by-Step Solution:}
\begin{enumerate}
    \item \textbf{Equation 1:} Keep as is.
    \begin{itemize}
        \item \ch{2Al(s) + 3/2 O2(g) -> Al2O3(s)}, $\Delta H_1' = -1601$ kJ
    \end{itemize}
    \item \textbf{Equation 2:} Reverse to get \ch{Fe2O3(s)} as a reactant.
    \begin{itemize}
        \item \ch{Fe2O3(s) -> 2Fe(s) + 3/2 O2(g)}, $\Delta H_2' = +821$ kJ
    \end{itemize}
    \item \textbf{Add the equations:} The \ch{3/2 O2(g)} cancels out.
    \[ \ch{2Al(s) + Fe2O3(s) -> Al2O3(s) + 2Fe(s)} \]
    \item \textbf{Sum the enthalpies:}
    \[ \Delta H_{rxn} = -1601 \text{ kJ} + 821 \text{ kJ} = -780 \text{ kJ} \]
\end{enumerate}
\textbf{Answer:} \textbf{-780 kJ}.

\section{Topic 9: Using the Summation Law to Calculate $\Delta H_{rxn}^\circ$}
\subsection{Key Principles}
\begin{itemize}
    \item The summation law is a direct application of Hess's Law using tabulated standard enthalpies of formation ($\Delta H_f^\circ$).
    \item \textbf{Memorization Required: The Formula}
    \[ \Delta H_{rxn}^\circ = \sum n\Delta H_f^\circ(\text{products}) - \sum m\Delta H_f^\circ(\text{reactants}) \]
    where $n$ and $m$ are the stoichiometric coefficients from the balanced equation.
    \item \textbf{Process:}
        \begin{enumerate}
            \item Write the balanced chemical equation.
            \item Look up the $\Delta H_f^\circ$ for each product and reactant in a table (like \textit{lecture\_slides.pdf, slide 40}).
            \item Multiply each $\Delta H_f^\circ$ by its stoichiometric coefficient.
            \item Sum the values for all products.
            \item Sum the values for all reactants.
            \item Subtract the total for the reactants from the total for the products.
        \end{enumerate}
    \item Remember: $\Delta H_f^\circ$ for any element in its standard state (e.g., \ch{O2(g)}, \ch{N2(g)}, \ch{C(graphite)}) is \textbf{0 kJ/mol}.
\end{itemize}

\subsection{Worked Examples}
\subsubsection{Example 1}
\textbf{Problem:} Calculate the standard enthalpy change for the reaction: \ch{4NH3(g) + 5O2(g) -> 4NO(g) + 6H2O(g)}.
Given: $\Delta H_f^\circ$[\ch{NH3(g)}] = -45.9 kJ/mol, $\Delta H_f^\circ$[\ch{NO(g)}] = +90.3 kJ/mol, $\Delta H_f^\circ$[\ch{H2O(g)}] = -241.8 kJ/mol.
\textbf{Source:} lecture\_slides.pdf, slides 42-43.
\textbf{Analysis:} This is a direct application of the summation law formula.
\textbf{Step-by-Step Solution:}
\begin{enumerate}
    \item \textbf{Sum of products:}
    \begin{align*}
    \sum n\Delta H_f^\circ(\text{products}) &= [4 \cdot \Delta H_f^\circ(\ch{NO(g)})] + [6 \cdot \Delta H_f^\circ(\ch{H2O(g)})] \\
    &= [4 \text{ mol} \times (+90.3 \frac{\text{kJ}}{\text{mol}})] + [6 \text{ mol} \times (-241.8 \frac{\text{kJ}}{\text{mol}})] \\
    &= (+361.2 \text{ kJ}) + (-1450.8 \text{ kJ}) = -1089.6 \text{ kJ}
    \end{align*}
    \item \textbf{Sum of reactants:} Remember $\Delta H_f^\circ$[\ch{O2(g)}] = 0.
    \begin{align*}
    \sum m\Delta H_f^\circ(\text{reactants}) &= [4 \cdot \Delta H_f^\circ(\ch{NH3(g)})] + [5 \cdot \Delta H_f^\circ(\ch{O2(g)})] \\
    &= [4 \text{ mol} \times (-45.9 \frac{\text{kJ}}{\text{mol}})] + [5 \text{ mol} \times (0 \frac{\text{kJ}}{\text{mol}})] \\
    &= -183.6 \text{ kJ}
    \end{align*}
    \item \textbf{Calculate $\Delta H_{rxn}^\circ$:}
    \[ \Delta H_{rxn}^\circ = (\text{products}) - (\text{reactants}) = (-1089.6 \text{ kJ}) - (-183.6 \text{ kJ}) = -906.0 \text{ kJ} \]
\end{enumerate}
\textbf{Answer:} \textbf{-906.0 kJ}.

\subsubsection{Example 2}
\textbf{Problem:} Calculate the heat of decomposition for: \ch{CaCO3(s) -> CaO(s) + CO2(g)}.
Given: $\Delta H_f^\circ$[\ch{CaCO3(s)}] = -1206.9 kJ/mol, $\Delta H_f^\circ$[\ch{CaO(s)}] = -635.6 kJ/mol, $\Delta H_f^\circ$[\ch{CO2(g)}] = -393.5 kJ/mol.
\textbf{Source:} Questions and Problems (textbook).pdf, Chapter 10, Problem 10.61.
\textbf{Analysis:} Apply the summation law to the given reaction. All coefficients are 1.
\textbf{Step-by-Step Solution:}
\begin{enumerate}
    \item \textbf{Sum of products:}
    \[ [1 \cdot \Delta H_f^\circ(\ch{CaO(s)})] + [1 \cdot \Delta H_f^\circ(\ch{CO2(g)})] = (-635.6) + (-393.5) = -1029.1 \text{ kJ} \]
    \item \textbf{Sum of reactants:}
    \[ [1 \cdot \Delta H_f^\circ(\ch{CaCO3(s)})] = -1206.9 \text{ kJ} \]
    \item \textbf{Calculate $\Delta H_{rxn}^\circ$:}
    \[ \Delta H_{rxn}^\circ = (-1029.1 \text{ kJ}) - (-1206.9 \text{ kJ}) = +177.8 \text{ kJ} \]
\end{enumerate}
\textbf{Answer:} \textbf{+177.8 kJ}.

\subsubsection{Example 3}
\textbf{Problem:} Calculate the heat of combustion for: \ch{2H2S(g) + 3O2(g) -> 2H2O(l) + 2SO2(g)}.
Given: $\Delta H_f^\circ$[\ch{H2S(g)}] = -20.15 kJ/mol, $\Delta H_f^\circ$[\ch{H2O(l)}] = -285.8 kJ/mol, $\Delta H_f^\circ$[\ch{SO2(g)}] = -296.4 kJ/mol.
\textbf{Source:} Questions and Problems (textbook).pdf, Chapter 10, Problem 10.62b.
\textbf{Analysis:} Apply the summation law, paying close attention to the stoichiometric coefficients.
\textbf{Step-by-Step Solution:}
\begin{enumerate}
    \item \textbf{Sum of products:}
    \[ [2 \cdot (-285.8)] + [2 \cdot (-296.4)] = -571.6 - 592.8 = -1164.4 \text{ kJ} \]
    \item \textbf{Sum of reactants:} ($\Delta H_f^\circ$[\ch{O2(g)}] = 0)
    \[ [2 \cdot (-20.15)] + [3 \cdot (0)] = -40.3 \text{ kJ} \]
    \item \textbf{Calculate $\Delta H_{rxn}^\circ$:}
    \[ \Delta H_{rxn}^\circ = (-1164.4 \text{ kJ}) - (-40.3 \text{ kJ}) = -1124.1 \text{ kJ} \]
\end{enumerate}
\textbf{Answer:} \textbf{-1124.1 kJ}.

\subsubsection{Example 4}
\textbf{Problem:} Calculate the heat of combustion for: \ch{2C2H2(g) + 5O2(g) -> 4CO2(g) + 2H2O(l)}.
Given: $\Delta H_f^\circ$[\ch{C2H2(g)}] = +226.6 kJ/mol, $\Delta H_f^\circ$[\ch{CO2(g)}] = -393.5 kJ/mol, $\Delta H_f^\circ$[\ch{H2O(l)}] = -285.8 kJ/mol.
\textbf{Source:} Questions and Problems (textbook).pdf, Chapter 10, Problem 10.63b.
\textbf{Analysis:} Another direct application of the summation law.
\textbf{Step-by-Step Solution:}
\begin{enumerate}
    \item \textbf{Sum of products:}
    \[ [4 \cdot (-393.5)] + [2 \cdot (-285.8)] = -1574.0 - 571.6 = -2145.6 \text{ kJ} \]
    \item \textbf{Sum of reactants:}
    \[ [2 \cdot (+226.6)] + [5 \cdot (0)] = +453.2 \text{ kJ} \]
    \item \textbf{Calculate $\Delta H_{rxn}^\circ$:}
    \[ \Delta H_{rxn}^\circ = (-2145.6 \text{ kJ}) - (+453.2 \text{ kJ}) = -2598.8 \text{ kJ} \]
\end{enumerate}
\textbf{Answer:} \textbf{-2598.8 kJ}.

\section{Topic 10: Calculating an Unknown $\Delta H_f^\circ$}
\subsection{Key Principles}
\begin{itemize}
    \item This is an algebraic variation of the summation law. In these problems, you are given the overall $\Delta H_{rxn}^\circ$ and all but one of the necessary $\Delta H_f^\circ$ values.
    \item The strategy is to set up the summation law equation as usual, substitute all the known values, and then solve for the single unknown variable (the missing $\Delta H_f^\circ$).
    \item Be very careful with signs and algebra, especially when subtracting negative numbers. It's easy to make a mistake.
\end{itemize}

\subsection{Worked Examples}
\subsubsection{Example 1}
\textbf{Problem:} Calculate the standard enthalpy of formation ($\Delta H_f^\circ$) of acetaldehyde (\ch{CH3CHO}) given the reaction: \ch{CH3CHO(g) + 5/2 O2(g) -> 2H2O(l) + 2CO2(g)}, with $\Delta H_{rxn}^\circ = -1194$ kJ.
Given: $\Delta H_f^\circ[\ch{H2O(l)}] = -285.8$ kJ/mol, $\Delta H_f^\circ[\ch{CO2(g)}] = -393.5$ kJ/mol.
\textbf{Source:} Hess's law, bond dissociation, standard enthalpy.pdf, Problem 5.
\textbf{Analysis:} Set up the summation law equation and solve for the unknown $\Delta H_f^\circ(\ch{CH3CHO})$.
\textbf{Step-by-Step Solution:}
\begin{enumerate}
    \item Write the summation law equation:
    \[ \Delta H_{rxn}^\circ = [2\Delta H_f^\circ(\ch{H2O}) + 2\Delta H_f^\circ(\ch{CO2})] - [\Delta H_f^\circ(\ch{CH3CHO}) + \frac{5}{2}\Delta H_f^\circ(\ch{O2})] \]
    \item Substitute known values. Let $x = \Delta H_f^\circ(\ch{CH3CHO})$.
    \[ -1194 = [2(-285.8) + 2(-393.5)] - [x + \frac{5}{2}(0)] \]
    \item Simplify the equation:
    \[ -1194 = [-571.6 - 787.0] - x \]
    \[ -1194 = -1358.6 - x \]
    \item Solve for $x$:
    \[ x = -1358.6 + 1194 = -164.6 \]
\end{enumerate}
\textbf{Answer:} \textbf{-165 kJ/mol}.

\subsubsection{Example 2}
\textbf{Problem:} The standard enthalpy change for the thermal decomposition \ch{AgNO3(s) -> AgNO2(s) + 1/2 O2(g)} is $+78.67$ kJ. The standard enthalpy of formation of \ch{AgNO3(s)} is $-123.02$ kJ/mol. Calculate the standard enthalpy of formation of \ch{AgNO2(s)}.
\textbf{Source:} Questions and Problems (textbook).pdf, Chapter 10, Problem 10.91.
\textbf{Analysis:} Use the summation law. The overall $\Delta H_{rxn}^\circ$ is given, and we need to find $\Delta H_f^\circ(\ch{AgNO2})$. Remember $\Delta H_f^\circ$ of \ch{O2} is zero.
\textbf{Step-by-Step Solution:}
\begin{enumerate}
    \item Write the summation law equation:
    \[ \Delta H_{rxn}^\circ = [\Delta H_f^\circ(\ch{AgNO2}) + \frac{1}{2}\Delta H_f^\circ(\ch{O2})] - [\Delta H_f^\circ(\ch{AgNO3})] \]
    \item Substitute known values. Let $x = \Delta H_f^\circ(\ch{AgNO2})$.
    \[ +78.67 = [x + \frac{1}{2}(0)] - [-123.02] \]
    \item Simplify and solve for $x$:
    \[ +78.67 = x + 123.02 \]
    \[ x = 78.67 - 123.02 = -44.35 \]
\end{enumerate}
\textbf{Answer:} \textbf{-44.35 kJ/mol}.

\subsubsection{Example 3}
\textbf{Problem:} Given the reaction \ch{NH3(g) + 3F2(g) -> NF3(g) + 3HF(g)} with $\Delta H_{rxn}^\circ = -881.2$ kJ/mol, determine the standard enthalpy of formation of \ch{NF3}.
Given: $\Delta H_f^\circ$[\ch{NH3(g)}] = -46.3 kJ/mol, $\Delta H_f^\circ$[\ch{HF(g)}] = -271.6 kJ/mol.
\textbf{Source:} Questions and Problems (textbook).pdf, Chapter 10, Problem 10.89.
\textbf{Analysis:} Another inverse summation law problem. Remember that \ch{F2(g)} is an element in its standard state.
\textbf{Step-by-Step Solution:}
\begin{enumerate}
    \item Write the summation law equation:
    \[ \Delta H_{rxn}^\circ = [\Delta H_f^\circ(\ch{NF3}) + 3\Delta H_f^\circ(\ch{HF})] - [\Delta H_f^\circ(\ch{NH3}) + 3\Delta H_f^\circ(\ch{F2})] \]
    \item Substitute known values. Let $x = \Delta H_f^\circ(\ch{NF3})$.
    \[ -881.2 = [x + 3(-271.6)] - [-46.3 + 3(0)] \]
    \item Simplify and solve for $x$:
    \[ -881.2 = [x - 814.8] - [-46.3] \]
    \[ -881.2 = x - 814.8 + 46.3 \]
    \[ -881.2 = x - 768.5 \]
    \[ x = -881.2 + 768.5 = -112.7 \]
\end{enumerate}
\textbf{Answer:} \textbf{-112.7 kJ/mol}.

\subsubsection{Example 4}
\textbf{Problem:} Calculate the standard enthalpy of formation of methanol (\ch{CH3OH(l)}) from its heat of combustion, $\Delta H_{comb}^\circ = -726.4$ kJ/mol. The combustion reaction is \ch{CH3OH(l) + 3/2 O2(g) -> CO2(g) + 2H2O(l)}.
Given: $\Delta H_f^\circ$[\ch{CO2(g)}] = -393.5 kJ/mol, $\Delta H_f^\circ$[\ch{H2O(l)}] = -285.8 kJ/mol.
\textbf{Source:} Adapted from Questions and Problems (textbook).pdf, Chapter 10, Problems 10.47 and 10.19.
\textbf{Analysis:} The heat of combustion is the $\Delta H_{rxn}^\circ$ for the combustion reaction. We can use this and the summation law to find the unknown $\Delta H_f^\circ$ for methanol.
\textbf{Step-by-Step Solution:}
\begin{enumerate}
    \item Write the summation law for the combustion reaction:
    \[ \Delta H_{comb}^\circ = [\Delta H_f^\circ(\ch{CO2}) + 2\Delta H_f^\circ(\ch{H2O})] - [\Delta H_f^\circ(\ch{CH3OH}) + \frac{3}{2}\Delta H_f^\circ(\ch{O2})] \]
    \item Substitute known values. Let $x = \Delta H_f^\circ(\ch{CH3OH})$.
    \[ -726.4 = [(-393.5) + 2(-285.8)] - [x + \frac{3}{2}(0)] \]
    \item Simplify the equation:
    \[ -726.4 = [-393.5 - 571.6] - x \]
    \[ -726.4 = -965.1 - x \]
    \item Solve for $x$:
    \[ x = -965.1 + 726.4 = -238.7 \]
\end{enumerate}
\textbf{Answer:} \textbf{-238.7 kJ/mol}.

\end{document}