\documentclass{article}

\usepackage{amsmath}
\usepackage{amssymb}
\usepackage[margin=1in]{geometry}
\usepackage{chemformula}

\title{Homework 10: Energy Changes in Chemical Reactions}
\author{Tashfeen Omran}
\date{October 2025}

\begin{document}

\maketitle

\section{Quiz Details}
\begin{itemize}
    \item \textbf{When:} Monday, November 3, 2025, during class.
    \item \textbf{Format:} Closed notes.
    \item \textbf{Allowed Materials:} A scientific calculator. Phones are not permitted.
\end{itemize}

\section{Key Topics}
This quiz will cover the quantitative aspects of heat transfer in chemical and physical processes. Expect questions on the following major topics:
\begin{enumerate}
    \item \textbf{The First Law of Thermodynamics:} Calculating changes in internal energy ($\Delta U$) using heat ($q$) and work ($w$), and correctly applying sign conventions for each.
    \item \textbf{Enthalpy and Calorimetry:} Using thermochemical equations to perform stoichiometric calculations and using calorimetry data ($q=mC_s\Delta T$) to determine heat transfer and enthalpy changes.
    \item \textbf{Hess's Law and Standard Enthalpies of Formation:} Manipulating given chemical equations to find the enthalpy of a target reaction and using the summation law with standard enthalpy of formation ($\Delta H_f^\circ$) values to find the standard enthalpy of reaction ($\Delta H_{rxn}^\circ$).
\end{enumerate}

\section{Phased Study Plan}

\subsection{Phase 1: Foundation (Friday/Saturday)}
\begin{itemize}
    \item \textbf{Core Concepts Review:} Read through the lecture slides to solidify your understanding of the foundational principles.
    \begin{itemize}
        \item Thermodynamics, Systems, Heat, Work: \textit{lecture\_slides.pdf}, slides 2-8.
        \item Enthalpy, Endothermic/Exothermic, Thermochemical Stoichiometry: \textit{lecture\_slides.pdf}, slides 9-16.
        \item Calorimetry and Specific Heat: \textit{lecture\_slides.pdf}, slides 19-28.
        \item Hess's Law \& Standard Enthalpies of Formation: \textit{lecture\_slides.pdf}, slides 29-43.
    \end{itemize}
    \item \textbf{Memorize Key Formulas \& Definitions:}
    \begin{itemize}
        \item \textbf{Internal Energy:} The total energy of a system. The change is calculated as:
        \[ \Delta U = q + w \]
        \item \textbf{Pressure-Volume Work:} Work done by or on a system due to volume change.
        \[ w = -P\Delta V \]
        \item \textbf{Heat Transfer (Specific Heat):} The most common heat calculation equation.
        \[ q = mC_s\Delta T \]
        Where $q$ is heat, $m$ is mass, $C_s$ is specific heat capacity, and $\Delta T$ is the change in temperature ($T_{final} - T_{initial}$).
        \item \textbf{Heat Transfer (Heat Capacity):} Used for an object as a whole, like a calorimeter.
        \[ q = C\Delta T \]
        \item \textbf{Calorimetry Principle:} Heat lost by the system equals heat gained by the surroundings.
        \[ q_{system} = -q_{surroundings} \]
        \item \textbf{Standard Enthalpy of Reaction (Summation Law):}
        \[ \Delta H_{rxn}^\circ = \sum n\Delta H_f^\circ(\text{products}) - \sum m\Delta H_f^\circ(\text{reactants}) \]
    \end{itemize}
    \item \textbf{Conceptual Understanding \& Memorization:}
    \begin{itemize}
        \item \textbf{Sign Conventions:} ALL signs are from the \textbf{system's perspective}.
            \begin{itemize}
                \item Heat absorbed by system (endothermic) $\implies q$ is \textbf{positive (+)}.
                \item Heat released by system (exothermic) $\implies q$ is \textbf{negative (-)}.
                \item Work done ON the system (compression) $\implies w$ is \textbf{positive (+)}.
                \item Work done BY the system (expansion) $\implies w$ is \textbf{negative (-)}.
            \end{itemize}
        \item \textbf{Standard State:} The standard state for thermodynamic data is 1 atm pressure, 25 $^\circ$C (298 K), and 1 M concentration for solutions.
        \item \textbf{Enthalpy of Formation ($\Delta H_f^\circ$):} The $\Delta H_f^\circ$ for any element in its most stable form (e.g., \ch{O2(g)}, \ch{C(graphite)}, \ch{Na(s)}) is \textbf{zero}.
    \end{itemize}
\end{itemize}

\subsection{Phase 2: Practice \& Application (Sunday)}
\begin{itemize}
    \item \textbf{Problem Drills:} Redo these problems from scratch and check your work against the solutions.
    \begin{itemize}
        \item Internal Energy: \textit{Internal energy, enthalpy, thermochemical equations.pdf}, Problems 1, 2, 4.
        \item Thermochemical Stoichiometry: \textit{Internal energy, enthalpy, thermochemical equations.pdf}, Problems 7, 8.
        \item Calorimetry: \textit{Specific heat capacity and calorimetry.pdf}, Problems 2, 5, 7.
        \item Hess's Law: \textit{Hess's law, bond dissociation, standard enthalpy.pdf}, Problems 1, 3.
        \item Enthalpies of Formation: \textit{Hess's law, bond dissociation, standard enthalpy.pdf}, Problems 5, 6.
    \end{itemize}
    \item \textbf{Unit Conversion Practice:}
    \begin{itemize}
        \item Energy: Joules (J) $\leftrightarrow$ kilojoules (kJ) where $1000 \text{ J} = 1 \text{ kJ}$.
        \item Stoichiometry: Always convert grams to moles before using mole ratios from thermochemical equations.
    \end{itemize}
\end{itemize}

\subsection{Phase 3: Final Mastery (Monday)}
\begin{itemize}
    \item \textbf{Final Review \& Prep:} Quickly review the sign conventions and the key formulas. Rework one problem from each major topic to ensure you have the process down. Get a good night's sleep.
\end{itemize}

\section{Explanations of Practice Problems}

\subsection{Topic 1: First Law of Thermodynamics and Sign Conventions}
\subsubsection{Key Principles}
\begin{itemize}
    \item The First Law of Thermodynamics is the law of conservation of energy: energy cannot be created or destroyed, only transferred. Mathematically, the change in a system's internal energy ($\Delta U$) is the sum of the heat transferred ($q$) and the work done ($w$).
    \item The most important and error-prone part of these problems is assigning the correct sign to $q$ and $w$ based on the wording. Always think from the point of view of the system.
    \item \textbf{Heat (q):} "Absorbs heat," "energy flows into," "heat is added" $\implies q > 0$ (+). "Releases heat," "evolves heat," "energy flows out of" $\implies q < 0$ (-).
    \item \textbf{Work (w):} "Work done \textbf{on} the system," "gas is compressed" $\implies w > 0$ (+). "Work done \textbf{by} the system," "expansion work," "piston performs work" $\implies w < 0$ (-).
\end{itemize}

\subsubsection{Example 1}
\textbf{Problem:} The internal energy of a system increased by 982 J when it absorbed 492 J of heat. How much work was done? Was work done by or on the system?

\textbf{Source:} Internal energy, enthalpy, thermochemical equations.pdf, Problem 1.

\textbf{Analysis:} This is a direct application of the First Law equation, $\Delta U = q + w$. We need to identify the given values with their correct signs and then solve for the unknown variable, $w$. The sign of the final answer for $w$ will tell us if work was done on or by the system.

\textbf{Step-by-Step Solution:}
\begin{enumerate}
    \item \textbf{Assign values and signs:}
    \begin{itemize}
        \item "internal energy... increased by 982 J" $\implies \Delta U = +982 \text{ J}$.
        \item "absorbed 492 J of heat" $\implies q = +492 \text{ J}$.
    \end{itemize}
    \item \textbf{Rearrange the First Law equation to solve for work ($w$):}
    \[ \Delta U = q + w \implies w = \Delta U - q \]
    \item \textbf{Substitute values and calculate:}
    \[ w = (+982 \text{ J}) - (+492 \text{ J}) = +490 \text{ J} \]
    \item \textbf{Interpret the sign of work:} Since $w$ is positive, work was done \textbf{on} the system.
\end{enumerate}

\textbf{Answer:} \textbf{+490 J; Work was done on the system}.

\subsubsection{Example 2}
\textbf{Problem:} The internal energy for the combustion of methane in a cylinder has decreased from 1892.4 kJ to 1000.0 kJ. If the combustion reaction causes a connected piston to perform 482 kJ of expansion work, how much heat is exchanged? Is the heat gained or lost by the system?

\textbf{Source:} Internal energy, enthalpy, thermochemical equations.pdf, Problem 2.

\textbf{Analysis:} We are given the initial and final internal energy, which allows us to calculate $\Delta U$. We are also told the system performs expansion work, which determines the sign of $w$. We will solve the First Law equation for $q$.

\textbf{Step-by-Step Solution:}
\begin{enumerate}
    \item \textbf{Calculate $\Delta U$:}
    \[ \Delta U = U_{final} - U_{initial} = 1000.0 \text{ kJ} - 1892.4 \text{ kJ} = -892.4 \text{ kJ} \]
    \item \textbf{Assign a sign to work ($w$):}
    \begin{itemize}
        \item "piston to perform 482 kJ of expansion work" means the system does work on the surroundings. $\implies w = -482 \text{ kJ}$.
    \end{itemize}
    \item \textbf{Rearrange the First Law equation to solve for heat ($q$):}
    \[ \Delta U = q + w \implies q = \Delta U - w \]
    \item \textbf{Substitute values and calculate:}
    \[ q = (-892.4 \text{ kJ}) - (-482 \text{ kJ}) = -892.4 \text{ kJ} + 482 \text{ kJ} = -410.4 \text{ kJ} \]
    \item \textbf{Interpret the sign of heat:} Since $q$ is negative, heat was \textbf{lost} by the system.
\end{enumerate}

\textbf{Answer:} \textbf{-410.4 kJ; Heat is lost by the system}. (Note: The solution key has a slight rounding difference).

\subsubsection{Example 3}
\textbf{Problem:} Roughly 900 J of heat are added to a system, and 200 J of work are done to the system. What is the total change in internal energy?

\textbf{Source:} Internal energy, enthalpy, thermochemical equations.pdf, Problem 4.

\textbf{Analysis:} A straightforward calculation of $\Delta U$ after assigning the correct signs to $q$ and $w$.

\textbf{Step-by-Step Solution:}
\begin{enumerate}
    \item \textbf{Assign values and signs:}
    \begin{itemize}
        \item "heat are added to a system" $\implies q = +900 \text{ J}$.
        \item "work are done to the system" (same as on the system) $\implies w = +200 \text{ J}$.
    \end{itemize}
    \item \textbf{Use the First Law equation to solve for $\Delta U$:}
    \[ \Delta U = q + w \]
    \item \textbf{Substitute values and calculate:}
    \[ \Delta U = (+900 \text{ J}) + (+200 \text{ J}) = +1100 \text{ J} \]
\end{enumerate}

\textbf{Answer:} \textbf{+1100 J}.

\subsubsection{Example 4}
\textbf{Problem:} State whether steaming condensing on a cold window is endothermic or exothermic.

\textbf{Source:} Internal energy, enthalpy, thermochemical equations.pdf, Problem 6a.

\textbf{Analysis:} We must determine if heat is absorbed by the system (endothermic) or released by it (exothermic). The "system" is the water vapor (steam). Condensation is the phase change from gas to liquid.

\textbf{Step-by-Step Solution:}
\begin{enumerate}
    \item \textbf{Define the process:} The process is \ch{H2O(g) -> H2O(l)}.
    \item \textbf{Analyze energy change:} For a gas to become a liquid, its molecules must slow down and get closer together. This means the system must lose kinetic energy. That energy is released to the surroundings (the cold window) in the form of heat.
    \item \textbf{Classify the process:} Since heat is released from the system, the process is \textbf{exothermic}.
\end{enumerate}

\textbf{Answer:} \textbf{Exothermic (heat is released)}.

\subsection{Topic 2: Thermochemical Stoichiometry and Calorimetry}
\subsubsection{Key Principles}
\begin{itemize}
    \item A \textbf{thermochemical equation} is a balanced chemical equation that includes the enthalpy change, $\Delta H_{rxn}$. The $\Delta H$ value corresponds to the mole quantities in the balanced equation.
    \item The $\Delta H_{rxn}$ can be used as a stoichiometric conversion factor to convert between moles of a reactant/product and the amount of heat released/absorbed.
    \item \textbf{Calorimetry} is the experimental measurement of heat flow. The main principle is the conservation of energy: any heat released by a reaction or hot object ($q_{system}$) must be absorbed by the surroundings ($q_{surroundings}$), so $q_{system} = -q_{surroundings}$.
    \item In coffee-cup calorimetry, the surroundings are typically the water and the calorimeter itself. The heat absorbed is calculated using $q=mC_s\Delta T$. In bomb calorimetry, the heat is absorbed by the calorimeter as a whole, so $q_{cal} = C_{cal}\Delta T$.
\end{itemize}

\subsubsection{Example 1}
\textbf{Problem:} Consider the reaction: \ch{4Fe(s) + 3O2(g) -> 2Fe2O3(s)}, $\Delta H_{rxn} = -1652$ kJ. How much heat is released when 9.0 moles Fe reacts with excess oxygen?

\textbf{Source:} Internal energy, enthalpy, thermochemical equations.pdf, Problem 8b.

\textbf{Analysis:} This is a thermochemical stoichiometry problem. The given $\Delta H$ corresponds to the reaction of 4 moles of Fe. We can use this as a conversion factor to find the heat for 9.0 moles of Fe.

\textbf{Step-by-Step Solution:}
\begin{enumerate}
    \item \textbf{Identify the conversion factor from the thermochemical equation:}
    \[ \frac{-1652 \text{ kJ}}{4 \text{ mol } \ch{Fe}} \]
    \item \textbf{Use dimensional analysis to calculate the heat for 9.0 moles:}
    \[ 9.0 \text{ mol } \ch{Fe} \times \frac{-1652 \text{ kJ}}{4 \text{ mol } \ch{Fe}} = -3717 \text{ kJ} \]
\end{enumerate}

\textbf{Answer:} \textbf{-3700 kJ} (to 2 significant figures).

\subsubsection{Example 2}
\textbf{Problem:} If 1756.8 kJ of energy is released in the reaction \ch{CH4(g) + 2O2(g) -> CO2(g) + 2H2O(l)} ($\Delta H_{rxn} = -890$ kJ/mol), how many grams of \ch{H2O} will form?

\textbf{Source:} Internal energy, enthalpy, thermochemical equations.pdf, Problem 7b.

\textbf{Analysis:} This is the reverse of the previous example. We will use the thermochemical equation to convert from kJ to moles of product (\ch{H2O}), and then use the molar mass of water to find the grams.

\textbf{Step-by-Step Solution:}
\begin{enumerate}
    \item \textbf{Identify the conversion factor:} The equation shows that for every -890 kJ of heat released, 2 moles of \ch{H2O} are formed. "Energy is released" means the heat term is negative.
    \[ \frac{2 \text{ mol } \ch{H2O}}{-890 \text{ kJ}} \]
    \item \textbf{Use dimensional analysis to convert kJ to moles of \ch{H2O}:}
    \[ -1756.8 \text{ kJ} \times \frac{2 \text{ mol } \ch{H2O}}{-890 \text{ kJ}} = 3.948 \text{ mol } \ch{H2O} \]
    \item \textbf{Convert moles of \ch{H2O} to grams:} The molar mass of \ch{H2O} is approximately 18.02 g/mol.
    \[ 3.948 \text{ mol } \ch{H2O} \times \frac{18.02 \text{ g } \ch{H2O}}{1 \text{ mol } \ch{H2O}} = 71.14 \text{ g } \ch{H2O} \]
\end{enumerate}

\textbf{Answer:} \textbf{71.1 g \ch{H2O}}.

\subsubsection{Example 3}
\textbf{Problem:} A 12.9-gram sample of an unknown metal at 26.5°C is placed in a Styrofoam cup containing 50.0 grams of water at 88.6°C. The final equilibrium temperature is 87.1°C. Calculate the specific heat capacity of the unknown metal. The specific heat of water is 4.18 J/g·°C.

\textbf{Source:} Specific heat capacity and calorimetry.pdf, Problem 5.

\textbf{Analysis:} This is a heat transfer problem. The heat lost by the hot water ($q_{water}$) is gained by the colder metal ($q_{metal}$). Therefore, $q_{metal} = -q_{water}$. We can calculate $q_{water}$ since we have all the variables, and then use that value to solve for the specific heat of the metal ($C_{s, metal}$).

\textbf{Step-by-Step Solution:}
\begin{enumerate}
    \item \textbf{Set up the core equation:}
    \[ q_{metal} = -q_{water} \]
    \[ (m \cdot C_s \cdot \Delta T)_{metal} = -(m \cdot C_s \cdot \Delta T)_{water} \]
    \item \textbf{Calculate the temperature changes ($\Delta T = T_{final} - T_{initial}$):}
    \begin{itemize}
        \item $\Delta T_{metal} = 87.1^\circ\text{C} - 26.5^\circ\text{C} = 60.6^\circ\text{C}$
        \item $\Delta T_{water} = 87.1^\circ\text{C} - 88.6^\circ\text{C} = -1.5^\circ\text{C}$
    \end{itemize}
    \item \textbf{Plug in the known values:}
    \[ (12.9 \text{ g}) \cdot C_{s, metal} \cdot (60.6^\circ\text{C}) = -[(50.0 \text{ g}) \cdot (4.18 \frac{\text{J}}{\text{g}\cdot^\circ\text{C}}) \cdot (-1.5^\circ\text{C})] \]
    \item \textbf{Simplify and solve for $C_{s, metal}$:}
    \[ (781.74 \text{ g}\cdot^\circ\text{C}) \cdot C_{s, metal} = -(-313.5 \text{ J}) \]
    \[ (781.74 \text{ g}\cdot^\circ\text{C}) \cdot C_{s, metal} = 313.5 \text{ J} \]
    \[ C_{s, metal} = \frac{313.5 \text{ J}}{781.74 \text{ g}\cdot^\circ\text{C}} = 0.401 \frac{\text{J}}{\text{g}\cdot^\circ\text{C}} \]
\end{enumerate}

\textbf{Answer:} \textbf{0.40 J/g·°C} (to 2 significant figures).

\subsubsection{Example 4}
\textbf{Problem:} The complete combustion of cinnamaldehyde (\ch{C9H8O}) in a bomb calorimeter ($C_{cal} = 3.640$ kJ/°C) caused a temperature change from 23.75°C to 41.26°C. Calculate the mass (in grams) of \ch{C9H8O} required for this combustion if the $\Delta H_{rxn} = -5107$ kJ/mole.

\textbf{Source:} Specific heat capacity and calorimetry.pdf, Problem 8.

\textbf{Analysis:} This is a two-part problem. First, use the calorimeter data to find the total heat released by the reaction ($q_{rxn}$). Second, use that heat value and the given molar enthalpy of reaction ($\Delta H_{rxn}$) to stoichiometrically calculate the mass of the reactant.

\textbf{Step-by-Step Solution:}
\begin{enumerate}
    \item \textbf{Calculate the heat absorbed by the calorimeter ($q_{cal}$):}
    \[ \Delta T = 41.26^\circ\text{C} - 23.75^\circ\text{C} = 17.51^\circ\text{C} \]
    \[ q_{cal} = C_{cal} \times \Delta T = (3.640 \frac{\text{kJ}}{^\circ\text{C}}) \times (17.51^\circ\text{C}) = 63.7364 \text{ kJ} \]
    \item \textbf{Determine the heat released by the reaction ($q_{rxn}$):} The reaction is the system, and the calorimeter is the surroundings.
    \[ q_{rxn} = -q_{cal} = -63.7364 \text{ kJ} \]
    \item \textbf{Use $q_{rxn}$ and $\Delta H_{rxn}$ to find the mass of \ch{C9H8O}:} This is a stoichiometry problem. The molar mass of \ch{C9H8O} is 132.2 g/mol.
    \[ -63.7364 \text{ kJ} \times \frac{1 \text{ mol } \ch{C9H8O}}{-5107 \text{ kJ}} \times \frac{132.2 \text{ g } \ch{C9H8O}}{1 \text{ mol } \ch{C9H8O}} = 1.650 \text{ g } \ch{C9H8O} \]
\end{enumerate}

\textbf{Answer:} \textbf{1.65 g \ch{C9H8O}}.

\subsection{Topic 3: Hess's Law and Standard Enthalpies of Formation}
\subsubsection{Key Principles}
\begin{itemize}
    \item \textbf{Hess's Law:} Since enthalpy is a state function, the enthalpy change for an overall reaction is the sum of the enthalpy changes for the individual steps that make up the overall reaction.
    \item \textbf{Rules for Manipulating Equations:}
        \begin{enumerate}
            \item If you reverse a reaction, you must change the sign of its $\Delta H$.
            \item If you multiply a reaction's coefficients by a factor, you must multiply its $\Delta H$ by the same factor.
        \end{enumerate}
    \item \textbf{Strategy for Hess's Law:} Manipulate the given "step" equations (by reversing or multiplying) so that they add up to the target equation. Then, add the manipulated $\Delta H$ values to get the final $\Delta H$.
    \item \textbf{Standard Enthalpy of Formation ($\Delta H_f^\circ$):} A specific type of enthalpy change for forming 1 mole of a compound from its elements in their standard states.
    \item \textbf{Summation Law:} A direct method to calculate $\Delta H_{rxn}^\circ$ using tabulated $\Delta H_f^\circ$ values: $\Delta H_{rxn}^\circ = \sum (\text{moles} \times \Delta H_f^\circ)_{\text{products}} - \sum (\text{moles} \times \Delta H_f^\circ)_{\text{reactants}}$.
\end{itemize}

\subsubsection{Example 1}
\textbf{Problem:} Given the equation for the combustion of methane: \ch{CH4(g) + 2O2(g) -> CO2(g) + 2H2O(l)}, $\Delta H_{rxn} = -890.4$ kJ/mol. Determine the $\Delta H_{rxn}$ for: \ch{2CO2(g) + 4H2O(l) -> 2CH4(g) + 4O2(g)}.

\textbf{Source:} lecture\_slides.pdf, slides 32-33, Problem 3.

\textbf{Analysis:} We need to manipulate the given source equation to match the target equation. The target equation is the reverse of the source equation, and its coefficients are multiplied by 2. We must apply both corresponding changes to the $\Delta H$ value.

\textbf{Step-by-Step Solution:}
\begin{enumerate}
    \item \textbf{Reverse the source equation:} This switches reactants and products and changes the sign of $\Delta H$.
    \[ \ch{CO2(g) + 2H2O(l) -> CH4(g) + 2O2(g)} \quad \Delta H_{rxn} = +890.4 \text{ kJ/mol} \]
    \item \textbf{Multiply the reversed equation by 2:} This doubles all coefficients and doubles the $\Delta H$ value.
    \[ \ch{2CO2(g) + 4H2O(l) -> 2CH4(g) + 4O2(g)} \quad \Delta H_{rxn} = 2 \times (+890.4 \text{ kJ/mol}) \]
    \item \textbf{Calculate the final $\Delta H$:}
    \[ \Delta H_{rxn} = +1780.8 \text{ kJ/mol} \]
\end{enumerate}

\textbf{Answer:} \textbf{+1780.8 kJ/mol}.

\subsubsection{Example 2}
\textbf{Problem:} Using the following data, find the enthalpy of reaction ($\Delta H_{rxn}$) for the hydrogenation of ethylene: \ch{C2H4(g) + H2(g) -> C2H6(g)}.
Given:
1) \ch{C2H4(g) + 3O2(g) -> 2CO2(g) + 2H2O(l)} $\quad \Delta H = -1411$ kJ
2) \ch{C2H6(g) + 7/2 O2(g) -> 2CO2(g) + 3H2O(l)} $\quad \Delta H = -1560$ kJ
3) \ch{H2(g) + 1/2 O2(g) -> H2O(l)} $\quad \Delta H = -285.8$ kJ

\textbf{Source:} Hess's law, bond dissociation, standard enthalpy.pdf, Problem 2.

\textbf{Analysis:} This is a classic Hess's Law problem. We need to arrange the three given equations so they sum to the target equation.

\textbf{Step-by-Step Solution:}
\begin{enumerate}
    \item \textbf{Equation 1:} The target has 1 \ch{C2H4} as a reactant. Equation 1 also has 1 \ch{C2H4} as a reactant. So, we use Equation 1 as is.
    \begin{itemize}
        \item \ch{C2H4(g) + 3O2(g) -> 2CO2(g) + 2H2O(l)} $\quad \Delta H_1 = -1411$ kJ
    \end{itemize}
    \item \textbf{Equation 2:} The target has 1 \ch{C2H6} as a product. Equation 2 has 1 \ch{C2H6} as a reactant. So, we must reverse Equation 2 and change the sign of its $\Delta H$.
    \begin{itemize}
        \item \ch{2CO2(g) + 3H2O(l) -> C2H6(g) + 7/2 O2(g)} $\quad \Delta H_2 = +1560$ kJ
    \end{itemize}
    \item \textbf{Equation 3:} The target has 1 \ch{H2} as a reactant. Equation 3 also has 1 \ch{H2} as a reactant. So, we use Equation 3 as is.
    \begin{itemize}
        \item \ch{H2(g) + 1/2 O2(g) -> H2O(l)} $\quad \Delta H_3 = -285.8$ kJ
    \end{itemize}
    \item \textbf{Add the manipulated equations and cancel intermediates:}
    \begin{align*}
        \ch{C2H4(g) + 3O2(g)} &\rightarrow \ch{2CO2(g) + 2H2O(l)} \\
        \ch{2CO2(g) + 3H2O(l)} &\rightarrow \ch{C2H6(g) + 7/2 O2(g)} \\
        \ch{H2(g) + 1/2 O2(g)} &\rightarrow \ch{H2O(l)} \\
        \hline
        \ch{C2H4(g) + H2(g) + (3 + 1/2)O2(g) + 2CO2(g) + 3H2O(l)} &\rightarrow \ch{C2H6(g) + 2CO2(g) + (2+1)H2O(l) + 7/2 O2(g)} \\
        \text{Cancel intermediates:} \ch{2CO2}, \ch{3H2O}, \text{ and } \ch{7/2 O2} \text{ cancel out.} \\
        \hline
        \ch{C2H4(g) + H2(g)} &\rightarrow \ch{C2H6(g)}
    \end{align*}
    \item \textbf{Add the corresponding enthalpy changes:}
    \[ \Delta H_{rxn} = \Delta H_1 + \Delta H_2 + \Delta H_3 = (-1411 \text{ kJ}) + (+1560 \text{ kJ}) + (-285.8 \text{ kJ}) = -136.8 \text{ kJ} \]
\end{enumerate}

\textbf{Answer:} \textbf{-137 kJ} (to the nearest integer as in the key).

\subsubsection{Example 3}
\textbf{Problem:} Calculate the standard enthalpy ($\Delta H^\circ$) for the oxidation of benzene: \ch{C6H6(l) + 3/2 O2(g) -> 6C(s) + 3H2O(l)}.
Given: $\Delta H_f^\circ[\ch{H2O(l)}] = -285.8$ kJ/mol, $\Delta H_f^\circ[\ch{C6H6(l)}] = +49.0$ kJ/mol.

\textbf{Source:} Hess's law, bond dissociation, standard enthalpy.pdf, Problem 6.

\textbf{Analysis:} We are given standard enthalpies of formation and asked to find the standard enthalpy of reaction. This is a direct application of the summation law. Remember that the $\Delta H_f^\circ$ for elements in their standard state (\ch{O2(g)} and \ch{C(s, graphite)}) is 0 kJ/mol.

\textbf{Step-by-Step Solution:}
\begin{enumerate}
    \item \textbf{Write the summation law equation:}
    \[ \Delta H_{rxn}^\circ = \sum n\Delta H_f^\circ(\text{products}) - \sum m\Delta H_f^\circ(\text{reactants}) \]
    \item \textbf{Expand for the specific reaction:}
    \[ \Delta H_{rxn}^\circ = [6 \cdot \Delta H_f^\circ(\ch{C(s)}) + 3 \cdot \Delta H_f^\circ(\ch{H2O(l)})] - [1 \cdot \Delta H_f^\circ(\ch{C6H6(l)}) + \frac{3}{2} \cdot \Delta H_f^\circ(\ch{O2(g)})] \]
    \item \textbf{Substitute the known values:}
    \[ \Delta H_{rxn}^\circ = [6 \cdot (0 \text{ kJ/mol}) + 3 \cdot (-285.8 \text{ kJ/mol})] - [1 \cdot (+49.0 \text{ kJ/mol}) + \frac{3}{2} \cdot (0 \text{ kJ/mol})] \]
    \item \textbf{Calculate the final result:}
    \[ \Delta H_{rxn}^\circ = [0 - 857.4 \text{ kJ}] - [+49.0 + 0 \text{ kJ}] \]
    \[ \Delta H_{rxn}^\circ = -857.4 \text{ kJ} - 49.0 \text{ kJ} = -906.4 \text{ kJ} \]
\end{enumerate}

\textbf{Answer:} \textbf{-906.4 kJ}.

\subsubsection{Example 4}
\textbf{Problem:} Calculate the standard enthalpy of formation ($\Delta H_f^\circ$) of acetaldehyde (\ch{CH3CHO}) given the following information:
Reaction: \ch{CH3CHO(g) + 5/2 O2(g) -> 2H2O(l) + 2CO2(g)}, $\Delta H_{rxn}^\circ = -1194$ kJ
Given: $\Delta H_f^\circ[\ch{H2O(l)}] = -285.8$ kJ/mol, $\Delta H_f^\circ[\ch{CO2(g)}] = -393.5$ kJ/mol.

\textbf{Source:} Hess's law, bond dissociation, standard enthalpy.pdf, Problem 5.

\textbf{Analysis:} This is an inverse application of the summation law. We are given the overall $\Delta H_{rxn}^\circ$ and the $\Delta H_f^\circ$ for all other species. We can set up the summation law equation and solve for the one unknown, $\Delta H_f^\circ(\ch{CH3CHO})$.

\textbf{Step-by-Step Solution:}
\begin{enumerate}
    \item \textbf{Write the summation law equation:}
    \[ \Delta H_{rxn}^\circ = \sum n\Delta H_f^\circ(\text{products}) - \sum m\Delta H_f^\circ(\text{reactants}) \]
    \item \textbf{Expand for the specific reaction:}
    \[ \Delta H_{rxn}^\circ = [2 \cdot \Delta H_f^\circ(\ch{H2O(l)}) + 2 \cdot \Delta H_f^\circ(\ch{CO2(g)})] - [1 \cdot \Delta H_f^\circ(\ch{CH3CHO(g)}) + \frac{5}{2} \cdot \Delta H_f^\circ(\ch{O2(g)})] \]
    \item \textbf{Substitute the known values, with $x$ as the unknown:}
    \[ -1194 \text{ kJ} = [2 \cdot (-285.8 \text{ kJ}) + 2 \cdot (-393.5 \text{ kJ})] - [x + \frac{5}{2} \cdot (0 \text{ kJ})] \]
    \item \textbf{Simplify the equation:}
    \[ -1194 \text{ kJ} = [-571.6 \text{ kJ} - 787.0 \text{ kJ}] - x \]
    \[ -1194 \text{ kJ} = -1358.6 \text{ kJ} - x \]
    \item \textbf{Solve for $x$:}
    \[ x = -1358.6 \text{ kJ} + 1194 \text{ kJ} \]
    \[ x = -164.6 \text{ kJ} \]
\end{enumerate}

\textbf{Answer:} \textbf{-165 kJ}.

\end{document}