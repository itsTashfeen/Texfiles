\documentclass{article}

\usepackage{amsmath}
\usepackage{amssymb}
\usepackage[margin=1in]{geometry}
\usepackage{chemformula}

\title{Homework 11: Gas Laws and Kinetic Molecular Theory}
\author{Tashfeen Omran}
\date{October 2025}

\begin{document}

\maketitle

\section{Quiz Details}
\begin{itemize}
    \item \textbf{When:} Tuesday, October 28th, 7:00 AM - 7:30 AM.
    \item \textbf{Format:} Closed notes.
    \item \textbf{Allowed Materials:} A scientific calculator. Phones are not permitted.
\end{itemize}

\section{Key Topics}
The quiz will cover all aspects of gas behavior discussed in Chapter 11. You must have a strong command of the following 13 topics:
\begin{enumerate}
    \item Pressure Unit Conversions
    \item The Empirical Gas Laws (Boyle's, Charles's, Gay-Lussac's)
    \item Proportionality of Gas Parameters
    \item The Combined Gas Law
    \item Standard Temperature and Pressure (STP)
    \item The Ideal Gas Law ($PV=nRT$)
    \item Molar Mass and Density Calculations
    \item Gas Stoichiometry
    \item Partial Pressures, Total Pressure, and Mole Fraction
    \item Gas Collection Over Water
    \item Kinetic Molecular Theory: Speed vs. Molar Mass
    \item Diffusion and Effusion Rates
    \item Calculating Ratios of Molecular Speeds (Graham's Law)
\end{enumerate}

\section{Phased Study Plan}
\subsection{Phase 1: Foundation (Friday/Saturday)}
\begin{itemize}
    \item \textbf{Core Concepts Review:} Read through \texttt{lecture\_slides.pdf}, focusing on slides 7-23 (Gas Laws), slides 31-43 (Gas Mixtures and Collection), and slides 45-50 (KMT and Graham's Law).
    \item \textbf{Memorize Key Formulas:}
    \begin{itemize}
        \item \textbf{Boyle's Law:} $P_1V_1 = P_2V_2$ (constant n, T)
        \item \textbf{Charles's Law:} $\frac{V_1}{T_1} = \frac{V_2}{T_2}$ (constant n, P)
        \item \textbf{Gay-Lussac's Law:} $\frac{P_1}{T_1} = \frac{P_2}{T_2}$ (constant n, V)
        \item \textbf{Combined Gas Law:} $\frac{P_1V_1}{T_1} = \frac{P_2V_2}{T_2}$ (constant n)
        \item \textbf{Ideal Gas Law:} $PV = nRT$ (where $R = 0.08206 \frac{\text{L}\cdot\text{atm}}{\text{mol}\cdot\text{K}}$)
        \item \textbf{Molar Mass from Ideal Gas Law:} $M = \frac{dRT}{P}$ where $d$ is density.
        \item \textbf{Dalton's Law:} $P_{total} = P_A + P_B + P_C + \dots$
        \item \textbf{Mole Fraction and Partial Pressure:} $P_A = \chi_A P_{total}$ where $\chi_A = \frac{n_A}{n_{total}}$
        \item \textbf{Graham's Law of Effusion:} $\frac{\text{rate}_A}{\text{rate}_B} = \sqrt{\frac{M_B}{M_A}}$
    \end{itemize}
    \item \textbf{Conceptual Understanding:}
    \begin{itemize}
        \item Temperature (\textbf{T}) must \textbf{ALWAYS} be in Kelvin (K) for all gas law calculations. K = $^\circ$C + 273.15.
        \item Standard Temperature and Pressure (STP) is defined as exactly 273.15 K (0$^\circ$C) and 1 atm. At STP, 1 mole of any ideal gas occupies 22.4 L.
        \item Lighter gases (smaller molar mass) move faster, diffuse faster, and effuse faster than heavier gases at the same temperature.
    \end{itemize}
\end{itemize}

\subsection{Phase 2: Practice \& Application (Sunday)}
\begin{itemize}
    \item \textbf{Problem Drills:} Redo problems from the homework solutions. Do not just read them.
    \begin{itemize}
        \item \texttt{Gas Law Problems.pdf}: \#1-7
        \item \texttt{Molar mass, density, gas stoichiometry, gas mixtures.pdf}: \#1-7
        \item \texttt{Kinetic molecular theory.pdf}: \#1-7
    \end{itemize}
    \item \textbf{Unit Conversion Practice:} Master these conversions.
    \begin{itemize}
        \item Pressure: mmHg to atm ($760 \text{ mmHg} = 1 \text{ atm}$), torr to atm ($760 \text{ torr} = 1 \text{ atm}$).
        \item Temperature: Celsius to Kelvin ($K = ^\circ\text{C} + 273.15$).
    \end{itemize}
\end{itemize}

\subsection{Phase 3: Final Mastery (Monday)}
\begin{itemize}
    \item \textbf{Final Review \& Prep:} Briefly review all formulas and key concepts. Explain the KMT graphs to yourself one last time. Get a good night's sleep.
\end{itemize}

\section{Explanations of Practice Problems}

\section{Topic 1: Pressure Unit Conversions}
\subsection{Key Principles}
\begin{itemize}
    \item The standard unit for pressure in the Ideal Gas Law is the atmosphere (atm). Most problems will require you to convert other units, like mmHg or torr, into atm.
    \item \textbf{Memorization Required:} You must know the key conversion factors.
    \item $1 \text{ atm} = 760 \text{ mmHg}$
    \item $1 \text{ atm} = 760 \text{ torr}$
    \item Therefore, $1 \text{ mmHg} = 1 \text{ torr}$.
\end{itemize}

\subsection{Worked Examples}
\subsubsection{Example 1}
\textbf{Problem:} Convert 684 mmHg to atm.
\textbf{Source:} lecture\_slides.pdf, slide 6, Problem A.
\textbf{Analysis:} This is a direct conversion using the factor $1 \text{ atm} = 760 \text{ mmHg}$. We will set up the dimensional analysis so that mmHg cancels out.
\textbf{Step-by-Step Solution:}
\[ 684 \text{ mmHg} \times \frac{1 \text{ atm}}{760 \text{ mmHg}} = 0.900 \text{ atm} \]
\textbf{Answer:} \textbf{0.900 atm}.

\subsubsection{Example 2}
\textbf{Problem:} Convert 788 torr to atm.
\textbf{Source:} lecture\_slides.pdf, slide 6, Problem C.
\textbf{Analysis:} This is a direct conversion using the factor $1 \text{ atm} = 760 \text{ torr}$.
\textbf{Step-by-Step Solution:}
\[ 788 \text{ torr} \times \frac{1 \text{ atm}}{760 \text{ torr}} = 1.0368 \text{ atm} \]
\textbf{Answer:} \textbf{1.04 atm}.

\subsubsection{Example 3}
\textbf{Problem:} Convert 0.977 atm to torr.
\textbf{Source:} lecture\_slides.pdf, slide 6, Problem B.
\textbf{Analysis:} This is a conversion from atm to torr. We use the same conversion factor, but inverted.
\textbf{Step-by-Step Solution:}
\[ 0.977 \text{ atm} \times \frac{760 \text{ torr}}{1 \text{ atm}} = 742.52 \text{ torr} \]
\textbf{Answer:} \textbf{743 torr}.

\subsubsection{Example 4}
\textbf{Problem:} The pressure on planet Quibblefritz is calculated to be 718 torr. What is this pressure in atm?
\textbf{Source:} Adapted from Gas Law Problems.pdf, Problem 6.
\textbf{Analysis:} A simple conversion from torr to atm.
\textbf{Step-by-Step Solution:}
\[ 718 \text{ torr} \times \frac{1 \text{ atm}}{760 \text{ torr}} = 0.9447 \text{ atm} \]
\textbf{Answer:} \textbf{0.945 atm}.

\section{Topic 2: The Empirical Gas Laws}
\subsection{Key Principles}
\begin{itemize}
    \item These laws relate two gas properties while holding the other two constant. They are used to compare the initial (1) and final (2) states of a gas.
    \item \textbf{Boyle's Law ($P_1V_1 = P_2V_2$):} At constant temperature and moles, pressure and volume are inversely proportional. As pressure increases, volume decreases.
    \item \textbf{Charles's Law ($\frac{V_1}{T_1} = \frac{V_2}{T_2}$):} At constant pressure and moles, volume and absolute temperature are directly proportional. As temperature increases, volume increases.
    \item \textbf{Gay-Lussac's Law ($\frac{P_1}{T_1} = \frac{P_2}{T_2}$):} At constant volume and moles, pressure and absolute temperature are directly proportional. As temperature increases, pressure increases.
\end{itemize}

\subsection{Worked Examples}
\subsubsection{Example 1 (Gay-Lussac's Law)}
\textbf{Problem:} A gas sample in a sealed flask has a pressure of 792.84 torr at a temperature of 21.26$^\circ$C. What is the pressure if the flask is heated to 75.35$^\circ$C?
\textbf{Source:} Gas Law Problems.pdf, Page 1, Problem 1.
\textbf{Analysis:} The flask is "sealed" (constant n) and "rigid" (constant V). Pressure and temperature are changing. This is a Gay-Lussac's Law problem. We must convert temperatures to Kelvin first.
\textbf{Step-by-Step Solution:}
\begin{enumerate}
    \item Identify variables: $P_1 = 792.84$ torr, $T_1 = 21.26^\circ$C, $P_2 = ?$, $T_2 = 75.35^\circ$C.
    \item Convert temperatures to Kelvin:
    \begin{itemize}
        \item $T_1 = 21.26 + 273.15 = 294.41$ K
        \item $T_2 = 75.35 + 273.15 = 348.50$ K
    \end{itemize}
    \item Rearrange Gay-Lussac's Law to solve for $P_2$: $P_2 = \frac{P_1 T_2}{T_1}$.
    \item Substitute values and calculate:
    \[ P_2 = \frac{(792.84 \text{ torr})(348.50 \text{ K})}{294.41 \text{ K}} = 938.50 \text{ torr} \]
\end{enumerate}
\textbf{Answer:} \textbf{938.50 torr}.

\subsubsection{Example 2 (Boyle's Law)}
\textbf{Problem:} A gas sample in a syringe has a volume of 35.26 mL at 0.9854 atm. The plunger is depressed, reducing the volume to 22.33 mL. What is the new pressure, assuming constant temperature?
\textbf{Source:} Gas Law Problems.pdf, Page 1, Problem 2.
\textbf{Analysis:} Temperature and moles are constant, while pressure and volume change. This is a Boyle's Law problem.
\textbf{Step-by-Step Solution:}
\begin{enumerate}
    \item Identify variables: $P_1 = 0.9854$ atm, $V_1 = 35.26$ mL, $P_2 = ?$, $V_2 = 22.33$ mL.
    \item Rearrange Boyle's Law to solve for $P_2$: $P_2 = \frac{P_1 V_1}{V_2}$.
    \item Substitute values and calculate:
    \[ P_2 = \frac{(0.9854 \text{ atm})(35.26 \text{ mL})}{22.33 \text{ mL}} = 1.556 \text{ atm} \]
\end{enumerate}
\textbf{Answer:} \textbf{1.556 atm}.

\subsubsection{Example 3 (Charles's Law)}
\textbf{Problem:} A sample of nitrogen in a flexible container has a volume of 42.748 L at -24.36$^\circ$C. The temperature is increased to 8.97$^\circ$C. Assuming pressure remains constant, what is the new volume?
\textbf{Source:} Gas Law Problems.pdf, Page 1, Problem 3.
\textbf{Analysis:} A "flexible container" at constant pressure implies that volume can change. Moles are also constant. Volume and temperature are changing. This is a Charles's Law problem.
\textbf{Step-by-Step Solution:}
\begin{enumerate}
    \item Identify variables: $V_1 = 42.748$ L, $T_1 = -24.36^\circ$C, $V_2 = ?$, $T_2 = 8.97^\circ$C.
    \item Convert temperatures to Kelvin:
    \begin{itemize}
        \item $T_1 = -24.36 + 273.15 = 248.79$ K
        \item $T_2 = 8.97 + 273.15 = 282.12$ K
    \end{itemize}
    \item Rearrange Charles's Law to solve for $V_2$: $V_2 = \frac{V_1 T_2}{T_1}$.
    \item Substitute values and calculate:
    \[ V_2 = \frac{(42.748 \text{ L})(282.12 \text{ K})}{248.79 \text{ K}} = 48.456 \text{ L} \]
\end{enumerate}
\textbf{Answer:} The provided key shows 14.456 L, which corresponds to an initial volume of 12.748 L, likely a typo in my source file. Based on the problem as written, the answer is \textbf{48.456 L}.

\subsubsection{Example 4 (Boyle's Law)}
\textbf{Problem:} A sample of air occupies 3.8 L when the pressure is 1.2 atm. What volume does it occupy at 6.6 atm, assuming constant temperature?
\textbf{Source:} Questions and Problems (textbook).pdf, Chapter 11, Problem 11.32a.
\textbf{Analysis:} Constant temperature and amount of gas, with changing pressure and volume. This is a Boyle's Law problem.
\textbf{Step-by-Step Solution:}
\begin{enumerate}
    \item Identify variables: $P_1 = 1.2$ atm, $V_1 = 3.8$ L, $P_2 = 6.6$ atm, $V_2 = ?$.
    \item Rearrange Boyle's Law to solve for $V_2$: $V_2 = \frac{P_1 V_1}{P_2}$.
    \item Substitute values and calculate:
    \[ V_2 = \frac{(1.2 \text{ atm})(3.8 \text{ L})}{6.6 \text{ atm}} = 0.69 \text{ L} \]
\end{enumerate}
\textbf{Answer:} \textbf{0.69 L}.

\section{Topic 3: Proportionality of Gas Parameters}
\subsection{Key Principles}
\begin{itemize}
    \item The empirical gas laws are based on proportionalities between variables.
    \item \textbf{Direct Proportionality ($X \propto Y$):} As Y increases, X increases by the same factor (e.g., volume and temperature in Charles's Law). The graph is a straight line through the origin.
    \item \textbf{Inverse Proportionality ($X \propto 1/Y$):} As Y increases, X decreases by the same factor (e.g., pressure and volume in Boyle's Law). The graph is a hyperbola.
    \item These relationships are key to conceptually understanding gas behavior without calculations.
\end{itemize}

\subsection{Worked Examples}
\subsubsection{Example 1}
\textbf{Problem:} Explain why a helium weather balloon expands as it rises in the air, assuming the temperature remains constant.
\textbf{Source:} Questions and Problems (textbook).pdf, Chapter 11, Problem 11.29.
\textbf{Analysis:} This question asks for a conceptual explanation based on gas law proportionalities. As a balloon rises, the external atmospheric pressure decreases. We need to explain how this affects the balloon's volume.
\textbf{Step-by-Step Solution:}
\begin{enumerate}
    \item Identify the relationship: At constant temperature and moles of gas, pressure and volume are related by Boyle's Law.
    \item State the proportionality: Boyle's Law states that pressure and volume are inversely proportional ($V \propto 1/P$).
    \item Apply to the scenario: As the balloon rises, the external pressure decreases. To maintain pressure equilibrium, the internal pressure must also decrease. Because of the inverse relationship, a decrease in pressure causes an increase in volume.
\end{enumerate}
\textbf{Answer:} As the balloon rises, the atmospheric pressure decreases. According to Boyle's Law, pressure and volume are inversely proportional, so the volume of the helium inside the balloon expands to match the lower external pressure.

\subsubsection{Example 2}
\textbf{Problem:} The absolute temperature of a gas in a flexible container is doubled at constant pressure and amount of gas. What happens to the volume?
\textbf{Source:} Adapted from Questions and Problems (textbook).pdf, Chapter 11, Problem 11.40(2).
\textbf{Analysis:} This requires understanding the direct proportionality from Charles's Law.
\textbf{Step-by-Step Solution:}
\begin{enumerate}
    \item Identify the relationship: At constant pressure and moles, volume and temperature are related by Charles's Law.
    \item State the proportionality: Charles's Law states that volume and absolute temperature are directly proportional ($V \propto T$).
    \item Apply to the scenario: If the absolute temperature is doubled, the volume must also double to maintain the direct proportionality.
\end{enumerate}
\textbf{Answer:} The volume will double.

\subsubsection{Example 3}
\textbf{Problem:} The pressure on a gas in a piston is tripled at constant temperature and amount of gas. What happens to the volume?
\textbf{Source:} Adapted from Questions and Problems (textbook).pdf, Chapter 11, Problem 11.40(1).
\textbf{Analysis:} This requires understanding the inverse proportionality from Boyle's Law.
\textbf{Step-by-Step Solution:}
\begin{enumerate}
    \item Identify the relationship: At constant temperature and moles, pressure and volume are related by Boyle's Law.
    \item State the proportionality: Pressure and volume are inversely proportional ($V \propto 1/P$).
    \item Apply to the scenario: If the pressure is tripled, the volume must decrease by a factor of three (be reduced to one-third of its original value) to maintain the inverse relationship.
\end{enumerate}
\textbf{Answer:} The volume is reduced to one-third of its original value.

\subsubsection{Example 4}
\textbf{Problem:} If a sealed, rigid container of gas is heated, what happens to the pressure inside?
\textbf{Source:} Conceptual Question.
\textbf{Analysis:} A "sealed, rigid container" means moles ($n$) and volume ($V$) are constant. We are heating it, so temperature ($T$) increases. We need to describe the effect on pressure ($P$).
\textbf{Step-by-Step Solution:}
\begin{enumerate}
    \item Identify the relationship: At constant volume and moles, pressure and temperature are related by Gay-Lussac's Law.
    \item State the proportionality: Pressure and absolute temperature are directly proportional ($P \propto T$).
    \item Apply to the scenario: As the temperature increases, the pressure must also increase.
\end{enumerate}
\textbf{Answer:} The pressure increases.

\section{Topic 4: The Combined Gas Law}
\subsection{Key Principles}
\begin{itemize}
    \item The Combined Gas Law merges Boyle's, Charles's, and Gay-Lussac's laws into a single, more versatile equation.
    \item It is used when comparing two states (initial and final) of a gas where the number of moles ($n$) is constant, but pressure, volume, and temperature may all change.
    \item The formula is:
    \[ \frac{P_1V_1}{T_1} = \frac{P_2V_2}{T_2} \]
    \item If any variable is held constant (e.g., $T_1 = T_2$), it can be canceled from both sides, simplifying the equation to one of the empirical laws (in this case, Boyle's Law).
\end{itemize}

\subsection{Worked Examples}
\subsubsection{Example 1}
\textbf{Problem:} A 1.500 L balloon is filled with helium at 305.9 K and 745.9 mmHg. The balloon rises to an altitude where the pressure is 154.2 mmHg and the temperature is 250.6 K. What is the volume of the balloon at this altitude?
\textbf{Source:} Gas Law Problems.pdf, Page 2, Problem 5.
\textbf{Analysis:} The amount of gas ($n$) is constant, but P, V, and T all change. This requires the Combined Gas Law. Pressures are both in mmHg, so they will cancel and do not need to be converted to atm.
\textbf{Step-by-Step Solution:}
\begin{enumerate}
    \item Identify variables:
    \begin{itemize}
        \item Initial (1): $P_1 = 745.9$ mmHg, $V_1 = 1.500$ L, $T_1 = 305.9$ K
        \item Final (2): $P_2 = 154.2$ mmHg, $V_2 = ?$, $T_2 = 250.6$ K
    \end{itemize}
    \item Rearrange the Combined Gas Law to solve for $V_2$: $V_2 = \frac{P_1 V_1 T_2}{T_1 P_2}$.
    \item Substitute values and calculate:
    \[ V_2 = \frac{(745.9 \text{ mmHg})(1.500 \text{ L})(250.6 \text{ K})}{(305.9 \text{ K})(154.2 \text{ mmHg})} = 5.944 \text{ L} \]
\end{enumerate}
\textbf{Answer:} \textbf{5.944 L}.

\subsubsection{Example 2}
\textbf{Problem:} A gas-filled balloon having a volume of 2.50 L at 1.2 atm and 20$^\circ$C is allowed to rise to the stratosphere, where the temperature and pressure are -23$^\circ$C and 3.00 x 10$^{-3}$ atm, respectively. Calculate the final volume of the balloon.
\textbf{Source:} Questions and Problems (textbook).pdf, Chapter 11, Problem 11.48.
\textbf{Analysis:} This is another Combined Gas Law problem. We must convert both temperatures to Kelvin before solving for the final volume, $V_2$.
\textbf{Step-by-Step Solution:}
\begin{enumerate}
    \item Identify variables:
    \begin{itemize}
        \item Initial (1): $P_1 = 1.2$ atm, $V_1 = 2.50$ L, $T_1 = 20^\circ$C
        \item Final (2): $P_2 = 3.00 \times 10^{-3}$ atm, $V_2 = ?$, $T_2 = -23^\circ$C
    \end{itemize}
    \item Convert temperatures to Kelvin:
    \begin{itemize}
        \item $T_1 = 20 + 273.15 = 293.15$ K
        \item $T_2 = -23 + 273.15 = 250.15$ K
    \end{itemize}
    \item Rearrange the Combined Gas Law for $V_2$: $V_2 = \frac{P_1 V_1 T_2}{T_1 P_2}$.
    \item Substitute values and calculate:
    \[ V_2 = \frac{(1.2 \text{ atm})(2.50 \text{ L})(250.15 \text{ K})}{(293.15 \text{ K})(3.00 \times 10^{-3} \text{ atm})} = 853.5 \text{ L} \]
\end{enumerate}
\textbf{Answer:} \textbf{8.5 x 10$^2$ L}.

\subsubsection{Example 3}
\textbf{Problem:} A gas at 572 mmHg and 35.0$^\circ$C occupies a volume of 6.15 L. Calculate its volume at STP.
\textbf{Source:} Questions and Problems (textbook).pdf, Chapter 11, Problem 11.52.
\textbf{Analysis:} This problem requires us to know the conditions for STP. We will use the Combined Gas Law, converting all units to be consistent.
\textbf{Step-by-Step Solution:}
\begin{enumerate}
    \item Identify variables:
    \begin{itemize}
        \item Initial (1): $P_1 = 572$ mmHg, $V_1 = 6.15$ L, $T_1 = 35.0^\circ$C
        \item Final (2, STP): $P_2 = 1$ atm, $V_2 = ?$, $T_2 = 0^\circ$C
    \end{itemize}
    \item Convert all units to be consistent. Let's use atm and K.
    \begin{itemize}
        \item $P_1 = 572 \text{ mmHg} \times \frac{1 \text{ atm}}{760 \text{ mmHg}} = 0.7526$ atm
        \item $T_1 = 35.0 + 273.15 = 308.15$ K
        \item $P_2 = 1.00$ atm
        \item $T_2 = 0 + 273.15 = 273.15$ K
    \end{itemize}
    \item Rearrange for $V_2$: $V_2 = \frac{P_1 V_1 T_2}{T_1 P_2}$.
    \item Substitute values:
    \[ V_2 = \frac{(0.7526 \text{ atm})(6.15 \text{ L})(273.15 \text{ K})}{(308.15 \text{ K})(1.00 \text{ atm})} = 4.10 \text{ L} \]
\end{enumerate}
\textbf{Answer:} \textbf{4.10 L}.

\subsubsection{Example 4}
\textbf{Problem:} An ideal gas originally at 0.85 atm and 66$^\circ$C was allowed to expand until its final volume, pressure, and temperature were 94 mL, 0.60 atm, and 45$^\circ$C, respectively. What was its initial volume?
\textbf{Source:} Questions and Problems (textbook).pdf, Chapter 11, Problem 11.50.
\textbf{Analysis:} A Combined Gas Law problem where the unknown is the initial volume, $V_1$.
\textbf{Step-by-Step Solution:}
\begin{enumerate}
    \item Identify variables:
    \begin{itemize}
        \item Initial (1): $P_1 = 0.85$ atm, $V_1 = ?$, $T_1 = 66^\circ$C
        \item Final (2): $P_2 = 0.60$ atm, $V_2 = 94$ mL, $T_2 = 45^\circ$C
    \end{itemize}
    \item Convert temperatures to Kelvin:
    \begin{itemize}
        \item $T_1 = 66 + 273.15 = 339.15$ K
        \item $T_2 = 45 + 273.15 = 318.15$ K
    \end{itemize}
    \item Rearrange the Combined Gas Law for $V_1$: $V_1 = \frac{P_2 V_2 T_1}{T_2 P_1}$.
    \item Substitute values:
    \[ V_1 = \frac{(0.60 \text{ atm})(94 \text{ mL})(339.15 \text{ K})}{(318.15 \text{ K})(0.85 \text{ atm})} = 70.7 \text{ mL} \]
\end{enumerate}
\textbf{Answer:} \textbf{71 mL}.

\section{Topic 5: Standard Temperature and Pressure (STP)}
\subsection{Key Principles}
\begin{itemize}
    \item \textbf{Memorization Required:} STP is a specific set of reference conditions for gases.
        \begin{itemize}
            \item Standard Temperature = 0$^\circ$C or 273.15 K
            \item Standard Pressure = 1 atm (exactly)
        \end{itemize}
    \item At STP, the volume of 1 mole of any ideal gas is 22.4 L. This is called the \textbf{molar volume}.
    \item This molar volume can be used as a direct conversion factor between moles and liters, but \textbf{only at STP}.
    \item For problems not at STP, you must use the Ideal Gas Law ($PV=nRT$).
\end{itemize}

\subsection{Worked Examples}
\subsubsection{Example 1}
\textbf{Problem:} At STP, what is the density of a gas (in g/L) if the molar mass of this gas is 66.1 g/mole?
\textbf{Source:} Molar mass, density, gas stoichiometry, gas mixtures.pdf, Problem 2.
\textbf{Analysis:} We can use the molar mass and molar volume at STP to find the density. Density is mass/volume. For 1 mole of gas at STP, the mass is the molar mass (66.1 g) and the volume is the molar volume (22.4 L).
\textbf{Step-by-Step Solution:}
\begin{enumerate}
    \item Identify the mass of 1 mole of the gas: $m = 66.1$ g.
    \item Identify the volume of 1 mole of the gas at STP: $V = 22.4$ L.
    \item Calculate density:
    \[ d = \frac{\text{mass}}{\text{volume}} = \frac{66.1 \text{ g}}{22.4 \text{ L}} = 2.95 \text{ g/L} \]
\end{enumerate}
\textbf{Answer:} \textbf{2.95 g/L}.

\subsubsection{Example 2}
\textbf{Problem:} Calculate the volume of 110.4 g \ch{CH4} gas (in L) at STP.
\textbf{Source:} Molar mass, density, gas stoichiometry, gas mixtures.pdf, Problem 3.
\textbf{Analysis:} This is a two-step conversion: first convert grams to moles, then use the molar volume at STP to convert moles to liters.
\textbf{Step-by-Step Solution:}
\begin{enumerate}
    \item The molar mass of \ch{CH4} is $12.01 + 4(1.008) = 16.042$ g/mol.
    \item Set up the dimensional analysis:
    \[ 110.4 \text{ g } \ch{CH4} \times \frac{1 \text{ mol } \ch{CH4}}{16.042 \text{ g } \ch{CH4}} \times \frac{22.4 \text{ L}}{1 \text{ mol } \ch{CH4}} \]
    \item Calculate the result:
    \[ = 6.882 \text{ mol } \ch{CH4} \times 22.4 \frac{\text{L}}{\text{mol}} = 154.15 \text{ L} \]
\end{enumerate}
\textbf{Answer:} \textbf{154.3 L \ch{CH4}}.

\subsubsection{Example 3}
\textbf{Problem:} At STP, 0.280 L of a gas weighs 0.400 g. Calculate the molar mass of the gas.
\textbf{Source:} Questions and Problems (textbook).pdf, Chapter 11, Problem 11.54.
\textbf{Analysis:} We can use the molar volume at STP to find out how many moles are in 0.280 L. Then, we can use the definition of molar mass (grams/mole).
\textbf{Step-by-Step Solution:}
\begin{enumerate}
    \item Convert the volume at STP to moles:
    \[ n = 0.280 \text{ L} \times \frac{1 \text{ mol}}{22.4 \text{ L}} = 0.0125 \text{ mol} \]
    \item Calculate the molar mass:
    \[ M = \frac{\text{mass}}{\text{moles}} = \frac{0.400 \text{ g}}{0.0125 \text{ mol}} = 32.0 \text{ g/mol} \]
\end{enumerate}
\textbf{Answer:} \textbf{32.0 g/mol}.

\subsubsection{Example 4}
\textbf{Problem:} Calculate the volume (in liters) of 124.3 g of \ch{CO2} at STP.
\textbf{Source:} Questions and Problems (textbook).pdf, Chapter 11, Problem 11.51.
\textbf{Analysis:} This is the same process as Example 2.
\textbf{Step-by-Step Solution:}
\begin{enumerate}
    \item The molar mass of \ch{CO2} is $12.01 + 2(16.00) = 44.01$ g/mol.
    \item Set up the dimensional analysis:
    \[ 124.3 \text{ g } \ch{CO2} \times \frac{1 \text{ mol } \ch{CO2}}{44.01 \text{ g } \ch{CO2}} \times \frac{22.41 \text{ L}}{1 \text{ mol } \ch{CO2}} \]
    (Using the more precise molar volume of 22.41 L/mol)
    \item Calculate the result:
    \[ = 2.824 \text{ mol } \ch{CO2} \times 22.41 \frac{\text{L}}{\text{mol}} = 63.29 \text{ L} \]
\end{enumerate}
\textbf{Answer:} \textbf{63.29 L}.

\section{Topic 6: The Ideal Gas Law}
\subsection{Key Principles}
\begin{itemize}
    \item The Ideal Gas Law, $PV=nRT$, relates all four gas variables (Pressure, Volume, moles, Temperature) for a gas in a single state. It is the most powerful and versatile of the gas laws.
    \item \textbf{Memorization Required:} You must know the formula and the value and units of the ideal gas constant, $R$.
        \begin{itemize}
            \item $R = 0.08206 \frac{\text{L}\cdot\text{atm}}{\text{mol}\cdot\text{K}}$
        \end{itemize}
    \item For the units to cancel correctly, the variables in the equation must have units that match the units in R:
        \begin{itemize}
            \item $P$ must be in atmospheres (atm).
            \item $V$ must be in Liters (L).
            \item $n$ must be in moles (mol).
            \item $T$ must be in Kelvin (K).
        \end{itemize}
\end{itemize}

\subsection{Worked Examples}
\subsubsection{Example 1}
\textbf{Problem:} A 2.2854 g sample of propane (\ch{C3H8}; 44.0972 g/mol) is placed in a 1.5763 L flask at 375.91 K. What is the pressure of this sample?
\textbf{Source:} Gas Law Problems.pdf, Page 3, Problem 7.
\textbf{Analysis:} This problem describes a single state of a gas, so we use the Ideal Gas Law. We are asked for pressure ($P$). We have V and T. We need to find n by converting the given mass to moles.
\textbf{Step-by-Step Solution:}
\begin{enumerate}
    \item Convert mass to moles ($n$):
    \[ n = 2.2854 \text{ g } \ch{C3H8} \times \frac{1 \text{ mol } \ch{C3H8}}{44.0972 \text{ g } \ch{C3H8}} = 0.051826 \text{ mol} \]
    \item Rearrange the Ideal Gas Law to solve for $P$: $P = \frac{nRT}{V}$.
    \item Substitute all values into the equation:
    \[ P = \frac{(0.051826 \text{ mol})(0.08206 \frac{\text{L}\cdot\text{atm}}{\text{mol}\cdot\text{K}})(375.91 \text{ K})}{1.5763 \text{ L}} = 1.014 \text{ atm} \]
\end{enumerate}
\textbf{Answer:} \textbf{1.014 atm}.

\subsubsection{Example 2}
\textbf{Problem:} A 50.0-L cylinder of nitrogen, \ch{N2}, has a pressure of 17.1 atm at 23$^\circ$C. What is the mass of nitrogen in the cylinder?
\textbf{Source:} lecture\_slides.pdf, slide 22.
\textbf{Analysis:} This requires a two-step process. First, use the Ideal Gas Law to solve for the number of moles, $n$. Second, convert the moles of \ch{N2} to grams using its molar mass.
\textbf{Step-by-Step Solution:}
\begin{enumerate}
    \item Convert temperature to Kelvin: $T = 23 + 273.15 = 296.15$ K.
    \item Rearrange the Ideal Gas Law to solve for $n$: $n = \frac{PV}{RT}$.
    \item Substitute values to find moles:
    \[ n = \frac{(17.1 \text{ atm})(50.0 \text{ L})}{(0.08206 \frac{\text{L}\cdot\text{atm}}{\text{mol}\cdot\text{K}})(296.15 \text{ K})} = 35.18 \text{ mol} \]
    \item Convert moles to grams. The molar mass of \ch{N2} is $2 \times 14.01 = 28.02$ g/mol.
    \[ 35.18 \text{ mol } \ch{N2} \times \frac{28.02 \text{ g } \ch{N2}}{1 \text{ mol } \ch{N2}} = 985.8 \text{ g} \]
\end{enumerate}
\textbf{Answer:} \textbf{986 g \ch{N2}}.

\subsubsection{Example 3}
\textbf{Problem:} What volume will 9.8 moles of sulfur hexafluoride (\ch{SF6}) gas occupy if the temperature and pressure of the gas are 105$^\circ$C and 9.4 atm, respectively?
\textbf{Source:} Questions and Problems (textbook).pdf, Chapter 11, Problem 11.46.
\textbf{Analysis:} A direct application of the Ideal Gas Law, solving for volume ($V$).
\textbf{Step-by-Step Solution:}
\begin{enumerate}
    \item Convert temperature to Kelvin: $T = 105 + 273.15 = 378.15$ K.
    \item Rearrange the Ideal Gas Law for $V$: $V = \frac{nRT}{P}$.
    \item Substitute values:
    \[ V = \frac{(9.8 \text{ mol})(0.08206 \frac{\text{L}\cdot\text{atm}}{\text{mol}\cdot\text{K}})(378.15 \text{ K})}{9.4 \text{ atm}} = 32.35 \text{ L} \]
\end{enumerate}
\textbf{Answer:} \textbf{32 L}.

\subsubsection{Example 4}
\textbf{Problem:} A 3.684 g sample of Freon-22 (\ch{CHF2Cl}; 134.94 g/mol is a typo, it should be 86.47 g/mol) is placed in a 0.4792 L flask and the pressure is 1.539 atm. What is the Celsius temperature of the sample?
\textbf{Source:} Gas Law Problems.pdf, Page 3, Problem 8.
\textbf{Analysis:} We will use the Ideal Gas Law to solve for Temperature. We must first convert the mass of Freon-22 to moles. The final answer must be converted from Kelvin to Celsius. (Using correct Molar Mass for \ch{CHFCl2} of 86.47 g/mol).
\textbf{Step-by-Step Solution:}
\begin{enumerate}
    \item Convert mass to moles ($n$):
    \[ n = 3.684 \text{ g } \ch{CHF2Cl} \times \frac{1 \text{ mol}}{86.47 \text{ g}} = 0.04260 \text{ mol} \]
    \item Rearrange the Ideal Gas Law for $T$: $T = \frac{PV}{nR}$.
    \item Substitute values:
    \[ T = \frac{(1.539 \text{ atm})(0.4792 \text{ L})}{(0.04260 \text{ mol})(0.08206 \frac{\text{L}\cdot\text{atm}}{\text{mol}\cdot\text{K}})} = 211.2 \text{ K} \]
    \item Convert Kelvin to Celsius: $^\circ\text{C} = K - 273.15 = 211.2 - 273.15 = -61.95^\circ$C.
\end{enumerate}
\textbf{Answer:} The source document has a calculation error due to the wrong molar mass. The correct answer is \textbf{-62.0$^\circ$C}.

\section{Topic 7: Molar Mass and Density Calculations}
\subsection{Key Principles}
\begin{itemize}
    \item The Ideal Gas Law can be combined with the definitions of density ($d = m/V$) and moles ($n=m/M$) to derive two very useful equations.
    \item \textbf{Memorization Required: The Formulas}
        \begin{itemize}
            \item To find Molar Mass from density: $M = \frac{dRT}{P}$
            \item To find density from Molar Mass: $d = \frac{MP}{RT}$
        \end{itemize}
    \item These equations provide a direct link between the macroscopic property of density and the microscopic property of molar mass for a gas.
    \item As always, ensure all variables are in units consistent with R (atm, L, K).
\end{itemize}

\subsection{Worked Examples}
\subsubsection{Example 1}
\textbf{Problem:} Determine the molar mass of an unknown gas in a container at -50.0$^\circ$C and 6.00 atm pressure. The density of this gas is 14.5 g/L.
\textbf{Source:} Molar mass, density, gas stoichiometry, gas mixtures.pdf, Problem 1.
\textbf{Analysis:} This is a direct application of the formula $M = \frac{dRT}{P}$. We just need to convert the temperature to Kelvin first.
\textbf{Step-by-Step Solution:}
\begin{enumerate}
    \item Convert temperature to Kelvin: $T = -50.0 + 273.15 = 223.15$ K.
    \item List the known variables: $d=14.5$ g/L, $P=6.00$ atm, $T=223.15$ K.
    \item Substitute into the formula:
    \[ M = \frac{(14.5 \frac{\text{g}}{\text{L}})(0.08206 \frac{\text{L}\cdot\text{atm}}{\text{mol}\cdot\text{K}})(223.15 \text{ K})}{6.00 \text{ atm}} = 44.3 \text{ g/mol} \]
\end{enumerate}
\textbf{Answer:} \textbf{44.3 g/mol}.

\subsubsection{Example 2}
\textbf{Problem:} Calculate the density of hydrogen bromide (\ch{HBr}) gas in g/L at 733 mmHg and 46$^\circ$C.
\textbf{Source:} Questions and Problems (textbook).pdf, Chapter 11, Problem 11.59.
\textbf{Analysis:} This requires using the formula $d = \frac{MP}{RT}$. We need to find the molar mass of \ch{HBr} and convert pressure and temperature to the correct units.
\textbf{Step-by-Step Solution:}
\begin{enumerate}
    \item Find the molar mass of \ch{HBr}: $M = 1.008 + 79.90 = 80.91$ g/mol.
    \item Convert temperature to Kelvin: $T = 46 + 273.15 = 319.15$ K.
    \item Convert pressure to atm: $P = 733 \text{ mmHg} \times \frac{1 \text{ atm}}{760 \text{ mmHg}} = 0.9645$ atm.
    \item Substitute into the formula:
    \[ d = \frac{(80.91 \frac{\text{g}}{\text{mol}})(0.9645 \text{ atm})}{(0.08206 \frac{\text{L}\cdot\text{atm}}{\text{mol}\cdot\text{K}})(319.15 \text{ K})} = 2.98 \text{ g/L} \]
\end{enumerate}
\textbf{Answer:} \textbf{2.98 g/L}.

\subsubsection{Example 3}
\textbf{Problem:} A 2.10-L vessel contains 4.65 g of a gas at 1.00 atm and 27.0$^\circ$C. What is the molar mass of the gas?
\textbf{Source:} Questions and Problems (textbook).pdf, Chapter 11, Problem 11.58.
\textbf{Analysis:} We can solve this in two ways: (1) Calculate the density first, then use $M=dRT/P$. (2) Use $PV=nRT$ to find moles, then use $M=m/n$. Let's use the first method.
\textbf{Step-by-Step Solution:}
\begin{enumerate}
    \item Calculate the density: $d = \frac{m}{V} = \frac{4.65 \text{ g}}{2.10 \text{ L}} = 2.214$ g/L.
    \item Convert temperature to Kelvin: $T = 27.0 + 273.15 = 300.15$ K.
    \item Use the molar mass formula:
    \[ M = \frac{dRT}{P} = \frac{(2.214 \frac{\text{g}}{\text{L}})(0.08206 \frac{\text{L}\cdot\text{atm}}{\text{mol}\cdot\text{K}})(300.15 \text{ K})}{1.00 \text{ atm}} = 54.5 \text{ g/mol} \]
\end{enumerate}
\textbf{Answer:} \textbf{54.5 g/mol}.

\subsubsection{Example 4}
\textbf{Problem:} A 1.16 g sample of an unknown compound is placed in a 0.250 L flask and heated to 563 K. Upon complete vaporization, the pressure is 1.03 atm. What is the molar mass of the compound?
\textbf{Source:} Gas Law Problems.pdf, Page 4, Problem 10.
\textbf{Analysis:} This is another molar mass problem. We can use the modified ideal gas law $M = \frac{mRT}{PV}$ which is derived by substituting $n=m/M$ into $PV=nRT$.
\textbf{Step-by-Step Solution:}
\begin{enumerate}
    \item Identify variables: $m = 1.16$ g, $R = 0.08206$, $T=563$ K, $P=1.03$ atm, $V=0.250$ L.
    \item Substitute into the formula $M = \frac{mRT}{PV}$:
    \[ M = \frac{(1.16 \text{ g})(0.08206 \frac{\text{L}\cdot\text{atm}}{\text{mol}\cdot\text{K}})(563 \text{ K})}{(1.03 \text{ atm})(0.250 \text{ L})} = 208 \text{ g/mol} \]
\end{enumerate}
\textbf{Answer:} \textbf{208 g/mol}.

\section{Topic 8: Gas Stoichiometry}
\subsection{Key Principles}
\begin{itemize}
    \item Gas stoichiometry problems involve chemical reactions where at least one reactant or product is a gas.
    \item The Ideal Gas Law is the bridge between the macroscopic properties of a gas (P, V, T) and the stoichiometric amount (moles, n).
    \item The central path is: Quantity of A $\rightarrow$ Moles of A $\xrightarrow{\text{Mole Ratio}}$ Moles of B $\rightarrow$ Quantity of B.
    \item If you are given P, V, and T of a gas, you use $PV=nRT$ to find its moles. If you need to find P, V, or T of a gas, you use stoichiometry to find its moles first, then use $PV=nRT$.
    \item At STP, you can use the molar volume shortcut (1 mol = 22.4 L).
\end{itemize}

\subsection{Worked Examples}
\subsubsection{Example 1 (STP)}
\textbf{Problem:} How many liters of \ch{Cl2} at STP is required to react with 50.0 grams of \ch{Na}? The reaction is: \ch{2Na(s) + Cl2(g) -> 2NaCl(s)}.
\textbf{Source:} lecture\_slides.pdf, slide 28.
\textbf{Analysis:} This is a standard stoichiometry problem, with the final step involving a gas at STP. Path: g Na $\rightarrow$ mol Na $\rightarrow$ mol \ch{Cl2} $\rightarrow$ L \ch{Cl2}.
\textbf{Step-by-Step Solution:}
\begin{enumerate}
    \item Set up the dimensional analysis:
    \[ 50.0 \text{ g } \ch{Na} \times \frac{1 \text{ mol } \ch{Na}}{22.99 \text{ g } \ch{Na}} \times \frac{1 \text{ mol } \ch{Cl2}}{2 \text{ mol } \ch{Na}} \times \frac{22.4 \text{ L } \ch{Cl2}}{1 \text{ mol } \ch{Cl2}} \]
    \item Calculate the result:
    \[ = 2.175 \text{ mol Na} \times \frac{1 \text{ mol } \ch{Cl2}}{2 \text{ mol } \ch{Na}} \times 22.4 \frac{\text{L}}{\text{mol}} = 1.0875 \text{ mol } \ch{Cl2} \times 22.4 \frac{\text{L}}{\text{mol}} = 24.36 \text{ L} \]
\end{enumerate}
\textbf{Answer:} \textbf{24.4 L \ch{Cl2}}.

\subsubsection{Example 2 (Non-STP)}
\textbf{Problem:} What volume of oxygen gas is produced when 22.2 grams of potassium chlorate is decomposed at 26.0$^\circ$C and a pressure of 735 mmHg? The reaction is \ch{2KClO3(s) -> 2KCl(s) + 3O2(g)}. Molar mass of \ch{KClO3} is 122.55 g/mol.
\textbf{Source:} lecture\_slides.pdf, slide 30.
\textbf{Analysis:} This is a gas stoichiometry problem not at STP. Path: g \ch{KClO3} $\rightarrow$ mol \ch{KClO3} $\rightarrow$ mol \ch{O2}. Then, use $PV=nRT$ to find the volume of \ch{O2}.
\textbf{Step-by-Step Solution:}
\begin{enumerate}
    \item Convert mass of \ch{KClO3} to moles of \ch{O2}:
    \[ 22.2 \text{ g } \ch{KClO3} \times \frac{1 \text{ mol } \ch{KClO3}}{122.55 \text{ g } \ch{KClO3}} \times \frac{3 \text{ mol } \ch{O2}}{2 \text{ mol } \ch{KClO3}} = 0.2717 \text{ mol } \ch{O2} \]
    \item Convert T and P to the correct units for the Ideal Gas Law:
    \begin{itemize}
        \item $T = 26.0 + 273.15 = 299.15$ K
        \item $P = 735 \text{ mmHg} \times \frac{1 \text{ atm}}{760 \text{ mmHg}} = 0.9671$ atm
    \end{itemize}
    \item Rearrange $PV=nRT$ to solve for V: $V = \frac{nRT}{P}$.
    \item Substitute values:
    \[ V = \frac{(0.2717 \text{ mol})(0.08206 \frac{\text{L}\cdot\text{atm}}{\text{mol}\cdot\text{K}})(299.15 \text{ K})}{0.9671 \text{ atm}} = 6.90 \text{ L} \]
\end{enumerate}
\textbf{Answer:} \textbf{6.90 L \ch{O2}}.

\subsubsection{Example 3}
\textbf{Problem:} What volume of \ch{NH3} (in L) must be used at 244 torr and 35$^\circ$C to produce 2.3 kg of \ch{HCl} gas? Reaction: \ch{2NH3(g) + 3Cl2(g) -> N2(g) + 6HCl(g)}.
\textbf{Source:} Molar mass, density, gas stoichiometry, gas mixtures.pdf, Problem 6.
\textbf{Analysis:} This is a product-to-reactant stoichiometry problem. Path: kg \ch{HCl} $\rightarrow$ g \ch{HCl} $\rightarrow$ mol \ch{HCl} $\rightarrow$ mol \ch{NH3}. Then use $PV=nRT$ to find the volume of \ch{NH3}.
\textbf{Step-by-Step Solution:}
\begin{enumerate}
    \item Convert mass of \ch{HCl} to moles of \ch{NH3}. Molar mass of \ch{HCl} is 36.46 g/mol.
    \[ 2.3 \text{ kg } \ch{HCl} \times \frac{1000 \text{ g}}{1 \text{ kg}} \times \frac{1 \text{ mol } \ch{HCl}}{36.46 \text{ g } \ch{HCl}} \times \frac{2 \text{ mol } \ch{NH3}}{6 \text{ mol } \ch{HCl}} = 21.0 \text{ mol } \ch{NH3} \]
    \item Convert T and P to the correct units:
    \begin{itemize}
        \item $T = 35 + 273.15 = 308.15$ K
        \item $P = 244 \text{ torr} \times \frac{1 \text{ atm}}{760 \text{ torr}} = 0.321$ atm
    \end{itemize}
    \item Use $V = \frac{nRT}{P}$ to find the volume of \ch{NH3}:
    \[ V = \frac{(21.0 \text{ mol})(0.08206 \frac{\text{L}\cdot\text{atm}}{\text{mol}\cdot\text{K}})(308.15 \text{ K})}{0.321 \text{ atm}} = 1653 \text{ L} \]
\end{enumerate}
\textbf{Answer:} \textbf{1.7 x 10$^3$ L \ch{NH3}}.

\subsubsection{Example 4}
\textbf{Problem:} If 9.0 L of \ch{NO} is combined with excess \ch{O2} at STP, what is the volume in liters of the \ch{NO2} produced? Reaction: \ch{2NO(g) + O2(g) -> 2NO2(g)}.
\textbf{Source:} Questions and Problems (textbook).pdf, Chapter 11, Problem 11.86.
\textbf{Analysis:} According to Avogadro's law, at constant temperature and pressure (like STP), the ratio of volumes is equal to the ratio of moles. We can use the stoichiometric coefficients directly on the volumes.
\textbf{Step-by-Step Solution:}
\begin{enumerate}
    \item Identify the mole ratio from the balanced equation: 2 mol \ch{NO} produces 2 mol \ch{NO2}. This is a 1:1 ratio.
    \item Apply this ratio to the volumes:
    \[ 9.0 \text{ L } \ch{NO} \times \frac{2 \text{ L } \ch{NO2}}{2 \text{ L } \ch{NO}} = 9.0 \text{ L } \ch{NO2} \]
\end{enumerate}
\textbf{Answer:} \textbf{9.0 L}.

\section{Topic 9: Partial Pressures \& Mole Fraction}
\subsection{Key Principles}
\begin{itemize}
    \item \textbf{Dalton's Law of Partial Pressures:} The total pressure of a gas mixture is the sum of the partial pressures of each individual gas. $P_{total} = P_A + P_B + \dots$
    \item \textbf{Partial Pressure ($P_A$):} The pressure a single gas in the mixture would exert if it were the only gas in the container. It can be calculated using the ideal gas law for that gas alone: $P_A V = n_A RT$.
    \item \textbf{Mole Fraction ($\chi_A$):} The ratio of the moles of one component to the total moles in the mixture. It is a dimensionless quantity. $\chi_A = \frac{n_A}{n_{total}}$.
    \item The partial pressure of a gas is equal to its mole fraction multiplied by the total pressure: $P_A = \chi_A P_{total}$.
\end{itemize}

\subsection{Worked Examples}
\subsubsection{Example 1}
\textbf{Problem:} A 10.0 L flask contains 1.03 g \ch{O2} and 0.572 g \ch{CO2} at 291.15 K.
a) What are the partial pressures of \ch{O2} and \ch{CO2}?
b) What is the total pressure?
c) What is the mole fraction of \ch{O2}?
\textbf{Source:} lecture\_slides.pdf, slides 38-40.
\textbf{Analysis:} First, we must convert the mass of each gas to moles. Then, we can use the Ideal Gas Law for each gas individually to find its partial pressure. The total pressure is the sum of the partial pressures. The mole fraction can be found from either the ratio of moles or the ratio of pressures.
\textbf{Step-by-Step Solution:}
\begin{enumerate}
    \item \textbf{Convert masses to moles:}
    \begin{itemize}
        \item Moles \ch{O2} ($n_{O2}$): $1.03 \text{ g} \times \frac{1 \text{ mol}}{32.00 \text{ g}} = 0.03218$ mol
        \item Moles \ch{CO2} ($n_{CO2}$): $0.572 \text{ g} \times \frac{1 \text{ mol}}{44.01 \text{ g}} = 0.01300$ mol
    \end{itemize}
    \item \textbf{a) Calculate partial pressures using $P = nRT/V$:}
    \begin{itemize}
        \item $P_{O2} = \frac{(0.03218 \text{ mol})(0.08206 \frac{\text{L}\cdot\text{atm}}{\text{mol}\cdot\text{K}})(291.15 \text{ K})}{10.0 \text{ L}} = 0.0769$ atm
        \item $P_{CO2} = \frac{(0.01300 \text{ mol})(0.08206 \frac{\text{L}\cdot\text{atm}}{\text{mol}\cdot\text{K}})(291.15 \text{ K})}{10.0 \text{ L}} = 0.0311$ atm
    \end{itemize}
    \item \textbf{b) Calculate total pressure:}
    \[ P_{total} = P_{O2} + P_{CO2} = 0.0769 \text{ atm} + 0.0311 \text{ atm} = 0.1080 \text{ atm} \]
    \item \textbf{c) Calculate mole fraction of \ch{O2}:}
    \[ \chi_{O2} = \frac{P_{O2}}{P_{total}} = \frac{0.0769 \text{ atm}}{0.1080 \text{ atm}} = 0.712 \]
    (Alternatively, using moles: $n_{total} = 0.03218 + 0.01300 = 0.04518$ mol. $\chi_{O2} = \frac{0.03218}{0.04518} = 0.712$)
\end{enumerate}
\textbf{Answer:} a) \textbf{$P_{O2}=0.0769$ atm, $P_{CO2}=0.0311$ atm}; b) \textbf{$P_{total}=0.1080$ atm}; c) \textbf{$\chi_{O2}=0.712$}.

\subsubsection{Example 2}
\textbf{Problem:} A cylinder has a volume of 20.0 L and contains 1813 g of \ch{CH4} and 336 g of \ch{O2} at 22.0$^\circ$C. Calculate the partial pressure of each gas and the total pressure.
\textbf{Source:} Molar mass, density, gas stoichiometry, gas mixtures.pdf, Problem 7a.
\textbf{Analysis:} Same procedure as Example 1: convert mass to moles for each gas, then use the Ideal Gas Law to find each partial pressure, and finally sum them for the total pressure.
\textbf{Step-by-Step Solution:}
\begin{enumerate}
    \item \textbf{Convert masses to moles:}
    \begin{itemize}
        \item Moles \ch{CH4}: $1813 \text{ g} \times \frac{1 \text{ mol}}{16.04 \text{ g}} = 113.0$ mol
        \item Moles \ch{O2}: $336 \text{ g} \times \frac{1 \text{ mol}}{32.00 \text{ g}} = 10.50$ mol
    \end{itemize}
    \item \textbf{Convert temperature to Kelvin:} $T = 22.0 + 273.15 = 295.15$ K.
    \item \textbf{Calculate partial pressures:}
    \begin{itemize}
        \item $P_{CH4} = \frac{(113.0 \text{ mol})(0.08206)(295.15 \text{ K})}{20.0 \text{ L}} = 136.9$ atm
        \item $P_{O2} = \frac{(10.50 \text{ mol})(0.08206)(295.15 \text{ K})}{20.0 \text{ L}} = 12.7$ atm
    \end{itemize}
    \item \textbf{Calculate total pressure:}
    \[ P_{total} = P_{CH4} + P_{O2} = 136.9 \text{ atm} + 12.7 \text{ atm} = 149.6 \text{ atm} \]
\end{enumerate}
\textbf{Answer:} \textbf{$P_{CH4}=136.9$ atm, $P_{O2}=12.7$ atm, $P_{total}=149.6$ atm}.

\subsubsection{Example 3}
\textbf{Problem:} A mixture of gases contains 0.31 mol \ch{CH4}, 0.25 mol \ch{C2H6}, and 0.29 mol \ch{C3H8}. The total pressure is 1.50 atm. Calculate the partial pressures of the gases.
\textbf{Source:} Questions and Problems (textbook).pdf, Chapter 11, Problem 11.76.
\textbf{Analysis:} We are given moles and total pressure. The most direct method is to calculate the mole fraction of each gas and then use $P_A = \chi_A P_{total}$.
\textbf{Step-by-Step Solution:}
\begin{enumerate}
    \item \textbf{Calculate total moles:} $n_{total} = 0.31 + 0.25 + 0.29 = 0.85$ mol.
    \item \textbf{Calculate mole fractions:}
    \begin{itemize}
        \item $\chi_{CH4} = \frac{0.31}{0.85} = 0.365$
        \item $\chi_{C2H6} = \frac{0.25}{0.85} = 0.294$
        \item $\chi_{C3H8} = \frac{0.29}{0.85} = 0.341$
    \end{itemize}
    \item \textbf{Calculate partial pressures:}
    \begin{itemize}
        \item $P_{CH4} = 0.365 \times 1.50 \text{ atm} = 0.547$ atm
        \item $P_{C2H6} = 0.294 \times 1.50 \text{ atm} = 0.441$ atm
        \item $P_{C3H8} = 0.341 \times 1.50 \text{ atm} = 0.512$ atm
    \end{itemize}
\end{enumerate}
\textbf{Answer:} \textbf{$P_{CH4} \approx 0.55$ atm, $P_{C2H6} \approx 0.44$ atm, $P_{C3H8} \approx 0.51$ atm}.

\subsubsection{Example 4}
\textbf{Problem:} A 2.5-L flask at 15$^\circ$C contains a mixture of \ch{N2}, He, and Ne at partial pressures of 0.32 atm for \ch{N2}, 0.15 atm for He, and 0.42 atm for Ne. Calculate the total pressure.
\textbf{Source:} Questions and Problems (textbook).pdf, Chapter 11, Problem 11.77a.
\textbf{Analysis:} This is a direct application of Dalton's Law of Partial Pressures.
\textbf{Step-by-Step Solution:}
\begin{enumerate}
    \item \textbf{State Dalton's Law:} $P_{total} = P_{N2} + P_{He} + P_{Ne}$.
    \item \textbf{Substitute values and sum:}
    \[ P_{total} = 0.32 \text{ atm} + 0.15 \text{ atm} + 0.42 \text{ atm} = 0.89 \text{ atm} \]
\end{enumerate}
\textbf{Answer:} \textbf{0.89 atm}.

\section{Topic 10: Gas Collection Over Water}
\subsection{Key Principles}
\begin{itemize}
    \item When a gas is collected by bubbling it through water, the collected gas is not pure. It is a mixture of the desired gas and water vapor.
    \item The total pressure measured inside the collection tube ($P_{total}$) is the sum of the partial pressure of the collected gas ($P_{gas}$) and the partial pressure of the water vapor ($P_{H2O}$).
    \item $P_{total} = P_{gas} + P_{H2O}$
    \item The vapor pressure of water ($P_{H2O}$) depends only on the temperature. You will be given this value or a table to find it (like in \textit{lecture\_slides.pdf, slide 42}).
    \item \textbf{The Key Step:} Before using the collected gas pressure in an ideal gas law calculation, you MUST first subtract the vapor pressure of water to find the partial pressure of the dry gas.
    \[ P_{gas} = P_{total} - P_{H2O} \]
\end{itemize}

\subsection{Worked Examples}
\subsubsection{Example 1}
\textbf{Problem:} If 0.0875 L of nitrogen gas was collected over water at 296.15 K (23$^\circ$C) and 727 mmHg, how many grams of \ch{NH4NO2} has decomposed? Reaction: \ch{NH4NO2(s) -> N2(g) + 2H2O(l)}. The vapor pressure of water at 23$^\circ$C is 21.1 mmHg (Note: slide has 22.1, likely a typo, using 21.1 from standard tables).
\textbf{Source:} lecture\_slides.pdf, slides 43-44.
\textbf{Analysis:} This is a gas stoichiometry problem that involves collecting a gas over water. First, we must find the partial pressure of the dry \ch{N2}. Then use $PV=nRT$ to find the moles of \ch{N2}. Finally, use stoichiometry to find the mass of \ch{NH4NO2}.
\textbf{Step-by-Step Solution:}
\begin{enumerate}
    \item \textbf{Find the partial pressure of \ch{N2}:}
    \[ P_{N2} = P_{total} - P_{H2O} = 727 \text{ mmHg} - 21.1 \text{ mmHg} = 705.9 \text{ mmHg} \]
    \item \textbf{Convert $P_{N2}$ to atm:}
    \[ 705.9 \text{ mmHg} \times \frac{1 \text{ atm}}{760 \text{ mmHg}} = 0.9288 \text{ atm} \]
    \item \textbf{Use the Ideal Gas Law to find moles of \ch{N2}:} $n = \frac{PV}{RT}$.
    \[ n_{N2} = \frac{(0.9288 \text{ atm})(0.0875 \text{ L})}{(0.08206 \frac{\text{L}\cdot\text{atm}}{\text{mol}\cdot\text{K}})(296.15 \text{ K})} = 0.00334 \text{ mol } \ch{N2} \]
    \item \textbf{Use stoichiometry to find grams of \ch{NH4NO2}:} The mole ratio is 1:1. The molar mass of \ch{NH4NO2} is 64.06 g/mol.
    \[ 0.00334 \text{ mol } \ch{N2} \times \frac{1 \text{ mol } \ch{NH4NO2}}{1 \text{ mol } \ch{N2}} \times \frac{64.06 \text{ g } \ch{NH4NO2}}{1 \text{ mol } \ch{NH4NO2}} = 0.214 \text{ g} \]
\end{enumerate}
\textbf{Answer:} \textbf{0.214 g \ch{NH4NO2}}.

\subsubsection{Example 2}
\textbf{Problem:} The hydrogen gas from the reaction \ch{Zn(s) + 2HCl(aq) -> ZnCl2(aq) + H2(g)} is collected over water at 25.0$^\circ$C. The volume of the gas is 7.80 L, and the total pressure is 0.980 atm. Calculate the amount of zinc metal in grams consumed. (Vapor pressure of water at 25$^\circ$C = 23.8 mmHg).
\textbf{Source:} Questions and Problems (textbook).pdf, Chapter 11, Problem 11.97.
\textbf{Analysis:} Similar to the previous problem. Find partial pressure of \ch{H2}, use ideal gas law to find moles \ch{H2}, then use stoichiometry to find grams of \ch{Zn}.
\textbf{Step-by-Step Solution:}
\begin{enumerate}
    \item \textbf{Convert pressures to be in the same units (atm):}
    \[ P_{H2O} = 23.8 \text{ mmHg} \times \frac{1 \text{ atm}}{760 \text{ mmHg}} = 0.0313 \text{ atm} \]
    \item \textbf{Find the partial pressure of \ch{H2}:}
    \[ P_{H2} = P_{total} - P_{H2O} = 0.980 \text{ atm} - 0.0313 \text{ atm} = 0.9487 \text{ atm} \]
    \item \textbf{Convert temperature to Kelvin:} $T = 25.0 + 273.15 = 298.15$ K.
    \item \textbf{Use the Ideal Gas Law to find moles of \ch{H2}:}
    \[ n_{H2} = \frac{(0.9487 \text{ atm})(7.80 \text{ L})}{(0.08206 \frac{\text{L}\cdot\text{atm}}{\text{mol}\cdot\text{K}})(298.15 \text{ K})} = 0.303 \text{ mol } \ch{H2} \]
    \item \textbf{Use stoichiometry to find grams of Zn:} The mole ratio is 1:1. Molar mass of Zn is 65.39 g/mol.
    \[ 0.303 \text{ mol } \ch{H2} \times \frac{1 \text{ mol } \ch{Zn}}{1 \text{ mol } \ch{H2}} \times \frac{65.39 \text{ g } \ch{Zn}}{1 \text{ mol } \ch{Zn}} = 19.8 \text{ g} \]
\end{enumerate}
\textbf{Answer:} \textbf{19.8 g Zn}.

\subsubsection{Example 3}
\textbf{Problem:} A piece of sodium metal reacts completely with water: \ch{2Na(s) + 2H2O(l) -> 2NaOH(aq) + H2(g)}. The hydrogen gas is collected over water at 25.0$^\circ$C. The volume is 246 mL at 1.00 atm. Calculate grams of sodium used. (Vapor pressure of water at 25$^\circ$C = 0.0313 atm).
\textbf{Source:} Questions and Problems (textbook).pdf, Chapter 11, Problem 11.96.
\textbf{Analysis:} Another problem following the same pattern.
\textbf{Step-by-Step Solution:}
\begin{enumerate}
    \item \textbf{Find the partial pressure of \ch{H2}:}
    \[ P_{H2} = P_{total} - P_{H2O} = 1.00 \text{ atm} - 0.0313 \text{ atm} = 0.9687 \text{ atm} \]
    \item \textbf{Convert volume to L and temperature to K:} $V = 0.246$ L, $T = 298.15$ K.
    \item \textbf{Use the Ideal Gas Law to find moles of \ch{H2}:}
    \[ n_{H2} = \frac{(0.9687 \text{ atm})(0.246 \text{ L})}{(0.08206)(298.15 \text{ K})} = 0.00974 \text{ mol } \ch{H2} \]
    \item \textbf{Use stoichiometry to find grams of Na:} The mole ratio is 2 mol Na to 1 mol \ch{H2}. Molar mass of Na is 22.99 g/mol.
    \[ 0.00974 \text{ mol } \ch{H2} \times \frac{2 \text{ mol } \ch{Na}}{1 \text{ mol } \ch{H2}} \times \frac{22.99 \text{ g } \ch{Na}}{1 \text{ mol } \ch{Na}} = 0.448 \text{ g} \]
\end{enumerate}
\textbf{Answer:} \textbf{0.45 g Na}.

\subsubsection{Example 4}
\textbf{Problem:} A mixture of helium and neon gases is collected over water at 28.0$^\circ$C and 745 mmHg. If the partial pressure of helium is 368 mmHg, what is the partial pressure of neon? (Vapor pressure of water at 28$^\circ$C = 28.3 mmHg).
\textbf{Source:} Questions and Problems (textbook).pdf, Chapter 11, Problem 11.79.
\textbf{Analysis:} This problem uses Dalton's Law for a collected gas mixture. The total pressure is the sum of the partial pressures of He, Ne, and water vapor.
\textbf{Step-by-Step Solution:}
\begin{enumerate}
    \item \textbf{State Dalton's Law for this mixture:} $P_{total} = P_{He} + P_{Ne} + P_{H2O}$.
    \item \textbf{Rearrange to solve for $P_{Ne}$:} $P_{Ne} = P_{total} - P_{He} - P_{H2O}$.
    \item \textbf{Substitute the given values:}
    \[ P_{Ne} = 745 \text{ mmHg} - 368 \text{ mmHg} - 28.3 \text{ mmHg} = 348.7 \text{ mmHg} \]
\end{enumerate}
\textbf{Answer:} \textbf{349 mmHg}.

\section{Topic 11: KMT: Speed vs. Molar Mass}
\subsection{Key Principles}
\begin{itemize}
    \item The Kinetic Molecular Theory (KMT) states that at a given temperature, all gases in a mixture have the \textbf{same average kinetic energy}.
    \item Kinetic energy is defined as $KE = \frac{1}{2}mv^2$. For this to be constant across different gases, a gas with a larger mass ($m$) must have a smaller average velocity ($v$), and vice versa.
    \item This means there is an \textbf{inverse relationship} between molar mass and molecular speed: lighter gases move faster, heavier gases move slower.
    \item On a velocity distribution graph for different gases at the same temperature:
        \begin{itemize}
            \item The peak for the \textbf{heavier gas} will be further to the left (lower speed) and taller.
            \item The peak for the \textbf{lighter gas} will be further to the right (higher speed) and shorter/flatter.
        \end{itemize}
\end{itemize}

\subsection{Worked Examples}
\subsubsection{Example 1}
\textbf{Problem:} Consider the graph comparing the velocity distribution of \ch{PH3} and Kr at the same temperature. Identify which curve represents Kr.
\textbf{Source:} Kinetic molecular theory.pdf, Problem 2a.
\textbf{Analysis:} We need to compare the molar masses of \ch{PH3} and Kr. The heavier gas will be the slower one (peak to the left).
\textbf{Step-by-Step Solution:}
\begin{enumerate}
    \item Calculate molar masses:
    \begin{itemize}
        \item M(\ch{PH3}) = $30.97 + 3(1.008) \approx 34.0$ g/mol.
        \item M(Kr) = $83.80$ g/mol.
    \end{itemize}
    \item Compare: Kr is significantly heavier than \ch{PH3}.
    \item Relate to the graph: The heavier gas (Kr) will have a lower average speed. Curve A has its peak at a lower velocity than Curve B. Therefore, Curve A represents Kr.
\end{enumerate}
\textbf{Answer:} \textbf{Curve A}.

\subsubsection{Example 2}
\textbf{Problem:} Using the same graph as Example 1, which curve represents the gas with a higher velocity?
\textbf{Source:} Kinetic molecular theory.pdf, Problem 2d.
\textbf{Analysis:} The gas with the higher velocity will be the lighter gas. Its curve will be shifted to the right.
\textbf{Step-by-Step Solution:}
\begin{enumerate}
    \item From the previous example, we know \ch{PH3} is lighter than Kr.
    \item Lighter gases move faster.
    \item Curve B has its peak at a higher velocity (further right) than Curve A.
    \item Therefore, Curve B represents the gas with the higher velocity.
\end{enumerate}
\textbf{Answer:} \textbf{Curve B}.

\subsubsection{Example 3}
\textbf{Problem:} The velocity distribution for four different molecules (A, B, C, D) at the same temperature is shown. Which gas has the largest molar mass?
\textbf{Source:} Kinetic molecular theory.pdf, Problem 3a.
\textbf{Analysis:} The gas with the largest molar mass will be the heaviest, and therefore the slowest. We need to find the curve with the lowest average speed (peak furthest to the left).
\textbf{Step-by-Step Solution:}
\begin{enumerate}
    \item The relationship is: largest molar mass $\iff$ heaviest gas $\iff$ slowest average speed.
    \item On the graph, the horizontal axis is molecular speed. Lower speed is to the left.
    \item Curve A has its peak further to the left than any other curve, indicating it has the lowest average speed.
    \item Therefore, Gas A has the largest molar mass.
\end{enumerate}
\textbf{Answer:} \textbf{Gas A}.

\subsubsection{Example 4}
\textbf{Problem:} True or false: Gas "A" has a smaller molar mass than gas "D".
\textbf{Source:} Kinetic molecular theory.pdf, Problem 3c.
\textbf{Analysis:} We compare the positions of the peaks for A and D. The peak further to the right corresponds to a higher speed and thus a smaller molar mass.
\textbf{Step-by-Step Solution:}
\begin{enumerate}
    \item Curve D's peak is much further to the right than Curve A's peak.
    \item This means Gas D has a much higher average speed than Gas A.
    \item Since speed is inversely proportional to the square root of molar mass, the faster gas (D) must have a smaller molar mass than the slower gas (A).
    \item The statement that Gas "A" has a smaller molar mass than gas "D" is therefore false.
\end{enumerate}
\textbf{Answer:} \textbf{False}.

\section{Topic 12: Diffusion and Effusion}
\subsection{Key Principles}
\begin{itemize}
    \item \textbf{Diffusion:} The mixing of gases as a result of their random motion and frequent collisions.
    \item \textbf{Effusion:} The escape of gas molecules through a tiny hole into a vacuum.
    \item \textbf{Graham's Law of Effusion:} The rate of effusion of a gas is inversely proportional to the square root of its molar mass ($M$).
    \[ \text{rate} \propto \frac{1}{\sqrt{M}} \]
    \item This means that lighter gases effuse (and diffuse) more rapidly than heavier gases.
    \item Because rate is distance/time, the time it takes for a gas to effuse is directly proportional to the square root of its molar mass ($time \propto \sqrt{M}$). A heavier gas will take longer to effuse.
\end{itemize}

\subsection{Worked Examples}
\subsubsection{Example 1}
\textbf{Problem:} Which gas effuses most rapidly: A, B, C, or D from the graph in Topic 11, Example 3?
\textbf{Source:} Kinetic molecular theory.pdf, Problem 3b.
\textbf{Analysis:} The gas that effuses most rapidly is the one with the highest average speed, which corresponds to the one with the lowest molar mass. We look for the peak that is furthest to the right.
\textbf{Step-by-Step Solution:}
\begin{enumerate}
    \item Rate of effusion is fastest for the gas with the highest speed.
    \item The graph shows that Gas D has the highest average speed (peak is furthest right).
    \item Therefore, Gas D effuses most rapidly.
\end{enumerate}
\textbf{Answer:} \textbf{Gas D}.

\subsubsection{Example 2}
\textbf{Problem:} Which curve represents the gas that diffuses more slowly, \ch{PH3} or Kr? (Using the graph from Topic 11, Example 1).
\textbf{Source:} Kinetic molecular theory.pdf, Problem 2c.
\textbf{Analysis:} The gas that diffuses more slowly is the heavier gas (larger molar mass).
\textbf{Step-by-Step Solution:}
\begin{enumerate}
    \item We know from before that Kr (M = 83.8 g/mol) is heavier than \ch{PH3} (M = 34.0 g/mol).
    \item The heavier gas diffuses more slowly.
    \item Curve A represents Kr. Therefore, Curve A represents the gas that diffuses more slowly.
\end{enumerate}
\textbf{Answer:} \textbf{Curve A}.

\subsubsection{Example 3}
\textbf{Problem:} Arrange the following species in order of increasing rate of effusion: \ch{Cl2}, Ar, HI.
\textbf{Source:} Kinetic molecular theory.pdf, Problem 6.
\textbf{Analysis:} The rate of effusion is inversely proportional to the square root of the molar mass. To arrange them in order of increasing rate, we must first arrange them in order of decreasing molar mass (heaviest to lightest).
\textbf{Step-by-Step Solution:}
\begin{enumerate}
    \item Calculate the molar masses:
    \begin{itemize}
        \item M(HI) = $126.9 + 1.008 = 127.9$ g/mol
        \item M(\ch{Cl2}) = $2 \times 35.45 = 70.90$ g/mol
        \item M(Ar) = $39.95$ g/mol
    \end{itemize}
    \item Order by molar mass (heaviest to lightest): HI > \ch{Cl2} > Ar.
    \item The rate of effusion will be the reverse order (slowest to fastest): HI < \ch{Cl2} < Ar.
\end{enumerate}
\textbf{Answer:} The order of increasing rate of effusion is \textbf{HI < \ch{Cl2} < Ar}.

\subsubsection{Example 4}
\textbf{Problem:} It takes 54 seconds for \ch{N2O} gas to effuse from a container. How much time would it take for the same amount of \ch{I2} gas to effuse under the same conditions?
\textbf{Source:} Kinetic molecular theory.pdf, Problem 7.
\textbf{Analysis:} The time of effusion is directly proportional to the square root of the molar mass. We can set up a ratio: $\frac{\text{time}_A}{\text{time}_B} = \sqrt{\frac{M_A}{M_B}}$.
\textbf{Step-by-Step Solution:}
\begin{enumerate}
    \item Calculate molar masses:
    \begin{itemize}
        \item M(\ch{N2O}) = $2(14.01) + 16.00 = 44.02$ g/mol
        \item M(\ch{I2}) = $2 \times 126.9 = 253.8$ g/mol
    \end{itemize}
    \item Set up the ratio equation. Let A = \ch{I2} and B = \ch{N2O}.
    \[ \frac{\text{time}_{I2}}{\text{time}_{N2O}} = \sqrt{\frac{M_{I2}}{M_{N2O}}} \]
    \item Rearrange and solve for time$_{I2}$:
    \[ \text{time}_{I2} = \text{time}_{N2O} \times \sqrt{\frac{M_{I2}}{M_{N2O}}} = 54 \text{ s} \times \sqrt{\frac{253.8 \text{ g/mol}}{44.02 \text{ g/mol}}} \]
    \[ \text{time}_{I2} = 54 \text{ s} \times \sqrt{5.765} = 54 \text{ s} \times 2.40 = 129.6 \text{ s} \]
\end{enumerate}
\textbf{Answer:} \textbf{130 seconds}.

\section{Topic 13: Ratio of Molecular Speeds (Graham's Law)}
\subsection{Key Principles}
\begin{itemize}
    \item Graham's Law provides a quantitative relationship to compare the rates (or speeds) of effusion/diffusion of two different gases, A and B.
    \item \textbf{Memorization Required: The Formula}
    \[ \frac{\text{rate}_A}{\text{rate}_B} = \frac{u_{rms, A}}{u_{rms, B}} = \sqrt{\frac{M_B}{M_A}} \]
    \item Note the inverse relationship: Gas A is in the numerator on the left, but its molar mass $M_A$ is in the denominator on the right.
    \item This equation can be used to find the rate ratio if molar masses are known, or to find an unknown molar mass if the rate ratio is known.
\end{itemize}

\subsection{Worked Examples}
\subsubsection{Example 1}
\textbf{Problem:} An unknown gas diffuses 1.12 times faster than argon (Ar) through a porous membrane. Calculate the molar mass of the unknown gas.
\textbf{Source:} Kinetic molecular theory.pdf, Problem 4.
\textbf{Analysis:} This is a direct application of Graham's Law to find an unknown molar mass. Let the unknown gas be X. The ratio of rates is given as $\frac{\text{rate}_X}{\text{rate}_{Ar}} = 1.12$.
\textbf{Step-by-Step Solution:}
\begin{enumerate}
    \item Look up the molar mass of Argon: M(Ar) = 39.95 g/mol.
    \item Set up Graham's Law:
    \[ \frac{\text{rate}_X}{\text{rate}_{Ar}} = \sqrt{\frac{M_{Ar}}{M_X}} \implies 1.12 = \sqrt{\frac{39.95 \text{ g/mol}}{M_X}} \]
    \item Square both sides to remove the square root:
    \[ (1.12)^2 = \frac{39.95}{M_X} \implies 1.2544 = \frac{39.95}{M_X} \]
    \item Rearrange and solve for $M_X$:
    \[ M_X = \frac{39.95 \text{ g/mol}}{1.2544} = 31.84 \text{ g/mol} \]
\end{enumerate}
\textbf{Answer:} \textbf{31.8 g/mol}.

\subsubsection{Example 2}
\textbf{Problem:} Calculate the diffusion rate of hydrobromic acid (\ch{HBr}) relative to Xenon (Xe).
\textbf{Source:} Kinetic molecular theory.pdf, Problem 5.
\textbf{Analysis:} The question asks for the ratio of the rates, $\frac{\text{rate}_{HBr}}{\text{rate}_{Xe}}$. We can calculate this using their molar masses.
\textbf{Step-by-Step Solution:}
\begin{enumerate}
    \item Find the molar masses:
    \begin{itemize}
        \item M(\ch{HBr}) = $1.008 + 79.90 = 80.91$ g/mol.
        \item M(Xe) = $131.29$ g/mol.
    \end{itemize}
    \item Set up the Graham's Law ratio:
    \[ \frac{\text{rate}_{HBr}}{\text{rate}_{Xe}} = \sqrt{\frac{M_{Xe}}{M_{HBr}}} = \sqrt{\frac{131.29 \text{ g/mol}}{80.91 \text{ g/mol}}} \]
    \item Calculate the result:
    \[ \frac{\text{rate}_{HBr}}{\text{rate}_{Xe}} = \sqrt{1.6227} = 1.27 \]
\end{enumerate}
\textbf{Answer:} \ch{HBr} diffuses \textbf{1.27 times faster} than Xe.

\subsubsection{Example 3}
\textbf{Problem:} Determine the molar mass and identity of a diatomic gas that moves 4.67 times as fast as \ch{CO2}.
\textbf{Source:} lecture\_slides.pdf, slide 54.
\textbf{Analysis:} Similar to Example 1. Let the unknown diatomic gas be X. The rate ratio is $\frac{\text{rate}_X}{\text{rate}_{CO2}} = 4.67$. We need to solve for $M_X$.
\textbf{Step-by-Step Solution:}
\begin{enumerate}
    \item Molar mass of \ch{CO2} is 44.01 g/mol.
    \item Set up Graham's Law:
    \[ 4.67 = \sqrt{\frac{M_{CO2}}{M_X}} = \sqrt{\frac{44.01 \text{ g/mol}}{M_X}} \]
    \item Square both sides:
    \[ (4.67)^2 = \frac{44.01}{M_X} \implies 21.81 = \frac{44.01}{M_X} \]
    \item Solve for $M_X$:
    \[ M_X = \frac{44.01 \text{ g/mol}}{21.81} = 2.018 \text{ g/mol} \]
    \item Identify the diatomic gas: A diatomic gas with a molar mass of ~2 g/mol is hydrogen, \ch{H2}.
\end{enumerate}
\textbf{Answer:} Molar Mass = \textbf{2.02 g/mol}; Identity = \textbf{\ch{H2}}.

\subsubsection{Example 4}
\textbf{Problem:} How much faster does \ch{^235UF6} effuse than \ch{^238UF6}? Molar mass of F is 19.00 g/mol.
\textbf{Source:} Questions and Problems (textbook).pdf, Chapter 11, Problem 11.12.
\textbf{Analysis:} We need to find the ratio of the rates, $\frac{\text{rate}_{235}}{\text{rate}_{238}}$. This requires calculating the molar mass of each isotopic compound.
\textbf{Step-by-Step Solution:}
\begin{enumerate}
    \item Calculate the molar masses:
    \begin{itemize}
        \item M(\ch{^235UF6}) = $235.04 + 6(19.00) = 349.04$ g/mol
        \item M(\ch{^238UF6}) = $238.05 + 6(19.00) = 352.05$ g/mol
    \end{itemize}
    \item Set up the Graham's Law ratio:
    \[ \frac{\text{rate}_{235}}{\text{rate}_{238}} = \sqrt{\frac{M_{238}}{M_{235}}} = \sqrt{\frac{352.05 \text{ g/mol}}{349.04 \text{ g/mol}}} \]
    \item Calculate the result:
    \[ \frac{\text{rate}_{235}}{\text{rate}_{238}} = \sqrt{1.0086} = 1.0043 \]
\end{enumerate}
\textbf{Answer:} \ch{^235UF6} effuses \textbf{1.0043 times faster} than \ch{^238UF6}.

\end{document}