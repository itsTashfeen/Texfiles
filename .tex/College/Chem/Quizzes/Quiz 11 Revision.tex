\documentclass{article}
\usepackage{amsmath}
\usepackage{amssymb}
\usepackage[margin=1in]{geometry}

\title{Homework 11: Gas Laws and Kinetic Molecular Theory}
\author{Tashfeen Omran}
\date{October 2025}

\begin{document}

\maketitle

\section{Quiz Details}
\begin{itemize}
    \item \textbf{When:} Tuesday, October 28th, 7:00 AM - 7:30 AM.
    \item \textbf{Format:} Closed notes.
    \item \textbf{Allowed Materials:} A scientific calculator. Phones are not permitted.
\end{itemize}

\section{Key Topics}
The quiz will focus on three main areas from Chapter 11. You must be able to:
\begin{enumerate}
    \item Use the empirical gas laws (Boyle's law, Charles' law, etc.) to solve for a missing variable (i.e., volume, pressure, or temperature).
    \item Use the ideal gas law to calculate volume, pressure, temperature, or moles of a gas.
    \item Read a velocity distribution diagram and answer questions about a molecule moving at different temperatures, based on the principles of the Kinetic Molecular Theory.
\end{enumerate}

\section{Phased Study Plan}

\subsection{Phase 1: Foundation (Friday/Saturday)}
\begin{itemize}
    \item \textbf{Core Concepts Review:} Carefully read through your lecture notes and the provided homework solutions for Chapter 11. Focus on understanding the relationships between pressure (P), volume (V), temperature (T), and moles (n).
    \item \textbf{Memorize Key Formulas:} You must know these formulas by heart.
        \begin{itemize}
            \item \textbf{Boyle's Law:} $P_1V_1 = P_2V_2$ (constant n, T)
            \item \textbf{Charles's Law:} $\frac{V_1}{T_1} = \frac{V_2}{T_2}$ (constant n, P)
            \item \textbf{Gay-Lussac's Law:} $\frac{P_1}{T_1} = \frac{P_2}{T_2}$ (constant n, V)
            \item \textbf{Combined Gas Law:} $\frac{P_1V_1}{T_1} = \frac{P_2V_2}{T_2}$ (constant n)
            \item \textbf{Ideal Gas Law:} $PV = nRT$ (where R = $0.08206 \frac{\text{L} \cdot \text{atm}}{\text{mol} \cdot \text{K}}$)
        \end{itemize}
    \item \textbf{Conceptual Understanding:}
        \begin{itemize}
            \item Understand that for all gas laws, temperature (\textbf{T}) \textbf{must} be in Kelvin (K). K = $^{\circ}$C + 273.15.
            \item Review the core principles of the Kinetic Molecular Theory (KMT), especially the direct relationship between average kinetic energy and absolute temperature (Kelvin).
        \end{itemize}
\end{itemize}

\subsection{Phase 2: Practice \& Application (Sunday)}
\begin{itemize}
    \item \textbf{Problem Drills:} Redo the problems selected in the "Explanations of Practice Problems" section below without looking at the solutions first. Focus on setting up the problem correctly, identifying the right formula, and ensuring all your units are consistent.
    \item \textbf{Unit Conversion Practice:} Pay close attention to converting pressures (e.g., mmHg to atm, where 760 mmHg = 1 atm) and temperatures (Celsius to Kelvin). This is a common source of errors.
\end{itemize}

\subsection{Phase 3: Final Mastery (Monday)}
\begin{itemize}
    \item \textbf{Final Review \& Prep:}
        \begin{itemize}
            \item Briefly review all key topics and formulas one last time.
            \item Look at the KMT graphs and explain to yourself what they mean before reading the explanation.
            \item Get a good night's sleep.
        \end{itemize}
\end{itemize}

\section{Explanations of Practice Problems}
Below are detailed walkthroughs of questions from your provided files that are relevant to the quiz topics.

\subsection{Topic 1: Empirical and Ideal Gas Laws}
These laws describe the relationships between P, V, T, and n. The most critical skill is identifying which variables are changing and which are held constant to select the correct formula.

\subsubsection{Key Principles}
\begin{itemize}
    \item \textbf{Empirical Laws (Boyle's, Charles', etc.):} Used when comparing two states (initial and final) of a gas where the amount of gas (moles) is constant.
    \item \textbf{Ideal Gas Law:} Used to describe a single state of a gas. It relates all four variables at one specific moment.
    \item \textbf{The Golden Rule of Temperature:} ALWAYS convert temperature to Kelvin. The relationships described by the gas laws only work with the absolute temperature scale.
\end{itemize}

\subsubsection{Example 1: Using the Combined Gas Law}
(from "Gas Law Problems", problem 5)

\textbf{Problem:} A 1.500 L balloon is filled with helium at 305.9 K and 745.9 mmHg. The balloon rises to an altitude where the pressure is 154.2 mmHg and the temperature is 250.6 K. What is the volume of the balloon at this altitude?

\textbf{Analysis:} The amount of helium gas (n) in the balloon is constant. Pressure, volume, and temperature are all changing. This requires the Combined Gas Law.

\textbf{Step 1: Identify variables.}
\begin{itemize}
    \item Initial State (1): $P_1 = 745.9 \text{ mmHg}$, $V_1 = 1.500 \text{ L}$, $T_1 = 305.9 \text{ K}$
    \item Final State (2): $P_2 = 154.2 \text{ mmHg}$, $V_2 = ?$, $T_2 = 250.6 \text{ K}$
\end{itemize}
Notice that the pressures are both in mmHg, so no conversion is necessary for them to cancel. Temperature is already in Kelvin.

\textbf{Step 2: State the formula.}
Since n is constant, the Combined Gas Law is:
\[ \frac{P_1V_1}{T_1} = \frac{P_2V_2}{T_2} \]

\textbf{Step 3: Rearrange the formula to solve for the unknown ($V_2$).}
\[ V_2 = \frac{P_1V_1T_2}{T_1P_2} \]

\textbf{Step 4: Substitute values and calculate.}
\[ V_2 = \frac{(745.9 \text{ mmHg})(1.500 \text{ L})(250.6 \text{ K})}{(305.9 \text{ K})(154.2 \text{ mmHg})} \]
\[ V_2 = \frac{280367.835}{47169.78} \text{ L} \approx 5.944 \text{ L} \]
The units mmHg and K cancel out, leaving L, which is correct for volume.

\textbf{Answer:} The final volume of the balloon is approximately \textbf{5.944 L}.

\subsubsection{Example 2: Using the Ideal Gas Law}
(from "Gas Law Problems", problem 7)

\textbf{Problem:} A 2.2854 g sample of propane (C$_3$H$_8$; 44.0972 g/mol) is placed in a 1.5763 L flask at 375.91 K. What is the pressure of this sample?

\textbf{Analysis:} This problem describes a single state of a gas and asks for one of the properties (pressure). This requires the Ideal Gas Law, $PV=nRT$.

\textbf{Step 1: Identify variables and convert units.}
\begin{itemize}
    \item $P = ?$
    \item $V = 1.5763 \text{ L}$
    \item $T = 375.91 \text{ K}$
    \item $R = 0.08206 \frac{\text{L} \cdot \text{atm}}{\text{mol} \cdot \text{K}}$
    \item We need moles (n). We are given mass, so we must convert mass to moles using the molar mass.
\end{itemize}
\[ n = 2.2854 \text{ g C}_3\text{H}_8 \times \frac{1 \text{ mol C}_3\text{H}_8}{44.0972 \text{ g C}_3\text{H}_8} \approx 0.051826 \text{ mol} \]

\textbf{Step 2: State the formula and rearrange for the unknown (P).}
\[ PV = nRT \implies P = \frac{nRT}{V} \]

\textbf{Step 3: Substitute values and calculate.}
\[ P = \frac{(0.051826 \text{ mol})(0.08206 \frac{\text{L} \cdot \text{atm}}{\text{mol} \cdot \text{K}})(375.91 \text{ K})}{1.5763 \text{ L}} \]
\[ P \approx 1.014 \text{ atm} \]
The units mol, L, and K cancel, leaving atm, which is correct for pressure.

\textbf{Answer:} The pressure of the sample is approximately \textbf{1.014 atm}.

\subsection{Topic 2: Kinetic Molecular Theory (KMT) and Velocity Diagrams}
KMT explains the macroscopic properties of gases based on the behavior of their molecules. For the quiz, focus on the relationship between temperature, molar mass, and molecular speed.

\subsubsection{Key Principles}
\begin{itemize}
    \item \textbf{Temperature and Kinetic Energy:} The average kinetic energy of gas molecules is directly proportional to the absolute temperature (in Kelvin). Higher T means higher average KE, which means faster moving molecules.
    \item \textbf{Molecular Speed Distribution:} At any given temperature, molecules in a gas sample move at a range of speeds. A velocity distribution curve shows the number of molecules moving at each speed.
    \item \textbf{Shape of the Curve:} As temperature increases, the curve flattens and spreads out. The peak (most probable speed) and the average speed both shift to the right (higher velocity).
    \item \textbf{Molar Mass and Speed:} At the same temperature, all gases have the same average kinetic energy. For this to be true, molecules with a smaller molar mass must move faster, and molecules with a larger molar mass must move slower.
\end{itemize}

\subsubsection{Example 3: Interpreting Velocity vs. Temperature}
(from "Kinetic molecular theory", problem 1)

\textbf{Problem:} The graph shows the velocity distribution for the same molecule at three different temperatures: 25$^{\circ}$C, 50$^{\circ}$C, and 75$^{\circ}$C.
\begin{itemize}
    \item a) Which curve represents the molecule at 50$^{\circ}$C?
    \item b) Which curve represents the molecule at 75$^{\circ}$C?
    \item c) Which curve represents the molecule at 25$^{\circ}$C?
\end{itemize}
\textbf{Analysis:} We need to relate the temperature to the shape and position of the curves.
\begin{itemize}
    \item \textbf{Lowest Temperature (25$^{\circ}$C):} Molecules move the slowest. This corresponds to the curve with its peak furthest to the left. The curve is also the tallest and narrowest, indicating less variation in speeds. This is \textbf{Curve A}.
    \item \textbf{Highest Temperature (75$^{\circ}$C):} Molecules move the fastest. This corresponds to the curve with its peak furthest to the right. The curve is also the flattest and widest, indicating a greater range of speeds. This is \textbf{Curve C}.
    \item \textbf{Intermediate Temperature (50$^{\circ}$C):} This will be the curve between the other two. This is \textbf{Curve B}.
\end{itemize}

\textbf{Answers:}
\begin{itemize}
    \item a) 50$^{\circ}$C is \textbf{Curve B}.
    \item b) 75$^{\circ}$C is \textbf{Curve C}.
    \item c) 25$^{\circ}$C is \textbf{Curve A}.
\end{itemize}

\end{document}