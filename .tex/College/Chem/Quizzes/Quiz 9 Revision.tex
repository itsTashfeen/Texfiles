\documentclass{article}

%========================================================================================
% PREAMBLE
% This section loads all necessary packages and defines the document's overall layout.
%========================================================================================
\usepackage[letterpaper, margin=1in, textwidth=6.5in]{geometry} % Manages page layout (margins, paper size).
\usepackage{amsmath}        % Provides advanced environments for equations (like align*).
\usepackage{amssymb}        % Provides additional mathematical symbols.
\usepackage{amsfonts}       % Provides mathematical fonts.
\usepackage{enumitem}       % Allows for customization of list environments (e.g., adding space between items).
\usepackage{chemformula}    % Simplifies writing chemical formulas (e.g., \ch{H2O}).
\usepackage{siunitx}        % Provides standardized formatting for numbers and units (e.g., \SI{1.00}{g/mol}).
\usepackage{hyperref}       % Creates hyperlinks within the document (useful for navigation in PDF viewers).
\hypersetup{
    colorlinks=true,
    linkcolor=blue,
    filecolor=magenta,      
    urlcolor=cyan,
}

% DOCUMENT METADATA
\title{Study Guide for Chemistry Quiz (Chapter 9 Topics)}
\author{Quiz Date: Tuesday, October 21st}
\date{}

\begin{document}

\maketitle % Generates the title.

This guide is tailored specifically for the four topics on your upcoming quiz. It includes key concepts, rules to memorize, and worked examples with references to your course materials.

%========================================================================================
% TOPIC 1: SOLUBILITY RULES
%========================================================================================
\section*{Topic 1: Using Solubility Rules}

\subsection*{Key Concept}
You must be able to use the provided solubility rules to determine if an ionic compound will dissolve in water (\textbf{soluble}) or form a solid precipitate (\textbf{insoluble}).

\begin{itemize}[itemsep=5pt]
    \item \textbf{Primary Study Source:} \texttt{Chapter 9 Chemical reactions in aqueous solutions.pdf}, \textbf{Slide 34 (TABLE 4.1)}.
    \item \textbf{Soluble compounds} dissociate completely into ions in water. They are strong electrolytes.
    \item \textbf{Insoluble compounds} do not dissolve in water. They are considered weak electrolytes because a negligible amount of ions are formed.
\end{itemize}

\subsection*{How to Apply the Rules}
\begin{enumerate}[itemsep=5pt]
    \item \textbf{Identify the ions} that make up the compound (one cation, one anion).
    \item \textbf{Check the "Generally Soluble" Section First:} Find one of the ions in this section.
    \begin{itemize}
        \item If the other ion in your compound is \textbf{NOT} listed as an exception, the compound is \textbf{SOLUBLE}.
        \item If the other ion \textbf{IS} listed as an exception, the compound is \textbf{INSOLUBLE}.
    \end{itemize}
    \item \textbf{Check the "Generally Insoluble" Section if Needed:} If your ion was not in the first section, find it here.
    \begin{itemize}
        \item If the other ion in your compound is \textbf{NOT} listed as an exception, the compound is \textbf{INSOLUBLE}.
        \item If the other ion \textbf{IS} listed as an exception, the compound is \textbf{SOLUBLE}.
    \end{itemize}
\end{enumerate}

\subsection*{Practice Problems}
\begin{itemize}[itemsep=5pt]
    \item \textbf{Practice Source:} \texttt{Questions and Problems}, Page 404, Problems \textbf{9.22 \& 9.23}.
    \item \textbf{Practice Source:} \texttt{Chapter 9 Chemical reactions in aqueous solutions.pdf}, Slides \textbf{36 \& 37}.
\end{itemize}
\begin{enumerate}[itemsep=5pt]
    \item \textbf{Is \ch{CaCO3} soluble or insoluble?}
    \begin{itemize}
        \item \textbf{Ions:} \ch{Ca^2+} and \ch{CO3^2-}.
        \item \textbf{Rule:} The \ch{CO3^2-} ion is in the "Generally Insoluble" category. \ch{Ca^2+} is not listed as an exception.
        \item \textbf{Answer:} Therefore, \ch{CaCO3} is \textbf{insoluble}.
    \end{itemize}
    \item \textbf{Is \ch{K2S} soluble or insoluble?}
    \begin{itemize}
        \item \textbf{Ions:} \ch{K+} and \ch{S^2-}.
        \item \textbf{Rule:} The \ch{S^2-} ion is "Generally Insoluble". However, \ch{K+} is an alkali metal ion and is listed as an exception.
        \item \textbf{Answer:} Therefore, \ch{K2S} is \textbf{soluble}.
    \end{itemize}
    \item \textbf{Is \ch{PbCl2} soluble or insoluble?}
    \begin{itemize}
        \item \textbf{Ions:} \ch{Pb^2+} and \ch{Cl-}.
        \item \textbf{Rule:} The \ch{Cl-} ion is "Generally Soluble". However, \ch{Pb^2+} is listed as an exception.
        \item \textbf{Answer:} Therefore, \ch{PbCl2} is \textbf{insoluble}.
    \end{itemize}
\end{enumerate}

\newpage

%========================================================================================
% TOPIC 2: pH CALCULATIONS
%========================================================================================
\section*{Topic 2: pH Calculations}

\subsection*{Key Concepts and Formulas}
You must be able to calculate pH from [\ch{H3O+}] and determine if the solution is acidic, basic, or neutral, while applying the correct significant figure rules.
\begin{itemize}[itemsep=5pt]
    \item \textbf{Primary Study Source:} \texttt{Chapter 9 Chemical reactions in aqueous solutions.pdf}, \textbf{Slides 16-19}.
    \item \textbf{Formula to find pH:} \[ \text{pH} = -\log[\ch{H3O+}] \]
    \item \textbf{pH Scale:} At \SI{25}{\celsius}, a pH $<$ 7 is \textbf{acidic}, pH = 7 is \textbf{neutral}, and pH $>$ 7 is \textbf{basic}.
\end{itemize}

\subsection*{Crucial Rule for Significant Figures}
\begin{itemize}[itemsep=5pt]
    \item The number of \textbf{significant figures} in your [\ch{H3O+}] concentration value determines the number of \textbf{decimal places} in your final pH answer.
\end{itemize}

\subsection*{Practice Problems}
\begin{itemize}[itemsep=5pt]
    \item \textbf{Practice Source:} \texttt{Chapter 9 Chemical reactions in aqueous solutions.pdf}, \textbf{Slide 19}.
    \item \textbf{Practice Source:} \texttt{Questions and Problems}, Page 407, Problems \textbf{9.80 \& 9.81}.
\end{itemize}
\begin{enumerate}[itemsep=5pt]
    \item \textbf{Calculate the pH for [\ch{H3O+}] = \SI{4.3e-8}{M} and classify the solution.}
    \begin{itemize}
        \item \textbf{Calculation:} \( \text{pH} = -\log(4.3 \times 10^{-8}) \)
        \item \textbf{Sig Figs:} The concentration \num{4.3e-8} has \textbf{two} significant figures. Therefore, the pH value must have \textbf{two} decimal places.
        \item \textbf{Answer:} pH = \textbf{7.37}. Since 7.37 > 7, the solution is \textbf{basic}.
    \end{itemize}
    \item \textbf{Calculate the pH for [\ch{H3O+}] = \SI{8.82e-4}{M} and classify the solution.}
    \begin{itemize}
        \item \textbf{Calculation:} \( \text{pH} = -\log(8.82 \times 10^{-4}) \)
        \item \textbf{Sig Figs:} The concentration \num{8.82e-4} has \textbf{three} significant figures. Therefore, the pH value must have \textbf{three} decimal places.
        \item \textbf{Answer:} pH = \textbf{3.054}. Since 3.054 < 7, the solution is \textbf{acidic}.
    \end{itemize}
\end{enumerate}

%========================================================================================
% TOPIC 3: OXIDATION NUMBERS
%========================================================================================
\section*{Topic 3: Assigning Oxidation Numbers}

\subsection*{Key Concept}
You must memorize and apply the seven rules for assigning oxidation numbers to an atom in a compound or ion.

\subsection*{The Seven Rules (Memorize These)}
\begin{itemize}[itemsep=5pt]
    \item \textbf{Primary Study Source:} \texttt{Chapter 9 Chemical reactions in aqueous solutions.pdf}, \textbf{Slides 48 \& 49}.
\end{itemize}
\begin{enumerate}[label=\textbf{Rule \arabic*:}, itemsep=5pt]
    \item \textbf{Elements:} The oxidation number of an atom in its elemental form is \textbf{0} (e.g., \ch{Na}, \ch{O2}, \ch{S8}).
    \item \textbf{Monatomic Ions:} The oxidation number equals the \textbf{charge} of the ion (e.g., \ch{Na+} is +1; \ch{Cl-} is -1).
    \item \textbf{Oxygen:} The oxidation number is usually \textbf{-2}. (Exception: -1 in peroxides like \ch{H2O2}).
    \item \textbf{Hydrogen:} The oxidation number is \textbf{+1} with nonmetals and \textbf{-1} with metals.
    \item \textbf{Halogens:} The oxidation number is usually \textbf{-1}. (Exception: positive when bonded to oxygen or a more electronegative halogen). Fluorine is \textbf{always -1}.
    \item \textbf{Neutral Compounds:} The \textbf{sum} of all oxidation numbers must equal \textbf{zero}.
    \item \textbf{Polyatomic Ions:} The \textbf{sum} of all oxidation numbers must equal the \textbf{charge of the ion}.
\end{enumerate}

\subsection*{Practice Problems}
\begin{itemize}[itemsep=5pt]
    \item \textbf{Practice Source:} \texttt{Oxidation states and redox reactions.pdf} (worksheet).
    \item \textbf{Practice Source:} \texttt{Chapter 9 Chemical reactions in aqueous solutions.pdf}, \textbf{Slides 50-54}.
    \item \textbf{Practice Source:} \texttt{Questions and Problems}, Page 405, Problems \textbf{9.49 - 9.51}.
\end{itemize}
\begin{enumerate}[itemsep=5pt]
    \item \textbf{Determine the oxidation number for Mn in \ch{KMnO4}.}
    \begin{itemize}
        \item \textbf{Rules:} K is +1 (Group 1 ion). O is -2 (Rule 3). The sum must be 0 (Rule 6). Let Mn = x.
        \item \textbf{Equation:} \( (+1) + (x) + 4(-2) = 0 \)
        \item \textbf{Solve:} \( 1 + x - 8 = 0 \implies x = +7 \). The oxidation number of Mn is \textbf{+7}.
    \end{itemize}
    \item \textbf{Determine the oxidation number for N in \ch{NO3-}.}
    \begin{itemize}
        \item \textbf{Rules:} O is -2 (Rule 3). The sum must be -1 (Rule 7). Let N = x.
        \item \textbf{Equation:} \( (x) + 3(-2) = -1 \)
        \item \textbf{Solve:} \( x - 6 = -1 \implies x = +5 \). The oxidation number of N is \textbf{+5}.
    \end{itemize}
\end{enumerate}

\newpage

%========================================================================================
% TOPIC 4: BALANCING REDOX REACTIONS
%========================================================================================
\section*{Topic 4: Balancing Redox Reactions}

\subsection*{Key Concepts}
A redox reaction involves the transfer of electrons, identified by changes in oxidation numbers. You must be able to balance the atoms and the charge.
\begin{itemize}[itemsep=5pt]
    \item \textbf{Primary Study Source:} \texttt{Chapter 9 Chemical reactions in aqueous solutions.pdf}, \textbf{Slides 58-61}.
    \item \textbf{Oxidation:} Loss of electrons (oxidation number increases).
    \item \textbf{Reduction:} Gain of electrons (oxidation number decreases).
\end{itemize}

\subsection*{The Half-Reaction Method}
\begin{enumerate}[label=\textbf{Step \arabic*:}, itemsep=5pt]
    \item \textbf{Split into Half-Reactions:} Write separate equations for the oxidation and reduction processes.
    \item \textbf{Balance Atoms:} Balance all atoms other than O and H.
    \item \textbf{Balance Charge with Electrons:} Add electrons (\ch{e-}) to the more positive side of each equation to make the charges equal on both sides.
    \item \textbf{Equalize Electrons:} Multiply the half-reactions by integers to make the number of electrons lost in oxidation equal the number of electrons gained in reduction.
    \item \textbf{Combine:} Add the balanced half-reactions and cancel the electrons.
\end{enumerate}

\subsection*{Practice Problem}
\begin{itemize}[itemsep=5pt]
    \item \textbf{Practice Source:} \texttt{Oxidation states and redox reactions.pdf}, **Part 3**.
    \item \textbf{Practice Source:} \texttt{Chapter 9 Chemical reactions in aqueous solutions.pdf}, **Slides 59 \& 61}.
\end{itemize}
\textbf{Balance the following redox reaction: \ch{Sn(s) + H+(aq) -> Sn^2+(aq) + H2(g)}}
\begin{itemize}[itemsep=5pt]
    \item \textbf{Step 1: Split into Half-Reactions}
    \begin{align*}
        \text{Oxidation:} & \quad \ch{Sn(s) -> Sn^2+(aq)} \\
        \text{Reduction:} & \quad \ch{H+(aq) -> H2(g)}
    \end{align*}
    \item \textbf{Step 2: Balance Atoms}
    \begin{align*}
        \text{Oxidation:} & \quad \ch{Sn(s) -> Sn^2+(aq)} \quad (\text{Sn is balanced}) \\
        \text{Reduction:} & \quad \ch{2H+(aq) -> H2(g)} \quad (\text{H is now balanced})
    \end{align*}
    \item \textbf{Step 3: Balance Charge with Electrons}
    \begin{align*}
        \text{Oxidation:} & \quad \ch{Sn(s) -> Sn^2+(aq) + 2e-} \quad (\text{Charge is 0 on both sides}) \\
        \text{Reduction:} & \quad \ch{2H+(aq) + 2e- -> H2(g)} \quad (\text{Charge is 0 on both sides})
    \end{align*}
    \item \textbf{Step 4: Equalize Electrons}
        \begin{itemize}
            \item The oxidation half-reaction loses 2 electrons.
            \item The reduction half-reaction gains 2 electrons.
            \item The electrons are already balanced (2 = 2). No multiplication is needed.
        \end{itemize}
    \item \textbf{Step 5: Combine}
        \begin{itemize}
            \item Add the two equations:
            \[ \ch{Sn(s) + 2H+(aq) + 2e- -> Sn^2+(aq) + 2e- + H2(g)} \]
            \item Cancel the electrons (\ch{2e-}) from both sides.
            \item \textbf{Final Balanced Equation:}
            \[ \ch{Sn(s) + 2H+(aq) -> Sn^2+(aq) + H2(g)} \]
        \end{itemize}
\end{itemize}

\end{document}