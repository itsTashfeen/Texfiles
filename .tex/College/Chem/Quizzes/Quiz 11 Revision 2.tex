\documentclass{article}
\usepackage{amsmath}
\usepackage{amssymb}
\usepackage{geometry}
\usepackage{chemformula}
\geometry{margin=1in}

\title{Homework 11: Gas Laws and Kinetic Molecular Theory}
\author{Tashfeen Omran}
\date{October 2025}

\begin{document}

\maketitle

\section{Quiz Details}
\begin{itemize}
    \item \textbf{When:} Tuesday, October 28th, 7:00 AM to 7:30 AM
    \item \textbf{Format:} Closed notes
    \item \textbf{Allowed Materials:} A scientific calculator. Phones are not permitted.
\end{itemize}

\section{Key Topics}
\begin{enumerate}
    \item The Empirical Gas Laws (Boyle's, Charles's, Gay-Lussac's)
    \item The Ideal Gas Law and Molar Mass Calculation
    \item Kinetic Molecular Theory and Graham's Law of Effusion
\end{enumerate}

\section{Phased Study Plan}
\subsection{Phase 1: Foundation (Friday/Saturday)}
\begin{itemize}
    \item \textbf{Core Concepts Review:} Read through \texttt{lecture\_slides.pdf}, focusing on slides 7-23 and 45-50. These slides introduce the properties of gases, the individual gas laws, the ideal gas law, and the kinetic molecular theory.
    \item \textbf{Memorize Key Formulas:}
        \begin{itemize}
            \item \textbf{Boyle's Law:} $P_1V_1 = P_2V_2$ (constant $n, T$)
            \item \textbf{Charles's Law:} $\frac{V_1}{T_1} = \frac{V_2}{T_2}$ (constant $n, P$)
            \item \textbf{Gay-Lussac's Law:} $\frac{P_1}{T_1} = \frac{P_2}{T_2}$ (constant $n, V$)
            \item \textbf{Combined Gas Law:} $\frac{P_1V_1}{T_1} = \frac{P_2V_2}{T_2}$ (constant $n$)
            \item \textbf{Ideal Gas Law:} $PV = nRT$
            \begin{itemize}
                \item $P$: Pressure (atm)
                \item $V$: Volume (L)
                \item $n$: Moles
                \item $R$: Ideal Gas Constant ($0.08206 \frac{\text{L} \cdot \text{atm}}{\text{mol} \cdot \text{K}}$)
                \item $T$: Temperature (K)
            \end{itemize}
            \item \textbf{Molar Mass from Ideal Gas Law:} $\mathcal{M} = \frac{dRT}{P}$ where $d$ is density.
            \item \textbf{Graham's Law of Effusion:} $\frac{\text{rate}_A}{\text{rate}_B} = \sqrt{\frac{\mathcal{M}_B}{\mathcal{M}_A}}$
        \end{itemize}
    \item \textbf{Conceptual Understanding:}
        \begin{itemize}
            \item Temperature must \textbf{always} be in Kelvin (K) for gas law calculations ($K = \text{°C} + 273.15$).
            \item Standard Temperature and Pressure (STP) is defined as exactly $273.15$ K ($0$°C) and $1$ atm.
            \item According to Kinetic Molecular Theory, gas speed is directly proportional to temperature and inversely proportional to molar mass. Lighter gases move faster at the same temperature.
        \end{itemize}
\end{itemize}

\subsection{Phase 2: Practice \& Application (Sunday)}
\begin{itemize}
    \item \textbf{Problem Drills:} Redo the problems from the \texttt{homework\_solutions.pdf} files. Specifically, focus on "Gas Law Problems" \#1-6, "Molar mass, density, gas stoichiometry, gas mixtures" \#1, 4, and "Kinetic molecular theory" \#1-5.
    \item \textbf{Unit Conversion Practice:} Be very comfortable with the following conversions:
        \begin{itemize}
            \item Pressure: mmHg to atm ($760 \text{ mmHg} = 1 \text{ atm}$), torr to atm ($760 \text{ torr} = 1 \text{ atm}$)
            \item Temperature: Celsius to Kelvin ($K = \text{°C} + 273.15$)
            \item Mass to Moles: grams $\leftrightarrow$ moles using molar mass.
        \end{itemize}
\end{itemize}

\subsection{Phase 3: Final Mastery (Monday)}
\begin{itemize}
    \item \textbf{Final Review \& Prep:} Quickly review all the key formulas one last time. Work through one or two problems from each key topic to ensure you have the process down. Get a good night's sleep.
\end{itemize}

\section{Explanations of Practice Problems}

\subsection{Topic 1: The Empirical Gas Laws}
\subsubsection{Key Principles}
\begin{itemize}
    \item The empirical gas laws relate two properties (like pressure and volume) while holding the other two (moles and temperature) constant.
    \item Boyle's Law ($P_1V_1 = P_2V_2$) shows an \textit{inverse} relationship between pressure and volume. As pressure increases, volume decreases.
    \item Charles's Law ($\frac{V_1}{T_1} = \frac{V_2}{T_2}$) and Gay-Lussac's Law ($\frac{P_1}{T_1} = \frac{P_2}{T_2}$) show a \textit{direct} relationship with temperature. As temperature increases, volume and pressure also increase.
    \item The Combined Gas Law ($\frac{P_1V_1}{T_1} = \frac{P_2V_2}{T_2}$) is used when both temperature, pressure, and volume change, but the amount of gas (moles) is constant.
\end{itemize}

\subsubsection{Example 1 (Gay-Lussac's Law)}
\textbf{Problem:} A gas sample in a sealed flask has a pressure of 792.84 torr at a temperature of 21.26°C. Assuming the volume of the flask cannot change, what is the pressure if the flask is heated to 75.35°C?

\textbf{Source:} homework\_solutions.pdf, Gas Law Problems, Page 1, Problem 1

\textbf{Analysis:} This is a Gay-Lussac's Law problem because volume and moles are constant, while pressure and temperature are changing. The strategy is to convert both temperatures to Kelvin and then use the formula $\frac{P_1}{T_1} = \frac{P_2}{T_2}$ to solve for the final pressure, $P_2$.

\textbf{Step-by-Step Solution:}
\begin{enumerate}
    \item Identify initial conditions: $P_1 = 792.84 \text{ torr}$, $T_1 = 21.26 \text{ °C}$.
    \item Identify final conditions: $P_2 = ?$, $T_2 = 75.35 \text{ °C}$.
    \item Convert temperatures to Kelvin:
    \[ T_1 = 21.26 + 273.15 = 294.41 \text{ K} \]
    \[ T_2 = 75.35 + 273.15 = 348.50 \text{ K} \]
    \item Rearrange Gay-Lussac's Law to solve for $P_2$:
    \[ P_2 = \frac{P_1 T_2}{T_1} \]
    \item Substitute the values and solve:
    \[ P_2 = \frac{(792.84 \text{ torr})(348.50 \text{ K})}{294.41 \text{ K}} = 938.50 \text{ torr} \]
\end{enumerate}
\textbf{Answer:} \textbf{938.50 torr}

\subsubsection{Example 2 (Boyle's Law)}
\textbf{Problem:} A gas sample in a syringe has a volume of 35.26 mL at 0.9854 atm. The plunger of the syringe is depressed such that the volume of the syringe is reduced to 22.33 mL. What is the new pressure in the syringe, assuming there has been no change in the temperature?

\textbf{Source:} homework\_solutions.pdf, Gas Law Problems, Page 1, Problem 2

\textbf{Analysis:} This is a Boyle's Law problem as temperature and the amount of gas are constant. The volume is decreasing, so we expect the pressure to increase. We will use the formula $P_1V_1 = P_2V_2$ to solve for the final pressure, $P_2$.

\textbf{Step-by-Step Solution:}
\begin{itemize}
    \item Initial Conditions: $P_1 = 0.9854 \text{ atm}$, $V_1 = 35.26 \text{ mL}$.
    \item Final Conditions: $P_2 = ?$, $V_2 = 22.33 \text{ mL}$.
    \item Rearrange Boyle's Law to solve for $P_2$:
    \[ P_2 = \frac{P_1 V_1}{V_2} \]
    \item Substitute the values and calculate:
    \[ P_2 = \frac{(0.9854 \text{ atm})(35.26 \text{ mL})}{22.33 \text{ mL}} = 1.556 \text{ atm} \]
\end{itemize}
\textbf{Answer:} \textbf{1.556 atm}

\subsubsection{Example 3 (Combined Gas Law)}
\textbf{Problem:} A 1.500 L balloon is filled with helium at 305.9 K and 745.9 mmHg. The balloon is released and rises to an altitude where the pressure is 154.2 mmHg and the temperature is 250.6 K. What is the volume of the balloon at this altitude?

\textbf{Source:} homework\_solutions.pdf, Gas Law Problems, Page 2, Problem 5

\textbf{Analysis:} This is a combined gas law problem because pressure, volume, and temperature are all changing, while the amount of helium gas inside the balloon remains constant. We will use the equation $\frac{P_1V_1}{T_1} = \frac{P_2V_2}{T_2}$ and solve for the final volume, $V_2$. The pressure units are consistent (mmHg), so no conversion is needed.

\textbf{Step-by-Step Solution:}
\begin{enumerate}
    \item Initial Conditions: $P_1 = 745.9 \text{ mmHg}$, $V_1 = 1.500 \text{ L}$, $T_1 = 305.9 \text{ K}$.
    \item Final Conditions: $P_2 = 154.2 \text{ mmHg}$, $V_2 = ?$, $T_2 = 250.6 \text{ K}$.
    \item Rearrange the Combined Gas Law to solve for $V_2$:
    \[ V_2 = \frac{P_1 V_1 T_2}{T_1 P_2} \]
    \item Substitute the values and solve:
    \[ V_2 = \frac{(745.9 \text{ mmHg})(1.500 \text{ L})(250.6 \text{ K})}{(305.9 \text{ K})(154.2 \text{ mmHg})} = 5.944 \text{ L} \]
\end{enumerate}
\textbf{Answer:} \textbf{5.944 L}

\subsection{Topic 2: The Ideal Gas Law}
\subsubsection{Key Principles}
\begin{itemize}
    \item The Ideal Gas Law ($PV=nRT$) relates all four properties of a gas (Pressure, Volume, moles, Temperature) at a single point in time. It is the most versatile gas law.
    \item It can be rearranged to solve for any of the four variables if the others are known.
    \item A common application is to find the molar mass ($\mathcal{M}$) of a gas. By combining the ideal gas law with the definitions of moles ($n = m/\mathcal{M}$) and density ($d = m/V$), we get the equation $\mathcal{M} = \frac{dRT}{P}$.
    \item For stoichiometry problems involving gases, the ideal gas law is used to convert between the volume of a gas and moles.
\end{itemize}

\subsubsection{Example 1 (Solving for Molar Mass)}
\textbf{Problem:} A 1.16 g sample of an unknown compound is placed in a 0.250 L flask and heated to 563 K. Upon complete vaporization of the sample, the pressure is 1.03 atm. What is the molar mass of the compound?

\textbf{Source:} homework\_solutions.pdf, Gas Law Problems, Page 4, Problem 10

\textbf{Analysis:} This is a molar mass determination problem. The strategy involves two main steps. First, use the Ideal Gas Law, $PV=nRT$, to solve for the number of moles ($n$) of the gas. Second, use the definition of molar mass, $\mathcal{M} = \frac{\text{mass}}{\text{moles}}$, to calculate the final answer.

\textbf{Step-by-Step Solution:}
\begin{enumerate}
    \item List the known variables: $P = 1.03 \text{ atm}$, $V = 0.250 \text{ L}$, $T = 563 \text{ K}$, and mass = $1.16 \text{ g}$. The gas constant $R = 0.08206 \frac{\text{L} \cdot \text{atm}}{\text{mol} \cdot \text{K}}$.
    \item Rearrange the Ideal Gas Law to solve for moles ($n$):
    \[ n = \frac{PV}{RT} \]
    \item Substitute the values to find the number of moles:
    \[ n = \frac{(1.03 \text{ atm})(0.250 \text{ L})}{(0.08206 \frac{\text{L} \cdot \text{atm}}{\text{mol} \cdot \text{K}})(563 \text{ K})} = 0.005574 \text{ mol} \]
    \item Use the mass and the calculated moles to find the molar mass ($\mathcal{M}$):
    \[ \mathcal{M} = \frac{\text{mass}}{n} = \frac{1.16 \text{ g}}{0.005574 \text{ mol}} = 208 \text{ g/mol} \]
\end{enumerate}
\textbf{Answer:} \textbf{208 g/mol}

\subsubsection{Example 2 (Solving for Molar Mass from Density)}
\textbf{Problem:} Determine the molar mass of an unknown gas in a container at -50.0°C and 6.00 atm pressure. The density of this gas is 14.5 g/L.

\textbf{Source:} homework\_solutions.pdf, Molar mass, density..., Page 1, Problem 1

\textbf{Analysis:} This problem directly provides density, pressure, and temperature, making it ideal for the molar mass equation derived from the ideal gas law: $\mathcal{M} = \frac{dRT}{P}$. The first step is to convert the temperature from Celsius to Kelvin.

\textbf{Step-by-Step Solution:}
\begin{itemize}
    \item List the knowns: $d = 14.5 \text{ g/L}$, $P = 6.00 \text{ atm}$, $T = -50.0 \text{ °C}$.
    \item Convert temperature to Kelvin:
    \[ T = -50.0 + 273.15 = 223.15 \text{ K} \]
    \item Substitute the values into the molar mass equation:
    \[ \mathcal{M} = \frac{(14.5 \frac{\text{g}}{\text{L}})(0.08206 \frac{\text{L} \cdot \text{atm}}{\text{mol} \cdot \text{K}})(223.15 \text{ K})}{6.00 \text{ atm}} \]
    \item Calculate the final result:
    \[ \mathcal{M} = 44.3 \text{ g/mol} \]
\end{itemize}
\textbf{Answer:} \textbf{44.3 g/mol}

\subsection{Topic 3: Kinetic Molecular Theory and Effusion}
\subsubsection{Key Principles}
\begin{itemize}
    \item The Kinetic Molecular Theory (KMT) describes the behavior of gas particles. Key ideas are that particles are in constant, random motion, and their average kinetic energy is proportional to the absolute temperature in Kelvin.
    \item A velocity distribution curve shows the number of molecules moving at different speeds. As temperature increases, the curve flattens and spreads out to the right, meaning the average speed increases and the range of speeds becomes wider.
    \item For different gases at the same temperature, lighter gases (lower molar mass) will have a faster average speed. Their distribution curve will be flatter and more spread out than that of a heavier gas.
    \item Graham's Law of Effusion states that the rate of effusion of a gas is inversely proportional to the square root of its molar mass ($\text{rate} \propto 1/\sqrt{\mathcal{M}}$). This means lighter gases effuse faster.
\end{itemize}

\subsubsection{Example 1 (Velocity Distribution vs. Temperature)}
\textbf{Problem:} The velocity distribution below details the same molecule moving at 25°C, 50°C, and 75°C. Which curve represents the molecule at 75°C? Which curve represents the molecule moving at the highest speed?

\textbf{Source:} homework\_solutions.pdf, Kinetic molecular theory, Page 1, Problem 1

\textbf{Analysis:} According to the Kinetic Molecular Theory, as the temperature of a gas increases, the average kinetic energy and average speed of its molecules increase. On a velocity distribution graph, this is shown by the peak of the curve shifting to the right (higher velocity) and the curve becoming flatter and wider. The highest temperature will correspond to the curve that is shifted furthest to the right.

\textbf{Step-by-Step Solution:}
\begin{enumerate}
    \item Identify the temperatures: 25°C, 50°C, 75°C. The highest temperature is 75°C.
    \item Examine the curves A, B, and C. Curve A has the lowest average speed (peak is furthest left). Curve C has the highest average speed (peak is furthest right).
    \item Match the temperatures to the curves. The lowest temperature (25°C) corresponds to the slowest curve (A). The highest temperature (75°C) corresponds to the fastest curve (C). The intermediate temperature (50°C) corresponds to the middle curve (B).
    \item The molecule moving at the highest speed is the one at the highest temperature.
\end{enumerate}
\textbf{Answer:} Curve \textbf{C} represents the molecule at 75°C. Curve \textbf{C} also represents the molecule moving at the highest speed.

\subsubsection{Example 2 (Graham's Law)}
\textbf{Problem:} An unknown gas diffuses 1.12 times faster than argon (\ch{Ar}) through a porous membrane. Calculate the molar mass of the unknown gas.

\textbf{Source:} homework\_solutions.pdf, Kinetic molecular theory, Page 3, Problem 4

\textbf{Analysis:} This is a Graham's Law problem. We are given the ratio of the rates and need to find the molar mass of the unknown gas (let's call it X). The strategy is to use the equation $\frac{\text{rate}_X}{\text{rate}_{Ar}} = \sqrt{\frac{\mathcal{M}_{Ar}}{\mathcal{M}_X}}$. We will look up the molar mass of Argon and then solve for $\mathcal{M}_X$.

\textbf{Step-by-Step Solution:}
\begin{enumerate}
    \item Write down the known information. The molar mass of Argon (\ch{Ar}) is $39.95$ g/mol. The ratio of the rates is $\frac{\text{rate}_X}{\text{rate}_{Ar}} = 1.12$.
    \item Set up Graham's Law equation:
    \[ 1.12 = \sqrt{\frac{39.95 \text{ g/mol}}{\mathcal{M}_X}} \]
    \item To solve for $\mathcal{M}_X$, we first square both sides of the equation:
    \[ (1.12)^2 = \frac{39.95 \text{ g/mol}}{\mathcal{M}_X} \]
    \[ 1.2544 = \frac{39.95 \text{ g/mol}}{\mathcal{M}_X} \]
    \item Rearrange the equation to solve for $\mathcal{M}_X$:
    \[ \mathcal{M}_X = \frac{39.95 \text{ g/mol}}{1.2544} = 31.8 \text{ g/mol} \]
\end{enumerate}
\textbf{Answer:} \textbf{31.8 g/mol}

\end{document}