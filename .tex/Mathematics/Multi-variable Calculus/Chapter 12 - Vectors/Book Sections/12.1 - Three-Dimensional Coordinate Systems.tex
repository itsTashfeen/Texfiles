\documentclass[11pt]{article}
\usepackage[utf8]{inputenc}
\usepackage[T1]{fontenc}
\usepackage{amsmath}
\usepackage{amssymb}
\usepackage{geometry}
\usepackage{graphicx}
\usepackage{fancyhdr}
\usepackage{enumitem}
\usepackage{tikz}

% Geometry setup
\geometry{a4paper, margin=1in}

% Header setup
\pagestyle{fancy}
\fancyhf{}
\rhead{Tashfeen Omran}
\lhead{Homework Chapter 12.1: 3D Coordinate Systems}
\rfoot{Page \thepage}

\title{Homework Chapter 12.1: Three-Dimensional Coordinate Systems}
\author{Tashfeen Omran}
\date{January 2026}

\begin{document}

\maketitle
\tableofcontents
\newpage

\section{Part 1: Introduction, Context, and Prerequisites}

\subsection{Core Concepts: The Geometry of Space}
In single-variable calculus, we operated on the 2D Cartesian plane ($\mathbb{R}^2$), defined by two perpendicular axes ($x$ and $y$). To study the calculus of space (curves, surfaces, and volumes), we extend this to three dimensions ($\mathbb{R}^3$).

\begin{itemize}
    \item \textbf{Coordinate Axes:} We introduce a third axis, the $z$-axis, perpendicular to both $x$ and $y$. The orientation is determined by the \textbf{Right-Hand Rule}: if you curl the fingers of your right hand from the positive $x$-axis to the positive $y$-axis, your thumb points in the direction of the positive $z$-axis.
    \item \textbf{Coordinate Planes:} The axes form three planes:
    \begin{itemize}
        \item The $xy$-plane ($z=0$, the "floor").
        \item The $xz$-plane ($y=0$, the "left wall").
        \item The $yz$-plane ($x=0$, the "right wall").
    \end{itemize}
    \item \textbf{Coordinates:} A point $P$ is represented as an ordered triple $(x, y, z)$.
    \item \textbf{Octants:} These planes divide space into eight regions called octants. The \textit{first octant} is where $x, y,$ and $z$ are all positive.
\end{itemize}

\subsection{Intuition and Derivation}
The fundamental formulas in 3D are logical extensions of 2D geometry using the Pythagorean Theorem.
\begin{itemize}
    \item \textbf{Distance Formula:} To find the distance between points $P_1(x_1, y_1, z_1)$ and $P_2(x_2, y_2, z_2)$, we imagine a rectangular box with these points as opposite corners. Applying Pythagoras twice (once for the base diagonal, once for the vertical height) yields the formula:
    \[ |P_1P_2| = \sqrt{(x_2-x_1)^2 + (y_2-y_1)^2 + (z_2-z_1)^2} \]
    \item \textbf{Spheres:} A sphere is the set of all points distance $r$ from a center $(h,k,l)$. By squaring the distance formula, we get the standard equation of a sphere.
\end{itemize}

\subsection{Historical Context}
The development of Analytic Geometry in three dimensions is largely credited to René Descartes and Pierre de Fermat in the 17th century. Their insight was that geometric position could be encoded as numbers. This allowed problems regarding shape, intersection, and volume—previously solved via complex Greek geometric arguments—to be solved algebraically. This unification is the prerequisite for Newton and Leibniz to develop Calculus.

\subsection{Key Formulas}
\begin{enumerate}
    \item \textbf{Distance Formula:} $d = \sqrt{(x_2-x_1)^2 + (y_2-y_1)^2 + (z_2-z_1)^2}$
    \item \textbf{Equation of a Sphere:} $(x-h)^2 + (y-k)^2 + (z-l)^2 = r^2$
    \item \textbf{Midpoint Formula:} $M = \left( \frac{x_1+x_2}{2}, \frac{y_1+y_2}{2}, \frac{z_1+z_2}{2} \right)$
\end{enumerate}

\subsection{Prerequisites}
\begin{itemize}
    \item \textbf{Completing the Square:} Essential for converting expanded quadratic equations (e.g., $x^2 + 4x + ...$) into standard circle/sphere form $(x+2)^2$.
    \item \textbf{Inequalities:} Understanding how to represent regions (e.g., the interior of a circle is $x^2+y^2 < r^2$).
    \item \textbf{Basic Trigonometry:} Used for geometric visualization.
\end{itemize}

\newpage

\section{Part 2: Detailed Homework Solutions}

\subsection*{Problem 1}
\textbf{Question:} Suppose you start at the origin, move along the x-axis a distance of 4 units in the positive direction, and then move downward a distance of 3 units. What are the coordinates of your position?

\textbf{Solution:}
1. Start at Origin: $(0, 0, 0)$.
2. Move along positive x-axis by 4: Add 4 to the x-coordinate. New position: $(4, 0, 0)$.
3. Move downward by 3: "Downward" implies the negative z-direction. Subtract 3 from the z-coordinate. New position: $(4, 0, -3)$.

\textbf{Final Answer:} $(4, 0, -3)$

\subsection*{Problem 2}
\textbf{Question:} Sketch the points $(1, 5, 3), (0, 2, -3), (-3, 0, 2),$ and $(2, -2, -1)$.

\textbf{Solution:}
\textit{(Descriptive Sketching Steps)}
\begin{itemize}
    \item \textbf{(1, 5, 3):} Start at origin. Move 1 unit forward (x), 5 units right (y), 3 units up (z). Point is in the first octant.
    \item \textbf{(0, 2, -3):} Start at origin. Do not move in x. Move 2 units right (y). Move 3 units down (z). Point is in the $yz$-plane (lower half).
    \item \textbf{(-3, 0, 2):} Start at origin. Move 3 units backward (negative x). Do not move in y. Move 2 units up (z). Point is in the $xz$-plane.
    \item \textbf{(2, -2, -1):} Start at origin. Move 2 units forward (x). Move 2 units left (negative y). Move 1 unit down (z). Point is in the bottom-front-left octant.
\end{itemize}

\subsection*{Problem 3}
\textbf{Question:} Which of the points $A(-4, 0, -1), B(3, 1, -5),$ and $C(2, 4, 6)$ is closest to the yz-plane? Which point lies in the xz-plane?

\textbf{Solution:}
1. \textbf{Closest to yz-plane:} The distance from a point $(x,y,z)$ to the yz-plane is given by the absolute value of the x-coordinate, $|x|$.
   \begin{itemize}
       \item Distance for A: $|-4| = 4$.
       \item Distance for B: $|3| = 3$.
       \item Distance for C: $|2| = 2$.
   \end{itemize}
   Point C has the smallest distance (2).

2. \textbf{Lies in the xz-plane:} A point lies in the xz-plane if its y-coordinate is 0.
   \begin{itemize}
       \item y-coordinate of A is 0.
       \item y-coordinate of B is 1.
       \item y-coordinate of C is 4.
   \end{itemize}
   Point A lies in the xz-plane.

\textbf{Final Answer:} C is closest to the yz-plane. A lies in the xz-plane.

\subsection*{Problem 4}
\textbf{Question:} What are the projections of the point $(2, 3, 5)$ on the xy-, yz-, and xz-planes? Find the length of the diagonal of the box.

\textbf{Solution:}
1. \textbf{Projections:} To project onto a coordinate plane, set the coordinate perpendicular to that plane to 0.
   \begin{itemize}
       \item xy-plane (set $z=0$): $(2, 3, 0)$.
       \item yz-plane (set $x=0$): $(0, 3, 5)$.
       \item xz-plane (set $y=0$): $(2, 0, 5)$.
   \end{itemize}
2. \textbf{Diagonal Length:} The box has vertices at the origin $(0,0,0)$ and $P(2,3,5)$. The diagonal is the distance from the origin to $P$.
   \[ d = \sqrt{(2-0)^2 + (3-0)^2 + (5-0)^2} = \sqrt{2^2 + 3^2 + 5^2} = \sqrt{4 + 9 + 25} = \sqrt{38} \]

\textbf{Final Answer:} Projections: $(2, 3, 0), (0, 3, 5), (2, 0, 5)$. Diagonal: $\sqrt{38}$.

\subsection*{Problem 5}
\textbf{Question:} What does the equation $x = 4$ represent in $\mathbb{R}^2$? What does it represent in $\mathbb{R}^3$?

\textbf{Solution:}
\begin{itemize}
    \item In $\mathbb{R}^2$: The set of points $\{(4, y) \mid y \in \mathbb{R}\}$. This is a \textbf{vertical line} passing through $x=4$.
    \item In $\mathbb{R}^3$: The set of points $\{(4, y, z) \mid y, z \in \mathbb{R}\}$. Since $y$ and $z$ are free variables, they span a 2D surface. This is a \textbf{plane} parallel to the $yz$-plane, passing through $x=4$.
\end{itemize}

\subsection*{Problem 6}
\textbf{Question:} What does $y = 3$ represent in $\mathbb{R}^3$? What does $z = 5$ represent? What does the pair of equations $y = 3, z = 5$ represent?

\textbf{Solution:}
\begin{itemize}
    \item $y=3$: A plane parallel to the $xz$-plane, intersecting the y-axis at 3.
    \item $z=5$: A plane parallel to the $xy$-plane (a horizontal plane), 5 units up.
    \item $y=3, z=5$: The intersection of the two planes described above. This intersection is a \textbf{line} parallel to the x-axis, passing through $(0, 3, 5)$.
\end{itemize}

\subsection*{Problem 7}
\textbf{Question:} Describe and sketch the surface in $\mathbb{R}^3$ represented by $x + y = 2$.

\textbf{Solution:}
This is a linear equation involving $x$ and $y$, but no $z$.
Since $z$ is missing, it is a free variable.
In the $xy$-plane ($z=0$), $x+y=2$ is a line with x-intercept 2 and y-intercept 2.
Since $z$ can be anything, we "extrude" this line up and down along the z-axis.
\textbf{Description:} A vertical plane that intersects the xy-plane in the line $y = -x + 2$.

\subsection*{Problem 8}
\textbf{Question:} Describe and sketch the surface in $\mathbb{R}^3$ represented by $x^2 + z^2 = 9$.

\textbf{Solution:}
The equation involves $x$ and $z$, but no $y$.
In the $xz$-plane ($y=0$), $x^2 + z^2 = 9$ is a circle with radius 3 centered at the origin.
Since $y$ is free, this circle extends infinitely along the y-axis.
\textbf{Description:} A circular cylinder of radius 3 centered on the y-axis.

\subsection*{Problem 9}
\textbf{Question:} Find the distance between $(3, 5, -2)$ and $(-1, 1, -4)$.

\textbf{Solution:}
\[ d = \sqrt{(-1 - 3)^2 + (1 - 5)^2 + (-4 - (-2))^2} \]
\[ d = \sqrt{(-4)^2 + (-4)^2 + (-2)^2} \]
\[ d = \sqrt{16 + 16 + 4} = \sqrt{36} = 6 \]

\textbf{Final Answer:} 6

\subsection*{Problem 10}
\textbf{Question:} Find the distance between $(-6, -3, 0)$ and $(2, 4, 5)$.

\textbf{Solution:}
\[ d = \sqrt{(2 - (-6))^2 + (4 - (-3))^2 + (5 - 0)^2} \]
\[ d = \sqrt{(8)^2 + (7)^2 + (5)^2} \]
\[ d = \sqrt{64 + 49 + 25} = \sqrt{138} \]

\textbf{Final Answer:} $\sqrt{138}$

\subsection*{Problem 11}
\textbf{Question:} Find lengths of sides of triangle PQR with $P(3, -2, -3), Q(7, 0, 1), R(1, 2, 1)$. Is it a right triangle? Is it isosceles?

\textbf{Solution:}
Calculate squared lengths to avoid roots initially:
\begin{itemize}
    \item $|PQ|^2 = (7-3)^2 + (0-(-2))^2 + (1-(-3))^2 = 4^2 + 2^2 + 4^2 = 16 + 4 + 16 = 36$. So $|PQ|=6$.
    \item $|QR|^2 = (1-7)^2 + (2-0)^2 + (1-1)^2 = (-6)^2 + 2^2 + 0^2 = 36 + 4 + 0 = 40$. So $|QR|=\sqrt{40}$.
    \item $|RP|^2 = (3-1)^2 + (-2-2)^2 + (-3-1)^2 = 2^2 + (-4)^2 + (-4)^2 = 4 + 16 + 16 = 36$. So $|RP|=6$.
\end{itemize}
\textbf{Analysis:}
\begin{enumerate}
    \item Isosceles? Yes, $|PQ| = |RP| = 6$.
    \item Right triangle? Check Pythagorean theorem: $a^2 + b^2 = c^2$.
    The sides squared are 36, 40, 36.
    Does $36 + 36 = 40$? No. Does $36 + 40 = 36$? No.
    It is not a right triangle.
\end{enumerate}

\textbf{Final Answer:} Sides: $6, \sqrt{40}, 6$. Isosceles: Yes. Right: No.

\subsection*{Problem 12}
\textbf{Question:} $P(2, -1, 0), Q(4, 1, 1), R(4, -5, 4)$. Right or Isosceles?

\textbf{Solution:}
\begin{itemize}
    \item $|PQ|^2 = (4-2)^2 + (1-(-1))^2 + (1-0)^2 = 2^2 + 2^2 + 1^2 = 4+4+1 = 9$. So $|PQ|=3$.
    \item $|QR|^2 = (4-4)^2 + (-5-1)^2 + (4-1)^2 = 0^2 + (-6)^2 + 3^2 = 0+36+9 = 45$. So $|QR|=\sqrt{45} = 3\sqrt{5}$.
    \item $|RP|^2 = (2-4)^2 + (-1-(-5))^2 + (0-4)^2 = (-2)^2 + 4^2 + (-4)^2 = 4+16+16 = 36$. So $|RP|=6$.
\end{itemize}
\textbf{Analysis:}
\begin{enumerate}
    \item Isosceles? Sides are 3, $3\sqrt{5}$, 6. No equal sides.
    \item Right triangle? Check squares: 9, 45, 36.
    Note that $9 + 36 = 45$. This satisfies $|PQ|^2 + |RP|^2 = |QR|^2$.
\end{enumerate}

\textbf{Final Answer:} Scalene. Right triangle (hypotenuse QR).

\subsection*{Problem 13}
\textbf{Question:} Determine whether the points lie on a straight line.
(a) $A(2, 4, 2), B(3, 7, -2), C(1, 3, 3)$
(b) $D(0, -5, 5), E(1, -2, 4), F(3, 4, 2)$

\textbf{Solution:}
Points are collinear if the sum of the shorter distances equals the longest distance (or if vectors are scalar multiples). We use the distance method.

\textbf{(a)}
\begin{itemize}
    \item $|AB| = \sqrt{(3-2)^2 + (7-4)^2 + (-2-2)^2} = \sqrt{1^2 + 3^2 + (-4)^2} = \sqrt{1+9+16} = \sqrt{26}$.
    \item $|BC| = \sqrt{(1-3)^2 + (3-7)^2 + (3-(-2))^2} = \sqrt{(-2)^2 + (-4)^2 + 5^2} = \sqrt{4+16+25} = \sqrt{45} = 3\sqrt{5}$.
    \item $|AC| = \sqrt{(1-2)^2 + (3-4)^2 + (3-2)^2} = \sqrt{(-1)^2 + (-1)^2 + 1^2} = \sqrt{3}$.
\end{itemize}
Approximations: $\sqrt{26} \approx 5.1$, $3\sqrt{5} \approx 6.7$, $\sqrt{3} \approx 1.73$.
Does $5.1 + 1.73 = 6.7$? No. $\sqrt{26} + \sqrt{3} \neq \sqrt{45}$.
\textbf{Answer (a):} Not collinear.

\textbf{(b)}
\begin{itemize}
    \item $|DE| = \sqrt{(1-0)^2 + (-2-(-5))^2 + (4-5)^2} = \sqrt{1 + 9 + 1} = \sqrt{11}$.
    \item $|EF| = \sqrt{(3-1)^2 + (4-(-2))^2 + (2-4)^2} = \sqrt{4 + 36 + 4} = \sqrt{44} = 2\sqrt{11}$.
    \item $|DF| = \sqrt{(3-0)^2 + (4-(-5))^2 + (2-5)^2} = \sqrt{9 + 81 + 9} = \sqrt{99} = 3\sqrt{11}$.
\end{itemize}
Check: $|DE| + |EF| = \sqrt{11} + 2\sqrt{11} = 3\sqrt{11} = |DF|$.
\textbf{Answer (b):} Yes, they are collinear.

\subsection*{Problem 14}
\textbf{Question:} Find the distance from $(4, -2, 6)$ to:
(a) xy-plane (b) yz-plane (c) xz-plane (d) x-axis (e) y-axis (f) z-axis.

\textbf{Solution:} Let $P(x,y,z) = (4, -2, 6)$.
\begin{itemize}
    \item (a) xy-plane: $|z| = |6| = 6$.
    \item (b) yz-plane: $|x| = |4| = 4$.
    \item (c) xz-plane: $|y| = |-2| = 2$.
    \item (d) x-axis: Distance is $\sqrt{y^2+z^2} = \sqrt{(-2)^2+6^2} = \sqrt{4+36} = \sqrt{40} = 2\sqrt{10}$.
    \item (e) y-axis: Distance is $\sqrt{x^2+z^2} = \sqrt{4^2+6^2} = \sqrt{16+36} = \sqrt{52} = 2\sqrt{13}$.
    \item (f) z-axis: Distance is $\sqrt{x^2+y^2} = \sqrt{4^2+(-2)^2} = \sqrt{16+4} = \sqrt{20} = 2\sqrt{5}$.
\end{itemize}

\subsection*{Problem 15}
\textbf{Question:} Equation of sphere with center $(-3, 2, 5)$ and radius 4. Intersection with yz-plane?

\textbf{Solution:}
Equation: $(x - (-3))^2 + (y - 2)^2 + (z - 5)^2 = 4^2 \implies (x+3)^2 + (y-2)^2 + (z-5)^2 = 16$.
Intersection with yz-plane ($x=0$):
Substitute $x=0$: $(0+3)^2 + (y-2)^2 + (z-5)^2 = 16$.
$9 + (y-2)^2 + (z-5)^2 = 16$.
$(y-2)^2 + (z-5)^2 = 7$.
This is a \textbf{circle} in the yz-plane centered at $(0, 2, 5)$ with radius $\sqrt{7}$.

\subsection*{Problem 16}
\textbf{Question:} Sphere center $(2, -6, 4)$, radius 5. Intersections with coordinate planes?

\textbf{Solution:}
Equation: $(x-2)^2 + (y+6)^2 + (z-4)^2 = 25$.
\begin{itemize}
    \item \textbf{xy-plane ($z=0$):} $(x-2)^2 + (y+6)^2 + (0-4)^2 = 25 \implies (x-2)^2 + (y+6)^2 + 16 = 25 \implies (x-2)^2 + (y+6)^2 = 9$. Circle, radius 3.
    \item \textbf{yz-plane ($x=0$):} $(0-2)^2 + (y+6)^2 + (z-4)^2 = 25 \implies 4 + (y+6)^2 + (z-4)^2 = 25 \implies (y+6)^2 + (z-4)^2 = 21$. Circle, radius $\sqrt{21}$.
    \item \textbf{xz-plane ($y=0$):} $(x-2)^2 + (0+6)^2 + (z-4)^2 = 25 \implies (x-2)^2 + 36 + (z-4)^2 = 25$. Since $36 > 25$, we get $(x-2)^2 + (z-4)^2 = -11$. No solution. \textbf{No intersection}.
\end{itemize}

\subsection*{Problem 17}
\textbf{Question:} Sphere passing through $(4, 3, -1)$ with center $(3, 8, 1)$.

\textbf{Solution:}
The radius is the distance between the center and the point.
$r = \sqrt{(4-3)^2 + (3-8)^2 + (-1-1)^2} = \sqrt{1^2 + (-5)^2 + (-2)^2} = \sqrt{1+25+4} = \sqrt{30}$.
Equation: $(x-3)^2 + (y-8)^2 + (z-1)^2 = 30$.

\subsection*{Problem 18}
\textbf{Question:} Sphere passing through origin, center $(1, 2, 3)$.

\textbf{Solution:}
Radius is distance from $(1, 2, 3)$ to $(0, 0, 0)$.
$r = \sqrt{1^2 + 2^2 + 3^2} = \sqrt{1+4+9} = \sqrt{14}$.
Equation: $(x-1)^2 + (y-2)^2 + (z-3)^2 = 14$.

\subsection*{Problem 19}
\textbf{Question:} Show equation represents a sphere, find center/radius: $x^2 + y^2 + z^2 + 8x - 2z = 8$.

\textbf{Solution:}
Group terms and complete the square:
$(x^2 + 8x) + y^2 + (z^2 - 2z) = 8$.
Add $(8/2)^2 = 16$ and $(-2/2)^2 = 1$ to both sides.
$(x^2 + 8x + 16) + y^2 + (z^2 - 2z + 1) = 8 + 16 + 1$.
$(x+4)^2 + y^2 + (z-1)^2 = 25$.
\textbf{Center:} $(-4, 0, 1)$. \textbf{Radius:} $\sqrt{25} = 5$.

\subsection*{Problem 20}
\textbf{Question:} $x^2 + y^2 + z^2 = 6x - 4y - 10z$.

\textbf{Solution:}
Move terms: $x^2 - 6x + y^2 + 4y + z^2 + 10z = 0$.
Complete squares:
$(x^2 - 6x + 9) + (y^2 + 4y + 4) + (z^2 + 10z + 25) = 0 + 9 + 4 + 25$.
$(x-3)^2 + (y+2)^2 + (z+5)^2 = 38$.
\textbf{Center:} $(3, -2, -5)$. \textbf{Radius:} $\sqrt{38}$.

\subsection*{Problem 21}
\textbf{Question:} $2x^2 + 2y^2 + 2z^2 - 2x + 4y + 1 = 0$.

\textbf{Solution:}
Divide by 2: $x^2 + y^2 + z^2 - x + 2y + 0.5 = 0$.
Group: $(x^2 - x) + (y^2 + 2y) + z^2 = -0.5$.
Complete squares:
$(x^2 - x + 0.25) + (y^2 + 2y + 1) + z^2 = -0.5 + 0.25 + 1$.
$(x - 0.5)^2 + (y+1)^2 + z^2 = 0.75$.
\textbf{Center:} $(0.5, -1, 0)$. \textbf{Radius:} $\sqrt{0.75} = \frac{\sqrt{3}}{2}$.

\subsection*{Problem 22}
\textbf{Question:} $4x^2 + 4y^2 + 4z^2 = 16x - 6y - 12$.

\textbf{Solution:}
Divide by 4: $x^2 + y^2 + z^2 = 4x - 1.5y - 3$.
Rearrange: $(x^2 - 4x) + (y^2 + 1.5y) + z^2 = -3$.
Complete squares:
$(x-2)^2 + (y + 0.75)^2 + z^2 = -3 + 4 + 0.75^2$.
$(x-2)^2 + (y + 3/4)^2 + z^2 = 1 + 9/16 = 25/16$.
\textbf{Center:} $(2, -0.75, 0)$. \textbf{Radius:} $5/4 = 1.25$.

\subsection*{Problem 23}
\textbf{Question:} Prove the Midpoint Formula.

\textbf{Solution:}
Let $P_1(x_1, y_1, z_1)$ and $P_2(x_2, y_2, z_2)$ be points. Let $M$ be the midpoint.
From geometry using similar triangles, the projection of the midpoint onto the x-axis must be the midpoint of the projections of $P_1$ and $P_2$. The midpoint of $x_1$ and $x_2$ on the real number line is $\frac{x_1+x_2}{2}$. Applying this logic to y and z coordinates independently yields $M(\frac{x_1+x_2}{2}, \frac{y_1+y_2}{2}, \frac{z_1+z_2}{2})$.

\subsection*{Problem 24}
\textbf{Question:} Find center of sphere if endpoints of diameter are $(5, 4, 3)$ and $(1, 6, -9)$. Find equation.

\textbf{Solution:}
Center is midpoint of diameter: $C = (\frac{5+1}{2}, \frac{4+6}{2}, \frac{3-9}{2}) = (3, 5, -3)$.
Radius is distance from Center to $(5,4,3)$:
$r^2 = (5-3)^2 + (4-5)^2 + (3-(-3))^2 = 2^2 + (-1)^2 + 6^2 = 4 + 1 + 36 = 41$.
Equation: $(x-3)^2 + (y-5)^2 + (z+3)^2 = 41$.

\subsection*{Problem 25}
\textbf{Question:} Equation of sphere center $(-1, 4, 5)$ touching (a) xy-plane, (b) yz-plane, (c) xz-plane.

\textbf{Solution:}
To touch a plane, the radius must equal distance to that plane.
\begin{itemize}
    \item (a) Touch xy ($z=0$): $r = |z| = 5$. Eq: $(x+1)^2 + (y-4)^2 + (z-5)^2 = 25$.
    \item (b) Touch yz ($x=0$): $r = |x| = 1$. Eq: $(x+1)^2 + (y-4)^2 + (z-5)^2 = 1$.
    \item (c) Touch xz ($y=0$): $r = |y| = 4$. Eq: $(x+1)^2 + (y-4)^2 + (z-5)^2 = 16$.
\end{itemize}

\subsection*{Problem 26}
\textbf{Question:} Closest coordinate plane to $(7, 3, 8)$? Equation of sphere touching it?

\textbf{Solution:}
Distances: to yz ($x=7$), to xz ($y=3$), to xy ($z=8$).
Smallest distance is 3 (to xz-plane).
Sphere center $(7, 3, 8)$, radius 3.
Eq: $(x-7)^2 + (y-3)^2 + (z-8)^2 = 9$.

\subsection*{Problems 27-42: Describe the Region}
\textbf{27.} $z = -2$: A horizontal plane 2 units below the xy-plane.
\textbf{28.} $x = 3$: A vertical plane parallel to the yz-plane at x=3.
\textbf{29.} $y > 1$: The half-space to the right of the plane $y=1$.
\textbf{30.} $x < 4$: The half-space behind the plane $x=4$.
\textbf{31.} $-1 \leq x \leq 2$: A vertical slab between planes $x=-1$ and $x=2$.
\textbf{32.} $z = y$: A plane passing through the x-axis, tilted at $45^\circ$ between xy and xz planes.
\textbf{33.} $x^2 + y^2 = 4, z = -1$: A circle of radius 2 centered on the z-axis, located in the plane $z = -1$.
\textbf{34.} $x^2 + y^2 = 4$: A circular cylinder of radius 2 along the z-axis.
\textbf{35.} $y^2 + z^2 \leq 25$: A solid cylinder of radius 5 along the x-axis.
\textbf{36.} $x^2 + z^2 \leq 25, 0 \leq y \leq 2$: A segment of a solid cylinder of radius 5 along the y-axis, cut between $y=0$ and $y=2$.
\textbf{37.} $x^2 + y^2 + z^2 = 4$: A sphere surface of radius 2 centered at origin.
\textbf{38.} $x^2 + y^2 + z^2 \leq 4$: A solid ball (sphere interior included) of radius 2.
\textbf{39.} $1 \leq x^2 + y^2 + z^2 \leq 5$: The solid shell between spheres of radius 1 and $\sqrt{5}$.
\textbf{40.} $1 \leq x^2 + y^2 \leq 5$: A thick cylindrical shell (tube) along the z-axis with inner radius 1 and outer radius $\sqrt{5}$.
\textbf{41.} $0 \leq x \leq 3, 0 \leq y \leq 3, 0 \leq z \leq 3$: A solid cube of side 3 in the first octant with one corner at origin.
\textbf{42.} $x^2 + y^2 + z^2 > 2z \implies x^2 + y^2 + (z-1)^2 > 1$: The region outside the sphere centered at $(0, 0, 1)$ with radius 1.

\subsection*{Problems 43-46: Write Inequalities}
\textbf{43.} Region between yz-plane and $x=5$: $0 \leq x \leq 5$.
\textbf{44.} Cylinder below $z=8$, above disk radius 2 in xy: $x^2 + y^2 \leq 4, 0 \leq z \leq 8$.
\textbf{45.} Between spheres radius $r$ and $R$: $r^2 < x^2 + y^2 + z^2 < R^2$.
\textbf{46.} Solid upper hemisphere radius 2: $x^2 + y^2 + z^2 \leq 4, z \geq 0$.

\subsection*{Problem 47}
\textbf{Question:} Line $L_1$ and its projection $L_2$ on xy-plane. Find P on $L_1$. Locate A, B, C intersections.

\textbf{Solution:}
(a) From the diagram (assuming standard perspective), $P$ corresponds to a specific point shown. Assuming $P$ is the intercept or marked point. Without coordinate labels on the graph ticks, we infer $x=1, y=1, z=1$ based on grid lines. Let's assume $P(1, 1, 1)$.
(b) $A$ (xy-int): Set $z=0$. $B$ (yz-int): Set $x=0$. $C$ (xz-int): Set $y=0$. The line appears to pass through axes. (Visual estimation only).

\subsection*{Problem 48}
\textbf{Question:} Distance from $P$ to $A(-1, 5, 3)$ is twice distance to $B(6, 2, -2)$. Show it's a sphere.

\textbf{Solution:}
Let $P(x,y,z)$. Given $|PA| = 2|PB|$. Square both sides: $|PA|^2 = 4|PB|^2$.
$(x+1)^2 + (y-5)^2 + (z-3)^2 = 4[(x-6)^2 + (y-2)^2 + (z+2)^2]$.
$x^2+2x+1 + ... = 4(x^2-12x+36 + ...)$.
$x^2+y^2+z^2+2x-10y-6z+35 = 4(x^2+y^2+z^2-12x-4y+4z+44)$.
Bring to right side: $3x^2 + 3y^2 + 3z^2 - 50x - 6y + 22z + (176 - 35) = 0$.
Dividing by 3 yields $x^2+y^2+z^2 + ... = 0$, which is a sphere.

\subsection*{Problem 49}
\textbf{Question:} Points equidistant from $A(-1, 5, 3)$ and $B(6, 2, -2)$.

\textbf{Solution:}
$|PA| = |PB| \implies |PA|^2 = |PB|^2$.
$(x+1)^2 + (y-5)^2 + (z-3)^2 = (x-6)^2 + (y-2)^2 + (z+2)^2$.
Expand: $x^2+2x+1 + y^2-10y+25 + z^2-6z+9 = x^2-12x+36 + y^2-4y+4 + z^2+4z+4$.
Cancel squared terms.
$2x - 10y - 6z + 35 = -12x - 4y + 4z + 44$.
$14x - 6y - 10z - 9 = 0$.
\textbf{Description:} A plane (specifically, the perpendicular bisector plane of segment AB).

\subsection*{Problem 50}
\textbf{Question:} Find the volume of the solid inside both spheres: $x^2+y^2+z^2+4x-2y+4z+5=0$ and $x^2+y^2+z^2=4$.

\textbf{Solution:}
Sphere 1: Complete squares $\to (x+2)^2 + (y-1)^2 + (z+2)^2 = -5 + 4 + 1 + 4 = 4$. Center $C_1(-2, 1, -2)$, Radius $r=2$.
Sphere 2: Center $O(0,0,0)$, Radius $r=2$.
Distance between centers $d = \sqrt{(-2)^2 + 1^2 + (-2)^2} = \sqrt{9} = 3$.
Since $d=3$ and radii are $2$, they overlap ($2+2 > 3$).
The intersection volume is composed of two symmetrical spherical caps.
Height of one cap $h = r - d/2 = 2 - 1.5 = 0.5$.
Volume of one cap $V_{cap} = \frac{\pi h^2}{3}(3r - h) = \frac{\pi (0.5)^2}{3}(3(2) - 0.5) = \frac{\pi (0.25)}{3}(5.5) = \frac{1.375\pi}{3} = \frac{11\pi}{24}$.
Total Volume = $2 \times \frac{11\pi}{24} = \frac{11\pi}{12}$.

\subsection*{Problem 51}
\textbf{Question:} Distance between spheres $x^2+y^2+z^2=4$ and $x^2+y^2+z^2=4x+4y+4z-11$.

\textbf{Solution:}
Sphere 1: $C_1(0,0,0)$, $r_1=2$.
Sphere 2: $(x-2)^2 + (y-2)^2 + (z-2)^2 = -11 + 4 + 4 + 4 = 1$. $C_2(2,2,2)$, $r_2=1$.
Distance between centers $d = \sqrt{2^2+2^2+2^2} = \sqrt{12} = 2\sqrt{3}$.
Distance between surfaces = $d - r_1 - r_2 = 2\sqrt{3} - 2 - 1 = 2\sqrt{3} - 3$.

\subsection*{Problem 52}
\textbf{Question:} Describe a solid: z-shadow is disk, y-shadow is square, x-shadow is isosceles triangle.

\textbf{Solution:}
The solid is the intersection of a circular cylinder along the z-axis ($x^2+y^2 \le r^2$) and a triangular prism oriented along the x-axis (creates triangle in yz, square in xz if lengths align).
Specifically: A cylinder $x^2+y^2 \le 1$ intersected with a wedge defined by planes intersecting above/below the cylinder.

\newpage

\section{Part 3: In-Depth Analysis}
\subsection{Problem Types}
\begin{enumerate}
    \item \textbf{Visualization/Sketching (1, 2, 5, 6, 27-42):} Requires mapping algebraic constraints to geometric regions (planes, cylinders).
    \item \textbf{Distance \& Projections (3, 4, 9-14, 26, 51):} Pure application of distance formulas or coordinate stripping (projections).
    \item \textbf{Sphere Algebra (15-22, 24-25):} Requires "Completing the Square" to identify center and radius.
    \item \textbf{Locus Problems (48-49):} Setting up an equation based on a distance rule (e.g., $d_1 = 2d_2$) and simplifying to find the surface type.
\end{enumerate}

\subsection{Key Algebraic Manipulations}
\textbf{Completing the Square (Problems 19-22):}
This is the most crucial technique.
Example from Problem 19:
\[ x^2 + 8x \rightarrow (x+4)^2 - 16 \]
Why? It converts the expanded polynomial into the geometric form $(x-h)^2$ required to read the center and radius.

\section{Part 4: Cheatsheet}
\subsection{Formulas}
\begin{itemize}
    \item Distance: $d = \sqrt{\Delta x^2 + \Delta y^2 + \Delta z^2}$
    \item Sphere: $(x-h)^2 + (y-k)^2 + (z-l)^2 = r^2$
\end{itemize}

\subsection{Tricks & Tips}
\begin{itemize}
    \item \textbf{Missing Variable = Cylinder:} If an equation like $x^2+y^2=4$ is in 3D but missing $z$, it describes a cylinder extending infinitely along the missing axis.
    \item \textbf{Linear Equation = Plane:} Any equation of form $Ax + By + Cz = D$ is a flat plane.
    \item \textbf{Completing the Square Shortcut:} Divide the linear coefficient by 2 and square it. $x^2 + bx \to (x + \frac{b}{2})^2 - (\frac{b}{2})^2$.
\end{itemize}

\section{Part 5: Conceptual Synthesis}
\subsection{The Big Picture}
This chapter bridges 2D analytic geometry and vector calculus. The core theme is \textbf{Generalization}: taking concepts like distance, lines, and circles and adding a dimension ($z$).
\subsection{Future Links}
\begin{itemize}
    \item \textbf{Vectors (12.2):} Points $(x,y,z)$ will become vectors $\langle x,y,z \rangle$.
    \item \textbf{Partial Derivatives (Ch 14):} We will study rates of change along the x, y, and z axes independently.
    \item \textbf{Multiple Integrals (Ch 15):} We will calculate volumes of these regions (spheres, cylinders) using $\iiint dV$.
\end{itemize}

\section{Part 6: Real-World Application (Finance)}
\subsection{Scenario: Portfolio Optimization}
In Modern Portfolio Theory (Markowitz), a portfolio consisting of 3 assets can be visualized in $\mathbb{R}^3$ where axes $x, y, z$ represent the weight of capital allocated to Apple, Google, and Tesla.
\subsection{Model Problem}
\textbf{Constraint:} $x + y + z = 1$ (Total capital is 100\%). This is a plane in $\mathbb{R}^3$.
\textbf{Objective:} Minimize Risk (Variance) $\sigma^2 = x^2\sigma_x^2 + y^2\sigma_y^2 + z^2\sigma_z^2 + ...$
This defines elliptical surfaces. The optimal portfolio is where the risk ellipsoid touches the budget plane.

\section{Part 7: Untested Concepts}
\textbf{General Cylinders:} The homework covered circular cylinders ($x^2+y^2=r^2$).
\textit{Untested:} Parabolic cylinders (e.g., $z = y^2$). This is a sheet of paper curved into a parabola, extended along the x-axis.

\section{Part 8: Diagnostic Testing ("Find the Flaw")}

\subsection{Flawed Problem 1: Projection}
\textbf{Problem:} Find the projection of $P(2, 3, 5)$ onto the plane $z=2$.
\textbf{Flawed Solution:} To project onto a plane, we set the variable to 0. So projection is $(2, 3, 0)$.
\textbf{The Error:} The projection is onto $z=2$, not the xy-plane ($z=0$).
\textbf{Correct Solution:} The projection keeps $x$ and $y$ but sets $z$ to the plane's value. Answer: $(2, 3, 2)$.

\subsection{Flawed Problem 2: Sphere Center}
\textbf{Problem:} Find center of $x^2 + y^2 + z^2 + 6x - 4y = 0$.
\textbf{Flawed Solution:} Complete squares: $(x+3)^2 + (y-2)^2 + z^2 = 0$. Center is $(3, -2, 0)$.
\textbf{The Error:} Center coordinates are the values subtracted from variables: $(x-h)$. $(x+3)$ means $h=-3$.
\textbf{Correct Solution:} Center is $(-3, 2, 0)$.

\subsection{Flawed Problem 3: Distance Logic}
\textbf{Problem:} Is triangle with sides 3, 4, 6 a right triangle?
\textbf{Flawed Solution:} $3^2 + 4^2 = 9 + 16 = 25$. Since $25 \neq 6^2 (36)$, it is not.
\textbf{The Error:} Correct logic, but often students assume hypotenuse is the last number without checking all combos. (This solution is actually correct, but the trap is usually hidden in failing to check if $\sqrt{a}+\sqrt{b} > \sqrt{c}$ for valid triangle).
\textit{Alternative Flaw:} "Distance from $(1,1,1)$ to $x$-axis is 1."
\textbf{Error:} Distance to axis is $\sqrt{y^2+z^2} = \sqrt{1+1} = \sqrt{2}$.

\subsection{Flawed Problem 4: Inequality}
\textbf{Problem:} Describe region $x^2 + y^2 < 4$.
\textbf{Flawed Solution:} A circle of radius 2 in the xy-plane.
\textbf{The Error:} In $\mathbb{R}^3$, a missing variable implies a cylinder.
\textbf{Correct Solution:} The interior of a cylinder of radius 2 centered on the z-axis.

\subsection{Flawed Problem 5: Collinearity}
\textbf{Problem:} Are $A, B, C$ collinear? $|AB|=2, |BC|=3, |AC|=6$.
\textbf{Flawed Solution:} $2+3 \neq 6$, so no.
\textbf{The Error:} While correct that they aren't collinear, the student must check if the \textit{sum of the two smaller} equals the larger. A common flaw is checking $AB+AC=BC$ when AC is the longest.

\end{document}