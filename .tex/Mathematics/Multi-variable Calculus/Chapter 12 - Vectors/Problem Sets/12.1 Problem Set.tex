\documentclass{article}
\usepackage{amsmath, amssymb, geometry, enumitem, graphicx}
\geometry{a4paper, margin=1in}

\title{Calculus III: 3D Coordinate Systems - Practice Problem Set}
\author{Based on Homework 12.1 Concepts}
\date{}

\begin{document}

\maketitle

\section*{Instructions}
This problem set is designed to test your understanding of Three-Dimensional Coordinate Systems. It covers the following concepts found in the provided study materials:
\begin{itemize}
    \item Coordinate Navigation and Plotting
    \item Distance Formulas (Points, Planes, Axes)
    \item Projections
    \item Triangle Geometry in 3D
    \item Spheres (Standard Form, General Form, Completing the Square)
    \item Surface and Region Identification (Planes, Cylinders, Inequalities)
\end{itemize}

\section*{Problem Set}

\begin{enumerate}
    % --- Navigation & Coordinates ---
    \item \textbf{(Coordinate Navigation)} A particle starts at the origin $(0,0,0)$. It moves 5 units along the positive $y$-axis, then 3 units along the negative $z$-axis, and finally 2 units along the positive $x$-axis. What are the final coordinates of the particle?

    \item \textbf{(Coordinate Navigation)} A point starts at $P(3, -2, 5)$. It moves 4 units in the negative $y$-direction and 2 units in the positive $z$-direction. What is the distance between the starting point and the ending point?

    % --- Distances & Projections ---
    \item \textbf{(Distance to Planes)} Find the distance from the point $P(-4, 7, 2)$ to:
    \begin{enumerate}
        \item The $xy$-plane.
        \item The $yz$-plane.
        \item The $xz$-plane.
    \end{enumerate}

    \item \textbf{(Distance to Axes)} Find the distance from the point $P(3, -4, 12)$ to:
    \begin{enumerate}
        \item The $x$-axis.
        \item The $y$-axis.
        \item The $z$-axis.
    \end{enumerate}

    \item \textbf{(Projections)} Find the coordinates of the projection of the point $Q(5, -9, 8)$ onto:
    \begin{enumerate}
        \item The $xy$-plane.
        \item The $xz$-plane.
    \end{enumerate}

    \item \textbf{(Distance Between Points)} Calculate the distance between point $A(1, 5, -2)$ and point $B(4, 1, -2)$.

    \item \textbf{(Closest Point)} Consider points $A(2, -5, 8)$ and $B(-3, 1, 1)$. Which point is closer to the $yz$-plane?

    % --- Midpoint ---
    \item \textbf{(Midpoint)} Find the midpoint of the line segment connecting $P(2, -3, 8)$ and $Q(6, 5, -2)$.

    % --- Triangle Geometry ---
    \item \textbf{(Triangle Properties)} Given the points $A(1, 2, 3)$, $B(4, 6, 3)$, and $C(1, 6, 3)$:
    \begin{enumerate}
        \item Find the lengths of the sides of triangle $ABC$.
        \item Determine if the triangle is a right triangle.
    \end{enumerate}

    \item \textbf{(Isosceles Triangle)} Determine if the triangle with vertices $P(0,0,0)$, $Q(2,2,0)$, and $R(2,0,2)$ is isosceles.

    % --- Spheres (Basics) ---
    \item \textbf{(Sphere Equation)} Find the standard equation of a sphere with center $C(2, -1, 4)$ and radius $r=5$.

    \item \textbf{(Sphere Center/Point)} Find the equation of the sphere centered at $(0, 5, -2)$ that passes through the origin.

    \item \textbf{(Sphere Diameter)} Find the equation of the sphere that has the points $A(2, 1, 4)$ and $B(4, 3, 10)$ as endpoints of a diameter.

    % --- Spheres (Completing the Square) ---
    \item \textbf{(General to Standard Form)} Show that the equation $x^2 + y^2 + z^2 - 2x + 6y = 15$ represents a sphere. Find its center and radius.

    \item \textbf{(General to Standard Form)} Find the center and radius of the sphere defined by $x^2 + y^2 + z^2 + 4x - 4y + 2z - 1 = 0$.

    \item \textbf{(Sphere Logic)} Does the equation $x^2 + y^2 + z^2 + 2x + 2y + 2z + 10 = 0$ represent a sphere? Explain why or why not.

    % --- Regions and Surfaces ---
    \item \textbf{(Surface Identification)} Describe the surface in $\mathbb{R}^3$ represented by the equation $x = 4$.

    \item \textbf{(Surface Identification)} Describe the surface in $\mathbb{R}^3$ represented by the equation $y = -2$ and $z = 5$. Determine the geometric object formed by their intersection.

    \item \textbf{(Cylinders)} Describe the surface defined by $x^2 + z^2 = 16$ in $\mathbb{R}^3$.

    \item \textbf{(Cylinders)} Describe the surface defined by $y^2 + z^2 = 4$. Which axis does this cylinder run along?

    \item \textbf{(Region Description)} Describe the region given by the inequalities $0 \le x \le 2$, $0 \le y \le 2$, and $0 \le z \le 2$.

    \item \textbf{(Region Description)} Describe the solid region defined by $x^2 + y^2 + z^2 \le 9$ and $z \ge 0$.

    \item \textbf{(Region Description)} Describe the set of points satisfying $xy = 0$ in $\mathbb{R}^3$.

    % --- Intersections ---
    \item \textbf{(Intersection: Sphere/Plane)} Find the equation of the intersection of the sphere $(x-1)^2 + y^2 + z^2 = 25$ and the plane $z = 0$. Describe this intersection.

    \item \textbf{(Intersection: DNE)} Determine the intersection of the sphere $x^2 + y^2 + (z-2)^2 = 4$ and the plane $z = 5$.

    \item \textbf{(Intersection: Point)} Find the intersection of the sphere $x^2 + y^2 + z^2 = 9$ and the plane $y = 3$.

    % --- Advanced/Conceptual ---
    \item \textbf{(Tangency)} Find the equation of the sphere centered at $(3, -4, 5)$ that is tangent to the $xy$-plane.

    \item \textbf{(Tangency)} Find the equation of the sphere centered at $(3, -4, 5)$ that is tangent to the $xz$-plane.

    \item \textbf{(Distance Point to Sphere)} Find the shortest distance from the point $P(0,0,0)$ to the surface of the sphere $(x-3)^2 + (y-4)^2 + z^2 = 4$.

    \item \textbf{(Complex Region)} Describe the region defined by $1 \le x^2 + y^2 + z^2 \le 4$.

\end{enumerate}

\newpage

\section*{Solutions}

\begin{enumerate}
    \item \textbf{Answer:} $(2, 5, -3)$.
    \textit{Reasoning:} Start at $(0,0,0)$. $y+5 \to (0,5,0)$. $z-3 \to (0,5,-3)$. $x+2 \to (2,5,-3)$.

    \item \textbf{Answer:} $2\sqrt{5}$.
    \textit{Reasoning:} Displacement vector is $\langle 0, -4, 2 \rangle$. Distance is magnitude: $\sqrt{0^2 + (-4)^2 + 2^2} = \sqrt{16+4} = \sqrt{20} = 2\sqrt{5}$.

    \item 
    \begin{enumerate}
        \item $|z| = |2| = 2$.
        \item $|x| = |-4| = 4$.
        \item $|y| = |7| = 7$.
    \end{enumerate}

    \item 
    \begin{enumerate}
        \item Distance to x-axis: $\sqrt{y^2 + z^2} = \sqrt{(-4)^2 + 12^2} = \sqrt{16+144} = \sqrt{160} = 4\sqrt{10}$.
        \item Distance to y-axis: $\sqrt{x^2 + z^2} = \sqrt{3^2 + 12^2} = \sqrt{9+144} = \sqrt{153}$.
        \item Distance to z-axis: $\sqrt{x^2 + y^2} = \sqrt{3^2 + (-4)^2} = \sqrt{9+16} = 5$.
    \end{enumerate}

    \item 
    \begin{enumerate}
        \item $(5, -9, 0)$. (Set $z=0$).
        \item $(5, 0, 8)$. (Set $y=0$).
    \end{enumerate}

    \item \textbf{Answer:} $5$.
    \textit{Reasoning:} $d = \sqrt{(4-1)^2 + (1-5)^2 + (-2 - (-2))^2} = \sqrt{3^2 + (-4)^2 + 0} = \sqrt{9+16} = 5$.

    \item \textbf{Answer:} Point $A$.
    \textit{Reasoning:} Distance to $yz$-plane is $|x|$. For $A$: $|2|=2$. For $B$: $|-3|=3$. $2 < 3$.

    \item \textbf{Answer:} $(4, 1, 3)$.
    \textit{Reasoning:} Midpoint = $(\frac{2+6}{2}, \frac{-3+5}{2}, \frac{8-2}{2}) = (\frac{8}{2}, \frac{2}{2}, \frac{6}{2}) = (4, 1, 3)$.

    \item 
    \begin{enumerate}
        \item $|AB| = \sqrt{(4-1)^2 + (6-2)^2 + 0} = \sqrt{9+16} = 5$. \\
              $|BC| = \sqrt{(1-4)^2 + (6-6)^2 + 0} = \sqrt{9} = 3$. \\
              $|AC| = \sqrt{(1-1)^2 + (6-2)^2 + 0} = \sqrt{16} = 4$.
        \item Yes. $3^2 + 4^2 = 9 + 16 = 25 = 5^2$. This is a 3-4-5 right triangle.
    \end{enumerate}

    \item \textbf{Answer:} Yes.
    \textit{Reasoning:} 
    $|PQ| = \sqrt{2^2 + 2^2 + 0} = \sqrt{8}$.
    $|PR| = \sqrt{2^2 + 0 + 2^2} = \sqrt{8}$.
    Since two sides are equal, it is isosceles.

    \item \textbf{Answer:} $(x-2)^2 + (y+1)^2 + (z-4)^2 = 25$.

    \item \textbf{Answer:} $x^2 + (y-5)^2 + (z+2)^2 = 29$.
    \textit{Reasoning:} $r = \text{dist}(C, \text{Origin}) = \sqrt{0^2+5^2+(-2)^2} = \sqrt{29}$. $r^2 = 29$.

    \item \textbf{Answer:} $(x-3)^2 + (y-2)^2 + (z-7)^2 = 11$.
    \textit{Reasoning:} Midpoint (Center) = $(3, 2, 7)$. Radius is distance from Center to $A$: $\sqrt{(2-3)^2 + (1-2)^2 + (4-7)^2} = \sqrt{1+1+9} = \sqrt{11}$.

    \item \textbf{Answer:} Center $(1, -3, 0)$, Radius $5$.
    \textit{Reasoning:} $(x^2-2x+1) + (y^2+6y+9) + z^2 = 15 + 1 + 9$.
    $(x-1)^2 + (y+3)^2 + z^2 = 25$.

    \item \textbf{Answer:} Center $(-2, 2, -1)$, Radius $\sqrt{10}$.
    \textit{Reasoning:} $(x+2)^2 + (y-2)^2 + (z+1)^2 = 1 + 4 + 4 + 1 = 10$.

    \item \textbf{Answer:} No.
    \textit{Reasoning:} Completing squares: $(x+1)^2 + (y+1)^2 + (z+1)^2 = -10 + 1 + 1 + 1 = -7$. Radius squared cannot be negative.

    \item \textbf{Answer:} A plane parallel to the $yz$-plane, passing through $x=4$.

    \item \textbf{Answer:} A line parallel to the $x$-axis.
    \textit{Reasoning:} Intersection of plane $y=-2$ and plane $z=5$.

    \item \textbf{Answer:} A circular cylinder along the $y$-axis with radius 4.

    \item \textbf{Answer:} A circular cylinder along the $x$-axis with radius 2.

    \item \textbf{Answer:} A cube of side length 2 in the first octant with one corner at the origin.

    \item \textbf{Answer:} A solid hemisphere of radius 3 centered at the origin, lying above the $xy$-plane.

    \item \textbf{Answer:} The union of the $xz$-plane ($y=0$) and the $yz$-plane ($x=0$).

    \item \textbf{Answer:} Circle in the $z=0$ plane with equation $(x-1)^2 + y^2 = 25$. Center $(1,0,0)$, radius 5.

    \item \textbf{Answer:} DNE (Empty Set).
    \textit{Reasoning:} Substitute $z=5$: $x^2+y^2+(5-2)^2 = 4 \implies x^2+y^2+9=4 \implies x^2+y^2=-5$. Impossible.

    \item \textbf{Answer:} The point $(0, 3, 0)$.
    \textit{Reasoning:} $x^2 + 3^2 + z^2 = 9 \implies x^2 + z^2 = 0 \implies x=0, z=0$.

    \item \textbf{Answer:} $(x-3)^2 + (y+4)^2 + (z-5)^2 = 25$.
    \textit{Reasoning:} Tangent to $xy$-plane means radius equals distance to $xy$-plane, which is $|z| = 5$.

    \item \textbf{Answer:} $(x-3)^2 + (y+4)^2 + (z-5)^2 = 16$.
    \textit{Reasoning:} Tangent to $xz$-plane means radius equals distance to $xz$-plane, which is $|y| = |-4| = 4$.

    \item \textbf{Answer:} $3$.
    \textit{Reasoning:} Center $C(3,4,0)$. Radius $r=2$. Distance from Origin to Center $d = \sqrt{3^2+4^2} = 5$. Distance to surface $= d - r = 5 - 2 = 3$.

    \item \textbf{Answer:} A spherical shell. The solid region between the sphere of radius 1 and the sphere of radius 2 (centered at origin).
\end{enumerate}

\newpage

\section*{Concept Checklist & Verification}

This section maps the problem numbers to the concepts identified in the provided homework files.

\begin{itemize}
    \item \textbf{Coordinate Navigation:} Problems 1, 2, 30(conceptually).
    \item \textbf{Distance Formula (Points):} Problems 2, 6, 8(conceptually).
    \item \textbf{Distance to Planes ($|x|, |y|, |z|$):} Problems 3, 7, 27, 28.
    \item \textbf{Distance to Axes (Pythagorean):} Problem 4.
    \item \textbf{Projections:} Problem 5.
    \item \textbf{Midpoint Formula:} Problems 8, 13.
    \item \textbf{Triangle Geometry (Right/Isosceles):} Problems 9, 10.
    \item \textbf{Sphere Equation (Center/Radius):} Problems 11, 25.
    \item \textbf{Sphere Equation (Center/Point):} Problem 12.
    \item \textbf{Sphere Equation (Diameter Endpoints):} Problem 13.
    \item \textbf{Completing the Square (Spheres):} Problems 14, 15, 16.
    \item \textbf{Surface Identification (Planes/Lines):} Problems 17, 18.
    \item \textbf{Surface Identification (Cylinders):} Problems 19, 20.
    \item \textbf{Describing Regions (Inequalities):} Problems 21, 22, 30.
    \item \textbf{Intersections (Sphere \& Plane):} Problems 23, 24, 25.
    \item \textbf{Tangency Concepts:} Problems 27, 28.
\end{itemize}

\end{document}