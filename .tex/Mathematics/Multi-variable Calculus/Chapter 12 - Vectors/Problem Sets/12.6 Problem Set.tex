\documentclass[12pt]{article}
\usepackage{amsmath, amssymb, graphicx, geometry, enumitem}
\geometry{a4paper, margin=1in}

\title{Calculus III Practice Set: Cylinders and Quadric Surfaces}
\author{Based on Homework 12.6 Study Guide}
\date{}

\begin{document}

\maketitle

\section*{Instructions}
Identify and describe the surfaces represented by the following equations. For quadric surfaces, specify the type (e.g., Ellipsoid, Hyperboloid of One Sheet) and the axis of symmetry or orientation. For shifted surfaces, identify the vertex or center.

\section{Part 1: Cylinders (Variable Missing)}
\textit{Concept: If a variable is missing, the surface is a cylinder parallel to that axis.}

\begin{enumerate}
    \item Identify and sketch the surface given by $x^2 + z^2 = 16$.
    \item Identify the surface given by $y = 4z^2$.
    \item Describe the surface $4x^2 + 25y^2 = 100$ in $\mathbb{R}^3$.
    \item Identify the surface $z = e^{-x}$.
    \item Identify the surface given by $x^2 - y^2 = 1$ in $\mathbb{R}^3$.
\end{enumerate}

\section{Part 2: Standard Quadric Classification}
\textit{Concept: Classify based on the standard forms $Ax^2 + By^2 + Cz^2 = J$. Pay attention to signs and powers.}

\begin{enumerate}[resume]
    \item Classify the surface: $x^2 + 4y^2 + z^2 = 1$.
    \item Classify and identify the axis of symmetry: $x^2 + y^2 - z^2 = 1$.
    \item Classify and identify the axis of symmetry: $-x^2 - y^2 + z^2 = 1$.
    \item Classify the surface: $y^2 = x^2 + z^2$.
    \item Classify and identify the direction of opening: $z = 3x^2 + 3y^2$.
    \item Classify the surface (Saddle): $z = y^2 - x^2$.
    \item Classify the surface: $9x^2 - 4y^2 + z^2 = 0$.
    \item Classify and identify the axis of symmetry: $x = 2y^2 + 5z^2$.
    \item Classify the surface: $4x^2 - y^2 + 9z^2 = 36$.
    \item Classify the surface: $x^2 + 2y^2 + 3z^2 = 0$.
\end{enumerate}

\section{Part 3: Shifted Surfaces (Completing the Square)}
\textit{Concept: Use completing the square to find the center $(h, k, l)$ and reduce to standard form.}

\begin{enumerate}[resume]
    \item Identify the surface and its center: $x^2 + y^2 + z^2 - 6x - 4y + 2z = 2$.
    \item Identify the surface and its vertex: $z = x^2 + y^2 - 4x + 6y + 13$.
    \item Identify the surface and its axis orientation: $x^2 - y^2 + z^2 - 4x - 2y = 0$.
    \item Identify the surface: $y = z^2 + 4z - 2x + 6$.
    \item Classify the surface: $4x^2 + y^2 - z^2 - 24x + 2y + 4z + 35 = 0$.
\end{enumerate}

\section{Part 4: Trace Analysis & Reverse Engineering}
\textit{Concept: Determine the 3D surface by analyzing 2D cross-sections (traces).}

\begin{enumerate}[resume]
    \item Find the equations of the traces for $x^2 + 4y^2 - z^2 = 4$ in the planes $z=0$, $x=0$, and $y=0$.
    \item A surface has traces that are ellipses in planes parallel to the $xy$-plane ($z=k$) and parabolas in planes parallel to the $xz$ and $yz$ planes. The surface opens upward. Write a possible equation.
    \item Identify the surface whose traces in $x=k$ are hyperbolas, traces in $y=k$ are hyperbolas, and traces in $z=k$ are circles ($k > 1$).
    \item If you rotate the line $z = 3y$ around the $z$-axis, what is the equation of the resulting surface?
    \item Match the description to a surface: "The cross-sections parallel to the $xz$-plane are parabolas opening down. The cross-sections parallel to the $yz$-plane are parabolas opening up."
\end{enumerate}

\section{Part 5: Conceptual & Diagnostic}
\textit{Concept: Test understanding of definitions, axes, and distinctions between similar forms.}

\begin{enumerate}[resume]
    \item Compare the surfaces $x^2 + y^2 - z^2 = 1$ and $x^2 + y^2 - z^2 = -1$. How do they differ?
    \item A student claims that $x^2 + z^2 = 9$ represents a circle. Why is this incorrect in the context of Multivariable Calculus?
    \item Determine the axis of symmetry for the hyperboloid $-2x^2 + y^2 + z^2 = 1$.
    \item True or False: The surface $x^2 + y^2 = z^2$ is an Elliptic Paraboloid. Explain.
    \item What is the trace of the Elliptic Paraboloid $z = \frac{x^2}{4} + \frac{y^2}{9}$ in the plane $z = -1$?
\end{enumerate}

\newpage
\section*{Answer Key and Solutions}

\subsection*{Part 1: Cylinders}
\begin{enumerate}
    \item \textbf{Circular Cylinder.} Variable $y$ is missing. Axis is the $y$-axis. Radius 4.
    \item \textbf{Parabolic Cylinder.} Variable $x$ is missing. Rulings parallel to the $x$-axis.
    \item \textbf{Elliptic Cylinder.} Variable $z$ is missing. Rulings parallel to the $z$-axis.
    \item \textbf{Exponential Cylinder.} Variable $y$ is missing. Shape defined by $z=e^{-x}$ extended along $y$.
    \item \textbf{Hyperbolic Cylinder.} Variable $z$ is missing. Hyperbola in $xy$-plane extruded along $z$.
\end{enumerate}

\subsection*{Part 2: Classification}
\begin{enumerate}[resume]
    \item \textbf{Ellipsoid.} All terms positive and squared, equals 1.
    \item \textbf{Hyperboloid of One Sheet.} One negative term ($-z^2$). Axis of symmetry is the $z$-axis.
    \item \textbf{Hyperboloid of Two Sheets.} Two negative terms ($-x^2, -y^2$) if rewritten as $-x^2 - y^2 + z^2 = 1$. Wait, standard form for 2-sheets has two negatives. Here, $z$ is positive. Axis is $z$.
    \item \textbf{Elliptic Cone.} Homogeneous degree 2 ($x^2 - y^2 + z^2 = 0 \implies y^2 = x^2 + z^2$). Axis is $y$.
    \item \textbf{Elliptic Paraboloid.} Linear $z$, quadratic $x, y$ same sign. Axis is $z$ (opens up).
    \item \textbf{Hyperbolic Paraboloid (Saddle).} Linear $z$, quadratic $x, y$ opposite signs.
    \item \textbf{Elliptic Cone.} Homogeneous ($9x^2 + z^2 = 4y^2$). Axis is $y$.
    \item \textbf{Elliptic Paraboloid.} Linear $x$. Axis is $x$ (opens positive direction).
    \item \textbf{Hyperboloid of One Sheet.} Divide by 36: $\frac{x^2}{9} - \frac{y^2}{36} + \frac{z^2}{4} = 1$. Axis is $y$ (negative term).
    \item \textbf{Point (Origin).} Sum of squares equals 0 implies $x=y=z=0$.
\end{enumerate}

\subsection*{Part 3: Shifted Surfaces}
\begin{enumerate}[resume]
    \item \textbf{Sphere/Ellipsoid.} Complete squares: $(x-3)^2 + (y-2)^2 + (z+1)^2 = 2 + 9 + 4 + 1 = 16$. Sphere centered at $(3, 2, -1)$ with radius 4.
    \item \textbf{Elliptic Paraboloid.} $z = (x-2)^2 + (y+3)^2 + 13 - 4 - 9 \implies z = (x-2)^2 + (y+3)^2$. Vertex at $(2, -3, 0)$.
    \item \textbf{Hyperboloid of One Sheet.} $(x-2)^2 - (y+1)^2 + z^2 = 4 - 1 = 3$. Axis parallel to $y$-axis. Center $(2, -1, 0)$.
    \item \textbf{Parabolic Cylinder.} Rearrange: $2x = z^2 + 4z - y + 6$. Actually, wait. The problem is $y = z^2 + 4z - 2x + 6$. This is linear in $y$ and $x$, quadratic in $z$. This is a Cylinder? No. Let's solve for linear variables. $2x + y = (z+2)^2 + 2$. This is a Parabolic Cylinder. The trace in any plane $z=k$ is a line $2x+y=C$. The surface is "folded" along a parabola.
    \item \textbf{Hyperboloid of Two Sheets.} $4(x-3)^2 + (y+1)^2 - (z-2)^2 = -35 + 36 + 1 - 4 = -2$.
    Divide by -2: $-2(x-3)^2 - \frac{1}{2}(y+1)^2 + \frac{1}{2}(z-2)^2 = 1$. Two negatives ($x, y$). Axis parallel to $z$-axis.
\end{enumerate}

\subsection*{Part 4: Trace Analysis}
\begin{enumerate}[resume]
    \item $z=0$: Ellipse ($x^2+4y^2=4$). $x=0$: Hyperbola ($4y^2-z^2=4$). $y=0$: Hyperbola ($x^2-z^2=4$).
    \item \textbf{Elliptic Paraboloid.} Form $z = Ax^2 + By^2$.
    \item \textbf{Hyperboloid of Two Sheets.} Standard form $-x^2 - y^2 + z^2 = 1$. Traces in $z=k$ ($k>1$) are circles $x^2+y^2 = k^2-1$. Traces in $x,y$ are hyperbolas.
    \item \textbf{Circular Cone.} $z^2 = 9(x^2 + y^2)$.
    \item \textbf{Hyperbolic Paraboloid (Saddle).} Equation form $z = y^2 - x^2$.
\end{enumerate}

\subsection*{Part 5: Conceptual}
\begin{enumerate}[resume]
    \item $x^2+y^2-z^2=1$ is a \textbf{Hyperboloid of One Sheet} (Axis $z$). $x^2+y^2-z^2=-1$ becomes $-x^2-y^2+z^2=1$, which is a \textbf{Hyperboloid of Two Sheets} (Axis $z$).
    \item In $\mathbb{R}^3$, an equation with one missing variable represents a \textbf{Cylinder}, not just a 2D curve. It is a Circular Cylinder.
    \item The variable with the \textbf{positive} coefficient determines the axis for a Hyperboloid of Two Sheets (since RHS=1). Here we have $-2x^2 + y^2 + z^2 = 1$. Two positives? Wait.
    Let's check PDF: "Two negative terms = Hyperboloid 2 Sheets". This equation has one negative ($-2x^2$) and two positives ($+y^2, +z^2$).
    This fits the "One Negative = Hyperboloid of One Sheet" rule.
    Therefore, the axis corresponds to the \textbf{negative variable}, which is the $x$-axis.
    \item \textbf{False.} $z^2 = x^2 + y^2$ is homogeneous. It is a \textbf{Cone}. A paraboloid has $z$ to the first power.
    \item \textbf{Empty Set.} $z = -1$ implies $-1 = x^2/4 + y^2/9$. Sum of squares cannot be negative.
\end{enumerate}

\section*{Concept Checklist & Problem Mapping}
\begin{itemize}
    \item \textbf{Cylinders (Type 1):} Problems 1, 2, 3, 4, 5.
    \item \textbf{Ellipsoids:} Problems 6, 21.
    \item \textbf{Hyperboloids of One Sheet:} Problems 7, 14, 18, 26, 28.
    \item \textbf{Hyperboloids of Two Sheets:} Problems 8, 20, 23, 26.
    \item \textbf{Elliptic Cones:} Problems 9, 12, 15, 24, 29.
    \item \textbf{Elliptic Paraboloids:} Problems 10, 13, 17, 22.
    \item \textbf{Hyperbolic Paraboloids (Saddle):} Problems 11, 25.
    \item \textbf{Completing the Square (Type 3):} Problems 16, 17, 18, 20.
    \item \textbf{Reverse Engineering Traces (Type 4):} Problems 21, 22, 23, 24, 25.
    \item \textbf{Diagnostic/Tricky Cases:} Problems 19, 27, 28, 29, 30.
\end{itemize}

\end{document}