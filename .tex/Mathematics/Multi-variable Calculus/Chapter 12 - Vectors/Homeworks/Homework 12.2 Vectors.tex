\documentclass[12pt, letterpaper]{article}
\usepackage[utf8]{inputenc}
\usepackage{amsmath, amssymb, amsthm}
\usepackage{geometry}
\usepackage{graphicx}
\usepackage{enumitem}
\usepackage{fancyhdr}
\usepackage{xcolor}

\geometry{margin=1in}

% Header and Footer
\pagestyle{fancy}
\fancyhf{}
\rhead{Tashfeen Omran}
\lhead{Homework 12.2 Vectors}
\rfoot{Page \thepage}

% Title Information
\title{\textbf{Comprehensive Guide \& Solutions: Homework 12.2 Vectors}}
\author{Tashfeen Omran}
\date{January 2026}

\begin{document}

\maketitle
\tableofcontents
\newpage

\section{Part 1: Introduction, Context, and Prerequisites}

\subsection{Core Concepts}
A \textbf{vector} is a mathematical object that possesses both \textbf{magnitude} (length or size) and \textbf{direction}. This distinguishes it from a \textbf{scalar}, which has only magnitude (represented by a real number). Geometrically, a vector is often represented as a directed line segment (an arrow).

If a vector $\mathbf{v}$ starts at point $A$ (initial point) and ends at point $B$ (terminal point), we denote it as $\vec{AB}$ or $\mathbf{v}$.

\textbf{Component Form:}
To perform algebraic operations, we place the initial point at the origin $(0,0)$. The coordinates of the terminal point then define the vector's components.
\[ \mathbf{v} = \langle v_1, v_2 \rangle \quad \text{(2D)} \quad \text{or} \quad \mathbf{v} = \langle v_1, v_2, v_3 \rangle \quad \text{(3D)} \]

\subsection{Intuition and Derivation}
The intuition for vectors comes from \textbf{displacement}. If you walk 3 miles North and then 4 miles East, your total distance walked is 7 miles, but your net displacement (vector sum) is 5 miles Northeast.

The formula for the magnitude of a vector is derived directly from the \textbf{Pythagorean Theorem}. In 2D, if $\mathbf{v} = \langle v_1, v_2 \rangle$, the length of the vector is the hypotenuse of a right triangle with legs $v_1$ and $v_2$.
\[ |\mathbf{v}| = \sqrt{v_1^2 + v_2^2} \]

\subsection{Historical Context}
The concept of vectors emerged in the 19th century as mathematicians sought to extend complex numbers to three dimensions. William Rowan Hamilton developed \textit{quaternions} in 1843, which included a vector part. Later, physicists and mathematicians like Josiah Willard Gibbs and Oliver Heaviside stripped away the quaternion framework to create the modern system of vector analysis we use today. This was driven by the need to model electromagnetic fields and fluid dynamics, where quantities like force and velocity have specific directions in 3D space.

\subsection{Key Formulas}
\begin{enumerate}
    \item \textbf{Vector from Two Points:} Given $A(x_1, y_1)$ and $B(x_2, y_2)$:
    \[ \vec{AB} = \langle x_2 - x_1, y_2 - y_1 \rangle \]
    \item \textbf{Magnitude (Length):}
    \[ |\mathbf{a}| = \sqrt{a_1^2 + a_2^2} \quad (\text{2D}) \quad \text{or} \quad \sqrt{a_1^2 + a_2^2 + a_3^2} \quad (\text{3D}) \]
    \item \textbf{Vector Addition:}
    \[ \langle a_1, a_2 \rangle + \langle b_1, b_2 \rangle = \langle a_1 + b_1, a_2 + b_2 \rangle \]
    \item \textbf{Scalar Multiplication:}
    \[ c \langle a_1, a_2 \rangle = \langle c a_1, c a_2 \rangle \]
    \item \textbf{Unit Vector (Normalization):} A vector with length 1 in the direction of $\mathbf{a}$:
    \[ \mathbf{u} = \frac{\mathbf{a}}{|\mathbf{a}|} \]
\end{enumerate}

\subsection{Prerequisites}
\begin{itemize}
    \item \textbf{Pythagorean Theorem:} Essential for finding distances and magnitudes.
    \item \textbf{Trigonometry:} Sine, Cosine, and Tangent are needed to decompose vectors or find angles (direction).
    \item \textbf{Basic Algebra:} Solving linear equations and manipulating roots.
    \item \textbf{Differentiation:} (For Problem 14) Using derivatives to find slopes of tangent lines.
\end{itemize}

\newpage

\section{Part 2: Detailed Homework Solutions}

\subsection*{Problem 1: Scalars vs. Vectors}
\textbf{Question:} Are the following quantities vectors or scalars? Explain.

\textbf{Solution:}
\begin{enumerate}[label=(\alph*)]
    \item \textbf{The cost of a theater ticket.}
    \begin{itemize}
        \item \textbf{Answer:} Scalar.
        \item \textbf{Reasoning:} Cost is defined only by a magnitude (a number, e.g., \$15) and has no direction.
    \end{itemize}
    \item \textbf{The current in a river.}
    \begin{itemize}
        \item \textbf{Answer:} Vector.
        \item \textbf{Reasoning:} A river current flows at a certain speed (magnitude) and heads downstream in a specific direction.
    \end{itemize}
    \item \textbf{The initial flight path from Houston to Dallas.}
    \begin{itemize}
        \item \textbf{Answer:} Vector.
        \item \textbf{Reasoning:} A flight path represents displacement from an origin to a destination, possessing both distance (magnitude) and direction.
    \end{itemize}
    \item \textbf{The population of the world.}
    \begin{itemize}
        \item \textbf{Answer:} Scalar.
        \item \textbf{Reasoning:} Population is a count (magnitude) with no associated direction.
    \end{itemize}
\end{enumerate}

\subsection*{Problem 2: Equal Vectors in a Parallelogram}
\textbf{Question:} Name all the equal vectors in the parallelogram shown. (Diagram shows vertices A, B, C, D going clockwise or counter-clockwise, with intersection E).
\textit{Note: Based on standard parallelogram vector properties:}
\begin{itemize}
    \item Opposite sides are parallel and equal in length.
    \item Diagonals bisect each other.
\end{itemize}

\textbf{Solution:}
\begin{enumerate}
    \item $\vec{AB}$ is equal to \textbf{$\vec{DC}$}. (Parallel, same length, same direction).
    \item $\vec{DA}$ is equal to \textbf{$\vec{CB}$}. (Parallel, same length, same direction).
    \item $\vec{DE}$ is equal to \textbf{$\vec{EB}$}. (Diagonals bisect; E is the midpoint, direction is D to B).
    \item $\vec{EA}$ is equal to \textbf{$\vec{CE}$}. (Diagonals bisect; E is midpoint, direction is C to A).
\end{enumerate}

\subsection*{Problem 3: Expressing Vectors using Midpoints}
\textbf{Question:} In the figure, the tip of $\mathbf{c}$ and the tail of $\mathbf{d}$ are both the midpoint of $QR$. Express $\mathbf{c}$ and $\mathbf{d}$ in terms of $\mathbf{a}$ and $\mathbf{b}$.
\textit{Setup:} Triangle $PQR$. $\mathbf{a} = \vec{QP}$, $\mathbf{b} = \vec{PR}$. The vector $\mathbf{a}+\mathbf{b} = \vec{QR}$.
Let $M$ be the midpoint of $QR$.
$\mathbf{c}$ starts at $Q$ and ends at $M$.
$\mathbf{d}$ starts at $M$ and ends at $R$.

\textbf{Solution:}
\begin{enumerate}
    \item Find $\vec{QR}$: Using the Triangle Law, $\vec{QP} + \vec{PR} = \vec{QR}$.
    \[ \vec{QR} = \mathbf{a} + \mathbf{b} \]
    \item Find $\mathbf{c}$: Vector $\mathbf{c}$ is the vector from $Q$ to the midpoint $M$. Since $M$ is the midpoint, $\vec{QM}$ is half the length of $\vec{QR}$ and in the same direction.
    \[ \mathbf{c} = \frac{1}{2}(\mathbf{a} + \mathbf{b}) \]
    \item Find $\mathbf{d}$: Vector $\mathbf{d}$ is the vector from $M$ to $R$. This is the second half of the segment $QR$. It is equal to $\mathbf{c}$.
    \[ \mathbf{d} = \frac{1}{2}(\mathbf{a} + \mathbf{b}) \]
\end{enumerate}

\subsection*{Problem 4: Finding Vector Representation (Standard Position)}
\textbf{Question:} Find a vector $\mathbf{a}$ with representation given by the directed line segment $\vec{AB}$.
$A(-3, 1)$, $B(5, 4)$. Draw $\vec{AB}$ and the equivalent representation starting at the origin.

\textbf{Solution:}
To find the component form, subtract the coordinates of the initial point $A$ from the terminal point $B$:
\[ \mathbf{a} = \langle x_B - x_A, y_B - y_A \rangle \]
\[ \mathbf{a} = \langle 5 - (-3), 4 - 1 \rangle \]
\[ \mathbf{a} = \langle 5 + 3, 3 \rangle \]
\[ \mathbf{a} = \langle 8, 3 \rangle \]

\textbf{Drawing:}
\begin{enumerate}
    \item $\vec{AB}$ starts at $(-3, 1)$ and goes to $(5, 4)$.
    \item The equivalent vector starts at $(0, 0)$ and goes to $(8, 3)$.
\end{enumerate}

\subsection*{Problem 5: Finding Vector Representation (Standard Position)}
\textbf{Question:} Find a vector $\mathbf{a}$ with representation given by the directed line segment $\vec{AB}$.
$A(1, 3)$, $B(0, 9)$.

\textbf{Solution:}
Subtract initial from terminal:
\[ \mathbf{a} = \langle 0 - 1, 9 - 3 \rangle \]
\[ \mathbf{a} = \langle -1, 6 \rangle \]

\textbf{Drawing:}
\begin{enumerate}
    \item $\vec{AB}$ starts at $(1, 3)$ and goes up/left to $(0, 9)$.
    \item The equivalent vector starts at $(0, 0)$ and goes to $(-1, 6)$.
\end{enumerate}

\subsection*{Problem 6: Vector Sum in 2D}
\textbf{Question:} Find the sum of the given vectors. $\mathbf{a} = \langle 2, -2 \rangle$, $\mathbf{b} = \langle -1, 8 \rangle$. Illustrate geometrically.

\textbf{Solution:}
Add corresponding components:
\[ \mathbf{a} + \mathbf{b} = \langle 2 + (-1), -2 + 8 \rangle \]
\[ \mathbf{a} + \mathbf{b} = \langle 1, 6 \rangle \]

\textbf{Geometric Illustration:}
Using the Triangle Law: Draw $\mathbf{a}$ starting at origin ending at $(2, -2)$. From that point, draw $\mathbf{b}$ (going left 1, up 8). The resultant vector connects the origin to $(1, 6)$.

\subsection*{Problem 7: Vector Sum in 3D}
\textbf{Question:} Find the sum of the given vectors $\mathbf{a} = \langle 1, 0, 4 \rangle$, $\mathbf{b} = \langle 0, 5, 0 \rangle$.
Geometric illustration details:
\begin{itemize}
    \item $\mathbf{a}$ starts at $(0,0,0)$ and ends at $(1,0,4)$.
    \item $\mathbf{b}$ starts at $(1,0,4)$ and ends at $(1+0, 0+5, 4+0) = (1,5,4)$.
    \item $\mathbf{a}+\mathbf{b}$ starts at $(0,0,0)$ and ends at $(1,5,4)$.
\end{itemize}

\textbf{Solution:}
\[ \mathbf{a} + \mathbf{b} = \langle 1+0, 0+5, 4+0 \rangle = \langle 1, 5, 4 \rangle \]

\subsection*{Problem 8: Vector Arithmetic Operations}
\textbf{Question:} Find $\mathbf{a} + \mathbf{b}$, $8\mathbf{a} + 9\mathbf{b}$, $|\mathbf{a}|$, and $|\mathbf{a} - \mathbf{b}|$.
Given: $\mathbf{a} = \langle -6, 8 \rangle$, $\mathbf{b} = \langle 6, 3 \rangle$.

\textbf{Solution:}
1. \textbf{$\mathbf{a} + \mathbf{b}$:}
\[ \langle -6 + 6, 8 + 3 \rangle = \langle 0, 11 \rangle \]

2. \textbf{$8\mathbf{a} + 9\mathbf{b}$:}
\[ 8\langle -6, 8 \rangle + 9\langle 6, 3 \rangle = \langle -48, 64 \rangle + \langle 54, 27 \rangle \]
\[ = \langle -48 + 54, 64 + 27 \rangle = \langle 6, 91 \rangle \]

3. \textbf{$|\mathbf{a}|$:}
\[ \sqrt{(-6)^2 + (8)^2} = \sqrt{36 + 64} = \sqrt{100} = 10 \]

4. \textbf{$|\mathbf{a} - \mathbf{b}|$:}
First find $\mathbf{a} - \mathbf{b}$:
\[ \langle -6 - 6, 8 - 3 \rangle = \langle -12, 5 \rangle \]
Now find magnitude:
\[ \sqrt{(-12)^2 + (5)^2} = \sqrt{144 + 25} = \sqrt{169} = 13 \]

\subsection*{Problem 9: Vector Arithmetic with Basis Vectors}
\textbf{Question:} Find $\mathbf{a} + \mathbf{b}$, $8\mathbf{a} + 4\mathbf{b}$, $|\mathbf{a}|$, $|\mathbf{a} - \mathbf{b}|$.
Given: $\mathbf{a} = 9\mathbf{i} - 6\mathbf{j} + 4\mathbf{k}$, $\mathbf{b} = 5\mathbf{i} - 8\mathbf{k}$.
(Note: $\mathbf{b}$ has $0\mathbf{j}$).

\textbf{Solution:}
1. \textbf{$\mathbf{a} + \mathbf{b}$:}
\[ (9+5)\mathbf{i} + (-6+0)\mathbf{j} + (4-8)\mathbf{k} = 14\mathbf{i} - 6\mathbf{j} - 4\mathbf{k} \]

2. \textbf{$8\mathbf{a} + 4\mathbf{b}$:}
\[ 8(9, -6, 4) + 4(5, 0, -8) \]
\[ = \langle 72, -48, 32 \rangle + \langle 20, 0, -32 \rangle \]
\[ = \langle 92, -48, 0 \rangle \quad \text{or} \quad 92\mathbf{i} - 48\mathbf{j} \]

3. \textbf{$|\mathbf{a}|$:}
\[ \sqrt{9^2 + (-6)^2 + 4^2} = \sqrt{81 + 36 + 16} = \sqrt{133} \]

4. \textbf{$|\mathbf{a} - \mathbf{b}|$:}
\[ \mathbf{a} - \mathbf{b} = \langle 9-5, -6-0, 4-(-8) \rangle = \langle 4, -6, 12 \rangle \]
\[ \text{Magnitude} = \sqrt{4^2 + (-6)^2 + 12^2} = \sqrt{16 + 36 + 144} = \sqrt{196} = 14 \]

\subsection*{Problem 10: Finding a Unit Vector}
\textbf{Question:} Find a unit vector in the direction of $\langle 15, -12 \rangle$.

\textbf{Solution:}
1. Find magnitude $|\mathbf{v}|$:
\[ |\mathbf{v}| = \sqrt{15^2 + (-12)^2} = \sqrt{225 + 144} = \sqrt{369} \]
Simplify $\sqrt{369} = \sqrt{9 \times 41} = 3\sqrt{41}$.

2. Divide vector by magnitude:
\[ \mathbf{u} = \frac{\langle 15, -12 \rangle}{3\sqrt{41}} = \left\langle \frac{15}{3\sqrt{41}}, \frac{-12}{3\sqrt{41}} \right\rangle = \left\langle \frac{5}{\sqrt{41}}, -\frac{4}{\sqrt{41}} \right\rangle \]

\subsection*{Problem 11: Finding a Unit Vector (3D)}
\textbf{Question:} Find a unit vector in direction of $-4\mathbf{i} + 2\mathbf{j} - \mathbf{k}$.

\textbf{Solution:}
1. Vector is $\mathbf{v} = \langle -4, 2, -1 \rangle$.
2. Magnitude:
\[ |\mathbf{v}| = \sqrt{(-4)^2 + 2^2 + (-1)^2} = \sqrt{16 + 4 + 1} = \sqrt{21} \]
3. Unit Vector:
\[ \mathbf{u} = \left\langle -\frac{4}{\sqrt{21}}, \frac{2}{\sqrt{21}}, -\frac{1}{\sqrt{21}} \right\rangle \]
Alternatively: $-\frac{4}{\sqrt{21}}\mathbf{i} + \frac{2}{\sqrt{21}}\mathbf{j} - \frac{1}{\sqrt{21}}\mathbf{k}$.

\subsection*{Problem 12: Angle with Positive x-axis}
\textbf{Question:} Find the angle between $\mathbf{v} = \mathbf{i} + \sqrt{13}\mathbf{j}$ and the positive x-axis. Round to nearest degree.

\textbf{Solution:}
The vector is $\langle 1, \sqrt{13} \rangle$. Since both components are positive, it is in Quadrant I.
Using trigonometry:
\[ \tan \theta = \frac{y}{x} = \frac{\sqrt{13}}{1} = \sqrt{13} \]
\[ \theta = \tan^{-1}(\sqrt{13}) \]
Using a calculator: $\sqrt{13} \approx 3.6055$.
$\theta \approx \tan^{-1}(3.6055) \approx 74.49^\circ$.
Rounded to nearest degree: \textbf{74$^\circ$}.

\subsection*{Problem 13: Finding Component Form from Magnitude and Angle}
\textbf{Question:} Initial point is origin, terminal point in Quadrant II. Angle with positive x-axis is $\frac{5\pi}{6}$. $|\mathbf{v}| = 8$. Find $\mathbf{v}$.

\textbf{Solution:}
The components are given by:
\[ x = |\mathbf{v}| \cos \theta \quad \text{and} \quad y = |\mathbf{v}| \sin \theta \]
\[ x = 8 \cos\left(\frac{5\pi}{6}\right) = 8 \left( -\frac{\sqrt{3}}{2} \right) = -4\sqrt{3} \]
\[ y = 8 \sin\left(\frac{5\pi}{6}\right) = 8 \left( \frac{1}{2} \right) = 4 \]
Check Quadrant II: $x$ is negative, $y$ is positive. Correct.
\[ \mathbf{v} = \langle -4\sqrt{3}, 4 \rangle \]

\subsection*{Problem 14: Tangent Unit Vectors}
\textbf{Question:} Find unit vectors parallel to the tangent line to the parabola $y = x^2$ at the point $(5, 25)$.
(Find smaller i-component and larger i-component).

\textbf{Solution:}
1. Find the slope of the tangent line.
\[ y' = 2x \]
At $x=5$, slope $m = 2(5) = 10$.
2. Convert slope to a vector.
A slope of $10 = \frac{10}{1}$ means for every 1 unit right, go 10 units up.
Vector $\mathbf{v} = \langle 1, 10 \rangle$.
3. Find magnitude: $|\mathbf{v}| = \sqrt{1^2 + 10^2} = \sqrt{101}$.
4. Unit vectors parallel to this line can go in two directions:
Direction 1 (Up/Right): $\mathbf{u}_1 = \langle \frac{1}{\sqrt{101}}, \frac{10}{\sqrt{101}} \rangle$.
Direction 2 (Down/Left): $\mathbf{u}_2 = \langle -\frac{1}{\sqrt{101}}, -\frac{10}{\sqrt{101}} \rangle$.

\textbf{Answers:}
Smaller $\mathbf{i}$-component (negative): $\langle -\frac{1}{\sqrt{101}}, -\frac{10}{\sqrt{101}} \rangle$.
Larger $\mathbf{i}$-component (positive): $\langle \frac{1}{\sqrt{101}}, \frac{10}{\sqrt{101}} \rangle$.

\subsection*{Problem 15: Triangle and Parallelogram Law (Conceptual)}
\textbf{Question:} Fill in the blanks for drawing vector addition.
\textbf{Solution:}
\begin{enumerate}
    \item "...starting at the \textbf{initial} point of $\mathbf{a}$..." (Triangle Law).
    \item "...place $\mathbf{b}$ so it starts where $\mathbf{a}$ \textbf{starts}..." (Parallelogram Law - tail to tail).
\end{enumerate}

\subsection*{Problem 16: Geometric Proof}
\textbf{Question:} Prove the line joining midpoints of two sides of a triangle is parallel to the third side and half its length.
Given Triangle $ABC$, $D$ midpoint of $AB$, $E$ midpoint of $BC$.

\textbf{Solution:}
\begin{enumerate}
    \item Equation (1): $\vec{AB} + \vec{BC} = \vec{AC}$. (Vector addition).
    \item Equation (2): $\vec{DB} + \vec{BE} = \vec{DE}$. (Vector addition in small triangle).
    \item Midpoint definitions:
    $\vec{DB} = \frac{1}{2}\vec{AB}$ and $\vec{BE} = \frac{1}{2}\vec{BC}$.
    \item Substitute into Equation (2):
    \[ \frac{1}{2}\vec{AB} + \frac{1}{2}\vec{BC} = \vec{DE} \]
    Factor out $1/2$:
    \[ \frac{1}{2}(\vec{AB} + \vec{BC}) = \vec{DE} \]
    Substitute $\vec{AC}$ from Equation (1):
    \[ \frac{1}{2}\vec{AC} = \vec{DE} \]
    \item Conclusion: Vectors are parallel (one is scalar multiple of other) and magnitude of $\vec{DE}$ is half of $\vec{AC}$.
\end{enumerate}

\newpage

\section{Part 3: In-Depth Analysis of Problems and Techniques}

\subsection{A) Problem Types and General Strategies}
\begin{itemize}
    \item \textbf{Setup \& Drawing (Q1, Q2, Q4, Q5):} These test the definition of vectors. Strategy: Always visualize vectors as arrows. Remember that vectors are translation invariant (you can move them as long as length and angle stay the same).
    \item \textbf{Component Calculation (Q3, Q6, Q7):} Converting geometric descriptions into numbers. Strategy: "Terminal minus Initial".
    \item \textbf{Vector Arithmetic (Q8, Q9):} Linear combinations ($c\mathbf{a} + d\mathbf{b}$). Strategy: Treat $\mathbf{i}, \mathbf{j}, \mathbf{k}$ like variables ($x, y, z$) and combine like terms.
    \item \textbf{Normalization (Q10, Q11, Q14):} Finding direction only. Strategy: Divide the vector by its own length.
    \item \textbf{Calculus Integration (Q14):} Linking derivatives to vectors. Strategy: Slope $m$ becomes vector $\langle 1, m \rangle$.
    \item \textbf{Geometric Proofs (Q16):} Using vector algebra to prove geometry theorems. Strategy: Express everything in terms of base vectors (sides of the triangle) and look for scalar multiples.
\end{itemize}

\subsection{B) Key Algebraic and Calculus Manipulations}
\begin{itemize}
    \item \textbf{Terminal Minus Initial:} The most used technique. If $A=(x_1, y_1)$ and $B=(x_2, y_2)$, $\vec{AB} = \langle x_2-x_1, y_2-y_1 \rangle$.
    \item \textbf{The Magnitude Formula (Pythagorean):} Used in Q8-11. Crucial for normalization. $|\mathbf{v}| = \sqrt{v_1^2 + v_2^2}$.
    \item \textbf{Slope-to-Vector Conversion:} In Q14, we converted $y'=10$ to $\langle 1, 10 \rangle$. This is a specific trick: A line with slope $m$ has a direction vector $\langle 1, m \rangle$ (or any multiple).
    \item \textbf{Quadrant Analysis:} In Q13, we found $x = -4\sqrt{3}, y=4$. We had to verify this matched "Quadrant II" (negative $x$, positive $y$).
\end{itemize}

\section{Part 4: Cheatsheet and Tips}

\subsection{Summary of Formulas}
\begin{itemize}
    \item \textbf{Vector $\vec{P_1P_2}$:} $\langle x_2-x_1, y_2-y_1, z_2-z_1 \rangle$
    \item \textbf{Magnitude:} $|\mathbf{a}| = \sqrt{a_1^2 + a_2^2 + a_3^2}$
    \item \textbf{Unit Vector:} $\mathbf{u} = \frac{\mathbf{a}}{|\mathbf{a}|}$
    \item \textbf{Components from Angle:} $x = |\mathbf{v}|\cos\theta, \quad y = |\mathbf{v}|\sin\theta$
\end{itemize}

\subsection{Tricks and Shortcuts}
\begin{itemize}
    \item \textbf{"Head-to-Tail":} To add vectors geometrically, place the tail of the second on the head of the first. The result goes from the very first tail to the very last head.
    \item \textbf{The "Slope Vector":} If you need a vector parallel to a line $y=mx+b$, use $\langle 1, m \rangle$.
    \item \textbf{Pythagorean Triples:} Watch for triples like (3, 4, 5) or (5, 12, 13) in magnitude problems (like Problem 10: 15, 12 $\to$ scaled 3-4-5). It saves calculation time.
\end{itemize}

\subsection{Common Pitfalls}
\begin{itemize}
    \item \textbf{Subtracting Backwards:} Calculating $\vec{AB}$ as $A - B$ instead of $B - A$. Remember: \textbf{End minus Start}.
    \item \textbf{Notation Errors:} Writing a vector as a number (e.g., $|\mathbf{v}| = \langle 3, 4 \rangle$ is wrong; magnitude is a number).
    \item \textbf{Calculator Mode:} For Q12 and Q13, ensure your calculator is in Degree or Radian mode as required by the question.
\end{itemize}

\newpage

\section{Part 5: Conceptual Synthesis and The "Big Picture"}

\subsection{Thematic Connections}
The central theme of this topic is \textbf{Linearization}. We are moving from single-variable calculus (lines and curves on a plane) to multi-variable space. Vectors provide the language to describe "straight" movement in 3D space. This connects to:
\begin{itemize}
    \item \textbf{Parametric Equations:} A line in 3D is just a starting point plus a scaled direction vector: $\mathbf{r}(t) = \mathbf{r}_0 + t\mathbf{v}$.
    \item \textbf{Physics:} Force, velocity, and acceleration are all vectors. Newton's laws are vector equations.
\end{itemize}

\subsection{Forward and Backward Links}
\begin{itemize}
    \item \textbf{Backward:} This builds directly on trigonometry (finding components) and analytic geometry (coordinates).
    \item \textbf{Forward:} This is the bedrock for \textbf{Partial Derivatives} (Calc III). Later, you will differentiate vector-valued functions, compute gradients (vectors of derivatives), and solve line integrals (work done by a vector field). You cannot do Calc III without fluent vector skills.
\end{itemize}

\section{Part 6: Real-World Application (Finance/Econ Focus)}

\subsection{A) Concrete Scenarios}
\begin{enumerate}
    \item \textbf{Portfolio Optimization (Finance):}
    An investment portfolio consists of various assets (stocks, bonds, options). We can represent the quantity of each asset held as a vector $\mathbf{h} = \langle h_1, h_2, \dots, h_n \rangle$. The current market prices are another vector $\mathbf{p} = \langle p_1, p_2, \dots, p_n \rangle$. The total value of the portfolio is the "dot product" (introduced next chapter) of these vectors. Vectors are essential for calculating risk (covariance matrices) and optimizing returns in Modern Portfolio Theory (Markowitz).

    \item \textbf{State Transition in Markov Chains (Economics):}
    Economists use Markov chains to model transitions between states (e.g., employed vs. unemployed). The population distribution is represented as a state vector $\mathbf{s}$. To predict the economy next year, we multiply this vector by a transition matrix. The concept of "steady state" involves finding a vector that doesn't change direction under this transformation (an eigenvector).
\end{enumerate}

\subsection{B) Model Problem Setup: Portfolio Value}
\textbf{Scenario:} You manage a fund with 3 tech stocks: Apple, Google, and Microsoft.
\textbf{Variables:}
\begin{itemize}
    \item Let vector $\mathbf{q} = \langle q_A, q_G, q_M \rangle$ represent the \textit{quantity} of shares owned for each company.
    \item Let vector $\mathbf{p} = \langle p_A, p_G, p_M \rangle$ represent the current \textit{price} per share for each.
\end{itemize}
\textbf{Equation:}
To find the total capital $V$ (a scalar), we perform the operation:
\[ V = q_A p_A + q_G p_G + q_M p_M \]
While this looks like simple algebra, treating these as vectors allows us to use calculus to calculate how the portfolio value changes if the price vector moves in a specific direction $\mathbf{d}$ (the directional derivative).

\newpage

\section{Part 7: Common Variations and Untested Concepts}

The homework focused heavily on basics. Here are standard concepts often found in Vector chapters that were missing:

\subsection{1. The Dot Product (Scalar Product)}
Often introduced in 12.2 or 12.3. It multiplies two vectors to get a scalar.
\textbf{Formula:} $\mathbf{a} \cdot \mathbf{b} = a_1b_1 + a_2b_2$.
\textbf{Use:} Determining if vectors are orthogonal (perpendicular). If $\mathbf{a} \cdot \mathbf{b} = 0$, they are perpendicular.

\subsection{2. Standard Basis Uniqueness}
Any vector in 2D can be uniquely expressed as $a\mathbf{i} + b\mathbf{j}$.
\textbf{Example:} Express $\langle 3, 5 \rangle$ using standard basis.
\textbf{Solution:} $3\mathbf{i} + 5\mathbf{j}$.

\subsection{3. Relative Velocity}
A classic application problem.
\textbf{Example:} A plane flies North at 200 mph. Wind blows East at 50 mph. Find the ground speed.
\textbf{Solution:} Add vectors $\langle 0, 200 \rangle + \langle 50, 0 \rangle = \langle 50, 200 \rangle$.
Speed $= \sqrt{50^2 + 200^2}$.

\section{Part 8: Advanced Diagnostic Testing: "Find the Flaw"}

\textbf{Instructions:} Below are 5 solutions containing specific errors. Your job is to find the error and correct it.

\subsection*{Flawed Problem 1}
\textbf{Problem:} Find vector $\vec{AB}$ from $A(2, 5)$ to $B(6, 1)$.
\textbf{Flawed Solution:}
\[ \vec{AB} = \langle 2-6, 5-1 \rangle = \langle -4, 4 \rangle \]
\textbf{Error:} Subtracted Initial minus Terminal.
\textbf{Correction:} Must be Terminal minus Initial. $\langle 6-2, 1-5 \rangle = \langle 4, -4 \rangle$.

\subsection*{Flawed Problem 2}
\textbf{Problem:} Find magnitude of $\mathbf{v} = \langle -3, 4 \rangle$.
\textbf{Flawed Solution:}
\[ |\mathbf{v}| = \sqrt{-3^2 + 4^2} = \sqrt{-9 + 16} = \sqrt{7} \]
\textbf{Error:} Square of a negative number is positive. Order of operations error in notation $(-3)^2$ vs $-3^2$.
\textbf{Correction:} $\sqrt{(-3)^2 + 4^2} = \sqrt{9+16} = \sqrt{25} = 5$.

\subsection*{Flawed Problem 3}
\textbf{Problem:} Find a unit vector for $\mathbf{v} = \langle 2, 2 \rangle$.
\textbf{Flawed Solution:}
Magnitude is $\sqrt{8}$. Unit vector is $\mathbf{u} = \sqrt{8} \langle 2, 2 \rangle = \langle 2\sqrt{8}, 2\sqrt{8} \rangle$.
\textbf{Error:} Multiplied by magnitude instead of dividing.
\textbf{Correction:} $\mathbf{u} = \frac{1}{\sqrt{8}} \langle 2, 2 \rangle = \langle \frac{2}{\sqrt{8}}, \frac{2}{\sqrt{8}} \rangle = \langle \frac{1}{\sqrt{2}}, \frac{1}{\sqrt{2}} \rangle$.

\subsection*{Flawed Problem 4}
\textbf{Problem:} $\mathbf{a} = \langle 1, 2 \rangle$, $\mathbf{b} = \langle 3, 4 \rangle$. Find $|\mathbf{a} + \mathbf{b}|$.
\textbf{Flawed Solution:}
\[ |\mathbf{a}| = \sqrt{5}, \quad |\mathbf{b}| = 5. \quad \text{Sum} = \sqrt{5} + 5. \]
\textbf{Error:} Assumed $|\mathbf{a} + \mathbf{b}| = |\mathbf{a}| + |\mathbf{b}|$. (Triangle Inequality violation).
\textbf{Correction:} Add vectors first: $\langle 4, 6 \rangle$. Magnitude is $\sqrt{16+36} = \sqrt{52}$.

\subsection*{Flawed Problem 5}
\textbf{Problem:} Find vector parallel to $y = 3x$.
\textbf{Flawed Solution:}
Slope is 3. Vector is $\langle 3, 1 \rangle$.
\textbf{Error:} Slope $m = \frac{\Delta y}{\Delta x} = \frac{3}{1}$. The y-component corresponds to the rise (3).
\textbf{Correction:} Vector is $\langle 1, 3 \rangle$.

\end{document}