\documentclass{article}
\usepackage[utf8]{inputenc}
\usepackage{amsmath}
\usepackage{amssymb}
\usepackage{geometry}
\usepackage{graphicx}
\usepackage{fancyhdr}
\usepackage{enumitem}
\usepackage{tcolorbox}

% Geometry settings
\geometry{a4paper, margin=1in}

% Header settings
\pagestyle{fancy}
\fancyhead[L]{Calculus III: 3D Coordinate Systems}
\fancyhead[R]{Homework 12.1 Solutions}

% Title Information
\title{Homework 12.1: Three-Dimensional Coordinate Systems}
\author{Tashfeen Omran}
\date{January 2026}

\begin{document}

\maketitle

\tableofcontents
\newpage

\section{Part 1: Comprehensive Introduction, Context, and Prerequisites}

\subsection{Core Concepts}
We exist in a three-dimensional world, yet until now, most of your calculus journey has taken place on a flat, two-dimensional piece of paper ($\mathbb{R}^2$). This topic extends Cartesian geometry into space ($\mathbb{R}^3$).

\begin{itemize}
    \item \textbf{The Coordinate Axes:} We introduce a third axis, the $z$-axis, which is perpendicular to both the $x$ and $y$ axes. The arrangement follows the \textbf{Right-Hand Rule}: if you curl the fingers of your right hand from the positive $x$-axis toward the positive $y$-axis, your thumb points in the direction of the positive $z$-axis.
    \item \textbf{Coordinate Planes:} The axes form three coordinate planes:
    \begin{itemize}
        \item The $xy$-plane (where $z = 0$).
        \item The $yz$-plane (where $x = 0$).
        \item The $xz$-plane (where $y = 0$).
    \end{itemize}
    These planes divide space into eight \textbf{octants}. The first octant is where $x, y,$ and $z$ are all positive.
    \item \textbf{Coordinates:} A point is represented as an ordered triple $(x, y, z)$.
\end{itemize}

\subsection{Intuition and Derivation: The Distance Formula}
In 2D, the distance between two points is derived from the Pythagorean Theorem: $d = \sqrt{(\Delta x)^2 + (\Delta y)^2}$.
In 3D, we apply the Pythagorean Theorem twice. First, to find the diagonal along the base (in the $xy$-plane), and second, to include the vertical height ($z$).
\[
d = \sqrt{(\Delta x)^2 + (\Delta y)^2 + (\Delta z)^2}
\]
This simply means that distance is the square root of the sum of the squared differences in each component.

\subsection{Historical Context}
The development of Analytic Geometry in 3D is largely credited to French mathematicians roughly in the 17th century, expanding on René Descartes' work. The motivation was to apply algebra to physical space—essential for physics, astronomy, and later, fluid dynamics. Without this framework, we could not describe planetary motion, electromagnetic fields, or financial volatility surfaces mathematically.

\subsection{Key Formulas}
\begin{enumerate}
    \item \textbf{Distance Formula:} Distance between $P_1(x_1, y_1, z_1)$ and $P_2(x_2, y_2, z_2)$:
    \[ |P_1 P_2| = \sqrt{(x_2-x_1)^2 + (y_2-y_1)^2 + (z_2-z_1)^2} \]
    \item \textbf{Equation of a Sphere:} With center $(h, k, l)$ and radius $r$:
    \[ (x-h)^2 + (y-k)^2 + (z-l)^2 = r^2 \]
    \item \textbf{Projections:}
    \begin{itemize}
        \item Projection onto $xy$-plane: $(x, y, 0)$
        \item Projection onto $yz$-plane: $(0, y, z)$
        \item Projection onto $xz$-plane: $(x, 0, z)$
    \end{itemize}
\end{enumerate}

\subsection{Prerequisites}
\begin{itemize}
    \item \textbf{Completing the Square:} Essential for converting general equations of spheres into standard form (e.g., transforming $x^2 + 6x$ into $(x+3)^2 - 9$).
    \item \textbf{Pythagorean Theorem:} The bedrock of distance calculations.
    \item \textbf{Basic Algebra:} Solving linear systems and manipulating exponents.
\end{itemize}

\newpage

\section{Part 2: Detailed Homework Solutions}

\subsection*{Problem 1 (SCalcET9 12.1.001)}
\textbf{Problem:} Suppose you start at the origin, move along the $x$-axis a distance of 2 units in the positive direction, and then move downward a distance of 9 units. What are the coordinates of your position?

\textbf{Solution:}
\begin{enumerate}
    \item \textbf{Start:} Origin $(0, 0, 0)$.
    \item \textbf{Move 1 (x-axis):} Move 2 units positive along $x$. $y$ and $z$ remain 0.
    \[ (0, 0, 0) \to (2, 0, 0) \]
    \item \textbf{Move 2 (Downward):} "Downward" corresponds to the negative $z$-direction. Move 9 units down.
    \[ z_{\text{new}} = 0 - 9 = -9 \]
    The $x$ and $y$ coordinates do not change during a vertical move.
    \[ (2, 0, 0) \to (2, 0, -9) \]
\end{enumerate}

\textbf{Answer:}
\[ (x, y, z) = (2, 0, -9) \]

\hrulefill

\subsection*{Problem 2 (SCalcET9 12.1.003.MI)}
\textbf{Problem:} Use the given points $A(-3, 0, -7)$, $B(4, 2, -5)$, $C(2, 3, 3)$.
\begin{itemize}
    \item Which of the points is closest to the $yz$-plane?
    \item Which point lies in the $xz$-plane?
\end{itemize}

\textbf{Solution:}
\textbf{Part A: Closest to the $yz$-plane}
The distance from a point $(x, y, z)$ to the $yz$-plane is simply the absolute value of the $x$-coordinate, $|x|$.
\begin{itemize}
    \item For $A(-3, 0, -7)$: Distance $= |-3| = 3$.
    \item For $B(4, 2, -5)$: Distance $= |4| = 4$.
    \item For $C(2, 3, 3)$: Distance $= |2| = 2$.
\end{itemize}
Comparing the distances $3, 4, 2$, the smallest is 2. Therefore, point $C$ is closest.

\textbf{Part B: Lies in the $xz$-plane}
A point lies in the $xz$-plane if and only if its $y$-coordinate is $0$.
\begin{itemize}
    \item $A$: $y = 0$. (Yes)
    \item $B$: $y = 2$. (No)
    \item $C$: $y = 3$. (No)
\end{itemize}
Therefore, point $A$ lies in the $xz$-plane.

\textbf{Answers:}
\begin{itemize}
    \item Closest to $yz$-plane: \textbf{C}
    \item Lies in $xz$-plane: \textbf{A}
\end{itemize}

\hrulefill

\subsection*{Problem 3 (SCalcET9 12.1.003.MI.SA)}
\textbf{Problem:} $A(8, 0, 1), B(8, 8, 5), C(2, 6, 6)$.
Which is closest to the $yz$-plane? Which lies in the $xz$-plane?

\textbf{Solution:}
Similar logic to Problem 2.
\textbf{Closest to $yz$-plane:} Calculate $|x|$ for each.
\begin{itemize}
    \item $A$: $|8| = 8$
    \item $B$: $|8| = 8$
    \item $C$: $|2| = 2$
\end{itemize}
$C$ has the smallest $|x|$ value.

\textbf{Lies in $xz$-plane:} Check for $y=0$.
\begin{itemize}
    \item $A$: $y=0$ (Yes)
    \item $B$: $y=8$ (No)
    \item $C$: $y=6$ (No)
\end{itemize}

\textbf{Answer:}
Closest to $yz$-plane: \textbf{Point C}. Point in $xz$-plane: \textbf{Point A}.

\hrulefill

\subsection*{Problem 4 (SCalcET9 12.1.004)}
\textbf{Problem:} Consider the point $(4, 5, 6)$.
\begin{enumerate}
    \item Projection on $xy$-plane?
    \item Projection on $yz$-plane?
    \item Projection on $xz$-plane?
    \item Draw a rectangular box.
\end{enumerate}

\textbf{Solution:}
To project a point onto a coordinate plane, set the coordinate perpendicular to that plane to 0.
\begin{enumerate}
    \item \textbf{$xy$-plane:} Set $z = 0$. Projection is $(4, 5, 0)$.
    \item \textbf{$yz$-plane:} Set $x = 0$. Projection is $(0, 5, 6)$.
    \item \textbf{$xz$-plane:} Set $y = 0$. Projection is $(4, 0, 6)$.
    \item \textbf{Box Diagonal:} The box has vertices at the origin and $(4,5,6)$. The length of the diagonal is the distance from $(0,0,0)$ to $(4,5,6)$.
    \[ d = \sqrt{4^2 + 5^2 + 6^2} = \sqrt{16 + 25 + 36} = \sqrt{77} \]
\end{enumerate}

\textbf{Answers:}
\begin{itemize}
    \item $(x, y, z) = (4, 5, 0)$
    \item $(x, y, z) = (0, 5, 6)$
    \item $(x, y, z) = (4, 0, 6)$
\end{itemize}

\hrulefill

\subsection*{Problem 5 (SCalcET9 12.1.006)}
\textbf{Problem:}
\begin{enumerate}
    \item What does $y=9$ represent in $\mathbb{R}^3$?
    \item What does $z=2$ represent?
    \item What does the pair $y=9, z=2$ represent?
\end{enumerate}

\textbf{Solution:}
\begin{enumerate}
    \item In $\mathbb{R}^3$, a single linear equation in one variable represents a \textbf{plane} perpendicular to that axis. $y=9$ is a plane parallel to the $xz$-plane, passing through $y=9$.
    \item Similarly, $z=2$ is a \textbf{plane} parallel to the $xy$-plane, passing through $z=2$.
    \item The intersection of two non-parallel planes is a \textbf{line}. The set of points where $y=9$ AND $z=2$ allows $x$ to vary freely. This is a line parallel to the $x$-axis.
\end{enumerate}

\textbf{Answers:}
\begin{itemize}
    \item $y=9$: \textbf{a plane}
    \item $z=2$: \textbf{a plane}
    \item Pair: \textbf{a line}
\end{itemize}

\hrulefill

\subsection*{Problem 6 (SCalcET9 12.1.010)}
\textbf{Problem:} Find the distance between $(-2, -6, 1)$ and $(5, 3, 4)$.

\textbf{Solution:}
Apply the distance formula:
\[ d = \sqrt{(x_2-x_1)^2 + (y_2-y_1)^2 + (z_2-z_1)^2} \]
Substitute the values:
\[ d = \sqrt{(5 - (-2))^2 + (3 - (-6))^2 + (4 - 1)^2} \]
\[ d = \sqrt{(7)^2 + (9)^2 + (3)^2} \]
\[ d = \sqrt{49 + 81 + 9} \]
\[ d = \sqrt{139} \]

\textbf{Answer:} $\sqrt{139}$

\hrulefill

\subsection*{Problem 7 (SCalcET9 12.1.011)}
\textbf{Problem:} Find lengths of sides of triangle PQR. $P(5, -2, 0), Q(9, 0, 4), R(11, -4, 0)$. Determine if it is a right triangle or isosceles.

\textbf{Solution:}
Calculate the lengths of the three sides: $|PQ|$, $|QR|$, $|RP|$.

\textbf{1. Length $|PQ|$:}
\[ |PQ| = \sqrt{(9-5)^2 + (0-(-2))^2 + (4-0)^2} = \sqrt{4^2 + 2^2 + 4^2} = \sqrt{16+4+16} = \sqrt{36} = 6 \]

\textbf{2. Length $|QR|$:}
\[ |QR| = \sqrt{(11-9)^2 + (-4-0)^2 + (0-4)^2} = \sqrt{2^2 + (-4)^2 + (-4)^2} = \sqrt{4+16+16} = \sqrt{36} = 6 \]

\textbf{3. Length $|RP|$:}
\[ |RP| = \sqrt{(5-11)^2 + (-2-(-4))^2 + (0-0)^2} = \sqrt{(-6)^2 + (2)^2 + 0} = \sqrt{36+4} = \sqrt{40} = 2\sqrt{10} \]

\textbf{Analysis:}
\begin{itemize}
    \item \textbf{Isosceles?} Yes, $|PQ| = |QR| = 6$.
    \item \textbf{Right Triangle?} Check Pythagorean theorem ($a^2 + b^2 = c^2$) where $c$ is the longest side ($|RP| = \sqrt{40}$).
    Sum of squares of shorter sides: $|PQ|^2 + |QR|^2 = 6^2 + 6^2 = 36 + 36 = 72$.
    Square of longest side: $|RP|^2 = 40$.
    $72 \neq 40$. Therefore, it is \textbf{not} a right triangle.
\end{itemize}

\textbf{Answers:}
\begin{itemize}
    \item $|PQ| = 6$
    \item $|QR| = 6$
    \item $|RP| = \sqrt{40}$ (or $2\sqrt{10}$)
    \item Right triangle? \textbf{No}
    \item Isosceles triangle? \textbf{Yes}
\end{itemize}

\hrulefill

\subsection*{Problem 8 (SCalcET9 12.1.014)}
\textbf{Problem:} Find distance from $(4, -3, 8)$ to:
(a) $xy$-plane, (b) $yz$-plane, (c) $xz$-plane, (d) $x$-axis, (e) $y$-axis, (f) $z$-axis.

\textbf{Solution:}
Let point $P = (x, y, z) = (4, -3, 8)$.

\textbf{Distances to Planes (drop the coordinate of the plane):}
\begin{itemize}
    \item (a) To $xy$-plane: $|z| = |8| = 8$.
    \item (b) To $yz$-plane: $|x| = |4| = 4$.
    \item (c) To $xz$-plane: $|y| = |-3| = 3$.
\end{itemize}

\textbf{Distances to Axes (Pythagorean of the other two):}
\begin{itemize}
    \item (d) To $x$-axis (distance to point $(4,0,0)$): $\sqrt{y^2 + z^2} = \sqrt{(-3)^2 + 8^2} = \sqrt{9 + 64} = \sqrt{73}$.
    \item (e) To $y$-axis (distance to point $(0,-3,0)$): $\sqrt{x^2 + z^2} = \sqrt{4^2 + 8^2} = \sqrt{16 + 64} = \sqrt{80} = 4\sqrt{5}$.
    \item (f) To $z$-axis (distance to point $(0,0,8)$): $\sqrt{x^2 + y^2} = \sqrt{4^2 + (-3)^2} = \sqrt{16 + 9} = \sqrt{25} = 5$.
\end{itemize}

\textbf{Answers:}
(a) 8, (b) 4, (c) 3, (d) $\sqrt{73}$, (e) $\sqrt{80}$, (f) 5.

\hrulefill

\subsection*{Problem 9 (SCalcET9 12.1.016)}
\textbf{Problem:} Sphere center $(3, -11, 8)$, radius 10.
1. Equation of sphere.
2. Intersection with coordinate planes (enter equation or DNE).

\textbf{Solution:}
\textbf{1. Sphere Equation:}
Formula: $(x-h)^2 + (y-k)^2 + (z-l)^2 = r^2$.
\[ (x-3)^2 + (y-(-11))^2 + (z-8)^2 = 10^2 \]
\[ (x-3)^2 + (y+11)^2 + (z-8)^2 = 100 \]

\textbf{2. Intersections:}
\begin{itemize}
    \item \textbf{$xy$-plane ($z=0$):} Substitute $z=0$ into the equation.
    \[ (x-3)^2 + (y+11)^2 + (0-8)^2 = 100 \]
    \[ (x-3)^2 + (y+11)^2 + 64 = 100 \]
    \[ (x-3)^2 + (y+11)^2 = 36 \]
    This is a circle.
    
    \item \textbf{$xz$-plane ($y=0$):} Substitute $y=0$.
    \[ (x-3)^2 + (0+11)^2 + (z-8)^2 = 100 \]
    \[ (x-3)^2 + 121 + (z-8)^2 = 100 \]
    \[ (x-3)^2 + (z-8)^2 = 100 - 121 = -21 \]
    A sum of squares cannot be negative. \textbf{DNE} (Does Not Exist/No intersection).

    \item \textbf{$yz$-plane ($x=0$):} Substitute $x=0$.
    \[ (0-3)^2 + (y+11)^2 + (z-8)^2 = 100 \]
    \[ 9 + (y+11)^2 + (z-8)^2 = 100 \]
    \[ (y+11)^2 + (z-8)^2 = 91 \]
    This is a circle.
\end{itemize}

\textbf{Answers:}
\begin{itemize}
    \item Eq: $(x-3)^2 + (y+11)^2 + (z-8)^2 = 100$
    \item $xy$: $(x-3)^2 + (y+11)^2 = 36$
    \item $xz$: DNE
    \item $yz$: $(y+11)^2 + (z-8)^2 = 91$
\end{itemize}

\hrulefill

\subsection*{Problem 10 (SCalcET9 12.1.017.MI)}
\textbf{Problem:} Equation of sphere passing through $(1, 8, 3)$ with center $(3, 3, -3)$.

\textbf{Solution:}
The radius $r$ is the distance between the center $C(3, 3, -3)$ and the point on the surface $P(1, 8, 3)$.
\[ r = \sqrt{(1-3)^2 + (8-3)^2 + (3-(-3))^2} \]
\[ r = \sqrt{(-2)^2 + (5)^2 + (6)^2} \]
\[ r = \sqrt{4 + 25 + 36} = \sqrt{65} \]
So $r^2 = 65$.
The equation is:
\[ (x-3)^2 + (y-3)^2 + (z+3)^2 = 65 \]

\textbf{Answer:}
$(x-3)^2 + (y-3)^2 + (z-(-3))^2 = 65$

\hrulefill

\subsection*{Problem 11 (SCalcET9 12.1.017.MI.SA)}
\textbf{Problem:} Sphere passes through $(6, 4, -2)$ with center $(4, 7, 4)$.

\textbf{Solution:}
Calculate radius squared ($r^2$):
\[ r^2 = (6-4)^2 + (4-7)^2 + (-2-4)^2 \]
\[ r^2 = (2)^2 + (-3)^2 + (-6)^2 \]
\[ r^2 = 4 + 9 + 36 = 49 \]
Equation:
\[ (x-4)^2 + (y-7)^2 + (z-4)^2 = 49 \]

\textbf{Answer:}
$(x-4)^2 + (y-7)^2 + (z-4)^2 = 49$

\hrulefill

\subsection*{Problem 12 (SCalcET9 12.1.020)}
\textbf{Problem:} $x^2 + y^2 + z^2 + 10x - 8y + 6z + 46 = 0$. Write in standard form. Find center and radius.

\textbf{Solution:}
Group variables and move the constant 46 to the right side:
\[ (x^2 + 10x) + (y^2 - 8y) + (z^2 + 6z) = -46 \]
Complete the square for each variable:
\begin{itemize}
    \item $x$: add $(10/2)^2 = 25$.
    \item $y$: add $(-8/2)^2 = 16$.
    \item $z$: add $(6/2)^2 = 9$.
\end{itemize}
Add these to \textbf{both} sides:
\[ (x^2 + 10x + 25) + (y^2 - 8y + 16) + (z^2 + 6z + 9) = -46 + 25 + 16 + 9 \]
\[ (x+5)^2 + (y-4)^2 + (z+3)^2 = -46 + 50 \]
\[ (x+5)^2 + (y-4)^2 + (z+3)^2 = 4 \]

\textbf{Analysis:}
Center $(h, k, l) = (-5, 4, -3)$.
Radius $r^2 = 4 \implies r = 2$.

\textbf{Answers:}
\begin{itemize}
    \item Standard Form: $(x+5)^2 + (y-4)^2 + (z+3)^2 = 4$
    \item Center: $(-5, 4, -3)$
    \item Radius: $2$
\end{itemize}

\hrulefill

\subsection*{Problem 13 (SCalcET9 12.1.027)}
\textbf{Problem:} Describe the region $z = -5$.

\textbf{Solution:}
The equation $z = -5$ fixes the $z$-coordinate but allows $x$ and $y$ to be anything. This creates a flat surface.
It is a plane parallel to the $xy$-plane (where $z=0$).
Since $-5$ is negative, it is 5 units \textbf{below} the $xy$-plane.

\textbf{Answer:}
The equation $z = -5$ represents a plane, parallel to the $xy$-plane and \textbf{5} units \textbf{below} it.

\hrulefill

\subsection*{Problem 14 (SCalcET9 12.1.033)}
\textbf{Problem:} Describe the region $x^2 + y^2 = 9, z = -1$.

\textbf{Solution:}
\begin{enumerate}
    \item $z = -1$ defines a specific plane parallel to the $xy$-plane.
    \item Inside that plane, $x^2 + y^2 = 9$ describes a circle with radius 3 centered at the $z$-axis (where $x=0, y=0$).
\end{enumerate}

\textbf{Answer:}
Because $z = -1$, all points in the region must lie in the \textbf{horizontal} plane $z = -1$. In addition, $x^2 + y^2 = 9$, so the region consists of all points that lie on \textbf{a circle} with radius \textbf{3} and center on the \textbf{z}-axis that is contained in the plane $z = -1$.

\hrulefill

\subsection*{Problem 15 (SCalcET9 12.1.034)}
\textbf{Problem:} Describe the region $x^2 + y^2 = 64$.

\textbf{Solution:}
The equation lacks the variable $z$. This means $z$ can be any real number.
In the $xy$-plane ($z=0$), this is a circle of radius $\sqrt{64} = 8$.
Since $z$ is free, we extend this circle infinitely up and down along the $z$-axis. This creates a cylinder.

\textbf{Answer:}
Here $x^2 + y^2 = 64$ with no restrictions on \textbf{z}, so a point in the region must lie on a circle of radius \textbf{8} and center on the \textbf{z}-axis, but it could be in any horizontal plane $z = k$ (parallel to the \textbf{xy}-plane). Thus, the region consists of all possible circles $x^2 + y^2 = 64, z = k$ and is therefore a \textbf{circular cylinder} with radius \textbf{8} whose axis is the \textbf{z}-axis.

\hrulefill

\subsection*{Problem 16 (SCalcET9 12.1.041)}
\textbf{Problem:} Describe $0 \le x \le 6, 0 \le y \le 6, 0 \le z \le 6$.

\textbf{Solution:}
These inequalities restrict all three variables to be between 0 and 6.
This describes a solid box (cube) in the first octant (where all coords are positive).

\textbf{Answer:}
The inequalities $0 \le x \le 6, 0 \le y \le 6, 0 \le z \le 6$ represent the set of all points in $\mathbb{R}^3$ that lie on or between the planes $x = \textbf{0}, x = \textbf{6}, y = \textbf{0}, y = \textbf{6}, z = \textbf{0}, z = \textbf{6}$ in the \textbf{first} octant.

\newpage

\section{Part 3: In-Depth Analysis}

\subsection{A) Problem Types and General Approach}
\begin{enumerate}
    \item \textbf{Coordinate Navigation (Q1):}
    \textit{Strategy:} Treat movements as simple addition/subtraction to the specific $(x, y, z)$ component involved. "Down" affects $z$, "West/East" affects $y$, etc.

    \item \textbf{Distance and Projections (Q2, Q3, Q4, Q6, Q8):}
    \textit{Strategy:}
    \begin{itemize}
        \item For distance between points: Use the 3D Distance Formula.
        \item For distance to a plane: Take the absolute value of the \textit{missing} variable (e.g., to $xz$-plane, use $|y|$).
        \item For distance to an axis: Use Pythagorean theorem on the \textit{other two} variables (e.g., to $x$-axis, use $\sqrt{y^2+z^2}$).
    \end{itemize}

    \item \textbf{Triangle Geometry (Q7):}
    \textit{Strategy:} Calculate all three side lengths first. Then check for Right ($a^2+b^2=c^2$) or Isosceles (two sides equal) properties.

    \item \textbf{Sphere Equations (Q9, Q10, Q11, Q12):}
    \textit{Strategy:}
    \begin{itemize}
        \item If Center/Radius given: Plug directly into $(x-h)^2 + ... = r^2$.
        \item If Center/Point given: Calc distance first to find $r$, then plug in.
        \item If General Equation given: Use \textbf{Completing the Square} to find standard form.
    \end{itemize}

    \item \textbf{Surface Identification (Q5, Q13, Q14, Q15):}
    \textit{Strategy:} Look for missing variables.
    \begin{itemize}
        \item Missing 2 variables (e.g., $z=5$) $\to$ Plane.
        \item Missing 1 variable (e.g., $x^2+y^2=9$) $\to$ Cylinder.
    \end{itemize}
\end{enumerate}

\subsection{B) Key Manipulations}
\textbf{Completing the Square (Problem 12):}
This was the most advanced algebraic manipulation in the set.
To transform $x^2 + 10x$ into a perfect square, we take the coefficient of $x$ (which is 10), divide by 2 (getting 5), and square it (getting 25). We add 25 to both sides.
\textit{Why necessary?} The general form $x^2+y^2+z^2+Ax+By+Cz+D=0$ hides the center and radius. The standard form $(x-h)^2...=r^2$ reveals them instantly.

\newpage

\section{Part 4: Cheatsheet and Tips}

\begin{tcolorbox}[title=Essential Formulas]
\begin{itemize}
    \item \textbf{Distance:} $d = \sqrt{\Delta x^2 + \Delta y^2 + \Delta z^2}$
    \item \textbf{Sphere:} $(x-h)^2 + (y-k)^2 + (z-l)^2 = r^2$
    \item \textbf{Completing the Square:} $x^2 + bx \to (x + \frac{b}{2})^2 - (\frac{b}{2})^2$
\end{itemize}
\end{tcolorbox}

\begin{tcolorbox}[title=Shortcuts & Tricks]
\begin{itemize}
    \item \textbf{Distance to Plane:} Just grab the absolute value of the letter \textit{not} in the plane's name.
    \item \textbf{Distance to Axis:} Square root of the sum of squares of the letters \textit{not} in the axis name.
    \item \textbf{Is it a Cylinder?} If an equation has $x^2$ and $y^2$ (or other pairs) but is completely missing the third variable, it is a cylinder extending along the missing axis.
\end{itemize}
\end{tcolorbox}

\begin{tcolorbox}[title=Common Pitfalls]
\begin{itemize}
    \item \textbf{The "Axis Distance" Trap:} When asked for distance to the $x$-axis, students often just write $|x|$. NO! That is the distance along the axis. Distance \textit{to} the axis is $\sqrt{y^2+z^2}$.
    \item \textbf{Intersection DNE:} When finding the intersection of a sphere and a plane, if you end up with $(something)^2 = \text{negative number}$, the intersection does not exist.
\end{itemize}
\end{tcolorbox}

\newpage

\section{Part 5: Conceptual Synthesis}

\subsection{A) Thematic Connections}
The core theme of this topic is \textbf{Dimensional Extension}. Just as Calculus I and II taught you to analyze curves in 2D, Calculus III generalizes these concepts to surfaces and solids in 3D. The "Free Variable" concept (e.g., $z$ is missing in $x^2+y^2=1$, so $z$ is free) is the fundamental building block for understanding cylinders and, later, partial derivatives (where we hold variables constant).

\subsection{B) Forward and Backward Links}
\begin{itemize}
    \item \textbf{Backward:} This relies heavily on the Pythagorean Theorem and Circle Equations from Pre-Calculus.
    \item \textbf{Forward:}
    \begin{itemize}
        \item \textbf{Vectors (12.2):} Points will become vectors. $(x,y,z)$ becomes $\langle x,y,z \rangle$.
        \item \textbf{Quadric Surfaces (12.6):} We will move beyond spheres and planes to paraboloids and hyperboloids.
        \item \textbf{Multiple Integration (Ch 15):} You will eventually integrate functions $f(x,y,z)$ over the 3D regions (boxes and spheres) you just learned to describe.
    \end{itemize}
\end{itemize}

\section{Part 6: Real-World Application and Modeling (Finance Focus)}

\subsection{A) Concrete Scenarios}
\begin{enumerate}
    \item \textbf{Portfolio Optimization (Finance):} In Modern Portfolio Theory, a portfolio consisting of three assets can be represented as a point $(w_1, w_2, w_3)$ in 3D space, where $w_i$ is the weight of asset $i$. The constraint that capital is fully invested is the plane equation $w_1 + w_2 + w_3 = 1$. Finding the "minimum variance" portfolio involves finding the point on this plane closest to the origin (adjusted for the covariance metric).
    \item \textbf{Volatility Surfaces (Derivatives Pricing):} In options trading, implied volatility is often plotted against strike price ($x$) and time to maturity ($y$). The resulting volatility ($\sigma = z$) creates a 3D surface known as the "volatility smile" or surface. Traders look for mispricing by analyzing the distance of current market data points from this theoretical surface.
\end{enumerate}

\subsection{B) Model Problem Setup}
\textbf{Scenario:} A quantitative analyst is modeling a 3-asset portfolio.
\textbf{Variables:} Let $x, y, z$ be the dollar amounts invested in Asset A, Asset B, and Asset C.
\textbf{Constraint:} The total budget is \$10,000.
\textbf{Equation:} The set of all possible portfolios lies on the plane:
\[ x + y + z = 10,000 \]
\textbf{Optimization:} To minimize risk (simplified as minimizing the sum of squares of investments), the analyst must minimize distance from the origin to this plane.
\textbf{Math Task:} Minimize $d = \sqrt{x^2+y^2+z^2}$ subject to $x+y+z=10000$.

\newpage

\section{Part 7: Common Variations and Untested Concepts}

Your homework covered the basics well, but omitted two critical standard concepts:

\subsection{1. The Midpoint Formula}
\textbf{Concept:} Finding the exact center between two points $P_1$ and $P_2$.
\textbf{Formula:} $M = \left( \frac{x_1+x_2}{2}, \frac{y_1+y_2}{2}, \frac{z_1+z_2}{2} \right)$.
\textbf{Example:} Find midpoint of $(2, 4, 0)$ and $(4, 8, 10)$.
\textbf{Solution:} $M = (\frac{2+4}{2}, \frac{4+8}{2}, \frac{0+10}{2}) = (3, 6, 5)$.

\subsection{2. Equation of a Sphere given Diameter Endpoints}
\textbf{Concept:} If you are given the endpoints of a diameter, the Center is the Midpoint, and the Radius is half the distance (or distance from Center to one end).
\textbf{Example:} Sphere with diameter endpoints $A(0,0,0)$ and $B(2,0,0)$.
\textbf{Solution:}
Center = Midpoint = $(1, 0, 0)$.
Radius = Distance from $(1,0,0)$ to $(0,0,0) = 1$.
Equation: $(x-1)^2 + y^2 + z^2 = 1$.

\section{Part 8: Advanced Diagnostic Testing: "Find the Flaw"}

\textbf{Instructions:} Below are 5 solutions containing a single critical error. Find it.

\subsection*{Problem A}
\textbf{Task:} Find distance from $P(3, 4, 12)$ to the $x$-axis.
\textbf{Flawed Solution:}
The distance to the $x$-axis is simply the $x$-coordinate.
$d = |3| = 3$.
\textbf{Final Answer:} 3.
\begin{itemize}
    \item \textbf{Error:} The student calculated distance to the $yz$-plane, not the $x$-axis.
    \item \textbf{Correction:} Distance to $x$-axis is $\sqrt{y^2+z^2} = \sqrt{4^2+12^2} = \sqrt{16+144} = \sqrt{160}$.
\end{itemize}

\subsection*{Problem B}
\textbf{Task:} Find center of sphere $x^2 + y^2 + z^2 - 6x = 0$.
\textbf{Flawed Solution:}
Complete the square for $x$.
$(x-3)^2 + y^2 + z^2 = 0$.
Center is $(3, 0, 0)$. Radius is 0.
\textbf{Final Answer:} Center $(3,0,0)$.
\begin{itemize}
    \item \textbf{Error:} When adding $(-3)^2=9$ to the left side to complete the square, the student failed to add 9 to the right side.
    \item \textbf{Correction:} $(x-3)^2 + y^2 + z^2 = 9$. Radius is 3.
\end{itemize}

\subsection*{Problem C}
\textbf{Task:} Describe $x^2 + z^2 = 4$.
\textbf{Flawed Solution:}
This is a circle of radius 2 centered at the origin in the $xz$-plane.
\textbf{Final Answer:} A circle.
\begin{itemize}
    \item \textbf{Error:} In $\mathbb{R}^3$, if a variable ($y$) is missing, it is a surface (cylinder), not a curve (circle).
    \item \textbf{Correction:} It is a circular cylinder opening along the $y$-axis.
\end{itemize}

\subsection*{Problem D}
\textbf{Task:} Find projection of $(2, 3, 5)$ onto the $xy$-plane.
\textbf{Flawed Solution:}
To project onto $xy$, we set $x$ and $y$ to 0.
Projection is $(0, 0, 5)$.
\textbf{Final Answer:} $(0, 0, 5)$.
\begin{itemize}
    \item \textbf{Error:} The student set the wrong variables to zero; they projected onto the $z$-axis.
    \item \textbf{Correction:} Set $z=0$. Projection is $(2, 3, 0)$.
\end{itemize}

\subsection*{Problem E}
\textbf{Task:} Intersection of sphere $(x)^2 + y^2 + z^2 = 1$ and plane $z = 2$.
\textbf{Flawed Solution:}
Substitute $z=2$: $x^2 + y^2 + 4 = 1$.
$x^2 + y^2 = -3$.
This is a circle with radius $\sqrt{-3}$.
\textbf{Final Answer:} Circle with imaginary radius.
\begin{itemize}
    \item \textbf{Error:} A squared radius cannot be negative in real coordinate geometry; the student failed to interpret the result as "No Intersection."
    \item \textbf{Correction:} The intersection is the empty set (DNE).
\end{itemize}

\end{document}