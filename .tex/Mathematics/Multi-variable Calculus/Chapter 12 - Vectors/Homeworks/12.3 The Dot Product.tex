\documentclass[11pt]{article}
\usepackage[utf8]{inputenc}
\usepackage{geometry}
\geometry{a4paper, margin=1in}
\usepackage{amsmath}
\usepackage{amssymb}
\usepackage{amsthm}
\usepackage{graphicx}
\usepackage{enumitem}
\usepackage{fancyhdr}
\usepackage{xcolor}

% Header/Footer Setup
\pagestyle{fancy}
\fancyhf{}
\lhead{Homework 12.3: The Dot Product}
\rhead{Tashfeen Omran}
\cfoot{\thepage}

% Custom Environments
\newtheorem*{theorem}{Theorem}
\theoremstyle{definition}
\newtheorem{problem}{Problem}
\newtheorem*{solution}{Solution}

\title{\textbf{Homework 12.3: The Dot Product}\\ \large Comprehensive Study Guide and Solutions}
\author{Tashfeen Omran}
\date{February 2026}

\begin{document}

\maketitle
\tableofcontents
\newpage

\section{Part 1: Comprehensive Introduction, Context, and Prerequisites}

\subsection{Core Concepts}
The \textbf{Dot Product} (also known as the scalar product) is a fundamental operation that takes two vectors and returns a single scalar (a real number). Unlike vector addition, which results in a new vector, the dot product collapses information about magnitude and direction into a single value.

Algebraically, for two vectors $\mathbf{a} = \langle a_1, a_2, a_3 \rangle$ and $\mathbf{b} = \langle b_1, b_2, b_3 \rangle$, the dot product is the sum of the products of their corresponding components:
\[ \mathbf{a} \cdot \mathbf{b} = a_1b_1 + a_2b_2 + a_3b_3 \]

Geometrically, the dot product relates the magnitudes of the vectors and the cosine of the angle $\theta$ between them:
\[ \mathbf{a} \cdot \mathbf{b} = |\mathbf{a}| |\mathbf{b}| \cos \theta \]

This geometric definition allows us to calculate angles between vectors in $3$D space—a task that is incredibly difficult using classical geometry alone.

\subsection{Intuition and Derivation}
Why does the formula $\mathbf{a} \cdot \mathbf{b} = a_1b_1 + a_2b_2$ relate to the cosine of the angle?
The derivation comes directly from the \textbf{Law of Cosines}. Consider a triangle formed by vectors $\mathbf{a}$, $\mathbf{b}$, and their difference $\mathbf{a} - \mathbf{b}$. The Law of Cosines states:
\[ |\mathbf{a} - \mathbf{b}|^2 = |\mathbf{a}|^2 + |\mathbf{b}|^2 - 2|\mathbf{a}||\mathbf{b}|\cos\theta \]
By expanding the left side using the distance formula (algebraic definition of magnitude) and simplifying, the terms cancel out nicely to reveal that the remaining term $|\mathbf{a}||\mathbf{b}|\cos\theta$ is exactly equal to the component-wise sum $a_1b_1 + a_2b_2$. This bridges the gap between coordinate geometry (algebra) and Euclidean geometry (angles and lengths).

\subsection{Historical Context and Motivation}
In the 19th century, mathematicians like William Rowan Hamilton and Hermann Grassmann were attempting to generalize complex numbers to three dimensions. Hamilton developed "Quaternions," which included both a scalar part and a vector part. The multiplication of quaternions involved a "scalar part" (which evolved into the dot product, representing geometric similarity) and a "vector part" (which evolved into the cross product). Later, physicists Josiah Willard Gibbs and Oliver Heaviside stripped away the complex quaternion framework to create the modern vector analysis we use today, specifically to solve problems in electromagnetism and mechanics where "work done" and "flux" required projecting one vector onto another.

\subsection{Key Formulas}
\begin{enumerate}
    \item \textbf{Algebraic Definition (3D):} \[ \mathbf{a} \cdot \mathbf{b} = a_1b_1 + a_2b_2 + a_3b_3 \]
    \item \textbf{Geometric Definition:} \[ \mathbf{a} \cdot \mathbf{b} = |\mathbf{a}||\mathbf{b}|\cos\theta \]
    \item \textbf{Angle Between Vectors:} \[ \cos\theta = \frac{\mathbf{a} \cdot \mathbf{b}}{|\mathbf{a}||\mathbf{b}|} \]
    \item \textbf{Orthogonality Condition:} Vectors $\mathbf{a}$ and $\mathbf{b}$ are orthogonal (perpendicular) if and only if $\mathbf{a} \cdot \mathbf{b} = 0$.
    \item \textbf{Scalar Projection of $\mathbf{b}$ onto $\mathbf{a}$ (Component):} \[ \text{comp}_{\mathbf{a}}\mathbf{b} = \frac{\mathbf{a} \cdot \mathbf{b}}{|\mathbf{a}|} \]
    \item \textbf{Vector Projection of $\mathbf{b}$ onto $\mathbf{a}$:} \[ \text{proj}_{\mathbf{a}}\mathbf{b} = \left( \frac{\mathbf{a} \cdot \mathbf{b}}{|\mathbf{a}|^2} \right) \mathbf{a} \]
\end{enumerate}

\subsection{Prerequisites}
To master this topic, you must be proficient in:
\begin{itemize}
    \item \textbf{Vector Magnitude:} Calculating $|\mathbf{v}| = \sqrt{v_1^2 + v_2^2 + v_3^2}$.
    \item \textbf{Trigonometry:} Specifically the values of cosine for standard angles ($30^{\circ}, 45^{\circ}, 60^{\circ}, 90^{\circ}$) and the range of the arccos function $[0, \pi]$.
    \item \textbf{Algebra:} Distributive property (FOILing) with variables.
\end{itemize}

\newpage
\section{Part 2: Detailed Homework Solutions}

\begin{problem}[1]
Which of the following expressions are meaningful? Which are meaningless? Explain.
\end{problem}
\begin{solution}
\textbf{(a)} $(\mathbf{a} \cdot \mathbf{b}) \cdot \mathbf{c}$
\begin{itemize}
    \item Analysis: $\mathbf{a} \cdot \mathbf{b}$ results in a scalar. The operation then becomes (Scalar) $\cdot$ (Vector). The dot product is only defined between two vectors.
    \item Conclusion: Meaningless.
    \item Fill-in: The expression is meaningless because it is the dot product of a scalar and a vector.
\end{itemize}

\textbf{(b)} $(\mathbf{a} \cdot \mathbf{b})\mathbf{c}$
\begin{itemize}
    \item Analysis: $\mathbf{a} \cdot \mathbf{b}$ is a scalar. A scalar times a vector is standard scalar multiplication.
    \item Conclusion: Meaningful.
    \item Fill-in: The expression is meaningful because it is a scalar multiple of a vector.
\end{itemize}

\textbf{(c)} $|\mathbf{a}|(\mathbf{b} \cdot \mathbf{c})$
\begin{itemize}
    \item Analysis: $|\mathbf{a}|$ is a scalar. $\mathbf{b} \cdot \mathbf{c}$ is a scalar. The product of two real numbers is valid.
    \item Conclusion: Meaningful.
    \item Fill-in: The expression is meaningful because it is the product of two scalars.
\end{itemize}

\textbf{(d)} $\mathbf{a} \cdot (\mathbf{b} + \mathbf{c})$
\begin{itemize}
    \item Analysis: $\mathbf{b} + \mathbf{c}$ is a vector. $\mathbf{a}$ is a vector. The dot product of two vectors is valid.
    \item Conclusion: Meaningful.
    \item Fill-in: The expression is meaningful because it is the dot product of two vectors.
\end{itemize}

\textbf{(e)} $\mathbf{a} \cdot \mathbf{b} + \mathbf{c}$
\begin{itemize}
    \item Analysis: $\mathbf{a} \cdot \mathbf{b}$ is a scalar. $\mathbf{c}$ is a vector. You cannot add a scalar to a vector.
    \item Conclusion: Meaningless.
    \item Fill-in: The expression is meaningless because it is the sum of a scalar and a vector.
\end{itemize}

\textbf{(f)} $|\mathbf{a}| \cdot (\mathbf{b} + \mathbf{c})$
\begin{itemize}
    \item Analysis: $|\mathbf{a}|$ is a scalar. $(\mathbf{b} + \mathbf{c})$ is a vector. The dot ($\cdot$) symbol usually denotes a dot product between two vectors. If interpreted as scalar multiplication, it is valid, but the notation specifically asks about dot product validity. In vector calculus context, "dot" implies dot product.
    \item Conclusion: Meaningless (as a dot product).
    \item Fill-in: The expression is meaningless because it is the dot product of a scalar and a vector.
\end{itemize}
\end{solution}

\begin{problem}[2]
Find $\mathbf{a} \cdot \mathbf{b}$. \\
$\mathbf{a} = \langle 9, -6 \rangle, \mathbf{b} = \langle 7, 8 \rangle$
\end{problem}
\begin{solution}
Formula: $\mathbf{a} \cdot \mathbf{b} = a_1b_1 + a_2b_2$
\[ \mathbf{a} \cdot \mathbf{b} = (9)(7) + (-6)(8) \]
\[ \mathbf{a} \cdot \mathbf{b} = 63 - 48 \]
\[ \mathbf{a} \cdot \mathbf{b} = 15 \]
\textbf{Answer:} $15$
\end{solution}

\begin{problem}[3]
Find $\mathbf{a} \cdot \mathbf{b}$. \\
$\mathbf{a} = \langle 6, -5, 2 \rangle, \mathbf{b} = \langle 5, 6, -1 \rangle$
\end{problem}
\begin{solution}
Formula: $\mathbf{a} \cdot \mathbf{b} = a_1b_1 + a_2b_2 + a_3b_3$
\[ \mathbf{a} \cdot \mathbf{b} = (6)(5) + (-5)(6) + (2)(-1) \]
\[ \mathbf{a} \cdot \mathbf{b} = 30 - 30 - 2 \]
\[ \mathbf{a} \cdot \mathbf{b} = -2 \]
\textbf{Answer:} $-2$
\end{solution}

\begin{problem}[4]
Find $\mathbf{a} \cdot \mathbf{b}$. \\
$\mathbf{a} = \langle p, -p, 9p \rangle, \mathbf{b} = \langle 4q, q, -q \rangle$
\end{problem}
\begin{solution}
\[ \mathbf{a} \cdot \mathbf{b} = (p)(4q) + (-p)(q) + (9p)(-q) \]
\[ \mathbf{a} \cdot \mathbf{b} = 4pq - pq - 9pq \]
Combine like terms:
\[ \mathbf{a} \cdot \mathbf{b} = -6pq \]
\textbf{Answer:} $-6pq$
\end{solution}

\begin{problem}[5]
Find $\mathbf{a} \cdot \mathbf{b}$. \\
$\mathbf{a} = 5\mathbf{i} + \mathbf{j}, \mathbf{b} = \mathbf{i} - 5\mathbf{j} + \mathbf{k}$
\end{problem}
\begin{solution}
First, write the vectors in component form:
$\mathbf{a} = \langle 5, 1, 0 \rangle$ (Note: the $\mathbf{k}$ component is 0)
$\mathbf{b} = \langle 1, -5, 1 \rangle$
\[ \mathbf{a} \cdot \mathbf{b} = (5)(1) + (1)(-5) + (0)(1) \]
\[ \mathbf{a} \cdot \mathbf{b} = 5 - 5 + 0 \]
\[ \mathbf{a} \cdot \mathbf{b} = 0 \]
\textbf{Answer:} $0$
\end{solution}

\begin{problem}[6]
Find $\mathbf{a} \cdot \mathbf{b}$. \\
$|\mathbf{a}| = 4, |\mathbf{b}| = 8$, the angle between $\mathbf{a}$ and $\mathbf{b}$ is $30^{\circ}$.
\end{problem}
\begin{solution}
Formula: $\mathbf{a} \cdot \mathbf{b} = |\mathbf{a}||\mathbf{b}|\cos\theta$
\[ \mathbf{a} \cdot \mathbf{b} = (4)(8)\cos(30^{\circ}) \]
\[ \mathbf{a} \cdot \mathbf{b} = 32 \left( \frac{\sqrt{3}}{2} \right) \]
\[ \mathbf{a} \cdot \mathbf{b} = 16\sqrt{3} \]
\textbf{Answer:} $16\sqrt{3}$
\end{solution}

\begin{problem}[7]
If $\mathbf{u}$ is a unit vector, find $\mathbf{u} \cdot \mathbf{v}$ and $\mathbf{u} \cdot \mathbf{w}$. (Assume $\mathbf{v}$ and $\mathbf{w}$ are also unit vectors.)
\textit{Image Description: An equilateral triangle formed by vectors $\mathbf{u}, \mathbf{v}, \mathbf{w}$. $\mathbf{u}$ goes up the left side. $\mathbf{v}$ goes down the right side. $\mathbf{w}$ goes left across the bottom. The vectors form a closed cycle.}
\end{problem}
\begin{solution}
Since $\mathbf{u}, \mathbf{v}, \mathbf{w}$ form an equilateral triangle, the interior angles are all $60^{\circ}$.
To find the dot product $\mathbf{u} \cdot \mathbf{v}$, we need the angle $\theta$ between them when they are placed \textbf{tail-to-tail}.
In the diagram, the head of $\mathbf{u}$ meets the tail of $\mathbf{v}$. The interior angle is $60^{\circ}$. If we extend $\mathbf{u}$ straight up to place the tails together, the angle between them is the supplementary angle:
\[ \theta = 180^{\circ} - 60^{\circ} = 120^{\circ} \]
Since they are unit vectors, magnitudes are $1$.
\[ \mathbf{u} \cdot \mathbf{v} = (1)(1)\cos(120^{\circ}) = -\frac{1}{2} \]

Now for $\mathbf{u} \cdot \mathbf{w}$. The tail of $\mathbf{u}$ meets the head of $\mathbf{w}$. If we slide $\mathbf{w}$ so its tail meets $\mathbf{u}$'s tail, the angle is again the exterior angle:
\[ \theta = 180^{\circ} - 60^{\circ} = 120^{\circ} \]
\[ \mathbf{u} \cdot \mathbf{w} = (1)(1)\cos(120^{\circ}) = -\frac{1}{2} \]

\textbf{Answer:}
$\mathbf{u} \cdot \mathbf{v} = -0.5$ \\
$\mathbf{u} \cdot \mathbf{w} = -0.5$
\end{solution}

\begin{problem}[8]
Find the angle between the vectors. (Exact and approximate). \\
$\mathbf{a} = \mathbf{i} - 5\mathbf{j} = \langle 1, -5 \rangle$ \\
$\mathbf{b} = -5\mathbf{i} + 12\mathbf{j} = \langle -5, 12 \rangle$
\end{problem}
\begin{solution}
Step 1: Find $\mathbf{a} \cdot \mathbf{b}$.
\[ \mathbf{a} \cdot \mathbf{b} = (1)(-5) + (-5)(12) = -5 - 60 = -65 \]
Step 2: Find magnitudes.
\[ |\mathbf{a}| = \sqrt{1^2 + (-5)^2} = \sqrt{1 + 25} = \sqrt{26} \]
\[ |\mathbf{b}| = \sqrt{(-5)^2 + 12^2} = \sqrt{25 + 144} = \sqrt{169} = 13 \]
Step 3: Use cosine formula.
\[ \cos\theta = \frac{\mathbf{a} \cdot \mathbf{b}}{|\mathbf{a}||\mathbf{b}|} = \frac{-65}{13\sqrt{26}} \]
Reduce the fraction:
\[ \cos\theta = \frac{-5}{\sqrt{26}} \]
Exact angle: $\theta = \arccos\left(-\frac{5}{\sqrt{26}}\right)$
Approximate: $\theta \approx 168.69^{\circ}$. The question asks for nearest degree.
\textbf{Answer:}
Exact: $\arccos\left(-\frac{5}{\sqrt{26}}\right)$
Approximate: $169^{\circ}$
\end{solution}

\begin{problem}[9]
Find the angle between the vectors. (Exact and approximate). \\
$\mathbf{a} = \langle -1, 4, 5 \rangle, \mathbf{b} = \langle 3, 7, 1 \rangle$
\end{problem}
\begin{solution}
Step 1: Find $\mathbf{a} \cdot \mathbf{b}$.
\[ \mathbf{a} \cdot \mathbf{b} = (-1)(3) + (4)(7) + (5)(1) = -3 + 28 + 5 = 30 \]
Step 2: Find magnitudes.
\[ |\mathbf{a}| = \sqrt{(-1)^2 + 4^2 + 5^2} = \sqrt{1 + 16 + 25} = \sqrt{42} \]
\[ |\mathbf{b}| = \sqrt{3^2 + 7^2 + 1^2} = \sqrt{9 + 49 + 1} = \sqrt{59} \]
Step 3: Use cosine formula.
\[ \cos\theta = \frac{30}{\sqrt{42}\sqrt{59}} = \frac{30}{\sqrt{2478}} \]
Exact: $\theta = \arccos\left(\frac{30}{\sqrt{2478}}\right)$
Approximate: $\arccos(0.6026) \approx 52.9^{\circ} \to 53^{\circ}$.
\textbf{Answer:}
Exact: $\arccos\left(\frac{30}{\sqrt{2478}}\right)$
Approximate: $53^{\circ}$
\end{solution}

\begin{problem}[10]
Find the three angles of the triangle with vertices $P(1,0), Q(0,2), R(2,3)$ to the nearest degree.
\end{problem}
\begin{solution}
To find $\angle RPQ$ (angle at P), we need vectors $\vec{PQ}$ and $\vec{PR}$.
\[ \vec{PQ} = \langle 0-1, 2-0 \rangle = \langle -1, 2 \rangle \]
\[ \vec{PR} = \langle 2-1, 3-0 \rangle = \langle 1, 3 \rangle \]
\[ |\vec{PQ}| = \sqrt{1+4} = \sqrt{5}, \quad |\vec{PR}| = \sqrt{1+9} = \sqrt{10} \]
\[ \vec{PQ} \cdot \vec{PR} = (-1)(1) + (2)(3) = -1 + 6 = 5 \]
\[ \cos P = \frac{5}{\sqrt{5}\sqrt{10}} = \frac{5}{\sqrt{50}} = \frac{5}{5\sqrt{2}} = \frac{1}{\sqrt{2}} \]
\[ P = \arccos\left(\frac{1}{\sqrt{2}}\right) = 45^{\circ} \]

To find $\angle PQR$ (angle at Q), we need $\vec{QP}$ and $\vec{QR}$.
\[ \vec{QP} = -\vec{PQ} = \langle 1, -2 \rangle \]
\[ \vec{QR} = \langle 2-0, 3-2 \rangle = \langle 2, 1 \rangle \]
\[ |\vec{QP}| = \sqrt{5}, \quad |\vec{QR}| = \sqrt{4+1} = \sqrt{5} \]
\[ \vec{QP} \cdot \vec{QR} = (1)(2) + (-2)(1) = 2 - 2 = 0 \]
Since the dot product is 0, the angle is $90^{\circ}$.

To find $\angle QRP$ (angle at R):
Since sum of angles is $180^{\circ}$, $R = 180 - 45 - 90 = 45^{\circ}$.
\textbf{Answer:}
$\angle RPQ = 45^{\circ}$
$\angle PQR = 90^{\circ}$
$\angle QRP = 45^{\circ}$
\end{solution}

\begin{problem}[11]
Determine whether the given vectors are orthogonal, parallel, or neither.
\end{problem}
\begin{solution}
\textbf{(a)} $\mathbf{a} = \langle 9, 6 \rangle, \mathbf{b} = \langle -4, 6 \rangle$
\begin{itemize}
    \item Orthogonal Check: $\mathbf{a} \cdot \mathbf{b} = (9)(-4) + (6)(6) = -36 + 36 = 0$.
    \item Result: Orthogonal.
\end{itemize}

\textbf{(b)} $\mathbf{a} = \langle 6, 7, -4 \rangle, \mathbf{b} = \langle 5, -1, 7 \rangle$
\begin{itemize}
    \item Orthogonal Check: $30 - 7 - 28 = -5 \neq 0$.
    \item Parallel Check: Is $\mathbf{a} = k\mathbf{b}$? $6/5 \neq 7/(-1)$. No.
    \item Result: Neither.
\end{itemize}

\textbf{(c)} $\mathbf{a} = -8\mathbf{i} + 4\mathbf{j} + 12\mathbf{k}, \mathbf{b} = 6\mathbf{i} - 3\mathbf{j} - 9\mathbf{k}$
\begin{itemize}
    \item Parallel Check: Ratio of components.
    $a_1/b_1 = -8/6 = -4/3$.
    $a_2/b_2 = 4/-3 = -4/3$.
    $a_3/b_3 = 12/-9 = -4/3$.
    Since ratios are constant, $\mathbf{a} = (-4/3)\mathbf{b}$.
    \item Result: Parallel.
\end{itemize}

\textbf{(d)} $\mathbf{a} = 2\mathbf{i} - \mathbf{j} + 2\mathbf{k}, \mathbf{b} = 3\mathbf{i} + 2\mathbf{j} - 2\mathbf{k}$
\begin{itemize}
    \item Orthogonal Check: $(2)(3) + (-1)(2) + (2)(-2) = 6 - 2 - 4 = 0$.
    \item Result: Orthogonal.
\end{itemize}
\end{solution}

\begin{problem}[12]
Use vectors to decide whether the triangle with vertices $P(3, -5, -3), Q(4, -2, -5), R(8, -4, -6)$ is right-angled.
\end{problem}
\begin{solution}
Calculate direction vectors for the sides:
$\vec{PQ} = \langle 1, 3, -2 \rangle$
$\vec{QR} = \langle 4, -2, -1 \rangle$
$\vec{PR} = \langle 5, 1, -3 \rangle$

Check dot products:
1. $\vec{PQ} \cdot \vec{QR} = (1)(4) + (3)(-2) + (-2)(-1) = 4 - 6 + 2 = 0$.
Since this dot product is 0, these sides are perpendicular.
\textbf{Answer:} Yes, it is right-angled.
\end{solution}

\begin{problem}[13]
Same concept as 12, step-by-step.
$P(-1, -2, -1), Q(0, 1, -3), R(4, -1, -4)$.
\end{problem}
\begin{solution}
$\vec{QP} = P - Q = \langle -1-0, -2-1, -1-(-3) \rangle = \langle -1, -3, 2 \rangle$.
$\vec{QR} = R - Q = \langle 4-0, -1-1, -4-(-3) \rangle = \langle 4, -2, -1 \rangle$.
$\vec{QP} \cdot \vec{QR} = (-1)(4) + (-3)(-2) + (2)(-1) = -4 + 6 - 2 = 0$.
Since the dot product is 0, the angle is $90^{\circ}$.
\textbf{Answer:}
$\vec{QP} = \langle -1, -3, 2 \rangle$
$\vec{QR} = \langle 4, -2, -1 \rangle$
$\vec{QP} \cdot \vec{QR} = 0$
Yes, it is right-angled.
\end{solution}

\begin{problem}[14]
Find $x$ such that the angle between $\langle 4, 1, -1 \rangle$ and $\langle 1, x, 0 \rangle$ is $45^{\circ}$.
\end{problem}
\begin{solution}
$\mathbf{a} = \langle 4, 1, -1 \rangle, \mathbf{b} = \langle 1, x, 0 \rangle$.
$\mathbf{a} \cdot \mathbf{b} = 4 + x$.
$|\mathbf{a}| = \sqrt{16 + 1 + 1} = \sqrt{18} = 3\sqrt{2}$.
$|\mathbf{b}| = \sqrt{1 + x^2}$.
Formula: $\mathbf{a} \cdot \mathbf{b} = |\mathbf{a}||\mathbf{b}|\cos(45^{\circ})$
\[ 4 + x = (3\sqrt{2})\sqrt{1 + x^2} \left( \frac{1}{\sqrt{2}} \right) \]
\[ 4 + x = 3\sqrt{1 + x^2} \]
Square both sides:
\[ (4+x)^2 = 9(1+x^2) \]
\[ 16 + 8x + x^2 = 9 + 9x^2 \]
\[ 0 = 8x^2 - 8x - 7 \]
Use quadratic formula: $x = \frac{8 \pm \sqrt{64 - 4(8)(-7)}}{16} = \frac{8 \pm \sqrt{64 + 224}}{16} = \frac{8 \pm \sqrt{288}}{16}$.
$\sqrt{288} = \sqrt{144 \cdot 2} = 12\sqrt{2}$.
$x = \frac{8 \pm 12\sqrt{2}}{16} = \frac{2 \pm 3\sqrt{2}}{4}$.
Check validity: We squared the equation, so we must check for extraneous solutions.
For $x \approx 1.56$, $4+x > 0$. RHS $> 0$. Valid.
For $x \approx -0.56$, $4+x > 0$. RHS $> 0$. Valid.
\textbf{Answer:} $\frac{2 + 3\sqrt{2}}{4}, \frac{2 - 3\sqrt{2}}{4}$
\end{solution}

\begin{problem}[15]
Find a unit vector orthogonal to both $\mathbf{i}+\mathbf{j}$ and $\mathbf{i}+\mathbf{k}$.
\end{problem}
\begin{solution}
Vectors: $\mathbf{a} = \langle 1, 1, 0 \rangle$, $\mathbf{b} = \langle 1, 0, 1 \rangle$.
Let vector $\mathbf{v} = \langle x, y, z \rangle$.
$\mathbf{v} \cdot \mathbf{a} = x + y = 0 \implies y = -x$.
$\mathbf{v} \cdot \mathbf{b} = x + z = 0 \implies z = -x$.
Let $x = 1$. Then $y = -1, z = -1$.
$\mathbf{v} = \langle 1, -1, -1 \rangle$.
Magnitude $|\mathbf{v}| = \sqrt{1+1+1} = \sqrt{3}$.
Unit vector $\mathbf{u} = \frac{1}{\sqrt{3}}\langle 1, -1, -1 \rangle$.
Note: The negative version $\langle -1/\sqrt{3}, 1/\sqrt{3}, 1/\sqrt{3} \rangle$ is also correct, but usually one is sufficient.
\textbf{Answer:} $\langle \frac{1}{\sqrt{3}}, -\frac{1}{\sqrt{3}}, -\frac{1}{\sqrt{3}} \rangle$
\end{solution}

\begin{problem}[16]
Find direction cosines and angles for $\langle -9, 4, 5 \rangle$.
\end{problem}
\begin{solution}
$|\mathbf{v}| = \sqrt{81 + 16 + 25} = \sqrt{122}$.
Direction cosines: $\cos\alpha = \frac{-9}{\sqrt{122}}, \cos\beta = \frac{4}{\sqrt{122}}, \cos\gamma = \frac{5}{\sqrt{122}}$.
Angles:
$\alpha = \arccos(-9/\sqrt{122}) \approx 144.6^{\circ}$
$\beta = \arccos(4/\sqrt{122}) \approx 68.8^{\circ}$
$\gamma = \arccos(5/\sqrt{122}) \approx 63.1^{\circ}$
\end{solution}

\begin{problem}[17]
Find scalar and vector projections of $\mathbf{b}$ onto $\mathbf{a}$.
$\mathbf{a} = \langle -3, 4 \rangle, \mathbf{b} = \langle 3, 6 \rangle$.
\end{problem}
\begin{solution}
$\mathbf{a} \cdot \mathbf{b} = -9 + 24 = 15$.
$|\mathbf{a}| = \sqrt{9+16} = 5$.
Scalar proj: $\frac{\mathbf{a} \cdot \mathbf{b}}{|\mathbf{a}|} = \frac{15}{5} = 3$.
Vector proj: (Scalar proj)$\frac{\mathbf{a}}{|\mathbf{a}|} = 3 \frac{\langle -3, 4 \rangle}{5} = \langle -1.8, 2.4 \rangle$.
\textbf{Answer:} Scalar: $3$; Vector: $\langle -\frac{9}{5}, \frac{12}{5} \rangle$.
\end{solution}

\begin{problem}[18]
$\mathbf{a} = \langle 1, 6 \rangle, \mathbf{b} = \langle 3, 5 \rangle$.
\end{problem}
\begin{solution}
$\mathbf{a} \cdot \mathbf{b} = 3 + 30 = 33$.
$|\mathbf{a}| = \sqrt{1+36} = \sqrt{37}$.
Scalar proj: $\frac{33}{\sqrt{37}}$.
Vector proj: $\frac{33}{\sqrt{37}} \frac{\langle 1, 6 \rangle}{\sqrt{37}} = \frac{33}{37}\langle 1, 6 \rangle = \langle \frac{33}{37}, \frac{198}{37} \rangle$.
\end{solution}

\begin{problem}[19]
$\mathbf{a} = \langle -1, 4, 8 \rangle, \mathbf{b} = \langle 14, 1, 2 \rangle$.
\end{problem}
\begin{solution}
$\mathbf{a} \cdot \mathbf{b} = -14 + 4 + 16 = 6$.
$|\mathbf{a}| = \sqrt{1+16+64} = \sqrt{81} = 9$.
Scalar proj: $6/9 = 2/3$.
Vector proj: $\frac{2}{3} \frac{\langle -1, 4, 8 \rangle}{9} = \langle -\frac{2}{27}, \frac{8}{27}, \frac{16}{27} \rangle$.
\end{solution}

\begin{problem}[20]
$\text{orth}_{\mathbf{a}}\mathbf{b}$. $\mathbf{a} = \langle 1, 4 \rangle, \mathbf{b} = \langle 2, 3 \rangle$.
\end{problem}
\begin{solution}
$\text{orth}_{\mathbf{a}}\mathbf{b} = \mathbf{b} - \text{proj}_{\mathbf{a}}\mathbf{b}$.
$\mathbf{a} \cdot \mathbf{b} = 2 + 12 = 14$. $|\mathbf{a}|^2 = 17$.
$\text{proj}_{\mathbf{a}}\mathbf{b} = \frac{14}{17}\langle 1, 4 \rangle = \langle \frac{14}{17}, \frac{56}{17} \rangle$.
$\text{orth} = \langle 2, 3 \rangle - \langle \frac{14}{17}, \frac{56}{17} \rangle = \langle \frac{34-14}{17}, \frac{51-56}{17} \rangle = \langle \frac{20}{17}, -\frac{5}{17} \rangle$.
Graph: Choose the graph where $\text{proj}$ is parallel to $\mathbf{a}$ and $\text{orth}$ is perpendicular.
\textbf{Answer:} $\langle \frac{20}{17}, -\frac{5}{17} \rangle$.
\end{solution}

\newpage
\section{Part 3: In-Depth Analysis of Problems and Techniques}

\subsection{A) Problem Types and General Approach}

\textbf{Type 1: The "Meaningful vs. Meaningless" Checks (Problem 1)}
These conceptual problems test your understanding of input/output types. The strategy is to track the data type at each step.
\begin{itemize}
    \item Vector $\cdot$ Vector $\to$ Scalar
    \item Scalar $\cdot$ Vector $\to$ Undefined (Standard multiplication is used, not dot)
    \item Scalar + Vector $\to$ Undefined
\end{itemize}

\textbf{Type 2: Basic Computation (Problems 2, 3, 4, 5)}
Straightforward application of the algebraic formula $\sum a_ib_i$. The strategy is simply careful arithmetic.

\textbf{Type 3: Geometry from Algebra (Problems 8, 9, 10, 14, 16)}
These require converting component forms into angles using $\theta = \arccos(\frac{\mathbf{a}\cdot\mathbf{b}}{|\mathbf{a}||\mathbf{b}|})$.
For Problem 10 (Triangle angles), the key strategy is defining vectors radiating \emph{from} the vertex in question (e.g., $\vec{PQ}$ and $\vec{PR}$ for angle $P$).

\textbf{Type 4: Orthogonality/Parallelism Checks (Problems 11, 12, 13, 15)}
\begin{itemize}
    \item Orthogonal: Check if dot product is 0.
    \item Parallel: Check if components are proportional ($\frac{a_1}{b_1} = \frac{a_2}{b_2} = \frac{a_3}{b_3}$).
\end{itemize}

\textbf{Type 5: Projections (Problems 17-20)}
Using the projection formulas. Problem 20 introduces the "Orthogonal Component" ($\text{orth}_{\mathbf{a}}\mathbf{b}$), which is just the original vector minus the projection.

\subsection{B) Key Manipulations}

\textbf{The "Tail-to-Tail" Adjustment (Problem 7)}
In the equilateral triangle problem, the vectors were head-to-tail. A crucial manipulation was shifting the vectors to be tail-to-tail to find the correct angle for the dot product ($\theta_{exterior} = 180 - \theta_{interior}$). Using the interior angle directly results in the wrong sign.

\textbf{Quadratic Solving for Geometry (Problem 14)}
We set up the dot product equation equal to the geometric definition and solved for a variable coordinate $x$. This required squaring both sides to remove square roots, necessitating a check for extraneous solutions (though both were valid here).

\textbf{System of Equations for Orthogonality (Problem 15)}
To find a vector orthogonal to two others, we set up a system: $x+y=0$ and $x+z=0$. This is an algebraic alternative to the cross product, relying purely on the definition of the dot product.

\newpage
\section{Part 4: "Cheatsheet" and Tips for Success}

\subsection{Summary of Formulas}
\begin{itemize}
    \item $\mathbf{a} \cdot \mathbf{b} = a_1b_1 + a_2b_2 + a_3b_3$
    \item $\cos\theta = \frac{\mathbf{a} \cdot \mathbf{b}}{|\mathbf{a}||\mathbf{b}|}$
    \item $\text{Scalar Comp} = \frac{\mathbf{a} \cdot \mathbf{b}}{|\mathbf{a}|}$
    \item $\text{Vector Proj} = \left( \frac{\mathbf{a} \cdot \mathbf{b}}{|\mathbf{a}|^2} \right) \mathbf{a}$
\end{itemize}

\subsection{Tips and Shortcuts}
\begin{itemize}
    \item \textbf{Perpendicular check:} If the question asks "Are they orthogonal?", calculate the dot product immediately. If it's 0, you're done.
    \item \textbf{Direction Cosines:} The components of a unit vector \emph{are} the direction cosines. If you normalize $\mathbf{v}$ (divide by magnitude), the resulting components are $\cos\alpha, \cos\beta, \cos\gamma$.
    \item \textbf{Projection Denominator:} The scalar projection divides by magnitude $|\mathbf{a}|$. The vector projection divides by magnitude squared $|\mathbf{a}|^2$.
\end{itemize}

\subsection{Common Pitfalls}
\begin{itemize}
    \item \textbf{Degree/Radian Mode:} Ensure your calculator is in Degree mode if the question asks for degrees (like Problem 10).
    \item \textbf{Projection Direction:} $\text{proj}_{\mathbf{a}}\mathbf{b}$ is a vector parallel to $\mathbf{a}$, not $\mathbf{b}$. Students often mistakenly multiply by $\mathbf{b}$ at the end.
    \item \textbf{Angle Definition:} The angle $\theta$ is always between $0$ and $\pi$ ($0$ to $180^{\circ}$).
\end{itemize}

\newpage
\section{Part 5: Conceptual Synthesis and The "Big Picture"}

\subsection{Thematic Connections}
The central theme of the Dot Product is \textbf{Interaction}. It measures how much one vector "goes in the direction" of another.
\begin{itemize}
    \item In \textbf{Physics}, this appears as \textbf{Work}. Force interacting with Displacement. Only the part of Force parallel to Displacement counts.
    \item In \textbf{Data Science}, this appears as \textbf{Similarity}. The dot product of two normalized data vectors tells you how correlated they are (Cosine Similarity).
\end{itemize}

\subsection{Forward and Backward Links}
\textbf{Backward:} This relies on the Law of Cosines from trigonometry.
\textbf{Forward:}
\begin{itemize}
    \item \textbf{Planes (Ch 12.5):} The equation of a plane is defined using the dot product: $\mathbf{n} \cdot (\mathbf{r} - \mathbf{r}_0) = 0$.
    \item \textbf{Line Integrals (Ch 16):} $\int \mathbf{F} \cdot d\mathbf{r}$ sums up tangential interactions along a curve.
    \item \textbf{Flux Integrals (Ch 16):} $\iint \mathbf{F} \cdot \mathbf{n} \, dS$ sums up flow passing through a surface.
\end{itemize}

\section{Part 6: Real-World Application and Modeling}

\subsection{Concrete Scenarios: Financial Focus}
\begin{enumerate}
    \item \textbf{Portfolio Variance:} In finance, a portfolio is a vector of weights $\mathbf{w}$. The risk (variance) of the portfolio involves the dot product of the weight vector with the covariance matrix of asset returns. The scalar value $\mathbf{w}^T \Sigma \mathbf{w}$ (a form of dot product) gives the total portfolio risk.
    \item \textbf{Correlation of Assets:} An analyst wants to know if two stocks move together. By treating the daily returns of Stock A and Stock B as high-dimensional vectors, the correlation coefficient is exactly the cosine of the angle between these vectors (Cosine Similarity). If the angle is 0 (cosine 1), they move perfectly in sync. If 90 degrees (dot product 0), they are uncorrelated.
\end{enumerate}

\subsection{Model Problem Setup: Correlation}
\textbf{Scenario:} Calculating the correlation between Tech Stock X and Tech Stock Y over 3 days.
\textbf{Variables:}
Vector $\mathbf{x} = \langle x_1, x_2, x_3 \rangle$ representing returns of Stock X.
Vector $\mathbf{y} = \langle y_1, y_2, y_3 \rangle$ representing returns of Stock Y.
\textbf{Equation:}
To find the correlation $\rho$ (which is $\cos \theta$):
\[ \rho = \frac{\mathbf{x} \cdot \mathbf{y}}{|\mathbf{x}||\mathbf{y}|} = \frac{x_1y_1 + x_2y_2 + x_3y_3}{\sqrt{x_1^2+x_2^2+x_3^2}\sqrt{y_1^2+y_2^2+y_3^2}} \]

\newpage
\section{Part 7: Common Variations and Untested Concepts}

Your homework covered the basics thoroughly, but missed a few standard applications found in most Calculus III curricula.

\subsection{1. Work Done by a Constant Force}
\textbf{Concept:} Work is the dot product of Force and Displacement vectors.
\textbf{Example:} A sled is pulled $50$ meters along flat ground by a force of $20$ N applied at an angle of $30^{\circ}$ above the horizontal. Find the work done.
\textbf{Solution:}
$\mathbf{F} \cdot \mathbf{d} = |\mathbf{F}| |\mathbf{d}| \cos\theta = (20)(50)\cos(30^{\circ}) = 1000(\frac{\sqrt{3}}{2}) = 500\sqrt{3}$ Joules.

\subsection{2. Cauchy-Schwarz Inequality}
\textbf{Concept:} $|\mathbf{a} \cdot \mathbf{b}| \leq |\mathbf{a}| |\mathbf{b}|$. This is a rigorous proof concept derived from the fact that $|\cos\theta| \leq 1$. It is fundamental in proving the Triangle Inequality in linear algebra.

\section{Part 8: Advanced Diagnostic Testing: "Find the Flaw"}

Below are five solutions that contain specific errors. Find the error and correct it.

\textbf{Problem A: Projections}
Find the vector projection of $\mathbf{b}=\langle 1, 2 \rangle$ onto $\mathbf{a}=\langle 3, 0 \rangle$.
\textit{Flawed Solution:}
$\text{proj}_{\mathbf{a}}\mathbf{b} = \frac{\mathbf{a} \cdot \mathbf{b}}{|\mathbf{b}|^2} \mathbf{b} = \frac{3}{5} \langle 1, 2 \rangle = \langle 0.6, 1.2 \rangle$.
\begin{itemize}
    \item \textbf{Error:} The formula used projects onto $\mathbf{b}$ instead of $\mathbf{a}$.
    \item \textbf{Correction:} Formula is $\frac{\mathbf{a} \cdot \mathbf{b}}{|\mathbf{a}|^2}\mathbf{a}$. Correct answer: $\frac{3}{9}\langle 3, 0 \rangle = \langle 1, 0 \rangle$.
\end{itemize}

\textbf{Problem B: Angle Calculation}
Find the angle between $\mathbf{a}=\langle 1, 1 \rangle$ and $\mathbf{b}=\langle -1, -1 \rangle$.
\textit{Flawed Solution:}
$\mathbf{a} \cdot \mathbf{b} = -2$. $|\mathbf{a}|=\sqrt{2}, |\mathbf{b}|=\sqrt{2}$.
$\cos\theta = \frac{-2}{2} = -1$.
$\theta = \cos^{-1}(1) = 0^{\circ}$ because $\cos$ is even.
\begin{itemize}
    \item \textbf{Error:} Incorrect inverse cosine evaluation; $\cos^{-1}(-1)$ is not $0$.
    \item \textbf{Correction:} $\arccos(-1) = 180^{\circ}$ (or $\pi$ radians). The vectors point in opposite directions.
\end{itemize}

\textbf{Problem C: Orthogonality}
Find $k$ if $\mathbf{a}=\langle k, 2 \rangle$ and $\mathbf{b}=\langle k, -2 \rangle$ are orthogonal.
\textit{Flawed Solution:}
$\mathbf{a} \cdot \mathbf{b} = k^2 - 4$.
For orthogonality, dot product equals 1.
$k^2 - 4 = 1 \implies k = \sqrt{5}$.
\begin{itemize}
    \item \textbf{Error:} Fundamental Misunderstanding: Orthogonality requires the dot product to be $0$, not $1$.
    \item \textbf{Correction:} $k^2 - 4 = 0 \implies k = \pm 2$.
\end{itemize}

\textbf{Problem D: Associative Property}
Evaluate $(\mathbf{a} \cdot \mathbf{b}) \cdot \mathbf{c}$ where $\mathbf{a}=\mathbf{i}, \mathbf{b}=\mathbf{j}, \mathbf{c}=\mathbf{k}$.
\textit{Flawed Solution:}
$\mathbf{a} \cdot (\mathbf{b} \cdot \mathbf{c}) = \mathbf{i} \cdot (0) = 0$.
So the answer is $0$.
\begin{itemize}
    \item \textbf{Error:} Treating the dot product as associative. The expression is actually meaningless (Scalar dot Vector).
    \item \textbf{Correction:} The expression is undefined/meaningless.
\end{itemize}

\textbf{Problem E: Work Done}
Force $\mathbf{F} = \langle 2, 3 \rangle$ moves an object from $(1,1)$ to $(4,5)$. Find Work.
\textit{Flawed Solution:}
$W = \mathbf{F} \cdot \text{Point} = \langle 2, 3 \rangle \cdot \langle 4, 5 \rangle = 8 + 15 = 23$.
\begin{itemize}
    \item \textbf{Error:} Dotting Force with the final position coordinate instead of the displacement vector.
    \item \textbf{Correction:} Displacement $\mathbf{d} = \langle 4-1, 5-1 \rangle = \langle 3, 4 \rangle$. Work $W = (2)(3) + (3)(4) = 6 + 12 = 18$.
\end{itemize}

\end{document}