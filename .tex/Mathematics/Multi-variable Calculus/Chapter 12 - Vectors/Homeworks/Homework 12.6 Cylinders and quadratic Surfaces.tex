\documentclass[12pt]{article}
\usepackage{amsmath, amssymb, amsthm}
\usepackage{geometry}
\usepackage{graphicx}
\usepackage{fancyhdr}
\usepackage{enumitem}
\usepackage{booktabs}
\usepackage{tikz}
\usepackage{pgfplots}
\pgfplotsset{compat=1.18}

% Geometry settings
\geometry{letterpaper, margin=1in}

% Header and Footer
\pagestyle{fancy}
\fancyhf{}
\rhead{Calculus III: Cylinders and Quadric Surfaces}
\lhead{Homework 12.6 Study Guide}
\cfoot{\thepage}

% Title Information
\title{\textbf{Homework 12.6: Cylinders and Quadric Surfaces}\\ \large Comprehensive Study Guide and Solutions}
\author{Tashfeen Omran}
\date{January 2026}

\begin{document}

\maketitle
\tableofcontents
\newpage

\section{Part 1: Introduction, Context, and Prerequisites}

\subsection{Core Concepts}
This topic extends the study of \textbf{conic sections} (circles, ellipses, parabolas, hyperbolas) from two dimensions ($\mathbb{R}^2$) into three dimensions ($\mathbb{R}^3$). 
\begin{itemize}
    \item \textbf{Cylinders:} In Calculus III, a "cylinder" is not just a soda can shape. It is any surface defined by an equation that is \textbf{missing one variable} (e.g., $z = x^2$). This implies the curve $z=x^2$ exists in the $xz$-plane and is "extruded" or translated infinitely along the missing axis (the $y$-axis).
    \item \textbf{Quadric Surfaces:} These are the 3D analogs of conic sections, defined by second-degree equations in $x, y, z$. The general form is $Ax^2 + By^2 + Cz^2 + Dxy + Eyz + Fxz + Gx + Hy + Iz + J = 0$. By aligning axes, we usually work with standard forms like $Ax^2 + By^2 + Cz^2 = J$.
    \item \textbf{Traces:} To visualize a 3D surface, we slice it with planes parallel to the coordinate planes (e.g., $z=k$). The resulting 2D curves are called \textit{traces}.
\end{itemize}

\subsection{Intuition and Derivation}
The formulas for quadric surfaces are derived by rotating or stretching 2D conics.
\begin{itemize}
    \item \textbf{The Ellipsoid:} Think of a unit sphere $x^2+y^2+z^2=1$. If you stretch it by factor $a$ in $x$, $b$ in $y$, and $c$ in $z$, you get $\frac{x^2}{a^2} + \frac{y^2}{b^2} + \frac{z^2}{c^2} = 1$.
    \item \textbf{Paraboloids:} Derived from rotating a parabola. If you spin $z=x^2$ around the z-axis, you get circular traces: $z = x^2 + y^2$.
\end{itemize}

\subsection{Key Formulas (Standard Forms)}
\begin{enumerate}
    \item \textbf{Ellipsoid:} $\frac{x^2}{a^2} + \frac{y^2}{b^2} + \frac{z^2}{c^2} = 1$ (All terms positive).
    \item \textbf{Hyperboloid of One Sheet:} $\frac{x^2}{a^2} + \frac{y^2}{b^2} - \frac{z^2}{c^2} = 1$ (One negative term; axis corresponds to the negative variable).
    \item \textbf{Hyperboloid of Two Sheets:} $-\frac{x^2}{a^2} - \frac{y^2}{b^2} + \frac{z^2}{c^2} = 1$ (Two negative terms; axis corresponds to the positive variable).
    \item \textbf{Elliptic Cone:} $\frac{z^2}{c^2} = \frac{x^2}{a^2} + \frac{y^2}{b^2}$ (Homogeneous, degree 2).
    \item \textbf{Elliptic Paraboloid:} $\frac{z}{c} = \frac{x^2}{a^2} + \frac{y^2}{b^2}$ (One linear, two quadratic same sign).
    \item \textbf{Hyperbolic Paraboloid (Saddle):} $\frac{z}{c} = \frac{x^2}{a^2} - \frac{y^2}{b^2}$ (One linear, two quadratic opposite signs).
\end{enumerate}

\subsection{Prerequisites}
\begin{itemize}
    \item \textbf{Completing the Square:} Essential for converting general equations (e.g., $x^2 - 4x \dots$) into standard forms $(x-2)^2 \dots$.
    \item \textbf{Conic Sections:} Recognizing $x^2 - y^2 = 1$ as a hyperbola and $x^2 + 2y^2 = 1$ as an ellipse.
\end{itemize}

\newpage

\section{Part 2: Detailed Homework Solutions}

\subsection*{Problem 1 (Page 2)}
\textbf{Question:} 
(a) What does $y=x^2$ represent as a curve in $\mathbb{R}^2$?
(b) What does it represent as a surface in $\mathbb{R}^3$?
(c) What does $z=y^2$ represent?

\textbf{Solution:}
\begin{itemize}
    \item[(a)] In 2D ($\mathbb{R}^2$), the equation $y=x^2$ describes a set of points forming a curve. This is the standard equation of a \textbf{parabola}.
    \item[(b)] In 3D ($\mathbb{R}^3$), the equation $y=x^2$ lacks the variable $z$. This means $z$ can be any value. We take the parabola $y=x^2$ in the $xy$-plane and extend it vertically along the $z$-axis. This creates a surface called a \textbf{parabolic cylinder}.
    \item[(c)] The equation $z=y^2$ lacks the variable $x$. This is a parabola in the $yz$-plane extended infinitely along the $x$-axis. It is also a \textbf{parabolic cylinder}.
\end{itemize}

\subsection*{Problem 2 (Page 3)}
\textbf{Question:} Describe the surface $x^2 + z^2 = 8$. Sketch traces.

\textbf{Solution:}
\textbf{Description:} The equation is missing the variable $y$. In the $xz$-plane, $x^2 + z^2 = 8$ is a circle with radius $\sqrt{8} = 2\sqrt{2}$. Since $y$ is free, this circle is extruded along the $y$-axis.
\\
\textbf{Surface Type:} \textbf{Circular cylinder} (oriented along the $y$-axis).

\textbf{Cross Sections (Traces):}
\begin{itemize}
    \item At $y = -8$: The equation is still $x^2 + z^2 = 8$. (Circle)
    \item At $y = 0$: The equation is $x^2 + z^2 = 8$. (Circle)
    \item At $y = 8$: The equation is $x^2 + z^2 = 8$. (Circle)
\end{itemize}
\textbf{Note:} All cross sections perpendicular to the axis of the cylinder are identical.

\subsection*{Problem 3 (Page 5)}
\textbf{Question:} Describe the surface $y^2 + 6z^2 = 6$.

\textbf{Solution:}
\textbf{Description:} The variable $x$ is missing. In the $yz$-plane, we have $\frac{y^2}{6} + \frac{z^2}{1} = 1$, which is an ellipse. Extruding this along the $x$-axis creates a cylinder.
\\
\textbf{Surface Type:} \textbf{Elliptic cylinder}.

\textbf{Cross Sections:}
\begin{itemize}
    \item At $x = -6$: $y^2 + 6z^2 = 6$.
    \item At $x = 0$: $y^2 + 6z^2 = 6$.
    \item At $x = 6$: $y^2 + 6z^2 = 6$.
\end{itemize}

\subsection*{Problem 4 (Page 7)}
\textbf{Question:} Write an equation for the surface shown.
\\
\textbf{Analysis:} The graph shows a surface that forms a wave shape. The "rulings" (straight lines) are parallel to the $x$-axis (the axis coming out of the page to the left). The cross-section in the $yz$-plane looks like a bell curve or a localized wave. Since the shape does not change as we move along $x$, it is a cylinder.
\\
The curve passes through $z=1$ at $y=0$ and decays as $y \to \pm \infty$. A common function for this shape is the Gaussian function.
\\
\textbf{Equation:} \[ z = e^{-y^2} \]
(Alternatively, depending on the specific function bank of the text, it could be $z = \frac{1}{1+y^2}$ or a single period of a cosine like $z = \cos y$). Given the decay shown, $z = e^{-y^2}$ is the best fit.

\subsection*{Problem 5 (Page 8)}
\textbf{Question:} Write an equation for the surface shown.
\\
\textbf{Analysis:} The surface extends infinitely along the $y$-axis without changing shape. The cross-section is in the $xz$-plane. The curve passes through the origin and looks like a cubic function $z = x^3$.
\\
\textbf{Equation:} \[ z = x^3 \]

\subsection*{Problem 6 (Page 9--10)}
\textbf{Question:} Consider $x^2 + y^2 - z^2 = 16$.
\begin{itemize}
    \item[(a)] Find traces and identify the surface.
    \item[(b)] How is the graph affected if changed to $x^2 - y^2 + z^2 = 16$?
    \item[(c)] How is it affected if changed to $x^2 + y^2 + 2y - z^2 = 15$?
\end{itemize}

\textbf{Solution:}
(a) \textbf{Traces:}
\begin{itemize}
    \item $x = k$: $y^2 - z^2 = 16 - k^2$. This is a \textbf{hyperbola}.
    \item $y = k$: $x^2 - z^2 = 16 - k^2$. This is a \textbf{hyperbola}.
    \item $z = k$: $x^2 + y^2 = 16 + k^2$. This is a \textbf{circle} (radius $\sqrt{16+k^2}$).
\end{itemize}
\textbf{Identification:} The equation $\frac{x^2}{16} + \frac{y^2}{16} - \frac{z^2}{16} = 1$ has one negative term ($z$).
\\
\textbf{Surface:} \textbf{Hyperboloid of one sheet} (Axis of symmetry is the $z$-axis).

(b) \textbf{Change to $x^2 - y^2 + z^2 = 16$:}
Now the negative sign is on the $y$ term. The axis of symmetry becomes the $y$-axis.
\\
\textbf{Answer:} The graph is rotated so that its axis is the \textbf{$y$-axis}.

(c) \textbf{Change to $x^2 + y^2 + 2y - z^2 = 15$:}
Complete the square for $y$:
\[ x^2 + (y^2 + 2y + 1) - 1 - z^2 = 15 \]
\[ x^2 + (y+1)^2 - z^2 = 16 \]
This is the same shape as part (a), but the center has moved from $(0,0,0)$ to $(0, -1, 0)$.
\\
\textbf{Answer:} The graph is shifted one unit in the \textbf{negative $y$-direction}.

\subsection*{Problem 7 (Page 11)}
\textbf{Question:} Sketch/identify $x = y^2 + 8z^2$. Find traces.

\textbf{Solution:}
This equation involves one linear variable ($x$) and two squared variables ($y, z$) with the same sign ($+$).
\\
\textbf{Surface:} \textbf{Elliptic Paraboloid}. The axis is the $x$-axis. It opens in the positive $x$ direction.

\textbf{Equations for Cross Sections:}
\begin{itemize}
    \item $z=0 \implies x = y^2$ (Parabola).
    \item $y=0 \implies x = 8z^2$ (Parabola).
    \item $x=-8 \implies -8 = y^2 + 8z^2$ (No real solution / DNE).
    \item $x=0 \implies 0 = y^2 + 8z^2$ (Point at origin).
    \item $x=8 \implies 8 = y^2 + 8z^2$ (Ellipse).
\end{itemize}

\subsection*{Problem 8 (Page 13)}
\textbf{Question:} Sketch/identify $9x^2 + 4y^2 + 4z^2 = 36$.

\textbf{Solution:}
Divide by 36:
\[ \frac{x^2}{4} + \frac{y^2}{9} + \frac{z^2}{9} = 1 \]
All signs are positive.
\\
\textbf{Surface:} \textbf{Ellipsoid}. (Specifically a prolate spheroid since $b=c=3 > a=2$).

\textbf{Traces:}
\begin{itemize}
    \item $z=0 \implies 9x^2 + 4y^2 = 36$ (Ellipse).
    \item $y=0 \implies 9x^2 + 4z^2 = 36$ (Ellipse).
    \item $x=0 \implies 4y^2 + 4z^2 = 36 \implies y^2 + z^2 = 9$ (Circle).
\end{itemize}

\subsection*{Problem 9 (Page 15)}
\textbf{Question:} Sketch/identify $z^2 - 36x^2 - y^2 = 36$.

\textbf{Solution:}
Divide by 36:
\[ \frac{z^2}{36} - x^2 - \frac{y^2}{36} = 1 \]
There are two negative signs ($x$ and $y$) and one positive ($z$).
\\
\textbf{Surface:} \textbf{Hyperboloid of Two Sheets}.
The axis of symmetry is the $z$-axis (corresponding to the positive term).

\textbf{Traces:}
\begin{itemize}
    \item $z = -7$: $(-7)^2 - 36x^2 - y^2 = 36 \implies 49 - 36 = 36x^2 + y^2 \implies 36x^2 + y^2 = 13$ (Ellipse).
    \item $z = 0$: $-36x^2 - y^2 = 36$ (Impossible, sum of negatives cannot be positive). DNE.
    \item $z = 7$: Same as $z=-7$. $36x^2 + y^2 = 13$ (Ellipse).
    \item $y = 0$: $z^2 - 36x^2 = 36$ (Hyperbola).
    \item $x = 0$: $z^2 - y^2 = 36$ (Hyperbola).
\end{itemize}

\subsection*{Problem 10 (Page 17)}
\textbf{Question:} Sketch/identify $9x^2 - y^2 + 3z^2 = 0$.

\textbf{Solution:}
Rewrite as $y^2 = 9x^2 + 3z^2$.
This is a homogeneous degree 2 equation (all terms quadratic, constant is 0). It represents a cone.
\\
\textbf{Surface:} \textbf{Elliptic Cone}. Axis is the $y$-axis (the variable isolated on one side).

\textbf{Traces:}
\begin{itemize}
    \item $z=0 \implies 9x^2 - y^2 = 0 \implies y = \pm 3x$ (Two Lines).
    \item $y=-9 \implies 9x^2 - (-9)^2 + 3z^2 = 0 \implies 9x^2 + 3z^2 = 81$ (Ellipse).
    \item $y=9 \implies 9x^2 + 3z^2 = 81$ (Ellipse).
    \item $x=0 \implies -y^2 + 3z^2 = 0 \implies y = \pm \sqrt{3}z$ (Two Lines).
\end{itemize}

\subsection*{Problems 11--14 (Matching Equations to Graphs)}
\textit{Note: The graphs labeled in the problem analysis refer to standard shapes. Since the screenshots provided in the homework PDF may show generic placeholders, the solutions below derive the correct classification and orientation based on the equations.}

\textbf{Problem 11:} $x^2 + 25y^2 + 36z^2 = 1$.
\\
All coefficients are positive. This is an \textbf{Ellipsoid}.
\\
Intercepts: $x=\pm 1, y=\pm 1/5, z=\pm 1/6$. The longest axis is the $x$-axis.

\textbf{Problem 12:} $16x^2 + 9y^2 + z^2 = 1$.
\\
All coefficients are positive. This is an \textbf{Ellipsoid}.
\\
Intercepts: $x=\pm 1/4, y=\pm 1/3, z=\pm 1$. The longest axis is the $z$-axis.
\\
\textit{Matching:} In the provided images, Graph 6 is an ellipsoid elongated vertically ($z$-axis). This matches Problem 12 perfectly.

\textbf{Problem 13:} $-x^2 + y^2 - z^2 = 5$.
\\
Rewrite: $y^2 - x^2 - z^2 = 5 \implies \frac{y^2}{5} - \frac{x^2}{5} - \frac{z^2}{5} = 1$.
\\
Two negative terms ($x, z$). This is a \textbf{Hyperboloid of Two Sheets}.
\\
Axis: The positive term is $y$, so the axis of symmetry is the \textbf{$y$-axis} (the "cups" open left and right).

\textbf{Problem 14:} $y^2 = x^2 + 3z^2$.
\\
Rewrite: $y^2 - x^2 - 3z^2 = 0$.
\\
Homogeneous equation. This is an \textbf{Elliptic Cone}.
\\
Axis: The variable alone on the LHS is $y$, so the axis is the \textbf{$y$-axis}.

\subsection*{Problem 15 (Page 23)}
\textbf{Question:} Sketch a surface with traces:
\begin{itemize}
    \item Traces in $x=k$ are hyperbolas (opening vertically along $z$).
    \item Traces in $y=k$ are concentric circles.
\end{itemize}
\textbf{Solution:}
Traces in $y=k$ being circles implies rotational symmetry about the $y$-axis, or an equation like $x^2 + z^2 = R(y)^2$.
\\
Traces in $x=k$ being hyperbolas $z^2 - y^2 = C$ implies a difference of squares involving $z$ and $y$.
\\
Combining these:
Consider the \textbf{Hyperboloid of One Sheet} with axis $y$.
Equation: $x^2 + z^2 - y^2 = 1$.
\begin{itemize}
    \item Check $y=k$: $x^2 + z^2 = 1 + k^2$ (Circles). Matches.
    \item Check $x=k$: $z^2 - y^2 = 1 - k^2$ (Hyperbolas). Matches.
\end{itemize}
\textbf{Identification:} \textbf{Hyperboloid of One Sheet}.

\subsection*{Problem 16 (Page 27)}
\textbf{Question:} Sketch a surface with traces:
\begin{itemize}
    \item Traces in $x=k$ are ellipses.
    \item Traces in $z=k$ are hyperbolas (opening along $x$).
\end{itemize}
\textbf{Solution:}
Traces in $x=k$ being ellipses means for fixed $x$, $Ay^2 + Bz^2 = C$ where $A, B$ have the same sign. Let's assume minus signs for standard form: $-y^2 - z^2 = \text{const}$.
\\
Traces in $z=k$ being hyperbolas opening along $x$ means $x^2 - y^2 = C$ (positive $x^2$, negative $y^2$).
\\
Signs: $x^2 (+), y^2 (-), z^2 (-)$.
\\
Equation: $x^2 - y^2 - z^2 = 1$.
\\
This is a \textbf{Hyperboloid of Two Sheets} (axis is $x$).
\begin{itemize}
    \item Check $x=k$: $-y^2 - z^2 = 1 - k^2 \implies y^2 + z^2 = k^2 - 1$. (Ellipses/Circles if $k^2 > 1$). Matches.
    \item Check $z=k$: $x^2 - y^2 = 1 + k^2$. (Hyperbolas). Matches.
\end{itemize}
\textbf{Identification:} \textbf{Hyperboloid of Two Sheets}.

\subsection*{Problem 17 (Page 31)}
\textbf{Question:} $x^2 + y^2 - 2x - 8y - z + 17 = 0$. Reduce, classify, sketch.

\textbf{Solution:}
\textbf{Step 1: Complete the Square.}
Group $x$ and $y$ terms:
\[ (x^2 - 2x) + (y^2 - 8y) = z - 17 \]
Add constants to complete squares ($(\frac{-2}{2})^2=1$ and $(\frac{-8}{2})^2=16$):
\[ (x^2 - 2x + 1) + (y^2 - 8y + 16) = z - 17 + 1 + 16 \]
\[ (x-1)^2 + (y-4)^2 = z \]
\textbf{Step 2: Classify.}
Form: $z = \frac{(x-h)^2}{a^2} + \frac{(y-k)^2}{b^2}$.
This is an \textbf{Elliptic Paraboloid}.
Vertex: $(1, 4, 0)$. Opens up (positive $z$).

\textbf{Step 3: Traces.}
\begin{itemize}
    \item $z=-1$: $(x-1)^2 + (y-4)^2 = -1$. Impossible. DNE.
    \item $z=1$: $(x-1)^2 + (y-4)^2 = 1$. Circle.
    \item $y=0$: $(x-1)^2 + (-4)^2 = z \implies z = (x-1)^2 + 16$. Parabola.
    \item $y=4$: $(x-1)^2 = z$. Parabola.
    \item $x=0$: $1 + (y-4)^2 = z$. Parabola.
\end{itemize}

\subsection*{Problem 18 (Page 33)}
\textbf{Question:} Sketch $z = 8x^2$.
\\
\textbf{Solution:}
Notice that the equation of the graph $z=8x^2$ doesn't involve $y$. This means that any vertical plane with equation $y=k$ intersects the graph in a curve with equation $z=8x^2$. So these vertical traces are \textbf{parabolas}.
\\
The graph is formed by taking the parabola $z=8x^2$ in the $xz$-plane and moving it in the direction of the \textbf{$y$-axis}. The graph is a surface, called a \textbf{parabolic cylinder}. Here the rulings of the cylinder are parallel to the \textbf{$y$-axis}.

\newpage

\section{Part 3: In-Depth Analysis}

\subsection{A) Problem Types and Approaches}
\begin{enumerate}
    \item \textbf{Type 1: Cylinder Identification (Problems 1-5, 18)}
    \textit{Recognition:} One variable is completely missing from the equation (e.g., $z = 8x^2$ has no $y$).
    \textit{Strategy:} Identify the curve in the 2D plane of the present variables. State that the surface is this curve "extruded" along the axis of the missing variable.

    \item \textbf{Type 2: Standard Quadric Classification (Problems 8, 9, 11-14)}
    \textit{Recognition:} Equation involves $x^2, y^2, z^2$ and perhaps constants.
    \textit{Strategy:} Put into standard form $=1$ or $=0$. Count the positive and negative signs.
    \begin{itemize}
        \item All $+ = 1$: Ellipsoid.
        \item One $- = 1$: Hyperboloid 1 Sheet.
        \item Two $- = 1$: Hyperboloid 2 Sheets.
        \item Homogeneous (equals 0): Cone.
    \end{itemize}

    \item \textbf{Type 3: Completing the Square (Problems 6, 17)}
    \textit{Recognition:} Presence of linear terms like $-2x$ or $-8y$.
    \textit{Strategy:} Group variables, add $(\frac{b}{2})^2$ to both sides, and factor into $(x-h)^2$. This reveals the center shift.

    \item \textbf{Type 4: Reverse Engineering from Traces (Problems 15, 16)}
    \textit{Recognition:} You are given descriptions of 2D cross-sections (e.g., "Traces in $x=k$ are ellipses").
    \textit{Strategy:} Write generic 2D equations for the traces and deduce the 3D sign pattern. (e.g., Ellipses in $x$ traces $\to$ $y^2, z^2$ same sign).
\end{enumerate}

\subsection{B) Key Manipulations}
\begin{itemize}
    \item \textbf{Completing the Square:} Used in Problem 17 ($x^2 - 2x \to (x-1)^2 - 1$). Crucial for finding the vertex of shifted surfaces. Without this, you cannot identify the location or true axis of the surface.
    \item \textbf{Normalizing Constants:} Used in Problem 8. Dividing $9x^2 + 4y^2 + 4z^2 = 36$ by 36 to get $=1$ allows direct comparison with the standard ellipsoid formula $\frac{x^2}{a^2} + \dots = 1$ to find intercepts.
    \item \textbf{Identifying Missing Variables:} The simplest but most powerful trick. In Problem 18, noting $y$ is missing immediately classifies it as a cylinder along $y$.
\end{itemize}

\section{Part 4: Cheatsheet and Tips}

\subsection*{Summary of Signs (Right Hand Side = 1)}
\begin{tabular}{@{}llll@{}}
\toprule
\textbf{Signs of $x^2, y^2, z^2$} & \textbf{Surface} & \textbf{Axis/Orientation} \\ \midrule
$+, +, +$ & Ellipsoid & Longest axis $\leftrightarrow$ smallest coeff \\
$+, +, -$ & Hyperboloid (1 Sheet) & Axis corresponds to the \textbf{negative} variable \\
$+, -, -$ & Hyperboloid (2 Sheets) & Axis corresponds to the \textbf{positive} variable \\
$+, +, 0$ (RHS=0) & Elliptic Cone & Axis corresponds to variable alone on one side \\
Linear $z$, $+, +$ & Elliptic Paraboloid & Axis is the linear variable \\
Linear $z$, $+,-$ & Hyperbolic Paraboloid & Saddle shape \\
Missing Variable & Cylinder & Parallel to missing axis \\ \bottomrule
\end{tabular}

\subsection*{Tips for Success}
\begin{itemize}
    \item \textbf{The "Finger Trick":} If you want the trace in the $xy$-plane ($z=0$), literally cover the $z$ term with your finger. The remaining equation is the trace.
    \item \textbf{Axis of Symmetry:}
    \begin{itemize}
        \item Paraboloids: The variable to the first power ($z = x^2+y^2 \to$ axis $z$).
        \item Hyperboloid 1-Sheet: The variable with the negative sign.
        \item Hyperboloid 2-Sheets: The variable with the positive sign.
    \end{itemize}
\end{itemize}

\newpage

\section{Part 5: Conceptual Synthesis}
\textbf{Thematic Connections:} This topic connects Algebra (conics) with Multivariable Calculus. Just as integration in Calc II used "slices" to find volumes, understanding surfaces via "traces" (slicing with planes) is the geometric foundation for \textbf{Triple Integrals} and \textbf{Partial Derivatives} which you will study next.

\textbf{Forward Links:}
\begin{itemize}
    \item \textbf{Partial Derivatives:} You will soon calculate $\frac{\partial z}{\partial x}$ for these surfaces, which represents the slope of the tangent line to the trace in the $xz$-plane.
    \item \textbf{Optimization:} You will find max/min points on these surfaces (the peaks of paraboloids or ellipsoids) using Lagrange Multipliers.
\end{itemize}

\section{Part 6: Real-World Application}
\textbf{Scenario: Satellite Dish Design (Paraboloid)}
Telecommunications engineers design satellite dishes as \textbf{Paraboloids of Revolution} ($z = c(x^2+y^2)$). Why? A property of parabolas is that all incoming parallel signals reflect to a single focus point.
\\
\textbf{Scenario: Nuclear Cooling Towers (Hyperboloid)}
Cooling towers are \textbf{Hyperboloids of One Sheet}. This shape is structurally strong (ruled surface using straight steel beams) and aids airflow for cooling.

\section{Part 7: Common Variations (Untested Concepts)}
\textbf{Missing Concept: The Hyperbolic Paraboloid (Saddle)}
Your homework covered the Elliptic Paraboloid but barely touched the Hyperbolic one.
\\
\textbf{Equation:} $z = y^2 - x^2$.
\\
\textbf{Example:} Sketch traces for $z = y^2 - x^2$.
\begin{itemize}
    \item $z=0$: $y^2 - x^2 = 0 \implies y = \pm x$ (Intersecting Lines).
    \item $x=0$: $z = y^2$ (Parabola opening up).
    \item $y=0$: $z = -x^2$ (Parabola opening down).
\end{itemize}
This opposing concavity creates a "saddle" point at the origin.

\section{Part 8: Advanced Diagnostic Testing ("Find the Flaw")}

\textbf{Problem A:} Identify the surface $x^2 + y^2 - z = 0$.
\\
\textit{Flawed Solution:} The equation has $x^2$ and $y^2$ positive and $z$ negative. Since there is one negative sign, it is a Hyperboloid of One Sheet.
\\
\textbf{Error:} The student treated the linear $z$ as a squared term.
\\
\textbf{Correction:} The variable $z$ is to the first power. This is an \textbf{Elliptic Paraboloid}.

\textbf{Problem B:} Describe the trace of $x^2 + y^2 - z^2 = 1$ at $z=1$.
\\
\textit{Flawed Solution:} At $z=1$, $x^2 + y^2 - 1 = 1 \implies x^2 + y^2 = 0$. This is a circle of radius 0 (a point).
\\
\textbf{Error:} Algebraic mistake moving the constant.
\\
\textbf{Correction:} $x^2 + y^2 = 1 + 1 = 2$. It is a \textbf{Circle} of radius $\sqrt{2}$.

\textbf{Problem C:} Identify $x^2 = y^2 + z^2$.
\\
\textit{Flawed Solution:} Rewrite as $x^2 - y^2 - z^2 = 0$. Two negative signs means Hyperboloid of Two Sheets.
\\
\textbf{Error:} The constant on the right is 0, not 1.
\\
\textbf{Correction:} Homogeneous equations (RHS=0) represent \textbf{Cones}. This is a Cone along the x-axis.

\end{document}