\documentclass[11pt]{article}
\usepackage{amsmath, amssymb, geometry}
\geometry{margin=1in}

\title{Homework 12.2 Vectors}
\author{Tashfeen Omran}
\date{January 2026}

\begin{document}

\maketitle

\tableofcontents
\newpage

\section{Comprehensive Introduction, Context, and Prerequisites}

\subsection{Core Concepts}

Vectors are fundamental mathematical objects that possess both \textit{magnitude} (size or length) and \textit{direction}. Unlike scalars, which are simple numbers representing only magnitude, vectors encode directional information, making them essential for describing physical quantities like velocity, force, and displacement.

\subsubsection{Fundamental Definitions}

\textbf{Scalar:} A quantity that has only magnitude. Examples include temperature ($25^\circ$C), mass ($5$ kg), and time ($3$ seconds).

\textbf{Vector:} A quantity that has both magnitude and direction. Examples include velocity ($50$ mph northeast), force ($10$ N downward), and displacement ($3$ meters to the right).

\textbf{Notation:} Vectors are typically denoted by boldface letters ($\mathbf{v}$), letters with arrows ($\vec{v}$), or in component form $\langle a, b \rangle$ or $\langle a, b, c \rangle$.

\textbf{Components:} In two dimensions, a vector $\mathbf{v} = \langle a, b \rangle$ has horizontal component $a$ and vertical component $b$. In three dimensions, $\mathbf{v} = \langle a, b, c \rangle$ has components along the $x$, $y$, and $z$ axes.

\textbf{Standard Basis Vectors:}
\begin{itemize}
    \item In 2D: $\mathbf{i} = \langle 1, 0 \rangle$ and $\mathbf{j} = \langle 0, 1 \rangle$
    \item In 3D: $\mathbf{i} = \langle 1, 0, 0 \rangle$, $\mathbf{j} = \langle 0, 1, 0 \rangle$, and $\mathbf{k} = \langle 0, 0, 1 \rangle$
\end{itemize}

Any vector can be written as a linear combination of basis vectors: $\mathbf{v} = a\mathbf{i} + b\mathbf{j} + c\mathbf{k} = \langle a, b, c \rangle$.

\subsection{Intuition and Derivation}

\subsubsection{Why Vectors?}

Consider describing the motion of an airplane. Simply stating "the plane is traveling at 500 mph" is incomplete—we need to know the direction. A vector naturally packages both pieces of information: speed (magnitude) and heading (direction).

\subsubsection{Geometric vs. Algebraic Perspective}

\textbf{Geometric:} A vector is visualized as an arrow. The length of the arrow represents magnitude, and the arrow points in the direction of the vector. Two vectors are equal if they have the same length and direction, regardless of their position in space.

\textbf{Algebraic:} A vector is represented by its components. This representation allows us to perform algebraic operations easily.

\subsubsection{Finding Vector from Two Points}

If $A = (x_1, y_1, z_1)$ and $B = (x_2, y_2, z_2)$, the vector from $A$ to $B$ is:
\[
\overrightarrow{AB} = \langle x_2 - x_1, y_2 - y_1, z_2 - z_1 \rangle
\]

This formula comes from the idea that to get from $A$ to $B$, we need to move $(x_2 - x_1)$ units in the $x$-direction, $(y_2 - y_1)$ units in the $y$-direction, and $(z_2 - z_1)$ units in the $z$-direction.

\subsubsection{Magnitude Derivation}

The magnitude (or length) of a vector $\mathbf{v} = \langle a, b, c \rangle$ comes from the three-dimensional Pythagorean theorem:
\[
|\mathbf{v}| = \sqrt{a^2 + b^2 + c^2}
\]

In two dimensions: $|\mathbf{v}| = \sqrt{a^2 + b^2}$.

This represents the distance from the origin to the point $(a, b, c)$ in space.

\subsection{Historical Context and Motivation}

The concept of vectors emerged gradually throughout the 18th and 19th centuries as mathematicians and physicists grappled with problems involving quantities that possessed both magnitude and direction. Early work on complex numbers by mathematicians like Caspar Wessel and Jean-Robert Argand in the late 1700s provided a two-dimensional geometric interpretation of numbers, planting the seeds for vector thinking.

The formal development of vector analysis is largely credited to William Rowan Hamilton, who in 1843 discovered quaternions while trying to extend complex numbers to three dimensions. Though quaternions themselves proved less practical for many applications, Hamilton's work led directly to the modern concept of vectors. Simultaneously, Hermann Grassmann was developing his own theory of "extension" which included vector-like objects. The system we use today was largely synthesized and popularized by Josiah Willard Gibbs and Oliver Heaviside in the 1880s, who recognized that many physical quantities in mechanics, electricity, and magnetism naturally required both magnitude and direction to be fully described. This mathematical framework became indispensable for formulating physical laws in a coordinate-independent manner, revolutionizing physics and engineering.

\subsection{Key Formulas}

\subsubsection{Vector from Two Points}
\[
\overrightarrow{AB} = \langle x_2 - x_1, y_2 - y_1, z_2 - z_1 \rangle
\]

\subsubsection{Magnitude}
\[
|\mathbf{v}| = \sqrt{a^2 + b^2 + c^2} \quad \text{for } \mathbf{v} = \langle a, b, c \rangle
\]

\subsubsection{Vector Addition}
\[
\mathbf{a} + \mathbf{b} = \langle a_1 + b_1, a_2 + b_2, a_3 + b_3 \rangle
\]

\subsubsection{Scalar Multiplication}
\[
c\mathbf{v} = \langle ca, cb, cc \rangle \quad \text{for } \mathbf{v} = \langle a, b, c \rangle
\]

\subsubsection{Unit Vector}
\[
\mathbf{u} = \frac{\mathbf{v}}{|\mathbf{v}|} = \frac{1}{|\mathbf{v}|}\mathbf{v}
\]

\subsubsection{Angle with Positive $x$-axis (2D)}
For $\mathbf{v} = \langle a, b \rangle$:
\[
\theta = \arctan\left(\frac{b}{a}\right) \quad \text{(adjust for quadrant)}
\]

\subsubsection{Vector from Magnitude and Direction}
If $|\mathbf{v}| = r$ and the angle with the positive $x$-axis is $\theta$:
\[
\mathbf{v} = \langle r\cos\theta, r\sin\theta \rangle
\]

\subsection{Prerequisites}

To successfully work with vectors, you should be comfortable with:

\begin{enumerate}
    \item \textbf{Coordinate Geometry:} Understanding points in 2D and 3D space, the distance formula, and the midpoint formula.
    \item \textbf{Pythagorean Theorem:} Both in 2D and its extension to 3D.
    \item \textbf{Trigonometry:} Right triangle trigonometry, the unit circle, and inverse trigonometric functions (especially $\arctan$).
    \item \textbf{Algebra:} Manipulating expressions, solving equations, and working with radicals.
    \item \textbf{Basic Geometry:} Understanding parallel lines, parallelograms, and triangle properties.
\end{enumerate}

\section{Detailed Homework Solutions}

\subsection{Problem 1: Scalars vs. Vectors}

\textbf{Problem:} Are the following quantities vectors or scalars?

\subsubsection{Part (a): The cost of a theater ticket}

\textbf{Solution:}

The cost of a theater ticket is a \textbf{scalar} because it has \textbf{only magnitude (no direction)}.

A cost is simply a numerical value (e.g., \$15) with no directional component.

\subsubsection{Part (b): The current in a river}

\textbf{Solution:}

The current in a river is a \textbf{vector} because it has \textbf{both magnitude and direction}.

The current has a speed (magnitude) and flows in a particular direction (e.g., northeast at $3$ m/s).

\subsubsection{Part (c): The initial flight path from Houston to Dallas}

\textbf{Solution:}

The initial flight path from Houston to Dallas is a \textbf{vector} because it has \textbf{both magnitude and direction}.

The flight path has a distance (magnitude) and a specific heading or bearing (direction).

\subsubsection{Part (d): The population of the world}

\textbf{Solution:}

The population of the world is a \textbf{scalar} because it has \textbf{only magnitude (no direction)}.

Population is simply a count (e.g., 8 billion people) with no directional component.

\subsection{Problem 2: Equal Vectors in a Parallelogram}

\textbf{Problem:} Name all the equal vectors in the parallelogram shown.

\textbf{Solution:}

In a parallelogram, opposite sides are parallel and equal in length. Vectors are equal if they have the same magnitude and direction.

Let's denote the parallelogram with vertices at points, and the vectors along the sides.

Based on parallelogram properties:

\begin{itemize}
    \item $\overrightarrow{AB}$ is equal to $\overrightarrow{DC}$
    \item $\overrightarrow{CB}$ is equal to $\overrightarrow{DE}$ (assuming $E$ is positioned such that this forms a parallelogram pattern)
    \item $\overrightarrow{EB}$ is equal to $\overrightarrow{??}$ (depends on the specific diagram)
    \item $\overrightarrow{DC}$ is equal to $\overrightarrow{AB}$
\end{itemize}

Without the specific diagram, the general principle is: vectors along opposite sides of a parallelogram are equal.

\textbf{Standard Answer Pattern:}
\begin{itemize}
    \item $\overrightarrow{AB} = \overrightarrow{DC}$
    \item $\overrightarrow{CB} = \overrightarrow{??}$ (parallel side)
    \item $\overrightarrow{EB} = \overrightarrow{??}$ (depends on point $E$)
    \item $\overrightarrow{DC} = \overrightarrow{AB}$
\end{itemize}

\subsection{Problem 3: Expressing Vectors in Terms of Others}

\textbf{Problem:} The tip of $\mathbf{c}$ and the tail of $\mathbf{d}$ are both the midpoint of $QR$. Express $\mathbf{c}$ and $\mathbf{d}$ in terms of $\mathbf{a}$ and $\mathbf{b}$.

\textbf{Solution:}

Let's denote the starting point as $P$, and let $\mathbf{a} = \overrightarrow{PQ}$ and $\mathbf{b} = \overrightarrow{PR}$.

The midpoint $M$ of $QR$ divides the segment into two equal parts.

The position of $M$ from $P$ is:
\[
\overrightarrow{PM} = \overrightarrow{PQ} + \overrightarrow{QM}
\]

Since $M$ is the midpoint of $QR$:
\[
\overrightarrow{QM} = \frac{1}{2}\overrightarrow{QR} = \frac{1}{2}(\overrightarrow{PR} - \overrightarrow{PQ}) = \frac{1}{2}(\mathbf{b} - \mathbf{a})
\]

Therefore:
\[
\overrightarrow{PM} = \mathbf{a} + \frac{1}{2}(\mathbf{b} - \mathbf{a}) = \mathbf{a} + \frac{1}{2}\mathbf{b} - \frac{1}{2}\mathbf{a} = \frac{1}{2}\mathbf{a} + \frac{1}{2}\mathbf{b} = \frac{1}{2}(\mathbf{a} + \mathbf{b})
\]

If $\mathbf{c}$ ends at $M$:
\[
\boxed{\mathbf{c} = \frac{1}{2}(\mathbf{a} + \mathbf{b})}
\]

For $\mathbf{d}$, which starts at $M$ and goes to $R$:
\[
\mathbf{d} = \overrightarrow{MR} = \overrightarrow{PR} - \overrightarrow{PM} = \mathbf{b} - \frac{1}{2}(\mathbf{a} + \mathbf{b}) = \mathbf{b} - \frac{1}{2}\mathbf{a} - \frac{1}{2}\mathbf{b} = -\frac{1}{2}\mathbf{a} + \frac{1}{2}\mathbf{b} = \frac{1}{2}(\mathbf{b} - \mathbf{a})
\]

\[
\boxed{\mathbf{d} = \frac{1}{2}(\mathbf{b} - \mathbf{a})}
\]

\subsection{Problem 4: Vector from Two Points}

\textbf{Problem:} Find a vector $\mathbf{a}$ with representation given by the directed line segment $\overrightarrow{AB}$ where $A(-3, 1)$ and $B(5, 4)$.

\textbf{Solution:}

Using the formula for the vector from point $A$ to point $B$:
\[
\overrightarrow{AB} = \langle x_2 - x_1, y_2 - y_1 \rangle
\]

Substituting the coordinates:
\[
\overrightarrow{AB} = \langle 5 - (-3), 4 - 1 \rangle = \langle 5 + 3, 3 \rangle = \langle 8, 3 \rangle
\]

\[
\boxed{\mathbf{a} = \langle 8, 3 \rangle}
\]

The equivalent representation starting at the origin is the vector from $(0, 0)$ to $(8, 3)$.

\subsection{Problem 5: Vector from Two Points}

\textbf{Problem:} Find a vector $\mathbf{a}$ with representation given by the directed line segment $\overrightarrow{AB}$ where $A(1, 3)$ and $B(0, 9)$.

\textbf{Solution:}

Using the formula:
\[
\overrightarrow{AB} = \langle x_2 - x_1, y_2 - y_1 \rangle
\]

Substituting:
\[
\overrightarrow{AB} = \langle 0 - 1, 9 - 3 \rangle = \langle -1, 6 \rangle
\]

\[
\boxed{\mathbf{a} = \langle -1, 6 \rangle}
\]

The equivalent representation starting at the origin is the vector from $(0, 0)$ to $(-1, 6)$.

\subsection{Problem 6: Vector Addition}

\textbf{Problem:} Find the sum of the given vectors $\mathbf{a} = \langle 2, -2 \rangle$ and $\mathbf{b} = \langle -1, 8 \rangle$.

\textbf{Solution:}

Vector addition is performed component-wise:
\[
\mathbf{a} + \mathbf{b} = \langle a_1 + b_1, a_2 + b_2 \rangle
\]

Substituting:
\[
\mathbf{a} + \mathbf{b} = \langle 2 + (-1), -2 + 8 \rangle = \langle 1, 6 \rangle
\]

\[
\boxed{\mathbf{a} + \mathbf{b} = \langle 1, 6 \rangle}
\]

\textbf{Geometric Illustration:} Place the tail of $\mathbf{b}$ at the tip of $\mathbf{a}$. The sum $\mathbf{a} + \mathbf{b}$ is the vector from the tail of $\mathbf{a}$ to the tip of $\mathbf{b}$. Alternatively, form a parallelogram with $\mathbf{a}$ and $\mathbf{b}$ as adjacent sides; the diagonal from the common starting point is $\mathbf{a} + \mathbf{b}$.

\subsection{Problem 7: 3D Vector Addition and Graphing}

\textbf{Problem:} Find the sum of the given vectors $\mathbf{a} = \langle 1, 0, 4 \rangle$ and $\mathbf{b} = \langle 0, 5, 0 \rangle$. Illustrate geometrically.

\textbf{Solution:}

Component-wise addition:
\[
\mathbf{a} + \mathbf{b} = \langle 1 + 0, 0 + 5, 4 + 0 \rangle = \langle 1, 5, 4 \rangle
\]

\textbf{Geometric Illustration:}

$\mathbf{a}$ starts at $(0, 0, 0)$ and ends at $(1, 0, 4)$.

$\mathbf{b}$ starts at $(1, 0, 4)$ (tip of $\mathbf{a}$) and ends at $(1, 5, 4)$.

$\mathbf{a} + \mathbf{b}$ starts at $(0, 0, 0)$ and ends at $(1, 5, 4)$.

\textbf{Answers for the fill-in-the-blank:}
\begin{itemize}
    \item $\mathbf{a}$ starts at $(x, y, z) = (0, 0, 0)$ and ends at $(x, y, z) = (1, 0, 4)$
    \item $\mathbf{b}$ starts at $(x, y, z) = (1, 0, 4)$ and ends at $(x, y, z) = (1, 5, 4)$
    \item $\mathbf{a} + \mathbf{b}$ starts at $(x, y, z) = (0, 0, 0)$ and ends at $(x, y, z) = (1, 5, 4)$
\end{itemize}

\subsection{Problem 8: Vector Operations}

\textbf{Problem:} Find $\mathbf{a} + \mathbf{b}$, $8\mathbf{a} + 9\mathbf{b}$, $|\mathbf{a}|$, and $|\mathbf{a} - \mathbf{b}|$ where $\mathbf{a} = \langle -6, 8 \rangle$ and $\mathbf{b} = \langle 6, 3 \rangle$.

\textbf{Solution:}

\textbf{(a)} $\mathbf{a} + \mathbf{b}$:
\[
\mathbf{a} + \mathbf{b} = \langle -6 + 6, 8 + 3 \rangle = \langle 0, 11 \rangle
\]

\[
\boxed{\mathbf{a} + \mathbf{b} = \langle 0, 11 \rangle}
\]

\textbf{(b)} $8\mathbf{a} + 9\mathbf{b}$:
\[
8\mathbf{a} = 8\langle -6, 8 \rangle = \langle -48, 64 \rangle
\]
\[
9\mathbf{b} = 9\langle 6, 3 \rangle = \langle 54, 27 \rangle
\]
\[
8\mathbf{a} + 9\mathbf{b} = \langle -48 + 54, 64 + 27 \rangle = \langle 6, 91 \rangle
\]

\[
\boxed{8\mathbf{a} + 9\mathbf{b} = \langle 6, 91 \rangle}
\]

\textbf{(c)} $|\mathbf{a}|$:
\[
|\mathbf{a}| = \sqrt{(-6)^2 + 8^2} = \sqrt{36 + 64} = \sqrt{100} = 10
\]

\[
\boxed{|\mathbf{a}| = 10}
\]

\textbf{(d)} $|\mathbf{a} - \mathbf{b}|$:

First, find $\mathbf{a} - \mathbf{b}$:
\[
\mathbf{a} - \mathbf{b} = \langle -6 - 6, 8 - 3 \rangle = \langle -12, 5 \rangle
\]

Then:
\[
|\mathbf{a} - \mathbf{b}| = \sqrt{(-12)^2 + 5^2} = \sqrt{144 + 25} = \sqrt{169} = 13
\]

\[
\boxed{|\mathbf{a} - \mathbf{b}| = 13}
\]

\subsection{Problem 9: 3D Vector Operations}

\textbf{Problem:} Find $\mathbf{a} + \mathbf{b}$, $8\mathbf{a} + 4\mathbf{b}$, $|\mathbf{a}|$, and $|\mathbf{a} - \mathbf{b}|$ where $\mathbf{a} = 9\mathbf{i} - 6\mathbf{j} + 4\mathbf{k}$ and $\mathbf{b} = 5\mathbf{i} - 8\mathbf{k}$.

\textbf{Solution:}

First, convert to component form:
\[
\mathbf{a} = \langle 9, -6, 4 \rangle, \quad \mathbf{b} = \langle 5, 0, -8 \rangle
\]

\textbf{(a)} $\mathbf{a} + \mathbf{b}$:
\[
\mathbf{a} + \mathbf{b} = \langle 9 + 5, -6 + 0, 4 + (-8) \rangle = \langle 14, -6, -4 \rangle
\]

\[
\boxed{\mathbf{a} + \mathbf{b} = \langle 14, -6, -4 \rangle = 14\mathbf{i} - 6\mathbf{j} - 4\mathbf{k}}
\]

\textbf{(b)} $8\mathbf{a} + 4\mathbf{b}$:
\[
8\mathbf{a} = 8\langle 9, -6, 4 \rangle = \langle 72, -48, 32 \rangle
\]
\[
4\mathbf{b} = 4\langle 5, 0, -8 \rangle = \langle 20, 0, -32 \rangle
\]
\[
8\mathbf{a} + 4\mathbf{b} = \langle 72 + 20, -48 + 0, 32 + (-32) \rangle = \langle 92, -48, 0 \rangle
\]

\[
\boxed{8\mathbf{a} + 4\mathbf{b} = \langle 92, -48, 0 \rangle = 92\mathbf{i} - 48\mathbf{j}}
\]

\textbf{(c)} $|\mathbf{a}|$:
\[
|\mathbf{a}| = \sqrt{9^2 + (-6)^2 + 4^2} = \sqrt{81 + 36 + 16} = \sqrt{133}
\]

\[
\boxed{|\mathbf{a}| = \sqrt{133}}
\]

\textbf{(d)} $|\mathbf{a} - \mathbf{b}|$:

First:
\[
\mathbf{a} - \mathbf{b} = \langle 9 - 5, -6 - 0, 4 - (-8) \rangle = \langle 4, -6, 12 \rangle
\]

Then:
\[
|\mathbf{a} - \mathbf{b}| = \sqrt{4^2 + (-6)^2 + 12^2} = \sqrt{16 + 36 + 144} = \sqrt{196} = 14
\]

\[
\boxed{|\mathbf{a} - \mathbf{b}| = 14}
\]

\subsection{Problem 10: Unit Vector}

\textbf{Problem:} Find a unit vector that has the same direction as the given vector $\langle 15, -12 \rangle$.

\textbf{Solution:}

First, find the magnitude of the vector:
\[
|\mathbf{v}| = \sqrt{15^2 + (-12)^2} = \sqrt{225 + 144} = \sqrt{369} = \sqrt{9 \cdot 41} = 3\sqrt{41}
\]

The unit vector is:
\[
\mathbf{u} = \frac{\mathbf{v}}{|\mathbf{v}|} = \frac{\langle 15, -12 \rangle}{3\sqrt{41}} = \left\langle \frac{15}{3\sqrt{41}}, \frac{-12}{3\sqrt{41}} \right\rangle = \left\langle \frac{5}{\sqrt{41}}, \frac{-4}{\sqrt{41}} \right\rangle
\]

Rationalizing:
\[
\mathbf{u} = \left\langle \frac{5\sqrt{41}}{41}, \frac{-4\sqrt{41}}{41} \right\rangle
\]

\[
\boxed{\mathbf{u} = \left\langle \frac{5\sqrt{41}}{41}, \frac{-4\sqrt{41}}{41} \right\rangle \text{ or } \left\langle \frac{5}{\sqrt{41}}, \frac{-4}{\sqrt{41}} \right\rangle}
\]

\subsection{Problem 11: Unit Vector in 3D}

\textbf{Problem:} Find a unit vector that has the same direction as $-4\mathbf{i} + 2\mathbf{j} - \mathbf{k}$.

\textbf{Solution:}

The vector is $\mathbf{v} = \langle -4, 2, -1 \rangle$.

Magnitude:
\[
|\mathbf{v}| = \sqrt{(-4)^2 + 2^2 + (-1)^2} = \sqrt{16 + 4 + 1} = \sqrt{21}
\]

Unit vector:
\[
\mathbf{u} = \frac{\mathbf{v}}{|\mathbf{v}|} = \frac{\langle -4, 2, -1 \rangle}{\sqrt{21}} = \left\langle \frac{-4}{\sqrt{21}}, \frac{2}{\sqrt{21}}, \frac{-1}{\sqrt{21}} \right\rangle
\]

Rationalized form:
\[
\mathbf{u} = \left\langle \frac{-4\sqrt{21}}{21}, \frac{2\sqrt{21}}{21}, \frac{-\sqrt{21}}{21} \right\rangle
\]

\[
\boxed{\mathbf{u} = \left\langle \frac{-4\sqrt{21}}{21}, \frac{2\sqrt{21}}{21}, \frac{-\sqrt{21}}{21} \right\rangle}
\]

\subsection{Problem 12: Angle with Positive $x$-axis}

\textbf{Problem:} What is the angle between the vector $\mathbf{i} + \sqrt{3}\mathbf{j}$ and the positive direction of the $x$-axis?

\textbf{Solution:}

The vector is $\mathbf{v} = \langle 1, \sqrt{3} \rangle$.

The angle $\theta$ with the positive $x$-axis is:
\[
\theta = \arctan\left(\frac{b}{a}\right) = \arctan\left(\frac{\sqrt{3}}{1}\right) = \arctan(\sqrt{3})
\]

Since $\tan(60^\circ) = \sqrt{3}$:
\[
\theta = 60^\circ
\]

\[
\boxed{\theta = 60^\circ}
\]

\subsection{Problem 13: Vector from Magnitude and Angle}

\textbf{Problem:} The initial point of a vector $\mathbf{v}$ in $\mathbb{R}^2$ is the origin and the terminal point is in quadrant II. If $\mathbf{v}$ makes an angle $\frac{5\pi}{6}$ with the positive $x$-axis and $|\mathbf{v}| = 8$, find $\mathbf{v}$ in component form.

\textbf{Solution:}

Given magnitude $r = 8$ and angle $\theta = \frac{5\pi}{6}$, the component form is:
\[
\mathbf{v} = \langle r\cos\theta, r\sin\theta \rangle
\]

Calculate:
\[
\cos\left(\frac{5\pi}{6}\right) = -\frac{\sqrt{3}}{2}, \quad \sin\left(\frac{5\pi}{6}\right) = \frac{1}{2}
\]

Therefore:
\[
\mathbf{v} = \left\langle 8 \cdot \left(-\frac{\sqrt{3}}{2}\right), 8 \cdot \frac{1}{2} \right\rangle = \langle -4\sqrt{3}, 4 \rangle
\]

\[
\boxed{\mathbf{v} = \langle -4\sqrt{3}, 4 \rangle}
\]

\subsection{Problem 14: Tangent Line to Parabola}

\textbf{Problem:} Find the unit vectors that are parallel to the tangent line to the parabola $y = x^2$ at the point $(5, 25)$.

\textbf{Solution:}

First, find the slope of the tangent line by taking the derivative:
\[
\frac{dy}{dx} = 2x
\]

At $x = 5$:
\[
\frac{dy}{dx}\bigg|_{x=5} = 2(5) = 10
\]

The slope is $m = 10$, so the direction vector of the tangent line is $\mathbf{v} = \langle 1, 10 \rangle$ (since for every $1$ unit in the $x$-direction, we move $10$ units in the $y$-direction).

Find the magnitude:
\[
|\mathbf{v}| = \sqrt{1^2 + 10^2} = \sqrt{1 + 100} = \sqrt{101}
\]

The unit vector in this direction is:
\[
\mathbf{u}_1 = \frac{\langle 1, 10 \rangle}{\sqrt{101}} = \left\langle \frac{1}{\sqrt{101}}, \frac{10}{\sqrt{101}} \right\rangle = \left\langle \frac{\sqrt{101}}{101}, \frac{10\sqrt{101}}{101} \right\rangle
\]

The opposite direction is:
\[
\mathbf{u}_2 = -\mathbf{u}_1 = \left\langle -\frac{\sqrt{101}}{101}, -\frac{10\sqrt{101}}{101} \right\rangle
\]

\textbf{Smaller $i$-component:}
\[
\boxed{\left\langle -\frac{\sqrt{101}}{101}, -\frac{10\sqrt{101}}{101} \right\rangle}
\]

\textbf{Larger $i$-component:}
\[
\boxed{\left\langle \frac{\sqrt{101}}{101}, \frac{10\sqrt{101}}{101} \right\rangle}
\]

\subsection{Problem 15: Vector Addition Example}

\textbf{Problem:} Draw the sum of vectors $\mathbf{a}$ and $\mathbf{b}$ shown in the figure.

\textbf{Solution:}

First, we translate $\mathbf{b}$ and place its tail at the tip of $\mathbf{a}$, being careful to draw a copy of $\mathbf{b}$ that has the same length and direction. Then we draw the vector $\mathbf{a} + \mathbf{b}$ starting at the \textbf{initial} point of $\mathbf{a}$ and ending at the terminal point of the copy of $\mathbf{b}$.

Alternatively, we could place $\mathbf{b}$ so it starts where $\mathbf{a}$ \textbf{starts} and construct $\mathbf{a} + \mathbf{b}$ by the Parallelogram Law.

\subsection{Problem 16: Midpoint Theorem Proof}

\textbf{Problem:} Use vectors to prove that the line joining the midpoints of two sides of a triangle is parallel to the third side and half its length.

\textbf{Solution:}

Consider triangle $ABC$ with points $D$ and $E$ as midpoints of $AB$ and $BC$, respectively.

By the definition of vector addition:
\[
(1) \quad \overrightarrow{AB} + \overrightarrow{BC} = \overrightarrow{AC}
\]
\[
(2) \quad \overrightarrow{DB} + \overrightarrow{BE} = \overrightarrow{DE}
\]

Now by the definition of midpoints:
\[
\overrightarrow{DB} = \frac{1}{2}\overrightarrow{AB} \quad \text{and} \quad \overrightarrow{BE} = \frac{1}{2}\overrightarrow{BC}
\]

Substituting these expressions into equation (2):
\[
\overrightarrow{DE} = \overrightarrow{DB} + \overrightarrow{BE} = \frac{1}{2}\overrightarrow{AB} + \frac{1}{2}\overrightarrow{BC}
\]

Factor out $\frac{1}{2}$:
\[
\overrightarrow{DE} = \frac{1}{2}(\overrightarrow{AB} + \overrightarrow{BC})
\]

From equation (1), $\overrightarrow{AB} + \overrightarrow{BC} = \overrightarrow{AC}$, so:
\[
\overrightarrow{DE} = \frac{1}{2}\overrightarrow{AC}
\]

This shows that $\overrightarrow{DE}$ is a scalar multiple of $\overrightarrow{AC}$, which means $\overrightarrow{DE}$ and $\overrightarrow{AC}$ are parallel. Moreover, the magnitude of $\overrightarrow{DE}$ is half the magnitude of $\overrightarrow{AC}$, as was to be shown.

\textbf{Fill-in answers:}
\begin{itemize}
    \item $(1)$ $\overrightarrow{AB} + \overrightarrow{BC} = \overrightarrow{AC}$
    \item $(2)$ $\overrightarrow{DB} + \overrightarrow{BE} = \overrightarrow{DE}$
    \item $\overrightarrow{DB} = \frac{1}{2}\overrightarrow{AB}$, $\overrightarrow{BE} = \frac{1}{2}\overrightarrow{BC}$
    \item Substituting gives: $\overrightarrow{DE} = \frac{1}{2}\overrightarrow{AB} + \frac{1}{2}\overrightarrow{BC}$
    \item $\overrightarrow{DE} = \frac{1}{2}\overrightarrow{AC}$
    \item Therefore $\overrightarrow{AC}$ and $\overrightarrow{DE}$ are parallel
\end{itemize}

\section{In-Depth Analysis of Problems and Techniques}

\subsection{Problem Types and General Approach}

\subsubsection{Type 1: Scalar vs. Vector Identification}

\textbf{Problems:} 1

\textbf{Strategy:} Ask two questions: (1) Does the quantity have magnitude? (2) Does the quantity have direction? If both yes, it's a vector. If only magnitude, it's a scalar.

\textbf{Examples of Scalars:} Temperature, mass, time, cost, population, distance (without direction)

\textbf{Examples of Vectors:} Velocity, force, displacement, acceleration, current (in a river)

\subsubsection{Type 2: Geometric Vector Equality}

\textbf{Problems:} 2

\textbf{Strategy:} Two vectors are equal if they have the same magnitude and direction, regardless of position. In parallelograms, opposite sides are equal vectors. Use properties of parallel sides.

\subsubsection{Type 3: Expressing Vectors in Terms of Others}

\textbf{Problems:} 3, 16

\textbf{Strategy:} Use vector addition/subtraction and properties like midpoints ($\overrightarrow{AM} = \frac{1}{2}\overrightarrow{AB}$ if $M$ is the midpoint of $AB$). Break complex paths into simpler vector combinations.

\subsubsection{Type 4: Finding Vectors from Coordinates}

\textbf{Problems:} 4, 5

\textbf{Strategy:} Use $\overrightarrow{AB} = \langle x_2 - x_1, y_2 - y_1, z_2 - z_1 \rangle$. The vector points from the first point to the second point.

\subsubsection{Type 5: Vector Arithmetic}

\textbf{Problems:} 6, 7, 8, 9

\textbf{Strategy:} Perform operations component-wise:
\begin{itemize}
    \item Addition: $\langle a_1, a_2, a_3 \rangle + \langle b_1, b_2, b_3 \rangle = \langle a_1 + b_1, a_2 + b_2, a_3 + b_3 \rangle$
    \item Scalar multiplication: $c\langle a_1, a_2, a_3 \rangle = \langle ca_1, ca_2, ca_3 \rangle$
    \item Magnitude: $|\mathbf{v}| = \sqrt{a_1^2 + a_2^2 + a_3^2}$
\end{itemize}

\subsubsection{Type 6: Unit Vectors}

\textbf{Problems:} 10, 11, 14

\textbf{Strategy:} 
\begin{enumerate}
    \item Find the magnitude $|\mathbf{v}|$
    \item Divide the vector by its magnitude: $\mathbf{u} = \frac{\mathbf{v}}{|\mathbf{v}|}$
    \item Rationalize denominators if needed
\end{enumerate}

For tangent lines, first find the derivative to get the slope, then create a direction vector.

\subsubsection{Type 7: Angle Problems}

\textbf{Problems:} 12, 13

\textbf{Strategy:} 
\begin{itemize}
    \item To find angle from vector: $\theta = \arctan\left(\frac{b}{a}\right)$ for $\mathbf{v} = \langle a, b \rangle$
    \item To find vector from angle and magnitude: $\mathbf{v} = \langle r\cos\theta, r\sin\theta \rangle$
    \item Pay attention to quadrant to adjust angle
\end{itemize}

\subsubsection{Type 8: Geometric Proofs with Vectors}

\textbf{Problems:} 16

\textbf{Strategy:} 
\begin{enumerate}
    \item Express all relevant quantities as vectors
    \item Use vector addition rules ($\overrightarrow{AB} + \overrightarrow{BC} = \overrightarrow{AC}$)
    \item Apply given properties (midpoint $\Rightarrow$ vector is half)
    \item Show the desired relationship through algebraic manipulation
\end{enumerate}

\subsection{Key Algebraic and Calculus Manipulations}

\subsubsection{Component-wise Operations}

All vector arithmetic happens component by component. This is the fundamental trick that makes vector calculations tractable.

\textbf{Example (Problem 8):} $8\mathbf{a} + 9\mathbf{b}$ with $\mathbf{a} = \langle -6, 8 \rangle$, $\mathbf{b} = \langle 6, 3 \rangle$:
\[
8\langle -6, 8 \rangle + 9\langle 6, 3 \rangle = \langle -48, 64 \rangle + \langle 54, 27 \rangle = \langle 6, 91 \rangle
\]

\subsubsection{Magnitude Calculation via Pythagorean Theorem}

The magnitude formula is the 3D Pythagorean theorem. Recognize perfect squares to simplify.

\textbf{Example (Problem 8):} $|\mathbf{a}| = \sqrt{(-6)^2 + 8^2} = \sqrt{36 + 64} = \sqrt{100} = 10$

This is a $3$-$4$-$5$ Pythagorean triple scaled by $2$.

\subsubsection{Factoring Out Scalars}

When proving geometric relationships, factor out common scalar multiples.

\textbf{Example (Problem 16):}
\[
\overrightarrow{DE} = \frac{1}{2}\overrightarrow{AB} + \frac{1}{2}\overrightarrow{BC} = \frac{1}{2}(\overrightarrow{AB} + \overrightarrow{BC}) = \frac{1}{2}\overrightarrow{AC}
\]

This shows $\overrightarrow{DE}$ is parallel to $\overrightarrow{AC}$ and half its length.

\subsubsection{Rationalizing Denominators}

Multiply by $\frac{\sqrt{n}}{\sqrt{n}}$ to eliminate radicals in denominators.

\textbf{Example (Problem 10):}
\[
\frac{5}{\sqrt{41}} = \frac{5}{\sqrt{41}} \cdot \frac{\sqrt{41}}{\sqrt{41}} = \frac{5\sqrt{41}}{41}
\]

\subsubsection{Trigonometric Values}

Memorize unit circle values for common angles:
\begin{itemize}
    \item $\cos\left(\frac{5\pi}{6}\right) = -\frac{\sqrt{3}}{2}$, $\sin\left(\frac{5\pi}{6}\right) = \frac{1}{2}$ (Problem 13)
    \item $\arctan(\sqrt{3}) = 60^\circ = \frac{\pi}{3}$ (Problem 12)
\end{itemize}

\subsubsection{Derivative as Slope}

For tangent line problems, the derivative gives the slope of the tangent line.

\textbf{Example (Problem 14):} For $y = x^2$ at $x = 5$:
\[
\frac{dy}{dx} = 2x \implies \text{slope at } x=5 \text{ is } m = 10
\]

Direction vector: $\langle 1, 10 \rangle$ (rise over run).

\subsubsection{Vector Subtraction for Finding Vectors Between Points}

$\overrightarrow{AB} = B - A$ component-wise.

\textbf{Example (Problem 4):} $A(-3, 1)$, $B(5, 4)$:
\[
\overrightarrow{AB} = \langle 5-(-3), 4-1 \rangle = \langle 8, 3 \rangle
\]

\subsubsection{Midpoint Halving}

If $M$ is the midpoint of $AB$, then $\overrightarrow{AM} = \frac{1}{2}\overrightarrow{AB}$.

\textbf{Example (Problem 3, 16):} This property is crucial for expressing vectors through midpoints.

\section{Cheatsheet and Tips for Success}

\subsection{Essential Formulas}

\begin{enumerate}
    \item \textbf{Vector from points:} $\overrightarrow{AB} = \langle x_2-x_1, y_2-y_1, z_2-z_1 \rangle$
    \item \textbf{Magnitude:} $|\mathbf{v}| = \sqrt{a^2 + b^2 + c^2}$
    \item \textbf{Unit vector:} $\mathbf{u} = \frac{\mathbf{v}}{|\mathbf{v}|}$
    \item \textbf{Vector from angle:} $\mathbf{v} = \langle r\cos\theta, r\sin\theta \rangle$
    \item \textbf{Angle from vector:} $\theta = \arctan\left(\frac{b}{a}\right)$
\end{enumerate}

\subsection{Quick Tricks and Shortcuts}

\begin{enumerate}
    \item \textbf{Pythagorean Triples:} Recognize $(3,4,5)$, $(5,12,13)$, $(8,15,17)$ and their multiples for quick magnitude calculations.
    \item \textbf{Unit Circle Mastery:} Memorize $\cos\theta$ and $\sin\theta$ for $0, \frac{\pi}{6}, \frac{\pi}{4}, \frac{\pi}{3}, \frac{\pi}{2}$ and their multiples.
    \item \textbf{Component-wise Everything:} Always work component-by-component for addition, subtraction, and scalar multiplication.
    \item \textbf{Zero Vector Check:} If components sum to zero after addition, you've likely found a special geometric property.
    \item \textbf{Parallel Vectors:} Two vectors are parallel if one is a scalar multiple of the other: $\mathbf{v} = k\mathbf{w}$.
\end{enumerate}

\subsection{Common Pitfalls and Traps}

\begin{enumerate}
    \item \textbf{Order Matters in Subtraction:} $\overrightarrow{AB} \neq \overrightarrow{BA}$. The vector goes FROM the first point TO the second.
    \item \textbf{Magnitude is Always Positive:} $|\mathbf{v}| \geq 0$. Don't forget the square root!
    \item \textbf{Angle Quadrant:} $\arctan$ gives values in $(-\frac{\pi}{2}, \frac{\pi}{2})$. Adjust for quadrants II, III, IV.
    \item \textbf{Unit Vector Check:} After finding a unit vector, verify $|\mathbf{u}| = 1$.
    \item \textbf{Rationalize or Don't:} Some problems want rationalized denominators, others don't. Read carefully.
    \item \textbf{$\mathbf{i}, \mathbf{j}, \mathbf{k}$ Notation:} $5\mathbf{i} - 8\mathbf{k} = \langle 5, 0, -8 \rangle$. Don't forget the zero for missing components!
    \item \textbf{Scalar vs. Vector Result:} Magnitude gives a scalar (number), not a vector.
\end{enumerate}

\subsection{Problem Recognition Guide}

\begin{itemize}
    \item \textbf{"Find the vector from $A$ to $B$"} $\Rightarrow$ Use $\overrightarrow{AB} = B - A$
    \item \textbf{"Find a unit vector"} $\Rightarrow$ Calculate $\frac{\mathbf{v}}{|\mathbf{v}|}$
    \item \textbf{"Express in terms of..."} $\Rightarrow$ Use vector addition/subtraction chains
    \item \textbf{"Angle with $x$-axis"} $\Rightarrow$ Use $\arctan\left(\frac{b}{a}\right)$ and adjust quadrant
    \item \textbf{"Given magnitude and angle"} $\Rightarrow$ Use $\langle r\cos\theta, r\sin\theta \rangle$
    \item \textbf{"Tangent line to curve"} $\Rightarrow$ Find derivative for slope, create direction vector
    \item \textbf{"Prove using vectors"} $\Rightarrow$ Express everything as vectors, use properties
\end{itemize}

\section{Conceptual Synthesis and The Big Picture}

\subsection{Thematic Connections}

\subsubsection{Core Theme: Direction Matters}

The fundamental theme of vector mathematics is that \textit{direction is a first-class citizen}. Unlike the real number line where we only care about "how much," vectors encode "how much and which way." This parallels:

\begin{itemize}
    \item \textbf{Complex Numbers (Earlier):} $z = a + bi$ also has two components and geometric interpretation. Complex numbers are 2D vectors with additional multiplication structure.
    \item \textbf{Parametric Equations (Earlier):} $x = f(t)$, $y = g(t)$ describe curves by giving position vectors at each time $t$.
    \item \textbf{Functions of Several Variables (Later):} Input becomes multi-dimensional, requiring vector notation.
\end{itemize}

\subsubsection{Algebraic Structure}

Vectors form a \textit{vector space}: you can add them and multiply by scalars, and these operations satisfy nice properties (associativity, commutativity, distributivity). This algebraic structure appears throughout mathematics:

\begin{itemize}
    \item \textbf{Linear Algebra:} Vectors, matrices, and linear transformations
    \item \textbf{Differential Equations:} Solutions form vector spaces
    \item \textbf{Functional Analysis:} Functions themselves can be treated as vectors
\end{itemize}

\subsection{Forward and Backward Links}

\subsubsection{Backward: Building on Foundations}

\textbf{From Coordinate Geometry:} Plotting points, distance formula, and midpoint formula naturally extend to vector operations. The distance formula $d = \sqrt{(x_2-x_1)^2 + (y_2-y_1)^2}$ is exactly vector magnitude.

\textbf{From Trigonometry:} Converting between polar and Cartesian coordinates ($r, \theta \leftrightarrow x, y$) is precisely the relationship between a vector's magnitude/direction and its components.

\textbf{From Algebra:} Vector addition and scalar multiplication use the same algebraic properties (distributive, associative) as real numbers, making calculations familiar.

\subsubsection{Forward: Essential for Future Topics}

\textbf{Dot Product and Cross Product (Next):} These operations combine vectors to produce scalars (dot) or new vectors (cross), enabling calculations of angles, projections, areas, and volumes.

\textbf{Vector-Valued Functions (Soon):} Functions $\mathbf{r}(t) = \langle x(t), y(t), z(t) \rangle$ describe curves in space. Derivatives $\mathbf{r}'(t)$ give velocity vectors, and integrals give displacement.

\textbf{Multivariable Calculus:} Partial derivatives, gradients, directional derivatives, and line integrals all fundamentally rely on vector notation and operations.

\textbf{Differential Equations:} Systems of ODEs are naturally expressed using vector notation: $\mathbf{x}' = A\mathbf{x}$.

\textbf{Physics and Engineering:} Nearly every physical quantity (force, momentum, electric field, magnetic field) is a vector, making this foundational for all applied mathematics.

\section{Real-World Application and Modeling}

\subsection{Concrete Scenarios in Finance and Economics}

\subsubsection{Scenario 1: Portfolio Allocation Vector}

An investment portfolio can be represented as a vector where each component represents the proportion invested in different asset classes. For example, $\mathbf{p} = \langle 0.4, 0.3, 0.2, 0.1 \rangle$ means 40\% stocks, 30\% bonds, 20\% real estate, 10\% cash.

Vector operations model rebalancing: if market movements change your portfolio to $\mathbf{p}_{\text{current}}$ and you want to return to target allocation $\mathbf{p}_{\text{target}}$, the rebalancing vector is:
\[
\mathbf{r} = \mathbf{p}_{\text{target}} - \mathbf{p}_{\text{current}}
\]

The magnitude $|\mathbf{r}|$ measures how far you've drifted from target allocation.

\subsubsection{Scenario 2: Multi-Factor Risk Model}

In quantitative finance, asset returns are often modeled as depending on multiple risk factors (market risk, size factor, value factor, momentum). The exposure of a stock to these factors can be represented as a vector.

For instance, stock $A$ might have factor exposures $\mathbf{f}_A = \langle 1.2, 0.8, -0.3, 0.5 \rangle$ representing its sensitivities to four factors. The expected return is then:
\[
E[R_A] = \mathbf{f}_A \cdot \mathbf{r}_{\text{factors}}
\]

where $\mathbf{r}_{\text{factors}}$ is the vector of expected factor returns (this uses the dot product, coming in the next section).

\subsubsection{Scenario 3: State-Space Models in Econometrics}

Economic systems are often modeled using state-space representations where the state of the economy is a vector $\mathbf{x}_t = \langle \text{GDP}_t, \text{inflation}_t, \text{unemployment}_t, \text{interest rate}_t \rangle$.

The evolution of the economy is modeled as:
\[
\mathbf{x}_{t+1} = A\mathbf{x}_t + \mathbf{w}_t
\]

where $A$ is a matrix describing relationships between variables and $\mathbf{w}_t$ is a shock vector. Forecasting future states requires tracking vectors through time.

\subsection{Model Problem Setup}

\textbf{Chosen Scenario:} Portfolio Drift and Rebalancing

\textbf{Setup:}

An investor has a target portfolio allocation across $n = 5$ asset classes:
\[
\mathbf{p}_{\text{target}} = \langle 0.35, 0.25, 0.20, 0.15, 0.05 \rangle
\]

representing 35\% stocks, 25\% bonds, 20\% international equities, 15\% commodities, and 5\% cash.

After 6 months of market movements, the current portfolio has drifted to:
\[
\mathbf{p}_{\text{current}} = \langle 0.42, 0.22, 0.18, 0.12, 0.06 \rangle
\]

\textbf{Questions to Answer:}

1. What is the rebalancing vector $\mathbf{r} = \mathbf{p}_{\text{target}} - \mathbf{p}_{\text{current}}$?

2. What is the magnitude of drift $|\mathbf{r}|$?

3. If the investor has a total portfolio value of $V = \$1{,}000{,}000$, how many dollars need to be moved in each asset class?

\textbf{Mathematical Model:}

\[
\mathbf{r} = \mathbf{p}_{\text{target}} - \mathbf{p}_{\text{current}} = \langle 0.35-0.42, 0.25-0.22, 0.20-0.18, 0.15-0.12, 0.05-0.06 \rangle
\]

\[
|\mathbf{r}| = \sqrt{\sum_{i=1}^{5} r_i^2}
\]

\[
\mathbf{D} = V \cdot \mathbf{r} = \text{dollar amounts to rebalance}
\]

This model uses vector subtraction to find the rebalancing needed, vector magnitude to quantify overall drift, and scalar multiplication to convert proportions to dollar amounts.

\section{Common Variations and Untested Concepts}

\subsection{Concept 1: Projections of Vectors}

\textbf{Description:} The projection of vector $\mathbf{a}$ onto vector $\mathbf{b}$ is the "shadow" of $\mathbf{a}$ in the direction of $\mathbf{b}$.

\textbf{Formula:}
\[
\text{proj}_{\mathbf{b}}\mathbf{a} = \frac{\mathbf{a} \cdot \mathbf{b}}{|\mathbf{b}|^2}\mathbf{b}
\]

(This requires the dot product, covered in section 12.3)

\textbf{Example:} Project $\mathbf{a} = \langle 3, 4 \rangle$ onto $\mathbf{b} = \langle 1, 0 \rangle$.

\[
\mathbf{a} \cdot \mathbf{b} = 3(1) + 4(0) = 3
\]
\[
|\mathbf{b}|^2 = 1^2 + 0^2 = 1
\]
\[
\text{proj}_{\mathbf{b}}\mathbf{a} = \frac{3}{1}\langle 1, 0 \rangle = \langle 3, 0 \rangle
\]

This is the horizontal component of $\mathbf{a}$.

\subsection{Concept 2: Distance from Point to Line}

\textbf{Description:} Finding the perpendicular distance from a point to a line in space uses vector projections.

\textbf{Method:} If point $P$ is at position $\mathbf{p}$ and the line passes through $Q$ with direction $\mathbf{d}$:
\begin{enumerate}
    \item Find vector $\mathbf{v} = \mathbf{p} - \mathbf{q}$ from $Q$ to $P$
    \item Project $\mathbf{v}$ onto $\mathbf{d}$: this is the parallel component
    \item The perpendicular distance is $|\mathbf{v} - \text{proj}_{\mathbf{d}}\mathbf{v}|$
\end{enumerate}

\textbf{Example:} Find distance from $P(3, 4)$ to the line through origin with direction $\langle 1, 1 \rangle$.

\[
\mathbf{v} = \langle 3, 4 \rangle - \langle 0, 0 \rangle = \langle 3, 4 \rangle
\]
\[
\mathbf{d} = \langle 1, 1 \rangle
\]
\[
\text{proj}_{\mathbf{d}}\mathbf{v} = \frac{\langle 3,4\rangle \cdot \langle 1,1\rangle}{|\langle 1,1\rangle|^2}\langle 1,1\rangle = \frac{7}{2}\langle 1,1\rangle = \left\langle \frac{7}{2}, \frac{7}{2}\right\rangle
\]
\[
\mathbf{v} - \text{proj}_{\mathbf{d}}\mathbf{v} = \left\langle 3-\frac{7}{2}, 4-\frac{7}{2}\right\rangle = \left\langle -\frac{1}{2}, \frac{1}{2}\right\rangle
\]
\[
\text{Distance} = \left|\left\langle -\frac{1}{2}, \frac{1}{2}\right\rangle\right| = \sqrt{\frac{1}{4} + \frac{1}{4}} = \frac{\sqrt{2}}{2}
\]

\subsection{Concept 3: Linear Combinations and Span}

\textbf{Description:} Any vector that can be written as $c_1\mathbf{v}_1 + c_2\mathbf{v}_2 + \cdots + c_n\mathbf{v}_n$ for scalars $c_i$ is a linear combination of the vectors $\mathbf{v}_i$.

The \textit{span} of a set of vectors is the set of all possible linear combinations.

\textbf{Example:} Can $\mathbf{w} = \langle 5, 1 \rangle$ be written as a linear combination of $\mathbf{v}_1 = \langle 2, 1 \rangle$ and $\mathbf{v}_2 = \langle 1, -1 \rangle$?

We need to find $c_1$ and $c_2$ such that:
\[
c_1\langle 2, 1 \rangle + c_2\langle 1, -1 \rangle = \langle 5, 1 \rangle
\]

This gives the system:
\[
\begin{cases}
2c_1 + c_2 = 5 \\
c_1 - c_2 = 1
\end{cases}
\]

Adding the equations: $3c_1 = 6 \Rightarrow c_1 = 2$

Substituting: $2 - c_2 = 1 \Rightarrow c_2 = 1$

Therefore: $\mathbf{w} = 2\mathbf{v}_1 + 1\mathbf{v}_2$

\subsection{Concept 4: Parametric Equations of Lines}

\textbf{Description:} A line through point $P_0$ with direction vector $\mathbf{d}$ can be described parametrically as:
\[
\mathbf{r}(t) = \mathbf{r}_0 + t\mathbf{d}
\]

where $\mathbf{r}_0$ is the position vector of $P_0$ and $t \in \mathbb{R}$.

\textbf{Example:} Find the parametric equation of the line through $P(2, -1, 3)$ parallel to $\mathbf{d} = \langle 1, 2, -2 \rangle$.

\[
\mathbf{r}(t) = \langle 2, -1, 3 \rangle + t\langle 1, 2, -2 \rangle = \langle 2+t, -1+2t, 3-2t \rangle
\]

Or in component form:
\[
x = 2 + t, \quad y = -1 + 2t, \quad z = 3 - 2t
\]

\subsection{Concept 5: Vector Equations of Planes}

\textbf{Description:} A plane through point $P_0$ with two direction vectors $\mathbf{u}$ and $\mathbf{v}$ (not parallel) can be described as:
\[
\mathbf{r}(s, t) = \mathbf{r}_0 + s\mathbf{u} + t\mathbf{v}
\]

\textbf{Example:} Find the vector equation of the plane through origin with direction vectors $\mathbf{u} = \langle 1, 0, 1 \rangle$ and $\mathbf{v} = \langle 0, 1, 1 \rangle$.

\[
\mathbf{r}(s, t) = \langle 0, 0, 0 \rangle + s\langle 1, 0, 1 \rangle + t\langle 0, 1, 1 \rangle = \langle s, t, s+t \rangle
\]

\section{Advanced Diagnostic Testing: Find the Flaw}

\subsection{Problem 1: Finding a Vector from Two Points}

\textbf{Given:} Find the vector from $A(2, 5)$ to $B(-1, 3)$.

\textbf{Flawed Solution:}

Using the formula $\overrightarrow{AB} = \langle x_1 - x_2, y_1 - y_2 \rangle$:
\[
\overrightarrow{AB} = \langle 2 - (-1), 5 - 3 \rangle = \langle 3, 2 \rangle
\]

\textbf{Your Task:} Find the error, explain why it's wrong, and provide the correct solution.

\vspace{1cm}

\textbf{Error:} The components are subtracted in the wrong order; the formula should be $(x_2 - x_1, y_2 - y_1)$, not $(x_1 - x_2, y_1 - y_2)$.

\textbf{Correct Solution:}
\[
\overrightarrow{AB} = \langle -1 - 2, 3 - 5 \rangle = \langle -3, -2 \rangle
\]

\subsection{Problem 2: Finding a Unit Vector}

\textbf{Given:} Find a unit vector in the direction of $\mathbf{v} = \langle 3, -4 \rangle$.

\textbf{Flawed Solution:}

The magnitude is:
\[
|\mathbf{v}| = \sqrt{3^2 + (-4)^2} = \sqrt{9 + 16} = \sqrt{25} = 5
\]

The unit vector is:
\[
\mathbf{u} = \frac{\mathbf{v}}{|\mathbf{v}|} = \langle 3, -4 \rangle - 5 = \langle -2, -9 \rangle
\]

\textbf{Your Task:} Find the error, explain why it's wrong, and provide the correct solution.

\vspace{1cm}

\textbf{Error:} Division by a scalar means dividing each component, not subtracting the scalar from the vector.

\textbf{Correct Solution:}
\[
\mathbf{u} = \frac{1}{5}\langle 3, -4 \rangle = \left\langle \frac{3}{5}, -\frac{4}{5} \right\rangle
\]

\subsection{Problem 3: Vector Magnitude After Operations}

\textbf{Given:} If $\mathbf{a} = \langle 1, 2 \rangle$ and $\mathbf{b} = \langle 3, 1 \rangle$, find $|\mathbf{a} + \mathbf{b}|$.

\textbf{Flawed Solution:}

First find the magnitudes:
\[
|\mathbf{a}| = \sqrt{1^2 + 2^2} = \sqrt{5}
\]
\[
|\mathbf{b}| = \sqrt{3^2 + 1^2} = \sqrt{10}
\]

Therefore:
\[
|\mathbf{a} + \mathbf{b}| = |\mathbf{a}| + |\mathbf{b}| = \sqrt{5} + \sqrt{10}
\]

\textbf{Your Task:} Find the error, explain why it's wrong, and provide the correct solution.

\vspace{1cm}

\textbf{Error:} The magnitude of a sum is not equal to the sum of magnitudes; you must first add the vectors, then find the magnitude.

\textbf{Correct Solution:}
\[
\mathbf{a} + \mathbf{b} = \langle 1+3, 2+1 \rangle = \langle 4, 3 \rangle
\]
\[
|\mathbf{a} + \mathbf{b}| = \sqrt{4^2 + 3^2} = \sqrt{16 + 9} = \sqrt{25} = 5
\]

\subsection{Problem 4: Angle with the $x$-axis}

\textbf{Given:} Find the angle that $\mathbf{v} = \langle -2, 2 \rangle$ makes with the positive $x$-axis.

\textbf{Flawed Solution:}

Using the formula:
\[
\theta = \arctan\left(\frac{2}{-2}\right) = \arctan(-1) = -45^\circ
\]

\textbf{Your Task:} Find the error, explain why it's wrong, and provide the correct solution.

\vspace{1cm}

\textbf{Error:} The vector is in quadrant II (negative $x$, positive $y$), but $\arctan(-1) = -45^\circ$ corresponds to quadrant IV; we need to adjust for the correct quadrant.

\textbf{Correct Solution:}

Since the vector is in quadrant II:
\[
\theta = 180^\circ + \arctan\left(\frac{2}{-2}\right) = 180^\circ - 45^\circ = 135^\circ
\]

Or using radians: $\theta = \frac{3\pi}{4}$

\subsection{Problem 5: Parallel Vectors}

\textbf{Given:} Determine if $\mathbf{a} = \langle 6, -9 \rangle$ and $\mathbf{b} = \langle -4, 6 \rangle$ are parallel.

\textbf{Flawed Solution:}

Two vectors are parallel if they have the same magnitude. Let's check:
\[
|\mathbf{a}| = \sqrt{6^2 + (-9)^2} = \sqrt{36 + 81} = \sqrt{117} = 3\sqrt{13}
\]
\[
|\mathbf{b}| = \sqrt{(-4)^2 + 6^2} = \sqrt{16 + 36} = \sqrt{52} = 2\sqrt{13}
\]

Since $|\mathbf{a}| \neq |\mathbf{b}|$, the vectors are not parallel.

\textbf{Your Task:} Find the error, explain why it's wrong, and provide the correct solution.

\vspace{1cm}

\textbf{Error:} Parallel vectors must be scalar multiples of each other, not have the same magnitude; vectors can be parallel with different lengths.

\textbf{Correct Solution:}

Check if $\mathbf{a} = k\mathbf{b}$ for some scalar $k$:
\[
\langle 6, -9 \rangle = k\langle -4, 6 \rangle
\]

From the first component: $6 = -4k \Rightarrow k = -\frac{3}{2}$

Check the second: $-9 = 6k = 6(-\frac{3}{2}) = -9$ \checkmark

Yes, $\mathbf{a} = -\frac{3}{2}\mathbf{b}$, so they are parallel (pointing in opposite directions).

\section{Complete LaTeX Document}

The document you are currently reading is the complete LaTeX source. To use it:

\begin{enumerate}
    \item Copy all content from \texttt{\textbackslash documentclass} to \texttt{\textbackslash end\{document\}}
    \item Save as a \texttt{.tex} file
    \item Compile with any LaTeX compiler (pdfLaTeX, XeLaTeX, etc.)
\end{enumerate}

\end{document}