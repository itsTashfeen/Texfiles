\documentclass[12pt]{article}
\usepackage[utf8]{inputenc}
\usepackage{amsmath, amssymb, amsthm}
\usepackage{geometry}
\usepackage{graphicx}
\usepackage{enumitem}
\usepackage{fancyhdr}

% Page Geometry
\geometry{a4paper, margin=1in}

% Header
\pagestyle{fancy}
\fancyhead[L]{Calculus III}
\fancyhead[R]{Functions of Several Variables}

\title{\textbf{Practice Set: Functions of Several Variables}\\
\large Based on Homework 14.1 Study Guide}
\author{Comprehensive Review \& Problem Set}
\date{}

\begin{document}

\maketitle

\section*{Concept Check List}
Before beginning the problems, review the following concepts derived from the study guide. These are the key areas tested in this problem set.

\begin{enumerate}
    \item \textbf{Function Evaluation:} Substituting scalars and expressions into $f(x,y)$.
    \item \textbf{Domain - Rational Functions:} Denominators $\neq 0$.
    \item \textbf{Domain - Radical Functions:} Arguments of even roots $\ge 0$.
    \item \textbf{Domain - Logarithmic Functions:} Arguments of logarithms $> 0$.
    \item \textbf{Domain - Visualizing Regions:} Identifying lines, circles, parabolas, and ellipses as boundaries; determining solid vs. dashed lines.
    \item \textbf{Domain - 3 Variables:} Extending domain rules to $f(x,y,z)$.
    \item \textbf{Range Analysis:} Determining possible output values (using properties of squares, exponentials, and roots).
    \item \textbf{Level Curves (Contour Maps):} Finding equations for $z=k$ and identifying shapes (Circles, Ellipses, Hyperbolas, Lines).
    \item \textbf{Topographic Intuition:} Relating contour spacing to steepness (Gradient intuition).
    \item \textbf{Traces \& Cross-Sections:} Fixing one variable to analyze 2D curves.
    \item \textbf{Surface Identification:} Matching equations to Quadric Surfaces (Paraboloids, Cones, Spheres, Ellipsoids) and Cylinders.
    \item \textbf{Applications:} Interpreting tabular data and word problems.
\end{enumerate}

\newpage

\section*{Problem Set}

\subsection*{Part 1: Evaluation}

\textbf{1.} Given $f(x, y) = \frac{x^3 + 2y}{x - y}$, evaluate $f(2, 1)$.

\textbf{2.} Given $g(x, y) = x^2 e^{xy}$, find $g(2, -0.5)$.

\textbf{3.} Given $h(x, y) = 3x^2 - y$, find and simplify the expression for $h(x+t, y)$.

\textbf{4.} Given $f(x, y) = \sqrt{x^2 + y^2}$, find the value of $f(t, t)$ assuming $t > 0$.

\subsection*{Part 2: Domain and Range}
\textit{Determine the domain of the following functions. Describe the region geometrically (e.g., "inside the circle," "above the line"). For sketching, note whether boundaries are solid or dashed.}

\textbf{5.} $f(x, y) = \frac{x + y}{x^2 + y^2 - 16}$

\textbf{6.} $f(x, y) = \sqrt{2x - y}$

\textbf{7.} $f(x, y) = \ln(y - x^2)$

\textbf{8.} $f(x, y) = \frac{\sqrt{x}}{y^2 - 1}$

\textbf{9.} $f(x, y) = \sqrt[4]{xy}$

\textbf{10.} $f(x, y) = \ln(16 - x^2 - y^2)$

\textbf{11.} $f(x, y) = \frac{1}{\sqrt{x + y - 2}}$

\textbf{12.} $f(x, y) = e^{\sqrt{1 - x^2}}$

\textbf{13.} $f(x, y, z) = \ln(z - x^2 - y^2)$

\textbf{14.} $g(x, y, z) = \sqrt{36 - 4x^2 - 9y^2 - z^2}$

\textbf{15.} \textbf{(Range)} Find the range of $z = 5 - \sqrt{x^2 + y^2}$.

\textbf{16.} \textbf{(Range)} Find the range of $z = e^{- (x^2 + y^2)}$.

\subsection*{Part 3: Level Curves and Contour Maps}
\textit{For the following functions, describe the geometric shape of the level curves $f(x,y) = k$.}

\textbf{17.} $z = x^2 + 4y^2$

\textbf{18.} $z = y - x^2$

\textbf{19.} $z = \sqrt{x^2 + y^2}$

\textbf{20.} $z = xy$

\textbf{21.} $z = \frac{y}{x}$

\textbf{22.} Consider the function $f(x, y) = \sqrt{100 - x^2 - y^2}$.
\begin{enumerate}[label=(\alph*)]
    \item Determine the domain and range.
    \item Sketch the level curves for $k = 0, 6, 8$.
    \item As $k$ increases from 0 to 10, do the level curves get closer together or farther apart? What does this imply about the shape of the surface?
\end{enumerate}

\textbf{23.} Which of the following functions corresponds to a contour map consisting of parallel, equally spaced straight lines?
\begin{enumerate}[label=(\Alph*)]
    \item $z = x^2 + y$
    \item $z = 3x - 2y + 5$
    \item $z = xy$
    \item $z = \sin(x)$
\end{enumerate}

\subsection*{Part 4: Surface Identification and Traces}

\textbf{24.} Identify the surface defined by $z = x^2$. Explain why the variable $y$ is missing and what that implies about the graph.

\textbf{25.} Match the equation $x^2 + y^2 + z^2 = 25$ to its surface type. Describe its traces in the planes $z=0$, $z=3$, and $z=5$.

\textbf{26.} Consider the surface $z = x^2 - y^2$.
\begin{enumerate}[label=(\alph*)]
    \item What is the shape of the trace when $z=0$?
    \item What is the shape of the trace when $x=0$?
    \item What is the common name for this surface?
\end{enumerate}

\textbf{27.} Match the equation to the description:
\begin{itemize}
    \item \textbf{A:} $z = \sqrt{x^2+y^2}$
    \item \textbf{B:} $z = x^2 + y^2$
\end{itemize}
\textit{Description 1:} Level curves are equally spaced circles.
\textit{Description 2:} Level curves are circles that get closer together as $z$ increases.

\textbf{28.} Identify the surface $z = \sin(y)$. Describe the cross-sections for a fixed value of $x$.

\subsection*{Part 5: Applications and Interpretation}

\textbf{29.} \textbf{(Table Interpretation)} The "Heat Index" $I(T, h)$ is a function of temperature $T$ (in $^{\circ}$F) and humidity $h$ (\%).
\[
\begin{array}{c|ccc}
T \setminus h & 40 & 50 & 60 \\
\hline
80 & 80 & 81 & 82 \\
85 & 86 & 88 & 90 \\
90 & 91 & 95 & 100 \\
\end{array}
\]
\begin{enumerate}[label=(\alph*)]
    \item Evaluate $I(90, 50)$ and interpret its meaning.
    \item If $T=85$ is held constant, estimate the rate of change of the Heat Index as humidity increases from 50 to 60.
\end{enumerate}

\textbf{30.} \textbf{(Conceptual)} In a financial model, the cost $C$ is given by $C(x, y) = 1000 + 5x + 2y$, where $x$ is labor hours and $y$ is material units. Describe the shape of the level curves of this cost function. What does moving along a level curve represent in economic terms?

\newpage

\section*{Answer Key and Solutions}

\subsection*{Part 1: Evaluation}
\textbf{1. Answer: 10.}
Substitute $x=2, y=1$: $\frac{2^3 + 2(1)}{2 - 1} = \frac{8+2}{1} = 10$.

\textbf{2. Answer: $4e^{-1}$ or $4/e$.}
Substitute $x=2, y=-0.5$: $(2)^2 e^{2(-0.5)} = 4e^{-1}$.

\textbf{3. Answer: $3x^2 + 6xt + 3t^2 - y$.}
Substitute $x \to (x+t)$: $3(x+t)^2 - y = 3(x^2 + 2xt + t^2) - y$.

\textbf{4. Answer: $t\sqrt{2}$.}
$f(t,t) = \sqrt{t^2 + t^2} = \sqrt{2t^2} = |t|\sqrt{2}$. Since $t>0$, result is $t\sqrt{2}$.

\subsection*{Part 2: Domain and Range}
\textbf{5. Domain:} All real numbers except points on the circle $x^2 + y^2 = 16$.
\textit{Reason:} Denominator $\neq 0$.

\textbf{6. Domain:} $y \le 2x$. Region on or below the line $y=2x$.
\textit{Reason:} Radicand $\ge 0 \implies 2x - y \ge 0$. Solid boundary.

\textbf{7. Domain:} $y > x^2$. Region strictly above the parabola $y=x^2$.
\textit{Reason:} Log argument $> 0$. Dashed boundary.

\textbf{8. Domain:} $x \ge 0$ AND $y \neq 1, y \neq -1$.
\textit{Reason:} Square root $x \ge 0$ (right half-plane) minus the two horizontal lines $y=\pm 1$.

\textbf{9. Domain:} Quadrants I and III (including axes).
\textit{Reason:} Even root requires $xy \ge 0$. This happens if both $x,y \ge 0$ OR both $x,y \le 0$.

\textbf{10. Domain:} Interior of the circle $x^2 + y^2 < 16$.
\textit{Reason:} $16 - x^2 - y^2 > 0 \implies x^2 + y^2 < 16$. Dashed boundary radius 4.

\textbf{11. Domain:} $y > -x + 2$. Region strictly above the line $y = -x + 2$.
\textit{Reason:} Denom $\neq 0$ AND root $\ge 0 \implies x+y-2 > 0$.

\textbf{12. Domain:} $-1 \le x \le 1$. (Infinite strip in $y$-direction).
\textit{Reason:} Root requires $1-x^2 \ge 0 \implies x^2 \le 1$. $y$ can be anything.

\textbf{13. Domain:} Inside the paraboloid $z > x^2 + y^2$.
\textit{Reason:} Log argument $z - (x^2+y^2) > 0$.

\textbf{14. Domain:} Interior and surface of the ellipsoid $\frac{x^2}{9} + \frac{y^2}{4} + \frac{z^2}{36} \le 1$.
\textit{Reason:} Radicand $\ge 0$.

\textbf{15. Range:} $(-\infty, 5]$.
\textit{Reason:} $\sqrt{\dots} \ge 0$. Max value is $5 - 0 = 5$. No lower bound.

\textbf{16. Range:} $(0, 1]$.
\textit{Reason:} $x^2+y^2 \ge 0$, so $-(x^2+y^2) \le 0$. $e^{\text{negative}} \in (0, 1]$.

\subsection*{Part 3: Level Curves}
\textbf{17. Ellipses.} Equation $\frac{x^2}{k} + \frac{y^2}{k/4} = 1$ (for $k>0$).
\textbf{18. Parabolas.} $y = x^2 + k$.
\textbf{19. Circles.} $x^2+y^2 = k^2$ (Concentrically spaced).
\textbf{20. Hyperbolas.} $y = k/x$.
\textbf{21. Lines.} $y = kx$ (Lines passing through origin).
\textbf{22. (a)} Domain: Circle $r=10$. Range: $[0, 10]$.
\textbf{(b)} $k=0 \to r=10$; $k=6 \to r=8$; $k=8 \to r=6$.
\textbf{(c)} Closer together. The surface (hemisphere) gets steeper near the base ($z=0$).
\textbf{23. B.} This is a plane. Linear functions of $x$ and $y$ yield parallel linear contours.

\subsection*{Part 4: Surface Identification}
\textbf{24. Parabolic Cylinder.} $y$ is missing, meaning the curve $z=x^2$ is extended infinitely along the $y$-axis (like a trough).
\textbf{25. Sphere.} Center $(0,0,0)$ radius 5.
$z=0$: Circle radius 5. $z=3$: Circle radius 4. $z=5$: Point $(0,0)$.
\textbf{26. (a)} $x = \pm y$ (Intersecting lines). \textbf{(b)} $z = -y^2$ (Parabola opening down). \textbf{(c)} Hyperbolic Paraboloid (Saddle).
\textbf{27. A matches Desc 1} (Cone - linear slope). \textbf{B matches Desc 2} (Paraboloid - increasing slope).
\textbf{28. Sine Cylinder.} Wave shape extending along the x-axis. Fixed $x$ yields $z=\sin(y)$.

\subsection*{Part 5: Applications}
\textbf{29. (a)} 95. At $90^\circ$F and 50\% humidity, it feels like $95^\circ$F.
\textbf{(b)} Change is $90-88 = 2$ degrees over 10\% humidity. Rate $\approx 0.2$ deg/\%.
\textbf{30.} Lines ($5x + 2y = C - 1000$). Moving along a level curve means keeping total cost constant while trading labor for materials.

\newpage

\section*{Concept Check List Mapping}
\textit{This table maps the problems to the concepts listed at the start of the document.}

\begin{table}[h]
\centering
\begin{tabular}{|l|l|}
\hline
\textbf{Concept / Problem Type} & \textbf{Relevant Question \#} \\ \hline
1. Function Evaluation & 1, 2, 3, 4, 29(a) \\ \hline
2. Domain - Rational & 5, 11 \\ \hline
3. Domain - Radical & 6, 8, 9, 11, 12, 14 \\ \hline
4. Domain - Logarithmic & 7, 10, 13 \\ \hline
5. Domain - Visualizing Regions & 5, 6, 7, 10, 11, 12 \\ \hline
6. Domain - 3 Variables & 13, 14 \\ \hline
7. Range Analysis & 15, 16, 22(a) \\ \hline
8. Level Curves - Identification & 17, 18, 19, 20, 21, 23 \\ \hline
9. Topographic Intuition (Spacing) & 22(c), 27 \\ \hline
10. Traces \& Cross-Sections & 26, 28 \\ \hline
11. Surface Identification & 24, 25, 26, 27, 28 \\ \hline
12. Applications/Interpretation & 29, 30 \\ \hline
\end{tabular}
\end{table}

\end{document}