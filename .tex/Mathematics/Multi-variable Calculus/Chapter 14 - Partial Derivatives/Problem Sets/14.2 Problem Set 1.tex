\documentclass[12pt]{article}
\usepackage{amsmath, amssymb, amsthm}
\usepackage{graphicx}
\usepackage{enumitem}
\usepackage{geometry}
\usepackage{hyperref}

\geometry{margin=1in}

\title{Homework 14.2 Practice: Multivariable Limits and Continuity\\
\large Based on Homework 14.2 Study Guide}
\author{Calculus III Practice Set}
\date{}

\begin{document}

\maketitle

\section*{Part 1: Problem Set}

\textbf{Instructions:} Evaluate the following limits, determine continuity, or identify the set of points where the function is continuous. If a limit does not exist, state "DNE" and provide a brief justification (e.g., Two-Path Test).

\subsection*{Topic A: Direct Substitution and Algebraic Simplification}

\begin{enumerate}
    \item \textbf{Problem 1:} Find the limit using direct substitution.
    \[ \lim_{(x,y) \to (2, -1)} (x^3 y^2 - 4xy + 5) \]

    \item \textbf{Problem 2:} Find the limit involving trigonometric and exponential functions.
    \[ \lim_{(x,y) \to (\pi, 0)} e^{x} \cos(x + y) \]

    \item \textbf{Problem 3:} Find the limit. (Hint: Factor the numerator).
    \[ \lim_{(x,y) \to (2, 2)} \frac{x^2 - y^2}{2x - 2y} \]

    \item \textbf{Problem 4:} Find the limit. (Hint: Difference of cubes factoring).
    \[ \lim_{(x,y) \to (1, 1)} \frac{x^3 - y^3}{x - y} \]

    \item \textbf{Problem 5:} Find the limit. (Hint: Use the conjugate).
    \[ \lim_{(x,y) \to (0, 0)} \frac{\sqrt{x^2 + y^2 + 9} - 3}{x^2 + y^2} \]
\end{enumerate}

\subsection*{Topic B: The Two-Path Test (Proving Non-Existence)}

\begin{enumerate}[resume]
    \item \textbf{Problem 6:} Show that the limit does not exist by approaching along the x-axis and the y-axis.
    \[ \lim_{(x,y) \to (0,0)} \frac{3x - y}{3x + y} \]

    \item \textbf{Problem 7:} Show that the limit does not exist by approaching along the lines $y=x$ and $y=-x$.
    \[ \lim_{(x,y) \to (0,0)} \frac{xy}{2x^2 + 3y^2} \]

    \item \textbf{Problem 8:} Show that the limit does not exist. (Hint: Degrees are unequal. Try a linear path $y=mx$ vs a parabolic path $x=y^2$).
    \[ \lim_{(x,y) \to (0,0)} \frac{xy^2}{x^2 + y^4} \]

    \item \textbf{Problem 9:} Determine if the limit exists. (Hint: Consider the path $y = x^3$).
    \[ \lim_{(x,y) \to (0,0)} \frac{x^3 y}{x^6 + y^2} \]

    \item \textbf{Problem 10:} Determine if the limit exists.
    \[ \lim_{(x,y) \to (0,0)} \frac{x^4 - y^2}{x^4 + y^2} \]
\end{enumerate}

\subsection*{Topic C: Polar Coordinates and Squeeze Theorem}

\begin{enumerate}[resume]
    \item \textbf{Problem 11:} Use polar coordinates ($x = r\cos\theta, y=r\sin\theta$) to find the limit.
    \[ \lim_{(x,y) \to (0,0)} \frac{x^3 + y^3}{x^2 + y^2} \]

    \item \textbf{Problem 12:} Use polar coordinates to find the limit.
    \[ \lim_{(x,y) \to (0,0)} \frac{5x^2 y^2}{\sqrt{x^2 + y^2}} \]

    \item \textbf{Problem 13:} Use polar coordinates to find the limit.
    \[ \lim_{(x,y) \to (0,0)} (x^2 + y^2) \ln(x^2 + y^2) \]

    \item \textbf{Problem 14:} Use polar coordinates to show the limit DNE (dependence on $\theta$).
    \[ \lim_{(x,y) \to (0,0)} \frac{x^2}{x^2 + y^2} \]

    \item \textbf{Problem 15:} Find the limit using the Squeeze Theorem or Polar Coordinates.
    \[ \lim_{(x,y) \to (0,0)} \frac{x^2 \sin^2(y)}{x^2 + y^2} \]
\end{enumerate}

\subsection*{Topic D: Substitutions and "Single Variable" Limits}

\begin{enumerate}[resume]
    \item \textbf{Problem 16:} Evaluate by using a substitution $u = xy$.
    \[ \lim_{(x,y) \to (0,0)} \frac{\sin(xy)}{xy} \]

    \item \textbf{Problem 17:} Evaluate by using a substitution $u = x^2 + y^2$.
    \[ \lim_{(x,y) \to (0,0)} \frac{1 - \cos(x^2 + y^2)}{x^2 + y^2} \]

    \item \textbf{Problem 18:} Evaluate.
    \[ \lim_{(x,y) \to (1,1)} \frac{x^2 - xy}{\sqrt{x} - \sqrt{y}} \]
\end{enumerate}

\subsection*{Topic E: Continuity and Domain Analysis}

\begin{enumerate}[resume]
    \item \textbf{Problem 19:} Describe the set of points where the function is continuous.
    \[ f(x,y) = \frac{1}{x^2 - y} \]

    \item \textbf{Problem 20:} Describe the set of points where the function is continuous. (Focus on the square root and denominator).
    \[ f(x,y) = \frac{\sqrt{x + y}}{x - y} \]

    \item \textbf{Problem 21:} Describe the set of points where the function is continuous.
    \[ f(x,y) = \ln(4 - x^2 - y^2) \]

    \item \textbf{Problem 22:} Determine where the function is discontinuous.
    \[ f(x,y) = \begin{cases} \frac{x^2 y}{x^2 + y^2} & \text{if } (x,y) \neq (0,0) \\ 0 & \text{if } (x,y) = (0,0) \end{cases} \]

    \item \textbf{Problem 23:} Determine where the function is discontinuous.
    \[ f(x,y) = \begin{cases} \frac{x^2 - y^2}{x - y} & \text{if } x \neq y \\ 0 & \text{if } x = y \end{cases} \]

    \item \textbf{Problem 24:} Is the following function continuous at $(0,0)$?
    \[ g(x,y) = \begin{cases} \frac{x y}{x^2 + y^2} & \text{if } (x,y) \neq (0,0) \\ 0 & \text{if } (x,y) = (0,0) \end{cases} \]
\end{enumerate}

\subsection*{Topic F: Advanced Concepts (Definitions and Theory)}

\begin{enumerate}[resume]
    \item \textbf{Problem 25 (Conceptual):} To prove that $\lim_{(x,y) \to (0,0)} f(x,y) = 0$ using the definition, we must show that for every $\epsilon > 0$, there exists a $\delta > 0$ such that if $0 < \sqrt{x^2+y^2} < \delta$, then $|f(x,y)| < \epsilon$.
    If we find that $|f(x,y)| \le 2\sqrt{x^2+y^2}$, what value should we choose for $\delta$ in terms of $\epsilon$?

    \item \textbf{Problem 26 (Find the Flaw):} A student claims that because $\lim_{x \to 0} f(x,0) = 0$ and $\lim_{y \to 0} f(0,y) = 0$, the limit exists and equals 0. Explain why this reasoning is flawed.

    \item \textbf{Problem 27 (Composition):} Find the set of points where $h(x,y) = e^{1/(x^2+y^2)}$ is continuous. What happens to the limit as $(x,y) \to (0,0)$?

    \item \textbf{Problem 28 (Limits at Infinity):} Evaluate the limit or show it DNE by converting to polar and letting $r \to \infty$.
    \[ \lim_{(x,y) \to (\infty, \infty)} \frac{xy}{x^2 + y^2} \]

    \item \textbf{Problem 29 (3 Variables):} Find the limit.
    \[ \lim_{(x,y,z) \to (0,0,0)} \frac{x^2 + y^2 + z^2}{\sqrt{x^2 + y^2 + z^2 + 1} - 1} \]

    \item \textbf{Problem 30 (Removable vs. Essential):} Consider $f(x,y) = \frac{\sin(x^2+y^2)}{x^2+y^2}$ for $(x,y) \ne (0,0)$. How should $f(0,0)$ be defined to make the function continuous everywhere?
\end{enumerate}

\newpage
\section*{Part 2: Detailed Solutions}

\begin{enumerate}
    \item \textbf{Answer: 13.}
    Polynomials are continuous everywhere.
    $2^3(-1)^2 - 4(2)(-1) + 5 = 8(1) + 8 + 5 = 21$.

    \item \textbf{Answer: $-e^{\pi}$.}
    Continuous function. Substitution:
    $e^{\pi} \cos(\pi + 0) = e^{\pi} (-1) = -e^{\pi}$.

    \item \textbf{Answer: 2.}
    Factor numerator: $(x-y)(x+y)$. Factor denominator: $2(x-y)$.
    $\lim \frac{(x-y)(x+y)}{2(x-y)} = \lim \frac{x+y}{2} = \frac{2+2}{2} = 2$.

    \item \textbf{Answer: 3.}
    Difference of cubes: $x^3-y^3 = (x-y)(x^2+xy+y^2)$.
    $\lim \frac{(x-y)(x^2+xy+y^2)}{x-y} = 1^2 + (1)(1) + 1^2 = 3$.

    \item \textbf{Answer: 1/6.}
    Multiply by conjugate $\frac{\sqrt{u}+3}{\sqrt{u}+3}$.
    $\frac{(x^2+y^2+9)-9}{(x^2+y^2)(\sqrt{x^2+y^2+9}+3)} = \frac{x^2+y^2}{(x^2+y^2)(\dots)} = \frac{1}{\sqrt{0+9}+3} = \frac{1}{6}$.

    \item \textbf{Answer: DNE.}
    Along x-axis ($y=0$): $\lim \frac{3x}{3x} = 1$.
    Along y-axis ($x=0$): $\lim \frac{-y}{y} = -1$.
    $1 \neq -1$, so DNE.

    \item \textbf{Answer: DNE.}
    Along $y=x$: $\frac{x^2}{2x^2+3x^2} = \frac{1}{5}$.
    Along $y=-x$: $\frac{-x^2}{2x^2+3x^2} = -\frac{1}{5}$.
    Values differ.

    \item \textbf{Answer: DNE.}
    Along $x=0$: 0.
    Along $x=y^2$: $\frac{(y^2)(y^2)}{(y^2)^2 + y^4} = \frac{y^4}{2y^4} = \frac{1}{2}$.
    $0 \neq 1/2$.

    \item \textbf{Answer: DNE.}
    Along $y=0$: 0.
    Along $y=x^3$: $\frac{x^3(x^3)}{x^6+(x^3)^2} = \frac{x^6}{2x^6} = \frac{1}{2}$.
    $0 \neq 1/2$.

    \item \textbf{Answer: DNE.}
    Along x-axis ($y=0$): $\frac{x^4}{x^4} = 1$.
    Along y-axis ($x=0$): $\frac{-y^2}{y^2} = -1$.

    \item \textbf{Answer: 0.}
    $x=r\cos\theta, y=r\sin\theta$.
    $\frac{r^3(\cos^3\theta+\sin^3\theta)}{r^2} = r(\cos^3\theta+\sin^3\theta)$.
    As $r \to 0$, $0 \cdot (\text{bounded}) = 0$.

    \item \textbf{Answer: 0.}
    $\frac{5(r\cos\theta)^2(r\sin\theta)^2}{r} = \frac{5r^4 \cos^2\theta \sin^2\theta}{r} = 5r^3 \cos^2\theta \sin^2\theta$.
    As $r \to 0$, limit is 0.

    \item \textbf{Answer: 0.}
    $(r^2) \ln(r^2) = 2 r^2 \ln(r)$.
    Let $t = r$. $\lim_{t \to 0} t^2 \ln t = \lim \frac{\ln t}{t^{-2}}$. Use L'Hopital: $\frac{1/t}{-2t^{-3}} = \frac{t^3}{-2t} = -\frac{t^2}{2} \to 0$.

    \item \textbf{Answer: DNE.}
    $\frac{r^2 \cos^2\theta}{r^2} = \cos^2\theta$.
    The limit depends on the angle $\theta$. (e.g., 1 at $\theta=0$, 0 at $\theta=\pi/2$).

    \item \textbf{Answer: 0.}
    $0 \le \frac{x^2 \sin^2 y}{x^2+y^2} \le \frac{x^2 (1)}{x^2+y^2} \le 1$.
    Wait, better bound: $\frac{x^2}{x^2+y^2} \le 1$, so $|f| \le \sin^2 y$. As $y \to 0$, $\sin^2 y \to 0$.
    Limit is 0.

    \item \textbf{Answer: 1.}
    Let $u = xy$. As $(x,y) \to (0,0)$, $u \to 0$.
    $\lim_{u \to 0} \frac{\sin u}{u} = 1$.

    \item \textbf{Answer: 0.}
    Let $u = x^2+y^2$. $\lim_{u \to 0} \frac{1-\cos u}{u}$.
    L'Hopital: $\frac{\sin u}{1} \to 0$.

    \item \textbf{Answer: 0.}
    Factor top: $x(x-y) = x(\sqrt{x}-\sqrt{y})(\sqrt{x}+\sqrt{y})$.
    $\lim \frac{x(\sqrt{x}-\sqrt{y})(\sqrt{x}+\sqrt{y})}{\sqrt{x}-\sqrt{y}} = \lim x(\sqrt{x}+\sqrt{y})$.
    At $(1,1)$: $1(1+1) = 2$.
    Wait, re-calculation: $x(x-y) \to x^2-xy$.
    Top factors to $x(\sqrt{x}-\sqrt{y})(\sqrt{x}+\sqrt{y})$. Denom cancels.
    Result $x(\sqrt{x}+\sqrt{y})$. Plug in (1,1) $\to 1(1+1) = 2$.
    (Solution: 2).

    \item \textbf{Answer: Continuous everywhere except points where $y = x^2$.}
    Rational function is continuous where denominator $\ne 0$.

    \item \textbf{Answer: Continuous where $x+y \ge 0$ AND $x \ne y$.}
    Must satisfy square root domain and denominator non-zero.
    Set notation: $\{(x,y) | x+y \ge 0, x \neq y\}$.

    \item \textbf{Answer: Continuous where $x^2 + y^2 < 4$.}
    Log argument must be positive. This is the interior of a circle with radius 2.

    \item \textbf{Answer: Continuous everywhere (including origin).}
    Check limit at $(0,0)$ via Polar: $\frac{r^3 \cos^2\theta \sin\theta}{r^2} = r \cos^2\theta \sin\theta \to 0$.
    Limit (0) = Function Value (0).

    \item \textbf{Answer: Discontinuous along $y=x$.}
    Away from $y=x$, it simplifies to $x+y$. Limit as $x \to y$ is $2x$.
    However, at $x=y$, defined as 0.
    Limit is $x+x = 2x$. Function is 0. Unless $x=0$, $2x \ne 0$.
    Discontinuous at all points on line $y=x$ except possibly origin. At origin limit is 0, value is 0.
    Strictly speaking, the "formula" creates the discontinuity $x=y$.

    \item \textbf{Answer: No (Discontinuous).}
    Limit via polar: $\frac{r^2 \cos\theta \sin\theta}{r^2} = \cos\theta \sin\theta$. Depends on $\theta$.
    Limit DNE, so cannot be continuous.

    \item \textbf{Answer: $\delta = \epsilon / 2$.}
    We want $|f| < \epsilon$. We have $|f| \le 2\sqrt{x^2+y^2}$.
    So we need $2\sqrt{x^2+y^2} < \epsilon \implies \sqrt{x^2+y^2} < \epsilon/2$.
    Since $\sqrt{x^2+y^2} < \delta$, set $\delta = \epsilon/2$.

    \item \textbf{Answer: The "Two-Path" Trap.}
    Checking specific paths (like axes) only proves non-existence if they differ. If they match, it proves nothing. There are infinite directions to approach from (spirals, parabolas, etc.).

    \item \textbf{Answer: Continuous for $(x,y) \neq (0,0)$.}
    At origin: as $(x,y) \to (0,0)$, $x^2+y^2 \to 0^+$, so $1/(x^2+y^2) \to \infty$.
    $e^{\infty} \to \infty$. Limit DNE (Infinite).

    \item \textbf{Answer: DNE.}
    Polar: $x \to \infty$ becomes $r \to \infty$.
    Expression: $\cos\theta \sin\theta$.
    This oscillates based on angle $\theta$. DNE.

    \item \textbf{Answer: 2.}
    Rationalize: Multiply by $\sqrt{...}+1$.
    $\frac{(x^2+y^2+z^2)(\sqrt{...}+1)}{x^2+y^2+z^2+1-1} = \sqrt{x^2+y^2+z^2+1} + 1$.
    Sub $(0,0,0) \to \sqrt{1}+1 = 2$.

    \item \textbf{Answer: $f(0,0) = 1$.}
    Use polar/substitution $u = r^2$. $\lim_{u \to 0} \frac{\sin u}{u} = 1$.
    Defining $f(0,0)=1$ makes it continuous.
\end{enumerate}

\newpage
\section*{Part 3: Concept Check List & Matrix}

The following matrix maps the problems in this set to the concepts defined in the Homework 14.2 PDF.

\begin{center}
\begin{tabular}{|p{8cm}|p{5cm}|}
\hline
\textbf{Concept / Problem Type} & \textbf{Problem Numbers} \\
\hline
\textbf{1. Direct Substitution} & \\
- Simple Polynomials & 1 \\
- Trig/Exponentials & 2 \\
\hline
\textbf{2. Algebraic Manipulation} & \\
- Factoring (Diff of Squares/Cubes) & 3, 4, 18 \\
- Conjugates & 5, 29 \\
\hline
\textbf{3. Two-Path Test (Proving DNE)} & \\
- Axis Paths & 6, 10 \\
- Linear Paths ($y=mx$) & 6, 7 \\
- Curved/Parabolic Paths ($y=x^2$) & 8, 9 \\
\hline
\textbf{4. Polar Coordinates / Squeeze Theorem} & \\
- Existence (Limit is 0) & 11, 12, 13, 15, 22 \\
- Non-Existence (Angular Dependence) & 14, 24, 28 \\
\hline
\textbf{5. Substitutions ($u=xy, u=r^2$)} & 16, 17, 30 \\
\hline
\textbf{6. Continuity Analysis} & \\
- Domain (Rational functions) & 19, 20, 23 \\
- Domain (Logarithms/Roots) & 20, 21 \\
- Piecewise Functions & 22, 23, 24, 30 \\
\hline
\textbf{7. Advanced / Conceptual} & \\
- Epsilon-Delta Logic & 25 \\
- "Find the Flaw" (Diagnostic) & 26 \\
- Composition Limits & 27 \\
- Limits at Infinity & 28 \\
- 3-Variable Limits & 29 \\
\hline
\end{tabular}
\end{center}

\end{document}