\documentclass[12pt]{article}
\usepackage{amsmath}
\usepackage{amssymb}
\usepackage{geometry}
\usepackage{enumitem}
\geometry{margin=1in}

\title{Calculus III - Section 14.1\\Functions of Several Variables\\Problem Set}
\author{Comprehensive Practice Problems}
\date{January 2026}

\begin{document}

\maketitle

\section*{Instructions}
This problem set contains 30 problems covering all concepts from Section 14.1: Functions of Several Variables. Work through each problem carefully, showing all steps. A concept reference guide is provided at the end.

\section{Function Evaluation}

\textbf{Problem 1.} Given $f(x, y) = \frac{x^2 - 3y}{2x + y^2}$, evaluate:
\begin{enumerate}[label=(\alph*)]
    \item $f(2, 1)$
    \item $f(-1, 3)$
    \item $f(0, -2)$
\end{enumerate}

\textbf{Problem 2.} Given $g(x, y) = x^3y - 2xy^2 + 5$, find:
\begin{enumerate}[label=(\alph*)]
    \item $g(1, 2)$
    \item $g(-2, -1)$
    \item $g(x+h, y)$
    \item $g(2t, t)$
\end{enumerate}

\textbf{Problem 3.} For $h(x, y, z) = e^{xy} + \ln(z - x)$, evaluate:
\begin{enumerate}[label=(\alph*)]
    \item $h(0, 5, 3)$
    \item $h(1, 2, 5)$
    \item $h(-1, 3, 2)$
\end{enumerate}

\textbf{Problem 4.} Given $f(x, y) = \sqrt{x^2 + y^2} \cdot \sin(\frac{\pi x}{y})$, find:
\begin{enumerate}[label=(\alph*)]
    \item $f(3, 4)$
    \item $f(1, 2)$
    \item $f(y, y)$
\end{enumerate}

\section{Domain Problems - Two Variables}

\textbf{Problem 5.} Find and sketch the domain of $f(x, y) = \ln(x + 2y - 4)$.

\textbf{Problem 6.} Find and sketch the domain of $g(x, y) = \sqrt{16 - x^2 - y^2}$.

\textbf{Problem 7.} Determine the domain of $h(x, y) = \frac{1}{\sqrt{y - x^2}}$ and sketch it in the $xy$-plane.

\textbf{Problem 8.} Find the domain of $f(x, y) = \frac{x - y}{x^2 + y^2 - 9}$ and describe it using inequalities.

\textbf{Problem 9.} Find and sketch the domain of $k(x, y) = \sqrt[3]{2x + 5y}$. (Hint: Odd roots have different restrictions than even roots)

\textbf{Problem 10.} Determine the domain of $f(x, y) = \ln(x^2 - y) + \sqrt{y - 1}$.

\textbf{Problem 11.} Find the domain of $g(x, y) = \frac{\sqrt{x}}{y - 3}$.

\section{Domain and Range - Combined}

\textbf{Problem 12.} For $f(x, y) = e^{\sqrt{4 - x^2 - y^2}}$:
\begin{enumerate}[label=(\alph*)]
    \item Find and sketch the domain
    \item Determine the range
\end{enumerate}

\textbf{Problem 13.} For $g(x, y) = \frac{xy}{x^2 + y^2 + 1}$:
\begin{enumerate}[label=(\alph*)]
    \item State the domain
    \item Find the range (Hint: Consider extreme values)
\end{enumerate}

\textbf{Problem 14.} For $h(x, y) = \arcsin(x + y)$:
\begin{enumerate}[label=(\alph*)]
    \item Find the domain
    \item State the range
\end{enumerate}

\section{Three Variable Functions}

\textbf{Problem 15.} Find the domain of $f(x, y, z) = \ln(16 - x^2 - y^2 - 4z^2)$ and describe the geometric shape.

\textbf{Problem 16.} Determine the domain of $g(x, y, z) = \frac{1}{\sqrt{z - \sqrt{x^2 + y^2}}}$ and write the inequality that defines it.

\textbf{Problem 17.} For $h(x, y, z) = \sqrt{x^2 + y^2 + z^2 - 25}$:
\begin{enumerate}[label=(\alph*)]
    \item Evaluate $h(3, 4, 0)$
    \item Find and describe the domain geometrically
\end{enumerate}

\textbf{Problem 18.} Find the domain of $f(x, y, z) = \ln(z) + \sqrt{9 - x^2 - y^2}$ and describe it in words.

\section{Level Curves and Contour Maps}

\textbf{Problem 19.} For $f(x, y) = 4x^2 + 9y^2$, find the level curves for $k = 0, 36, 144$ and identify the type of curves.

\textbf{Problem 20.} Sketch several level curves of $f(x, y) = xy$ for $k = -4, -1, 0, 1, 4$. What type of curves are these?

\textbf{Problem 21.} For $g(x, y) = y - x^3$, sketch the level curves for $k = -1, 0, 1, 2$.

\textbf{Problem 22.} Match the function $f(x, y) = \cos(x)\sin(y)$ with its contour map description:
\begin{enumerate}[label=(\alph*)]
    \item Concentric circles
    \item Parallel lines
    \item Rectangular grid pattern
    \item Hyperbolas
\end{enumerate}

\textbf{Problem 23.} Two contour maps are shown for $z = \sqrt{x^2 + y^2}$ (cone) and $z = \frac{1}{\sqrt{x^2 + y^2}}$ (inverted cone). Explain how the spacing of level curves differs between these two surfaces.

\section{Cylindrical Surfaces and Cross Sections}

\textbf{Problem 24.} For $f(x, y) = \cos(y)$:
\begin{enumerate}[label=(\alph*)]
    \item Describe the surface in words
    \item Find the cross section in the plane $x = 1$
    \item Find the cross section in the plane $y = \pi/4$
\end{enumerate}

\textbf{Problem 25.} For $g(x, y) = 4 - y^2$:
\begin{enumerate}[label=(\alph*)]
    \item Sketch the surface
    \item Find equations for the cross sections at $x = -2, 0, 2$
    \item Find the cross section in the $xz$-plane (where $y = 0$)
\end{enumerate}

\textbf{Problem 26.} Consider $f(x, y) = e^x$. Find the cross sections:
\begin{enumerate}[label=(\alph*)]
    \item In planes parallel to the $xz$-plane (fix $y = k$)
    \item In planes parallel to the $yz$-plane (fix $x = 0, 1, -1$)
\end{enumerate}

\section{Table Interpretation}

\textbf{Problem 27.} The table below shows the wind chill temperature $W = f(T, v)$ where $T$ is the actual temperature in °F and $v$ is the wind speed in mph.

\begin{center}
\begin{tabular}{|c|c|c|c|c|c|}
\hline
$T \backslash v$ & 5 & 10 & 15 & 20 & 25 \\
\hline
30 & 27 & 21 & 17 & 15 & 13 \\
20 & 16 & 9 & 4 & 2 & 0 \\
10 & 4 & -4 & -9 & -12 & -15 \\
0 & -7 & -16 & -22 & -26 & -29 \\
\hline
\end{tabular}
\end{center}

\begin{enumerate}[label=(\alph*)]
    \item Find and interpret $f(20, 15)$
    \item For what value of $v$ is $f(10, v) = -4$?
    \item For what value of $T$ is $f(T, 20) = 2$?
    \item Compare how $f(30, v)$ and $f(0, v)$ change as $v$ increases
\end{enumerate}

\section{Advanced Domain Problems}

\textbf{Problem 28.} Find the domain of $f(x, y) = \sqrt{\ln(x + y)}$ and explain your reasoning carefully.

\textbf{Problem 29.} Determine the domain of $g(x, y) = \arctan\left(\frac{y}{x}\right) + \sqrt{1 - x^2 - y^2}$.

\textbf{Problem 30.} For the function $h(x, y, z) = \frac{\sqrt{z^2 - x^2 - y^2}}{x^2 + y^2 - 4}$:
\begin{enumerate}[label=(\alph*)]
    \item Find the domain
    \item Describe the domain geometrically as the region between two surfaces
\end{enumerate}

\newpage

\section*{Solutions}

\subsection*{Problem 1}
\begin{enumerate}[label=(\alph*)]
    \item $f(2, 1) = \frac{(2)^2 - 3(1)}{2(2) + (1)^2} = \frac{4 - 3}{4 + 1} = \frac{1}{5}$
    \item $f(-1, 3) = \frac{(-1)^2 - 3(3)}{2(-1) + (3)^2} = \frac{1 - 9}{-2 + 9} = \frac{-8}{7} = -\frac{8}{7}$
    \item $f(0, -2) = \frac{(0)^2 - 3(-2)}{2(0) + (-2)^2} = \frac{0 + 6}{0 + 4} = \frac{6}{4} = \frac{3}{2}$
\end{enumerate}

\subsection*{Problem 2}
\begin{enumerate}[label=(\alph*)]
    \item $g(1, 2) = (1)^3(2) - 2(1)(2)^2 + 5 = 2 - 8 + 5 = -1$
    \item $g(-2, -1) = (-2)^3(-1) - 2(-2)(-1)^2 + 5 = 8 - 4 + 5 = 9$
    \item $g(x+h, y) = (x+h)^3y - 2(x+h)y^2 + 5$
    \item $g(2t, t) = (2t)^3(t) - 2(2t)(t)^2 + 5 = 8t^4 - 4t^3 + 5$
\end{enumerate}

\subsection*{Problem 3}
\begin{enumerate}[label=(\alph*)]
    \item $h(0, 5, 3) = e^{(0)(5)} + \ln(3 - 0) = e^0 + \ln(3) = 1 + \ln(3)$
    \item $h(1, 2, 5) = e^{(1)(2)} + \ln(5 - 1) = e^2 + \ln(4) = e^2 + 2\ln(2)$
    \item $h(-1, 3, 2) = e^{(-1)(3)} + \ln(2 - (-1)) = e^{-3} + \ln(3)$
\end{enumerate}

\subsection*{Problem 4}
\begin{enumerate}[label=(\alph*)]
    \item $f(3, 4) = \sqrt{3^2 + 4^2} \cdot \sin\left(\frac{\pi \cdot 3}{4}\right) = 5 \cdot \sin\left(\frac{3\pi}{4}\right) = 5 \cdot \frac{\sqrt{2}}{2} = \frac{5\sqrt{2}}{2}$
    \item $f(1, 2) = \sqrt{1^2 + 2^2} \cdot \sin\left(\frac{\pi \cdot 1}{2}\right) = \sqrt{5} \cdot \sin\left(\frac{\pi}{2}\right) = \sqrt{5} \cdot 1 = \sqrt{5}$
    \item $f(y, y) = \sqrt{y^2 + y^2} \cdot \sin\left(\frac{\pi y}{y}\right) = \sqrt{2y^2} \cdot \sin(\pi) = |y|\sqrt{2} \cdot 0 = 0$
\end{enumerate}

\subsection*{Problem 5}
For $\ln(x + 2y - 4)$ to be defined: $x + 2y - 4 > 0 \Rightarrow 2y > 4 - x \Rightarrow y > 2 - \frac{x}{2}$

Domain: All points strictly above the line $y = 2 - \frac{x}{2}$. 

Sketch: Line with slope $-\frac{1}{2}$ and $y$-intercept 2, dashed boundary, shade above.

\subsection*{Problem 6}
For $\sqrt{16 - x^2 - y^2}$ to be defined: $16 - x^2 - y^2 \geq 0 \Rightarrow x^2 + y^2 \leq 16$

Domain: All points on and inside the circle of radius 4 centered at origin.

Sketch: Solid circle with radius 4, shaded interior.

\subsection*{Problem 7}
Two conditions:
\begin{itemize}
    \item Denominator $\neq 0$: $y - x^2 \neq 0$
    \item Inside square root $> 0$: $y - x^2 > 0 \Rightarrow y > x^2$
\end{itemize}

Domain: All points strictly above the parabola $y = x^2$.

Sketch: Dashed parabola opening upward, shade region above (inside the cup).

\subsection*{Problem 8}
Denominator cannot be zero: $x^2 + y^2 - 9 \neq 0 \Rightarrow x^2 + y^2 \neq 9$

Domain: All points in the $xy$-plane except those on the circle $x^2 + y^2 = 9$.

In inequality form: $\{(x, y) \in \mathbb{R}^2 : x^2 + y^2 \neq 9\}$

\subsection*{Problem 9}
Cube roots (odd roots) are defined for all real numbers.

Domain: All of $\mathbb{R}^2$ (the entire $xy$-plane).

Sketch: The entire plane with no restrictions.

\subsection*{Problem 10}
Two conditions:
\begin{itemize}
    \item $\ln(x^2 - y)$: Need $x^2 - y > 0 \Rightarrow y < x^2$
    \item $\sqrt{y - 1}$: Need $y - 1 \geq 0 \Rightarrow y \geq 1$
\end{itemize}

Domain: $\{(x, y) : y < x^2 \text{ and } y \geq 1\} = \{(x, y) : 1 \leq y < x^2\}$

This requires $x^2 > 1$, so $|x| > 1$.

\subsection*{Problem 11}
Two conditions:
\begin{itemize}
    \item $\sqrt{x}$: Need $x \geq 0$
    \item Denominator: $y - 3 \neq 0 \Rightarrow y \neq 3$
\end{itemize}

Domain: $\{(x, y) : x \geq 0 \text{ and } y \neq 3\}$

Half-plane $x \geq 0$ with the horizontal line $y = 3$ removed.

\subsection*{Problem 12}
\begin{enumerate}[label=(\alph*)]
    \item Domain: $4 - x^2 - y^2 \geq 0 \Rightarrow x^2 + y^2 \leq 4$. 
    
    Disk of radius 2 centered at origin (solid boundary).
    
    \item Range: Let $u = \sqrt{4 - x^2 - y^2}$. Since $0 \leq x^2 + y^2 \leq 4$, we have $0 \leq u \leq 2$.
    
    Thus $z = e^u$ where $0 \leq u \leq 2$, giving $e^0 \leq z \leq e^2$.
    
    Range: $[1, e^2]$
\end{enumerate}

\subsection*{Problem 13}
\begin{enumerate}[label=(\alph*)]
    \item Domain: Denominator $x^2 + y^2 + 1 > 0$ for all $(x, y)$. 
    
    Domain: $\mathbb{R}^2$ (entire plane)
    
    \item For range, note that by substituting $x = r\cos\theta$, $y = r\sin\theta$:
    
    $g = \frac{r^2\cos\theta\sin\theta}{r^2 + 1} = \frac{r^2\sin(2\theta)/2}{r^2 + 1}$
    
    As $r \to \infty$, $g \to \frac{\sin(2\theta)}{2}$, and as $r \to 0$, $g \to 0$.
    
    Maximum value approaches $\frac{1}{2}$, minimum approaches $-\frac{1}{2}$.
    
    Range: $\left(-\frac{1}{2}, \frac{1}{2}\right)$
\end{enumerate}

\subsection*{Problem 14}
\begin{enumerate}[label=(\alph*)]
    \item Domain: For $\arcsin$, need $-1 \leq x + y \leq 1$.
    
    Domain: $\{(x, y) : -1 \leq x + y \leq 1\}$ (strip between parallel lines)
    
    \item Range: The range of $\arcsin$ is $\left[-\frac{\pi}{2}, \frac{\pi}{2}\right]$
\end{enumerate}

\subsection*{Problem 15}
Need: $16 - x^2 - y^2 - 4z^2 > 0 \Rightarrow x^2 + y^2 + 4z^2 < 16$

Dividing by 16: $\frac{x^2}{16} + \frac{y^2}{16} + \frac{z^2}{4} < 1$

Geometric shape: Interior of an ellipsoid with semi-axes $a = 4$, $b = 4$, $c = 2$.

\subsection*{Problem 16}
Two conditions:
\begin{itemize}
    \item Denominator $\neq 0$: $z - \sqrt{x^2 + y^2} \neq 0$
    \item Inside square root $> 0$: $z - \sqrt{x^2 + y^2} > 0$
\end{itemize}

Domain inequality: $z > \sqrt{x^2 + y^2}$

This is the region strictly above (outside) the cone $z = \sqrt{x^2 + y^2}$.

\subsection*{Problem 17}
\begin{enumerate}[label=(\alph*)]
    \item $h(3, 4, 0) = \sqrt{3^2 + 4^2 + 0^2 - 25} = \sqrt{9 + 16 - 25} = \sqrt{0} = 0$
    
    \item Domain: $x^2 + y^2 + z^2 - 25 \geq 0 \Rightarrow x^2 + y^2 + z^2 \geq 25$
    
    Geometric description: Points on and outside the sphere of radius 5 centered at origin.
\end{enumerate}

\subsection*{Problem 18}
Two conditions:
\begin{itemize}
    \item $\ln(z)$: Need $z > 0$
    \item $\sqrt{9 - x^2 - y^2}$: Need $9 - x^2 - y^2 \geq 0 \Rightarrow x^2 + y^2 \leq 9$
\end{itemize}

Domain: $\{(x, y, z) : x^2 + y^2 \leq 9 \text{ and } z > 0\}$

Description: A solid cylinder of radius 3 along the $z$-axis, for all positive $z$ values.

\subsection*{Problem 19}
Level curves: $4x^2 + 9y^2 = k$

For $k = 0$: $4x^2 + 9y^2 = 0 \Rightarrow x = 0, y = 0$ (single point, the origin)

For $k = 36$: $4x^2 + 9y^2 = 36 \Rightarrow \frac{x^2}{9} + \frac{y^2}{4} = 1$ (ellipse with $a = 3$, $b = 2$)

For $k = 144$: $\frac{x^2}{36} + \frac{y^2}{16} = 1$ (ellipse with $a = 6$, $b = 4$)

Type: Concentric ellipses centered at origin.

\subsection*{Problem 20}
Level curves: $xy = k \Rightarrow y = \frac{k}{x}$

These are rectangular hyperbolas:
\begin{itemize}
    \item $k = -4$: $y = -\frac{4}{x}$ (hyperbola in quadrants II and IV)
    \item $k = -1$: $y = -\frac{1}{x}$
    \item $k = 0$: $xy = 0$ (the coordinate axes)
    \item $k = 1$: $y = \frac{1}{x}$ (hyperbola in quadrants I and III)
    \item $k = 4$: $y = \frac{4}{x}$
\end{itemize}

Type: Rectangular hyperbolas.

\subsection*{Problem 21}
Level curves: $y - x^3 = k \Rightarrow y = x^3 + k$

These are cubic curves (vertical translations of $y = x^3$):
\begin{itemize}
    \item $k = -1$: $y = x^3 - 1$
    \item $k = 0$: $y = x^3$
    \item $k = 1$: $y = x^3 + 1$
    \item $k = 2$: $y = x^3 + 2$
\end{itemize}

Sketch shows family of S-shaped curves.

\subsection*{Problem 22}
Answer: (c) Rectangular grid pattern

The level curves $\cos(x)\sin(y) = k$ form a grid-like pattern because:
\begin{itemize}
    \item When $x = \frac{\pi}{2} + n\pi$, $\cos(x) = 0$, so $f = 0$ (vertical lines)
    \item When $y = n\pi$, $\sin(y) = 0$, so $f = 0$ (horizontal lines)
    \item Other level curves form rounded rectangular shapes
\end{itemize}

\subsection*{Problem 23}
For $z = \sqrt{x^2 + y^2}$ (cone):
\begin{itemize}
    \item Level curves: $\sqrt{x^2 + y^2} = k \Rightarrow x^2 + y^2 = k^2$
    \item Circles with radii $k$
    \item For $k = 1, 2, 3$: radii are 1, 2, 3 (equally spaced)
\end{itemize}

For $z = \frac{1}{\sqrt{x^2 + y^2}}$ (inverted cone):
\begin{itemize}
    \item Level curves: $x^2 + y^2 = \frac{1}{k^2}$
    \item For $k = 1, 2, 3$: radii are 1, $\frac{1}{2}$, $\frac{1}{3}$
    \item Circles get closer together as $k$ increases
\end{itemize}

The cone has evenly-spaced level curves; the inverted cone has level curves that bunch together near the origin.

\subsection*{Problem 24}
\begin{enumerate}[label=(\alph*)]
    \item The surface is a cylindrical wave extending along the $x$-axis, with sinusoidal variation in the $y$-direction.
    
    \item Cross section at $x = 1$: $z = \cos(y)$ (same for any $x$ value)
    
    \item Cross section at $y = \frac{\pi}{4}$: $z = \cos\left(\frac{\pi}{4}\right) = \frac{\sqrt{2}}{2}$ (horizontal line)
\end{enumerate}

\subsection*{Problem 25}
\begin{enumerate}[label=(\alph*)]
    \item Surface is a parabolic cylinder opening downward along the $x$-axis
    
    \item Cross sections at any $x$ value: $z = 4 - y^2$ (same parabola)
    
    At $x = -2, 0, 2$: all give $z = 4 - y^2$
    
    \item Cross section at $y = 0$: $z = 4 - 0^2 = 4$ (horizontal line)
\end{enumerate}

\subsection*{Problem 26}
\begin{enumerate}[label=(\alph*)]
    \item For any fixed $y = k$: $z = e^x$ (same exponential curve)
    
    \item For fixed $x$ values:
    \begin{itemize}
        \item $x = 0$: $z = e^0 = 1$ (horizontal line)
        \item $x = 1$: $z = e^1 = e$ (horizontal line)
        \item $x = -1$: $z = e^{-1} = \frac{1}{e}$ (horizontal line)
    \end{itemize}
\end{enumerate}

\subsection*{Problem 27}
\begin{enumerate}[label=(\alph*)]
    \item $f(20, 15) = 4$. 
    
    Interpretation: When the actual temperature is 20°F and the wind speed is 15 mph, the perceived temperature (wind chill) is 4°F.
    
    \item Look at row $T = 10$, find where value equals $-4$.
    
    This occurs at $v = 10$ mph.
    
    \item Look at column $v = 20$, find where value equals 2.
    
    This occurs at $T = 20$°F.
    
    \item $f(30, v)$: As $v$ increases from 5 to 25, values go from 27 to 13, decreasing by about 3-4° per 5 mph increment (relatively steady decrease).
    
    $f(0, v)$: As $v$ increases from 5 to 25, values go from $-7$ to $-29$, decreasing more rapidly (about 5-6° per 5 mph increment at higher speeds).
    
    The wind chill effect is more pronounced at lower actual temperatures.
\end{enumerate}

\subsection*{Problem 28}
Two conditions:
\begin{itemize}
    \item Inside $\ln$: Need $x + y > 0 \Rightarrow y > -x$
    \item Inside $\sqrt{\phantom{x}}$: Need $\ln(x + y) \geq 0 \Rightarrow x + y \geq 1$
\end{itemize}

Domain: $\{(x, y) : x + y \geq 1\}$

This is the region on and above the line $y = 1 - x$ (solid boundary).

\subsection*{Problem 29}
Two conditions:
\begin{itemize}
    \item $\arctan(y/x)$: Defined for all $(x, y)$ except where $x = 0$
    \item $\sqrt{1 - x^2 - y^2}$: Need $1 - x^2 - y^2 \geq 0 \Rightarrow x^2 + y^2 \leq 1$
\end{itemize}

Domain: $\{(x, y) : x^2 + y^2 \leq 1 \text{ and } x \neq 0\}$

The closed unit disk with the $y$-axis removed.

\subsection*{Problem 30}
\begin{enumerate}[label=(\alph*)]
    \item Three conditions:
    \begin{itemize}
        \item Numerator: $z^2 - x^2 - y^2 \geq 0 \Rightarrow z^2 \geq x^2 + y^2$
        \item Denominator: $x^2 + y^2 - 4 \neq 0 \Rightarrow x^2 + y^2 \neq 4$
        \item Combined with numerator: $z^2 \geq x^2 + y^2$ and $x^2 + y^2 \neq 4$
    \end{itemize}
    
    Domain: $\{(x, y, z) : z^2 \geq x^2 + y^2 \text{ and } x^2 + y^2 \neq 4\}$
    
    \item Geometric description: Points on and outside the double cone $z^2 = x^2 + y^2$, excluding points on the cylinder $x^2 + y^2 = 4$.
\end{enumerate}

\newpage

\section*{Concept Reference Guide}

This reference links each concept/problem type to the relevant question numbers.

\begin{enumerate}
    \item \textbf{Function Evaluation - Direct Substitution}
    \begin{itemize}
        \item Two variables: Problems 1, 2
        \item Three variables: Problem 3
        \item With special functions: Problem 4
    \end{itemize}
    
    \item \textbf{Function Evaluation - Algebraic Expressions}
    \begin{itemize}
        \item $f(x+h, y)$ form: Problem 2(c)
        \item $f(at, bt)$ form: Problem 2(d)
        \item $f(y, y)$ form: Problem 4(c)
    \end{itemize}
    
    \item \textbf{Domain - Logarithmic Functions}
    \begin{itemize}
        \item Basic log domain: Problem 5
        \item Combined with other restrictions: Problems 10, 28
        \item Three variables: Problems 15, 18
    \end{itemize}
    
    \item \textbf{Domain - Even Root Functions}
    \begin{itemize}
        \item Basic square root: Problem 6
        \item Square root in denominator: Problem 7
        \item Combined restrictions: Problems 10, 11, 12, 29
        \item Three variables: Problems 16, 17, 30
    \end{itemize}
    
    \item \textbf{Domain - Odd Root Functions}
    \begin{itemize}
        \item Cube root (no restriction): Problem 9
    \end{itemize}
    
    \item \textbf{Domain - Rational Functions}
    \begin{itemize}
        \item Simple denominator restriction: Problem 8
        \item Combined with other restrictions: Problems 11, 30
    \end{itemize}
    
    \item \textbf{Domain - Inverse Trigonometric Functions}
    \begin{itemize}
        \item Arcsine: Problem 14
        \item Arctangent: Problem 29
    \end{itemize}
    
    \item \textbf{Range Determination}
    \begin{itemize}
        \item With exponential: Problem 12
        \item With rational function: Problem 13
        \item With inverse trig: Problem 14
    \end{itemize}
    
    \item \textbf{Geometric Recognition}
    \begin{itemize}
        \item Circle: Problems 6, 8
        \item Parabola: Problem 7
        \item Sphere: Problem 17
        \item Ellipsoid: Problem 15
        \item Cone: Problems 16, 30
        \item Cylinder: Problem 18
        \item Ellipse: Problem 19
        \item Hyperbola: Problem 20
    \end{itemize}
    
    \item \textbf{Domain Sketching}
    \begin{itemize}
        \item Linear boundaries: Problem 5
        \item Circular regions: Problems 6, 8, 12, 29
        \item Parabolic regions: Problem 7
        \item Combined regions: Problems 10, 11, 14, 28, 29
    \end{itemize}
    
    \item \textbf{Three-Variable Domains}
    \begin{itemize}
        \item Ellipsoid interior: Problem 15
        \item Cone exterior: Problem 16
        \item Sphere exterior: Problem 17
        \item Cylindrical region: Problem 18
        \item Combined surfaces: Problem 30
    \end{itemize}
    
    \item \textbf{Level Curves - Identification}
    \begin{itemize}
        \item Ellipses: Problem 19
        \item Hyperbolas: Problem 20
        \item Cubic curves: Problem 21
        \item Grid patterns: Problem 22
    \end{itemize}
    
    \item \textbf{Level Curves - Spacing Analysis}
    \begin{itemize}
        \item Cone vs. inverted cone: Problem 23
    \end{itemize}
    
    \item \textbf{Cylindrical Surfaces}
    \begin{itemize}
        \item $f(x, y) = g(y)$ only: Problems 24, 25
        \item $f(x, y) = g(x)$ only: Problem 26
    \end{itemize}
    
    \item \textbf{Cross Sections}
    \begin{itemize}
        \item Fixing $x$: Problems 24, 25
        \item Fixing $y$: Problems 24, 25, 26
        \item Different values: Problems 24, 25, 26
    \end{itemize}
    
    \item \textbf{Table/Data Interpretation}
    \begin{itemize}
        \item Reading values: Problem 27(a)
        \item Finding inputs for given outputs: Problems 27(b), 27(c)
        \item Comparing rates of change: Problem 27(d)
    \end{itemize}
    
    \item \textbf{Advanced Domain Analysis}
    \begin{itemize}
        \item Nested functions: Problem 28
        \item Multiple restrictions: Problems 29, 30
        \item Geometric interpretation: Problem 30(b)
    \end{itemize}
    
    \item \textbf{Inequality Manipulation}
    \begin{itemize}
        \item Converting to standard form: Problems 6, 8, 15, 17
        \item Recognizing geometric shapes: Problems 15, 16, 17, 18, 30
    \end{itemize}
\end{enumerate}

\section*{Study Tips}

\begin{itemize}
    \item \textbf{The Big Three Domain Restrictions:}
    \begin{enumerate}
        \item Denominators cannot be zero
        \item Even roots require non-negative arguments
        \item Logarithms require strictly positive arguments
    \end{enumerate}
    
    \item \textbf{Boundary Line Rules:}
    \begin{itemize}
        \item Strict inequality ($>$ or $<$): Dashed line
        \item Non-strict inequality ($\geq$ or $\leq$): Solid line
    \end{itemize}
    
    \item \textbf{Missing Variable Pattern:}
    \begin{itemize}
        \item If $y$ is missing from $f(x, y)$, the surface is a cylinder along the $y$-axis
        \item If $x$ is missing, the cylinder extends along the $x$-axis
    \end{itemize}
    
    \item \textbf{Common Level Curve Patterns:}
    \begin{itemize}
        \item $x^2 + y^2 = k$: Circles (cone/paraboloid)
        \item $xy = k$: Hyperbolas (saddle surface)
        \item $ax + by = k$: Parallel lines (plane)
        \item $y = x^n + k$: Power curves (generalized cylinders)
    \end{itemize}
    
    \item \textbf{Three-Variable Shapes:}
    \begin{itemize}
        \item $x^2 + y^2 + z^2 = r^2$: Sphere
        \item $\frac{x^2}{a^2} + \frac{y^2}{b^2} + \frac{z^2}{c^2} = 1$: Ellipsoid
        \item $z = \sqrt{x^2 + y^2}$: Cone
        \item $x^2 + y^2 = r^2$: Cylinder (no $z$ restriction)
    \end{itemize}
\end{itemize}

\end{document}