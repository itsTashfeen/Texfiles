\documentclass[11pt]{article}
\usepackage[utf8]{inputenc}
\usepackage{amsmath, amssymb, amsthm, graphicx, enumitem, geometry, fancyhdr, hyperref}

% Geometry settings
\geometry{
    a4paper,
    total={170mm,257mm},
    left=20mm,
    top=20mm,
}

% Header settings
\pagestyle{fancy}
\fancyhead[L]{Homework 14.2 Limits and Continuity}
\fancyhead[R]{Tashfeen Omran}

% Title information
\title{\textbf{Homework 14.2: Multivariable Limits and Continuity}\\ \large A Comprehensive Study Guide and Solution Set}
\author{Tashfeen Omran}
\date{January 2026}

\begin{document}

\maketitle
\tableofcontents
\newpage

\section{Part 1: Introduction, Context, and Prerequisites}

\subsection{Core Concepts}
In single-variable calculus, a limit $\lim_{x \to a} f(x)$ exists if approaching $a$ from the left yields the same value as approaching from the right. In multivariable calculus, specifically for functions of two variables $z = f(x,y)$, the domain is a plane, not a line.

To find $\lim_{(x,y) \to (a,b)} f(x,y)$, we must approach the point $(a,b)$ from \textbf{infinite directions}. For the limit to exist, the function must approach the same value $L$ along \textit{every possible path} (lines, parabolas, spirals, etc.) passing through $(a,b)$.

\subsection{Intuition and Motivation}
Imagine a hiking terrain described by a surface $z = f(x,y)$.
\begin{itemize}
    \item \textbf{Limit Existence:} If you and your friends walk toward a specific map coordinate $(a,b)$ from different directions (North, East, South-West, spiraling in), you should all meet at the same altitude $L$.
    \item \textbf{Continuity:} Not only must you meet at the same altitude, but there must also be actual ground beneath your feet at that point. If there is a hole in the terrain (a removable discontinuity) or a vertical cliff (an infinite discontinuity), the function is not continuous there.
\end{itemize}

\subsection{Historical Context}
The rigorous definition of limits (using $\epsilon-\delta$) was formalized by Augustin-Louis Cauchy and Karl Weierstrass in the 19th century. This development was motivated by the need to put calculus on a solid logical foundation. Early mathematicians struggled with "pathological" functions—surfaces that looked smooth but behaved chaotically at specific points. The rigorous theory of limits was necessary to distinguish between "well-behaved" smooth surfaces and those with subtle tears or singularities.

\subsection{Key Formulas and Definitions}

\textbf{1. Definition of the Limit:}
\[ \lim_{(x,y) \to (a,b)} f(x,y) = L \]
means that for every $\epsilon > 0$, there exists a $\delta > 0$ such that if $0 < \sqrt{(x-a)^2 + (y-b)^2} < \delta$, then $|f(x,y) - L| < \epsilon$.

\textbf{2. Two-Path Test (To prove non-existence):}
If $f(x,y) \to L_1$ along path $C_1$ and $f(x,y) \to L_2$ along path $C_2$, and $L_1 \neq L_2$, then the limit \textbf{does not exist (DNE)}.

\textbf{3. Continuity:}
A function $f$ is continuous at $(a,b)$ if:
\[ \lim_{(x,y) \to (a,b)} f(x,y) = f(a,b) \]

\subsection{Prerequisites}
To master this topic, you must be proficient in:
\begin{itemize}
    \item \textbf{Factoring Polynomials:} Difference of squares ($a^2 - b^2$) and sum/difference of cubes.
    \item \textbf{L'Hôpital's Rule:} (Only valid after reducing a multivariable limit to a single variable limit).
    \item \textbf{Polar Coordinates:} Converting $x = r\cos\theta$, $y = r\sin\theta$, with $x^2+y^2 = r^2$.
    \item \textbf{Domain Restrictions:} $\sqrt{A} \implies A \ge 0$, $\ln(A) \implies A > 0$, $\frac{1}{A} \implies A \neq 0$.
\end{itemize}

\newpage

\section{Part 2: Detailed Homework Solutions}

\subsection*{Problem 1 (SCalcET9 14.2.005)}
\textbf{Find the limit:}
\[ \lim_{(x,y) \to (2,5)} (x^2y^3 - 7y^2) \]

\textbf{Solution:}
This is a polynomial function. Polynomials are continuous everywhere. We can use \textbf{Direct Substitution}.
\begin{align*}
    L &= (2)^2(5)^3 - 7(5)^2 \\
    &= (4)(125) - 7(25) \\
    &= 500 - 175 \\
    &= 325
\end{align*}

\textbf{Final Answer:} $325$

\hrulefill

\subsection*{Problem 2 (SCalcET9 14.2.008)}
\textbf{Find the limit:}
\[ \lim_{(x,y) \to (9,-3)} \frac{x^2y + xy^2}{x^2 - y^2} \]

\textbf{Solution:}
First, attempt direct substitution to check for indeterminacy.
\[ \text{Denominator: } x^2 - y^2 = (9)^2 - (-3)^2 = 81 - 9 = 72 \]
Since the denominator is not zero ($72 \neq 0$), the function is continuous at this point. We can substitute directly.
\begin{align*}
    \text{Numerator: } x^2y + xy^2 &= (9)^2(-3) + (9)(-3)^2 \\
    &= 81(-3) + 9(9) \\
    &= -243 + 81 \\
    &= -162
\end{align*}
\[ \text{Limit} = \frac{-162}{72} \]
Simplify by dividing numerator and denominator by 18:
\[ \frac{-162 \div 18}{72 \div 18} = \frac{-9}{4} \]
\textit{Alternative Algebraic Approach (Factoring):}
\[ \frac{xy(x+y)}{(x-y)(x+y)} = \frac{xy}{x-y} \]
Substitute $(9, -3)$:
\[ \frac{9(-3)}{9 - (-3)} = \frac{-27}{12} = -\frac{9}{4} \]

\textbf{Final Answer:} $-\frac{9}{4}$

\hrulefill

\subsection*{Problem 3 (SCalcET9 14.2.009)}
\textbf{Find the limit:}
\[ \lim_{(x,y) \to (3\pi/2, \pi)} y \sin(x-y) \]

\textbf{Solution:}
The function is a product of polynomial and trigonometric functions, which are continuous everywhere. Use Direct Substitution.
\begin{align*}
    L &= \pi \sin\left(\frac{3\pi}{2} - \pi\right) \\
    &= \pi \sin\left(\frac{\pi}{2}\right) \\
    &= \pi (1) \\
    &= \pi
\end{align*}

\textbf{Final Answer:} $\pi$

\hrulefill

\subsection*{Problem 4 (SCalcET9 14.2.020)}
\textbf{Find the limit, if it exists:}
\[ \lim_{(x,y) \to (\pi, 1/2)} 5e^{xy} \sin(xy) \]

\textbf{Solution:}
Exponentials and sine functions are continuous on their domains. Use Direct Substitution.
\begin{align*}
    L &= 5e^{(\pi)(1/2)} \sin\left((\pi)(1/2)\right) \\
    &= 5e^{\pi/2} \sin\left(\frac{\pi}{2}\right) \\
    &= 5e^{\pi/2} (1) \\
    &= 5e^{\pi/2}
\end{align*}

\textbf{Final Answer:} $5e^{\pi/2}$

\hrulefill

\subsection*{Problem 5 (SCalcET9 14.2.023)}
\textbf{Find the limit, if it exists:}
\[ \lim_{(x,y) \to (0,0)} \frac{xy^2 \cos(y)}{5x^2 + y^4} \]

\textbf{Solution:}
Direct substitution yields $0/0$. We must test different paths.

\textbf{Path 1: Approach along the x-axis ($y=0$)}
Let $y=0$. As $x \to 0$:
\[ \lim_{x \to 0} \frac{x(0)^2 \cos(0)}{5x^2 + 0} = \lim_{x \to 0} \frac{0}{5x^2} = 0 \]
Limit along this path is $0$.

\textbf{Path 2: Approach along the parabola $x = y^2$}
Substitute $x = y^2$ into the function:
\begin{align*}
    f(y^2, y) &= \frac{(y^2)(y^2) \cos(y)}{5(y^2)^2 + y^4} \\
    &= \frac{y^4 \cos(y)}{5y^4 + y^4} \\
    &= \frac{y^4 \cos(y)}{6y^4}
\end{align*}
Cancel $y^4$ (assuming $y \neq 0$):
\[ \lim_{y \to 0} \frac{\cos(y)}{6} = \frac{1}{6} \]
Limit along this path is $1/6$.

\textbf{Conclusion:}
Since $0 \neq 1/6$, the limits along two different paths are different.

\textbf{Final Answer:} DNE

\hrulefill

\subsection*{Problem 6 (SCalcET9 14.2.032.MI.SA)}
\textbf{Find the limit, if it exists:}
\[ \lim_{(x,y) \to (0,0)} \frac{xy}{\sqrt{x^2+y^2}} \]

\textbf{Solution:}
Direct substitution gives $0/0$. The term $\sqrt{x^2+y^2}$ suggests \textbf{Polar Coordinates}.
Let $x = r\cos\theta$ and $y = r\sin\theta$. As $(x,y) \to (0,0)$, $r \to 0^+$.
\begin{align*}
    \lim_{r \to 0^+} \frac{(r\cos\theta)(r\sin\theta)}{\sqrt{r^2}} &= \lim_{r \to 0^+} \frac{r^2 \cos\theta \sin\theta}{r} \\
    &= \lim_{r \to 0^+} r \cos\theta \sin\theta
\end{align*}
Since $-1 \le \cos\theta \sin\theta \le 1$, the term $\cos\theta \sin\theta$ is bounded.
As $r \to 0$, $r \times (\text{bounded value}) \to 0$.
The limit is $0$ regardless of $\theta$.

\textbf{Final Answer:} 0

\hrulefill

\subsection*{Problem 7 (SCalcET9 14.2.039)}
\textbf{Analyze continuity:}
\[ f(x,y) = e^{1/(x-y)} \]
\textbf{Graph the function and observe where it is discontinuous.}
\textbf{Explanation using composition of functions.}

\textbf{Solution:}
The exponential function $e^u$ is continuous everywhere. The discontinuity arises solely from the exponent $u = \frac{1}{x-y}$.
A rational function is discontinuous where the denominator is zero.
\[ x - y = 0 \implies y = x \]
Therefore, $f$ is discontinuous along the line $y=x$.

\textbf{Composition Explanation:}
Looking at $f(x,y) = e^{1/(x-y)}$ as a composition $(h \circ g)(x,y)$:
Let $g(x,y) = \frac{1}{x-y}$. This is a rational function and is continuous everywhere except where $x-y=0$.
Let $h(t) = e^t$. Since $h(t)$ is continuous \textbf{everywhere} (on $\mathbb{R}$), the composition is continuous everywhere except at the discontinuity of the inner function $g$.

\textbf{Fill-in-the-blanks:}
1. $f$ is discontinuous at: \textbf{The set of points $(x,y)$ such that $y=x$}.
2. Since $h(t) = e^t$ is continuous \textbf{everywhere}, the composition is continuous everywhere except at the discontinuity above.

\textbf{Final Answer:} Discontinuous at $y=x$.

\hrulefill

\subsection*{Problem 8 (SCalcET9 14.2.042)}
\textbf{Determine the set of points at which the function is continuous:}
\[ F(x,y) = \cos\left(\sqrt{1+x-y}\right) \]

\textbf{Solution:}
The cosine function is continuous everywhere. The square root function $\sqrt{u}$ is continuous for $u \ge 0$.
Therefore, we require the inside of the square root to be non-negative:
\[ 1 + x - y \ge 0 \]
Rearranging for $y$:
\[ 1 + x \ge y \quad \text{or} \quad y \le x + 1 \]
This describes the region on and below the line $y = x+1$.

\textbf{Final Answer:} $\{(x,y) \mid y \le x + 1\}$ (Select the last option in the screenshot).

\hrulefill

\subsection*{Problem 9 (SCalcET9 14.2.046)}
\textbf{Determine the set of points at which the function is continuous:}
\[ G(x,y) = \ln(4+x-y) \]

\textbf{Solution:}
The natural logarithm $\ln(u)$ is continuous only when $u > 0$.
\[ 4 + x - y > 0 \]
Rearranging for $y$:
\[ 4 + x > y \quad \text{or} \quad y < x + 4 \]
This describes the region strictly below the line $y = x+4$.

\textbf{Final Answer:} $\{(x,y) \mid y < x + 4\}$ (Select the first option in the screenshot).

\hrulefill

\subsection*{Problem 10 (SCalcET9 14.2.049)}
\textbf{Determine the set of points at which the function is continuous:}
\[ f(x,y) = \begin{cases} \dfrac{x^2y^3}{8x^2+y^2} & \text{if } (x,y) \neq (0,0) \\ 1 & \text{if } (x,y) = (0,0) \end{cases} \]

\textbf{Solution:}
The function is rational away from $(0,0)$ where the denominator is non-zero, so it is continuous for all $(x,y) \neq (0,0)$.
We must check continuity at $(0,0)$.
1. Find the limit as $(x,y) \to (0,0)$. Use Polar Coordinates.
\[ x = r\cos\theta, \quad y = r\sin\theta \]
\[ \lim_{r \to 0} \frac{(r^2\cos^2\theta)(r^3\sin^3\theta)}{8r^2\cos^2\theta + r^2\sin^2\theta} = \lim_{r \to 0} \frac{r^5 \cos^2\theta \sin^3\theta}{r^2(8\cos^2\theta + \sin^2\theta)} \]
\[ = \lim_{r \to 0} r^3 \left( \frac{\cos^2\theta \sin^3\theta}{8\cos^2\theta + \sin^2\theta} \right) \]
The denominator $8\cos^2\theta + \sin^2\theta \ge 1$ (it is never zero). The fraction is bounded.
As $r \to 0$, $r^3 \to 0$.
So, $\lim_{(x,y)\to(0,0)} f(x,y) = 0$.

2. Compare Limit to Defined Value.
$f(0,0)$ is defined as $1$.
Since Limit ($0$) $\neq$ Function Value ($1$), $f$ is \textbf{not} continuous at $(0,0)$.

\textbf{Conclusion:}
The function is continuous everywhere except $(0,0)$.
This corresponds to the set where $(x,y) \neq (0,0)$.

\textbf{Final Answer:} $\{(x,y) \mid (x,y) \neq (0,0)\}$ (Select the third option).

\newpage

\section{Part 3: In-Depth Analysis of Problems and Techniques}

\subsection{A) Problem Types and General Approach}
\begin{enumerate}
    \item \textbf{Direct Substitution (Problems 1, 2, 3, 4):}
    \begin{itemize}
        \item \textbf{Strategy:} Plug the numbers in immediately.
        \item \textbf{Check:} Ensure the denominator is not zero and you don't take the square root/log of invalid numbers. If you get a number, you are done.
    \end{itemize}

    \item \textbf{The Two-Path Test (Problem 5):}
    \begin{itemize}
        \item \textbf{Strategy:} Use this when substitution yields $0/0$ and the degrees of numerator and denominator look somewhat "balanced" or specific paths simplify the expression.
        \item \textbf{Common Paths:} Axis ($y=0$ or $x=0$), Linear ($y=mx$), Parabolic ($x=y^2$ or $y=x^2$).
    \end{itemize}

    \item \textbf{Polar Coordinates / Squeeze Theorem (Problem 6, 10):}
    \begin{itemize}
        \item \textbf{Strategy:} Use this when you have terms like $x^2+y^2$ in the denominator and direct sub yields $0/0$.
        \item \textbf{Method:} Swap $x \to r\cos\theta, y \to r\sin\theta$. If the limit depends only on $r$ and vanishes as $r \to 0$ (independent of $\theta$), the limit exists.
    \end{itemize}

    \item \textbf{Domain Continuity (Problems 7, 8, 9):}
    \begin{itemize}
        \item \textbf{Strategy:} Identify the restriction function (denominator, sqrt, log). Set up the inequality (e.g., inside of log $> 0$) and solve for $y$.
    \end{itemize}
\end{enumerate}

\subsection{B) Key Algebraic and Calculus Manipulations}
\begin{itemize}
    \item \textbf{The Parabolic Path Trick (Problem 5):}
    For $\frac{xy^2}{5x^2+y^4}$, notice that $x^2$ is degree 2 and $y^4$ is degree 4. To make them "compatible" (addable), we chose $x = y^2$, turning $x^2$ into $(y^2)^2 = y^4$. This allowed terms to combine in the denominator.

    \item \textbf{Polar Bounding (Problem 10):}
    When simplifying polar limits, we often isolate the "angular part" (functions of $\theta$). If the denominator is strictly non-zero (e.g., $8\cos^2\theta + \sin^2\theta \ge 1$), the whole angular fraction is bounded. An $r^n$ term multiplying a bounded term goes to 0.
\end{itemize}

\section{Part 4: "Cheatsheet" and Tips for Success}

\subsection*{Summary of Formulas}
\begin{itemize}
    \item \textbf{Polar Conversion:} $x = r\cos\theta, \quad y = r\sin\theta, \quad x^2+y^2 = r^2$.
    \item \textbf{Continuous Functions:} Polynomials, $\sin$, $\cos$, $e^x$ are continuous everywhere. Rationals, $\ln$, $\sqrt{x}$ are continuous on their domains.
\end{itemize}

\subsection*{Tricks and Shortcuts}
\begin{itemize}
    \item \textbf{Degree Analysis Shortcut:} Look at $f(x,y) = \frac{P(x,y)}{Q(x,y)}$. Sum the exponents of terms.
    \begin{itemize}
        \item If total degree of Numerator $>$ total degree of Denominator, Limit is likely 0 (Use Polar).
        \item If Num Degree $=$ Denom Degree, Limit likely DNE (Use Two-Path: $y=mx$).
        \item If you see $x^2+y^2$, immediately think \textbf{Polar}.
    \end{itemize}
\end{itemize}

\subsection*{Common Pitfalls}
\begin{itemize}
    \item \textbf{The "Two Paths Match" Trap:} Finding that $\lim_{x\to0} = 0$ and $\lim_{y\to0} = 0$ \textbf{does NOT} prove the limit is 0. You must use Polar or Squeeze theorem to prove existence. You can only use paths to prove \textit{non-existence}.
    \item \textbf{Algebraic Errors:} $(x+y)^2 \neq x^2 + y^2$.
\end{itemize}

\newpage

\section{Part 5: Conceptual Synthesis and The "Big Picture"}

\subsection{Thematic Connections}
The core theme of Multivariable Limits is \textbf{Locality vs. Directionality}. In 1D, you only have "left" and "right". In 2D, a point $(0,0)$ is surrounded by a 360-degree neighborhood. The behavior of the function must be consistent regardless of the angle of approach. This concept prepares you for \textbf{Directional Derivatives} (how slope changes based on direction) and \textbf{Gradients}.

\subsection{Forward and Backward Links}
\begin{itemize}
    \item \textbf{Backward:} Relies heavily on Limits (Calc I) and Parametric/Polar equations (Calc II).
    \item \textbf{Forward:} Continuity is required for \textbf{Partial Differentiation}. If a function is not continuous, it cannot be differentiable. Furthermore, defining Multiple Integrals (Volume under a surface) requires the surface to be largely continuous.
\end{itemize}

\section{Part 6: Real-World Application and Modeling}

\subsection{A) Concrete Scenarios (Finance \& Economics)}
\begin{enumerate}
    \item \textbf{Options Pricing (Black-Scholes Model):}
    The Black-Scholes equation for pricing European options relies on variables like Stock Price ($S$), Time ($t$), and Volatility ($\sigma$). As time approaches expiration ($t \to T$), the value of the option must converge continuously to the Payoff Function $\max(S-K, 0)$. Analyzing this limit ensures the model is consistent with contract rules at expiration.

    \item \textbf{Portfolio Correlation Singularities:}
    In Modern Portfolio Theory, risk is calculated using a correlation matrix of assets. If the correlation $\rho$ between two assets approaches 1 (perfect correlation), the determinant of the covariance matrix approaches 0. This creates a singularity when trying to invert the matrix for optimization. Understanding the limit behavior as $\rho \to 1$ is crucial for building stable algorithmic trading models.
\end{enumerate}

\subsection{B) Model Problem Setup: High-Frequency Trading Continuity}
\textbf{Scenario:} An algorithmic trading bot executes trades based on a liquidity function $L(v, t)$, where $v$ is trade volume and $t$ is time gap between trades.
\[ L(v, t) = \frac{v \cdot t}{\sqrt{v^2 + t^2}} \]
\textbf{Problem:} The algorithm crashes if $L(v,t)$ is undefined or jumps discontinuously.
\textbf{Task:} Determine if the liquidity score is stable as volume and time gap both vanish $(v, t) \to (0,0)$. This is exactly Problem 6 from your homework. We proved the limit is 0, so the function can be defined as $L(0,0)=0$ to ensure stability.

\newpage

\section{Part 7: Common Variations and Untested Concepts}

Your homework covered the basics, but omitted a few advanced standard curriculum concepts:

\subsection{1. The Epsilon-Delta Proof}
The homework relied on techniques to find answers, but not the rigorous proof.
\textbf{Concept:} Proving $\lim_{(x,y)\to(0,0)} \frac{3x^2y}{\sqrt{x^2+y^2}} = 0$ using definition.
\textbf{Example:}
We need to show $|\frac{3x^2y}{\sqrt{x^2+y^2}} - 0| < \epsilon$.
Using $|y| \le \sqrt{x^2+y^2}$ and $x^2 \le x^2+y^2$:
\[ \left| \frac{3x^2y}{\sqrt{x^2+y^2}} \right| \le \frac{3(x^2+y^2)\sqrt{x^2+y^2}}{\sqrt{x^2+y^2}} = 3(x^2+y^2) \]
We want $3(x^2+y^2) < \epsilon$, or $\sqrt{x^2+y^2} < \sqrt{\epsilon/3}$. Thus, choose $\delta = \sqrt{\epsilon/3}$.

\subsection{2. Limits at Infinity}
\textbf{Concept:} Limits where $x \to \infty$ or $y \to \infty$.
\textbf{Example:} $\lim_{(x,y) \to (\infty, \infty)} \frac{xy}{x^2+y^2}$.
Convert to polar: $r \to \infty$.
\[ \frac{r^2 \cos\theta \sin\theta}{r^2} = \cos\theta \sin\theta \]
As $r \to \infty$, the value oscillates between -0.5 and 0.5 depending on $\theta$. The limit DNE.

\section{Part 8: Advanced Diagnostic Testing: "Find the Flaw"}

\textbf{Problem 1: Path Independence?}
\textit{Student Work:} Evaluate $\lim_{(x,y)\to(0,0)} \frac{x^2 y}{x^4 + y^2}$.
Path 1 ($x=0$): Limit is 0. Path 2 ($y=x$): $\frac{x^3}{x^4+x^2} \approx \frac{x^3}{x^2} = x \to 0$.
\textit{Conclusion:} Since both paths are 0, the limit is 0.
\textbf{Flaw:} Checking linear paths is insufficient.
\textbf{Correction:} Use path $y=x^2$. Limit becomes $\frac{x^2(x^2)}{x^4+(x^2)^2} = \frac{x^4}{2x^4} = 1/2$. Limit DNE.

\textbf{Problem 2: L'Hôpital's Rule Misuse}
\textit{Student Work:} Evaluate $\lim_{(x,y)\to(0,0)} \frac{\sin(xy)}{xy}$.
Apply L'Hôpital's Rule to top and bottom: $\lim \frac{\cos(xy)(y)}{y} = \cos(0) = 1$.
\textbf{Flaw:} You cannot apply L'Hôpital's Rule directly to multivariable functions.
\textbf{Correction:} Let $u = xy$. As $(x,y)\to(0,0)$, $u\to 0$. $\lim_{u\to 0} \frac{\sin u}{u} = 1$ via single-variable Calc I definition.

\textbf{Problem 3: Direct Sub Algebra Error}
\textit{Student Work:} $\lim_{(x,y)\to(1,1)} \frac{x-y}{x^2-y^2} = \frac{0}{0}$.
Factor: $\frac{x-y}{(x-y)^2} = \frac{1}{x-y}$. Sub in (1,1) $\to 1/0 \to \infty$.
\textbf{Flaw:} Factoring $x^2-y^2$ gives $(x-y)(x+y)$, not $(x-y)^2$.
\textbf{Correction:} $\frac{x-y}{(x-y)(x+y)} = \frac{1}{x+y}$. Limit is $1/2$.

\textbf{Problem 4: Polar Bound Error}
\textit{Student Work:} $\lim_{(x,y)\to(0,0)} \frac{2x}{x^2+y+1}$. Use Polar.
$\frac{2r\cos\theta}{r^2+r\sin\theta+1}$. As $r\to 0$, num $\to 0$, den $\to 1$. Limit 0.
\textbf{Flaw:} This solution is actually correct, but the \textit{method} is overkill.
\textbf{Correction:} This is a continuous function. Just use direct substitution. $\frac{0}{0+0+1} = 0$. Using polar unnecessarily complicates simple problems.

\textbf{Problem 5: Continuity vs Limit}
\textit{Student Work:} Function $f(x,y) = \frac{xy}{x^2+y^2}$ if $\neq (0,0)$, else $0$.
Limit DNE (proved by paths).
\textit{Conclusion:} The function is continuous because $f(0,0)$ is defined.
\textbf{Flaw:} Existence of value $f(0,0)$ does not imply continuity. The limit must exist AND equal the value.
\textbf{Correction:} Since Limit DNE, function is discontinuous at $(0,0)$.

\end{document}