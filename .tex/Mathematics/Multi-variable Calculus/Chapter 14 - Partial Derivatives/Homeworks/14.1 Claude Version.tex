\documentclass[11pt]{article}
\usepackage{amsmath, amssymb, geometry}
\usepackage{graphicx}
\geometry{margin=1in}

\title{Homework 14.1: Functions of Several Variables}
\author{Tashfeen Omran}
\date{January 2026}

\begin{document}

\maketitle
\tableofcontents
\newpage

\section{Comprehensive Introduction, Context, and Prerequisites}

\subsection{Core Concepts}

A \textbf{function of several variables} is a rule that assigns to each ordered pair (or triple, quadruple, etc.) of real numbers in its domain a unique real number. Where single-variable calculus deals with functions of the form $y = f(x)$, multivariable calculus begins with functions like:

\begin{itemize}
    \item $z = f(x, y)$ --- a function of two variables
    \item $w = f(x, y, z)$ --- a function of three variables
    \item $u = f(x_1, x_2, \ldots, x_n)$ --- a function of $n$ variables
\end{itemize}

The \textbf{domain} of a function of several variables is the set of all input tuples for which the function is defined. For example, for $f(x, y) = \sqrt{x - y}$, the domain is $\{(x, y) : x \geq y\}$, which is the region on or above the line $y = x$ in the $xy$-plane.

The \textbf{range} is the set of all possible output values. For $f(x, y) = \sqrt{x - y}$, since the square root function outputs only non-negative values, the range is $[0, \infty)$.

\textbf{Visualization:} While the graph of $y = f(x)$ is a curve in 2D space, the graph of $z = f(x, y)$ is a \emph{surface} in 3D space. Each point on this surface has coordinates $(x, y, z)$ where $z = f(x, y)$.

\textbf{Level Curves and Contour Maps:} Since visualizing 3D surfaces on paper is challenging, we use \textbf{contour maps}. A level curve (or contour line) of $f(x, y)$ is a curve in the $xy$-plane along which $f$ has a constant value $k$. The equation of a level curve is $f(x, y) = k$. A collection of several level curves for different values of $k$ forms a contour map, similar to topographic maps showing elevation.

\subsection{Intuition and Derivation}

The transition from single-variable to multivariable functions is natural when modeling real-world phenomena. Temperature in a room depends on position: $T = f(x, y, z)$ where $(x, y, z)$ is a point in space. The price of a stock option depends on multiple factors: $V = f(S, t, \sigma, r)$ where $S$ is stock price, $t$ is time, $\sigma$ is volatility, and $r$ is the interest rate.

The fundamental idea is that output can depend on multiple independent inputs simultaneously. The domain restrictions arise from the mathematical operations involved:
\begin{itemize}
    \item Square roots require non-negative arguments: $\sqrt{g(x,y)} \Rightarrow g(x,y) \geq 0$
    \item Logarithms require positive arguments: $\ln(g(x,y)) \Rightarrow g(x,y) > 0$
    \item Rational functions require non-zero denominators: $\frac{1}{g(x,y)} \Rightarrow g(x,y) \neq 0$
\end{itemize}

Level curves emerge from the need to represent 3D information in 2D. Just as contour lines on a hiking map show points of equal elevation, level curves show points where a function takes the same value.

\subsection{Historical Context and Motivation}

The development of multivariable calculus emerged in the 18th century from the need to model physical phenomena that depend on multiple variables simultaneously. While Newton and Leibniz developed single-variable calculus to study motion and rates of change, scientists soon realized that most natural phenomena---gravitational fields, fluid flow, heat distribution, electromagnetic forces---cannot be adequately described by functions of a single variable.

Mathematicians like Leonhard Euler, Joseph-Louis Lagrange, and Carl Friedrich Gauss extended calculus to multiple dimensions to solve problems in physics, astronomy, and engineering. The study of planetary motion required understanding how gravitational force depends on position in three-dimensional space. The analysis of heat flow in metal plates led to partial differential equations involving temperature as a function of spatial coordinates and time. These practical necessities drove the mathematical framework we study today, laying the groundwork for modern physics, engineering optimization, economic modeling, and financial mathematics.

\subsection{Key Formulas}

\textbf{Function Evaluation:}
\[f(a, b) = \text{value obtained by substituting } x = a \text{ and } y = b \text{ into } f(x, y)\]

\textbf{Domain Determination:} The domain of $f(x, y)$ consists of all points $(x, y)$ for which the function expression is defined. Common restrictions:
\begin{align*}
\sqrt{g(x,y)} &\Rightarrow g(x,y) \geq 0 \\
\ln(g(x,y)) &\Rightarrow g(x,y) > 0 \\
\frac{1}{g(x,y)} &\Rightarrow g(x,y) \neq 0
\end{align*}

\textbf{Level Curves:} For $z = f(x, y)$, the level curve at height $k$ is:
\[f(x, y) = k\]

\textbf{Range:} The range is determined by analyzing all possible output values of the function over its entire domain.

\subsection{Prerequisites}

To master functions of several variables, you must be proficient in:

\begin{enumerate}
    \item \textbf{Algebra:}
    \begin{itemize}
        \item Manipulating expressions with multiple variables
        \item Solving systems of equations
        \item Working with inequalities (for domain determination)
    \end{itemize}
    
    \item \textbf{Single-Variable Calculus:}
    \begin{itemize}
        \item Understanding functions, domains, and ranges
        \item Exponential and logarithmic functions
        \item Trigonometric functions
    \end{itemize}
    
    \item \textbf{Analytic Geometry:}
    \begin{itemize}
        \item Graphing curves and regions in the $xy$-plane
        \item Understanding circles, parabolas, and other conic sections
        \item Three-dimensional coordinate systems
    \end{itemize}
    
    \item \textbf{Inequalities:}
    \begin{itemize}
        \item Graphing inequalities in 2D (shading regions)
        \item Understanding compound inequalities
    \end{itemize}
\end{enumerate}

\newpage
\section{Detailed Homework Solutions}

\subsection{Problem 1}

Given $f(x, y) = 3x - y^2$, find:

\subsubsection{Part (a): $f(1, 5)$}

Substitute $x = 1$ and $y = 5$:
\[f(1, 5) = 3(1) - (5)^2 = 3 - 25 = -22\]

\textbf{Answer:} $-22$

\subsubsection{Part (b): $f(-3, -1)$}

Substitute $x = -3$ and $y = -1$:
\[f(-3, -1) = 3(-3) - (-1)^2 = -9 - 1 = -10\]

\textbf{Answer:} $-10$

\subsubsection{Part (c): $f(x + h, y)$}

Substitute $x + h$ for $x$:
\[f(x + h, y) = 3(x + h) - y^2 = 3x + 3h - y^2\]

\textbf{Answer:} $3x + 3h - y^2$

\subsubsection{Part (d): $f(x, x)$}

Substitute $x$ for both variables:
\[f(x, x) = 3x - x^2\]

\textbf{Answer:} $3x - x^2$

\subsection{Problem 2}

Let $g(x, y) = x\ln(x^2 + y)$.

\subsubsection{Part (a): Evaluate $g(7, 1)$}

Substitute $x = 7$ and $y = 1$:
\[g(7, 1) = 7\ln(7^2 + 1) = 7\ln(49 + 1) = 7\ln(50)\]

\textbf{Answer:} $7\ln(50)$

\subsubsection{Part (b): Find and sketch the domain of $g$}

For $g(x, y) = x\ln(x^2 + y)$ to be defined, the argument of the logarithm must be positive:
\[x^2 + y > 0 \Rightarrow y > -x^2\]

The domain is the region \emph{above} the parabola $y = -x^2$ in the $xy$-plane (not including the parabola itself).

\textbf{Domain:} $\{(x, y) : y > -x^2\}$

\textbf{Sketch:} The region is all points above the downward-opening parabola $y = -x^2$. Shade the region above this parabola, using a dashed curve for the parabola to indicate it's not included.

\subsubsection{Part (c): Find the range of $g$}

Since $x$ can be any real number and $\ln(x^2 + y)$ can take any real value (as $x^2 + y$ ranges over all positive reals), the product $x\ln(x^2 + y)$ can produce any real number.

\textbf{Range:} $(-\infty, \infty)$ or $\mathbb{R}$

\subsection{Problem 3}

Let $h(x, y) = e^{y - x^2}$.

\subsubsection{Part (a): Evaluate $h(-1, 2)$}

Substitute $x = -1$ and $y = 2$:
\[h(-1, 2) = e^{2 - (-1)^2} = e^{2 - 1} = e^1 = e\]

\textbf{Answer:} $e$

\subsubsection{Part (b): Find and sketch the domain of $h$}

The exponential function $e^{y - x^2}$ is defined for all real values of $y - x^2$. Therefore, there are no restrictions.

\textbf{Domain:} $\{(x, y) : x, y \in \mathbb{R}\} = \mathbb{R}^2$

\textbf{Sketch:} The entire $xy$-plane.

\subsubsection{Part (c): Find the range of $h$}

Since $e^u > 0$ for all real $u$, and $y - x^2$ can be any real number (as $x$ and $y$ vary), we have:
\[h(x, y) = e^{y - x^2} \in (0, \infty)\]

\textbf{Range:} $(0, \infty)$

\subsection{Problem 4}

Let $f(x, y, z) = \ln(z - x^2 - y^2)$.

\subsubsection{Part (a): Evaluate $f(3, -4, 9)$}

Substitute $x = 3$, $y = -4$, and $z = 9$:
\[f(3, -4, 9) = \ln(9 - 3^2 - (-4)^2) = \ln(9 - 9 - 16) = \ln(-16)\]

Since $\ln(-16)$ is undefined (logarithm of a negative number), this point is not in the domain of $f$.

\textbf{Answer:} Undefined (or DNE)

\subsubsection{Part (b): Find the domain of $f$}

For the logarithm to be defined:
\[z - x^2 - y^2 > 0 \Rightarrow z > x^2 + y^2\]

\textbf{Domain:} $z > x^2 + y^2$

(This describes the region above the paraboloid $z = x^2 + y^2$ in three-dimensional space.)

\subsection{Problem 5}

Find and sketch the domain of $f(x, y) = \sqrt{x - 8y}$.

For the square root to be defined:
\[x - 8y \geq 0 \Rightarrow x \geq 8y\]

\textbf{Domain:} $\{(x, y) : x \geq 8y\}$

\textbf{Sketch:} This is the region on or below the line $x = 8y$ (or equivalently, $y = \frac{x}{8}$). Shade the region below and including this line.

\subsection{Problem 6}

Find and sketch the domain of $g(x, y) = \frac{x - y}{x + y}$.

The denominator cannot be zero:
\[x + y \neq 0 \Rightarrow y \neq -x\]

\textbf{Domain:} $\{(x, y) : x + y \neq 0\}$ or $\{(x, y) : y \neq -x\}$

\textbf{Sketch:} The entire $xy$-plane except the line $y = -x$. Draw the line $y = -x$ as a dashed line to show it's excluded.

\subsection{Problem 7}

Find and sketch the domain of $f(x, y, z) = \ln(36 - 9x^2 - 4y^2 - z^2)$.

For the logarithm to be defined:
\[36 - 9x^2 - 4y^2 - z^2 > 0 \Rightarrow 9x^2 + 4y^2 + z^2 < 36\]

Dividing by 36:
\[\frac{x^2}{4} + \frac{y^2}{9} + \frac{z^2}{36} < 1\]

\textbf{Domain:} $\frac{x^2}{4} + \frac{y^2}{9} + \frac{z^2}{36} < 1$

\textbf{Description:} This is the interior of an ellipsoid centered at the origin with semi-axes of lengths 2 (in the $x$-direction), 3 (in the $y$-direction), and 6 (in the $z$-direction).

\subsection{Problem 8 - Temperature-Humidity Index}

\subsubsection{Part (a): Value of $f(85, 70)$ and its meaning}

From the table, $f(85, 70) = 93$.

\textbf{Meaning:} When the actual temperature is $85^\circ$F and the relative humidity is 70\%, the perceived air temperature is approximately $93^\circ$F.

\subsubsection{Part (b): For what value of $h$ is $f(90, h) = 100$?}

Looking at the row where $T = 90$:
\begin{itemize}
    \item $f(90, 20) = 87$
    \item $f(90, 30) = 90$
    \item $f(90, 40) = 93$
    \item $f(90, 50) = 96$
    \item $f(90, 60) = 100$ \checkmark
\end{itemize}

\textbf{Answer:} $h = 60\%$

\subsubsection{Part (c): For what value of $T$ is $f(T, 40) = 86$?}

Looking at the column where $h = 40$:
\begin{itemize}
    \item $f(80, 40) = 79$
    \item $f(85, 40) = 86$ \checkmark
\end{itemize}

\textbf{Answer:} $T = 85^\circ$F

\subsubsection{Part (d): Meanings and comparison}

$I = f(80, h)$ means that $T$ is fixed at 80 and $h$ is allowed to vary, resulting in a function of $h$ that gives the humidex values for different relative humidities when the actual temperature is $80^\circ$F.

Similarly, $I = f(95, h)$ is a function of one variable that gives the humidex values for different relative humidities when the actual temperature is $95^\circ$F.

Looking at the rows corresponding to $T = 80$ and $T = 95$:
\begin{itemize}
    \item $f(80, h)$ \textbf{increases} at a relatively constant rate of approximately $1^\circ$F per 10\% relative humidity
    \item $f(95, h)$ \textbf{increases} more quickly (at first with an average rate of change of $3^\circ$F per 10\% relative humidity) and at \textbf{an increasing} rate (approximately $10^\circ$F per 10\% relative humidity for larger values of $h$)
\end{itemize}

\subsection{Problem 9}

Sketch the graph of $f(x, y) = x^2$.

This function depends only on $x$, not on $y$. For any fixed value of $x = a$, we have $z = a^2$ for all values of $y$. This creates a cylindrical surface: a parabola $z = x^2$ in the $xz$-plane, extended infinitely in the $y$-direction.

\textbf{Description:} Parabolic cylinder with axis along the $y$-axis, opening upward.

\subsection{Problem 10}

Sketch the graph of $f(x, y) = \sin(x)$ with cross sections.

\subsubsection{Cross section at $y = -\frac{\pi}{2}$}

Since $f$ doesn't depend on $y$, the cross section is:
\[z = \sin(x)\]

\subsubsection{Cross section at $y = 0$}

\[z = \sin(x)\]

\subsubsection{Cross section at $y = \frac{\pi}{2}$}

\[z = \sin(x)\]

\subsubsection{Cross section at $x = -\frac{\pi}{2}$}

For $x = -\frac{\pi}{2}$:
\[z = \sin\left(-\frac{\pi}{2}\right) = -1\]

\subsubsection{Cross section at $x = 0$}

For $x = 0$:
\[z = \sin(0) = 0\]

\subsubsection{Cross section at $x = \frac{\pi}{2}$}

For $x = \frac{\pi}{2}$:
\[z = \sin\left(\frac{\pi}{2}\right) = 1\]

\subsection{Problem 11 - Contour Maps}

Two contour maps are shown. One is for a cone, the other for a paraboloid.

For a cone $z = k\sqrt{x^2 + y^2}$, the level curves are circles with radii that increase \emph{linearly} with $z$. The spacing between contour lines is constant.

For a paraboloid $z = k(x^2 + y^2)$, the level curves are also circles, but their radii increase as the \emph{square root} of $z$. The spacing between contour lines decreases as you move outward.

\textbf{Answer:} Map II is the paraboloid. Map I is the cone. The cone's $z$-values change at a constant rate.

\subsection{Problem 12}

Tutorial exercise (to be completed interactively).

\subsection{Problem 13 - Contour Map Interpretation}

Given a contour map with level curves at $z = 1, 3, 5, 7, 9$ and the equation $x^2 + y^2 = 4.8$ for $z = 3$.

\subsubsection{Cross section at $z = 1$}

The level curve satisfies $f(x, y) = 1$. If the pattern is $x^2 + y^2 = c \cdot z$ for some constant $c$, then:
\[x^2 + y^2 = 1.6\]

\subsubsection{Cross section at $z = 3$}

Given: $x^2 + y^2 = 4.8$

\subsubsection{Cross section at $z = 5$}

\[x^2 + y^2 = 8.0\]

\subsubsection{Cross section at $z = 7$}

\[x^2 + y^2 = 11.2\]

\subsubsection{Cross section at $z = 9$}

\[x^2 + y^2 = 14.4\]

The general pattern is $x^2 + y^2 = 1.6z$.

\subsection{Problem 14}

Sketch the graph based on a given contour map. (Interactive graphing problem)

\subsection{Problem 15}

Sketch the graph based on a given contour map. (Interactive graphing problem)

\subsection{Problem 16}

Draw a contour map of $f(x, y) = x^2 - y^2$ showing several level curves.

The level curves satisfy:
\[x^2 - y^2 = k\]

These are hyperbolas:
\begin{itemize}
    \item For $k > 0$: hyperbolas opening horizontally
    \item For $k < 0$: hyperbolas opening vertically
    \item For $k = 0$: the pair of lines $y = \pm x$
\end{itemize}

\subsection{Problem 17}

Draw a contour map of $f(x, y) = ye^x$ showing several level curves.

The level curves satisfy:
\[ye^x = k \Rightarrow y = ke^{-x}\]

These are exponential curves:
\begin{itemize}
    \item For $k > 0$: exponentially decreasing curves in the upper half-plane
    \item For $k < 0$: exponentially increasing curves in the lower half-plane
    \item For $k = 0$: the $x$-axis
\end{itemize}

\subsection{Problem 18}

Draw a contour map of $f(x, y) = \sqrt{x^2 + y^2}$ showing several level curves.

The level curves satisfy:
\[\sqrt{x^2 + y^2} = k \Rightarrow x^2 + y^2 = k^2\]

These are concentric circles centered at the origin with radii $k \geq 0$.

\subsection{Problem 19}

Consider $z = \sin(xy)$.

\subsubsection{Part (a): Match with graph}

The function $z = \sin(xy)$ is periodic in the product $xy$. It oscillates between $-1$ and $1$. Along the axes ($x = 0$ or $y = 0$), we have $z = 0$. The function is symmetric about $y = x$ since $\sin(xy) = \sin(yx)$.

\textbf{Answer: Graph E}

\subsubsection{Part (b): Match with contour map}

The level curves satisfy $\sin(xy) = k$, which gives $xy = \arcsin(k) + 2\pi n$. These are rectangular hyperbolas. The contour map should show hyperbolic curves.

\textbf{Answer: Contour Map VI}

\subsubsection{Reasoning}

This function is \textbf{periodic} in both $x$ and $y$, and the function is \textbf{unchanged} when $x$ is interchanged with $y$, so its graph is \textbf{symmetric} about the plane $y = x$. In addition, the function is \textbf{zero} along the $x$- and $y$-axes. These conditions are satisfied only by \textbf{Graph E} and \textbf{Contour Map VI}.

\subsection{Problem 20}

Consider $z = \sin(x - y)$.

\subsubsection{Part (a): Match with graph}

The function $z = \sin(x - y)$ depends only on the difference $x - y$. Along lines of the form $x - y = c$, the function is constant. It's periodic along these diagonal lines.

\textbf{Answer: Graph F}

\subsubsection{Part (b): Match with contour map}

The level curves satisfy $\sin(x - y) = k$, which gives $x - y = \arcsin(k) + 2\pi n$. These are parallel straight lines.

\textbf{Answer: Contour Map I}

\subsubsection{Reasoning}

This function is \textbf{periodic} in both $x$ and $y$ but is \textbf{constant} along the lines $y = x + k$, a condition satisfied only by \textbf{Graph F} and \textbf{Contour Map I}.

\newpage
\section{In-Depth Analysis of Problems and Techniques}

\subsection{Problem Types and General Approach}

\subsubsection{Type 1: Function Evaluation Problems (Problems 1, 2a, 3a, 4a)}

\textbf{Strategy:} Direct substitution of given values into the function expression.

\textbf{Steps:}
\begin{enumerate}
    \item Identify the function formula
    \item Substitute the given values for each variable
    \item Simplify using order of operations
    \item Check if the result is defined (e.g., no division by zero, no logarithm of non-positive numbers)
\end{enumerate}

\subsubsection{Type 2: Domain Determination Problems (Problems 2b, 3b, 4b, 5, 6, 7)}

\textbf{Strategy:} Identify restrictions based on the mathematical operations in the function.

\textbf{Common restrictions:}
\begin{itemize}
    \item Square roots: argument $\geq 0$
    \item Logarithms: argument $> 0$
    \item Rational functions: denominator $\neq 0$
    \item Even roots: argument $\geq 0$
\end{itemize}

\textbf{Steps:}
\begin{enumerate}
    \item Identify all operations that impose restrictions
    \item Write inequality or equation for each restriction
    \item Solve the inequality/equation
    \item Express domain in set notation
    \item Sketch the region in the appropriate coordinate space
\end{enumerate}

\subsubsection{Type 3: Range Determination Problems (Problems 2c, 3c)}

\textbf{Strategy:} Analyze the possible output values over the entire domain.

\textbf{Steps:}
\begin{enumerate}
    \item Consider the range of inner functions
    \item Apply outer function transformations
    \item Check boundary behavior
    \item Consider if all intermediate values are achieved
\end{enumerate}

\subsubsection{Type 4: Function Composition and Substitution (Problem 1c, 1d)}

\textbf{Strategy:} Replace variables with expressions and simplify.

\textbf{Example:} For $f(x + h, y)$, replace every $x$ with $(x + h)$ and simplify.

\subsubsection{Type 5: Table Interpretation Problems (Problem 8)}

\textbf{Strategy:} Read values from the table and understand functional relationships.

\textbf{Key skills:}
\begin{itemize}
    \item Direct lookup for specific values
    \item Interpolation or pattern recognition
    \item Understanding rate of change from discrete data
\end{itemize}

\subsubsection{Type 6: Graph Sketching (Problems 9, 10)}

\textbf{Strategy:} Analyze how the function depends on each variable and construct the 3D surface.

\textbf{For functions of the form $z = f(x)$ only:} The graph is a cylinder (curve extruded along the $y$-axis).

\textbf{For general functions:} Examine cross-sections at various $x$ and $y$ values.

\subsubsection{Type 7: Level Curve Problems (Problems 13, 16, 17, 18)}

\textbf{Strategy:} Set $f(x, y) = k$ for various constants $k$ and sketch the resulting curves.

\textbf{Common level curves:}
\begin{itemize}
    \item $x^2 + y^2 = k$: circles
    \item $x^2 - y^2 = k$: hyperbolas
    \item $y = ke^{-x}$: exponential curves
    \item $ax + by = k$: parallel lines
\end{itemize}

\subsubsection{Type 8: Matching Problems (Problems 11, 19, 20)}

\textbf{Strategy:} Identify key features of the function and match with corresponding visual characteristics.

\textbf{Key features to check:}
\begin{itemize}
    \item Symmetry properties
    \item Behavior along axes
    \item Periodicity
    \item Constant values along specific curves or planes
    \item Shape of level curves (circles, hyperbolas, lines, etc.)
\end{itemize}

\subsection{Key Algebraic and Calculus Manipulations}

\subsubsection{Manipulation 1: Inequality Solving for Domains}

\textbf{Example from Problem 2:} For $g(x, y) = x\ln(x^2 + y)$, we need:
\[x^2 + y > 0 \Rightarrow y > -x^2\]

\textbf{Why it's crucial:} This transforms a functional requirement into a geometric region. The parabola $y = -x^2$ is the boundary, and the domain is the region above it.

\subsubsection{Manipulation 2: Completing Inequalities for 3D Domains}

\textbf{Example from Problem 7:} For $f(x, y, z) = \ln(36 - 9x^2 - 4y^2 - z^2)$:
\[36 - 9x^2 - 4y^2 - z^2 > 0\]

Rearranging:
\[9x^2 + 4y^2 + z^2 < 36\]

Dividing by 36 to get standard form:
\[\frac{x^2}{4} + \frac{y^2}{9} + \frac{z^2}{36} < 1\]

\textbf{Why it's crucial:} This reveals the domain as the interior of an ellipsoid, a fundamental 3D geometric shape.

\subsubsection{Manipulation 3: Exponential and Logarithmic Properties}

\textbf{Example from Problem 3:} $h(x, y) = e^{y - x^2}$

The range is $(0, \infty)$ because:
\begin{itemize}
    \item The exponential function $e^u$ is always positive
    \item As $u \to -\infty$, $e^u \to 0^+$
    \item As $u \to \infty$, $e^u \to \infty$
    \item Since $y - x^2$ can achieve any real value, $e^{y-x^2}$ achieves all positive values
\end{itemize}

\subsubsection{Manipulation 4: Recognizing Standard Forms}

\textbf{Example from Problem 18:} $f(x, y) = \sqrt{x^2 + y^2}$

The level curves $\sqrt{x^2 + y^2} = k$ immediately suggest:
\[x^2 + y^2 = k^2\]

Recognizing this as the equation of a circle with radius $k$ is essential.

\subsubsection{Manipulation 5: Symmetry Recognition}

\textbf{Example from Problem 19:} $z = \sin(xy)$

Since $xy = yx$, we have $\sin(xy) = \sin(yx)$, so $f(x, y) = f(y, x)$. This means the function is symmetric about the line $y = x$ in the domain.

\subsubsection{Manipulation 6: Understanding Function Dependencies}

\textbf{Example from Problem 20:} $z = \sin(x - y)$

The function depends only on the combination $x - y$, not on $x$ and $y$ independently. This means:
\begin{itemize}
    \item Along any line $x - y = c$, the function is constant
    \item The level curves are lines parallel to $y = x$
\end{itemize}

\newpage
\section{Cheatsheet and Tips for Success}

\subsection{Essential Formulas}

\begin{enumerate}
    \item \textbf{Domain of $\sqrt{g(x,y)}$:} $g(x,y) \geq 0$
    \item \textbf{Domain of $\ln(g(x,y))$:} $g(x,y) > 0$
    \item \textbf{Domain of $\frac{1}{g(x,y)}$:} $g(x,y) \neq 0$
    \item \textbf{Level curve:} $f(x, y) = k$ (equation in the $xy$-plane)
    \item \textbf{Range of $e^{g(x,y)}$:} $(0, \infty)$
    \item \textbf{Standard circle:} $x^2 + y^2 = r^2$
    \item \textbf{Standard ellipse:} $\frac{x^2}{a^2} + \frac{y^2}{b^2} = 1$
    \item \textbf{Standard hyperbola:} $\frac{x^2}{a^2} - \frac{y^2}{b^2} = \pm 1$ or $x^2 - y^2 = k$
    \item \textbf{Standard ellipsoid:} $\frac{x^2}{a^2} + \frac{y^2}{b^2} + \frac{z^2}{c^2} = 1$
\end{enumerate}

\subsection{Tricks and Shortcuts}

\begin{enumerate}
    \item \textbf{Quick domain check:} Look for square roots, logarithms, and denominators first.
    
    \item \textbf{For cylindrical surfaces:} If $z = f(x)$ (no $y$ dependence) or $z = f(y)$ (no $x$ dependence), the graph is a cylinder.
    
    \item \textbf{Symmetry shortcuts:}
    \begin{itemize}
        \item If $f(x, y) = f(y, x)$, the function is symmetric about $y = x$
        \item If $f(x, y) = f(-x, y)$, the function is symmetric about the $y$-axis
        \item If $f(x, y) = f(x, -y)$, the function is symmetric about the $x$-axis
    \end{itemize}
    
    \item \textbf{Level curve shortcut:} For $f(x, y) = g(x) + h(y)$, the level curves are more complex. But for $f(x, y) = g(x)h(y)$, set $g(x)h(y) = k$ and solve for $y$ in terms of $x$.
    
    \item \textbf{Range of compositions:}
    \begin{itemize}
        \item If the outer function is $e^u$, range is $(0, \infty)$
        \item If the outer function is $u^2$, range is $[0, \infty)$
        \item If the outer function is $\sin(u)$ or $\cos(u)$, range is $[-1, 1]$
    \end{itemize}
    
    \item \textbf{For matching problems:} Check axes behavior first (where $x = 0$ or $y = 0$), then check for periodicity, then check symmetry.
\end{enumerate}

\subsection{Common Pitfalls and Mistakes}

\begin{enumerate}
    \item \textbf{Forgetting strict vs. non-strict inequalities:}
    \begin{itemize}
        \item $\sqrt{g} \Rightarrow g \geq 0$ (includes equality)
        \item $\ln(g) \Rightarrow g > 0$ (strict inequality)
    \end{itemize}
    
    \item \textbf{Boundary inclusion errors:} When sketching domains, use solid curves for $\geq$ or $\leq$, dashed curves for $>$ or $<$.
    
    \item \textbf{Sign errors in substitution:} When evaluating $f(-3, -1)$, carefully track negative signs: $(-1)^2 = 1$, not $-1$.
    
    \item \textbf{Confusing $f(x, x)$ vs. $f(x, y)$:} In $f(x, x)$, both variables are replaced with the same value.
    
    \item \textbf{Range errors:} Don't confuse domain restrictions with range. $f(x, y) = \ln(x)$ has domain $x > 0$, but range $(-\infty, \infty)$.
    
    \item \textbf{Incomplete domain descriptions:} For three-variable functions, describe the 3D region clearly. "$z > x^2 + y^2$" is much clearer than "above a surface."
    
    \item \textbf{Misidentifying level curves:}
    \begin{itemize}
        \item $x^2 + y^2 = k$ gives circles (if $k > 0$)
        \item $x^2 - y^2 = k$ gives hyperbolas (if $k \neq 0$)
        \item Don't confuse these!
    \end{itemize}
    
    \item \textbf{Forgetting to check if a point is in the domain:} Before evaluating $f(a, b)$, verify $(a, b)$ is in the domain. Problem 4a is a trap: the point $(3, -4, 9)$ is not in the domain.
\end{enumerate}

\subsection{Recognition Tips}

\textbf{How to recognize problem type:}

\begin{itemize}
    \item "Evaluate $f(a, b)$" → Direct substitution
    \item "Find the domain" → Look for restrictions from operations
    \item "Find the range" → Analyze output values over entire domain
    \item "Sketch the graph" → Visualize 3D surface or analyze cross-sections
    \item "Draw a contour map" → Set $f(x, y) = k$ and sketch multiple level curves
    \item "Match the function with..." → Identify key features and compare
\end{itemize}

\newpage
\section{Conceptual Synthesis and The Big Picture}

\subsection{Thematic Connections}

\subsubsection{Core Theme: Extending Single-Variable Concepts to Higher Dimensions}

The central theme of multivariable functions is the natural extension of familiar single-variable concepts to situations where the output depends on multiple independent inputs.

\textbf{Connections to previous topics:}

\begin{enumerate}
    \item \textbf{From $y = f(x)$ to $z = f(x, y)$:}
    \begin{itemize}
        \item Single-variable: Graph is a curve in 2D
        \item Multivariable: Graph is a surface in 3D
    \end{itemize}
    
    \item \textbf{Domain and Range:}
    \begin{itemize}
        \item Single-variable: Domain is an interval or union of intervals on the real line
        \item Multivariable: Domain is a region in 2D or 3D space
    \end{itemize}
    
    \item \textbf{Visualization:}
    \begin{itemize}
        \item Single-variable: We directly graph $y = f(x)$
        \item Multivariable: We use contour maps (level curves) to represent 3D surfaces in 2D
    \end{itemize}
    
    \item \textbf{Function evaluation:}
    \begin{itemize}
        \item Single-variable: $f(a)$ is the $y$-value when $x = a$
        \item Multivariable: $f(a, b)$ is the $z$-value when $x = a$ and $y = b$
    \end{itemize}
\end{enumerate}

\subsection{Forward and Backward Links}

\subsubsection{Backward Links: Building on Previous Concepts}

Functions of several variables are a natural extension of:

\begin{enumerate}
    \item \textbf{Functions (Algebra and Precalculus):}
    \begin{itemize}
        \item The concept of a function as a rule assigning outputs to inputs
        \item Domain and range
        \item Function notation
    \end{itemize}
    
    \item \textbf{Coordinate Geometry:}
    \begin{itemize}
        \item Graphing in 2D
        \item Equations of curves (circles, parabolas, hyperbolas, ellipses)
        \item 3D coordinate systems
    \end{itemize}
    
    \item \textbf{Single-Variable Calculus:}
    \begin{itemize}
        \item Understanding graphs and level sets
        \item Behavior of exponential, logarithmic, and trigonometric functions
    \end{itemize}
\end{enumerate}

\subsubsection{Forward Links: Foundation for Future Topics}

Understanding multivariable functions is absolutely essential for:

\begin{enumerate}
    \item \textbf{Partial Derivatives (Chapter 14.3):}
    \begin{itemize}
        \item You cannot take partial derivatives without understanding multivariable functions
        \item $\frac{\partial f}{\partial x}$ measures how $f$ changes as $x$ varies while $y$ is held constant
    \end{itemize}
    
    \item \textbf{Gradients and Directional Derivatives (Chapter 14.5-14.6):}
    \begin{itemize}
        \item The gradient $\nabla f$ points in the direction of steepest ascent
        \item Understanding level curves is crucial: the gradient is always perpendicular to level curves
    \end{itemize}
    
    \item \textbf{Optimization (Chapter 14.7-14.8):}
    \begin{itemize}
        \item Finding maxima and minima of functions of several variables
        \item Applications to economics (utility maximization), engineering (design optimization), and finance (portfolio optimization)
    \end{itemize}
    
    \item \textbf{Multiple Integrals (Chapters 15):}
    \begin{itemize}
        \item Double and triple integrals extend the concept of area and volume
        \item The domain of integration is described using inequalities, just like domains of functions
    \end{itemize}
    
    \item \textbf{Vector Calculus (Chapters 16-17):}
    \begin{itemize}
        \item Vector fields assign vectors to points, a generalization of functions
        \item Line integrals, surface integrals, and theorems like Green's and Stokes' theorem
    \end{itemize}
\end{enumerate}

\newpage
\section{Real-World Application and Modeling}

\subsection{Concrete Scenarios}

\subsubsection{Scenario 1: Option Pricing in Quantitative Finance}

\textbf{Context:} An investment bank needs to price European call options on a stock. The value of the option depends on multiple factors simultaneously.

\textbf{The Problem:} The Black-Scholes model gives the price of a European call option as a function of several variables:
\[C = f(S, K, T, r, \sigma)\]

where:
\begin{itemize}
    \item $S$ = current stock price
    \item $K$ = strike price
    \item $T$ = time to maturity
    \item $r$ = risk-free interest rate
    \item $\sigma$ = volatility of the stock
\end{itemize}

\textbf{Why multivariable functions are essential:} To hedge their position, traders need to understand how the option price changes when \emph{any} of these variables change. They compute "Greeks" (Delta, Gamma, Vega, Theta, Rho), which are essentially partial derivatives of the pricing function with respect to each variable.

\subsubsection{Scenario 2: Portfolio Optimization in Asset Management}

\textbf{Context:} A portfolio manager must construct an optimal portfolio from $n$ different assets. The portfolio's expected return and risk both depend on how much capital is allocated to each asset.

\textbf{The Problem:} If $w_1, w_2, \ldots, w_n$ represent the weights (proportions) invested in each asset, then:
\begin{itemize}
    \item Expected return: $R(w_1, \ldots, w_n) = \sum_{i=1}^n w_i \mu_i$
    \item Portfolio variance (risk): $\sigma^2(w_1, \ldots, w_n) = \sum_{i=1}^n \sum_{j=1}^n w_i w_j \sigma_{ij}$
\end{itemize}

The manager must maximize return while keeping risk below a threshold, subject to $\sum w_i = 1$ and $w_i \geq 0$.

\textbf{Why multivariable functions are essential:} This is a constrained optimization problem in $n$ dimensions. Understanding the domain (the simplex of valid portfolios), level curves (portfolios with equal return or equal risk), and optimization techniques are all rooted in multivariable function theory.

\subsubsection{Scenario 3: Credit Risk Modeling}

\textbf{Context:} A credit analyst needs to assess the probability that a borrower will default on a loan.

\textbf{The Problem:} The default probability is modeled as:
\[P = f(\text{credit score}, \text{debt-to-income ratio}, \text{loan amount}, \text{interest rate}, \text{employment status})\]

This is a multivariable function where each input variable contributes to the overall risk assessment.

\textbf{Why multivariable functions are essential:} To price the loan correctly, the lender must understand:
\begin{itemize}
    \item The domain: What combinations of variables are even possible?
    \item Sensitivity: How does default probability change as each variable changes?
    \item Level sets: What combinations of variables give the same risk level?
\end{itemize}

All of these require a solid understanding of multivariable functions.

\subsection{Model Problem Setup}

\subsubsection{Detailed Setup: Simplified Black-Scholes Call Option Pricing}

Consider a simplified version of option pricing where we fix the strike price $K = 100$ and risk-free rate $r = 0.05$, leaving the call option value as a function of current stock price $S$ and volatility $\sigma$:

\[C(S, \sigma) = S \cdot N(d_1) - 100e^{-0.05T} \cdot N(d_2)\]

where:
\[d_1 = \frac{\ln(S/100) + (0.05 + \sigma^2/2)T}{\sigma\sqrt{T}}\]
\[d_2 = d_1 - \sigma\sqrt{T}\]

and $N(\cdot)$ is the cumulative standard normal distribution function, and we fix $T = 1$ year.

\textbf{Domain:} $S > 0$ and $\sigma > 0$ (stock prices and volatilities must be positive).

\textbf{Questions a trader might ask:}
\begin{enumerate}
    \item What is $C(110, 0.25)$? (Evaluate the function at a specific point)
    \item How does $C$ change as $S$ increases while $\sigma$ is held constant? (This leads to Delta, $\frac{\partial C}{\partial S}$)
    \item What combinations of $(S, \sigma)$ give an option value of \$15? (This is a level curve: $C(S, \sigma) = 15$)
    \item What is the range of possible option values? (Range analysis)
\end{enumerate}

\textbf{The integral/equation to solve:} While the Black-Scholes formula itself doesn't require integration in its final form, understanding the derivation requires solving a partial differential equation:

\[\frac{\partial C}{\partial t} + \frac{1}{2}\sigma^2 S^2 \frac{\partial^2 C}{\partial S^2} + rS\frac{\partial C}{\partial S} - rC = 0\]

This PDE is solved subject to the boundary condition $C(S, T) = \max(S - K, 0)$ at expiration.

\newpage
\section{Common Variations and Untested Concepts}

\subsection{Concept 1: Functions Defined Piecewise}

\textbf{What it is:} Some multivariable functions are defined using different formulas in different regions of the domain.

\textbf{Example:}
\[f(x, y) = \begin{cases}
x^2 + y^2 & \text{if } x^2 + y^2 \leq 1 \\
1 & \text{if } x^2 + y^2 > 1
\end{cases}\]

\textbf{Worked Example:} Evaluate $f(0.5, 0.5)$ and $f(2, 3)$.

\textbf{Solution:}
\begin{itemize}
    \item For $(0.5, 0.5)$: $x^2 + y^2 = 0.25 + 0.25 = 0.5 \leq 1$, so use the first formula:
    \[f(0.5, 0.5) = 0.5^2 + 0.5^2 = 0.5\]
    
    \item For $(2, 3)$: $x^2 + y^2 = 4 + 9 = 13 > 1$, so use the second formula:
    \[f(2, 3) = 1\]
\end{itemize}

\subsection{Concept 2: Limits of Multivariable Functions}

\textbf{What it is:} We say $\lim_{(x,y) \to (a,b)} f(x, y) = L$ if $f(x, y)$ gets arbitrarily close to $L$ as $(x, y)$ approaches $(a, b)$ along \emph{any} path.

\textbf{Key difference from single-variable:} In single-variable calculus, we can only approach from left or right. In multivariable calculus, there are infinitely many paths to approach a point.

\textbf{Example:} Does $\lim_{(x,y) \to (0,0)} \frac{xy}{x^2 + y^2}$ exist?

\textbf{Solution:} Test different paths:
\begin{itemize}
    \item Along $y = 0$: $\frac{x \cdot 0}{x^2 + 0} = 0 \to 0$
    \item Along $x = 0$: $\frac{0 \cdot y}{0 + y^2} = 0 \to 0$
    \item Along $y = x$: $\frac{x \cdot x}{x^2 + x^2} = \frac{x^2}{2x^2} = \frac{1}{2} \to \frac{1}{2}$
\end{itemize}

Since we get different limits along different paths, the limit does not exist.

\subsection{Concept 3: Continuity of Multivariable Functions}

\textbf{What it is:} A function $f(x, y)$ is continuous at $(a, b)$ if:
\begin{enumerate}
    \item $f(a, b)$ is defined
    \item $\lim_{(x,y) \to (a,b)} f(x, y)$ exists
    \item $\lim_{(x,y) \to (a,b)} f(x, y) = f(a, b)$
\end{enumerate}

\textbf{Example:} Is $f(x, y) = \frac{x^2 - y^2}{x - y}$ continuous at $(1, 1)$?

\textbf{Solution:} At $(1, 1)$, the function is $\frac{0}{0}$ (indeterminate form), so $f(1, 1)$ is not defined in the original form. However, we can simplify:
\[f(x, y) = \frac{(x-y)(x+y)}{x-y} = x + y \quad \text{for } x \neq y\]

So $\lim_{(x,y) \to (1,1)} f(x, y) = 1 + 1 = 2$. If we define $f(1, 1) = 2$, the function becomes continuous.

\subsection{Concept 4: Conic Sections in Level Curves}

\textbf{What it is:} Many important functions have level curves that are conic sections (circles, ellipses, parabolas, hyperbolas).

\textbf{Example:} Find and classify the level curves of $f(x, y) = x^2 + 4y^2$.

\textbf{Solution:} The level curves satisfy:
\[x^2 + 4y^2 = k\]

For $k > 0$, dividing by $k$:
\[\frac{x^2}{k} + \frac{y^2}{k/4} = 1\]

These are ellipses with semi-major axis $\sqrt{k}$ (horizontal) and semi-minor axis $\sqrt{k/4} = \frac{\sqrt{k}}{2}$ (vertical).

For $k = 0$, we get just the single point $(0, 0)$.

For $k < 0$, there are no real solutions (empty level curve).

\subsection{Concept 5: Quadric Surfaces}

\textbf{What it is:} Graphs of functions involving second-degree polynomials in three variables are called quadric surfaces.

\textbf{Common types:}
\begin{enumerate}
    \item \textbf{Ellipsoid:} $\frac{x^2}{a^2} + \frac{y^2}{b^2} + \frac{z^2}{c^2} = 1$
    \item \textbf{Hyperboloid of one sheet:} $\frac{x^2}{a^2} + \frac{y^2}{b^2} - \frac{z^2}{c^2} = 1$
    \item \textbf{Hyperboloid of two sheets:} $-\frac{x^2}{a^2} - \frac{y^2}{b^2} + \frac{z^2}{c^2} = 1$
    \item \textbf{Elliptic paraboloid:} $z = \frac{x^2}{a^2} + \frac{y^2}{b^2}$
    \item \textbf{Hyperbolic paraboloid (saddle):} $z = \frac{x^2}{a^2} - \frac{y^2}{b^2}$
\end{enumerate}

\textbf{Example:} Identify the surface $z = x^2 - y^2$.

\textbf{Solution:} This is a hyperbolic paraboloid (saddle surface). 
\begin{itemize}
    \item Along $y = 0$: $z = x^2$ (parabola opening upward)
    \item Along $x = 0$: $z = -y^2$ (parabola opening downward)
    \item Level curves: $x^2 - y^2 = k$ (hyperbolas)
\end{itemize}

\newpage
\section{Advanced Diagnostic Testing: Find the Flaw}

Each problem below contains a subtle but critical error. Your task is to identify the mistake, explain why it's wrong, and provide the correct solution.

\subsection{Flawed Problem 1}

\textbf{Problem:} Find the domain of $f(x, y) = \sqrt{9 - x^2 - y^2}$.

\textbf{Flawed Solution:}

For the square root to be defined:
\[9 - x^2 - y^2 \geq 0\]
\[x^2 + y^2 \geq 9\]

Therefore, the domain is the set of points outside or on the circle of radius 3.

\textbf{Domain:} $\{(x, y) : x^2 + y^2 \geq 9\}$

\vspace{0.3cm}
\noindent\textbf{Find the flaw, explain it in one sentence, and provide the correct answer.}

\subsection{Flawed Problem 2}

\textbf{Problem:} Evaluate $g(x, y) = \frac{x^2 - y^2}{x - y}$ at the point $(3, 3)$.

\textbf{Flawed Solution:}

Substitute $x = 3$ and $y = 3$:
\[g(3, 3) = \frac{3^2 - 3^2}{3 - 3} = \frac{9 - 9}{0} = \frac{0}{0} = 0\]

\textbf{Answer:} $0$

\vspace{0.3cm}
\noindent\textbf{Find the flaw, explain it in one sentence, and provide the correct answer.}

\subsection{Flawed Problem 3}

\textbf{Problem:} Find the range of $h(x, y) = e^{-x^2 - y^2}$.

\textbf{Flawed Solution:}

The function $e^u$ has range $(0, \infty)$ for all real $u$. Since $-x^2 - y^2$ can be any real number, the range of $h$ is $(0, \infty)$.

\textbf{Range:} $(0, \infty)$

\vspace{0.3cm}
\noindent\textbf{Find the flaw, explain it in one sentence, and provide the correct answer.}

\subsection{Flawed Problem 4}

\textbf{Problem:} Describe the level curves of $f(x, y) = \ln(x^2 + y^2)$.

\textbf{Flawed Solution:}

The level curves satisfy:
\[\ln(x^2 + y^2) = k\]
\[x^2 + y^2 = e^k\]

Since $e^k > 0$ for all real $k$, the level curves are circles of radius $\sqrt{e^k}$ for any real number $k$.

\vspace{0.3cm}
\noindent\textbf{Find the flaw, explain it in one sentence, and provide the correct answer.}

\subsection{Flawed Problem 5}

\textbf{Problem:} Find the domain of $f(x, y, z) = \frac{1}{\sqrt{x^2 + y^2 + z^2 - 1}}$.

\textbf{Flawed Solution:}

The square root requires:
\[x^2 + y^2 + z^2 - 1 \geq 0\]
\[x^2 + y^2 + z^2 \geq 1\]

And the denominator requires:
\[x^2 + y^2 + z^2 - 1 \neq 0\]
\[x^2 + y^2 + z^2 \neq 1\]

Combining these, the domain is:
\[\{(x, y, z) : x^2 + y^2 + z^2 > 1\}\]

\vspace{0.3cm}
\noindent\textbf{Find the flaw (if any), explain it, and provide the correct answer.}

\vspace{1cm}
\noindent\textit{Note: The purpose of these problems is to develop critical reading and error-detection skills essential for exams and real-world problem-solving.}

\newpage
\section{Answers to Diagnostic Problems}

\subsection{Answer to Flawed Problem 1}

\textbf{The Flaw:} The inequality was incorrectly reversed when rearranging; subtracting $9$ from both sides and multiplying by $-1$ should have flipped the inequality sign.

\textbf{Correct Solution:}
\[9 - x^2 - y^2 \geq 0\]
\[-x^2 - y^2 \geq -9\]
\[x^2 + y^2 \leq 9\]

\textbf{Correct Domain:} $\{(x, y) : x^2 + y^2 \leq 9\}$ (the closed disk of radius 3)

\subsection{Answer to Flawed Problem 2}

\textbf{The Flaw:} The expression $\frac{0}{0}$ is indeterminate, not equal to zero; the point $(3, 3)$ is not in the domain since it makes the denominator zero.

\textbf{Correct Solution:} The point $(3, 3)$ is not in the domain of $g$ because it makes the denominator $x - y = 0$. Therefore, $g(3, 3)$ is undefined (DNE).

Alternatively, we could simplify first:
\[g(x, y) = \frac{(x-y)(x+y)}{x-y} = x + y \quad \text{for } x \neq y\]

Then $\lim_{(x,y) \to (3,3)} g(x, y) = 6$, but $g(3, 3)$ itself is still undefined.

\textbf{Correct Answer:} Undefined (DNE)

\subsection{Answer to Flawed Problem 3}

\textbf{The Flaw:} The expression $-x^2 - y^2$ cannot be any real number because it's always non-positive (it equals zero at the origin and is negative everywhere else).

\textbf{Correct Solution:}

Since $x^2 \geq 0$ and $y^2 \geq 0$, we have:
\[-x^2 - y^2 \leq 0\]

with equality only at $(0, 0)$.

Therefore:
\[e^{-x^2 - y^2} \leq e^0 = 1\]

with the maximum value of 1 achieved at $(0, 0)$.

As $(x, y) \to \infty$ (moving away from the origin), $-x^2 - y^2 \to -\infty$, so $e^{-x^2 - y^2} \to 0^+$.

\textbf{Correct Range:} $(0, 1]$

\subsection{Answer to Flawed Problem 4}

\textbf{The Flaw:} The function $\ln(x^2 + y^2)$ requires $x^2 + y^2 > 0$, which excludes the origin, so level curves only exist for values of $k$ such that $e^k > 0$ (which is always true), but the domain restriction means we cannot include the point $(0, 0)$.

Actually, there's a more subtle issue: Since the domain excludes $(0, 0)$, all level curves with $k \in \mathbb{R}$ are valid circles, but the statement "for any real number $k$" is correct. The actual flaw is that we should note the domain of $f$ excludes the origin.

\textbf{Better Critique:} The solution should explicitly state that the domain of $f$ is $\{(x, y) : x^2 + y^2 > 0\} = \mathbb{R}^2 \setminus \{(0, 0)\}$ (all points except the origin).

\textbf{Correct Answer:} The level curves are circles $x^2 + y^2 = e^k$ of radius $r = e^{k/2}$ for any $k \in \mathbb{R}$, but the function is only defined for $(x, y) \neq (0, 0)$.

\subsection{Answer to Flawed Problem 5}

\textbf{The Flaw:} This solution is actually correct! There is no flaw.

The square root in the denominator requires $x^2 + y^2 + z^2 - 1 > 0$ (strictly greater than, not just $\geq$, because it's in the denominator). This gives:
\[x^2 + y^2 + z^2 > 1\]

\textbf{Correct Domain:} $\{(x, y, z) : x^2 + y^2 + z^2 > 1\}$ (the region outside the unit sphere)

\textit{Note: This problem tests whether you can recognize when a solution is actually correct, an important skill to avoid second-guessing yourself on exams.}

\end{document}