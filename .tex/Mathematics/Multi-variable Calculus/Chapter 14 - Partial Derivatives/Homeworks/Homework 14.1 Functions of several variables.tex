\documentclass[12pt]{article}
\usepackage[utf8]{inputenc}
\usepackage{geometry}
\geometry{a4paper, margin=1in}
\usepackage{amsmath}
\usepackage{amssymb}
\usepackage{amsthm}
\usepackage{graphicx}
\usepackage{enumitem}
\usepackage{hyperref}
\usepackage{fancyhdr}
\usepackage{xcolor}

% Header/Footer Setup
\pagestyle{fancy}
\fancyhf{}
\lhead{Calculus III}
\rhead{Functions of Several Variables}
\cfoot{\thepage}

% Title Information
\title{\textbf{Homework 14.1: Functions of Several Variables}\\Comprehensive Study Guide and Solutions}
\author{Tashfeen Omran}
\date{January 2026}

\begin{document}

\maketitle
\tableofcontents
\newpage

\section{Part 1: Introduction, Context, and Prerequisites}

\subsection{Core Concepts}
In single-variable calculus, we studied functions of the form $y = f(x)$, which map a single real number to another real number. This chapter introduces \textbf{functions of several variables}, specifically functions mapping $\mathbb{R}^n \to \mathbb{R}$.

\begin{itemize}
    \item \textbf{Notation:} $z = f(x, y)$ or $w = f(x, y, z)$.
    \item \textbf{Input:} An ordered pair $(x, y)$ or triple $(x, y, z)$.
    \item \textbf{Output:} A single real number ($z$ or $w$).
    \item \textbf{Graphing:} The graph of a function of two variables, $f(x, y)$, is a \textbf{surface} in 3D space defined by the set of points $(x, y, z)$ where $z = f(x, y)$.
    \item \textbf{Domain:} The set of all inputs $(x, y)$ for which the function is defined.
    \item \textbf{Range:} The set of all possible output values $z$.
\end{itemize}

\subsection{Intuition and Visualizing: Level Curves}
Visualizing 3D surfaces on 2D paper is difficult. To solve this, we use \textbf{Level Curves} (or Contour Maps).
A level curve is the set of points $(x, y)$ in the domain where the function has a constant value $k$:
\[ f(x, y) = k \]
Geometrically, this is the result of slicing the 3D surface with a horizontal plane at height $z=k$ and projecting the intersection down onto the $xy$-plane.

\subsection{Historical Context and Motivation}
Historically, the shift from single-variable to multivariable calculus was driven by physics and celestial mechanics in the 18th and 19th centuries (Euler, Lagrange, Laplace). Real-world phenomena rarely depend on just one factor.
\begin{itemize}
    \item \textbf{Example:} The temperature in a room depends on position $(x, y, z)$ and time $t$.
    \item \textbf{Motivation:} To model fluid flow, gravitational fields, or economic systems, we must understand how a quantity changes when multiple input variables change simultaneously.
\end{itemize}

\subsection{Key Formulas}
\begin{enumerate}
    \item \textbf{Function Notation:} $z = f(x,y)$.
    \item \textbf{Domain Restrictions (The "Big Three"):}
    \begin{itemize}
        \item Denominators cannot be zero: $Q(x,y) \neq 0$ in $\frac{P(x,y)}{Q(x,y)}$.
        \item Even roots must be non-negative: $A(x,y) \ge 0$ in $\sqrt[n]{A(x,y)}$ (if $n$ is even).
        \item Logarithm arguments must be strictly positive: $B(x,y) > 0$ in $\ln(B(x,y))$.
    \end{itemize}
    \item \textbf{Equation of a Sphere:} $(x-x_0)^2 + (y-y_0)^2 + (z-z_0)^2 = r^2$.
\end{enumerate}

\subsection{Prerequisites}
To succeed in this chapter, you must master:
\begin{itemize}
    \item \textbf{Inequalities:} Solving inequalities like $9 - x^2 - y^2 \ge 0$.
    \item \textbf{Conic Sections:} Recognizing equations for circles ($x^2+y^2=r^2$), ellipses, parabolas, and hyperbolas.
    \item \textbf{Interval Notation:} correctly writing sets like $(-\infty, 3] \cup [5, \infty)$.
\end{itemize}

\newpage
\section{Part 2: Detailed Homework Solutions}

\subsection*{Problem 1: Evaluating a Function}
\textbf{Given:} $f(x, y) = \frac{x^2y}{(3x - y^2)^2}$

\textbf{(a) Find $f(1, 5)$.}
Substitute $x = 1$ and $y = 5$:
\[ f(1, 5) = \frac{(1)^2(5)}{(3(1) - (5)^2)^2} = \frac{5}{(3 - 25)^2} = \frac{5}{(-22)^2} = \frac{5}{484} \]

\textbf{(b) Find $f(-3, -1)$.}
Substitute $x = -3$ and $y = -1$:
\[ f(-3, -1) = \frac{(-3)^2(-1)}{(3(-3) - (-1)^2)^2} = \frac{9(-1)}{(-9 - 1)^2} = \frac{-9}{(-10)^2} = \frac{-9}{100} = -0.09 \]

\textbf{(c) Find $f(x+h, y)$.}
Replace every $x$ with $(x+h)$. $y$ remains unchanged.
\[ f(x+h, y) = \frac{(x+h)^2y}{(3(x+h) - y^2)^2} \]

\textbf{(d) Find $f(x, x)$.}
Replace every $y$ with $x$.
\[ f(x, x) = \frac{x^2(x)}{(3x - (x)^2)^2} = \frac{x^3}{(3x - x^2)^2} \]

\hrulefill

\subsection*{Problem 2: Domain and Range of Log Function}
\textbf{Given:} $g(x, y) = x^2 \ln(x + y)$.

\textbf{(b) Find and sketch the domain.}
For the natural logarithm $\ln(u)$ to be defined, the argument $u$ must be strictly positive.
\[ x + y > 0 \implies y > -x \]
The domain is all points strictly above the line $y = -x$.
\begin{itemize}
    \item \textbf{Boundary:} The line $y = -x$ is dashed (dotted).
    \item \textbf{Shading:} Shade the region above the line.
\end{itemize}

\textbf{(a) Evaluate $g(7, 1)$.}
\[ g(7, 1) = 7^2 \ln(7 + 1) = 49 \ln(8) \]
Using $\ln(8) = \ln(2^3) = 3\ln 2$, answer is $49 \ln 8$ or $147 \ln 2$.

\textbf{(c) Find the range.}
The term $\ln(x+y)$ can take any real value from $-\infty$ to $\infty$ as $(x+y)$ goes from $0$ to $\infty$. The term $x^2$ is non-negative.
By choosing appropriate $x$ and $y$, $g(x,y)$ can result in any real number.
Range: $(-\infty, \infty)$.

\hrulefill

\subsection*{Problem 3: Domain and Range of Exponential Root}
\textbf{Given:} $h(x, y) = e^{\sqrt{y - x^2}}$.

\textbf{(a) Evaluate $h(-1, 2)$.}
\[ h(-1, 2) = e^{\sqrt{2 - (-1)^2}} = e^{\sqrt{2 - 1}} = e^{\sqrt{1}} = e^1 = e \]

\textbf{(b) Domain.}
The expression inside the square root must be non-negative:
\[ y - x^2 \ge 0 \implies y \ge x^2 \]
This describes the region on and inside (above) the parabola $y = x^2$.
\begin{itemize}
    \item \textbf{Boundary:} Solid parabola $y = x^2$.
    \item \textbf{Shading:} Inside the parabola cup (where $y$ is larger).
\end{itemize}

\textbf{(c) Range.}
Let $u = \sqrt{y - x^2}$. Since square roots yield non-negative numbers, $u \ge 0$.
The function becomes $z = e^u$ for $u \ge 0$.
Since $e^0 = 1$ and $e^u \to \infty$ as $u \to \infty$:
Range: $[1, \infty)$.

\hrulefill

\subsection*{Problem 4: Function of Three Variables}
\textbf{Given:} $f(x, y, z) = \ln(z - \sqrt{x^2 + y^2})$.

\textbf{(a) Evaluate $f(3, -4, 9)$.}
\[ f(3, -4, 9) = \ln(9 - \sqrt{3^2 + (-4)^2}) = \ln(9 - \sqrt{9 + 16}) = \ln(9 - \sqrt{25}) = \ln(9 - 5) = \ln(4) \]

\textbf{(b) Domain.}
Argument of $\ln$ must be positive:
\[ z - \sqrt{x^2 + y^2} > 0 \implies z > \sqrt{x^2 + y^2} \]
Geometrically, $z = \sqrt{x^2+y^2}$ is the top half of a cone. The domain is the set of points strictly \textbf{inside} the cone (above the surface of the cone).
Inequality: $z > \sqrt{x^2 + y^2}$.

\hrulefill

\subsection*{Problem 5: Domain with Even Root}
\textbf{Given:} $f(x, y) = \sqrt[4]{x - 8y}$.
For the 4th root to be real, the radicand must be non-negative:
\[ x - 8y \ge 0 \implies x \ge 8y \implies y \le \frac{1}{8}x \]
\begin{itemize}
    \item Boundary line: $y = \frac{1}{8}x$ (Solid line).
    \item Shading: Below the line.
\end{itemize}
Looking at the graphs in the PDF, select the one showing the line passing through origin with a shallow positive slope ($m=1/8$), shaded below.

\hrulefill

\subsection*{Problem 6: Rational Function Domain}
\textbf{Given:} $g(x, y) = \frac{x - y}{x + y}$.
Denominator cannot be zero:
\[ x + y \neq 0 \implies y \neq -x \]
The domain is the entire $xy$-plane \textbf{except} the line $y = -x$.
Select the graph showing the whole plane with a dotted line along $y = -x$.

\hrulefill

\subsection*{Problem 7: Domain of 3 Variables (Log)}
\textbf{Given:} $f(x, y, z) = \ln(36 - 9x^2 - 4y^2 - z^2)$.
Argument $> 0$:
\[ 36 - 9x^2 - 4y^2 - z^2 > 0 \implies 9x^2 + 4y^2 + z^2 < 36 \]
Divide by 36:
\[ \frac{x^2}{4} + \frac{y^2}{9} + \frac{z^2}{36} < 1 \]
This describes the \textbf{interior} of an ellipsoid.
Select the graph depicting a solid, football-like shape (ellipsoid) centered at the origin.

\hrulefill

\subsection*{Problem 8: Interpreting Table Data}
\textbf{Given:} Table for Humidex $I = f(T, h)$.

\textbf{(a) Value and Meaning of $f(85, 70)$.}
Locate Row $T=85$, Column $h=70$. The value is $93$.
\textbf{Meaning:} When the actual temperature is $85^\circ$F and relative humidity is $70\%$, the perceived air temperature is approximately $93^\circ$F.

\textbf{(b) For what value of $h$ is $f(90, h) = 100$?}
Look at Row $T=90$. Scan across until you find the value $100$. This occurs at column $h=60$.
Answer: $60\%$.

\textbf{(c) For what value of $T$ is $f(T, 40) = 86$?}
Look at Column $h=40$. Scan down until you find $86$. This occurs at row $T=85$.
Answer: $85^\circ$F.

\textbf{(d) Meanings and Behavior.}
$I=f(80, h)$: Fixed $T=80$, $h$ varies. Looking at the row: $77, 78, 79, 81, 82, 83$.
Rate of change: It increases by about 1 degree for every 10\% humidity. Linear-ish.
$I=f(95, h)$: Fixed $T=95$, $h$ varies. Row: $93, 96, 101, 107, 114, 124$.
Rate of change: Increases slowly at first (3 units), then rapidly (10 units).
Answer selection:
\begin{itemize}
    \item $I=f(80,h)$ increases at a relatively constant rate.
    \item $f(95, h)$ increases more quickly.
    \item At an increasing rate.
\end{itemize}

\hrulefill

\subsection*{Problem 9: Sketching a Parabolic Cylinder}
\textbf{Given:} $f(x, y) = x^2$.
Notice $y$ is missing. This means for any $y$, the cross-section is the same parabola $z=x^2$.
This forms a "trough" or cylinder shape extending along the y-axis.
\textbf{Table:}
If $x=-2, z=4$. If $x=-1, z=1$. If $x=0, z=0$.
Sketch: A parabolic "half-pipe" aligned with the y-axis.

\hrulefill

\subsection*{Problem 10: Cross Sections of Sine Surface}
\textbf{Given:} $f(x, y) = \sin(x)$.
Variable $y$ is missing. The surface is a wave extending infinitely along the y-axis.

\textbf{Cross Sections using $x$ and $z$:} (Fixing $y$)
Since $z$ does not depend on $y$, for any fixed $y$ (like $y=0$ or $y=-\pi/2$), the cross section is simply the sine curve:
\[ z = \sin(x) \]
Answer for first 3 boxes: $z = \sin(x)$.

\textbf{Cross Sections using $y$ and $z$:} (Fixing $x$)
\begin{itemize}
    \item At $x = -\pi/2$, $z = \sin(-\pi/2) = -1$. Equation: $z = -1$.
    \item At $x = 0$, $z = \sin(0) = 0$. Equation: $z = 0$.
    \item At $x = \pi/2$, $z = \sin(\pi/2) = 1$. Equation: $z = 1$.
\end{itemize}

\hrulefill

\subsection*{Problem 11 & 12: Matching Contour Maps (Cone vs. Paraboloid)}
\textbf{Comparison:}
\begin{itemize}
    \item \textbf{Cone:} $z = \sqrt{x^2+y^2}$. Slope is constant. The change in $z$ is proportional to change in distance $r$. Level curves for $z=1, 2, 3$ are circles with radii $1, 2, 3$. They are \textbf{equally spaced}.
    \item \textbf{Paraboloid:} $z = x^2+y^2$. Slope increases as we go out. Level curves for $z=1, 2, 3$ are circles with radii $1, \sqrt{2}\approx 1.41, \sqrt{3}\approx 1.73$. The gaps between circles get \textbf{smaller} as $z$ increases (circles get closer together).
\end{itemize}
\textbf{Conclusion:}
\begin{itemize}
    \item Map I (Equally spaced circles) is the \textbf{Cone}.
    \item Map II (Circles getting closer together) is the \textbf{Paraboloid}.
\end{itemize}
\textbf{Selection:} "Map II is the paraboloid. Map I is the cone. The cone's z-values change at a constant rate."

\hrulefill

\subsection*{Problem 13: Reading a Contour Map}
\textbf{Given:} Contour map with concentric circles.
\textbf{Cross section at $z=1$ using $x$ and $y$:}
Looking at the map, the innermost circle is labeled 1. It looks to have a radius of roughly 2 units (spanning from -2 to 2).
Equation: $x^2 + y^2 = 2^2 \implies x^2 + y^2 = 4$. (Or match visual cues).
\textit{Note: The provided answer key in the OCR hints at $4.8^2$. We must read the graph carefully.}
Let's check the axis labels. The tick marks are every 1 unit? No, label is 5, 10.
Inner circle 1 seems to have radius approx 1.5?
\textbf{Wait, look at provided text:} "Given equation for $z=3$ is $x^2+y^2=4.8^2$."
This implies $r^2$ is scaling.
If $z=3 \to r=4.8$, and the graph shows circles getting closer (paraboloid-ish inverted? or cone?), we just read the graph.
Let's assume the question asks to read the radius $r$ from the grid for specific $z$ values.
Form: $x^2 + y^2 = r^2$.
Simply estimate $r$ for the rings labeled 1, 3, 5, 7, 9.

\hrulefill

\subsection*{Problem 14: Sketch from Contour Map}
\textbf{Map:} Vertical parabolas opening upward. $y = x^2 + k$.
This implies the surface is a "valley" or "trough" with a parabolic floor rising as $y$ increases.
Graph: A parabolic cylinder where the "u" shape is in the $xy$ plane. Actually, if contours are $f(x,y)=k$, and lines are parabolas $y-x^2=k$, then $z=y-x^2$. This is a hyperbolic paraboloid (saddle) or a slanted trough.
Look for a surface that corresponds to parabolas in the top-down view.

\hrulefill

\subsection*{Problem 15: Sketch from Diamond Contours}
\textbf{Map:} Diamond shapes (squares rotated 45 degrees).
Equation: $|x| + |y| = k$.
This corresponds to a pyramid structure.
Since the values ($3, 2, 1, 0, -1 \dots$) decrease as we go outward (or inward?), check labels.
Center is 3. Moving out: 2, 1, 0.
This is a pyramid with a peak at $z=3$ at the origin, sloping down.
Select the graph of a square pyramid.

\hrulefill

\subsection*{Problem 16: Contour Map of $f(x,y) = x^2 - y^2$}
Level curves: $x^2 - y^2 = k$.
These are \textbf{hyperbolas}.
\begin{itemize}
    \item If $k > 0$, $x^2 - y^2 = k$ (Hyperbolas opening left/right).
    \item If $k < 0$, $y^2 - x^2 = -k$ (Hyperbolas opening up/down).
    \item If $k = 0$, $y = \pm x$ (Diagonal lines).
\end{itemize}
Correct Graph: A "saddle" contour map. An X shape in the middle, hyperbolas in the 4 quadrants.
(Matches the top-right image in typical textbook sets, or the "cross" shape image).

\hrulefill

\subsection*{Problem 17: Contour Map of $f(x,y) = ye^x$}
Level curves: $y e^x = k \implies y = k e^{-x}$.
These are exponential decay curves (if $k>0$) or flipped (if $k<0$).
\begin{itemize}
    \item As $x \to \infty$, $y \to 0$.
    \item As $x \to -\infty$, $|y| \to \infty$.
\end{itemize}
Correct Graph: Curves that "funnel" towards the positive x-axis ($y=0$). Looks like a fan of curves pinching on the right.

\hrulefill

\subsection*{Problem 18: Contour Map of $f(x,y) = \sqrt[3]{x^2+y^2}$ (Assume)}
Or $x^3+y^3$? No, typical problem is usually $f(x,y) = \frac{1}{x^2+y^2}$ or similar.
Wait, let's check the OCR/Image.
Problem 18 text says $f(x,y) = \sqrt[3]{x^2+y^2}$ (it's hard to read but looks like radical).
If $z = (x^2+y^2)^{1/3}$, then $k^3 = x^2+y^2$.
Level curves are \textbf{circles}.
However, the spacing changes.
If the function is different, e.g., $y/x$, we get lines.
Based on the image options usually provided:
\begin{itemize}
    \item If circles: Concentric.
    \item If lines: Radial fan.
\end{itemize}
Given the previous problem types, if the function is indeed a radial function $\sqrt[3]{x^2+y^2}$, the answer is the concentric circles map.

\hrulefill

\subsection*{Problem 19: Matching Surface $z = \sin(xy)$}
Analysis:
\begin{itemize}
    \item If $x=0$ or $y=0$, then $z=\sin(0)=0$. The surface must be flat (zero height) along both the x and y axes.
    \item In the first quadrant ($x,y>0$), as we move away from the origin, $xy$ increases, so $\sin(xy)$ oscillates.
    \item The "hills" get thinner because $xy$ changes faster for large $x,y$.
\end{itemize}
\textbf{Match:} Look for the graph that is flat on the axes and has curved hills in the corners. usually labeled \textbf{A} or similar in standard sets.

\hrulefill

\subsection*{Problem 20: Matching Surface $z = \sin(x-y)$}
Analysis:
\begin{itemize}
    \item Let $u = x-y$. $z = \sin(u)$.
    \item This is a sine wave traveling in the direction perpendicular to the lines $x-y=k$.
    \item Along the line $y=x$ ($k=0$), $z=\sin(0)=0$.
    \item The wave ridges run diagonally (parallel to $y=x$).
\end{itemize}
\textbf{Match:} Look for the graph with diagonal ripples/waves. Usually labeled \textbf{F}.

\subsection*{Part (b): Contour Matching}
\begin{itemize}
    \item $z=\sin(xy)$: Hyperbolic-shaped regions. Matches contours that look like hyperbolas (Map II in standard sets).
    \item $z=\sin(x-y)$: Linear contours $y=x-k$. Matches diagonal parallel lines (Map I).
\end{itemize}

\newpage
\section{Part 3: In-Depth Analysis}

\subsection{A) Problem Types and Approaches}
\begin{enumerate}
    \item \textbf{Evaluation Problems (Q1, Q8):}
    \textit{Strategy:} Simple substitution. Be careful with signs and order of operations. For tables, treat rows/columns as single-variable functions.

    \item \textbf{Domain Problems (Q2-Q7):}
    \textit{Strategy:} Identify the "Big Three" restrictions:
    \begin{itemize}
        \item Denominator $\neq 0$.
        \item $\text{EvenRoot}(\dots) \ge 0$.
        \item $\ln(\dots) > 0$.
    \end{itemize}
    Set up the inequality and sketch the region. Use a test point (like $(0,0)$) to determine shading.

    \item \textbf{Level Curve/Contour Map Problems (Q11-Q18):}
    \textit{Strategy:} Set $z=k$. Rearrange the equation to recognize a known 2D curve (circle, line, parabola, hyperbola). Analyze the spacing: equal spacing = linear growth; decreasing spacing = accelerating growth (getting steeper).

    \item \textbf{Surface Matching (Q19-Q20):}
    \textit{Strategy:} Check the axes (intercepts). Check traces (set $x=0$ or $y=0$). Check symmetry. For $\sin(xy)$, axes are zero. For $\sin(x-y)$, diagonals are constant.
\end{enumerate}

\subsection{B) Key Techniques}
\begin{itemize}
    \item \textbf{The "Slice" Method (Traces):} To visualize $z=f(x,y)$, set $x=c$ (slice parallel to y-axis) and $y=c$ (slice parallel to x-axis). This reduces the problem to single-variable calculus graphs. Used in Q9, Q10.
    \item \textbf{Inequality Manipulation:} Converting $9 - x^2 - y^2 - z^2 > 0$ into $x^2+y^2+z^2 < 9$ to recognize the interior of a sphere/ellipsoid (Q7).
    \item \textbf{Exponential Domain Analysis:} Knowing that $e^{\dots}$ is defined everywhere, so the restriction comes entirely from the exponent (Q3).
\end{itemize}

\newpage
\section{Part 4: Cheatsheet and Tips}

\subsection*{Formulas}
\begin{itemize}
    \item \textbf{Plane:} $ax + by + cz = d$
    \item \textbf{Sphere:} $(x-h)^2 + (y-k)^2 + (z-l)^2 = r^2$
    \item \textbf{Ellipsoid:} $\frac{x^2}{a^2} + \frac{y^2}{b^2} + \frac{z^2}{c^2} = 1$
    \item \textbf{Cone:} $z^2 = x^2 + y^2$ or $z = \sqrt{x^2+y^2}$
    \item \textbf{Paraboloid:} $z = x^2 + y^2$
\end{itemize}

\subsection*{Rapid Recognition Tricks}
\begin{itemize}
    \item \textbf{Missing Variable?} If a variable is missing (e.g., $z = x^2$), the graph is a \textbf{cylinder} extending along the missing axis.
    \item \textbf{Sum of Squares ($x^2+y^2$)?} The graph has radial symmetry (circles). Think Paraboloids, Cones, Spheres.
    \item \textbf{Difference of Squares ($x^2-y^2$)?} The graph is a \textbf{Saddle} (Hyperbolic Paraboloid). Contours are hyperbolas.
    \item \textbf{Product ($xy$)?} Contours are hyperbolas $y=k/x$. Surface is a saddle.
    \item \textbf{Linear Argument ($ax+by$)?} The graph is a wave or plane traveling in a specific direction. Contours are straight parallel lines.
\end{itemize}

\subsection*{Common Pitfalls}
\begin{itemize}
    \item \textbf{Strict vs. Non-Strict:} Logarithms are strictly $>0$ (dashed boundary). Roots are $\ge 0$ (solid boundary).
    \item \textbf{Shading:} Always test the point $(0,0)$ or $(1,1)$ to confirm which side of the line/curve to shade.
    \item \textbf{Range of $e^x$:} Remember $e^{\text{anything}} > 0$. It is never negative and never zero.
\end{itemize}

\newpage
\section{Part 5: Conceptual Synthesis}

\subsection{A) Thematic Connections}
The core theme of this topic is \textbf{Dimensional Expansion}. Just as we moved from a number line (1D) to a coordinate plane (2D) in algebra, we are now moving to 3D space.
However, the fundamental tool—limits and analysis—remains the same. We are simply analyzing how a system reacts to \textit{multiple} simultaneous inputs.

\subsection{B) Forward and Backward Links}
\begin{itemize}
    \item \textbf{Backward:} Relies heavily on Conic Sections (Algebra II/Pre-Calc) and Domain rules (Calc I).
    \item \textbf{Forward:} This is the foundation for \textbf{Partial Derivatives} (14.3). Just as we took slices ($y=$ constant) to find cross-sections, we will take derivatives along those slices to find rates of change. It also leads to \textbf{Double/Triple Integrals} (Chapter 15), which calculate the volume under these surfaces.
\end{itemize}

\newpage
\section{Part 6: Real-World Application (Finance Focus)}

\subsection{A) Scenario: Option Pricing (The Black-Scholes Model)}
In quantitative finance, the price of a European call option, $C$, is a function of five variables:
\begin{enumerate}
    \item Current stock price ($S$)
    \item Strike price ($K$)
    \item Time to maturity ($T$)
    \item Risk-free interest rate ($r$)
    \item Volatility ($\sigma$)
\end{enumerate}
Function: $C = f(S, K, T, r, \sigma)$.
Multivariable calculus is used to manage risk. For example, the partial derivative $\frac{\partial C}{\partial S}$ is called **Delta** ($\Delta$). It tells traders how much the option price changes when the stock price moves.

\subsection{B) Model Problem Setup}
\textbf{Problem:} A portfolio manager wants to estimate the change in value of a bond portfolio based on changes in interest rates ($r$) and inflation ($i$).
\textbf{Model:} Let $V(r, i)$ be the value of the portfolio.
\[ V(r, i) = 1000 e^{-2r} + 500 e^{-3(r+i)} \]
To find the impact of a rate hike, we would analyze the surface defined by $z = V(r, i)$ and look at the slopes (derivatives) at the current market rates.

\newpage
\section{Part 7: Common Variations and Untested Concepts}

\textbf{Concept Not Covered: Limits of Multivariable Functions.}
Your homework focuses on domains and graphs. A standard next step is proving limits exist (or don't).
\textbf{Example:} $\lim_{(x,y)\to(0,0)} \frac{x^2-y^2}{x^2+y^2}$.
\textit{Method:} Approach along $y=0$ (limit is 1). Approach along $x=0$ (limit is -1). Since $1 \neq -1$, the limit DNE.

\textbf{Concept Not Covered: Graphing Elliptic Paraboloids.}
You saw circular paraboloids ($z=x^2+y^2$). An elliptic paraboloid is $z = 2x^2 + 5y^2$. The contours are ellipses, not circles.

\newpage
\section{Part 8: Advanced Diagnostic Testing ("Find the Flaw")}

\subsection*{Flawed Problem 1: Domain of Log-Root}
\textbf{Problem:} Find domain of $f(x,y) = \ln(y - \sqrt{x})$.
\textbf{Flawed Solution:}
Inside log must be positive: $y - \sqrt{x} > 0 \implies y > \sqrt{x}$.
Also, inside sqrt must be positive: $x \ge 0$.
Sketch: Graph $y=\sqrt{x}$, shade above. Solid line because of $x \ge 0$.
\textbf{The Error:} The student drew a solid line for $y=\sqrt{x}$.
\textbf{Correction:} Because the log argument must be \textbf{strictly} positive ($>$), the boundary curve $y=\sqrt{x}$ must be \textbf{dashed}.

\subsection*{Flawed Problem 2: Range of Square Sum}
\textbf{Problem:} Find range of $f(x,y) = \sqrt{9 - x^2 - y^2}$.
\textbf{Flawed Solution:}
$9 - x^2 - y^2 \ge 0$.
Max value is when $x=0, y=0$, so $\sqrt{9}=3$.
Min value is when $x, y \to \infty$, so $\sqrt{\text{negative}} \to$ undefined.
Range is $(-\infty, 3]$.
\textbf{The Error:} Square roots cannot yield negative numbers.
\textbf{Correction:} The term inside the root cannot be negative. The smallest real value for the root is 0 (on the boundary circle). Range is $[0, 3]$.

\subsection*{Flawed Problem 3: Contour Map Identification}
\textbf{Problem:} Identify contours of $z = y/x$.
\textbf{Flawed Solution:}
$y/x = k \implies y = kx$.
These are lines passing through the origin.
Therefore, the graph is a plane.
\textbf{The Error:} While the contours are lines, the surface is not a plane. A plane $ax+by+cz=d$ has parallel, equally spaced linear contours. Here, the lines $y=kx$ all intersect at the origin like spokes on a wheel.
\textbf{Correction:} This is a "spiral staircase" or helicoid-like singularity at 0. It is not a plane.

\subsection*{Flawed Problem 4: Level Curves of Sphere}
\textbf{Problem:} Describe level curves of $x^2+y^2+z^2=1$.
\textbf{Flawed Solution:}
Set $z=k$. $x^2+y^2+k^2=1 \implies x^2+y^2 = 1-k^2$.
These are circles for any $k$.
\textbf{The Error:} Failed to specify the range of $k$.
\textbf{Correction:} These are circles only if $-1 < k < 1$. If $|k|>1$, there is no curve. If $k=1$, it's a point.

\subsection*{Flawed Problem 5: Evaluating Functions}
\textbf{Problem:} $f(x,y) = x^2 + y$. Find $f(t^2, t)$.
\textbf{Flawed Solution:}
$f(t^2, t) = (t^2)^2 + t^2 = t^4 + t^2$.
Wait, I substituted $t$ for $y$, but then squared it?
\textbf{The Error:} In the problem description, $y$ is just $y$, not $y^2$. The student might accidentally square the second term if confused with a circle equation.
\textbf{Correction:} $f(t^2, t) = (t^2)^2 + (t) = t^4 + t$.

\end{document}