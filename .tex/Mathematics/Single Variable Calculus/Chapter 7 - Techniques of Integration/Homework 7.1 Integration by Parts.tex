\documentclass{article}
\usepackage{amsmath}
\usepackage{amssymb}
\usepackage{geometry}
\geometry{a4paper, margin=1in}

\title{Homework 7.1 Integration by Parts}
\author{Tashfeen Omran}
\date{November 2025}

\begin{document}

\maketitle

\part{Comprehensive Introduction, Context, and Prerequisites}

\section{Core Concepts}
Integration by Parts is a fundamental technique of integration that is used to find the integral of a product of two functions. It is essentially the reverse of the product rule for differentiation. While u-substitution allows us to integrate composite functions, Integration by Parts provides a method for tackling products of functions, such as $\int x e^x \,dx$ or $\int x \sin(x) \,dx$.

The core idea is to take an integral that is difficult to solve, $\int u \,dv$, and transform it into an expression that includes a potentially much simpler integral, $\int v \,du$. The choice of which function in the product to call '$u$' and which to call '$dv$' is the key strategic decision in this process.

\section{Intuition and Derivation}
The formula for Integration by Parts is not arbitrary; it is derived directly from the product rule for differentiation.

Recall the product rule:
\[ \frac{d}{dx} [u(x)v(x)] = u'(x)v(x) + u(x)v'(x) \]
We can write this using differential notation as:
\[ d(uv) = v\,du + u\,dv \]
Now, if we integrate both sides of this equation, we get:
\[ \int d(uv) = \int v\,du + \int u\,dv \]
The integral of a differential, $\int d(uv)$, is simply the function itself, $uv$. So we have:
\[ uv = \int v\,du + \int u\,dv \]
By rearranging this equation to solve for $\int u\,dv$, we arrive at the celebrated formula for Integration by Parts:
\[ \int u \,dv = uv - \int v \,du \]
This formula is the heart of the technique. It tells us that we can trade the problem of integrating $u\,dv$ for the problem of integrating $v\,du$, with the hope that the second integral is easier to solve.

\section{Historical Context and Motivation}
The development of Integration by Parts is intrinsically linked to the invention of calculus by Isaac Newton and Gottfried Wilhelm Leibniz in the late 17th century. As they formalized the rules of differentiation and integration, they quickly encountered the need for methods to integrate more complex functions. The product rule for differentiation was well-understood, and reversing it to create a rule for integration was a natural and necessary step.

Mathematicians like the Bernoulli brothers and Leonhard Euler, working in the 18th century, extensively used and refined this technique. The primary motivation was to solve problems in physics and mechanics, which often involved differential equations. Many differential equations contain products of functions, and Integration by Parts became an indispensable tool for solving them. For instance, calculating the work done by a variable force or finding the center of mass of an object often leads to integrals of products, making this technique essential for the advancement of both mathematics and physical science.

\section{Key Formulas}
There are two primary forms of the Integration by Parts formula:

\begin{enumerate}
    \item \textbf{Indefinite Integral:} This is the most common form.
    \[ \int u \,dv = uv - \int v \,du \]
    \item \textbf{Definite Integral:} When integrating over an interval from $a$ to $b$.
    \[ \int_{a}^{b} u \,dv = [uv]_{a}^{b} - \int_{a}^{b} v \,du \]
    where $[uv]_{a}^{b} = u(b)v(b) - u(a)v(a)$.
\end{enumerate}

\section{Prerequisites}
To successfully master Integration by Parts, you must be proficient in the following prerequisite topics:
\begin{itemize}
    \item \textbf{Basic Differentiation Rules:} You must be able to quickly find derivatives of polynomial, exponential, logarithmic, and trigonometric functions.
    \item \textbf{The Product Rule:} Understanding the product rule is key to understanding the derivation and conceptual basis of the technique.
    \item \textbf{Basic Integration Rules:} You need to be fluent in reversing the differentiation rules to find antiderivatives (e.g., power rule, integrals of $e^x$, $\sin(x)$, $\cos(x)$, $\frac{1}{x}$, etc.).
    \item \textbf{U-Substitution:} Often, the integral $\int v \,du$ on the right side of the formula will require a u-substitution to be solved. Sometimes, an initial u-substitution is needed before applying Integration by Parts.
    \item \textbf{Algebraic Manipulation:} Skills such as simplifying fractions, factoring, and manipulating logarithmic and exponential expressions are crucial.
    \item \textbf{Trigonometric Identities:} You may need identities like $\sin(2x) = 2\sin(x)\cos(x)$ or $\tan^2(x) = \sec^2(x) - 1$ to simplify an integral before or after applying the technique.
\end{itemize}

\part{Detailed Homework Solutions}

\subsection*{Problem 1: Evaluate $\int xe^{6x} \,dx$}
\textbf{Setup:} We use integration by parts with $\int u \,dv = uv - \int v \,du$. We choose $u$ and $dv$ according to the LIATE principle (Log, Inverse Trig, Algebraic, Trig, Exponential).
Let $u = x$ and $dv = e^{6x} \,dx$.
\textbf{Steps:}
\begin{enumerate}
    \item Differentiate $u$: $du = dx$.
    \item Integrate $dv$: $v = \int e^{6x} \,dx = \frac{1}{6}e^{6x}$.
    \item Apply the formula:
    \begin{align*}
        \int xe^{6x} \,dx &= (x)\left(\frac{1}{6}e^{6x}\right) - \int \left(\frac{1}{6}e^{6x}\right) \,dx \\
        &= \frac{1}{6}xe^{6x} - \frac{1}{6} \int e^{6x} \,dx \\
        &= \frac{1}{6}xe^{6x} - \frac{1}{6} \left(\frac{1}{6}e^{6x}\right) + C \\
        &= \frac{1}{6}xe^{6x} - \frac{1}{36}e^{6x} + C
    \end{align*}
\end{enumerate}
\textbf{Final Answer:} $\frac{1}{6}xe^{6x} - \frac{1}{36}e^{6x} + C$

\subsection*{Problem 2: Evaluate $\int x\cos(9x) \,dx$}
\textbf{Setup:} Let $u = x$ and $dv = \cos(9x) \,dx$.
\textbf{Steps:}
\begin{enumerate}
    \item Differentiate $u$: $du = dx$.
    \item Integrate $dv$: $v = \int \cos(9x) \,dx = \frac{1}{9}\sin(9x)$.
    \item Apply the formula:
    \begin{align*}
        \int x\cos(9x) \,dx &= (x)\left(\frac{1}{9}\sin(9x)\right) - \int \left(\frac{1}{9}\sin(9x)\right) \,dx \\
        &= \frac{1}{9}x\sin(9x) - \frac{1}{9} \int \sin(9x) \,dx \\
        &= \frac{1}{9}x\sin(9x) - \frac{1}{9} \left(-\frac{1}{9}\cos(9x)\right) + C \\
        &= \frac{1}{9}x\sin(9x) + \frac{1}{81}\cos(9x) + C
    \end{align*}
\end{enumerate}
\textbf{Final Answer:} $\frac{1}{9}x\sin(9x) + \frac{1}{81}\cos(9x) + C$

\subsection*{Problem 3: Evaluate $\int w\ln(w) \,dw$}
\textbf{Setup:} Here, we have a logarithmic function. We choose it to be $u$.
Let $u = \ln(w)$ and $dv = w \,dw$.
\textbf{Steps:}
\begin{enumerate}
    \item Differentiate $u$: $du = \frac{1}{w} \,dw$.
    \item Integrate $dv$: $v = \int w \,dw = \frac{1}{2}w^2$.
    \item Apply the formula:
    \begin{align*}
        \int w\ln(w) \,dw &= (\ln(w))\left(\frac{1}{2}w^2\right) - \int \left(\frac{1}{2}w^2\right) \left(\frac{1}{w} \,dw\right) \\
        &= \frac{1}{2}w^2\ln(w) - \frac{1}{2} \int w \,dw \\
        &= \frac{1}{2}w^2\ln(w) - \frac{1}{2} \left(\frac{1}{2}w^2\right) + C \\
        &= \frac{1}{2}w^2\ln(w) - \frac{1}{4}w^2 + C
    \end{align*}
\end{enumerate}
\textbf{Final Answer:} $\frac{1}{2}w^2\ln(w) - \frac{1}{4}w^2 + C$

\subsection*{Problem 4: Evaluate $\int \frac{\ln(x)}{x^2} \,dx$}
\textbf{Setup:} Let $u = \ln(x)$ and $dv = \frac{1}{x^2} \,dx = x^{-2} \,dx$.
\textbf{Steps:}
\begin{enumerate}
    \item Differentiate $u$: $du = \frac{1}{x} \,dx$.
    \item Integrate $dv$: $v = \int x^{-2} \,dx = -x^{-1} = -\frac{1}{x}$.
    \item Apply the formula:
    \begin{align*}
        \int \frac{\ln(x)}{x^2} \,dx &= (\ln(x))\left(-\frac{1}{x}\right) - \int \left(-\frac{1}{x}\right) \left(\frac{1}{x} \,dx\right) \\
        &= -\frac{\ln(x)}{x} + \int \frac{1}{x^2} \,dx \\
        &= -\frac{\ln(x)}{x} + \int x^{-2} \,dx \\
        &= -\frac{\ln(x)}{x} - x^{-1} + C \\
        &= -\frac{\ln(x)}{x} - \frac{1}{x} + C
    \end{align*}
\end{enumerate}
\textbf{Final Answer:} $-\frac{\ln(x)}{x} - \frac{1}{x} + C$

\subsection*{Problem 5: Evaluate $\int \ln(\sqrt{x}) \,dx$}
\textbf{Setup:} First, simplify the integrand using the logarithm property $\ln(a^b) = b\ln(a)$.
\[ \int \ln(\sqrt{x}) \,dx = \int \ln(x^{1/2}) \,dx = \int \frac{1}{2}\ln(x) \,dx = \frac{1}{2} \int \ln(x) \,dx \]
Now we solve $\int \ln(x) \,dx$. Let $u = \ln(x)$ and $dv = dx$.
\textbf{Steps:}
\begin{enumerate}
    \item Differentiate $u$: $du = \frac{1}{x} \,dx$.
    \item Integrate $dv$: $v = \int dx = x$.
    \item Apply the formula to $\int \ln(x) \,dx$:
    \begin{align*}
        \int \ln(x) \,dx &= (\ln(x))(x) - \int (x)\left(\frac{1}{x} \,dx\right) \\
        &= x\ln(x) - \int 1 \,dx \\
        &= x\ln(x) - x + C_1
    \end{align*}
    \item Substitute back into the original problem:
    \[ \frac{1}{2} \int \ln(x) \,dx = \frac{1}{2}(x\ln(x) - x + C_1) = \frac{1}{2}x\ln(x) - \frac{1}{2}x + C \]
\end{enumerate}
\textbf{Final Answer:} $\frac{1}{2}x\ln(x) - \frac{1}{2}x + C$ (Note: The provided solution simplifies $x\ln(\sqrt{x}) - \frac{x}{2} + C$, which is equivalent.)

\subsection*{Problem 6: Evaluate $\int e^{8\theta}\sin(9\theta) \,d\theta$}
This is a "boomerang" integral requiring two applications of integration by parts. Let $I = \int e^{8\theta}\sin(9\theta) \,d\theta$.

\textbf{First Pass:} Let $u = \sin(9\theta)$ and $dv = e^{8\theta} \,d\theta$.
\begin{itemize}
    \item $du = 9\cos(9\theta) \,d\theta$
    \item $v = \frac{1}{8}e^{8\theta}$
\end{itemize}
\[ I = \frac{1}{8}e^{8\theta}\sin(9\theta) - \int \frac{1}{8}e^{8\theta} \cdot 9\cos(9\theta) \,d\theta = \frac{1}{8}e^{8\theta}\sin(9\theta) - \frac{9}{8}\int e^{8\theta}\cos(9\theta) \,d\theta \]

\textbf{Second Pass:} For the new integral, let $u_2 = \cos(9\theta)$ and $dv_2 = e^{8\theta} \,d\theta$.
\begin{itemize}
    \item $du_2 = -9\sin(9\theta) \,d\theta$
    \item $v_2 = \frac{1}{8}e^{8\theta}$
\end{itemize}
\[ \int e^{8\theta}\cos(9\theta) \,d\theta = \frac{1}{8}e^{8\theta}\cos(9\theta) - \int \frac{1}{8}e^{8\theta}(-9\sin(9\theta)) \,d\theta = \frac{1}{8}e^{8\theta}\cos(9\theta) + \frac{9}{8}\int e^{8\theta}\sin(9\theta) \,d\theta \]
The original integral $I$ has reappeared.

\textbf{Solve for I:} Substitute the result of the second pass back into the first equation.
\begin{align*}
    I &= \frac{1}{8}e^{8\theta}\sin(9\theta) - \frac{9}{8}\left[ \frac{1}{8}e^{8\theta}\cos(9\theta) + \frac{9}{8}I \right] \\
    I &= \frac{1}{8}e^{8\theta}\sin(9\theta) - \frac{9}{64}e^{8\theta}\cos(9\theta) - \frac{81}{64}I \\
    I + \frac{81}{64}I &= \frac{1}{8}e^{8\theta}\sin(9\theta) - \frac{9}{64}e^{8\theta}\cos(9\theta) \\
    \frac{145}{64}I &= \frac{8e^{8\theta}\sin(9\theta) - 9e^{8\theta}\cos(9\theta)}{64} \\
    I &= \frac{1}{145}e^{8\theta}(8\sin(9\theta) - 9\cos(9\theta)) + C
\end{align*}
\textbf{Final Answer:} $\frac{1}{145}e^{8\theta}(8\sin(9\theta) - 9\cos(9\theta)) + C$

\subsection*{Problem 7: Evaluate $\int e^{3\theta}\sin(4\theta) \,d\theta$}
This is another boomerang integral, laid out as a tutorial. Let $I = \int e^{3\theta}\sin(4\theta) \,d\theta$.
\textbf{Step 1 \& 2 (First Pass):} Let $u = \sin(4\theta)$ and $dv = e^{3\theta} \,d\theta$.
\begin{itemize}
    \item $du = 4\cos(4\theta) \,d\theta$
    \item $v = \frac{1}{3}e^{3\theta}$
\end{itemize}
\[ I = \frac{1}{3}e^{3\theta}\sin(4\theta) - \int \frac{1}{3}e^{3\theta}(4\cos(4\theta))\,d\theta = \frac{1}{3}e^{3\theta}\sin(4\theta) - \frac{4}{3}\int e^{3\theta}\cos(4\theta) \,d\theta \]
\textbf{Step 3 \& 4 (Second Pass):} For $\int e^{3\theta}\cos(4\theta) \,d\theta$, let $u_2 = \cos(4\theta)$ and $dv_2 = e^{3\theta} \,d\theta$.
\begin{itemize}
    \item $du_2 = -4\sin(4\theta) \,d\theta$
    \item $v_2 = \frac{1}{3}e^{3\theta}$
\end{itemize}
\[ \int e^{3\theta}\cos(4\theta) \,d\theta = \frac{1}{3}e^{3\theta}\cos(4\theta) - \int \frac{1}{3}e^{3\theta}(-4\sin(4\theta)) \,d\theta = \frac{1}{3}e^{3\theta}\cos(4\theta) + \frac{4}{3}I \]
\textbf{Step 5, 6, 7 (Solve for I):} Substitute back.
\begin{align*}
    I &= \frac{1}{3}e^{3\theta}\sin(4\theta) - \frac{4}{3}\left[ \frac{1}{3}e^{3\theta}\cos(4\theta) + \frac{4}{3}I \right] \\
    I &= \frac{1}{3}e^{3\theta}\sin(4\theta) - \frac{4}{9}e^{3\theta}\cos(4\theta) - \frac{16}{9}I \\
    I + \frac{16}{9}I &= \frac{1}{3}e^{3\theta}\sin(4\theta) - \frac{4}{9}e^{3\theta}\cos(4\theta) \\
    \frac{25}{9}I &= \frac{3e^{3\theta}\sin(4\theta) - 4e^{3\theta}\cos(4\theta)}{9} \\
    I &= \frac{1}{25}e^{3\theta}(3\sin(4\theta) - 4\cos(4\theta)) + C
\end{align*}
\textbf{Final Answer:} $\frac{1}{25}e^{3\theta}(3\sin(4\theta) - 4\cos(4\theta)) + C$

\subsection*{Problem 8: Evaluate $\int_{0}^{2\pi} x\sin(x)\cos(x) \,dx$}
\textbf{Setup:} First, use the identity $\sin(2x) = 2\sin(x)\cos(x)$, so $\sin(x)\cos(x) = \frac{1}{2}\sin(2x)$.
\[ \int_{0}^{2\pi} x\sin(x)\cos(x) \,dx = \int_{0}^{2\pi} x \left(\frac{1}{2}\sin(2x)\right) \,dx = \frac{1}{2} \int_{0}^{2\pi} x\sin(2x) \,dx \]
Now use integration by parts. Let $u = x$ and $dv = \sin(2x) \,dx$.
\textbf{Steps:}
\begin{enumerate}
    \item Differentiate $u$: $du = dx$.
    \item Integrate $dv$: $v = \int \sin(2x) \,dx = -\frac{1}{2}\cos(2x)$.
    \item Apply the definite integral formula:
    \begin{align*}
        \frac{1}{2} \int_{0}^{2\pi} x\sin(2x) \,dx &= \frac{1}{2} \left[ \left[x\left(-\frac{1}{2}\cos(2x)\right)\right]_{0}^{2\pi} - \int_{0}^{2\pi} \left(-\frac{1}{2}\cos(2x)\right) \,dx \right] \\
        &= \frac{1}{2} \left[ \left[-\frac{x}{2}\cos(2x)\right]_{0}^{2\pi} + \frac{1}{2} \int_{0}^{2\pi} \cos(2x) \,dx \right] \\
        &= \frac{1}{2} \left[ \left(-\frac{2\pi}{2}\cos(4\pi)\right) - \left(-\frac{0}{2}\cos(0)\right) + \frac{1}{2} \left[\frac{1}{2}\sin(2x)\right]_{0}^{2\pi} \right] \\
        &= \frac{1}{2} \left[ (-\pi \cdot 1) - 0 + \frac{1}{4}(\sin(4\pi) - \sin(0)) \right] \\
        &= \frac{1}{2} [-\pi + \frac{1}{4}(0-0)] = -\frac{\pi}{2}
    \end{align*}
\end{enumerate}
\textbf{Final Answer:} $-\frac{\pi}{2}$

\subsection*{Problem 9: Evaluate $\int_{0}^{t} 7e^s \sin(t-s) \,ds$}
Let $I = \int_{0}^{t} 7e^s \sin(t-s) \,ds$. This is a boomerang integral with respect to $s$.
\textbf{First Pass:} Let $u = 7\sin(t-s)$ and $dv = e^s \,ds$.
\begin{itemize}
    \item $du = 7\cos(t-s) \cdot (-1) \,ds = -7\cos(t-s) \,ds$ (by chain rule)
    \item $v = e^s$
\end{itemize}
\begin{align*}
    I &= [7e^s \sin(t-s)]_0^t - \int_0^t e^s (-7\cos(t-s)) \,ds \\
    &= (7e^t \sin(0) - 7e^0 \sin(t)) + 7 \int_0^t e^s \cos(t-s) \,ds \\
    &= -7\sin(t) + 7 \int_0^t e^s \cos(t-s) \,ds
\end{align*}
\textbf{Second Pass:} For the new integral, let $u_2 = \cos(t-s)$ and $dv_2 = e^s \,ds$.
\begin{itemize}
    \item $du_2 = -\sin(t-s) \cdot (-1) \,ds = \sin(t-s) \,ds$
    \item $v_2 = e^s$
\end{itemize}
\begin{align*}
    \int_0^t e^s \cos(t-s) \,ds &= [e^s \cos(t-s)]_0^t - \int_0^t e^s \sin(t-s) \,ds \\
    &= (e^t \cos(0) - e^0 \cos(t)) - \frac{1}{7} I \\
    &= (e^t - \cos(t)) - \frac{1}{7} I
\end{align*}
\textbf{Solve for I:} Substitute back.
\begin{align*}
    I &= -7\sin(t) + 7 \left[ (e^t - \cos(t)) - \frac{1}{7} I \right] \\
    I &= -7\sin(t) + 7e^t - 7\cos(t) - I \\
    2I &= 7e^t - 7\sin(t) - 7\cos(t) \\
    I &= \frac{7}{2}(e^t - \sin(t) - \cos(t))
\end{align*}
\textbf{Final Answer:} $\frac{7}{2}(e^t - \sin(t) - \cos(t))$

\subsection*{Problem 10: Evaluate $\int_{0}^{\pi} e^{\cos(t)}\sin(2t) \,dt$}
\textbf{Setup:} Use identity $\sin(2t) = 2\sin(t)\cos(t)$.
\[ \int_{0}^{\pi} e^{\cos(t)} 2\sin(t)\cos(t) \,dt \]
Now, use u-substitution. Let $x = \cos(t)$, so $dx = -\sin(t) \,dt$. We must also change the bounds.
\begin{itemize}
    \item When $t=0$, $x = \cos(0) = 1$.
    \item When $t=\pi$, $x = \cos(\pi) = -1$.
\end{itemize}
The integral becomes:
\[ \int_{1}^{-1} e^x \cdot 2x \cdot (-dx) = -2 \int_{1}^{-1} x e^x \,dx = 2 \int_{-1}^{1} x e^x \,dx \]
\textbf{Integration by Parts:} Let $u = x$ and $dv = e^x \,dx$.
\begin{itemize}
    \item $du = dx$
    \item $v = e^x$
\end{itemize}
\begin{align*}
    2 \int_{-1}^{1} x e^x \,dx &= 2 \left( [xe^x]_{-1}^{1} - \int_{-1}^{1} e^x \,dx \right) \\
    &= 2 \left( (1e^1 - (-1)e^{-1}) - [e^x]_{-1}^{1} \right) \\
    &= 2 \left( (e + e^{-1}) - (e^1 - e^{-1}) \right) \\
    &= 2 (e + e^{-1} - e + e^{-1}) \\
    &= 2 (2e^{-1}) = \frac{4}{e}
\end{align*}
\textbf{Final Answer:} $\frac{4}{e}$

\subsection*{Problem 11: Prove the reduction formula $\int (\ln(x))^n \,dx = x(\ln(x))^n - n\int(\ln(x))^{n-1} \,dx$}
\textbf{Setup:} Let $u = (\ln(x))^n$ and $dv = dx$.
\textbf{Steps:}
\begin{enumerate}
    \item Differentiate $u$: $du = n(\ln(x))^{n-1} \cdot \frac{1}{x} \,dx$ (by chain rule).
    \item Integrate $dv$: $v = \int dx = x$.
    \item Apply the formula:
    \begin{align*}
        \int (\ln(x))^n \,dx &= ((\ln(x))^n)(x) - \int x \left( n(\ln(x))^{n-1} \cdot \frac{1}{x} \,dx \right) \\
        &= x(\ln(x))^n - \int n(\ln(x))^{n-1} \,dx \\
        &= x(\ln(x))^n - n\int(\ln(x))^{n-1} \,dx
    \end{align*}
\end{enumerate}
\textbf{Final Answer:} The formula is proven.

\subsection*{Problem 12: Determine the reduction formula for $\int 9x^n e^x \,dx$}
\textbf{Setup:} Let $u = x^n$ and $dv = 9e^x \,dx$.
\textbf{Steps:}
\begin{enumerate}
    \item Differentiate $u$: $du = nx^{n-1} \,dx$.
    \item Integrate $dv$: $v = \int 9e^x \,dx = 9e^x$.
    \item Apply the formula:
    \begin{align*}
        \int 9x^n e^x \,dx &= (x^n)(9e^x) - \int (9e^x)(nx^{n-1} \,dx) \\
        &= 9x^n e^x - 9n \int x^{n-1}e^x \,dx
    \end{align*}
\end{enumerate}
\textbf{Final Answer:} The correct formula is $\int 9x^n e^x \,dx = 9x^n e^x - 9n \int x^{n-1}e^x \,dx$.

\subsection*{Problem 13: Determine the reduction formula for $\int 2\tan^n(x) \,dx$}
\textbf{Setup:} This problem is best solved by first splitting $\tan^n(x)$ and using an identity.
\[ \int 2\tan^n(x) \,dx = 2 \int \tan^{n-2}(x)\tan^2(x) \,dx = 2 \int \tan^{n-2}(x)(\sec^2(x) - 1) \,dx \]
\[ = 2 \int \tan^{n-2}(x)\sec^2(x) \,dx - 2 \int \tan^{n-2}(x) \,dx \]
The second integral is the reduced part. The first integral can be solved by u-substitution, letting $u=\tan(x)$, $du=\sec^2(x)dx$.
\[ 2 \int \tan^{n-2}(x)\sec^2(x) \,dx = 2 \int u^{n-2} \,du = 2 \frac{u^{n-1}}{n-1} = 2\frac{\tan^{n-1}(x)}{n-1} \]
Combining the parts:
\[ \int 2\tan^n(x) \,dx = 2\frac{\tan^{n-1}(x)}{n-1} - \int 2\tan^{n-2}(x) \,dx \]
\textbf{Final Answer:} The correct formula is $\int 2\tan^n(x) \,dx = 2\frac{\tan^{n-1}(x)}{n-1} - \int 2\tan^{n-2}(x) \,dx, (n \neq 1)$.

\subsection*{Problem 14: Evaluate $\int xe^{9x} \,dx$}
\textbf{Setup:} Let $u = x$ and $dv = e^{9x} \,dx$.
\textbf{Steps:}
\begin{enumerate}
    \item $du = dx$.
    \item $v = \int e^{9x} \,dx = \frac{1}{9}e^{9x}$.
    \item Apply the formula:
    \begin{align*}
        \int xe^{9x} \,dx &= \frac{1}{9}xe^{9x} - \int \frac{1}{9}e^{9x} \,dx \\
        &= \frac{1}{9}xe^{9x} - \frac{1}{9}\left(\frac{1}{9}e^{9x}\right) + C \\
        &= \frac{1}{9}xe^{9x} - \frac{1}{81}e^{9x} + C
    \end{align*}
\end{enumerate}
\textbf{Final Answer:} $\frac{1}{9}xe^{9x} - \frac{1}{81}e^{9x} + C$

\subsection*{Problem 15: Evaluate $\int te^{8t} \,dt$}
\textbf{Setup:} Let $u = t$ and $dv = e^{8t} \,dt$.
\textbf{Steps:}
\begin{enumerate}
    \item $du = dt$.
    \item $v = \int e^{8t} \,dt = \frac{1}{8}e^{8t}$.
    \item Apply the formula:
    \begin{align*}
        \int te^{8t} \,dt &= \frac{1}{8}te^{8t} - \int \frac{1}{8}e^{8t} \,dt \\
        &= \frac{1}{8}te^{8t} - \frac{1}{8}\left(\frac{1}{8}e^{8t}\right) + C \\
        &= \frac{1}{8}te^{8t} - \frac{1}{64}e^{8t} + C
    \end{align*}
\end{enumerate}
\textbf{Final Answer:} $\frac{1}{8}te^{8t} - \frac{1}{64}e^{8t} + C$

\subsection*{Problem 16: Evaluate $\int (\pi - x)\cos(\pi x) \,dx$}
\textbf{Setup:} Let $u = \pi - x$ and $dv = \cos(\pi x) \,dx$.
\textbf{Steps:}
\begin{enumerate}
    \item Differentiate $u$: $du = -1 \,dx$.
    \item Integrate $dv$: $v = \int \cos(\pi x) \,dx = \frac{1}{\pi}\sin(\pi x)$.
    \item Apply the formula:
    \begin{align*}
        \int (\pi - x)\cos(\pi x) \,dx &= (\pi-x)\left(\frac{1}{\pi}\sin(\pi x)\right) - \int \left(\frac{1}{\pi}\sin(\pi x)\right) (-1 \,dx) \\
        &= \frac{\pi-x}{\pi}\sin(\pi x) + \frac{1}{\pi} \int \sin(\pi x) \,dx \\
        &= \frac{\pi-x}{\pi}\sin(\pi x) + \frac{1}{\pi} \left(-\frac{1}{\pi}\cos(\pi x)\right) + C \\
        &= \frac{\pi-x}{\pi}\sin(\pi x) - \frac{1}{\pi^2}\cos(\pi x) + C
    \end{align*}
\end{enumerate}
\textbf{Final Answer:} $\frac{\pi-x}{\pi}\sin(\pi x) - \frac{1}{\pi^2}\cos(\pi x) + C$

\part{In-Depth Analysis of Problems and Techniques}

\subsection{Problem Types and General Approach}
The homework problems can be categorized into several distinct types, each with a specific strategy.

\subsubsection{Type 1: Polynomial times Exponential/Trigonometric}
\begin{itemize}
    \item \textbf{Problems:} 1, 2, 14, 15, 16, and the core part of 8 and 10.
    \item \textbf{General Approach:} The key is to choose the polynomial term as $u$ and the exponential or trigonometric term as $dv$. The reason this works is that differentiating the polynomial term reduces its degree with each step, eventually becoming zero or a constant, which terminates the process and makes the final integral trivial. This is the primary application of the LIATE/LIPET mnemonic.
\end{itemize}

\subsubsection{Type 2: Logarithmic Function Involved}
\begin{itemize}
    \item \textbf{Problems:} 3, 4, 5, 11.
    \item \textbf{General Approach:} Always choose the logarithmic term as $u$. The derivative of $\ln(x)$ is $1/x$, an algebraic function. This transforms the integral from containing a transcendental function to one containing (usually) only algebraic functions, which are often much simpler to integrate.
\end{itemize}

\subsubsection{Type 3: "Boomerang" Integrals (Exponential times Trig)}
\begin{itemize}
    \item \textbf{Problems:} 6, 7, 9.
    \item \textbf{General Approach:} These are the most algorithmically complex. The strategy is to apply integration by parts twice, consistently choosing the same type of function for $u$ (e.g., always the trig function). After the second application, the original integral, which we can call $I$, will reappear on the right side of the equation. The final step is to solve the resulting algebraic equation for $I$.
\end{itemize}

\subsubsection{Type 4: Reduction Formulas}
\begin{itemize}
    \item \textbf{Problems:} 11, 12, 13.
    \item \textbf{General Approach:} The goal is to express an integral with a power $n$ in terms of an integral with a smaller power (like $n-1$ or $n-2$). This is done by applying integration by parts to the general form of the integral. For trigonometric reduction formulas (like Problem 13), it's common to first use a trigonometric identity to split the integrand before integrating.
\end{itemize}

\subsubsection{Type 5: Integrals Requiring Initial Simplification}
\begin{itemize}
    \item \textbf{Problems:} 5, 8, 10.
    \item \textbf{General Approach:} Before applying integration by parts, the integrand must be simplified. This can involve using a logarithm property (Problem 5), a trigonometric identity (Problem 8), or a full u-substitution to transform the entire integral (Problem 10). A crucial lesson is to always look for simplifications before diving into a complex technique.
\end{itemize}

\subsection{Key Algebraic and Calculus Manipulations}
\begin{itemize}
    \item \textbf{LIATE/LIPET Mnemonic:} The most important strategic tool. It's a guideline for choosing $u$ in the order: \textbf{L}ogarithmic, \textbf{I}nverse Trig, \textbf{A}lgebraic (polynomials), \textbf{T}rigonometric, \textbf{E}xponential. The function type that appears first in the list should be your choice for $u$. This was the key to setting up nearly every problem.
    \item \textbf{Trigonometric Identities:} In Problem 8, using $\sin(x)\cos(x) = \frac{1}{2}\sin(2x)$ was essential to create a solvable form. In Problem 10, using $\sin(2t) = 2\sin(t)\cos(t)$ was necessary to set up the u-substitution.
    \item \textbf{Logarithm Properties:} In Problem 5, simplifying $\ln(\sqrt{x})$ to $\frac{1}{2}\ln(x)$ made the problem much cleaner.
    \item \textbf{Solving for the Integral Algebraically:} The "boomerang" technique in Problems 6, 7, and 9 hinges on the algebraic step of isolating the integral $I$. For example, getting to $I = \text{stuff} - kI$ and correctly transforming it to $(1+k)I = \text{stuff}$.
    \item \textbf{Chain Rule Application (Derivatives and Integrals):} Finding $du$ and $v$ constantly requires the chain rule. For instance, in Problem 2, integrating $dv = \cos(9x)dx$ gives $v = \frac{1}{9}\sin(9x)$, not $\sin(9x)$. This is a frequent point of error. Similarly, in Problem 9, differentiating $u=7\sin(t-s)$ with respect to $s$ gives $du = -7\cos(t-s)ds$.
    \item \textbf{Careful Handling of Definite Integrals:} In Problems 8, 9, and 10, one must remember to evaluate the $[uv]$ term at the bounds and carry the bounds over to the new integral $\int v\,du$. In Problem 10, changing the bounds of integration during the initial u-substitution was a critical step.
\end{itemize}

\part{"Cheatsheet" and Tips for Success}

\section{Summary of Formulas}
\begin{itemize}
    \item \textbf{Indefinite Integral:} $\int u \,dv = uv - \int v \,du$
    \item \textbf{Definite Integral:} $\int_{a}^{b} u \,dv = [uv]_{a}^{b} - \int_{a}^{b} v \,du$
\end{itemize}

\section{Tricks and Shortcuts}
\begin{itemize}
    \item \textbf{LIATE Rule:} To choose $u$, follow this priority: Log, Inverse Trig, Algebraic, Trig, Exponential.
    \item \textbf{Simplify First:} Always look for algebraic simplifications, trig identities, or log rules to apply before you start.
    \item \textbf{The Boomerang:} If you see an exponential multiplied by a sine or cosine, expect to use integration by parts twice and then solve for the integral algebraically.
    \item \textbf{Tabular Method (DI Method):} For integrals like $\int x^3 e^x \,dx$, where you'd apply the technique multiple times, the tabular method is a powerful shortcut (See Part 7).
\end{itemize}

\section{Common Pitfalls and Mistakes}
\begin{itemize}
    \item \textbf{Forgetting the Minus Sign:} The formula is $uv \mathbf{-} \int v\,du$. It's easy to write a plus sign by mistake.
    \item \textbf{Sign Errors in $v$ or $du$:} The integral of $\sin(x)$ is $-\cos(x)$. Forgetting this negative sign is a very common error that cascades through the entire problem.
    \item \textbf{Forgetting $+C$:} Your final answer for an indefinite integral must always include the constant of integration.
    \item \textbf{Algebraic Errors with Boomerangs:} When solving for $I$, be careful with fractions and signs. It's easy to add $\frac{16}{9}I$ to the wrong side or make a mistake with the common denominator.
    \item \textbf{Definite Integral Errors:} Forgetting to evaluate both terms, $[uv]_a^b$ and $\int_a^b v\,du$. A common mistake is to only evaluate the final antiderivative at the bounds.
\end{itemize}

\section{How to Recognize Problem Types}
\begin{itemize}
    \item \textbf{Product of different function types?} (e.g., polynomial and exponential) $\rightarrow$ This is a standard Integration by Parts problem. Use LIATE.
    \item \textbf{A single function you can't integrate directly?} (e.g., $\ln(x)$ or $\arctan(x)$) $\rightarrow$ Try Integration by Parts with $dv = dx$.
    \item \textbf{$e^{ax}\sin(bx)$ or $e^{ax}\cos(bx)$?} $\rightarrow$ This is a Boomerang problem. Prepare for two rounds.
    \item \textbf{An integral with a power $n$?} $\rightarrow$ This might be a reduction formula problem.
\end{itemize}

\part{Conceptual Synthesis and The "Big Picture"}

\section{Thematic Connections}
The core theme of Integration by Parts is \textbf{transformation}. It's a method for taking an integral that is difficult or impossible to solve in its current form and transforming it into an equivalent expression containing a simpler integral. This theme of transforming problems is one of the most powerful ideas in mathematics.

We've seen this theme before:
\begin{itemize}
    \item In algebra, we use logarithms to transform multiplication problems into addition problems.
    \item In trigonometry, we use identities to transform complex expressions into simpler ones.
    \item With u-substitution, we transform an integral in terms of $x$ into a simpler one in terms of $u$.
\end{itemize}
Integration by Parts is simply another, more powerful, tool in our arsenal of mathematical transformations. It strategically uses the inverse relationship between differentiation and integration (specifically, the product rule) to achieve this transformation.

\section{Forward and Backward Links}
\begin{itemize}
    \item \textbf{Backward Link (Foundation):} Integration by Parts is a direct and logical consequence of the \textbf{Product Rule for Differentiation}. Without a solid understanding of how to differentiate a product of functions, the derivation and application of this integration technique would be meaningless. It is the next logical step in our study of integration after mastering basic antiderivatives and u-substitution, as it addresses a class of functions (products) that those earlier methods cannot handle.

    \item \textbf{Forward Link (Application):} The skills learned here are not just for solving textbook integrals. They are crucial for several advanced topics:
        \begin{itemize}
            \item \textbf{Differential Equations:} Many methods for solving first-order ordinary differential equations, particularly those involving integrating factors, rely heavily on this technique.
            \item \textbf{Fourier Series:} In engineering and physics, Fourier series are used to represent complex periodic signals (like sound waves) as a sum of simple sines and cosines. Calculating the coefficients for these series requires evaluating integrals of the form $\int f(x)\sin(nx) \,dx$, which is a classic application of Integration by Parts.
            \item \textbf{Probability and Statistics:} As shown in the next section, calculating the mean, variance, and other moments of continuous probability distributions often requires Integration by Parts.
        \end{itemize}
\end{itemize}

\part{Real-World Application and Modeling}
\section{Concrete Scenarios in Finance, Economics, and Statistics}
\begin{enumerate}
    \item \textbf{Finance (Derivative Pricing):} The valuation of financial derivatives, such as options, is a cornerstone of modern finance. The famous Black-Scholes model provides a formula for this. The derivation of this formula and its applications involve calculating the expected future value of an asset. This expectation is an integral of the form $\int (\text{payoff}) \times (\text{probability density function}) \,d(\text{price})$. The log-normal distribution is often used for asset prices, leading to integrands like $S \cdot e^{-(\ln S - \mu)^2 / (2\sigma^2)}$. Evaluating these integrals or related quantities in stochastic calculus often requires sophisticated applications of Integration by Parts (known in that context as Itô's Lemma, which is a stochastic version of the chain/product rule).

    \item \textbf{Statistics (Expected Value of a Distribution):} The expected value (or mean) of a continuous random variable is its probability-weighted average, found by the integral $E[X] = \int_{-\infty}^{\infty} x f(x) \,dx$, where $f(x)$ is the probability density function (PDF). For many important distributions, this requires Integration by Parts. For example, the Gamma distribution, used to model waiting times, has a PDF of the form $f(x) \propto x^{k-1}e^{-x/\theta}$. Finding its mean involves integrating a polynomial term times an exponential term, a classic application of our technique.

    \item \textbf{Economics (Present Value of a Continuous Income Stream):} Economists model the value of an asset by calculating the present value of the income it will generate. If an asset generates a continuous income stream $I(t)$ over a period of $T$ years, its present value (PV) is given by the integral $PV = \int_0^T I(t)e^{-rt} \,dt$, where $r$ is the continuous interest rate. If the income stream $I(t)$ is not constant (e.g., $I(t) = kt$, representing linear growth), the integral becomes $\int_0^T kt e^{-rt} \,dt$. This is an integral of a polynomial times an exponential, which must be solved using Integration by Parts.
\end{enumerate}

\section{Model Problem Setup}
Let's set up the model for the \textbf{Present Value of a Continuous Income Stream}.

\begin{itemize}
    \item \textbf{Scenario:} A startup company projects that its profit stream will grow linearly over the next 5 years. The profit at time $t$ (in years) is modeled by the function $I(t) = 100,000t$. The company wants to find the present value of this entire 5-year profit stream, assuming a continuous interest rate of 6\%.

    \item \textbf{Variables:}
        \begin{itemize}
            \item $t$ = time in years
            \item $I(t) = 100,000t$ = income stream function (in dollars)
            \item $r = 0.06$ = continuous interest rate
            \item $T = 5$ = total time period in years
            \item $PV$ = Present Value (the quantity we want to find)
        \end{itemize}

    \item \textbf{Mathematical Model:} The formula for the present value of a continuous stream is:
        \[ PV = \int_0^T I(t)e^{-rt} \,dt \]
    Substituting our specific functions and variables, we get the definite integral that needs to be solved:
        \[ PV = \int_0^5 (100,000t)e^{-0.06t} \,dt \]
    To solve this, we would use Integration by Parts with:
        \begin{itemize}
            \item $u = 100,000t$
            \item $dv = e^{-0.06t} \,dt$
        \end{itemize}
    Solving this integral would give the company the total value, in today's dollars, of its projected profits for the next five years.
\end{itemize}

\part{Common Variations and Untested Concepts}
The provided homework set was quite thorough, but it omitted a few key variations and one extremely useful computational shortcut.

\section{The Tabular Method (DI Method)}
This technique is a shortcut for repeated applications of Integration by Parts, typically used on integrals of the form $\int (\text{polynomial}) \cdot (\text{easily integrable function}) \,dx$.

\textbf{Explanation:}
\begin{enumerate}
    \item Create two columns: one for "D" (derivatives) and one for "I" (integrals).
    \item Place the polynomial part in the "D" column and the other part in the "I" column.
    \item Repeatedly differentiate the "D" column until you get zero.
    \item Repeatedly integrate the "I" column the same number of times.
    \item Multiply the terms diagonally, starting with the first term in the "D" column and the second in the "I" column.
    \item Alternate the signs of these products: +, -, +, -, ...
    \item The sum of these signed products is the answer.
\end{enumerate}

\textbf{Example (Not in Homework):} Evaluate $\int x^3 e^{2x} \,dx$.
\begin{center}
\begin{tabular}{c c c}
    \textbf{Sign} & \textbf{D} & \textbf{I} \\ \hline
    + & $x^3$ & $e^{2x}$ \\
    - & $3x^2$ & $\frac{1}{2}e^{2x}$ \\
    + & $6x$ & $\frac{1}{4}e^{2x}$ \\
    - & $6$ & $\frac{1}{8}e^{2x}$ \\
    + & $0$ & $\frac{1}{16}e^{2x}$
\end{tabular}
\end{center}
\textbf{Solution:}
\begin{align*}
    \int x^3 e^{2x} \,dx &= +x^3\left(\frac{1}{2}e^{2x}\right) - 3x^2\left(\frac{1}{4}e^{2x}\right) + 6x\left(\frac{1}{8}e^{2x}\right) - 6\left(\frac{1}{16}e^{2x}\right) + C \\
    &= \frac{1}{2}x^3e^{2x} - \frac{3}{4}x^2e^{2x} + \frac{3}{4}xe^{2x} - \frac{3}{8}e^{2x} + C
\end{align*}

\section{Integrals of Inverse Trigonometric Functions}
Your homework did not include integrating a lone inverse trigonometric function.

\textbf{Explanation:} The strategy is similar to integrating $\ln(x)$; we let $u$ be the inverse trig function and $dv = dx$.

\textbf{Example (Not in Homework):} Evaluate $\int \arctan(x) \,dx$.
\begin{itemize}
    \item Let $u = \arctan(x)$ and $dv = dx$.
    \item Then $du = \frac{1}{1+x^2} \,dx$ and $v = x$.
\end{itemize}
Applying the formula:
\[ \int \arctan(x) \,dx = x\arctan(x) - \int x \cdot \frac{1}{1+x^2} \,dx \]
The new integral can be solved with a simple u-substitution. Let $w = 1+x^2$, so $dw = 2x\,dx$, which means $x\,dx = \frac{1}{2}dw$.
\begin{align*}
    \int \frac{x}{1+x^2} \,dx &= \int \frac{1}{w} \cdot \frac{1}{2}dw = \frac{1}{2}\ln|w| = \frac{1}{2}\ln(1+x^2)
\end{align*}
Combining the results:
\[ \int \arctan(x) \,dx = x\arctan(x) - \frac{1}{2}\ln(1+x^2) + C \]

\part{Advanced Diagnostic Testing: "Find the Flaw"}
Below are five problems with complete solutions. However, each solution contains one subtle but critical error. Your task is to find the flaw, explain why it's wrong, and provide the correct step and final answer.

\subsection*{Problem 1}
Evaluate $\int x^2 \sin(x) \,dx$.

\textbf{Flawed Solution:}
Use tabular integration.
\begin{center}
\begin{tabular}{c c c}
    \textbf{Sign} & \textbf{D} & \textbf{I} \\ \hline
    + & $x^2$ & $\sin(x)$ \\
    - & $2x$ & $-\cos(x)$ \\
    + & $2$ & $\sin(x)$ \\
    - & $0$ & $\cos(x)$
\end{tabular}
\end{center}
Multiplying diagonally:
\begin{align*}
    \int x^2 \sin(x) \,dx &= (x^2)(-\cos(x)) - (2x)(\sin(x)) + (2)(\cos(x)) + C \\
    &= -x^2\cos(x) - 2x\sin(x) + 2\cos(x) + C
\end{align*}

\subsection*{Problem 2}
Evaluate $\int_{1}^{e} \ln(x^2) \,dx$.

\textbf{Flawed Solution:}
First, simplify: $\int_{1}^{e} 2\ln(x) \,dx = 2 \int_{1}^{e} \ln(x) \,dx$.
Let $u = \ln(x)$ and $dv = dx$. Then $du = \frac{1}{x}dx$ and $v=x$.
\begin{align*}
    2 \int_{1}^{e} \ln(x) \,dx &= 2 \left( [x\ln(x)]_1^e - \int_1^e x \cdot \frac{1}{x} dx \right) \\
    &= 2 \left( [x\ln(x)]_1^e - \int_1^e 1 \,dx \right) \\
    &= 2 \left( [x\ln(x) - x]_1^e \right) \\
    &= 2 \left( (e\ln(e) - e) - (1\ln(1) - 1) \right) \\
    &= 2 \left( (e \cdot 1 - e) - (1 \cdot 0 - 1) \right) \\
    &= 2 \left( (0) - (-1) \right) = 2(1) = 2
\end{align*}
\textbf{Wait, the flaw is in my provided solution for Problem 2. Let me fix it.}
The original solution is: $2((e\ln(e)-e) - (1\ln(1)-1)) = 2((e-e)-(0-1)) = 2(0 - (-1)) = 2$. This is correct. Let me create a flawed one.

\textbf{Corrected Flawed Solution for Problem 2:}
\begin{align*}
    2 \int_{1}^{e} \ln(x) \,dx &= 2 \left( [x\ln(x)]_1^e - \int_1^e x \cdot \frac{1}{x} dx \right) \\
    &= 2 \left( (e\ln(e) - 1\ln(1)) - \int_1^e 1 \,dx \right) \\
    &= 2 \left( (e \cdot 1 - 0) - [x]_1^e \right) \\
    &= 2 \left( e - (e - 1) \right) \\
    &= 2(e - e + 1) = 2
\end{align*}
Ah, this flaw is too obvious and still gets the right answer by chance. Let me create a better flaw.

\textbf{Final Flawed Solution for Problem 2:}
Evaluate $\int_{0}^{\pi/4} x \sec^2(x) \,dx$.

\textbf{Flawed Solution:}
Let $u = x$ and $dv = \sec^2(x)dx$. Then $du=dx$ and $v=\tan(x)$.
\begin{align*}
    \int_{0}^{\pi/4} x \sec^2(x) \,dx &= [x\tan(x)]_{0}^{\pi/4} - \int_{0}^{\pi/4} \tan(x) \,dx \\
    &= \left(\frac{\pi}{4}\tan\left(\frac{\pi}{4}\right) - 0 \cdot \tan(0)\right) - [-\ln|\cos(x)|]_{0}^{\pi/4} \\
    &= \left(\frac{\pi}{4} \cdot 1 - 0\right) - \left(-\ln\left|\cos\left(\frac{\pi}{4}\right)\right| - (-\ln|\cos(0)|)\right) \\
    &= \frac{\pi}{4} - \left(-\ln\left(\frac{\sqrt{2}}{2}\right) - (-\ln(1))\right) \\
    &= \frac{\pi}{4} + \ln\left(\frac{\sqrt{2}}{2}\right) + \ln(1) \\
    &= \frac{\pi}{4} + \ln\left(\frac{\sqrt{2}}{2}\right)
\end{align*}

\subsection*{Problem 3}
Evaluate $\int e^x \cos(x) \,dx$. Let $I = \int e^x \cos(x) \,dx$.

\textbf{Flawed Solution:}
First pass: $u = \cos(x)$, $dv=e^x dx$. So $du = -\sin(x)dx$, $v=e^x$.
\[ I = e^x\cos(x) - \int e^x(-\sin(x))dx = e^x\cos(x) + \int e^x\sin(x)dx \]
Second pass: $u = e^x$, $dv=\sin(x)dx$. So $du = e^x dx$, $v = -\cos(x)$.
\begin{align*}
    \int e^x\sin(x)dx &= (e^x)(-\cos(x)) - \int (-\cos(x))(e^x dx) \\
    &= -e^x\cos(x) + \int e^x\cos(x)dx = -e^x\cos(x) + I
\end{align*}
Substitute back:
\[ I = e^x\cos(x) + [-e^x\cos(x) + I] \]
\[ I = I \]
This leads to a trivial identity, something is wrong.

\subsection*{Problem 4}
Evaluate $\int \frac{\ln(x)}{\sqrt{x}} \,dx$.

\textbf{Flawed Solution:}
Let $u=\ln(x)$ and $dv = x^{-1/2}dx$. Then $du = \frac{1}{x}dx$ and $v = \frac{1}{2}x^{1/2}$.
\begin{align*}
    \int \frac{\ln(x)}{\sqrt{x}} \,dx &= (\ln x)\left(\frac{1}{2}\sqrt{x}\right) - \int \left(\frac{1}{2}\sqrt{x}\right)\left(\frac{1}{x}dx\right) \\
    &= \frac{1}{2}\sqrt{x}\ln(x) - \frac{1}{2}\int x^{-1/2} dx \\
    &= \frac{1}{2}\sqrt{x}\ln(x) - \frac{1}{2}\left(\frac{1}{2}x^{1/2}\right) + C \\
    &= \frac{1}{2}\sqrt{x}\ln(x) - \frac{1}{4}\sqrt{x} + C
\end{align*}

\subsection*{Problem 5}
Prove the reduction formula for $\int \cos^n(x) \,dx$.

\textbf{Flawed Solution:}
Let $u = \cos^{n-1}(x)$ and $dv = \cos(x)dx$. Then $v = \sin(x)$.
For $du$, use chain rule: $du = (n-1)\cos^{n-2}(x) \cdot (\sin(x)) dx$.
\begin{align*}
    \int \cos^n(x) dx &= \cos^{n-1}(x)\sin(x) - \int \sin(x) \left((n-1)\cos^{n-2}(x)\sin(x)\right) dx \\
    &= \cos^{n-1}(x)\sin(x) - (n-1)\int \cos^{n-2}(x)\sin^2(x) dx \\
    &= \cos^{n-1}(x)\sin(x) - (n-1)\int \cos^{n-2}(x)(1-\cos^2(x)) dx \\
    &= \cos^{n-1}(x)\sin(x) - (n-1)\left[ \int \cos^{n-2}(x)dx - \int \cos^n(x)dx \right] \\
    I &= \cos^{n-1}(x)\sin(x) - (n-1)\int \cos^{n-2}(x)dx + (n-1)I \\
    I - (n-1)I &= \cos^{n-1}(x)\sin(x) - (n-1)\int \cos^{n-2}(x)dx \\
    (2-n)I &= \cos^{n-1}(x)\sin(x) - (n-1)\int \cos^{n-2}(x)dx \\
    I &= \frac{1}{2-n}\cos^{n-1}(x)\sin(x) - \frac{n-1}{2-n}\int \cos^{n-2}(x)dx
\end{align*}

\end{document}