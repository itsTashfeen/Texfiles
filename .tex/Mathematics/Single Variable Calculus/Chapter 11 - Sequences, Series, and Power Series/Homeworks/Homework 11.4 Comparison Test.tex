\documentclass{article}
\usepackage{amsmath}
\usepackage{amssymb}
\usepackage[margin=1in]{geometry}

\title{Homework 11.4 Comparison Tests for Series}
\author{Tashfeen Omran}
\date{November 2025}

\begin{document}

\maketitle

\section{Comprehensive Introduction, Context, and Prerequisites}

\subsection{Core Concepts}
The central idea behind the Comparison Tests is simple yet powerful: to determine if a complicated-looking infinite series converges or diverges, we can compare it to a simpler series whose behavior we already know. This allows us to leverage our knowledge of basic series types (like p-series and geometric series) to analyze more complex ones. We will focus on two main tests.

\subsubsection{The Direct Comparison Test (DCT)}
The Direct Comparison Test is the more intuitive of the two. It works by establishing a direct inequality between the terms of the series in question and a known series.

Let $\sum a_n$ and $\sum b_n$ be series with positive terms ($a_n > 0$, $b_n > 0$).
\begin{enumerate}
    \item If $a_n \le b_n$ for all $n$ (or for all $n$ past a certain point), and the "bigger" series $\sum b_n$ \textbf{converges}, then the "smaller" series $\sum a_n$ must also \textbf{converge}.
    \item If $a_n \ge b_n$ for all $n$ (or for all $n$ past a certain point), and the "smaller" series $\sum b_n$ \textbf{diverges}, then the "bigger" series $\sum a_n$ must also \textbf{diverge}.
\end{enumerate}

\subsubsection{The Limit Comparison Test (LCT)}
The Limit Comparison Test is often more practical than the DCT because finding a strict inequality can be tricky. The LCT instead compares the long-term behavior of the terms of two series.

Let $\sum a_n$ and $\sum b_n$ be series with positive terms. We compute the limit of the ratio of their terms:
\[ L = \lim_{n \to \infty} \frac{a_n}{b_n} \]
\begin{enumerate}
    \item If $L$ is a finite, positive number ($0 < L < \infty$), then both series share the same fate: either they \textbf{both converge} or they \textbf{both diverge}.
    \item If $L = 0$ and $\sum b_n$ converges, then $\sum a_n$ also converges.
    \item If $L = \infty$ and $\sum b_n$ diverges, then $\sum a_n$ also diverges.
\end{enumerate}
The first case ($0 < L < \infty$) is the most common and powerful application of this test.

\subsection{Intuition and Derivation}
\textbf{Direct Comparison Test:} Imagine two people, A and B, adding numbers to a pile. The terms $a_n$ and $b_n$ are the numbers they add at each step. If person A's numbers are always smaller than person B's ($a_n \le b_n$), and person B's final pile has a finite size ($\sum b_n$ converges), it's logical that person A's pile must also be finite ($\sum a_n$ converges). Conversely, if person B's numbers are smaller ($b_n \le a_n$), and their pile grows infinitely large ($\sum b_n$ diverges), then person A's larger pile must also be infinite ($\sum a_n$ diverges).

\textbf{Limit Comparison Test:} The "why" behind the LCT is that if $\lim_{n \to \infty} \frac{a_n}{b_n} = L$, it means that for very large $n$, the terms are approximately proportional: $a_n \approx L \cdot b_n$. Since $L$ is just a finite, non-zero constant, the series $\sum a_n$ behaves just like the series $\sum (L \cdot b_n) = L \sum b_n$. We know that multiplying a series by a constant doesn't change its convergence or divergence. Therefore, $\sum a_n$ and $\sum b_n$ must do the same thing.

\subsection{Historical Context and Motivation}
The study of infinite series has captivated mathematicians for centuries, dating back to Zeno's paradoxes in ancient Greece. However, a rigorous framework for determining their convergence or divergence only emerged with the development of calculus in the 17th century. Mathematicians like Newton, Leibniz, and the Bernoulli family made great strides, but it was in the 19th century, with the work of Cauchy and others, that the theory was placed on a firm logical footing.

The core motivation was that while it is often impossible to find the \textit{exact sum} of a series, the more fundamental question is whether a finite sum even exists. For applications in physics (like calculating gravitational forces), engineering, and later, finance, knowing that a series converges is essential. The comparison tests were a brilliant development because they allowed mathematicians to determine convergence without the burden of calculating the actual sum, simply by comparing a difficult series to a well-understood one.

\subsection{Key Formulas}
The essential formulas are the statements of the tests themselves:
\begin{itemize}
    \item \textbf{Direct Comparison Test (DCT):} Given $a_n, b_n > 0$.
    \begin{itemize}
        \item If $a_n \le b_n$ and $\sum b_n$ converges $\implies \sum a_n$ converges.
        \item If $a_n \ge b_n$ and $\sum b_n$ diverges $\implies \sum a_n$ diverges.
    \end{itemize}
    \item \textbf{Limit Comparison Test (LCT):} Given $a_n, b_n > 0$, compute $L = \lim_{n \to \infty} \frac{a_n}{b_n}$.
    \begin{itemize}
        \item If $0 < L < \infty \implies \sum a_n$ and $\sum b_n$ both converge or both diverge.
    \end{itemize}
\end{itemize}
To use these tests, you must have a library of known series to compare against:
\begin{itemize}
    \item \textbf{p-Series:} The series $\sum_{n=1}^\infty \frac{1}{n^p}$ converges if $p > 1$ and diverges if $p \le 1$.
    \item \textbf{Geometric Series:} The series $\sum_{n=1}^\infty ar^{n-1}$ converges if $|r| < 1$ and diverges if $|r| \ge 1$.
\end{itemize}

\subsection{Prerequisites}
\begin{itemize}
    \item \textbf{Sequences and Series Fundamentals:} Firm grasp of the definitions of sequences and series, and sigma notation ($\sum$).
    \item \textbf{Convergence/Divergence Tests:} Knowledge of the Test for Divergence, the Integral Test, p-series, and geometric series. These provide the "known" series for comparison.
    \item \textbf{Limits at Infinity:} Proficiency in evaluating limits of functions as the variable approaches infinity, including handling rational functions, exponential functions, and applying L'Hôpital's Rule when necessary.
    \item \textbf{Algebraic Manipulation:} Strong skills in working with inequalities, simplifying complex fractions, and manipulating exponents.
\end{itemize}

\section{Detailed Homework Solutions}

\subsection{Problem 1}
\textbf{Question:} Suppose $\sum a_n$ and $\sum b_n$ are series with positive terms and $\sum b_n$ is known to be convergent.

\textbf{(a) If $a_n > b_n$ for all $n$, what can you say about $\sum a_n$? Why?}
\textbf{Solution:} We cannot say anything. The Direct Comparison Test requires the series in question to be \textit{smaller} than a known convergent series to prove convergence. Knowing our series is \textit{larger} than a convergent one is inconclusive. For example, $\sum \frac{1}{n^2}$ converges, but the larger series $\sum \frac{1}{n}$ diverges, while another larger series $\sum \frac{1.1}{n^2}$ converges.
\textbf{Correct Answer:} We cannot say anything about $\sum a_n$.

\textbf{(b) If $a_n < b_n$ for all $n$, what can you say about $\sum a_n$? Why?}
\textbf{Solution:} The series $\sum a_n$ converges. This is a direct application of the Direct Comparison Test. Since the terms of $\sum a_n$ are positive and are less than the corresponding terms of a known convergent series $\sum b_n$, $\sum a_n$ must also converge.
\textbf{Correct Answer:} $\sum a_n$ converges by the Comparison Test.

\subsection{Problem 2}
\textbf{Question:} Suppose $\sum a_n$ and $\sum b_n$ are series with positive terms and $\sum b_n$ is known to be divergent.

\textbf{(a) If $a_n > b_n$ for all $n$, what can you say about $\sum a_n$? Why?}
\textbf{Solution:} The series $\sum a_n$ diverges. This is a direct application of the Direct Comparison Test. Since the terms of $\sum a_n$ are positive and are greater than the corresponding terms of a known divergent series $\sum b_n$, $\sum a_n$ must also diverge.
\textbf{Correct Answer:} $\sum a_n$ diverges by the Comparison Test.

\textbf{(b) If $a_n < b_n$ for all $n$, what can you say about $\sum a_n$? Why?}
\textbf{Solution:} We cannot say anything. The Direct Comparison Test requires the series in question to be \textit{larger} than a known divergent series to prove divergence. Knowing our series is \textit{smaller} than a divergent one is inconclusive. For example, the series $\sum \frac{1}{n}$ diverges, but the smaller series $\sum \frac{1}{n^2}$ converges, while another smaller series $\sum \frac{0.5}{n}$ diverges.
\textbf{Correct Answer:} We cannot say anything about $\sum a_n$.

\subsection{Problem 3}
\textbf{(a) Identify the error(s) in the proposed proof for the convergence of $\sum_{n=2}^{\infty} \frac{n}{n^3+5}$.}
\textbf{Proof steps:}
1. We have $\frac{n}{n^3+5} < \frac{n}{n^3} = \frac{1}{n^2}$ for all $n \le 2$.
2. The summation $\sum \frac{1}{n^2}$ converges because it is a p-series with $p=2>1$.
3. So $\sum \frac{n}{n^3+5}$ converges by part (i) of the direct comparison test.

\textbf{Solution:}
Step 1 contains a critical error. The inequality $\frac{n}{n^3+5} < \frac{1}{n^2}$ is correct, because the denominator $n^3+5$ is larger than $n^3$. However, for the comparison test to work, this inequality must hold for all $n$ from the start of the summation onwards, i.e., for $n \ge 2$. The proof incorrectly states the condition as $n \le 2$. Step 2 and Step 3 are logically correct, but they rely on the faulty condition in Step 1.
\textbf{Correct Answer:} The first sentence should say $n \ge 2$ instead of $n \le 2$.

\textbf{(b) Identify the error(s) in the proposed proof for the convergence of $\sum_{n=2}^{\infty} \frac{n}{n^3-5}$.}
\textbf{Proof steps:}
1. Use the limit comparison test with $a_n = \frac{n}{n^3-5}$ and $b_n = \frac{n^2}{1}$.
2. We take $\lim_{n \to \infty} \frac{a_n}{b_n} = \lim_{n \to \infty} \frac{n/(n^3-5)}{n^2/1} = \lim_{n \to \infty} \frac{n}{n^2(n^3-5)} = \dots = -1 < 0$.
3. Since $\sum \frac{1}{n^2}$ is a convergent p-series, the series $\sum \frac{n}{n^3-5}$ converges.

\textbf{Solution:} This proof has multiple errors.
\begin{itemize}
    \item \textbf{Error 1 (Step 1 \& 2):} The choice of $b_n$ is completely wrong. To analyze $a_n = \frac{n}{n^3-5}$, the dominant terms are $n/n^3 = 1/n^2$. So the correct choice for comparison is $b_n = \frac{1}{n^2}$. The proof uses $b_n = n^2$, which makes no sense.
    \item \textbf{Error 2 (Step 2):} The limit calculation is incorrect. Even with their flawed $b_n$, the limit would be $\lim_{n \to \infty} \frac{n}{n^2(n^3-5)} = 0$, not $-1$. If the correct $b_n = 1/n^2$ was used, the limit would be:
    \[ \lim_{n \to \infty} \frac{n/(n^3-5)}{1/n^2} = \lim_{n \to \infty} \frac{n^3}{n^3-5} = 1 \]
    The condition for the LCT is that the limit must be positive ($L>0$), not negative. The result "$-1<0$" is not a valid conclusion for the LCT. The correct result is $1>0$.
\end{itemize}
\textbf{Correct Answers:}
\begin{itemize}
    \item The first sentence should say $b_n = \frac{1}{n^2}$ instead of $b_n = \frac{n^2}{1}$.
    \item The second sentence should conclude with the statement $= 1 > 0$ instead of $= -1 < 0$.
\end{itemize}

\subsection{Problem 4}
\textbf{Question:} Determine whether the series $\sum_{n=1}^{\infty} \frac{1}{n^6+8}$ converges or diverges.
\textbf{Solution:} We can use the Direct Comparison Test. We want to compare to a known p-series. Let $a_n = \frac{1}{n^6+8}$. Let's choose $b_n = \frac{1}{n^6}$.
1. \textbf{Comparison:} For all $n \ge 1$, we have $n^6+8 > n^6$. Taking the reciprocal reverses the inequality: $\frac{1}{n^6+8} < \frac{1}{n^6}$.
2. \textbf{Known Series:} The series $\sum_{n=1}^{\infty} b_n = \sum_{n=1}^{\infty} \frac{1}{n^6}$ is a p-series with $p=6$. Since $p > 1$, this series converges.
3. \textbf{Conclusion:} Since our series $a_n$ is smaller than a convergent series $b_n$, by the Direct Comparison Test, our series also converges.
\textbf{Final Answer:} converges.

\subsection{Problem 5}
\textbf{Question:} Determine whether the series $\sum_{n=1}^{\infty} \frac{n+4}{n\sqrt{n}}$ converges or diverges.
\textbf{Solution:} Let $a_n = \frac{n+4}{n\sqrt{n}} = \frac{n+4}{n^{3/2}}$. This looks like a p-series. Let's use the Limit Comparison Test. To choose our comparison series $b_n$, we look at the dominant terms in $a_n$: $\frac{n}{n^{3/2}} = \frac{1}{n^{1/2}}$. So, let's choose $b_n = \frac{1}{n^{1/2}}$.
1. \textbf{Limit Calculation:}
\[ L = \lim_{n \to \infty} \frac{a_n}{b_n} = \lim_{n \to \infty} \frac{(n+4)/n^{3/2}}{1/n^{1/2}} = \lim_{n \to \infty} \frac{n+4}{n^{3/2}} \cdot \frac{n^{1/2}}{1} = \lim_{n \to \infty} \frac{n+4}{n} = \lim_{n \to \infty} \left(1 + \frac{4}{n}\right) = 1 \]
2. \textbf{Known Series:} The series $\sum b_n = \sum \frac{1}{n^{1/2}}$ is a p-series with $p=1/2$. Since $p \le 1$, this series diverges.
3. \textbf{Conclusion:} Since our limit is $L=1$ (which is finite and positive), and our comparison series $\sum b_n$ diverges, the original series $\sum a_n$ must also diverge by the Limit Comparison Test.
\textbf{Final Answer:} diverges.

\subsection{Problem 6}
\textbf{Question:} Determine whether the series $\sum_{n=1}^{\infty} \frac{n+9}{n\sqrt{n}}$ converges or diverges.
\textbf{Solution:} This is very similar to Problem 5. We can use either Direct or Limit Comparison. Let's demonstrate with Direct Comparison this time.
Let $a_n = \frac{n+9}{n\sqrt{n}}$. We want to show it's larger than a divergent series. Let's compare with $b_n = \frac{n}{n\sqrt{n}} = \frac{1}{\sqrt{n}}$.
1. \textbf{Comparison:} For all $n \ge 1$, we have $n+9 > n$. Since the denominator is positive, we can say $\frac{n+9}{n\sqrt{n}} > \frac{n}{n\sqrt{n}} = \frac{1}{\sqrt{n}}$.
2. \textbf{Known Series:} The series $\sum b_n = \sum \frac{1}{\sqrt{n}}$ is a p-series with $p=1/2$. Since $p \le 1$, it diverges.
3. \textbf{Conclusion:} Since our series $a_n$ is larger than a divergent series $b_n$, by the Direct Comparison Test, our series also diverges.
\textbf{Correct Answer:} The series diverges by the Direct Comparison Test. Each term is greater than that of a divergent $p$-series.

\subsection{Problem 7}
\textbf{Question:} Determine whether the series $\sum_{n=1}^{\infty} \frac{9^n}{3+10^n}$ converges or diverges.
\textbf{Solution:} This series has exponential terms, suggesting a comparison to a geometric series. Let's use the Limit Comparison Test.
Let $a_n = \frac{9^n}{3+10^n}$. The dominant terms are $\frac{9^n}{10^n}$. So let's choose $b_n = \frac{9^n}{10^n} = \left(\frac{9}{10}\right)^n$.
1. \textbf{Limit Calculation:}
\[ L = \lim_{n \to \infty} \frac{a_n}{b_n} = \lim_{n \to \infty} \frac{9^n/(3+10^n)}{(9/10)^n} = \lim_{n \to \infty} \frac{9^n}{3+10^n} \cdot \frac{10^n}{9^n} = \lim_{n \to \infty} \frac{10^n}{3+10^n} \]
To evaluate this limit, we divide the numerator and denominator by $10^n$:
\[ L = \lim_{n \to \infty} \frac{10^n/10^n}{(3/10^n) + (10^n/10^n)} = \lim_{n \to \infty} \frac{1}{3(1/10)^n + 1} = \frac{1}{0+1} = 1 \]
2. \textbf{Known Series:} The series $\sum b_n = \sum (\frac{9}{10})^n$ is a geometric series with ratio $r = 9/10$. Since $|r| < 1$, it converges.
3. \textbf{Conclusion:} Since our limit is $L=1$ (finite and positive), and our comparison series $\sum b_n$ converges, the original series $\sum a_n$ must also converge by the Limit Comparison Test.
\textbf{Final Answer:} converges.

\subsection{Problem 8}
\textbf{Question:} Determine whether the series $\sum_{n=2}^{\infty} \frac{1}{\ln(n)}$ converges or diverges.
\textbf{Solution:} We know that the natural logarithm function grows more slowly than any positive power of $n$. Specifically, we know the inequality $\ln(n) < n$ for all $n \ge 1$. Let's use the Direct Comparison Test.
Let $a_n = \frac{1}{\ln(n)}$. Let's compare with $b_n = \frac{1}{n}$.
1. \textbf{Comparison:} For $n \ge 2$, we have $\ln(n) < n$. Taking the reciprocal reverses the inequality: $\frac{1}{\ln(n)} > \frac{1}{n}$.
2. \textbf{Known Series:} The series $\sum b_n = \sum_{n=2}^{\infty} \frac{1}{n}$ is the harmonic series (a p-series with $p=1$). It diverges.
3. \textbf{Conclusion:} Since our series $a_n$ is larger than a divergent series $b_n$, by the Direct Comparison Test, our series also diverges.
\textbf{Final Answer:} diverges.

\subsection{Problem 9}
\textbf{Question:} Determine whether the series $\sum_{k=1}^{\infty} \frac{\sqrt[3]{k}}{\sqrt{k^3+9k+5}}$ converges or diverges.
\textbf{Solution:} Let $a_k = \frac{k^{1/3}}{\sqrt{k^3+9k+5}}$. Let's use the Limit Comparison Test. The dominant term in the numerator is $k^{1/3}$. The dominant term in the denominator is $\sqrt{k^3} = k^{3/2}$.
Our comparison term $b_k$ should be the ratio of these dominant terms: $b_k = \frac{k^{1/3}}{k^{3/2}} = k^{1/3 - 3/2} = k^{2/6 - 9/6} = k^{-7/6} = \frac{1}{k^{7/6}}$.
1. \textbf{Limit Calculation:}
\[ L = \lim_{k \to \infty} \frac{a_k}{b_k} = \lim_{k \to \infty} \frac{k^{1/3}/\sqrt{k^3+9k+5}}{1/k^{7/6}} = \lim_{k \to \infty} \frac{k^{1/3} \cdot k^{7/6}}{\sqrt{k^3+9k+5}} = \lim_{k \to \infty} \frac{k^{9/6}}{\sqrt{k^3+9k+5}} = \lim_{k \to \infty} \frac{k^{3/2}}{\sqrt{k^3+9k+5}} \]
Divide numerator and denominator by $k^{3/2} = \sqrt{k^3}$:
\[ L = \lim_{k \to \infty} \frac{k^{3/2}/k^{3/2}}{\sqrt{k^3+9k+5}/\sqrt{k^3}} = \lim_{k \to \infty} \frac{1}{\sqrt{1 + 9/k^2 + 5/k^3}} = \frac{1}{\sqrt{1+0+0}} = 1 \]
2. \textbf{Known Series:} The series $\sum b_k = \sum \frac{1}{k^{7/6}}$ is a p-series with $p=7/6$. Since $p > 1$, it converges.
3. \textbf{Conclusion:} Since our limit is $L=1$ (finite and positive), and our comparison series $\sum b_k$ converges, the original series $\sum a_k$ must also converge by the Limit Comparison Test.
\textbf{Final Answer:} converges.

\subsection{Problem 10}
\textbf{Question:} Determine whether the series $\sum_{n=1}^{\infty} \frac{1+\sin(n)}{e^n}$ converges or diverges.
\textbf{Solution:} The presence of $\sin(n)$ suggests using the Direct Comparison Test by bounding the term.
Let $a_n = \frac{1+\sin(n)}{e^n}$.
1. \textbf{Comparison:} We know that for all $n$, $-1 \le \sin(n) \le 1$.
Adding 1 to all parts gives $0 \le 1+\sin(n) \le 2$.
Since $e^n$ is always positive, we can divide by it without changing the inequalities:
\[ 0 \le \frac{1+\sin(n)}{e^n} \le \frac{2}{e^n} \]
So, we have $a_n \le \frac{2}{e^n}$. Let's choose $b_n = \frac{2}{e^n} = 2\left(\frac{1}{e}\right)^n$.
2. \textbf{Known Series:} The series $\sum b_n = \sum_{n=1}^{\infty} 2\left(\frac{1}{e}\right)^n$ is a constant multiple of a geometric series with ratio $r=1/e$. Since $e \approx 2.718$, we have $|r| = 1/e < 1$. Therefore, $\sum b_n$ converges.
3. \textbf{Conclusion:} Since our series $a_n$ is non-negative and is smaller than a convergent series $b_n$, by the Direct Comparison Test, our series also converges.
\textbf{Final Answer:} converges.

\subsection{Problem 11}
\textbf{Question:} Determine whether the series $\sum_{n=1}^{\infty} \frac{1}{n^n}$ converges or diverges.
\textbf{Solution:} Let's use the Direct Comparison Test. Let $a_n = \frac{1}{n^n}$. We need to find a simpler, larger series that converges. A p-series is a good candidate.
1. \textbf{Comparison:} Let's compare $n^n$ to $n^2$. For $n \ge 2$, we have $n^n \ge n^2$ (e.g., $2^2=2^2$, $3^3 > 3^2$, $4^4 > 4^2$, etc.). Taking the reciprocal reverses the inequality:
\[ \frac{1}{n^n} \le \frac{1}{n^2} \quad \text{for } n \ge 2 \]
Let's choose $b_n = \frac{1}{n^2}$.
2. \textbf{Known Series:} The series $\sum_{n=1}^{\infty} b_n = \sum_{n=1}^{\infty} \frac{1}{n^2}$ is a p-series with $p=2$. Since $p>1$, this series converges.
3. \textbf{Conclusion:} Since our series $a_n$ is smaller than a convergent series $b_n$ (for $n \ge 2$), by the Direct Comparison Test, our series also converges. (The first term doesn't affect convergence).
\textbf{Final Answer:} converges.

\subsection{Problem 12}
\textbf{Question:} Determine whether the series $\sum_{n=2}^{\infty} \frac{n+6}{(n+7)^5}$ converges or diverges.
\textbf{Solution:} Let $a_n = \frac{n+6}{(n+7)^5}$. Use the Limit Comparison Test. The dominant term in the numerator is $n$. The dominant term in the denominator is $n^5$. So, we compare with $b_n = \frac{n}{n^5} = \frac{1}{n^4}$.
1. \textbf{Limit Calculation:}
\[ L = \lim_{n \to \infty} \frac{a_n}{b_n} = \lim_{n \to \infty} \frac{(n+6)/(n+7)^5}{1/n^4} = \lim_{n \to \infty} \frac{n^4(n+6)}{(n+7)^5} \]
The degree of the numerator is 5. The degree of the denominator is 5. The limit is the ratio of the leading coefficients, which is $1/1 = 1$.
\[ L = \lim_{n \to \infty} \frac{n^5+6n^4}{n^5 + \dots} = 1 \]
2. \textbf{Known Series:} The comparison series $\sum b_n = \sum \frac{1}{n^4}$ is a p-series with $p=4$. Since $p>1$, it converges.
3. \textbf{Conclusion:} Since the limit $L=1$ is finite and positive, and $\sum b_n$ converges, the original series $\sum a_n$ also converges by the Limit Comparison Test.
\textbf{Correct Answer:} The series converges by the Limit Comparison Test with a convergent $p$-series.

\subsection{Problem 13}
\textbf{Question:} Determine whether the series $\sum_{n=1}^{\infty} \left(1+\frac{3}{n}\right)^3 e^{-n}$ converges or diverges.
\textbf{Solution:} Let $a_n = \left(1+\frac{3}{n}\right)^3 e^{-n}$. As $n \to \infty$, the term $\left(1+\frac{3}{n}\right)^3$ approaches $(1+0)^3 = 1$. This suggests the overall term behaves like $e^{-n}$. Let's use the Limit Comparison Test with $b_n = e^{-n} = \left(\frac{1}{e}\right)^n$.
1. \textbf{Limit Calculation:}
\[ L = \lim_{n \to \infty} \frac{a_n}{b_n} = \lim_{n \to \infty} \frac{(1+3/n)^3 e^{-n}}{e^{-n}} = \lim_{n \to \infty} \left(1+\frac{3}{n}\right)^3 = (1+0)^3 = 1 \]
2. \textbf{Known Series:} The comparison series $\sum b_n = \sum \left(\frac{1}{e}\right)^n$ is a geometric series with ratio $r = 1/e$. Since $|r| < 1$, it converges.
3. \textbf{Conclusion:} Since the limit $L=1$ is finite and positive, and $\sum b_n$ converges, the original series $\sum a_n$ also converges by the Limit Comparison Test.
\textbf{Correct Answer:} The series converges by the Limit Comparison Test with a convergent geometric series.

\subsection{Problem 14}
\textbf{Question:} Determine whether the series $\sum_{n=1}^{\infty} \frac{e^{1/n}}{n}$ converges or diverges.
\textbf{Solution:} Let $a_n = \frac{e^{1/n}}{n}$. As $n \to \infty$, the term $e^{1/n}$ approaches $e^0 = 1$. So, the entire term $a_n$ should behave like $\frac{1}{n}$. Let's use the Limit Comparison Test with the harmonic series, $b_n = \frac{1}{n}$.
1. \textbf{Limit Calculation:}
\[ L = \lim_{n \to \infty} \frac{a_n}{b_n} = \lim_{n \to \infty} \frac{e^{1/n}/n}{1/n} = \lim_{n \to \infty} e^{1/n} = e^0 = 1 \]
2. \textbf{Known Series:} The comparison series $\sum b_n = \sum \frac{1}{n}$ is the harmonic series, which diverges.
3. \textbf{Conclusion:} Since the limit $L=1$ is finite and positive, and $\sum b_n$ diverges, the original series $\sum a_n$ also diverges by the Limit Comparison Test.
The reasoning can also be framed using the Direct Comparison Test. For $n \ge 1$, $1/n > 0$, so $e^{1/n} > e^0 = 1$. This gives the inequality $\frac{e^{1/n}}{n} > \frac{1}{n}$. Since our series is larger than the divergent harmonic series, it must diverge.
\textbf{Correct Answer:} The series diverges by the Direct Comparison Test. Each term is greater than that of the harmonic series.

\subsection{Problem 15}
\textbf{Question:} Determine whether the series $\sum_{n=1}^{\infty} \frac{27}{n!}$ converges or diverges.
\textbf{Solution:} Let $a_n = \frac{27}{n!}$. We know that factorials grow very quickly, much faster than exponential functions. Let's compare it to a geometric series using the Direct Comparison Test.
1. \textbf{Comparison:} Let's find an inequality. We can compare $n!$ to $2^n$.
For $n=1, 1! = 1, 2^1=2$. $1! < 2^1$.
For $n=2, 2! = 2, 2^2=4$. $2! < 2^2$.
For $n=3, 3! = 6, 2^3=8$. $3! < 2^3$.
For $n \ge 4$, we have $n! = n \cdot (n-1) \cdots 4 \cdot 3 \cdot 2 \cdot 1 > 2 \cdot 2 \cdot 2 \cdot 2 \cdots 2 \cdot 1 = 2^{n-1}$.
So, $\frac{1}{n!} < \frac{1}{2^{n-1}}$ for $n \ge 4$.
Therefore, $\frac{27}{n!} < \frac{27}{2^{n-1}}$.
Let's use $b_n = \frac{27}{2^{n-1}} = 27\left(\frac{1}{2}\right)^{n-1}$.
2. \textbf{Known Series:} The series $\sum b_n = \sum 27\left(\frac{1}{2}\right)^{n-1}$ is a geometric series with ratio $r=1/2$. Since $|r|<1$, it converges.
3. \textbf{Conclusion:} Since our series $a_n$ is smaller than a convergent geometric series $b_n$ (for $n \ge 4$), by the Direct Comparison Test, our series converges.
\textbf{Correct Answer:} The series converges by the Direct Comparison Test. Each term is less than that of a convergent geometric series.

\section{In-Depth Analysis of Problems and Techniques}

\subsection{Problem Types and General Approach}
\begin{itemize}
    \item \textbf{Type 1: Conceptual Questions (Problems 1, 2).} These test the fundamental logic of the Comparison Tests. The approach is to carefully read the conditions and determine if they match a conclusive case of the DCT or an inconclusive one.
    \item \textbf{Type 2: Error Analysis (Problem 3).} These problems require a careful line-by-line validation of a proof. The strategy is to check every detail: the choice of comparison series, the setup of the inequality or limit, the algebra, and the final conclusion.
    \item \textbf{Type 3: Comparison with p-Series (Problems 4, 5, 6, 9, 12, 14).} These problems feature terms that are algebraic functions of $n$ (polynomials, roots). The best approach is the Limit Comparison Test. The strategy is to identify the "dominant term" of $a_n$ by taking the ratio of the highest power of $n$ in the numerator and denominator. This becomes your $b_n$, which will be a p-series.
    \item \textbf{Type 4: Comparison with Geometric Series (Problems 7, 10, 13, 15).} These problems involve exponential terms ($c^n$), factorials ($n!$), or other rapidly growing/shrinking functions. The strategy is to compare with a geometric series. LCT is great for clean ratios of exponentials (Prob 7, 13), while DCT is effective when you can establish a clear inequality, such as by bounding a trig function (Prob 10) or using known growth hierarchies (Prob 15).
    \item \textbf{Type 5: Comparison with Harmonic Series (Problem 8, 14).} These problems involve terms that grow slowly, like $\ln(n)$, or behave like $1/n$. The strategy is to compare with the divergent harmonic series $b_n = 1/n$, often using the DCT with a simple inequality like $\ln(n) < n$.
\end{itemize}

\subsection{Key Algebraic and Calculus Manipulations}
\begin{itemize}
    \item \textbf{Identifying Dominant Terms for LCT:} This is the single most important skill for this topic. In Problem 9, for $a_k = \frac{k^{1/3}}{\sqrt{k^3+9k+5}}$, recognizing that for large $k$, this behaves like $\frac{k^{1/3}}{\sqrt{k^3}} = \frac{1}{k^{7/6}}$, is crucial for choosing $b_k$ and making the limit calculation simple.
    \item \textbf{Establishing Inequalities for DCT:}
    \begin{itemize}
        \item \textbf{Bounding Trig Functions:} In Problem 10, the key step was using $-1 \le \sin(n) \le 1$ to bound the numerator, which simplifies the entire term to $a_n \le \frac{2}{e^n}$.
        \item \textbf{Modifying Denominators:} In Problem 4, making the denominator smaller ($n^6+8 \to n^6$) makes the fraction larger, which is a common way to create a simple comparison series.
        \item \textbf{Using Known Growth Rates:} In Problem 8, using the fundamental inequality $\ln(n) < n$ was the key. In Problem 15, knowing $n!$ grows faster than $2^n$ allowed for comparison with a geometric series.
    \end{itemize}
    \item \textbf{Evaluating Limits at Infinity:}
    \begin{itemize}
        \item \textbf{Dividing by the Highest Power:} In Problem 7, for $\lim \frac{10^n}{3+10^n}$, dividing everything by $10^n$ was the standard and effective technique. The same logic applies to polynomials, as seen in Problem 12.
        \item \textbf{Recognizing Embedded Limits:} In Problem 13 and 14, recognizing that $\lim (1+3/n) = 1$ and $\lim e^{1/n} = 1$ simplified the limit of the entire expression.
    \end{itemize}
    \item \textbf{Manipulating Exponents:} In Problem 9, the algebra step $k^{1/3}/k^{3/2} = k^{1/3-3/2} = k^{-7/6}$ was a necessary prerequisite to identifying the correct p-series for comparison.
\end{itemize}

\section{"Cheatsheet" and Tips for Success}
\subsection{Summary of Formulas}
\begin{itemize}
    \item \textbf{p-Series} $\sum \frac{1}{n^p}$: Converges if $p>1$, Diverges if $p \le 1$.
    \item \textbf{Geometric Series} $\sum ar^{n-1}$: Converges if $|r|<1$, Diverges if $|r| \ge 1$.
    \item \textbf{Direct Comparison Test (DCT)}:
    \begin{itemize}
        \item $a_n \le b_n$ and $\sum b_n$ Converges $\implies \sum a_n$ Converges.
        \item $a_n \ge b_n$ and $\sum b_n$ Diverges $\implies \sum a_n$ Diverges.
    \end{itemize}
    \item \textbf{Limit Comparison Test (LCT)}: Let $L = \lim_{n \to \infty} \frac{a_n}{b_n}$.
    \begin{itemize}
        \item If $0 < L < \infty$, then $\sum a_n$ and $\sum b_n$ do the same thing.
    \end{itemize}
\end{itemize}

\subsection{Tricks and Shortcuts}
\begin{itemize}
    \item \textbf{LCT is your workhorse:} When in doubt, try the Limit Comparison Test. It's usually more straightforward than finding a perfect inequality for the DCT.
    \item \textbf{Dominant Term Shortcut:} For any rational-like function, your comparison series $b_n$ is just the highest power of $n$ in the numerator over the highest power in the denominator. The LCT limit will almost always be 1.
    \item \textbf{Drop the Constants:} For LCT, you can ignore constants. For $\frac{n+6}{(n+7)^5}$, just think $\frac{n}{n^5} = \frac{1}{n^4}$.
\end{itemize}

\subsection{Common Pitfalls and How to Avoid Them}
\begin{itemize}
    \item \textbf{Mixing up DCT rules:} Remember: To prove CONVERGENCE, you need to be SMALLER. To prove DIVERGENCE, you need to be BIGGER. Any other comparison is inconclusive.
    \item \textbf{Forgetting Positivity:} The comparison tests only work for series with positive terms.
    \item \textbf{Algebraic Errors:} Be very careful when simplifying fractions and exponents. A small error here will lead you to compare with the wrong p-series.
    \item \textbf{LCT with L=0 or L=inf:} Remember these are one-way conclusions. If $L=0$, you can only conclude convergence if $\sum b_n$ converges. If $L=\infty$, you can only conclude divergence if $\sum b_n$ diverges.
\end{itemize}

\section{Conceptual Synthesis and The "Big Picture"}
\subsection{Thematic Connections}
The core theme of this topic is \textbf{determining the properties of an unknown object by comparing it to a known one}. This is one of the most fundamental strategies in all of mathematics and science. We see this theme repeatedly:
\begin{itemize}
    \item \textbf{The Squeeze Theorem for Limits:} We determine a difficult limit by "squeezing" it between two simpler functions with known, equal limits.
    \item \textbf{Comparison for Improper Integrals:} The convergence of an improper integral can be determined by comparing the integrand to a function whose integral is known to converge or diverge. This is a direct continuous analogue of the DCT for series.
    \item \textbf{Big O Notation in Computer Science:} To analyze an algorithm's complexity, we compare its performance to known growth functions (e.g., linear, quadratic, logarithmic), which is a form of limit comparison.
\end{itemize}

\subsection{Forward and Backward Links}
\begin{itemize}
    \item \textbf{Backward Link (Foundations):} The Comparison Tests are built directly upon two pillars: \textbf{p-series and geometric series}. Without this library of known series, we would have nothing to compare against. The tests are also fundamentally dependent on the concept of \textbf{limits at infinity}, which provides the language to describe the "long-term behavior" of a series' terms.
    \item \textbf{Forward Link (Future Applications):} The skills learned here are essential for the next major topics in series. The mindset of analyzing the long-term behavior of terms is the key idea behind the \textbf{Ratio and Root Tests}. Most importantly, when studying \textbf{Power Series} (e.g., $\sum c_n (x-a)^n$), we must determine the "interval of convergence." While the Ratio Test is often used to find the main interval, the Comparison Tests are frequently required to determine if the series converges or diverges at the \textbf{endpoints} of that interval.
\end{itemize}

\section{Real-World Application and Modeling}
\subsection{Concrete Scenarios in Finance and Economics}
\begin{enumerate}
    \item \textbf{Derivative Pricing:} In financial engineering, the price of a complex derivative security can sometimes be expressed as an infinite series expansion. A quantitative analyst might use the Limit Comparison Test to show that their new, complex pricing series behaves like a known, simpler series (e.g., for a standard option), thereby proving their model produces a stable, finite price rather than an infinite one.
    \item \textbf{Economic Shock Models:} Econometric models often describe how a shock to the economy (like a policy change) dissipates over time. The total economic impact is the sum of the effects in all future periods, an infinite series. For the economy to be considered stable, this series must converge. An economist would use a comparison test to prove that the shock's effects decrease quickly enough (e.g., by comparing them to a convergent geometric series) for the total impact to be finite.
    \item \textbf{Actuarial Science:} The present value of a "perpetuity" (a stream of payments that lasts forever) is calculated as an infinite series. If the payment amounts change over time according to a complex formula, an actuary would need to prove the series converges to ensure the product has a finite, calculable price. They could use the Direct Comparison Test by showing that their complex payment schedule is always less than a simple, known-to-be-summable schedule (like one that decreases geometrically).
\end{enumerate}

\subsection{Model Problem Setup: Valuing a Growth Perpetuity}
\textbf{Scenario:} A company plans to issue a special type of stock that pays a dividend forever. The dividend in year $n$ is projected to be $D_n = \frac{n}{n^3+50}$ dollars. A financial analyst needs to determine if the "intrinsic value" of this stock is finite, assuming a discount rate of 4\% per year.

\textbf{Model Setup:}
\begin{itemize}
    \item \textbf{Variables:}
    \begin{itemize}
        \item $n$: The year, from $n=1$ to $\infty$.
        \item $D_n$: The dividend payment in year $n$, $D_n = \frac{n}{n^3+50}$.
        \item $r$: The discount rate, $r=0.04$.
        \item $PV_n$: The present value of the dividend from year $n$.
        \item $P$: The total price (intrinsic value) of the stock.
    \end{itemize}
    \item \textbf{Formulation:} The present value of a future payment is $PV_n = \frac{D_n}{(1+r)^n}$. The total price is the sum of all future present values.
    \[ P = \sum_{n=1}^{\infty} PV_n = \sum_{n=1}^{\infty} \frac{n}{(n^3+50)(1.04)^n} \]
    \item \textbf{Equation to Analyze:} To determine if the price $P$ is finite, we must test the convergence of the series:
    \[ \sum_{n=1}^{\infty} \frac{n}{(n^3+50)(1.04)^n} \]
    \item \textbf{Application of Comparison Test:} Let $a_n = \frac{n}{(n^3+50)(1.04)^n}$. This looks complicated. We can use the Direct Comparison Test.
    First, let's simplify the algebraic part. For $n \ge 1$, $n^3+50 > n^3$, so $\frac{n}{n^3+50} < \frac{n}{n^3} = \frac{1}{n^2}$.
    This gives us the inequality:
    \[ a_n = \frac{n}{(n^3+50)(1.04)^n} < \frac{1}{n^2 (1.04)^n} \]
    This is better, but still complex. Let's make an even simpler comparison. Since $n^2 \ge 1$ for $n \ge 1$, we can say $\frac{1}{n^2} \le 1$. Therefore:
    \[ a_n < \frac{1}{n^2 (1.04)^n} \le \frac{1}{(1.04)^n} \]
    Let our comparison series be $b_n = \left(\frac{1}{1.04}\right)^n$. This is a geometric series with ratio $|r| = 1/1.04 < 1$, so $\sum b_n$ converges.
    Since $0 < a_n < b_n$ and $\sum b_n$ converges, the series for the price $P$ must also converge. The analyst can conclude that the stock has a finite intrinsic value.
\end{itemize}

\section{Common Variations and Untested Concepts}
\subsection{The LCT with Limit Zero or Infinity}
The homework primarily featured LCT cases where $0 < L < \infty$. It's important to understand the special cases.
\begin{itemize}
    \item \textbf{Case 1: $L=0$. } If $\lim \frac{a_n}{b_n} = 0$, it means $a_n$ is significantly smaller than $b_n$. The only conclusion is: If the larger series $\sum b_n$ \textbf{converges}, then $\sum a_n$ must also \textbf{converge}.
    \item \textbf{Case 2: $L=\infty$. } If $\lim \frac{a_n}{b_n} = \infty$, it means $a_n$ is significantly larger than $b_n$. The only conclusion is: If the smaller series $\sum b_n$ \textbf{diverges}, then $\sum a_n$ must also \textbf{diverge}.
\end{itemize}

\textbf{Worked Example (L=0 case):} Test the convergence of $\sum_{n=1}^{\infty} a_n = \sum_{n=1}^{\infty} \frac{\ln(n)}{n^3}$.
\begin{enumerate}
    \item \textbf{Setup:} The term looks like a p-series but is modified by $\ln(n)$. $\ln(n)$ grows slower than any power of $n$, so let's compare it to a p-series that we suspect converges, like $\sum b_n = \sum \frac{1}{n^2}$.
    \item \textbf{Limit Calculation:}
    \[ L = \lim_{n \to \infty} \frac{a_n}{b_n} = \lim_{n \to \infty} \frac{\ln(n)/n^3}{1/n^2} = \lim_{n \to \infty} \frac{n^2 \ln(n)}{n^3} = \lim_{n \to \infty} \frac{\ln(n)}{n} \]
    This is an indeterminate form $\frac{\infty}{\infty}$, so we use L'Hôpital's Rule:
    \[ L = \lim_{n \to \infty} \frac{1/n}{1} = 0 \]
    \item \textbf{Conclusion:} We found that $L=0$. Our comparison series was $\sum b_n = \sum \frac{1}{n^2}$, which is a convergent p-series ($p=2>1$). According to the rule, since $L=0$ and the comparison series converges, our original series $\sum a_n$ also \textbf{converges}.
\end{enumerate}

\section{Advanced Diagnostic Testing: "Find the Flaw"}
Below are five problems with flawed solutions. Your task is to find the error, explain it, and provide the correct solution.

\subsection{Problem 1}
\textbf{Problem:} Test the convergence of $\sum_{n=3}^{\infty} \frac{1}{n-2}$.
\begin{itemize}
    \item \textbf{Flawed Solution:} We will use the Limit Comparison Test. Let $a_n = \frac{1}{n-2}$ and let's compare to the convergent p-series $b_n = \frac{1}{n^2}$.
    \[ \lim_{n \to \infty} \frac{a_n}{b_n} = \lim_{n \to \infty} \frac{1/(n-2)}{1/n^2} = \lim_{n \to \infty} \frac{n^2}{n-2} = \infty \]
    Since the limit is infinity and the comparison series converges, the test is inconclusive. No conclusion can be drawn.
    \item \textbf{The Flaw:} The error is stopping at an inconclusive result due to a poor choice of comparison series.
    \item \textbf{Correct Solution:} A better choice is the divergent harmonic series $b_n = \frac{1}{n}$.
    \[ L = \lim_{n \to \infty} \frac{1/(n-2)}{1/n} = \lim_{n \to \infty} \frac{n}{n-2} = 1 \]
    Since $L=1$ is finite and positive and $\sum \frac{1}{n}$ diverges, the original series also \textbf{diverges}.
\end{itemize}

\subsection{Problem 2}
\textbf{Problem:} Test the convergence of $\sum_{n=1}^{\infty} \frac{n^2+n}{n^4+5}$.
\begin{itemize}
    \item \textbf{Flawed Solution:} We use the Direct Comparison Test. For $n \ge 1$, we have $n^2+n > 1$ and $n^4+5 > n^4$.
    \[ \frac{n^2+n}{n^4+5} > \frac{1}{n^4} \]
    The series $\sum \frac{1}{n^4}$ is a convergent p-series ($p=4>1$). Since our series is larger than a convergent series, it must diverge by the Direct Comparison Test.
    \item \textbf{The Flaw:} The error is the conclusion; being larger than a convergent series is an inconclusive case for the Direct Comparison Test.
    \item \textbf{Correct Solution:} Use the Limit Comparison Test with $b_n = \frac{n^2}{n^4} = \frac{1}{n^2}$.
    \[ L = \lim_{n \to \infty} \frac{(n^2+n)/(n^4+5)}{1/n^2} = \lim_{n \to \infty} \frac{n^4+n^3}{n^4+5} = 1 \]
    The comparison series $\sum \frac{1}{n^2}$ is a convergent p-series ($p=2>1$). Since $L=1$ is finite and positive, the original series also \textbf{converges}.
\end{itemize}

\subsection{Problem 3}
\textbf{Problem:} Test the convergence of $\sum_{n=1}^{\infty} \frac{3^n}{n!}$.
\begin{itemize}
    \item \textbf{Flawed Solution:} Let's use the Limit Comparison Test. Let's compare $a_n = \frac{3^n}{n!}$ with the divergent geometric series $b_n = (1.5)^n$.
    \[ L = \lim_{n \to \infty} \frac{3^n/n!}{(1.5)^n} = \lim_{n \to \infty} \frac{(3/1.5)^n}{n!} = \lim_{n \to \infty} \frac{2^n}{n!} = 0 \]
    Since the limit is $L=0$ and the comparison series $\sum b_n$ diverges, the original series $\sum a_n$ must also diverge.
    \item \textbf{The Flaw:} The error is a misapplication of the LCT case for $L=0$; a divergent comparison series with $L=0$ is an inconclusive result.
    \item \textbf{Correct Solution:} The LCT with this $b_n$ is inconclusive. A better test for this series is the Ratio Test, but to use comparison, we must establish an inequality. For $n \ge 4$, $n! > 3^n$. For example, $4! = 24 < 3^4=81$, $5! = 120 < 3^5=243$, $6! = 720 > 3^6=729$ is false. Let's try comparing to $b_n = (\frac{1}{2})^n$. For $n \ge 6$, we have $n! > 2 \cdot 6^n$. So $\frac{3^n}{n!} < \frac{3^n}{2 \cdot 6^n} = \frac{1}{2}(\frac{1}{2})^n$. Since $\sum \frac{1}{2}(\frac{1}{2})^n$ is a convergent geometric series, our series \textbf{converges} by DCT.
\end{itemize}

\subsection{Problem 4}
\textbf{Problem:} Test the convergence of $\sum_{n=1}^{\infty} \frac{\cos(n)+2}{n^{3/2}}$.
\begin{itemize}
    \item \textbf{Flawed Solution:} We use the Direct Comparison Test. We know that $-1 \le \cos(n) \le 1$. Adding 2, we get $1 \le \cos(n)+2 \le 3$.
    Therefore, we can establish the inequality:
    \[ \frac{\cos(n)+2}{n^{3/2}} \ge \frac{1}{n^{3/2}} \]
    The series $\sum \frac{1}{n^{3/2}}$ is a convergent p-series ($p=3/2 > 1$). Since our series is greater than a convergent series, no conclusion can be drawn from the Direct Comparison Test.
    \item \textbf{The Flaw:} The error is failing to use the other side of the inequality, which provides a conclusive result.
    \item \textbf{Correct Solution:} The solution correctly established $1 \le \cos(n)+2 \le 3$. From this, we should use the upper bound:
    \[ 0 < \frac{\cos(n)+2}{n^{3/2}} \le \frac{3}{n^{3/2}} \]
    The series $\sum \frac{3}{n^{3/2}}$ is a constant multiple of a convergent p-series ($p=3/2 > 1$) and therefore converges. Since our series is smaller than a convergent series, it \textbf{converges} by the Direct Comparison Test.
\end{itemize}

\subsection{Problem 5}
\textbf{Problem:} Test the convergence of $\sum_{n=1}^{\infty} \frac{\sqrt{n}}{n+1}$.
\begin{itemize}
    \item \textbf{Flawed Solution:} We use the Limit Comparison Test. Let $a_n = \frac{\sqrt{n}}{n+1}$. The dominant terms are $\frac{\sqrt{n}}{n} = \frac{1}{\sqrt{n}}$. We will compare with $b_n = \frac{1}{\sqrt{n}}$.
    \[ L = \lim_{n \to \infty} \frac{\sqrt{n}/(n+1)}{1/\sqrt{n}} = \lim_{n \to \infty} \frac{n}{n+1} = 1 \]
    The limit is $L=1$. The series $\sum b_n = \sum \frac{1}{\sqrt{n}}$ is the harmonic series, which converges. Therefore, the original series converges.
    \item \textbf{The Flaw:} The error is that the comparison series $\sum \frac{1}{\sqrt{n}}$ is a p-series with $p=1/2$, which diverges, not converges.
    \item \textbf{Correct Solution:} The setup and limit calculation are correct. The comparison series is $\sum b_n = \sum \frac{1}{n^{1/2}}$. This is a p-series with $p=1/2$. Since $p \le 1$, this series diverges. Since $L=1$ is finite and positive, the original series shares the same fate and also \textbf{diverges}.
\end{itemize}

\end{document}```