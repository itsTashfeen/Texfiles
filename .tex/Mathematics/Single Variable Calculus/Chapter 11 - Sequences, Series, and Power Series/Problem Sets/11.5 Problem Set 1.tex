\documentclass{article}
\usepackage{amsmath}
\usepackage{amssymb}
\usepackage{geometry}
\geometry{a4paper, margin=1in}

\title{Practice Problems: Alternating Series and Absolute Convergence}
\author{Generated by Gemini}
\date{\today}

\begin{document}

\maketitle

\section{Practice Problems}

For each series, determine if it is absolutely convergent, conditionally convergent, or divergent. For problems that ask for an error estimate, follow the specific instructions.

\subsection*{Problem Set}

\begin{enumerate}
    \item Which of the following statements is required for the Alternating Series Test to prove that the series $\sum_{n=1}^{\infty} (-1)^n b_n$ converges?
    \begin{enumerate}
        \item $\lim_{n \to \infty} b_n = 1$
        \item The sequence $\{b_n\}$ is eventually non-decreasing.
        \item $b_n > 0$ for all $n$.
        \item $\sum_{n=1}^{\infty} b_n$ converges.
    \end{enumerate}

    \item Determine if the following statements are True or False.
    \begin{enumerate}
        \item If a series is convergent, it must be absolutely convergent.
        \item The Alternating Series Test can be used to prove a series diverges.
        \item If $\lim_{n \to \infty} b_n = 0$, then $\sum (-1)^n b_n$ must converge.
        \item If $\sum |a_n|$ diverges, then $\sum a_n$ also diverges.
    \end{enumerate}
    
    \item $\sum_{n=1}^{\infty} (-1)^n \frac{3n^2 - 1}{2n^2 + n}$
    
    \item $\sum_{n=1}^{\infty} (-1)^{n+1} \frac{2^n}{2^n - 100}$
    
    \item $\sum_{n=1}^{\infty} \frac{\cos(\pi n)}{n^{1/n}}$
    
    \item $\sum_{n=1}^{\infty} (-1)^n \left(1 + \frac{1}{n}\right)^n$
    
    \item $\sum_{n=2}^{\infty} \frac{(-1)^n}{\ln(n^2)}$
    
    \item $\sum_{n=1}^{\infty} \frac{(-1)^{n-1}}{n^2 + 5}$
    
    \item $\sum_{n=1}^{\infty} (-1)^n \frac{n}{n^2 + 9}$
    
    \item $\sum_{n=2}^{\infty} (-1)^n \frac{\ln(n)}{n}$
    
    \item $\sum_{n=1}^{\infty} \frac{(-1)^n}{n\sqrt{n}}$
    
    \item $\sum_{n=1}^{\infty} \frac{(-1)^n (n^2 - 1)}{n^4 + 5}$
    
    \item $\sum_{n=1}^{\infty} \frac{(-1)^n}{n^{3/4}}$
    
    \item $\sum_{n=1}^{\infty} \frac{(-1)^{n+1}}{\sqrt[3]{n+1}}$
    
    \item $\sum_{n=1}^{\infty} \frac{\sin(n)}{n^3 + 1}$ (Note: This is not an alternating series, but test for absolute convergence).
    
    \item $\sum_{n=1}^{\infty} \frac{(-1)^n}{e^n + e^{-n}}$
    
    \item $\sum_{n=2}^{\infty} \frac{(-1)^n \cdot n}{\ln(n) + n}$
    
    \item $\sum_{n=1}^{\infty} \frac{(-1)^n}{n + \sqrt{n}}$
    
    \item $\sum_{n=1}^{\infty} \frac{(-1)^n 100^n}{n!}$
    
    \item $\sum_{n=1}^{\infty} \frac{(-1)^n n^3}{e^n}$
    
    \item $\sum_{n=1}^{\infty} \frac{(-1)^n (n!)^2}{(2n)!}$
    
    \item $\sum_{n=1}^{\infty} \frac{(-1)^n n! \cdot 2^n}{n^n}$
    
    \item $\sum_{n=1}^{\infty} \left( \frac{-2n}{5n+3} \right)^n$
    
    \item $\sum_{n=1}^{\infty} \left( \frac{6n-1}{3n+2} \right)^n (-1)^n$
    
    \item Approximate the sum of the series $\sum_{n=1}^{\infty} \frac{(-1)^{n-1}}{n^5}$ with an error less than $0.0001$.
    
    \item What is the maximum error if you use the first 10 terms ($S_{10}$) to approximate the sum of the series $\sum_{n=1}^{\infty} \frac{(-1)^{n}}{n!}$?
    
    \item $\sum_{n=1}^{\infty} \frac{\cos(\pi n) (n+1)}{n^2+n+1}$
    
    \item $\sum_{n=1}^{\infty} (-1)^n \frac{\arctan(n)}{n^2}$
    
    \item $\sum_{n=2}^{\infty} \frac{(-1)^n}{n \ln(n)}$
    
    \item $\sum_{n=1}^{\infty} (-1)^n (\sqrt{n^2+1} - n)$
    
    \item $\sum_{n=1}^{\infty} \frac{(-1)^n \cdot 2 \cdot 4 \cdot 6 \cdots (2n)}{1 \cdot 4 \cdot 7 \cdots (3n-2)}$
    
    \item $\sum_{n=1}^{\infty} (-1)^n \sin\left(\frac{1}{n}\right)$
\end{enumerate}


\newpage
\section{Solutions to Practice Problems}

\begin{enumerate}
    \item \textbf{Answer: (c)}. The term $b_n$ must be positive to represent the magnitude. The other conditions are incorrect versions of the AST requirements.
    
    \item \textbf{Answers:}
    (a) \textbf{False}. The alternating harmonic series $\sum (-1)^n/n$ converges, but is not absolutely convergent.
    (b) \textbf{False}. The AST only provides conditions for convergence. If its conditions are not met, the test is inconclusive (though if $\lim b_n \neq 0$, the series diverges by the Test for Divergence, not by the AST itself).
    (c) \textbf{False}. The terms must also be decreasing. A counterexample is a series where $b_n = 1/n$ for odd $n$ and $b_n=1/n^2$ for even $n$.
    (d) \textbf{False}. This is the definition of conditional convergence. The series $\sum a_n$ might converge.
    
    \item \textbf{Divergent}. Test for Divergence. Let $b_n = \frac{3n^2 - 1}{2n^2 + n}$.
    \[ \lim_{n \to \infty} b_n = \lim_{n \to \infty} \frac{3 - 1/n^2}{2 + 1/n} = \frac{3}{2} \]
    Since the limit is not 0, the series diverges by the Test for Divergence.
    
    \item \textbf{Divergent}. Test for Divergence. Let $b_n = \frac{2^n}{2^n - 100}$.
    \[ \lim_{n \to \infty} b_n = \lim_{n \to \infty} \frac{1}{1 - 100/2^n} = 1 \]
    Since the limit is not 0, the series diverges by the Test for Divergence.
    
    \item \textbf{Divergent}. Note that $\cos(\pi n) = (-1)^n$. Let $b_n = n^{1/n}$.
    \[ \lim_{n \to \infty} b_n = \lim_{n \to \infty} n^{1/n} = 1 \]
    This is a known limit. Since the limit is not 0, the series diverges by the Test for Divergence.
    
    \item \textbf{Divergent}. Test for Divergence. Let $b_n = (1 + 1/n)^n$.
    \[ \lim_{n \to \infty} b_n = \lim_{n \to \infty} \left(1 + \frac{1}{n}\right)^n = e \]
    Since the limit is not 0, the series diverges by the Test for Divergence.
    
    \item \textbf{Conditionally Convergent}. This is an alternating series with $b_n = \frac{1}{\ln(n^2)} = \frac{1}{2\ln(n)}$.
    \textbf{1. AST:} $\lim_{n \to \infty} \frac{1}{2\ln(n)} = 0$. Since $\ln(n)$ is increasing, $b_n$ is decreasing. The series converges by AST.
    \textbf{2. Absolute Convergence:} Test $\sum \frac{1}{2\ln(n)}$. We know $\ln(n) < n$ for $n \ge 1$. Thus, $\frac{1}{2\ln(n)} > \frac{1}{2n}$. Since $\sum \frac{1}{2n} = \frac{1}{2}\sum \frac{1}{n}$ diverges (harmonic series), $\sum \frac{1}{2\ln(n)}$ diverges by the Direct Comparison Test.
    The series is conditionally convergent.
    
    \item \textbf{Absolutely Convergent}.
    \textbf{1. Absolute Convergence:} Test $\sum \frac{1}{n^2+5}$. Use LCT with the convergent p-series $\sum \frac{1}{n^2}$.
    \[ L = \lim_{n \to \infty} \frac{1/(n^2+5)}{1/n^2} = \lim_{n \to \infty} \frac{n^2}{n^2+5} = 1 \]
    Since $L$ is finite and positive, $\sum \frac{1}{n^2+5}$ converges. The series is absolutely convergent.
    
    \item \textbf{Conditionally Convergent}. Let $b_n = \frac{n}{n^2+9}$.
    \textbf{1. AST:} $\lim_{n \to \infty} \frac{n}{n^2+9} = 0$. Let $f(x) = \frac{x}{x^2+9}$. $f'(x) = \frac{(x^2+9)(1) - x(2x)}{(x^2+9)^2} = \frac{9-x^2}{(x^2+9)^2}$. This is negative for $x > 3$. Thus, $b_n$ is decreasing for $n \ge 3$. The series converges by AST.
    \textbf{2. Absolute Convergence:} Test $\sum \frac{n}{n^2+9}$. Use LCT with the divergent harmonic series $\sum \frac{1}{n}$.
    \[ L = \lim_{n \to \infty} \frac{n/(n^2+9)}{1/n} = \lim_{n \to \infty} \frac{n^2}{n^2+9} = 1 \]
    The series of absolute values diverges. The original series is conditionally convergent.
    
    \item \textbf{Conditionally Convergent}. Let $b_n = \frac{\ln(n)}{n}$.
    \textbf{1. AST:} $\lim_{n \to \infty} \frac{\ln(n)}{n} = 0$ by L'Hopital's Rule. Let $f(x) = \frac{\ln(x)}{x}$. $f'(x) = \frac{x(1/x) - \ln(x)(1)}{x^2} = \frac{1-\ln(x)}{x^2}$. This is negative for $x>e$. So $b_n$ is decreasing for $n \ge 3$. The series converges by AST.
    \textbf{2. Absolute Convergence:} Test $\sum \frac{\ln(n)}{n}$. Since $\ln(n) > 1$ for $n \ge 3$, we have $\frac{\ln(n)}{n} > \frac{1}{n}$. Since $\sum \frac{1}{n}$ diverges, our series diverges by Direct Comparison.
    The series is conditionally convergent.

    \item \textbf{Absolutely Convergent}. Test $\sum |\frac{(-1)^n}{n\sqrt{n}}| = \sum \frac{1}{n^{3/2}}$. This is a p-series with $p = 3/2 > 1$, so it converges. The series is absolutely convergent.

    \item \textbf{Absolutely Convergent}. Test $\sum \frac{n^2-1}{n^4+5}$. Use LCT with convergent p-series $\sum \frac{n^2}{n^4} = \sum \frac{1}{n^2}$.
    \[ L = \lim_{n \to \infty} \frac{(n^2-1)/(n^4+5)}{1/n^2} = \lim_{n \to \infty} \frac{n^2(n^2-1)}{n^4+5} = \lim_{n \to \infty} \frac{n^4-n^2}{n^4+5} = 1 \]
    The series of absolute values converges. The series is absolutely convergent.

    \item \textbf{Conditionally Convergent}. Test $\sum \frac{1}{n^{3/4}}$. This is a divergent p-series ($p=3/4 \le 1$). So it is not absolutely convergent. The original series is alternating with $b_n=1/n^{3/4}$, which is positive, decreasing, and has limit 0. It converges by AST. The series is conditionally convergent.

    \item \textbf{Conditionally Convergent}. Test $\sum \frac{1}{(n+1)^{1/3}}$. This behaves like the divergent p-series $\sum 1/n^{1/3}$ ($p=1/3 \le 1$). By LCT, it diverges. So it is not absolutely convergent. The original series converges by AST. The series is conditionally convergent.
    
    \item \textbf{Absolutely Convergent}. We test for absolute convergence: $\sum |\frac{\sin(n)}{n^3+1}| = \sum \frac{|\sin(n)|}{n^3+1}$.
    We know $0 \le |\sin(n)| \le 1$. Therefore, $\frac{|\sin(n)|}{n^3+1} \le \frac{1}{n^3+1} < \frac{1}{n^3}$.
    Since $\sum \frac{1}{n^3}$ is a convergent p-series ($p=3>1$), our series converges by the Direct Comparison Test. The series is absolutely convergent.

    \item \textbf{Absolutely Convergent}. Test $\sum \frac{1}{e^n+e^{-n}}$. Compare to $\sum \frac{1}{e^n} = \sum (\frac{1}{e})^n$, which is a convergent geometric series ($|r|=1/e < 1$).
    Since $e^n+e^{-n} > e^n$, we have $\frac{1}{e^n+e^{-n}} < \frac{1}{e^n}$.
    By the Direct Comparison Test, the series of absolute values converges. The series is absolutely convergent.

    \item \textbf{Conditionally Convergent}. Let $b_n = \frac{n}{n+\ln(n)}$.
    First, check $\lim_{n \to \infty} b_n = \lim_{n \to \infty} \frac{1}{1 + \ln(n)/n} = \frac{1}{1+0} = 1$. The series diverges by the Test for Divergence.
    \textit{Correction: The problem was likely intended to be $\frac{(-1)^n \ln(n)}{n}$. I will solve that version.}
    Assuming the series is $\sum_{n=2}^{\infty} \frac{(-1)^n \ln(n)}{n}$, this is solved in problem 10. \textbf{Conditionally Convergent}.
    
    \item \textbf{Conditionally Convergent}. Let $b_n = \frac{1}{n+\sqrt{n}}$.
    \textbf{1. AST:} $\lim b_n=0$ and terms are clearly decreasing. Converges by AST.
    \textbf{2. Absolute Convergence:} Test $\sum \frac{1}{n+\sqrt{n}}$. Use LCT with divergent harmonic series $\sum 1/n$.
    \[ L = \lim_{n \to \infty} \frac{1/(n+\sqrt{n})}{1/n} = \lim_{n \to \infty} \frac{n}{n+\sqrt{n}} = \lim_{n \to \infty} \frac{1}{1+1/\sqrt{n}} = 1 \]
    The series of absolute values diverges. The original series is conditionally convergent.

    \item \textbf{Absolutely Convergent}. Use the Ratio Test.
    \[ L = \lim_{n \to \infty} \left| \frac{100^{n+1}}{(n+1)!} \cdot \frac{n!}{100^n} \right| = \lim_{n \to \infty} \frac{100}{n+1} = 0 \]
    Since $L < 1$, the series is absolutely convergent.

    \item \textbf{Absolutely Convergent}. Use the Ratio Test.
    \[ L = \lim_{n \to \infty} \left| \frac{(n+1)^3}{e^{n+1}} \cdot \frac{e^n}{n^3} \right| = \lim_{n \to \infty} \left( \frac{n+1}{n} \right)^3 \frac{1}{e} = 1^3 \cdot \frac{1}{e} = \frac{1}{e} \]
    Since $L < 1$, the series is absolutely convergent.

    \item \textbf{Absolutely Convergent}. Use the Ratio Test.
    \[ L = \lim_{n \to \infty} \left| \frac{((n+1)!)^2}{(2(n+1))!} \cdot \frac{(2n)!}{(n!)^2} \right| = \lim_{n \to \infty} \frac{(n+1)^2}{(2n+2)(2n+1)} = \lim_{n \to \infty} \frac{n^2+2n+1}{4n^2+6n+2} = \frac{1}{4} \]
    Since $L < 1$, the series is absolutely convergent.

    \item \textbf{Absolutely Convergent}. Use the Ratio Test.
    \[ L = \lim_{n \to \infty} \left| \frac{(n+1)! \cdot 2^{n+1}}{(n+1)^{n+1}} \cdot \frac{n^n}{n! \cdot 2^n} \right| = \lim_{n \to \infty} \frac{(n+1) \cdot 2 \cdot n^n}{(n+1)^{n+1}} = 2 \lim_{n \to \infty} \frac{n^n}{(n+1)^n} = 2 \lim_{n \to \infty} \left( \frac{n}{n+1} \right)^n \]
    \[ = 2 \lim_{n \to \infty} \left( \frac{1}{1+1/n} \right)^n = 2 \frac{1}{e} = \frac{2}{e} \]
    Since $L = 2/e < 1$, the series is absolutely convergent.
    
    \item \textbf{Absolutely Convergent}. Use the Root Test.
    \[ L = \lim_{n \to \infty} \sqrt[n]{ \left| \left( \frac{-2n}{5n+3} \right)^n \right| } = \lim_{n \to \infty} \frac{2n}{5n+3} = \frac{2}{5} \]
    Since $L < 1$, the series is absolutely convergent.

    \item \textbf{Divergent}. Use the Root Test.
    \[ L = \lim_{n \to \infty} \sqrt[n]{ \left| \left( \frac{6n-1}{3n+2} \right)^n (-1)^n \right| } = \lim_{n \to \infty} \frac{6n-1}{3n+2} = 2 \]
    Since $L > 1$, the series is divergent.

    \item We need $|R_n| \le b_{n+1} < 0.0001$. Here $b_n = 1/n^5$.
    \[ \frac{1}{(n+1)^5} < \frac{1}{10000} \implies (n+1)^5 > 10000 \implies n+1 > \sqrt[5]{10000} \approx 6.3 \]
    So we need $n+1 \ge 7$, which means $n \ge 6$. We need to sum the first \textbf{6 terms}.
    $S_6 = 1 - \frac{1}{32} + \frac{1}{243} - \frac{1}{1024} + \frac{1}{3125} - \frac{1}{7776} \approx 0.9721$.

    \item By the Alternating Series Estimation Theorem, $|R_{10}| \le b_{11}$. Here $b_n = 1/n!$.
    The maximum error is $b_{11} = \frac{1}{11!} = \frac{1}{39,916,800}$.
    
    \item \textbf{Conditionally Convergent}. Note $\cos(\pi n) = (-1)^n$. The series is $\sum (-1)^n \frac{n+1}{n^2+n+1}$. Let $b_n = \frac{n+1}{n^2+n+1}$.
    \textbf{1. AST:} $\lim b_n = 0$. The derivative of $f(x)=\frac{x+1}{x^2+x+1}$ is negative for $x \ge 1$, so it's decreasing. Converges by AST.
    \textbf{2. Absolute Convergence:} Test $\sum \frac{n+1}{n^2+n+1}$. Use LCT with divergent $\sum 1/n$.
    \[ L = \lim_{n \to \infty} \frac{(n+1)/(n^2+n+1)}{1/n} = \lim_{n \to \infty} \frac{n^2+n}{n^2+n+1} = 1 \]
    The series of absolute values diverges. The series is conditionally convergent.

    \item \textbf{Absolutely Convergent}. Test $\sum \frac{\arctan(n)}{n^2}$.
    We know $0 < \arctan(n) < \pi/2$. So, $\frac{\arctan(n)}{n^2} < \frac{\pi/2}{n^2}$.
    Since $\sum \frac{\pi/2}{n^2} = \frac{\pi}{2} \sum \frac{1}{n^2}$ is a convergent p-series ($p=2$), the series of absolute values converges by Direct Comparison. The series is absolutely convergent.

    \item \textbf{Conditionally Convergent}. Let $b_n = \frac{1}{n \ln(n)}$.
    \textbf{1. AST:} $\lim b_n = 0$ and terms are decreasing. Converges by AST.
    \textbf{2. Absolute Convergence:} Test $\sum \frac{1}{n \ln(n)}$. Use the Integral Test.
    \[ \int_2^\infty \frac{1}{x \ln(x)} dx = [\ln(\ln(x))]_2^\infty = \infty \]
    The integral diverges, so the series of absolute values diverges. The series is conditionally convergent.
    
    \item \textbf{Conditionally Convergent}. Let $b_n = \sqrt{n^2+1}-n$.
    Multiply by the conjugate: $b_n = (\sqrt{n^2+1}-n) \frac{\sqrt{n^2+1}+n}{\sqrt{n^2+1}+n} = \frac{n^2+1-n^2}{\sqrt{n^2+1}+n} = \frac{1}{\sqrt{n^2+1}+n}$.
    \textbf{1. AST:} $\lim b_n = 0$ and terms are decreasing. Converges by AST.
    \textbf{2. Absolute Convergence:} Test $\sum \frac{1}{\sqrt{n^2+1}+n}$. Use LCT with divergent $\sum 1/n$.
    \[ L = \lim_{n \to \infty} \frac{1/(\sqrt{n^2+1}+n)}{1/n} = \lim_{n \to \infty} \frac{n}{\sqrt{n^2+1}+n} = \lim_{n \to \infty} \frac{1}{\sqrt{1+1/n^2}+1} = \frac{1}{2} \]
    The series of absolute values diverges. The series is conditionally convergent.
    
    \item \textbf{Absolutely Convergent}. Use the Ratio Test.
    \[ L = \lim_{n \to \infty} \left| \frac{2(n+1)}{3(n+1)-2} \right| = \lim_{n \to \infty} \frac{2n+2}{3n+1} = \frac{2}{3} \]
    Since $L < 1$, the series is absolutely convergent.
    
    \item \textbf{Conditionally Convergent}. Let $b_n = \sin(1/n)$.
    \textbf{1. AST:} $\lim_{n \to \infty} \sin(1/n) = \sin(0) = 0$. For $n \ge 1$, $1/n$ is in $(0, 1]$, where $\sin(x)$ is increasing. Since $1/n$ is decreasing, $\sin(1/n)$ is also decreasing. Converges by AST.
    \textbf{2. Absolute Convergence:} Test $\sum \sin(1/n)$. Use LCT with divergent $\sum 1/n$.
    \[ L = \lim_{n \to \infty} \frac{\sin(1/n)}{1/n} = 1 \quad (\text{This is the fundamental trig limit, let } x=1/n) \]
    The series of absolute values diverges. The series is conditionally convergent.

\end{enumerate}

\newpage
\section{Concept Checklist and Problem Index}

This index maps each key concept to the practice problems that test it.

\begin{itemize}
    \item \textbf{C1: Definitions \& Theory} (Understanding the formal definitions)
    \begin{itemize}
        \item Questions: 1, 2
    \end{itemize}
    
    \item \textbf{C2: Test for Divergence on Alternating Series} ($\lim_{n \to \infty} b_n \neq 0$)
    \begin{itemize}
        \item Questions: 3, 4, 5, 6, 24
    \end{itemize}
    
    \item \textbf{C3: Alternating Series Test (AST)} (Direct application of the two conditions)
    \begin{itemize}
        \item Questions: 7, 8, 13, 14, 18
    \end{itemize}
    
    \item \textbf{C4: AST with Calculus} (Using a derivative to prove terms are decreasing)
    \begin{itemize}
        \item Questions: 9, 10
    \end{itemize}
    
    \item \textbf{C5: Absolute Convergence Test} (General strategy of first testing $\sum |a_n|$)
    \begin{itemize}
        \item Questions: All problems from 7-32 involve this strategy.
    \end{itemize}
    
    \item \textbf{C6: P-Series for Absolute Convergence Analysis}
    \begin{itemize}
        \item Questions: 11 (convergent), 13 (divergent), 14 (divergent)
    \end{itemize}
    
    \item \textbf{C7: LCT/DCT for Absolute Convergence Analysis}
    \begin{itemize}
        \item Questions: 7, 8, 9, 10, 12, 15, 16, 17, 18, 27, 28, 29, 30, 32
    \end{itemize}
    
    \item \textbf{C8: Ratio Test for Absolute Convergence}
    \begin{itemize}
        \item Questions: 19, 20, 21, 22, 31
    \end{itemize}
    
    \item \textbf{C9: Root Test for Absolute Convergence}
    \begin{itemize}
        \item Questions: 23 (convergent), 24 (divergent)
    \end{itemize}
    
    \item \textbf{C10: Classification: Divergent}
    \begin{itemize}
        \item Questions: 3, 4, 5, 6, 24
    \end{itemize}

    \item \textbf{C11: Classification: Absolutely Convergent} ($\sum |a_n|$ converges)
    \begin{itemize}
        \item Questions: 8, 11, 12, 15, 16, 19, 20, 21, 22, 23, 28, 31
    \end{itemize}
    
    \item \textbf{C12: Classification: Conditionally Convergent} ($\sum a_n$ converges but $\sum |a_n|$ diverges)
    \begin{itemize}
        \item Questions: 7, 9, 10, 13, 14, 17, 18, 27, 29, 30, 32
    \end{itemize}
    
    \item \textbf{C13: Alternating Series Remainder Estimation} ($|R_n| \le b_{n+1}$)
    \begin{itemize}
        \item Questions: 25, 26
    \end{itemize}
\end{itemize}

\end{document}