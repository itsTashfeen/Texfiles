\documentclass{article}
\usepackage{amsmath}
\usepackage{amssymb}
\usepackage{geometry}
\geometry{a4paper, margin=1in}
\usepackage{multicol}

\title{Problem Set 11.10: Taylor and Maclaurin Series}
\author{Generated by Gemini}
\date{November 2025}

\begin{document}

\maketitle

\part*{Problem Set}

\section{Direct Definition and Basic Concepts}

\subsection{Problems}
\begin{enumerate}
    \item If a function $f(x)$ is represented by the power series $\sum_{n=0}^{\infty} c_n (x-5)^n$, what is the formula for the coefficient $c_7$?
    \item A Taylor series for a function $f(x)$ is centered at $a=-2$. If the series is given by $\sum_{n=0}^{\infty} b_n(x+2)^n$, write the formula for $b_{10}$.
    \item Given that the Maclaurin series for a function $g(x)$ is $\sum_{n=0}^{\infty} a_n x^n$, provide the explicit formula for the coefficient $a_5$.
    \item The third-degree Taylor polynomial for a function $f(x)$ centered at $a=1$ is $T_3(x) = 2 - (x-1) + 3(x-1)^2 - 5(x-1)^3$. What are the values of $f(1)$, $f'(1)$, $f''(1)$, and $f'''(1)$?
    \item Let $f(x)$ have a Taylor series centered at $a=0$. If the series begins $3 + 2x - \frac{1}{2}x^2 + \frac{5}{3}x^3 + \dots$, what is the value of $f^{(3)}(0)$?
\end{enumerate}

\section{Constructing Series from a Derivative Formula}

\subsection{Problems}
\begin{enumerate}
    \setcounter{enumi}{5}
    \item Given $f^{(n)}(0) = n!$ for all $n \ge 0$, find the Maclaurin series for $f(x)$ and its radius of convergence.
    \item Find the Taylor series for a function $f(x)$ centered at $a=1$, if it is known that $f^{(n)}(1) = \frac{(-1)^n n!}{2^n}$. Determine the radius of convergence.
    \item If a function $g(x)$ has derivatives at $x=0$ given by $g^{(n)}(0) = (-1)^n \frac{(n+1)!}{3^n}$, find the Maclaurin series for $g(x)$ and its radius of convergence.
    \item A function $h(x)$ is centered at $a=-3$ and its derivatives are given by $h^{(n)}(-3) = \frac{10}{5^n (n+2)}$. Find the Taylor series for $h(x)$ and its radius of convergence.
    \item Determine the Maclaurin series and its radius of convergence for a function $f(x)$ where $f^{(n)}(0) = (2n)!$.
\end{enumerate}

\section{Constructing Series by Differentiating the Function}

\subsection{Problems}
\begin{enumerate}
    \setcounter{enumi}{10}
    \item Find the Maclaurin series for $f(x) = \cos(x)$ and its radius of convergence.
    \item Find the Taylor series for $f(x) = \ln(x)$ centered at $a=1$ and find its radius of convergence.
    \item Find the Maclaurin series for $f(x) = e^{-2x}$ and its radius of convergence.
    \item Find the Taylor series for $f(x) = \frac{1}{x}$ centered at $a=2$ and determine its radius of convergence.
    \item Find the Maclaurin series for $f(x) = \sinh(x) = \frac{e^x - e^{-x}}{2}$ and its radius of convergence. (Hint: Use the known series for $e^x$.)
    \item Find the first four non-zero terms of the Taylor series for $f(x) = \sqrt{x}$ centered at $a=4$.
    \item Find the Maclaurin series for $f(x) = (1-x)^{-1}$ (the geometric series).
    \item Find the Maclaurin series for $f(x) = \sin(3x)$.
    \item Find the Taylor series for $f(x) = x^3 - 2x + 4$ centered at $a=1$.
    \item Find the Maclaurin series for $f(x) = \frac{1}{(1+x)^2}$. (Hint: This is related to the derivative of a known series).
\end{enumerate}

\section{Constructing Series by Manipulating Known Series}

\subsection{Problems}
\begin{enumerate}
    \setcounter{enumi}{20}
    \item Find the Maclaurin series for $f(x) = e^{x^2}$ by substituting into the series for $e^x$. What is the radius of convergence?
    \item Find the Maclaurin series for $f(x) = \cos(\sqrt{x})$ for $x \ge 0$.
    \item Use the series for $\frac{1}{1-x}$ to find the Maclaurin series for $f(x) = \frac{1}{1+x^2}$. What is its radius of convergence?
    \item By integrating the series from the previous problem, find the Maclaurin series for $f(x) = \arctan(x)$.
    \item Find the Maclaurin series for $f(x) = x^2 \sin(x)$.
    \item Find the Maclaurin series for $f(x) = \frac{x}{1-2x}$.
    \item By differentiating the series for $\sin(x)$, find the series for $\cos(x)$.
    \item Find the Maclaurin series for $f(x) = \ln(1-x^2)$ by integrating a known series.
    \item Find the first three non-zero terms of the Maclaurin series for $f(x) = e^x \cos(x)$ by multiplying the respective series.
    \item Find the Maclaurin series for $f(x) = \frac{\sin(x)}{x}$. Use this series to evaluate $\lim_{x\to 0} \frac{\sin(x)}{x}$.
    \item Use a known series to evaluate the indefinite integral $\int \cos(x^3) dx$ as a power series.
    \item Find the Maclaurin series for $f(x) = \cosh(x) = \frac{e^x + e^{-x}}{2}$.
\end{enumerate}

\section{Binomial Series}

\subsection{Problems}
\begin{enumerate}
    \setcounter{enumi}{32}
    \item Use the binomial series to expand $f(x) = \frac{1}{\sqrt{1+x}}$ as a power series. State the radius of convergence.
    \item Find the first four terms of the binomial series for $f(x) = \sqrt[3]{1-x}$.
    \item Find the Maclaurin series for $f(x) = (1+x^2)^{-1/2}$.
    \item Use the binomial series to find the Maclaurin series for $f(x) = \arcsin(x)$. (Hint: Recall that $\arcsin(x) = \int \frac{1}{\sqrt{1-x^2}} dx$).
\end{enumerate}


\section{Radius and Interval of Convergence}

\subsection{Problems}
\begin{enumerate}
    \setcounter{enumi}{36}
    \item Find the radius and interval of convergence for the series $\sum_{n=0}^{\infty} \frac{(x-2)^n}{n!}$.
    \item Find the radius and interval of convergence for the series $\sum_{n=1}^{\infty} \frac{(x+1)^n}{n \cdot 3^n}$.
    \item Find the radius and interval of convergence for the series $\sum_{n=0}^{\infty} n!(2x-1)^n$.
    \item Find the radius and interval of convergence for the series $\sum_{n=1}^{\infty} \frac{(-1)^n x^n}{\sqrt{n}}$.
    \item Find the radius and interval of convergence for the series $\sum_{n=0}^{\infty} \frac{x^{2n}}{4^n}$.
    \item Find the radius and interval of convergence for the series $\sum_{n=1}^{\infty} \frac{(4x-5)^n}{n^2}$.
\end{enumerate}

\section{Taylor's Inequality and Error Estimation}

\subsection{Problems}
\begin{enumerate}
    \setcounter{enumi}{42}
    \item Use the Maclaurin polynomial of degree 3 for $f(x)=e^x$ to approximate $e^{0.1}$. Use Taylor's Inequality to estimate the accuracy of the approximation.
    \item Find the degree $n$ of the Taylor polynomial for $f(x)=\cos(x)$ centered at $a=0$ that is needed to estimate $\cos(0.2)$ with an error of less than $0.0001$.
    \item Let $f(x) = \ln(1+x)$. Use the third-degree Maclaurin polynomial to approximate $\ln(1.5)$. Use Taylor's Inequality to bound the remainder $R_3(0.5)$.
    \item Prove that the Maclaurin series for $f(x)=\sin(x)$ converges to $\sin(x)$ for all $x$.
    \item For the function $f(x) = \sqrt[3]{x}$, use the second-degree Taylor polynomial centered at $a=8$ to approximate $\sqrt[3]{9}$. Use Taylor's Inequality to estimate the error.
\end{enumerate}

\section{Mixed and Advanced Problems}

\subsection{Problems}
\begin{enumerate}
    \setcounter{enumi}{47}
    \item Find the sum of the series $1 - \frac{\pi^2}{2!} + \frac{\pi^4}{4!} - \frac{\pi^6}{6!} + \dots$.
    \item Find the sum of the series $\sum_{n=0}^{\infty} \frac{(-1)^n (\pi/3)^{2n+1}}{(2n+1)!}$.
    \item Find the sum of the series $\sum_{n=0}^{\infty} \frac{2^n}{n!}$.
    \item Use series to evaluate the limit $\lim_{x\to 0} \frac{e^x - 1 - x}{x^2}$.
    \item Use series to evaluate the limit $\lim_{x\to 0} \frac{x - \arctan(x)}{x^3}$.
    \item Find the Maclaurin series for the indefinite integral $\int e^{-x^2} dx$.
    \item Find the first three non-zero terms of the Taylor series for $f(x) = \tan(x)$ centered at $a=0$.
    \item Find the interval of convergence for the Taylor series of $f(x) = \frac{1}{3-x}$ centered at $a=1$.
    \item Use a Taylor polynomial of degree 5 to approximate the value of the definite integral $\int_0^{0.5} \frac{1}{1+x^4} dx$.
    \item If the Taylor series for $f(x)$ centered at $a=2$ is $\sum_{n=0}^{\infty} \frac{(n+1)}{3^n}(x-2)^n$, what is $f^{(4)}(2)$?
    \item Find the sum of the series $1 - \ln(2) + \frac{(\ln 2)^2}{2!} - \frac{(\ln 2)^3}{3!} + \dots$.
    \item Use series to solve the initial value problem $y' - y = 0$ with $y(0)=1$.
    \item Find the Maclaurin series for $f(x) = \sin^2(x)$. (Hint: Use the identity $\sin^2(x) = \frac{1-\cos(2x)}{2}$).
    \item Find the first four non-zero terms of the series for $f(x) = \sec(x)$ by performing long division of $1$ by the series for $\cos(x)$.
\end{enumerate}

\clearpage

\part*{Solutions}

\section{Direct Definition and Basic Concepts}
\begin{enumerate}
    \item The formula for the coefficients of a Taylor series centered at $a$ is $c_n = \frac{f^{(n)}(a)}{n!}$. Here, $a=5$ and we need $c_7$. So, $c_7 = \frac{f^{(7)}(5)}{7!}$.
    \item Here, the center is $a=-2$. The formula for the coefficients is $b_n = \frac{f^{(n)}(-2)}{n!}$. For $n=10$, we have $b_{10} = \frac{f^{(10)}(-2)}{10!}$.
    \item A Maclaurin series is centered at $a=0$. The formula is $a_n = \frac{g^{(n)}(0)}{n!}$. For $a_5$, we get $a_5 = \frac{g^{(5)}(0)}{5!}$.
    \item The general form is $T_3(x) = f(a) + f'(a)(x-a) + \frac{f''(a)}{2!}(x-a)^2 + \frac{f'''(a)}{3!}(x-a)^3$.
    Comparing coefficients with $T_3(x) = 2 - (x-1) + 3(x-1)^2 - 5(x-1)^3$:
    $f(1) = 2$.
    $f'(1) = -1$.
    $\frac{f''(1)}{2!} = 3 \implies f''(1) = 3 \cdot 2! = 6$.
    $\frac{f'''(1)}{3!} = -5 \implies f'''(1) = -5 \cdot 3! = -30$.
    \item The series is $f(x) = \sum \frac{f^{(n)}(0)}{n!}x^n$. The term with $x^3$ is $\frac{f^{(3)}(0)}{3!}x^3$.
    We are given this term is $\frac{5}{3}x^3$.
    So, $\frac{f^{(3)}(0)}{3!} = \frac{5}{3} \implies f^{(3)}(0) = \frac{5}{3} \cdot 3! = \frac{5}{3} \cdot 6 = 10$.
\end{enumerate}

\section{Constructing Series from a Derivative Formula}
\begin{enumerate}
    \setcounter{enumi}{5}
    \item $f(x) = \sum_{n=0}^{\infty} \frac{f^{(n)}(0)}{n!}x^n = \sum_{n=0}^{\infty} \frac{n!}{n!}x^n = \sum_{n=0}^{\infty} x^n$. This is the geometric series.
    Radius of convergence: Using the Ratio Test, $\lim_{n \to \infty} |\frac{x^{n+1}}{x^n}| = |x| < 1$. So $R=1$.
    \item $f(x) = \sum_{n=0}^{\infty} \frac{f^{(n)}(1)}{n!}(x-1)^n = \sum_{n=0}^{\infty} \frac{(-1)^n n!/2^n}{n!}(x-1)^n = \sum_{n=0}^{\infty} \frac{(-1)^n(x-1)^n}{2^n}$.
    Ratio Test: $\lim_{n \to \infty} |\frac{(x-1)^{n+1}/2^{n+1}}{(x-1)^n/2^n}| = \lim_{n \to \infty} |\frac{x-1}{2}| = \frac{|x-1|}{2} < 1 \implies |x-1|<2$. So $R=2$.
    \item $g(x) = \sum_{n=0}^{\infty} \frac{g^{(n)}(0)}{n!}x^n = \sum_{n=0}^{\infty} \frac{(-1)^n(n+1)!/3^n}{n!}x^n = \sum_{n=0}^{\infty} \frac{(-1)^n(n+1)}{3^n}x^n$.
    Ratio Test: $\lim_{n \to \infty} |\frac{(n+2)x^{n+1}/3^{n+1}}{(n+1)x^n/3^n}| = \lim_{n \to \infty} |\frac{x}{3} \frac{n+2}{n+1}| = \frac{|x|}{3} < 1 \implies |x|<3$. So $R=3$.
    \item $h(x) = \sum_{n=0}^{\infty} \frac{h^{(n)}(-3)}{n!}(x+3)^n = \sum_{n=0}^{\infty} \frac{10/(5^n(n+2))}{n!}(x+3)^n = \sum_{n=0}^{\infty} \frac{10(x+3)^n}{5^n n! (n+2)}$.
    Ratio Test: $\lim_{n \to \infty} |\frac{10(x+3)^{n+1}}{5^{n+1}(n+1)!(n+3)} \cdot \frac{5^n n!(n+2)}{10(x+3)^n}| = \lim_{n \to \infty} |\frac{x+3}{5(n+1)} \frac{n+2}{n+3}| = 0$.
    Since the limit is $0 < 1$ for all $x$, the radius of convergence is $R=\infty$.
    \item $f(x) = \sum_{n=0}^{\infty} \frac{f^{(n)}(0)}{n!}x^n = \sum_{n=0}^{\infty} \frac{(2n)!}{n!}x^n$.
    Ratio Test: $\lim_{n \to \infty} |\frac{(2(n+1))! x^{n+1}}{(n+1)!} \cdot \frac{n!}{(2n)! x^n}| = \lim_{n \to \infty} |\frac{(2n+2)(2n+1)x}{n+1}| = \infty$.
    The series converges only if $x=0$. So $R=0$.
\end{enumerate}

% Add solutions for the other sections in a similar, detailed manner.
\section{Constructing Series by Differentiating the Function}
\begin{multicols}{2}
\begin{enumerate}
    \setcounter{enumi}{10}
    \item $f(x) = \cos(x), f(0)=1$. $f'(x) = -\sin(x), f'(0)=0$. $f''(x) = -\cos(x), f''(0)=-1$. $f'''(x) = \sin(x), f'''(0)=0$. $f^{(4)}(x) = \cos(x), f^{(4)}(0)=1$. The pattern of derivatives at 0 is $1, 0, -1, 0, \dots$.
    Series: $1 - \frac{x^2}{2!} + \frac{x^4}{4!} - \dots = \sum_{n=0}^{\infty} \frac{(-1)^n x^{2n}}{(2n)!}$.
    Ratio test shows $R=\infty$.

    \item $f(x) = \ln(x), f(1)=0$. $f'(x) = \frac{1}{x}, f'(1)=1$. $f''(x) = -x^{-2}, f''(1)=-1$. $f'''(x) = 2x^{-3}, f'''(1)=2$. $f^{(n)}(x) = (-1)^{n-1}(n-1)!x^{-n}$. So $f^{(n)}(1)=(-1)^{n-1}(n-1)!$.
    Series: $\sum_{n=1}^{\infty} \frac{(-1)^{n-1}(n-1)!}{n!}(x-1)^n = \sum_{n=1}^{\infty} \frac{(-1)^{n-1}(x-1)^n}{n}$.
    Ratio test shows $R=1$.

    \item $f(x) = e^{-2x}, f(0)=1$. $f'(x) = -2e^{-2x}, f'(0)=-2$. $f''(x) = 4e^{-2x}, f''(0)=4$. $f^{(n)}(x) = (-2)^n e^{-2x}, f^{(n)}(0) = (-2)^n$.
    Series: $\sum_{n=0}^{\infty} \frac{(-2)^n}{n!}x^n = \sum_{n=0}^{\infty} \frac{(-2x)^n}{n!}$.
    Ratio test shows $R=\infty$.

    \item $f(x) = x^{-1}, f(2)=1/2$. $f'(x)=-x^{-2}, f'(2)=-1/4$. $f''(x)=2x^{-3}, f''(2)=2/8$. $f^{(n)}(2) = \frac{(-1)^n n!}{2^{n+1}}$.
    Series: $\sum_{n=0}^{\infty} \frac{(-1)^n n!/2^{n+1}}{n!}(x-2)^n = \sum_{n=0}^{\infty} \frac{(-1)^n(x-2)^n}{2^{n+1}}$.
    Ratio test shows $R=2$.

    \item Using known series: $e^x = \sum \frac{x^n}{n!}$, $e^{-x} = \sum \frac{(-x)^n}{n!}$.
    $\sinh(x) = \frac{1}{2}(\sum \frac{x^n}{n!} - \sum \frac{(-1)^n x^n}{n!}) = \frac{1}{2} \sum \frac{(1-(-1)^n)x^n}{n!}$.
    If n is even, $1-1=0$. If n is odd, $1-(-1)=2$.
    Series: $\sum_{k=0}^{\infty} \frac{x^{2k+1}}{(2k+1)!}$. $R=\infty$.

    \item $f(x)=x^{1/2}, f(4)=2$. $f'(x)=\frac{1}{2}x^{-1/2}, f'(4)=\frac{1}{4}$. $f''(x)=-\frac{1}{4}x^{-3/2}, f''(4)=-\frac{1}{32}$. $f'''(x)=\frac{3}{8}x^{-5/2}, f'''(4)=\frac{3}{256}$.
    $T_3(x) = 2 + \frac{1}{4}(x-4) - \frac{1/32}{2!}(x-4)^2 + \frac{3/256}{3!}(x-4)^3 = 2 + \frac{1}{4}(x-4) - \frac{1}{64}(x-4)^2 + \frac{1}{512}(x-4)^3$.

    \item This is the standard geometric series, $\sum_{n=0}^\infty x^n$.

    \item We know $\sin(u) = \sum_{n=0}^\infty \frac{(-1)^n u^{2n+1}}{(2n+1)!}$. Let $u=3x$.
    $\sin(3x) = \sum_{n=0}^\infty \frac{(-1)^n (3x)^{2n+1}}{(2n+1)!} = \sum_{n=0}^\infty \frac{(-1)^n 3^{2n+1} x^{2n+1}}{(2n+1)!}$.

    \item A polynomial is its own Taylor series. We just need to rewrite it in powers of $(x-1)$.
    Let $u=x-1 \implies x=u+1$.
    $(u+1)^3 - 2(u+1) + 4 = (u^3+3u^2+3u+1) - (2u+2) + 4 = u^3+3u^2+u+3$.
    $f(x) = 3 + (x-1) + 3(x-1)^2 + (x-1)^3$.

    \item We know $\frac{1}{1-u} = \sum_{n=0}^\infty u^n$. Let $u=-x$.
    $\frac{1}{1+x} = \sum_{n=0}^\infty (-x)^n = \sum_{n=0}^\infty (-1)^n x^n$.
    $\frac{d}{dx}(\frac{1}{1+x}) = \frac{-1}{(1+x)^2}$.
    So $\frac{1}{(1+x)^2} = -\frac{d}{dx}\sum_{n=0}^\infty (-1)^n x^n = -\sum_{n=1}^\infty (-1)^n n x^{n-1} = \sum_{n=1}^\infty (-1)^{n+1} n x^{n-1}$.
    Re-index with $k=n-1$: $\sum_{k=0}^\infty (-1)^{k+2}(k+1)x^k = \sum_{k=0}^\infty (-1)^k(k+1)x^k$.
\end{enumerate}
\end{multicols}

% Other solutions sections follow
\section{Constructing Series by Manipulating Known Series}
% Solutions 21-32
\begin{enumerate}
    \setcounter{enumi}{20}
    \item $e^u = \sum_{n=0}^{\infty} \frac{u^n}{n!}$. Let $u=x^2$. $e^{x^2} = \sum_{n=0}^{\infty} \frac{(x^2)^n}{n!} = \sum_{n=0}^{\infty} \frac{x^{2n}}{n!}$. Radius of convergence is $R=\infty$.
    \item $\cos(u) = \sum_{n=0}^{\infty} \frac{(-1)^n u^{2n}}{(2n)!}$. Let $u=\sqrt{x}$. $\cos(\sqrt{x}) = \sum_{n=0}^{\infty} \frac{(-1)^n (\sqrt{x})^{2n}}{(2n)!} = \sum_{n=0}^{\infty} \frac{(-1)^n x^n}{(2n)!}$.
    \item $\frac{1}{1-u} = \sum_{n=0}^{\infty} u^n$. Let $u=-x^2$. $\frac{1}{1+x^2} = \sum_{n=0}^{\infty} (-x^2)^n = \sum_{n=0}^{\infty} (-1)^n x^{2n}$. Radius of convergence is $|-x^2|<1 \implies |x|<1$, so $R=1$.
    \item $\arctan(x) = \int \frac{1}{1+x^2} dx = \int \sum_{n=0}^{\infty} (-1)^n x^{2n} dx = C + \sum_{n=0}^{\infty} \frac{(-1)^n x^{2n+1}}{2n+1}$. Since $\arctan(0)=0$, $C=0$. So $\arctan(x) = \sum_{n=0}^{\infty} \frac{(-1)^n x^{2n+1}}{2n+1}$.
    \item $\sin(x) = \sum_{n=0}^{\infty} \frac{(-1)^n x^{2n+1}}{(2n+1)!}$. $x^2 \sin(x) = x^2 \sum_{n=0}^{\infty} \frac{(-1)^n x^{2n+1}}{(2n+1)!} = \sum_{n=0}^{\infty} \frac{(-1)^n x^{2n+3}}{(2n+1)!}$.
    \item $\frac{1}{1-u} = \sum_{n=0}^\infty u^n$. Let $u=2x$. $\frac{1}{1-2x} = \sum_{n=0}^\infty (2x)^n = \sum_{n=0}^\infty 2^n x^n$. So, $\frac{x}{1-2x} = x \sum_{n=0}^\infty 2^n x^n = \sum_{n=0}^\infty 2^n x^{n+1}$.
    \item $\sin(x) = x - \frac{x^3}{3!} + \frac{x^5}{5!} - \dots$. Differentiating term-by-term: $1 - \frac{3x^2}{3!} + \frac{5x^4}{5!} - \dots = 1 - \frac{x^2}{2!} + \frac{x^4}{4!} - \dots = \cos(x)$.
    \item $\ln(1-u) = -\sum_{n=1}^\infty \frac{u^n}{n}$. Let $u=x^2$. $\ln(1-x^2) = -\sum_{n=1}^\infty \frac{(x^2)^n}{n} = -\sum_{n=1}^\infty \frac{x^{2n}}{n}$.
    \item $e^x = 1+x+\frac{x^2}{2}+\dots$, $\cos(x) = 1-\frac{x^2}{2}+\frac{x^4}{24}-\dots$.
    $e^x \cos(x) = (1+x+\frac{x^2}{2}+\dots)(1-\frac{x^2}{2}+\dots) = 1(1-\frac{x^2}{2}) + x(1) + \frac{x^2}{2}(1) + \dots = 1+x-\frac{x^2}{2}+\frac{x^2}{2}+\dots = 1+x+0x^2+\dots = 1+x - \frac{x^3}{3} + \dots$ (Need to expand further for $x^2, x^3$ terms). Correct expansion: $1(1-\frac{x^2}{2}) + x(1) + \frac{x^2}{2}(1) = 1+x$. $x^3$ term: $x(-\frac{x^2}{2}) + \frac{x^3}{6}(1) = -\frac{x^3}{2}+\frac{x^3}{6}=-\frac{x^3}{3}$. Result: $1+x-\frac{x^3}{3} + \dots$.
    \item $\sin(x) = \sum_{n=0}^{\infty} \frac{(-1)^n x^{2n+1}}{(2n+1)!} = x - \frac{x^3}{3!} + \dots$. $\frac{\sin(x)}{x} = \sum_{n=0}^{\infty} \frac{(-1)^n x^{2n}}{(2n+1)!} = 1 - \frac{x^2}{3!} + \frac{x^4}{5!} - \dots$. $\lim_{x\to 0} (1 - \frac{x^2}{3!} + \dots) = 1$.
    \item $\cos(u) = \sum_{n=0}^{\infty} \frac{(-1)^n u^{2n}}{(2n)!}$. Let $u=x^3$. $\cos(x^3) = \sum_{n=0}^{\infty} \frac{(-1)^n (x^3)^{2n}}{(2n)!} = \sum_{n=0}^{\infty} \frac{(-1)^n x^{6n}}{(2n)!}$. $\int \cos(x^3) dx = C + \sum_{n=0}^{\infty} \frac{(-1)^n x^{6n+1}}{(2n)!(6n+1)}$.
    \item $\cosh(x) = \frac{1}{2}(e^x+e^{-x}) = \frac{1}{2}(\sum \frac{x^n}{n!} + \sum \frac{(-x)^n}{n!}) = \frac{1}{2}\sum \frac{(1+(-1)^n)x^n}{n!}$. For odd n, terms are 0. For even n, terms are $2x^n/n!$. So $\cosh(x) = \sum_{k=0}^\infty \frac{x^{2k}}{(2k)!}$.
\end{enumerate}

\section{Binomial Series}
% Solutions 33-36
\begin{enumerate}
    \setcounter{enumi}{32}
    \item $f(x)=(1+x)^{-1/2}$. Here $k=-1/2$. The series is $1 + \sum_{n=1}^\infty \frac{(-\frac{1}{2})(-\frac{3}{2})\cdots(-\frac{1}{2}-n+1)}{n!} x^n = 1 + \sum_{n=1}^\infty \frac{(-1)^n 1\cdot 3\cdots(2n-1)}{2^n n!} x^n$. $R=1$.
    \item $f(x)=(1+(-x))^{1/3}$. Here $k=1/3$. $T_3(x) = 1 + \frac{1}{3}(-x) + \frac{\frac{1}{3}(-\frac{2}{3})}{2!}(-x)^2 + \frac{\frac{1}{3}(-\frac{2}{3})(-\frac{5}{3})}{3!}(-x)^3 = 1 - \frac{1}{3}x - \frac{1}{9}x^2 - \frac{5}{81}x^3$.
    \item $f(x)=(1+x^2)^{-1/2}$. Use result from Q33, substitute $x$ with $x^2$. $1 + \sum_{n=1}^\infty \frac{(-1)^n 1\cdot 3\cdots(2n-1)}{2^n n!} (x^2)^n = 1 + \sum_{n=1}^\infty \frac{(-1)^n 1\cdot 3\cdots(2n-1)}{2^n n!} x^{2n}$.
    \item From Q35, the series for $(1-u^2)^{-1/2}$ is $1 + \sum_{n=1}^\infty \frac{1\cdot 3\cdots(2n-1)}{2^n n!} u^{2n}$. Integrate term by term: $\arcsin(x) = C + x + \sum_{n=1}^\infty \frac{1\cdot 3\cdots(2n-1)}{2^n n! (2n+1)} x^{2n+1}$. Since $\arcsin(0)=0$, $C=0$.
\end{enumerate}


\section{Radius and Interval of Convergence}
% Solutions 37-42
\begin{enumerate}
    \setcounter{enumi}{36}
    \item Ratio test: $\lim_{n \to \infty} |\frac{(x-2)^{n+1}/(n+1)!}{(x-2)^n/n!}| = \lim_{n \to \infty} |\frac{x-2}{n+1}| = 0$. $R=\infty$, Interval: $(-\infty, \infty)$.
    \item Ratio test: $\lim_{n \to \infty} |\frac{(x+1)^{n+1}/((n+1)3^{n+1})}{(x+1)^n/(n \cdot 3^n)}| = \lim_{n \to \infty} |\frac{x+1}{3} \frac{n}{n+1}| = \frac{|x+1|}{3} < 1 \implies |x+1|<3$. $R=3$. Interval: $(-4, 2)$. Test endpoints: $x=2 \implies \sum \frac{3^n}{n 3^n} = \sum \frac{1}{n}$ (diverges). $x=-4 \implies \sum \frac{(-3)^n}{n 3^n} = \sum \frac{(-1)^n}{n}$ (converges). Interval: $[-4, 2)$.
    \item Ratio test: $\lim_{n \to \infty} |\frac{(n+1)!(2x-1)^{n+1}}{n!(2x-1)^n}| = \lim_{n \to \infty} |(n+1)(2x-1)| = \infty$ unless $x=1/2$. $R=0$. Interval: $\{1/2\}$.
    \item Ratio test: $\lim_{n \to \infty} |\frac{x^{n+1}/\sqrt{n+1}}{x^n/\sqrt{n}}| = |x|\lim_{n \to \infty} \sqrt{\frac{n}{n+1}} = |x| < 1$. $R=1$. Interval: $(-1,1)$. Test endpoints: $x=1 \implies \sum \frac{(-1)^n}{\sqrt{n}}$ (converges). $x=-1 \implies \sum \frac{(-1)^n(-1)^n}{\sqrt{n}} = \sum \frac{1}{\sqrt{n}}$ (diverges, p-series). Interval: $(-1, 1]$.
    \item Root test: $\lim_{n \to \infty} \sqrt[n]{|\frac{x^{2n}}{4^n}|} = \lim_{n \to \infty} \frac{|x^2|}{4} = \frac{x^2}{4} < 1 \implies x^2 < 4 \implies |x|<2$. $R=2$. Interval: $(-2,2)$. Test endpoints: $x=2 \implies \sum \frac{4^n}{4^n} = \sum 1$ (diverges). $x=-2 \implies \sum \frac{(-2)^{2n}}{4^n} = \sum \frac{4^n}{4^n} = \sum 1$ (diverges). Interval: $(-2, 2)$.
    \item Ratio test: $\lim_{n \to \infty} |\frac{(4x-5)^{n+1}/(n+1)^2}{(4x-5)^n/n^2}| = |4x-5| \lim_{n \to \infty} (\frac{n}{n+1})^2 = |4x-5|<1$. $|4(x-5/4)|<1 \implies |x-5/4|<1/4$. $R=1/4$. Interval: $(1, 3/2)$. Test endpoints: $x=3/2 \implies \sum \frac{(1)^n}{n^2}$ (converges, p-series). $x=1 \implies \sum \frac{(-1)^n}{n^2}$ (converges). Interval: $[1, 3/2]$.
\end{enumerate}

\section{Taylor's Inequality and Error Estimation}
% Solutions 43-47
\begin{enumerate}
    \setcounter{enumi}{42}
    \item $T_3(x) = 1+x+\frac{x^2}{2!}+\frac{x^3}{3!}$. $e^{0.1} \approx 1+0.1+\frac{0.01}{2}+\frac{0.001}{6} \approx 1.105166...$. Error: $R_3(0.1)$. $f^{(4)}(x)=e^x$. On $[0, 0.1]$, $e^x$ is increasing, so $|f^{(4)}(x)| \le e^{0.1} < e < 3$. Let $M=3$. $|R_3(0.1)| \le \frac{3}{4!} (0.1)^4 = \frac{3}{24}(0.0001) = 0.0000125$.
    \item We need $|R_n(0.2)| \le 0.0001$. For $f(x)=\cos(x)$, $|f^{(n+1)}(x)|$ is either $|\sin x|$ or $|\cos x|$, so $|f^{(n+1)}(x)| \le 1$. Let $M=1$. We need $\frac{1}{(n+1)!}|0.2|^{n+1} \le 0.0001$.
    $n=1: \frac{0.2^2}{2} = 0.02$. $n=2: \frac{0.2^3}{6} \approx 0.0013$. $n=3: \frac{0.2^4}{24} \approx 0.000067 < 0.0001$. So $n=3$ is sufficient.
    \item $f(x)=\ln(1+x)$, $T_3(x) = x-\frac{x^2}{2}+\frac{x^3}{3}$. $\ln(1.5) = f(0.5) \approx 0.5 - \frac{0.5^2}{2} + \frac{0.5^3}{3} \approx 0.5-0.125+0.04166=0.41666$.
    $f^{(4)}(x) = -6(1+x)^{-4}$. On $[0, 0.5]$, $|f^{(4)}(x)| = \frac{6}{(1+x)^4}$ is decreasing. Max value is at $x=0$, $M=6$. $|R_3(0.5)| \le \frac{6}{4!}|0.5|^4 = \frac{6}{24}(0.0625) = 0.015625$.
    \item For $f(x)=\sin(x)$, $|f^{(n+1)}(x)| \le 1$ for all $x,n$. Let $M=1$. By Taylor's Inequality, $|R_n(x)| \le \frac{1}{(n+1)!}|x|^{n+1}$. For any fixed $x$, $\lim_{n \to \infty} \frac{|x|^{n+1}}{(n+1)!} = 0$. Since the remainder goes to 0, the series converges to $\sin(x)$.
    \item $f(x)=x^{1/3}, a=8$. $f(8)=2, f'(8)=1/12, f''(8)=-1/144$. $T_2(x) = 2+\frac{1}{12}(x-8)-\frac{1}{288}(x-8)^2$. $\sqrt[3]{9} \approx T_2(9) = 2+\frac{1}{12}-\frac{1}{288} \approx 2.080$. $f'''(x)=\frac{10}{27}x^{-8/3}$. On $[8,9]$, $|f'''(x)|=\frac{10}{27x^{8/3}}$ is decreasing. Max is at $x=8$, $M=\frac{10}{27 \cdot 8^{8/3}} = \frac{10}{27 \cdot 256} = \frac{5}{3456}$. $|R_2(9)| \le \frac{5/3456}{3!}|9-8|^3 = \frac{5}{20736} \approx 0.00024$.
\end{enumerate}

\section{Mixed and Advanced Problems}
\begin{enumerate}
    \setcounter{enumi}{47}
    \item This is the series for $\cos(x) = \sum \frac{(-1)^n x^{2n}}{(2n)!}$ with $x=\pi$. So the sum is $\cos(\pi) = -1$.
    \item This is the series for $\sin(x)$ with $x=\pi/3$. The sum is $\sin(\pi/3) = \sqrt{3}/2$.
    \item This is the series for $e^x = \sum \frac{x^n}{n!}$ with $x=2$. The sum is $e^2$.
    \item $e^x=1+x+\frac{x^2}{2!}+\frac{x^3}{3!}+\dots$. $\lim_{x\to 0} \frac{(1+x+\frac{x^2}{2}+\dots) - 1 - x}{x^2} = \lim_{x\to 0} \frac{\frac{x^2}{2}+\frac{x^3}{6}+\dots}{x^2} = \lim_{x\to 0} (\frac{1}{2}+\frac{x}{6}+\dots) = \frac{1}{2}$.
    \item $\arctan(x)=x-\frac{x^3}{3}+\frac{x^5}{5}-\dots$. $\lim_{x\to 0} \frac{x-(x-\frac{x^3}{3}+\dots)}{x^3} = \lim_{x\to 0} \frac{\frac{x^3}{3}-\frac{x^5}{5}+\dots}{x^3} = \lim_{x\to 0} (\frac{1}{3}-\frac{x^2}{5}+\dots) = \frac{1}{3}$.
    \item $e^{-x^2} = \sum_{n=0}^\infty \frac{(-x^2)^n}{n!} = \sum_{n=0}^\infty \frac{(-1)^n x^{2n}}{n!}$. $\int e^{-x^2}dx = C + \sum_{n=0}^\infty \frac{(-1)^n x^{2n+1}}{n!(2n+1)}$.
    \item $f(x)=\tan x, f(0)=0$. $f'(x)=\sec^2 x, f'(0)=1$. $f''(x)=2\sec^2 x \tan x, f''(0)=0$. $f'''(x)=4\sec^2 x \tan^2 x + 2\sec^4 x, f'''(0)=2$. $T_3(x) = x + \frac{2}{3!}x^3 = x + \frac{x^3}{3}$.
    \item $f(x) = \frac{1}{3-x} = \frac{1}{2-(x-1)} = \frac{1/2}{1-\frac{x-1}{2}}$. This is a geometric series with $r=\frac{x-1}{2}$. Converges for $|\frac{x-1}{2}|<1 \implies |x-1|<2$. Interval is $(-1, 3)$.
    \item $\frac{1}{1+u} = 1-u+u^2-u^3+\dots$. Let $u=x^4$. $\frac{1}{1+x^4} = 1-x^4+x^8-\dots$. $\int_0^{0.5} (1-x^4) dx = [x-\frac{x^5}{5}]_0^{0.5} = 0.5 - \frac{0.5^5}{5} = 0.5 - \frac{0.03125}{5} = 0.5 - 0.00625 = 0.49375$.
    \item Coefficient of $(x-2)^n$ is $c_n = \frac{f^{(n)}(2)}{n!}$. We are given $c_n = \frac{n+1}{3^n}$. For $n=4$, $c_4 = \frac{f^{(4)}(2)}{4!} = \frac{4+1}{3^4} = \frac{5}{81}$. So $f^{(4)}(2) = \frac{5}{81} \cdot 4! = \frac{5 \cdot 24}{81} = \frac{40}{27}$.
    \item This is the series for $e^x$ with $x=-\ln(2)$. Sum is $e^{-\ln(2)} = e^{\ln(2^{-1})} = 2^{-1} = 1/2$.
    \item Assume $y=\sum_{n=0}^\infty c_n x^n$. Then $y'=\sum_{n=1}^\infty n c_n x^{n-1}$. $y'-y = \sum_{n=1}^\infty n c_n x^{n-1} - \sum_{n=0}^\infty c_n x^n = 0$. Re-index first sum: $\sum_{k=0}^\infty (k+1)c_{k+1}x^k - \sum_{k=0}^\infty c_k x^k=0$. This gives $(k+1)c_{k+1} - c_k=0$, or $c_{k+1}=\frac{c_k}{k+1}$. $y(0)=1 \implies c_0=1$. Then $c_1=c_0/1=1, c_2=c_1/2=1/2, c_3=c_2/3=1/6$. So $c_n=1/n!$. $y=\sum \frac{x^n}{n!} = e^x$.
    \item $\sin^2(x) = \frac{1}{2}(1-\cos(2x))$. $\cos(u)=\sum_{n=0}^\infty \frac{(-1)^n u^{2n}}{(2n)!}$. $\cos(2x)=\sum_{n=0}^\infty \frac{(-1)^n (2x)^{2n}}{(2n)!} = 1 - \frac{4x^2}{2!} + \frac{16x^4}{4!} - \dots$.
    $\sin^2(x) = \frac{1}{2}(1 - (1 - \frac{4x^2}{2!} + \frac{16x^4}{4!} - \dots)) = \frac{1}{2}(\frac{4x^2}{2!} - \frac{16x^4}{4!} + \dots) = \sum_{n=1}^\infty \frac{(-1)^{n+1} 2^{2n-1} x^{2n}}{(2n)!}$.
    \item $\sec(x) = \frac{1}{\cos x} = \frac{1}{1-\frac{x^2}{2}+\frac{x^4}{24}-\dots}$. Long division gives $1+\frac{x^2}{2}+\frac{5x^4}{24}+\frac{61x^6}{720}+\dots$.
\end{enumerate}

\clearpage

\part*{Concept Checklist and Problem Index}

\begin{multicols}{2}
\begin{itemize}
    \item \textbf{Direct Definition and Theory}
    \begin{itemize}
        \item Formula for Taylor/Maclaurin coefficients: 1, 2, 3
        \item Relating polynomial coefficients to derivatives: 4, 5, 57
    \end{itemize}

    \item \textbf{Constructing Series}
    \begin{itemize}
        \item From a given derivative formula: 6, 7, 8, 9, 10
        \item From scratch by differentiating a function: 11, 12, 13, 14, 16, 17, 19
        \item By substitution into a known series: 18, 21, 22, 23
        \item By differentiation of a known series: 20, 27
        \item By integration of a known series: 24, 28, 31, 36, 54
        \item By algebraic manipulation (multiplication, division, addition): 15, 25, 26, 29, 30, 32, 59, 60, 61
    \end{itemize}

    \item \textbf{Binomial Series}
    \begin{itemize}
        \item Basic expansion: 33, 34
        \item With substitution: 35
        \item Application (integration): 36
    \end{itemize}

    \item \textbf{Convergence Analysis}
    \begin{itemize}
        \item Using Ratio/Root Test for Radius of Convergence: 6, 7, 8, 9, 10, 11, 12, 13, 14, 21, 23
        \item Finding the full Interval of Convergence (testing endpoints): 37, 38, 39, 40, 41, 42, 55
    \end{itemize}

    \item \textbf{Applications and Error Analysis}
    \begin{itemize}
        \item Taylor's Inequality to bound error/remainder: 43, 44, 45, 47
        \item Proving a series converges to its function: 46
        \item Finding the sum of a known series: 48, 49, 50, 58
        \item Using series to evaluate limits: 30, 51, 52
        \item Using series for approximation (integrals, function values): 43, 45, 47, 56
        \item Using series to solve differential equations: 59
    \end{itemize}
\end{itemize}
\end{multicols}

\end{document}