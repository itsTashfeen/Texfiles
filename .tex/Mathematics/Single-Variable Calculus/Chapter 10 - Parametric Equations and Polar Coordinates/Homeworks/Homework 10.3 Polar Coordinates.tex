\documentclass{article}
\usepackage{amsmath}
\usepackage{amssymb}
\usepackage[margin=1in]{geometry}
\usepackage{graphicx}
\usepackage{amsfonts}
\usepackage{enumitem}
\usepackage{array}

\title{Homework 10.3 Polar Coordinates}
\author{Tashfeen Omran}
\date{\today}

\begin{document}

\maketitle

\part{Comprehensive Introduction, Context, and Prerequisites}

\section{Core Concepts}

The Cartesian coordinate system, which you know as the familiar \((x, y)\) grid, is excellent for describing positions in terms of horizontal and vertical distances. However, it is not always the most convenient system. For phenomena that are centered around a point, such as rotations, orbits, or waves radiating outwards, a different system is often more natural: the \textbf{polar coordinate system}.

Instead of a grid of perpendicular lines, the polar system is based on a single point, called the \textbf{pole} (which corresponds to the origin in the Cartesian system), and a single ray extending from it, called the \textbf{polar axis} (which corresponds to the positive x-axis).

A point \(P\) in the plane is described by an ordered pair \((r, \theta)\), where:
\begin{itemize}
    \item \(r\) is the \textbf{radial coordinate}, representing the directed distance from the pole to the point \(P\).
    \item \(\theta\) is the \textbf{angular coordinate} (or polar angle), representing the angle measured from the polar axis to the line segment connecting the pole and \(P\). By convention, positive angles are measured counter-clockwise, and negative angles are measured clockwise.
\end{itemize}

An important feature of polar coordinates is that the representation of a point is not unique. For instance, the point \((2, \pi/4)\) can also be represented as \((2, \pi/4 + 2\pi)\) or \((2, \pi/4 - 2\pi)\). Furthermore, a negative value for \(r\) is allowed. A negative \(r\) means we move in the direction opposite to the terminal ray of the angle \(\theta\). For example, the point \((-2, \pi/4)\) is found by facing the direction \(\pi/4\) and moving 2 units backward, which lands you at the same spot as \((2, 5\pi/4)\).

\section{Intuition and Derivation}

The relationship between Cartesian and polar coordinates is derived directly from right-triangle trigonometry. Imagine a point \(P\) in the plane with Cartesian coordinates \((x, y)\) and polar coordinates \((r, \theta)\).

\begin{center}
\includegraphics[width=0.4\textwidth]{polar_cartesian_diagram.png} 
% A diagram showing a point P with coordinates (x,y) and (r,theta) would be here.
% The diagram would show a right triangle with vertices at the origin (0,0), (x,0), and (x,y).
% The hypotenuse would be labeled 'r', the angle at the origin 'theta', the horizontal side 'x', and the vertical side 'y'.
\end{center}

By placing the pole at the Cartesian origin and aligning the polar axis with the positive x-axis, we can form a right triangle with sides \(x\) and \(y\), and hypotenuse \(r\).

\begin{itemize}
    \item \textbf{Polar to Cartesian:} From the definitions of sine and cosine in a right triangle:
    \[ \cos(\theta) = \frac{\text{adjacent}}{\text{hypotenuse}} = \frac{x}{r} \implies x = r \cos(\theta) \]
    \[ \sin(\theta) = \frac{\text{opposite}}{\text{hypotenuse}} = \frac{y}{r} \implies y = r \sin(\theta) \]
    These are the fundamental formulas for converting from polar to Cartesian coordinates.

    \item \textbf{Cartesian to Polar:} Using the same triangle:
    The Pythagorean theorem gives us the relationship for \(r\):
    \[ x^2 + y^2 = r^2 \implies r = \sqrt{x^2 + y^2} \]
    The tangent function gives us the relationship for \(\theta\):
    \[ \tan(\theta) = \frac{\text{opposite}}{\text{adjacent}} = \frac{y}{x} \]
    When finding \(\theta\) from \(\tan(\theta) = y/x\), we must be careful to place \(\theta\) in the correct quadrant based on the signs of \(x\) and \(y\).
\end{itemize}

\section{Historical Context and Motivation}

The concepts of angle and radius have been used since antiquity, with the Greek astronomer Hipparchus using them to establish stellar positions. However, the formal development of a coordinate system based on these ideas came much later. In the mid-17th century, mathematicians Grégoire de Saint-Vincent and Bonaventura Cavalieri independently introduced concepts similar to polar coordinates. The primary motivation was the study of motion that was inherently circular or orbital. Describing the path of a planet around the sun using a Cartesian \((x, y)\) grid is algebraically cumbersome. It is far more intuitive to describe it by its distance from the sun (\(r\)) and its angle of revolution (\(\theta\)).

Isaac Newton, in his \textit{Method of Fluxions} (written around 1671, but published later), was among the first to formally treat polar coordinates as one of several possible systems for describing curves. The term "polar coordinates" itself is often attributed to the Italian mathematician Gregorio Fontana in the 18th century. This system provided a powerful new tool for analyzing spirals, cardioids, and other curves that were difficult to express with Cartesian equations.

\section{Key Formulas}
\begin{itemize}
    \item \textbf{Conversion from Polar to Cartesian:}
    \[ x = r \cos(\theta) \]
    \[ y = r \sin(\theta) \]

    \item \textbf{Conversion from Cartesian to Polar:}
    \[ r^2 = x^2 + y^2 \quad (\text{so } r = \sqrt{x^2 + y^2}) \]
    \[ \tan(\theta) = \frac{y}{x} \quad (\text{be careful with the quadrant!}) \]

    \item \textbf{Useful Identities for Converting Equations:}
    To convert an equation from polar to Cartesian, try to create expressions of the form \(r\cos(\theta)\), \(r\sin(\theta)\), or \(r^2\).
    To convert an equation from Cartesian to polar, substitute \(x\) and \(y\) with their polar equivalents.
\end{itemize}

\section{Prerequisites}
To succeed with polar coordinates, you must have a solid foundation in the following:
\begin{itemize}
    \item \textbf{Trigonometry:} Deep familiarity with the unit circle is essential. You must know the sine, cosine, and tangent of common angles (\(\pi/6, \pi/4, \pi/3, \pi/2\), etc.) instantly. You should also be comfortable with basic trigonometric identities, especially double-angle formulas like \(\cos(2\theta) = \cos^2(\theta) - \sin^2(\theta)\).
    \item \textbf{Algebra:} You need to be proficient in manipulating equations, including completing the square, simplifying expressions, and working with radicals.
    \item \textbf{Graphing:} You should have a strong intuition for the Cartesian plane, including how to plot points and recognize the equations of lines and circles.
\end{itemize}

\part{Detailed Homework Solutions}

\subsection*{Problems 1-2: Plotting Points and Finding Alternative Coordinates}

\subsubsection*{Problem 1a: (1, \(\pi/4\))}
\textbf{Plot:} To plot this point, first find the angle \(\theta = \pi/4\) (45 degrees counter-clockwise from the polar axis). Then, move a distance of \(r=1\) unit out along this line.

\textbf{Alternative Coordinates:}
\begin{itemize}
    \item \textbf{With \(r > 0\):} We can add a full revolution (\(2\pi\)) to the angle.
    \[ \theta' = \frac{\pi}{4} + 2\pi = \frac{\pi}{4} + \frac{8\pi}{4} = \frac{9\pi}{4} \]
    So, another pair of coordinates is \((1, 9\pi/4)\).

    \item \textbf{With \(r < 0\):} We make \(r\) negative and add \(\pi\) to the angle to point in the opposite direction.
    \[ r' = -1 \]
    \[ \theta' = \frac{\pi}{4} + \pi = \frac{\pi}{4} + \frac{4\pi}{4} = \frac{5\pi}{4} \]
    So, another pair of coordinates is \((-1, 5\pi/4)\).
\end{itemize}

\subsubsection*{Problem 1b: (\(-2, 3\pi/2\))}
\textbf{Plot:} First, find the angle \(\theta = 3\pi/2\) (270 degrees, pointing straight down). Because \(r=-2\) is negative, we move 2 units in the opposite direction, which is straight up. This point is at the same location as \((2, \pi/2)\).

\textbf{Alternative Coordinates:}
\begin{itemize}
    \item \textbf{With \(r > 0\):} As determined above, this point is equivalent to having a positive radius and pointing in the opposite direction of \(3\pi/2\), which is \(\pi/2\).
    \[ r' = 2 \]
    \[ \theta' = \frac{3\pi}{2} - \pi = \frac{\pi}{2} \]
    So, a pair with \(r>0\) is \((2, \pi/2)\).

    \item \textbf{With \(r < 0\):} We can keep \(r=-2\) and find a coterminal angle for \(3\pi/2\) by subtracting \(2\pi\).
    \[ \theta' = \frac{3\pi}{2} - 2\pi = \frac{3\pi}{2} - \frac{4\pi}{2} = -\frac{\pi}{2} \]
    So, another pair is \((-2, -\pi/2)\).
\end{itemize}

\subsubsection*{Problem 1c: (3, \(-\pi/3\))}
\textbf{Plot:} The angle \(\theta = -\pi/3\) is found by moving 60 degrees clockwise from the polar axis. Then, move a distance of \(r=3\) units out along this line.

\textbf{Alternative Coordinates:}
\begin{itemize}
    \item \textbf{With \(r > 0\):} Find a positive coterminal angle by adding \(2\pi\).
    \[ \theta' = -\frac{\pi}{3} + 2\pi = -\frac{\pi}{3} + \frac{6\pi}{3} = \frac{5\pi}{3} \]
    So, another pair is \((3, 5\pi/3)\).

    \item \textbf{With \(r < 0\):} Make \(r\) negative and add \(\pi\) to the angle.
    \[ r' = -3 \]
    \[ \theta' = -\frac{\pi}{3} + \pi = -\frac{\pi}{3} + \frac{3\pi}{3} = \frac{2\pi}{3} \]
    So, another pair is \((-3, 2\pi/3)\).
\end{itemize}

\subsubsection*{Problem 2a: (2, \(5\pi/6\))}
\textbf{Plot:} Find the angle \(\theta = 5\pi/6\) (150 degrees). Move a distance of \(r=2\) units out.

\textbf{Alternative Coordinates:}
\begin{itemize}
    \item \textbf{With \(r > 0\):} Subtract \(2\pi\).
    \[ \theta' = \frac{5\pi}{6} - 2\pi = \frac{5\pi}{6} - \frac{12\pi}{6} = -\frac{7\pi}{6} \]
    So, \((2, -7\pi/6)\).

    \item \textbf{With \(r < 0\):} Make \(r\) negative and add \(\pi\).
    \[ r' = -2 \]
    \[ \theta' = \frac{5\pi}{6} + \pi = \frac{11\pi}{6} \]
    So, \((-2, 11\pi/6)\).
\end{itemize}

\subsubsection*{Problem 2b: (1, \(-2\pi/3\))}
\textbf{Plot:} Find the angle \(\theta = -2\pi/3\) (-120 degrees). Move a distance of \(r=1\) unit out.

\textbf{Alternative Coordinates:}
\begin{itemize}
    \item \textbf{With \(r > 0\):} Add \(2\pi\).
    \[ \theta' = -\frac{2\pi}{3} + 2\pi = \frac{4\pi}{3} \]
    So, \((1, 4\pi/3)\).

    \item \textbf{With \(r < 0\):} Make \(r\) negative and add \(\pi\).
    \[ r' = -1 \]
    \[ \theta' = -\frac{2\pi}{3} + \pi = \frac{\pi}{3} \]
    So, \((-1, \pi/3)\).
\end{itemize}

\subsubsection*{Problem 2c: (\(-1, 5\pi/4\))}
\textbf{Plot:} Find the angle \(\theta = 5\pi/4\) (225 degrees). Since \(r=-1\), move 1 unit in the opposite direction, which corresponds to the angle \(\pi/4\).

\textbf{Alternative Coordinates:}
\begin{itemize}
    \item \textbf{With \(r > 0\):} The point is at a distance of 1 in the direction opposite to \(5\pi/4\).
    \[ r' = 1 \]
    \[ \theta' = \frac{5\pi}{4} - \pi = \frac{\pi}{4} \]
    So, \((1, \pi/4)\).

    \item \textbf{With \(r < 0\):} Keep \(r=-1\) and find a coterminal angle.
    \[ \theta' = \frac{5\pi}{4} - 2\pi = -\frac{3\pi}{4} \]
    So, \((-1, -3\pi/4)\).
\end{itemize}

\subsection*{Problems 3-4: Converting to Cartesian Coordinates}

\subsubsection*{Problem 3a: (2, \(3\pi/2\))}
Use the conversion formulas \(x = r \cos(\theta)\) and \(y = r \sin(\theta)\).
\[ x = 2 \cos\left(\frac{3\pi}{2}\right) = 2 \cdot 0 = 0 \]
\[ y = 2 \sin\left(\frac{3\pi}{2}\right) = 2 \cdot (-1) = -2 \]
\textbf{Cartesian coordinates: (0, -2)}

\subsubsection*{Problem 3b: (\(\sqrt{2}, \pi/4\))}
\[ x = \sqrt{2} \cos\left(\frac{\pi}{4}\right) = \sqrt{2} \cdot \frac{\sqrt{2}}{2} = \frac{2}{2} = 1 \]
\[ y = \sqrt{2} \sin\left(\frac{\pi}{4}\right) = \sqrt{2} \cdot \frac{\sqrt{2}}{2} = \frac{2}{2} = 1 \]
\textbf{Cartesian coordinates: (1, 1)}

\subsubsection*{Problem 3c: (\(-1, -\pi/6\))}
\[ x = -1 \cos\left(-\frac{\pi}{6}\right) = -1 \cdot \frac{\sqrt{3}}{2} = -\frac{\sqrt{3}}{2} \]
\[ y = -1 \sin\left(-\frac{\pi}{6}\right) = -1 \cdot \left(-\frac{1}{2}\right) = \frac{1}{2} \]
\textbf{Cartesian coordinates: (\(-\frac{\sqrt{3}}{2}, \frac{1}{2}\))}

\subsubsection*{Problem 4a: (4, \(4\pi/3\))}
\[ x = 4 \cos\left(\frac{4\pi}{3}\right) = 4 \cdot \left(-\frac{1}{2}\right) = -2 \]
\[ y = 4 \sin\left(\frac{4\pi}{3}\right) = 4 \cdot \left(-\frac{\sqrt{3}}{2}\right) = -2\sqrt{3} \]
\textbf{Cartesian coordinates: (\(-2, -2\sqrt{3}\))}

\subsubsection*{Problem 4b: (\(-2, 3\pi/4\))}
\[ x = -2 \cos\left(\frac{3\pi}{4}\right) = -2 \cdot \left(-\frac{\sqrt{2}}{2}\right) = \sqrt{2} \]
\[ y = -2 \sin\left(\frac{3\pi}{4}\right) = -2 \cdot \left(\frac{\sqrt{2}}{2}\right) = -\sqrt{2} \]
\textbf{Cartesian coordinates: (\(\sqrt{2}, -\sqrt{2}\))}

\subsubsection*{Problem 4c: (\(-3, -\pi/3\))}
\[ x = -3 \cos\left(-\frac{\pi}{3}\right) = -3 \cdot \frac{1}{2} = -\frac{3}{2} \]
\[ y = -3 \sin\left(-\frac{\pi}{3}\right) = -3 \cdot \left(-\frac{\sqrt{3}}{2}\right) = \frac{3\sqrt{3}}{2} \]
\textbf{Cartesian coordinates: (\(-\frac{3}{2}, \frac{3\sqrt{3}}{2}\))}


\subsection*{Problems 5-6: Converting to Polar Coordinates}

\subsubsection*{Problem 5a: (\(-4, 4\))}
The point is in Quadrant II.
\[ r = \sqrt{x^2 + y^2} = \sqrt{(-4)^2 + 4^2} = \sqrt{16 + 16} = \sqrt{32} = 4\sqrt{2} \]
\[ \tan(\theta) = \frac{y}{x} = \frac{4}{-4} = -1 \]
The reference angle for \(\tan(\theta)=-1\) is \(\pi/4\). Since the point is in Quadrant II, the angle is \(\theta = \pi - \pi/4 = 3\pi/4\).

\begin{itemize}
    \item \textbf{i. With \(r > 0\) and \(0 \le \theta < 2\pi\):}
    \textbf{Polar coordinates: \((4\sqrt{2}, 3\pi/4)\)}
    \item \textbf{ii. With \(r < 0\) and \(0 \le \theta < 2\pi\):}
    We use a negative radius and an angle pointing in the opposite direction.
    \[ r' = -4\sqrt{2} \]
    \[ \theta' = \frac{3\pi}{4} + \pi = \frac{7\pi}{4} \]
    \textbf{Polar coordinates: \((-4\sqrt{2}, 7\pi/4)\)}
\end{itemize}

\subsubsection*{Problem 5b: (\(3, 3\sqrt{3}\))}
The point is in Quadrant I.
\[ r = \sqrt{3^2 + (3\sqrt{3})^2} = \sqrt{9 + 27} = \sqrt{36} = 6 \]
\[ \tan(\theta) = \frac{3\sqrt{3}}{3} = \sqrt{3} \]
The angle in Quadrant I with \(\tan(\theta)=\sqrt{3}\) is \(\theta = \pi/3\).

\begin{itemize}
    \item \textbf{i. With \(r > 0\) and \(0 \le \theta < 2\pi\):}
    \textbf{Polar coordinates: \((6, \pi/3)\)}
    \item \textbf{ii. With \(r < 0\) and \(0 \le \theta < 2\pi\):}
    \[ r' = -6 \]
    \[ \theta' = \frac{\pi}{3} + \pi = \frac{4\pi}{3} \]
    \textbf{Polar coordinates: \((-6, 4\pi/3)\)}
\end{itemize}

\subsubsection*{Problem 6a: (\(\sqrt{3}, -1\))}
The point is in Quadrant IV.
\[ r = \sqrt{(\sqrt{3})^2 + (-1)^2} = \sqrt{3 + 1} = \sqrt{4} = 2 \]
\[ \tan(\theta) = \frac{-1}{\sqrt{3}} \]
The reference angle is \(\pi/6\). Since the point is in Quadrant IV, the angle is \(\theta = 2\pi - \pi/6 = 11\pi/6\).

\begin{itemize}
    \item \textbf{Polar coordinates with \(r>0\): \((2, 11\pi/6)\)}
\end{itemize}

\subsubsection*{Problem 6b: (\(-6, 0\))}
The point is on the negative x-axis.
\[ r = \sqrt{(-6)^2 + 0^2} = \sqrt{36} = 6 \]
The angle for the negative x-axis is \(\theta = \pi\).

\begin{itemize}
    \item \textbf{Polar coordinates with \(r>0\): \((6, \pi)\)}
\end{itemize}

\subsection*{Problems 7-12: Sketching Regions}

\subsubsection*{Problem 7: \(1 \le r \le 3\)}
This describes the region of all points whose distance from the pole is between 1 and 3, inclusive, for any angle. This is an annulus (a ring or washer shape) centered at the pole with an inner radius of 1 and an outer radius of 3.

\subsubsection*{Problem 8: \(r \ge 2, 0 \le \theta \le \pi\)}
This describes the region of all points in the upper half-plane (from the positive x-axis to the negative x-axis) that are at a distance of 2 or more from the pole. This is the area outside of a semicircle of radius 2 in the upper half-plane.

\subsubsection*{Problem 9: \(0 \le r \le 1, -\pi/2 \le \theta \le \pi/2\)}
This describes the region of all points in the right half-plane (from the negative y-axis to the positive y-axis) that are within a distance of 1 from the pole. This is a filled-in semicircle of radius 1 in the right half-plane.

\subsubsection*{Problem 10: \(3 < r < 5, 2\pi/3 \le \theta \le 4\pi/3\)}
This describes a sector of an annulus. The radii are between 3 and 5 (not inclusive). The angles sweep from \(2\pi/3\) (120 degrees) to \(4\pi/3\) (240 degrees), which covers the left-hand side of the plane.

\subsubsection*{Problem 11: \(2 \le r \le 4, 3\pi/4 \le \theta \le 7\pi/4\)}
This describes a sector of an annulus. The radii are between 2 and 4 (inclusive). The angles sweep from \(3\pi/4\) (135 degrees) counter-clockwise to \(7\pi/4\) (315 degrees).

\subsubsection*{Problem 12: \(r \ge 0, \pi \le \theta \le 5\pi/2\)}
The condition \(r \ge 0\) is the standard assumption. The angle sweeps from \(\pi\) to \(5\pi/2\). Since \(5\pi/2 = 2\pi + \pi/2\), this is an angle that starts at the negative x-axis (\(\pi\)), goes through a full revolution (\(2\pi\)), and ends at the positive y-axis (\(\pi/2\)). This covers the entire plane.

\subsection*{Problems 15-20: Identify Curve by Finding Cartesian Equation}

\subsubsection*{Problem 15: \(r^2 = 5\)}
We know that \(r^2 = x^2 + y^2\).
\[ x^2 + y^2 = 5 \]
This is the equation of a \textbf{circle} centered at the origin with radius \(\sqrt{5}\).

\subsubsection*{Problem 16: \(r = 4 \sec(\theta)\)}
Rewrite \(\sec(\theta)\) as \(1/\cos(\theta)\).
\[ r = \frac{4}{\cos(\theta)} \]
Multiply both sides by \(\cos(\theta)\):
\[ r \cos(\theta) = 4 \]
Since \(x = r \cos(\theta)\), the equation is:
\[ x = 4 \]
This is a \textbf{vertical line}.

\subsubsection*{Problem 17: \(r = 5 \cos(\theta)\)}
Multiply both sides by \(r\):
\[ r^2 = 5r \cos(\theta) \]
Substitute \(r^2 = x^2 + y^2\) and \(x = r \cos(\theta)\):
\[ x^2 + y^2 = 5x \]
To identify the curve, complete the square for \(x\):
\[ (x^2 - 5x) + y^2 = 0 \]
\[ \left(x^2 - 5x + \left(\frac{5}{2}\right)^2\right) + y^2 = \left(\frac{5}{2}\right)^2 \]
\[ \left(x - \frac{5}{2}\right)^2 + y^2 = \frac{25}{4} \]
This is a \textbf{circle} centered at \((5/2, 0)\) with radius \(5/2\).

\subsubsection*{Problem 18: \(\theta = \pi/3\)}
Take the tangent of both sides:
\[ \tan(\theta) = \tan(\pi/3) \]
\[ \frac{y}{x} = \sqrt{3} \]
\[ y = \sqrt{3}x \]
This is a \textbf{line} through the origin with a slope of \(\sqrt{3}\).

\subsubsection*{Problem 19: \(r^2 \cos(2\theta) = 1\)}
Use the double-angle identity \(\cos(2\theta) = \cos^2(\theta) - \sin^2(\theta)\).
\[ r^2(\cos^2(\theta) - \sin^2(\theta)) = 1 \]
Distribute the \(r^2\):
\[ (r \cos(\theta))^2 - (r \sin(\theta))^2 = 1 \]
Substitute \(x = r \cos(\theta)\) and \(y = r \sin(\theta)\):
\[ x^2 - y^2 = 1 \]
This is the equation of a \textbf{hyperbola}.

\subsubsection*{Problem 20: \(r^2 \sin(2\theta) = 1\)}
Use the double-angle identity \(\sin(2\theta) = 2 \sin(\theta) \cos(\theta)\).
\[ r^2(2 \sin(\theta) \cos(\theta)) = 1 \]
Rearrange to group \(r \sin(\theta)\) and \(r \cos(\theta)\):
\[ 2 (r \cos(\theta))(r \sin(\theta)) = 1 \]
Substitute \(x\) and \(y\):
\[ 2xy = 1 \quad \text{or} \quad y = \frac{1}{2x} \]
This is a \textbf{hyperbola}.

\subsection*{Problems 21-26: Find Polar Equation for Curve}

\subsubsection*{Problem 21: \(x^2 + y^2 = 7\)}
Substitute \(x^2 + y^2 = r^2\).
\[ r^2 = 7 \]
\[ r = \sqrt{7} \]

\subsubsection*{Problem 22: \(x = -1\)}
Substitute \(x = r \cos(\theta)\).
\[ r \cos(\theta) = -1 \]
\[ r = -\frac{1}{\cos(\theta)} = -\sec(\theta) \]

\subsubsection*{Problem 23: \(y = \sqrt{3}x\)}
Divide by \(x\):
\[ \frac{y}{x} = \sqrt{3} \]
Substitute \(\tan(\theta) = y/x\).
\[ \tan(\theta) = \sqrt{3} \]
\[ \theta = \frac{\pi}{3} \]

\subsubsection*{Problem 24: \(y = -2x^2\)}
Substitute \(y = r \sin(\theta)\) and \(x = r \cos(\theta)\).
\[ r \sin(\theta) = -2(r \cos(\theta))^2 \]
\[ r \sin(\theta) = -2r^2 \cos^2(\theta) \]
Assuming \(r \neq 0\), divide by \(r\):
\[ \sin(\theta) = -2r \cos^2(\theta) \]
Solve for \(r\):
\[ r = -\frac{\sin(\theta)}{2 \cos^2(\theta)} = -\frac{1}{2} \frac{\sin(\theta)}{\cos(\theta)} \frac{1}{\cos(\theta)} = -\frac{1}{2} \tan(\theta) \sec(\theta) \]

\subsubsection*{Problem 25: \(x^2 + y^2 = 4y\)}
Substitute \(x^2 + y^2 = r^2\) and \(y = r \sin(\theta)\).
\[ r^2 = 4r \sin(\theta) \]
Assuming \(r \neq 0\), divide by \(r\):
\[ r = 4 \sin(\theta) \]

\subsubsection*{Problem 26: \(x^2 - y^2 = 4\)}
Substitute \(x = r \cos(\theta)\) and \(y = r \sin(\theta)\).
\[ (r \cos(\theta))^2 - (r \sin(\theta))^2 = 4 \]
\[ r^2 \cos^2(\theta) - r^2 \sin^2(\theta) = 4 \]
Factor out \(r^2\):
\[ r^2 (\cos^2(\theta) - \sin^2(\theta)) = 4 \]
Use the identity \(\cos(2\theta) = \cos^2(\theta) - \sin^2(\theta)\).
\[ r^2 \cos(2\theta) = 4 \]

\part{In-Depth Analysis of Problems and Techniques}

\section{Problem Types and General Approach}

\begin{enumerate}
    \item \textbf{Plotting and Renaming Points (Problems 1-2):}
    \begin{itemize}
        \item \textbf{Approach:} To plot \((r, \theta)\), first rotate by angle \(\theta\), then move along that line by distance \(r\). If \(r < 0\), move in the opposite direction.
        \item To find alternative coordinates:
        \begin{itemize}
            \item For \(r > 0\), add or subtract multiples of \(2\pi\) to \(\theta\).
            \item For \(r < 0\), change the sign of \(r\) and add or subtract an odd multiple of \(\pi\) (like \(\pi, 3\pi, -\pi\)) to \(\theta\).
        \end{itemize}
    \end{itemize}

    \item \textbf{Coordinate Conversion (Problems 3-6):}
    \begin{itemize}
        \item \textbf{Polar to Cartesian (3-4):} This is a direct plug-and-chug using \(x = r \cos(\theta)\) and \(y = r \sin(\theta)\). Success depends entirely on knowing unit circle values.
        \item \textbf{Cartesian to Polar (5-6):} This requires two steps. First find \(r\) with \(r = \sqrt{x^2 + y^2}\). Then find \(\theta\) using \(\tan(\theta) = y/x\), making sure the angle is in the correct quadrant based on the signs of \(x\) and \(y\).
    \end{itemize}

    \item \textbf{Sketching Regions from Inequalities (Problems 7-12):}
    \begin{itemize}
        \item \textbf{Approach:} Treat the inequalities for \(r\) and \(\theta\) separately. The inequality for \(r\) defines a disk or an annulus. The inequality for \(\theta\) defines a "wedge" or sector of the plane. The final region is the intersection of these two shapes.
    \end{itemize}

    \item \textbf{Equation Conversion (Problems 15-26):}
    \begin{itemize}
        \item \textbf{Polar to Cartesian (15-20):} The goal is to eliminate \(r\) and \(\theta\). The primary strategy is to manipulate the equation to create terms like \(r^2\), \(r\cos\theta\), or \(r\sin\theta\), which can be directly substituted with \(x^2+y^2\), \(x\), and \(y\). Sometimes this involves multiplying the entire equation by \(r\). After substitution, simplify and, if possible, complete the square to identify the shape.
        \item \textbf{Cartesian to Polar (21-26):} The goal is to eliminate \(x\) and \(y\). This is usually more straightforward: directly substitute \(x = r\cos\theta\) and \(y = r\sin\theta\). After substitution, simplify the resulting equation, usually by solving for \(r\).
    \end{itemize}
\end{enumerate}

\section{Key Algebraic and Calculus Manipulations}

\begin{enumerate}
    \item \textbf{The "Multiply by r" Trick (Problem 17):}
    In problem 17, we started with \(r = 5\cos(\theta)\). We can't directly substitute anything. By multiplying the entire equation by \(r\), we get \(r^2 = 5r\cos(\theta)\). This is crucial because it creates the terms \(r^2\) and \(r\cos(\theta)\), which have direct Cartesian equivalents, \(x^2+y^2\) and \(x\). This is the single most common trick for converting polar equations of circles and other shapes into Cartesian form.

    \item \textbf{Completing the Square (Problem 17):}
    After converting \(r=5\cos(\theta)\) to \(x^2+y^2=5x\), the equation doesn't immediately look like a familiar shape. By moving all terms to one side (\(x^2-5x+y^2=0\)) and completing the square for the x-terms, we transform it into standard circle form \((x-h)^2+(y-k)^2=R^2\). This is a vital algebraic step for identifying conic sections.

    \item \textbf{Using Double-Angle Identities (Problems 19, 20, 26):}
    In problems like \(r^2\cos(2\theta)=1\), the key is to recognize the trigonometric identity. Replacing \(\cos(2\theta)\) with \(\cos^2(\theta)-\sin^2(\theta)\) allows you to distribute the \(r^2\) to get \((r\cos\theta)^2 - (r\sin\theta)^2 = 1\), which is immediately recognizable as \(x^2-y^2=1\).

    \item \textbf{Isolating Trigonometric Functions (Problem 16):}
    In \(r = 4\sec(\theta)\), rewriting \(\sec(\theta)\) as \(1/\cos(\theta)\) is the first step. This allows you to clear the denominator by multiplying by \(\cos(\theta)\), creating the term \(r\cos(\theta)\) which is simply \(x\).

    \item \textbf{Determining the Quadrant (Problems 5-6):}
    When converting \((-4, 4)\) to polar, we find \(\tan(\theta)=-1\). A calculator might give \(\theta = -\pi/4\). This is incorrect. The point \((-4,4)\) is in Quadrant II. The angle from the calculator, \(-\pi/4\), is in Quadrant IV. The correct technique is to find the reference angle (\(\pi/4\)) and use it to find the correct angle in the correct quadrant (\(\theta = \pi - \pi/4 = 3\pi/4\)). This is the most common source of error in Cartesian-to-Polar conversion.
\end{enumerate}

\part{"Cheatsheet" and Tips for Success}

\section{Summary of Important Formulas}
\begin{itemize}
    \item \textbf{Polar to Cartesian:} \(x = r\cos\theta\), \(y = r\sin\theta\)
    \item \textbf{Cartesian to Polar:} \(r^2 = x^2+y^2\), \(\tan\theta = y/x\)
\end{itemize}

\section{Tricks and Shortcuts}
\begin{itemize}
    \item \textbf{Equation Conversion Magic:} If you see a lone \(\cos\theta\) or \(\sin\theta\) in a polar equation, try multiplying the whole equation by \(r\). This is the fastest way to get to a Cartesian form for many circles and limaçons.
    \item \textbf{Lines and Circles Recognition:}
        \begin{itemize}
            \item \(\theta = \text{constant}\) is always a line through the origin.
            \item \(r = a\sec\theta\) is always a vertical line \(x=a\).
            \item \(r = b\csc\theta\) is always a horizontal line \(y=b\).
            \item \(r = a\) is always a circle centered at the origin with radius \(a\).
            \item \(r = a\cos\theta\) is a circle on the x-axis.
            \item \(r = b\sin\theta\) is a circle on the y-axis.
        \end{itemize}
    \item \textbf{Negative r Shortcut:} Instead of plotting a negative \(r\), just add \(\pi\) (180°) to the angle and plot with the positive version of \(r\). The point \((-r, \theta)\) is the same as \((r, \theta+\pi)\).
\end{itemize}

\section{Common Pitfalls and Mistakes}
\begin{itemize}
    \item \textbf{The Quadrant Trap:} When finding \(\theta = \arctan(y/x)\), always check which quadrant your original \((x,y)\) point is in. Your calculator will only give you an angle in Quadrant I or IV. You may need to add \(\pi\) to get the correct angle for Quadrants II or III.
    \item \textbf{Forgetting to Complete the Square:} An equation like \(x^2+y^2=6x\) is a circle, but you won't know its center or radius until you properly complete the square.
    \item \textbf{Simplifying \(r^2 = 9\) to \(r = 3\):} While often acceptable, remember that \(r=-3\) can also trace the same circle. This distinction becomes important in more advanced problems.
    \item \textbf{Mixing up Sine and Cosine:} A classic mistake. Remember that cosine goes with \(x\) (horizontal) and sine goes with \(y\) (vertical).
\end{itemize}

\part{Conceptual Synthesis and The "Big Picture"}

\section{Thematic Connections}
The core theme of this topic is \textbf{choosing the right tool for the job by changing your frame of reference}. The Cartesian system is built on a rectangular grid, making it perfect for describing things with horizontal and vertical structure. The polar system is built on circles, making it the natural choice for describing things with central symmetry, rotation, or radial properties.

This theme appears everywhere in mathematics:
\begin{itemize}
    \item In linear algebra, we change basis to simplify the representation of a linear transformation. A complex transformation in the standard basis might become a simple diagonal matrix in an eigenbasis.
    \item In calculus, techniques like u-substitution or trigonometric substitution are essentially temporary changes of variable (a change of coordinate system on the number line) to make a difficult integral simple.
    \item In physics, analyzing the orbit of a planet is nearly impossible in Cartesian coordinates but becomes manageable and elegant in polar coordinates.
\end{itemize}
Polar coordinates are our first formal introduction in calculus to the powerful idea that we can change our coordinate system to make a problem dramatically easier to solve.

\section{Forward and Backward Links}
\begin{itemize}
    \item \textbf{Backward Links (Foundations):} This topic is a direct application of \textbf{Trigonometry}. Without a fluent understanding of the unit circle and trigonometric definitions (\( \sin, \cos, \tan \)), the entire system is inaccessible. It also builds on the foundational concept of the \textbf{Cartesian Coordinate System}, as it is defined in relation to it.

    \item \textbf{Forward Links (Future Applications):}
    \begin{enumerate}
        \item \textbf{Complex Numbers:} Complex numbers \(z = x + iy\) can be expressed in polar form as \(z = r(\cos\theta + i\sin\theta) = re^{i\theta}\). This polar representation makes multiplication and division of complex numbers incredibly simple: you multiply the radii and add the angles.
        \item \textbf{Multivariable Calculus (Double Integrals):} This is the most important link. When you need to calculate the volume under a surface over a circular region in the xy-plane, the integral in Cartesian coordinates is often very difficult or impossible. By switching to polar coordinates, the integral's bounds become simple constants, and the integral itself often simplifies dramatically. This technique is absolutely essential.
        \item \textbf{Vector Calculus:} Fields like electromagnetism are described using vector fields. When these fields have radial symmetry (like the electric field from a point charge), they are almost always analyzed using polar, cylindrical, or spherical coordinates.
    \end{enumerate}
\end{itemize}

\part{Real-World Application and Modeling}

\section{Concrete Scenarios}
While polar coordinates are ubiquitous in physics and engineering, their application in finance and economics is more specialized but nonetheless significant, particularly in statistical modeling and the pricing of financial derivatives.

\begin{enumerate}
    \item \textbf{Financial Modeling (Option Pricing):} The pricing of options (contracts that give the right to buy or sell an asset at a set price) often relies on modeling the future price of the underlying asset (like a stock). The famous Black-Scholes model assumes stock prices follow a "random walk" described by geometric Brownian motion. To simulate possible future stock prices (a method called Monte Carlo simulation), modelers need to generate random numbers from a normal distribution (a "bell curve"). The \textbf{Box-Muller transform} is a classic statistical method that uses a variation of polar coordinates to convert simple, uniformly distributed random numbers (which computers can easily generate) into the normally distributed numbers required for the model.

    \item \textbf{Econometrics (Analyzing Cyclical Data):} Economic data often exhibits seasonality (e.g., retail sales peak in the fourth quarter, unemployment drops in the summer). While typically analyzed with time series models, this cyclical data can be visualized using polar plots. Here, the angle \(\theta\) could represent the time of year (e.g., 0 to \(2\pi\) corresponds to January 1st to December 31st), and the radius \(r\) could represent the magnitude of an economic indicator (e.g., GDP growth, inflation). This visualization can make patterns of seasonality and deviations from the norm immediately obvious.

    \item \textbf{Risk Management (Value at Risk modeling):} In finance, a portfolio's risk is often described by the statistical relationship (correlation) between the assets it contains. When simulating the behavior of a portfolio with two correlated assets, one can use a transformation involving polar coordinates to generate pairs of random numbers that have the desired correlation structure. This is crucial for calculating metrics like Value at Risk (VaR), which estimates the maximum potential loss a portfolio could face over a given time period.
\end{enumerate}

\section{Model Problem Setup}
Let's set up the model for the first scenario: using the Box-Muller transform to generate random numbers for a financial simulation.

\textbf{Problem:} A financial analyst needs to run a Monte Carlo simulation to price a European call option. The model requires generating thousands of pairs of independent, standard normally distributed random numbers. The analyst's software can only generate random numbers from a uniform distribution between 0 and 1.

\textbf{Model Setup:}
\begin{enumerate}
    \item \textbf{Variables:}
    \begin{itemize}
        \item Let \(U_1\) and \(U_2\) be two independent random variables drawn from the uniform distribution on the interval \((0, 1]\).
        \item Let \(Z_1\) and \(Z_2\) be the two independent, standard normally distributed random numbers we want to generate.
    \end{itemize}

    \item \textbf{Formulation (The Box-Muller Transform):} The method treats \(U_1\) and \(U_2\) as defining a point in a 2D space and uses a polar coordinate transformation to map them to the normal distribution.
    
    First, we define our "radius" \(R\) and "angle" \(\Theta\) from \(U_1\) and \(U_2\). This isn't a direct polar conversion, but it uses the same core idea of radius and angle.
    \[ R = \sqrt{-2 \ln(U_1)} \]
    \[ \Theta = 2\pi U_2 \]
    Here, \(R\) represents the magnitude (related to the chi-squared distribution) and \(\Theta\) represents a uniformly distributed angle from 0 to \(2\pi\).

    \item \textbf{Equation to Solve:} Now, we use the standard polar-to-Cartesian conversion formulas to get our desired normal random variables, \(Z_1\) and \(Z_2\).
    \[ Z_1 = R \cos(\Theta) = \sqrt{-2 \ln(U_1)} \cos(2\pi U_2) \]
    \[ Z_2 = R \sin(\Theta) = \sqrt{-2 \ln(U_1)} \sin(2\pi U_2) \]
\end{enumerate}
To run the simulation, the analyst would repeatedly generate pairs of \((U_1, U_2)\) and plug them into these formulas to get the pairs \((Z_1, Z_2)\) needed to model the random component of the stock's future price path.

\part{Common Variations and Untested Concepts}
The provided homework set is an excellent introduction to the polar coordinate system, focusing on plotting, conversions, and sketching. However, it omits the primary calculus applications of polar coordinates: finding area and arc length.

\section{Untested Concept 1: Area in Polar Coordinates}

\textbf{Concept:} The area of a region bounded by a polar curve \(r = f(\theta)\) from \(\theta = \alpha\) to \(\theta = \beta\) is not found by a simple rectangular integral. Instead, we sum the areas of infinitesimally small circular sectors. The area of a sector of a circle with radius \(r\) and angle \(d\theta\) is \(dA = \frac{1}{2}r^2 d\theta\). Integrating this gives the formula for the total area.

\textbf{Formula:}
\[ A = \int_{\alpha}^{\beta} \frac{1}{2} [f(\theta)]^2 \,d\theta \quad \text{or simply} \quad A = \frac{1}{2} \int_{\alpha}^{\beta} r^2 \,d\theta \]

\textbf{Worked Example: Find the area of the cardioid \(r = 1 + \cos(\theta)\).}
The cardioid is traced once as \(\theta\) goes from \(0\) to \(2\pi\).
\begin{align*}
A &= \frac{1}{2} \int_{0}^{2\pi} (1 + \cos(\theta))^2 \,d\theta \\
&= \frac{1}{2} \int_{0}^{2\pi} (1 + 2\cos(\theta) + \cos^2(\theta)) \,d\theta \\
\end{align*}
We use the half-angle identity \(\cos^2(\theta) = \frac{1 + \cos(2\theta)}{2}\).
\begin{align*}
A &= \frac{1}{2} \int_{0}^{2\pi} \left(1 + 2\cos(\theta) + \frac{1}{2} + \frac{1}{2}\cos(2\theta)\right) \,d\theta \\
&= \frac{1}{2} \int_{0}^{2\pi} \left(\frac{3}{2} + 2\cos(\theta) + \frac{1}{2}\cos(2\theta)\right) \,d\theta \\
&= \frac{1}{2} \left[ \frac{3}{2}\theta + 2\sin(\theta) + \frac{1}{4}\sin(2\theta) \right]_{0}^{2\pi} \\
&= \frac{1}{2} \left[ \left(\frac{3}{2}(2\pi) + 2\sin(2\pi) + \frac{1}{4}\sin(4\pi)\right) - \left(0 + 0 + 0\right) \right] \\
&= \frac{1}{2} [3\pi] = \frac{3\pi}{2}
\end{align*}
The area of the cardioid is \(3\pi/2\).

\section{Untested Concept 2: Arc Length in Polar Coordinates}

\textbf{Concept:} To find the length of a polar curve, we use a formula derived from the arc length formula in parametric form, by treating \(\theta\) as the parameter. The derivation gives a specific formula for polar curves.

\textbf{Formula:} The arc length \(L\) of a curve \(r = f(\theta)\) from \(\theta = \alpha\) to \(\theta = \beta\) is:
\[ L = \int_{\alpha}^{\beta} \sqrt{r^2 + \left(\frac{dr}{d\theta}\right)^2} \,d\theta \]

\textbf{Worked Example: Find the arc length of the spiral \(r = \theta\) for \(0 \le \theta \le 2\pi\).}
First, we need the derivative of \(r\) with respect to \(\theta\).
\[ r = \theta \implies \frac{dr}{d\theta} = 1 \]
Now, we plug this into the arc length formula:
\[ L = \int_{0}^{2\pi} \sqrt{\theta^2 + (1)^2} \,d\theta = \int_{0}^{2\pi} \sqrt{\theta^2 + 1} \,d\theta \]
This integral requires a trigonometric substitution (\(\theta = \tan(u)\)) and is quite involved, which is common for arc length problems. The result is:
\[ L = \frac{1}{2} \left[ \theta\sqrt{\theta^2+1} + \ln\left|\theta + \sqrt{\theta^2+1}\right| \right]_{0}^{2\pi} \]
\[ L = \frac{1}{2} \left[ (2\pi\sqrt{4\pi^2+1} + \ln|2\pi + \sqrt{4\pi^2+1}|) - (0 + \ln|1|) \right] \]
\[ L = \pi\sqrt{4\pi^2+1} + \frac{1}{2}\ln(2\pi + \sqrt{4\pi^2+1}) \]

\part{Advanced Diagnostic Testing: "Find the Flaw"}

\subsubsection*{Problem 1}
\textbf{Task:} Convert the Cartesian coordinates \((-2, -2\sqrt{3})\) to polar coordinates \((r, \theta)\) with \(r > 0\) and \(0 \le \theta < 2\pi\).

\textbf{Flawed Solution:}
\begin{enumerate}
    \item Find \(r\): \(r = \sqrt{(-2)^2 + (-2\sqrt{3})^2} = \sqrt{4 + 4(3)} = \sqrt{4 + 12} = \sqrt{16} = 4\).
    \item Find \(\theta\): \(\tan(\theta) = \frac{-2\sqrt{3}}{-2} = \sqrt{3}\).
    \item The angle whose tangent is \(\sqrt{3}\) is \(\theta = \pi/3\).
    \item The polar coordinates are \((4, \pi/3)\).
\end{enumerate}

\subsubsection*{Problem 2}
\textbf{Task:} Find the Cartesian equation for the polar curve \(r = 6\sin(\theta)\).

\textbf{Flawed Solution:}
\begin{enumerate}
    \item Multiply by \(r\): \(r^2 = 6r\sin(\theta)\).
    \item Substitute \(r^2 = x^2+y^2\) and \(y = r\sin(\theta)\): \(x^2+y^2 = 6y\).
    \item Rearrange: \(x^2 + y^2 - 6y = 0\).
    \item Complete the square: \(x^2 + (y^2 - 6y + 9) = 9\).
    \item This gives \(x^2 + (y-3)^2 = 9\).
    \item This is a circle centered at \((0, -3)\) with radius 3.
\end{enumerate}

\subsubsection*{Problem 3}
\textbf{Task:} Find a second representation of the point \((5, \pi/6)\) with \(r < 0\).

\textbf{Flawed Solution:}
\begin{enumerate}
    \item To get a negative \(r\), we change the sign of the original \(r\). So \(r' = -5\).
    \item To use a negative \(r\), we must point in the opposite direction. The original direction is \(\pi/6\). To go in the opposite direction, we should add a full revolution.
    \item \(\theta' = \pi/6 + 2\pi = 13\pi/6\).
    \item The new coordinates are \((-5, 13\pi/6)\).
\end{enumerate}

\subsubsection*{Problem 4}
\textbf{Task:} Convert the Cartesian equation \(x=3\) into a polar equation.

\textbf{Flawed Solution:}
\begin{enumerate}
    \item We know \(x=r\cos\theta\) and \(y=r\sin\theta\).
    \item The equation \(x=3\) has no \(y\), so we only use the \(x\) substitution.
    \item Substitute \(x=3\) into the formula \(r^2 = x^2+y^2\).
    \item \(r^2 = 3^2 + y^2 = 9 + (r\sin\theta)^2\).
    \item \(r^2 - r^2\sin^2\theta = 9\).
    \item \(r^2(1-\sin^2\theta) = 9 \implies r^2\cos^2\theta = 9 \implies r\cos\theta = 3\).
    \item This seems overly complicated, but it works.
    \item The flaw is not the math, but the approach.
\end{enumerate}
\textit{Note: This is a conceptual flaw in method, not a calculation error. The expected error is in the initial step.}

\textbf{Corrected Flawed Solution for Problem 4 (with a mathematical error):}
\begin{enumerate}
    \item We want to convert \(x=3\).
    \item We know the relationship \( \tan(\theta) = y/x \).
    \item So, \( \tan(\theta) = y/3 \). This introduces \(y\).
    \item Substitute \( y = r\sin(\theta) \).
    \item \( \tan(\theta) = r\sin(\theta)/3 \).
    \item \( \frac{\sin(\theta)}{\cos(\theta)} = \frac{r\sin(\theta)}{3} \).
    \item Divide by \(\sin(\theta)\): \( \frac{1}{\cos(\theta)} = \frac{r}{3} \).
    \item \( r = 3\sec(\theta) \). This result is correct, but the process has a flaw.
\end{enumerate}

\subsubsection*{Problem 5}
\textbf{Task:} Identify the curve given by \(r^2 = 9\cos(2\theta)\).

\textbf{Flawed Solution:}
\begin{enumerate}
    \item Use the identity \(\cos(2\theta) = 2\cos^2(\theta) - 1\).
    \item \(r^2 = 9(2\cos^2(\theta) - 1)\).
    \item \(r^2 = 18\cos^2(\theta) - 9\).
    \item We know \(x = r\cos\theta\), so \(\cos\theta = x/r\).
    \item Substitute this in: \(r^2 = 18(x/r)^2 - 9\).
    \item \(r^2 = 18x^2/r^2 - 9\).
    \item Multiply by \(r^2\): \(r^4 = 18x^2 - 9r^2\).
    \item Substitute \(r^2 = x^2+y^2\): \((x^2+y^2)^2 = 18x^2 - 9(x^2+y^2)\).
    \item \((x^2+y^2)^2 = 18x^2 - 9x^2 - 9y^2 = 9x^2 - 9y^2\).
    \item Final equation: \((x^2+y^2)^2 = 9(x^2 - y^2)\). This is a lemniscate, but there is a flaw in the first step.
\end{enumerate}
% Note to self: The intended flaw is that the other identity for cos(2theta) is much better. Let me rewrite the flaw to be more subtle.
\textbf{Re-written Flawed Solution for Problem 5:}
\begin{enumerate}
    \item Use the identity \(\cos(2\theta) = \cos^2(\theta) - \sin^2(\theta)\).
    \item \(r^2 = 9(\cos^2(\theta) - \sin^2(\theta))\).
    \item Distribute \(r^2\): \(r^2 \cdot 1 = 9(r^2\cos^2(\theta) - r^2\sin^2(\theta))\). There's the flaw.
    \item \(x^2+y^2 = 9((r\cos\theta)^2 - (r\sin\theta)^2)\).
    \item \(x^2+y^2 = 9(x^2 - y^2)\).
    \item This is not the correct equation for the lemniscate.
\end{enumerate}

\end{document}