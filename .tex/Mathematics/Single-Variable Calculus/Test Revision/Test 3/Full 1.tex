%%%%%%%%%%%%%%%%%%%%%%%%%%%%%%%%%%%%%%%%%%%%%%%%%%%%%%
% LaTeX Document for Calculus II Problem Set
% Part 1: Sequences
% Generated by Gemini
%%%%%%%%%%%%%%%%%%%%%%%%%%%%%%%%%%%%%%%%%%%%%%%%%%%%%%

\documentclass[12pt]{article}

% PACKAGES
\usepackage[margin=1in]{geometry} % Set page margins
\usepackage{amsmath, amssymb, amsthm} % For advanced math typesetting
\usepackage{graphicx} % To include images
\usepackage{hyperref} % For clickable links and references
\usepackage{enumitem} % For customized lists
\usepackage{amsfonts} % For math fonts

% HYPERREF SETUP
\hypersetup{
    colorlinks=true,
    linkcolor=blue,
    filecolor=magenta,      
    urlcolor=cyan,
    pdftitle={Calculus II Problem Set: Sequences},
    pdfauthor={Gemini},
}

% DOCUMENT TITLE AND AUTHOR
\title{\textbf{Calculus II Problem Set \\ Part 1: Sequences}}
\author{Generated by Gemini}
\date{\today}

% CUSTOM ENVIRONMENTS (optional, for styling)
\theoremstyle{definition}
\newtheorem{problem}{Problem}[section]
\newtheorem*{solution}{Solution}

\begin{document}

\maketitle
\thispagestyle{empty}
\clearpage

\tableofcontents
\clearpage

\pagenumbering{arabic}

%%%%%%%%%%%%%%%%%%%%%%%%%%%%%%%%%%%%%%%%%%%%%%%%%%%%%%
% SECTION 1: FINDING THE N-TH TERM
%%%%%%%%%%%%%%%%%%%%%%%%%%%%%%%%%%%%%%%%%%%%%%%%%%%%%%
\section{Finding the n-th Term}
\textit{For each of the following sequences, find a formula for the general term \(a_n\), assuming the pattern of the first few terms continues. Assume \(n\) begins with 1.}

\begin{problem}
\(1, \frac{1}{2}, \frac{1}{3}, \frac{1}{4}, \dots\)
\end{problem}

\begin{problem}
\(5, 8, 11, 14, \dots\)
\end{problem}

\begin{problem}
\(\frac{1}{2}, -\frac{4}{3}, \frac{9}{4}, -\frac{16}{5}, \dots\)
\end{problem}

\begin{problem}
\(2, \frac{3}{4}, \frac{4}{9}, \frac{5}{16}, \dots\)
\end{problem}

\begin{problem}
\(1, 0, 1, 0, 1, 0, \dots\)
\end{problem}

\begin{problem}
\(\frac{1}{2}, \frac{1}{4}, \frac{1}{8}, \frac{1}{16}, \dots\)
\end{problem}

\begin{problem}
\(0, 3, 8, 15, 24, \dots\)
\end{problem}

\begin{problem}
\(-\frac{2}{3}, \frac{4}{9}, -\frac{8}{27}, \frac{16}{81}, \dots\)
\end{problem}

\begin{problem}
\(\frac{1}{1}, \frac{2}{3}, \frac{3}{5}, \frac{4}{7}, \dots\)
\end{problem}

\begin{problem}
\(1, -1, 1, -1, 1, \dots\)
\end{problem}

\begin{problem}
\(2, 6, 12, 20, 30, \dots\)
\textit{Hint: Look at the factors of each term.}
\end{problem}

\begin{problem}
\(\cos(0), \cos(\pi), \cos(2\pi), \cos(3\pi), \dots\)
\end{problem}

\begin{problem}
\(\frac{1}{2}, \frac{2}{5}, \frac{3}{10}, \frac{4}{17}, \dots\)
\end{problem}

\begin{problem}
\(1, \frac{1}{2}, \frac{1}{6}, \frac{1}{24}, \frac{1}{120}, \dots\)
\end{problem}

\begin{problem}
\(1, 5, 9, 13, 17, \dots\)
\end{problem}

\begin{problem}
\(\frac{\sqrt{1}}{3}, \frac{\sqrt{2}}{4}, \frac{\sqrt{3}}{5}, \frac{\sqrt{4}}{6}, \dots\)
\end{problem}

\begin{problem}
\(10, 5, \frac{5}{2}, \frac{5}{4}, \frac{5}{8}, \dots\)
\end{problem}

\begin{problem}
\(\frac{5}{1}, \frac{8}{3}, \frac{11}{5}, \frac{14}{7}, \dots\)
\end{problem}

\begin{problem}
\(\{0.9, 0.99, 0.999, 0.9999, \dots\}\)
\textit{Hint: Express each term as \(1 - \dots\)}
\end{problem}

\begin{problem}
\(0, \frac{1}{2}, 0, \frac{1}{2}, 0, \frac{1}{2}, \dots\)
\end{problem}

\begin{problem}
\(\frac{\ln 1}{1}, \frac{\ln 2}{2}, \frac{\ln 3}{3}, \frac{\ln 4}{4}, \dots\)
\end{problem}

\begin{problem}
\(5, -25, 125, -625, \dots\)
\end{problem}

\begin{problem}
\(\frac{1}{e}, \frac{2}{e^2}, \frac{3}{e^3}, \frac{4}{e^4}, \dots\)
\end{problem}

\begin{problem}
\(1, -\frac{1}{8}, \frac{1}{27}, -\frac{1}{64}, \dots\)
\end{problem}

\begin{problem}
\(1, 3, 1, 3, 1, 3, \dots\)
\end{problem}

\clearpage

%%%%%%%%%%%%%%%%%%%%%%%%%%%%%%%%%%%%%%%%%%%%%%%%%%%%%%
% SECTION 2: CONVERGENCE AND DIVERGENCE
%%%%%%%%%%%%%%%%%%%%%%%%%%%%%%%%%%%%%%%%%%%%%%%%%%%%%%
\section{Convergence and Divergence of Sequences}
\textit{Determine whether the sequence converges or diverges. If it converges, find the limit.}

\begin{problem}
\(a_n = \frac{3n^2 - 1}{10n + 5n^2}\)
\end{problem}

\begin{problem}
\(a_n = \frac{n}{n+1}\)
\end{problem}

\begin{problem}
\(a_n = n \sin\left(\frac{1}{n}\right)\)
\end{problem}

\begin{problem}
\(a_n = (-1)^n \frac{n}{n+1}\)
\end{problem}

\begin{problem}
\(a_n = \frac{\ln(n)}{n}\)
\end{problem}

\begin{problem}
\(a_n = \cos(n\pi)\)
\end{problem}

\begin{problem}
\(a_n = \frac{n!}{2^n}\)
\end{problem}

\begin{problem}
\(a_n = \frac{3^n}{n!}\)
\end{problem}

\begin{problem}
\(a_n = \arctan(n)\)
\end{problem}

\begin{problem}
\(a_n = \left(1 + \frac{1}{n}\right)^n\)
\end{problem}

\begin{problem}
\(a_n = \sqrt{n+1} - \sqrt{n}\)
\end{problem}

\begin{problem}
\(a_n = \frac{\sin(n)}{n}\)
\end{problem}

\begin{problem}
\(a_n = \frac{(-1)^n}{n^2}\)
\end{problem}

\begin{problem}
\(a_n = n e^{-n}\)
\end{problem}

\begin{problem}
\(a_n = \frac{n^3}{n^3 + 1}\)
\end{problem}

\begin{problem}
\(a_n = n^{1/n}\)
\end{problem}

\begin{problem}
\(a_n = \frac{2n + 1}{1 - 3n}\)
\end{problem}

\begin{problem}
\(a_n = \frac{4n^2 - 3}{3n^2 + n + 1}\)
\end{problem}

\begin{problem}
\(a_n = \frac{\sqrt{n}}{\ln(n)}\)
\end{problem}

\begin{problem}
\(a_n = \cos\left(\frac{2}{n}\right)\)
\end{problem}

\begin{problem}
\(a_n = \frac{n^2}{2n - 1}\)
\end{problem}

\begin{problem}
\(a_n = (-1)^{n+1} \frac{1}{\sqrt{n}}\)
\end{problem}

\begin{problem}
\(a_n = \tanh(n)\)
\end{problem}

\begin{problem}
\(a_n = \frac{(n+1)!}{n!}\)
\end{problem}

\begin{problem}
\(a_n = \frac{\cos^2(n)}{2^n}\)
\end{problem}

\clearpage

%%%%%%%%%%%%%%%%%%%%%%%%%%%%%%%%%%%%%%%%%%%%%%%%%%%%%%
% SOLUTIONS
%%%%%%%%%%%%%%%%%%%%%%%%%%%%%%%%%%%%%%%%%%%%%%%%%%%%%%
\section{Solutions}

\subsection{Solutions for Section 1: Finding the n-th Term}
\begin{enumerate}
    \item[\textbf{1.1}] The terms are the reciprocals of the natural numbers. \(a_n = \frac{1}{n}\)
    \item[\textbf{1.2}] This is an arithmetic sequence with first term \(a_1=5\) and common difference \(d=3\). The formula is \(a_n = a_1 + (n-1)d = 5 + (n-1)3 = 3n + 2\).
    \item[\textbf{1.3}] Signs are alternating: \((-1)^{n+1}\). Numerators are squares: \(n^2\). Denominators are \(n+1\). So, \(a_n = \frac{(-1)^{n+1}n^2}{n+1}\).
    \item[\textbf{1.4}] Numerators are \(n+1\). Denominators are squares: \(n^2\). So, \(a_n = \frac{n+1}{n^2}\).
    \item[\textbf{1.5}] The terms alternate between 1 and 0. We can write this using cosine or with \((-1)^n\). A simple form is \(a_n = \frac{1 + (-1)^{n+1}}{2}\).
    \item[\textbf{1.6}] This is a geometric sequence with first term \(a_1=1/2\) and common ratio \(r=1/2\). The formula is \(a_n = a_1 r^{n-1} = \frac{1}{2} \left(\frac{1}{2}\right)^{n-1} = \left(\frac{1}{2}\right)^n = \frac{1}{2^n}\).
    \item[\textbf{1.7}] The terms are one less than the perfect squares. \(a_n = n^2 - 1\).
    \item[\textbf{1.8}] This is a geometric sequence with ratio \(r = -2/3\). The first term is \(-2/3\). So, \(a_n = \left(-\frac{2}{3}\right)^n\).
    \item[\textbf{1.9}] Numerators are \(n\). Denominators are consecutive odd numbers, which can be written as \(2n-1\). So, \(a_n = \frac{n}{2n-1}\).
    \item[\textbf{1.10}] The terms alternate between 1 and -1. \(a_n = (-1)^{n+1}\) or \(a_n = (-1)^{n-1}\).
    \item[\textbf{1.11}] The terms are \(1 \cdot 2, 2 \cdot 3, 3 \cdot 4, 4 \cdot 5, 5 \cdot 6, \dots\). The formula is \(a_n = n(n+1)\).
    \item[\textbf{1.12}] The arguments of cosine are \(0, \pi, 2\pi, \dots\). This is \((n-1)\pi\). The terms are \(1, -1, 1, -1, \dots\). So, \(a_n = \cos((n-1)\pi) = (-1)^{n-1}\).
    \item[\textbf{1.13}] Numerators are \(n\). Denominators are one more than the squares, \(n^2+1\). So, \(a_n = \frac{n}{n^2+1}\).
    \item[\textbf{1.14}] The denominators are factorials: \(1!, 2!, 3!, \dots\). So, \(a_n = \frac{1}{n!}\).
    \item[\textbf{1.15}] Arithmetic sequence with \(a_1=1\) and \(d=4\). \(a_n = 1 + (n-1)4 = 4n - 3\).
    \item[\textbf{1.16}] Numerators are \(\sqrt{n}\). Denominators are \(n+2\). So, \(a_n = \frac{\sqrt{n}}{n+2}\).
    \item[\textbf{1.17}] Geometric sequence with \(a_1=10\) and \(r=1/2\). \(a_n = 10 \left(\frac{1}{2}\right)^{n-1}\).
    \item[\textbf{1.18}] Numerators are an arithmetic sequence \(3n+2\). Denominators are an arithmetic sequence of odd numbers \(2n-1\). So, \(a_n = \frac{3n+2}{2n-1}\).
    \item[\textbf{1.19}] \(a_1 = 1 - 0.1\), \(a_2 = 1 - 0.01\), \(a_3 = 1 - 0.001\). This can be written as \(a_n = 1 - \frac{1}{10^n}\).
    \item[\textbf{1.20}] The sequence is \(\frac{1-(-1)^n}{4}\).
    \item[\textbf{1.21}] \(a_n = \frac{\ln n}{n}\).
    \item[\textbf{1.22}] Geometric, with \(a_1=5, r=-5\). \(a_n = 5(-5)^{n-1} = (-1)^{n-1}5^n\).
    \item[\textbf{1.23}] \(a_n = \frac{n}{e^n}\).
    \item[\textbf{1.24}] The signs alternate \((-1)^{n+1}\). The denominators are cubes \(n^3\). \(a_n = \frac{(-1)^{n+1}}{n^3}\).
    \item[\textbf{1.25}] The sequence is \(2 + (-1)^{n+1}\).
\end{enumerate}

\subsection{Solutions for Section 2: Convergence and Divergence}
\begin{enumerate}
    \item[\textbf{2.1}] Divide numerator and denominator by \(n^2\): \(\lim_{n\to\infty} \frac{3 - 1/n^2}{10/n + 5} = \frac{3-0}{0+5} = \frac{3}{5}\). \textbf{Converges to 3/5}.
    \item[\textbf{2.2}] Divide by \(n\): \(\lim_{n\to\infty} \frac{1}{1 + 1/n} = \frac{1}{1+0} = 1\). \textbf{Converges to 1}.
    \item[\textbf{2.3}] Rewrite as \(\frac{\sin(1/n)}{1/n}\). Let \(x=1/n\). As \(n\to\infty\), \(x\to0\). The limit is \(\lim_{x\to0} \frac{\sin x}{x} = 1\). \textbf{Converges to 1}.
    \item[\textbf{2.4}] The term \(\frac{n}{n+1}\) approaches 1, but the \((-1)^n\) factor causes the sequence to oscillate between values close to 1 and -1. It does not approach a single value. \textbf{Diverges}.
    \item[\textbf{2.5}] Use L'Hôpital's Rule on the function \(f(x) = \frac{\ln x}{x}\). \(\lim_{x\to\infty} \frac{\ln x}{x} = \lim_{x\to\infty} \frac{1/x}{1} = 0\). \textbf{Converges to 0}.
    \item[\textbf{2.6}] The sequence is \(-1, 1, -1, 1, \dots\). It oscillates and does not approach a single limit. \textbf{Diverges}.
    \item[\textbf{2.7}] Factorials grow faster than exponentials. The terms \(a_n\) will grow without bound. \(\lim_{n\to\infty} \frac{n!}{2^n} = \infty\). \textbf{Diverges}.
    \item[\textbf{2.8}] Factorials grow faster than exponentials. The denominator grows much faster than the numerator. \(\lim_{n\to\infty} \frac{3^n}{n!} = 0\). \textbf{Converges to 0}.
    \item[\textbf{2.9}] As \(n\) approaches infinity, the argument of arctan goes to infinity. \(\lim_{n\to\infty} \arctan(n) = \frac{\pi}{2}\). \textbf{Converges to \(\pi/2\)}.
    \item[\textbf{2.10}] This is a standard limit form. \(\lim_{n\to\infty} \left(1 + \frac{1}{n}\right)^n = e\). \textbf{Converges to e}.
    \item[\textbf{2.11}] Multiply by the conjugate: \(a_n = (\sqrt{n+1} - \sqrt{n}) \frac{\sqrt{n+1} + \sqrt{n}}{\sqrt{n+1} + \sqrt{n}} = \frac{n+1-n}{\sqrt{n+1} + \sqrt{n}} = \frac{1}{\sqrt{n+1} + \sqrt{n}}\). The limit of this expression as \(n\to\infty\) is 0. \textbf{Converges to 0}.
    \item[\textbf{2.12}] Use the Squeeze Theorem. Since \(-1 \le \sin(n) \le 1\), we have \(-\frac{1}{n} \le \frac{\sin(n)}{n} \le \frac{1}{n}\). Since \(\lim_{n\to\infty} (-\frac{1}{n}) = 0\) and \(\lim_{n\to\infty} (\frac{1}{n}) = 0\), the limit of \(a_n\) is also 0. \textbf{Converges to 0}.
    \item[\textbf{2.13}] The absolute value is \(|a_n| = \frac{1}{n^2}\), which goes to 0. Therefore, the sequence itself must go to 0. \textbf{Converges to 0}.
    \item[\textbf{2.14}] Rewrite as \(\frac{n}{e^n}\). Use L'Hôpital's Rule on \(f(x) = \frac{x}{e^x}\). \(\lim_{x\to\infty} \frac{x}{e^x} = \lim_{x\to\infty} \frac{1}{e^x} = 0\). \textbf{Converges to 0}.
    \item[\textbf{2.15}] Divide by \(n^3\): \(\lim_{n\to\infty} \frac{1}{1 + 1/n^3} = 1\). \textbf{Converges to 1}.
    \item[\textbf{2.16}] This is a standard limit. Let \(y = n^{1/n}\). Then \(\ln y = \frac{\ln n}{n}\). We already know \(\lim_{n\to\infty} \frac{\ln n}{n} = 0\). So, \(\lim \ln y = 0\), which means \(\lim y = e^0 = 1\). \textbf{Converges to 1}.
    \item[\textbf{2.17}] Divide by \(n\): \(\lim_{n\to\infty} \frac{2 + 1/n}{1/n - 3} = \frac{2}{-3}\). \textbf{Converges to -2/3}.
    \item[\textbf{2.18}] Divide by \(n^2\): \(\lim_{n\to\infty} \frac{4 - 3/n^2}{3 + 1/n + 1/n^2} = \frac{4}{3}\). \textbf{Converges to 4/3}.
    \item[\textbf{2.19}] Polynomials (even roots) grow faster than logarithms. The limit is \(\infty\). \textbf{Diverges}.
    \item[\textbf{2.20}] As \(n\to\infty\), \(2/n \to 0\). Since cosine is continuous, \(\lim_{n\to\infty} \cos\left(\frac{2}{n}\right) = \cos(0) = 1\). \textbf{Converges to 1}.
    \item[\textbf{2.21}] The degree of the numerator (2) is greater than the degree of the denominator (1). The limit is \(\infty\). \textbf{Diverges}.
    \item[\textbf{2.22}] The absolute value \(|a_n| = \frac{1}{\sqrt{n}}\) goes to 0, so the sequence converges to 0. \textbf{Converges to 0}.
    \item[\textbf{2.23}] \(\tanh(n) = \frac{e^n - e^{-n}}{e^n + e^{-n}}\). Divide by \(e^n\): \(\frac{1 - e^{-2n}}{1 + e^{-2n}}\). As \(n\to\infty\), \(e^{-2n}\to0\). The limit is \(\frac{1-0}{1+0} = 1\). \textbf{Converges to 1}.
    \item[\textbf{2.24}] \(a_n = \frac{(n+1) \cdot n!}{n!} = n+1\). The limit is \(\infty\). \textbf{Diverges}.
    \item[\textbf{2.25}] Use the Squeeze Theorem. \(0 \le \cos^2(n) \le 1\), so \(0 \le \frac{\cos^2(n)}{2^n} \le \frac{1}{2^n}\). Since \(\lim_{n\to\infty} 0 = 0\) and \(\lim_{n\to\infty} \frac{1}{2^n} = 0\), the sequence converges to 0. \textbf{Converges to 0}.
\end{enumerate}

\end{document}