%%%%%%%%%%%%%%%%%%%%%%%%%%%%%%%%%%%%%%%%%%%%%%%%%%%%%%
% LaTeX Document for Calculus II Problem Set
% Part 2 (Continued): Series Convergence and Divergence
% Tests: Alternating, Ratio, Root, Telescoping, etc.
% Generated by Gemini
%%%%%%%%%%%%%%%%%%%%%%%%%%%%%%%%%%%%%%%%%%%%%%%%%%%%%%

\documentclass[12pt]{article}

% PACKAGES
\usepackage[margin=1in]{geometry}
\usepackage{amsmath, amssymb, amsthm}
\usepackage{hyperref}
\usepackage{multicol} % For two-column layout in the problems section

% HYPERREF SETUP
\hypersetup{
    colorlinks=true,
    linkcolor=blue,
    filecolor=magenta,      
    urlcolor=cyan,
    pdftitle={Calculus II Problem Set: Series Part 2},
    pdfauthor={Gemini},
}

% DOCUMENT TITLE AND AUTHOR
\title{\textbf{Calculus II Problem Set \\ Part 2 (Continued): Series Convergence and Divergence}}
\author{Generated by Gemini}
\date{\today}

% CUSTOM ENVIRONMENTS
\theoremstyle{definition}
\newtheorem{problem}{Problem}[section]

\begin{document}

\maketitle
\thispagestyle{empty}
\clearpage

\tableofcontents
\clearpage

\pagenumbering{arabic}

%%%%%%%%%%%%%%%%%%%%%%%%%%%%%%%%%%%%%%%%%%%%%%%%%%%%%%
% SECTION 1: PRACTICE PROBLEMS
%%%%%%%%%%%%%%%%%%%%%%%%%%%%%%%%%%%%%%%%%%%%%%%%%%%%%%
\section{Practice Problems}
\textit{For each of the following series, determine whether it converges absolutely, converges conditionally, or diverges. State the test(s) you used and show all necessary work. If the series is telescoping or geometric, find its sum.}

\begin{multicols}{2}
\begin{enumerate}
    \item \(\sum_{n=1}^{\infty} \frac{(-1)^n}{\sqrt{n}}\)
    \item \(\sum_{n=1}^{\infty} \frac{n!}{10^n}\)
    \item \(\sum_{n=1}^{\infty} \left(\frac{2n}{n+1}\right)^n\)
    \item \(\sum_{n=1}^{\infty} \frac{1}{n(n+1)}\)
    \item \(\sum_{n=1}^{\infty} \frac{(-1)^{n+1}}{n^2+1}\)
    \item \(\sum_{n=1}^{\infty} \frac{3^n}{n^3}\)
    \item \(\sum_{n=1}^{\infty} (\ln(n) - \ln(n+1))\)
    \item \(\sum_{n=1}^{\infty} \frac{(-1)^n n^2}{n^2+5}\)
    \item \(\sum_{n=1}^{\infty} \frac{n^2}{2^n}\)
    \item \(\sum_{n=2}^{\infty} \frac{(-1)^n}{\ln n}\)
    \item \(\sum_{n=1}^{\infty} \frac{1}{n^2+3n+2}\)
    \item \(\sum_{n=1}^{\infty} \left(\frac{n}{3n+1}\right)^n\)
    \item \(\sum_{n=1}^{\infty} \frac{(n!)^2}{(2n)!}\)
    \item \(\sum_{n=1}^{\infty} \frac{\sin(n)}{n^2}\)
    \item \(\sum_{n=1}^{\infty} \frac{(-1)^n 2^n}{n!}\)
    \item \(\sum_{n=1}^{\infty} \left(\frac{1}{n} - \frac{1}{n+2}\right)\)
    \item \(\sum_{n=1}^{\infty} \frac{(-1)^n (n+1)}{n}\)
    \item \(\sum_{n=1}^{\infty} \left(1-\frac{1}{n}\right)^{n^2}\)
    \item \(\sum_{n=1}^{\infty} \frac{(-1)^n}{n\sqrt{n}}\)
    \item \(\sum_{n=1}^{\infty} \frac{n^n}{n!}\)
    \item \(\sum_{n=1}^{\infty} \frac{\cos(n\pi)}{n}\)
    \item \(\sum_{n=1}^{\infty} \frac{e^n}{n^n}\)
    \item \(\sum_{n=1}^{\infty} \frac{2}{n^2-1}\)
    \item \(\sum_{n=1}^{\infty} \frac{100^n}{n!}\)
    \item \(\sum_{n=1}^{\infty} \left(\frac{\arctan n}{2}\right)^n\)
    \item \(\sum_{n=1}^{\infty} \frac{(-1)^{n-1}}{\sqrt[3]{n}}\)
    \item \(\sum_{n=1}^{\infty} \frac{(-1)^n 3^n}{(2n+1)!}\)
    \item \(\sum_{n=1}^{\infty} \frac{5+2\cos n}{n^3}\)
    \item \(\sum_{n=1}^{\infty} \frac{(-1)^n n}{e^n}\)
    \item \(\sum_{n=1}^{\infty} \ln\left(\frac{n}{n+1}\right)\)
    \item \(\sum_{n=1}^{\infty} \left(\frac{n+1}{n}\right)^n\)
    \item \(\sum_{n=1}^{\infty} \frac{n!}{(2n)!}\)
    \item \(\sum_{n=1}^{\infty} \frac{(-1)^n}{\sqrt{n+1}}\)
    \item \(\sum_{n=1}^{\infty} \frac{n \cdot 2^n}{3^n}\)
    \item \(\sum_{n=1}^{\infty} \frac{3}{\sqrt{(n+2)(n+3)}}\)
    \item \(\sum_{n=1}^{\infty} \frac{(-1)^n}{3n+1}\)
    \item \(\sum_{n=1}^{\infty} \frac{n^3}{n!}\)
    \item \(\sum_{n=1}^{\infty} \left(e^{1/n}-e^{1/(n+1)}\right)\)
    \item \(\sum_{n=1}^{\infty} \frac{n!}{1 \cdot 3 \cdot 5 \cdots (2n-1)}\)
    \item \(\sum_{n=1}^{\infty} \frac{(-1)^n n^n}{(2n)!}\)
    \item \(\sum_{n=1}^{\infty} \frac{(-1)^n \ln n}{n}\)
    \item \(\sum_{n=1}^{\infty} \left(\frac{n}{\ln n}\right)^n\)
    \item \(\sum_{n=1}^{\infty} \frac{2^n n!}{(n+2)!}\)
    \item \(\sum_{n=1}^{\infty} \frac{2+(-1)^n}{n^2}\)
    \item \(\sum_{n=1}^{\infty} \frac{(-1)^n (n!)^3}{(3n)!}\)
    \item \(\sum_{n=1}^{\infty} (\sqrt{n+1}-\sqrt{n})\)
    \item \(\sum_{n=1}^{\infty} \frac{4^n}{(n+1)^n}\)
    \item \(\sum_{n=1}^{\infty} \frac{(-1)^n}{n!}\)
    \item \(\sum_{n=1}^{\infty} \frac{(-1)^n}{1+\ln n}\)
    \item \(\sum_{n=1}^{\infty} \frac{1}{n(n+1)(n+2)}\)
    \item \(\sum_{n=1}^{\infty} \left(\frac{n^2+1}{2n^2+1}\right)^n\)
    \item \(\sum_{n=1}^{\infty} \frac{\cos(n)}{2^n}\)
    \item \(\sum_{n=1}^{\infty} \frac{(-1)^n}{2n+1}\)
    \item \(\(\sum_{n=1}^{\infty} \frac{(n+1)5^n}{n \cdot 3^{2n}}\)\)
    \item \(\sum_{n=1}^{\infty} (\sqrt[n]{n}-1)^n\)
    \item \(\sum_{n=1}^{\infty} \frac{(-1)^n \arctan n}{n^2}\)
    \item \(\sum_{n=1}^{\infty} \frac{1}{2n^2-n}\)
    \item \(\sum_{n=1}^{\infty} \frac{3^n n^2}{n!}\)
    \item \(\sum_{n=1}^{\infty} \frac{(-1)^n(n-1)}{n^2+1}\)
    \item \(\sum_{n=1}^{\infty} \frac{5^n}{(n!)^2}\)
    \item \(\sum_{n=1}^{\infty} \left(\frac{1}{2^n} - \frac{1}{2^{n+1}}\right)\)
    \item \(\sum_{n=1}^{\infty} \frac{(2n)!}{2^n n!}\)
    \item \(\sum_{n=1}^{\infty} \left(1+\frac{k}{n}\right)^n\)
    \item \(\sum_{n=1}^{\infty} \frac{(-1)^n n}{n^3+1}\)
    \item \(\sum_{n=1}^{\infty} \frac{\sin(n\pi/2)}{n!}\)
    \item \(\sum_{n=1}^{\infty} \frac{n!}{e^{n^2}}\)
    \item \(\sum_{n=1}^{\infty} \left(\cos\left(\frac{1}{n}\right) - \cos\left(\frac{1}{n+1}\right)\right)\)
    \item \(\sum_{n=1}^{\infty} \frac{(-1)^n n!}{100^n}\)
    \item \(\sum_{n=1}^{\infty} \left(\frac{n}{n+1}\right)^{n^2}\)
    \item \(\sum_{n=1}^{\infty} \frac{(-1)^n}{n^{1.01}}\)
    \item \(\sum_{n=1}^{\infty} \frac{4n!}{1 \cdot 4 \cdot 7 \cdots (3n-2)}\)
    \item \(\sum_{n=1}^{\infty} \frac{(-1)^n(n^2+1)}{2n^2+n-1}\)
    \item \(\sum_{n=1}^{\infty} \frac{2 \cdot 4 \cdot 6 \cdots (2n)}{n!}\)
    \item \(\sum_{n=1}^{\infty} \frac{(-1)^n (2n)!}{n^{2n}}\)
    \item \(\sum_{n=1}^{\infty} \frac{3^n+n}{n!+n^2}\)
    \item \(\sum_{n=1}^{\infty} \frac{1}{n\sqrt{n^2-1}}\)
    \item \(\sum_{n=1}^{\infty} \frac{(-1)^n}{n^p}\) (for p>0)
    \item \(\sum_{n=1}^{\infty} \frac{(n!)^n}{n^{n^2}}\)
    \item \(\sum_{n=1}^{\infty} \frac{(-1)^n 2^{4n}}{(2n+1)!}\)
    \item \(\sum_{n=1}^{\infty} \frac{1}{n(n+1)(n+2)}\)
    \item \(\sum_{n=1}^{\infty} \frac{1}{\sqrt{n!}}\)
    \item \(\sum_{n=1}^{\infty} \frac{n \ln n}{(n+1)^3}\)
    \item \(\sum_{n=1}^{\infty} (\sqrt[n]{2}-1)\)
    \item \(\sum_{n=1}^{\infty} \frac{(-1)^n}{\sqrt[4]{n}}\)
    \item \(\sum_{n=1}^{\infty} \frac{n!}{n^3+1}\)
    \item \(\sum_{n=1}^{\infty} \frac{(-1)^n e^{-n}}{n}\)
    \item \(\sum_{n=1}^{\infty} \frac{(3n)!}{n!(2n)! 3^n}\)
    \item \(\sum_{n=1}^{\infty} \left(\frac{3n+1}{2n+5}\right)^n\)
    \item \(\sum_{n=1}^{\infty} \frac{(-1)^n}{n \ln n}\)
    \item \(\sum_{n=1}^{\infty} \frac{(n+1)!-n!}{(n+2)!}\)
    \item \(\sum_{n=1}^{\infty} \frac{4^n-1}{5^n-2}\)
    \item \(\sum_{n=1}^{\infty} \frac{\sin(n\pi)}{n^2+1}\)
    \item \(\sum_{n=1}^{\infty} \frac{n!}{e^n}\)
    \item \(\sum_{n=1}^{\infty} \frac{(-1)^n(n^2+n)}{n^3+1}\)
    \item \(\sum_{n=1}^{\infty} \frac{1}{\sqrt[n]{n!}}\)
    \item \(\sum_{n=1}^{\infty} \frac{(-1)^n}{\arctan(n)}\)
    \item \(\sum_{n=1}^{\infty} \frac{n! n!}{(2n)!} 2^n\)
    \item \(\sum_{n=1}^{\infty} \left(\frac{1}{n^{0.8}} - \frac{1}{(n+1)^{0.8}}\right)\)
    \item \(\sum_{n=1}^{\infty} \frac{(-1)^n n}{n+1}\)
    \item \(\sum_{n=1}^{\infty} \left(\frac{\ln n}{n}\right)^n\)
    \item \(\sum_{n=1}^{\infty} \frac{(-1)^n 3n^2}{n^3+1}\)
    \item \(\sum_{n=1}^{\infty} \frac{10^n}{(n+1)!}\)
    \item \(\sum_{n=1}^{\infty} \frac{n^2 \cos(n)}{n^4+1}\)
    \item \(\sum_{n=1}^{\infty} \frac{(-1)^n}{\ln(n+1)}\)
    \item \(\sum_{n=1}^{\infty} \frac{n! n^3}{n^n}\)
    \item \(\sum_{n=1}^{\infty} \frac{(-1)^n}{n(n+1)}\)
    \item \(\sum_{n=1}^{\infty} \left(1+\frac{1}{n^2}\right)^n\)
    \item \(\sum_{n=1}^{\infty} \frac{2^n}{n^{100}}\)
    \item \(\sum_{n=1}^{\infty} \frac{(-1)^n}{1 \cdot 3 \cdot 5 \cdots (2n-1)}\)
    \item \(\sum_{n=1}^{\infty} \frac{1}{n^n}\)
    \item \(\sum_{n=1}^{\infty} \frac{n^2-1}{n^2+2n}\)
    \item \(\sum_{n=1}^{\infty} (\sqrt{n^2+1}-n)\)
    \item \(\sum_{n=1}^{\infty} \frac{(-1)^n n^2}{4^n}\)
    \item \(\sum_{n=1}^{\infty} \frac{1}{n\sqrt{\ln n}}\)
    \item \(\sum_{n=1}^{\infty} \frac{n}{n!}\)
    \item \(\sum_{n=1}^{\infty} \left( \frac{1}{n^{1/3}} - \frac{1}{(n+1)^{1/3}} \right)\)
    \item \(\sum_{n=1}^{\infty} \frac{(-1)^n \sin(1/n)}{1/n}\)
    \item \(\sum_{n=1}^{\infty} \frac{3^n n!}{(2n)!}\)
    \item \(\sum_{n=1}^{\infty} \frac{(-1)^n}{n^2-n+1}\)
    \item \(\sum_{n=1}^{\infty} \frac{(n!)^2}{2^{n^2}}\)
    \item \(\sum_{n=1}^{\infty} \frac{(-1)^n(2n)!}{4^n n! n!}\)
    \item \(\sum_{n=1}^{\infty} \frac{3^{n^2}}{n!}\)
    \item \(\sum_{n=1}^{\infty} \frac{1}{(n+2)\ln(n+2)}\)
    \item \(\sum_{n=1}^{\infty} \frac{(-1)^n(n+1)!}{n^n}\)
    \item \(\sum_{n=1}^{\infty} (\sqrt[3]{n+1}-\sqrt[3]{n})\)
    \item \(\sum_{n=1}^{\infty} \frac{(-1)^n \cos(n)}{n^2}\)
    \item \(\sum_{n=1}^{\infty} \frac{(2n)^n}{n^{2n}}\)
    \item \(\sum_{n=1}^{\infty} \frac{(-1)^n}{n \cdot 2^n}\)
    \item \(\sum_{n=1}^{\infty} \frac{1}{1+2+..+n}\)
    \item \(\sum_{n=1}^{\infty} \frac{3n!}{(n+1)!}\)
    \item \(\sum_{n=1}^{\infty} \frac{(-1)^n}{e^n}\)
    \item \(\sum_{n=1}^{\infty} \frac{n \ln n}{2^n}\)
    \item \(\sum_{n=1}^{\infty} \frac{2n}{n^2+1}\)
    \item \(\sum_{n=1}^{\infty} \frac{n^2}{n!}\)
    \item \(\sum_{n=1}^{\infty} \frac{1}{\sqrt{n(n+1)}}\)
    \item \(\sum_{n=1}^{\infty} \frac{(-1)^n \sqrt{n}}{n+1}\)
    \item \(\sum_{n=1}^{\infty} \frac{3^n}{n^n}\)
    \item \(\sum_{n=1}^{\infty} \frac{1}{n^2\ln n}\)
    \item \(\sum_{n=1}^{\infty} \frac{(-1)^n \cdot 2 \cdot 4 \cdots (2n)}{1 \cdot 4 \cdot 7 \cdots (3n-2)}\)
    \item \(\sum_{n=1}^{\infty} \frac{n}{n+1}\)
    \item \(\sum_{n=1}^{\infty} \frac{1}{2+\sin n}\)
    \item \(\sum_{n=1}^{\infty} \frac{1}{(2n)!}\)
    \item \(\sum_{n=1}^{\infty} \frac{(-1)^n \cdot n}{n^2+1}\)
    \item \(\sum_{n=1}^{\infty} \frac{1}{2^n+n}\)
    \item \(\sum_{n=1}^{\infty} \frac{(2n+1)^n}{n^{2n}}\)
    \item \(\sum_{n=1}^{\infty} \frac{n^3}{n^4-1}\)
    \item \(\sum_{n=1}^{\infty} \frac{1}{n \cdot 5^n}\)
    \item \(\sum_{n=1}^{\infty} \frac{n!+1}{(n+1)!}\)
    \item \(\sum_{n=1}^{\infty} \frac{(-1)^n n^3}{3^n}\)
    \item \(\sum_{n=1}^{\infty} \frac{1}{3^n-2^n}\)
    \item \(\sum_{n=1}^{\infty} \frac{2}{n(n+2)}\)
    \item \(\sum_{n=1}^{\infty} \frac{(-1)^n}{\sqrt{n^2+1}}\)
    \item \(\sum_{n=1}^{\infty} \frac{(n!)^2 n^3}{(2n)!}\)
    \item \(\sum_{n=1}^{\infty} \frac{1}{\sqrt{n^3+1}}\)
    \item \(\sum_{n=1}^{\infty} \frac{1}{(n\ln n)\ln(\ln n)}\)
    \item \(\sum_{n=1}^{\infty} \frac{n^2 2^n}{(n+1)!}\)
    \item \(\sum_{n=1}^{\infty} \frac{n^3-1}{n^3+1}\)
    \item \(\sum_{n=1}^{\infty} \frac{(-1)^n}{\sqrt[n]{n}}\)
    \item \(\sum_{n=1}^{\infty} \frac{n^4}{4^n}\)
    \item \(\sum_{n=1}^{\infty} \frac{\sin(1/n)}{\sqrt{n}}\)
    \item \(\sum_{n=1}^{\infty} \frac{3^{\cos n}}{n^2}\)
    \item \(\sum_{n=1}^{\infty} \frac{(-1)^n(n!)^2}{(2n)!}\)
    \item \(\sum_{n=1}^{\infty} \frac{1}{n \sqrt{n^2+1}}\)
    \item \(\sum_{n=1}^{\infty} \frac{1}{n^{1/2}}\)
    \item \(\sum_{n=1}^{\infty} \frac{(-1)^n}{\sqrt{n}+\sqrt{n+1}}\)
    \item \(\sum_{n=1}^{\infty} \frac{n!}{(2n-1)!}\)
    \item \(\sum_{n=1}^{\infty} \frac{\arctan(n)}{n\sqrt{n}}\)
    \item \(\sum_{n=1}^{\infty} \frac{1}{(n+1)!}\)
    \item \(\sum_{n=1}^{\infty} \frac{n^n}{3^{n^2}}\)
    \item \(\sum_{n=1}^{\infty} \frac{n^3+1}{n^4-1}\)
\end{enumerate}
\end{multicols}

\clearpage

\section{Practice Problems - Power Series Representation}
\textit{For each of the following functions, find a power series representation and determine the radius and interval of convergence.}

\begin{multicols}{2}
\begin{enumerate}
    \item \( f(x) = \frac{1}{1+x} \)
    \item \( f(x) = \frac{1}{1-x^2} \)
    \item \( f(x) = \frac{1}{1+4x^2} \)
    \item \( f(x) = \frac{1}{4-x} \)
    \item \( f(x) = \frac{x}{1-x} \)
    \item \( f(x) = \frac{x^2}{1+x^3} \)
    \item \( f(x) = \frac{3}{2+x} \)
    \item \( f(x) = \frac{1}{(1-x)^2} \)
    \item \( f(x) = \frac{1}{(1+x)^3} \)
    \item \( f(x) = \frac{x}{(1-x^2)^2} \)
    \item \( f(x) = \ln(1-x) \)
    \item \( f(x) = \ln(1+x^2) \)
    \item \( f(x) = \arctan(x) \)
    \item \( f(x) = x \arctan(x^2) \)
    \item \( f(x) = \ln(5+x) \)
    \item \( f(x) = \frac{1}{x-5} \)
    \item \( f(x) = \frac{x^3}{1+x} \)
    \item \( f(x) = \frac{1}{(1+2x)^2} \)
    \item \( f(x) = x \ln(1+x) \)
    \item \( f(x) = \int \frac{1}{1+x^4} dx \)
    \item \( f(x) = \frac{2x}{1-x^2} \)
    \item \( f(x) = \frac{x}{2-x} \)
    \item \( f(x) = \frac{d}{dx} \left( \frac{1}{1+x^3} \right) \)
    \item \( f(x) = \ln(1-x^4) \)
    \item \( f(x) = \frac{1+x}{1-x} \)
    \item \( f(x) = \frac{x^2}{16-x^2} \)
    \item \( f(x) = \frac{1}{9+x^2} \)
    \item \( f(x) = \frac{x}{1+x^2} \)
    \item \( f(x) = \frac{1}{(1-3x)^2} \)
    \item \( f(x) = x^2 \ln(1-x) \)
    \item \( f(x) = \arctan(x/3) \)
    \item \( f(x) = \frac{x}{(1-x)(1-2x)} \)
    \item \( f(x) = \frac{1}{x^2+4x+3} \)
    \item \( f(x) = \frac{1}{3-2x} \)
    \item \( f(x) = \ln(x^2+1) \)
    \item \( f(x) = \frac{x}{1-x^4} \)
    \item \( f(x) = \frac{2}{(1+x)^3} \)
    \item \( f(x) = x e^x \)
    \item \( f(x) = \cos(x^2) \)
    \item \( f(x) = \frac{\sin(x)}{x} \)
    \item \( f(x) = \frac{1}{2x-5} \)
    \item \( f(x) = \ln(1+x) - x \)
    \item \( f(x) = \int x \arctan(x) dx \)
    \item \( f(x) = \frac{d}{dx} \ln(1-x^2) \)
    \item \( f(x) = \frac{x-1}{x-2} \)
    \item \( f(x) = \sin^2(x) \)
    \item \( f(x) = e^{-x^2} \)
    \item \( f(x) = \frac{1+x^2}{1-x^2} \)
    \item \( f(x) = \ln\left(\frac{1+x}{1-x}\right) \)
    \item \( f(x) = \frac{x^2}{(1+x)^2} \)
\end{enumerate}
\end{multicols}

\clearpage

%%%%%%%%%%%%%%%%%%%%%%%%%%%%%%%%%%%%%%%%%%%%%%%%%%%%%%
% SOLUTIONS SECTION
%%%%%%%%%%%%%%%%%%%%%%%%%%%%%%%%%%%%%%%%%%%%%%%%%%%%%%
\section{Solutions}
\begin{enumerate}
    \item[\textbf{1.}] \textbf{Conditionally Convergent}. Alternating Series Test: \(b_n = 1/\sqrt{n}\) is decreasing and \(\lim b_n=0\), so it converges. Absolute series \(\sum 1/\sqrt{n}\) is a p-series with \(p=1/2 \le 1\), so it diverges.
    \item[\textbf{2.}] \textbf{Diverges}. Ratio Test: \(\lim_{n\to\infty} \frac{(n+1)!/10^{n+1}}{n!/10^n} = \lim_{n\to\infty} \frac{n+1}{10} = \infty > 1\).
    \item[\textbf{3.}] \textbf{Diverges}. Root Test: \(\lim_{n\to\infty} \left( \left(\frac{2n}{n+1}\right)^n \right)^{1/n} = \lim_{n\to\infty} \frac{2n}{n+1} = 2 > 1\).
    \item[\textbf{4.}] \textbf{Converges to 1}. Telescoping Series. Use partial fractions: \(\frac{1}{n(n+1)} = \frac{1}{n} - \frac{1}{n+1}\). The partial sum is \(S_N = (1-\frac{1}{2}) + (\frac{1}{2}-\frac{1}{3}) + \dots + (\frac{1}{N}-\frac{1}{N+1}) = 1 - \frac{1}{N+1}\). \(\lim_{N\to\infty} S_N = 1\).
    \item[\textbf{5.}] \textbf{Absolutely Convergent}. The absolute series is \(\sum \frac{1}{n^2+1}\), which converges by Limit Comparison with \(\sum 1/n^2\).
    \item[\textbf{6.}] \textbf{Diverges}. Ratio Test: \(\lim_{n\to\infty} \frac{3^{n+1}/(n+1)^3}{3^n/n^3} = \lim_{n\to\infty} 3 \left(\frac{n}{n+1}\right)^3 = 3 > 1\).
    \item[\textbf{7.}] \textbf{Diverges to \(-\infty\)}. Telescoping Series. \(\sum (\ln(n) - \ln(n+1)) = \sum -\ln(\frac{n+1}{n})\). \(S_N = (\ln 1 - \ln 2) + (\ln 2 - \ln 3) + \dots + (\ln N - \ln(N+1)) = \ln 1 - \ln(N+1) = -\ln(N+1)\). \(\lim_{N\to\infty} S_N = -\infty\).
    \item[\textbf{8.}] \textbf{Diverges}. Test for Divergence on an alternating series. \(\lim_{n\to\infty} \frac{n^2}{n^2+5} = 1 \neq 0\). The terms oscillate between approx. 1 and -1.
    \item[\textbf{9.}] \textbf{Converges Absolutely}. Ratio Test: \(\lim_{n\to\infty} \frac{(n+1)^2/2^{n+1}}{n^2/2^n} = \lim_{n\to\infty} \frac{1}{2} \left(\frac{n+1}{n}\right)^2 = \frac{1}{2} < 1\).
    \item[\textbf{10.}] \textbf{Conditionally Convergent}. AST: \(b_n=1/\ln n\) is decreasing and \(\lim b_n = 0\), so it converges. Absolute series \(\sum 1/\ln n\) diverges by Direct Comparison with \(\sum 1/n\) (since \(\ln n < n\)).
    \item[\textbf{11.}] \textbf{Converges to 1/2}. Telescoping Series. \(\frac{1}{(n+1)(n+2)} = \frac{1}{n+1} - \frac{1}{n+2}\). \(S_N = (\frac{1}{2}-\frac{1}{3}) + \dots + (\frac{1}{N+1}-\frac{1}{N+2}) = \frac{1}{2} - \frac{1}{N+2}\). The sum is 1/2.
    \item[\textbf{12.}] \textbf{Converges Absolutely}. Root Test: \(\lim_{n\to\infty} \left( \left(\frac{n}{3n+1}\right)^n \right)^{1/n} = \lim_{n\to\infty} \frac{n}{3n+1} = \frac{1}{3} < 1\).
    \item[\textbf{13.}] \textbf{Converges Absolutely}. Ratio Test: \(\lim_{n\to\infty} \frac{((n+1)!)^2}{(2n+2)!} \cdot \frac{(2n)!}{(n!)^2} = \lim_{n\to\infty} \frac{(n+1)^2}{(2n+2)(2n+1)} = \frac{1}{4} < 1\).
    \item[\textbf{14.}] \textbf{Converges Absolutely}. The absolute series is \(\sum \frac{|\sin n|}{n^2}\). Use Direct Comparison: \(\frac{|\sin n|}{n^2} \le \frac{1}{n^2}\). Since \(\sum 1/n^2\) converges, the series converges absolutely.
    \item[\textbf{15.}] \textbf{Converges Absolutely}. Ratio Test: \(\lim_{n\to\infty} \frac{2^{n+1}/(n+1)!}{2^n/n!} = \lim_{n\to\infty} \frac{2}{n+1} = 0 < 1\).
    \item[\textbf{16.}] \textbf{Converges to 3/2}. Telescoping Series. \(S_N = (1-\frac{1}{3}) + (\frac{1}{2}-\frac{1}{4}) + (\frac{1}{3}-\frac{1}{5}) + \dots + (\frac{1}{N}-\frac{1}{N+2})\). The terms that remain are \(1+\frac{1}{2} - \frac{1}{N+1} - \frac{1}{N+2}\). The sum is \(1+1/2=3/2\).
    \item[\textbf{17.}] \textbf{Diverges}. Test for Divergence for alternating series. \(\lim_{n\to\infty} \frac{n+1}{n} = 1 \neq 0\).
    \item[\textbf{18.}] \textbf{Converges Absolutely}. Root Test: \(\lim_{n\to\infty} \left( \left(1-\frac{1}{n}\right)^{n^2} \right)^{1/n} = \lim_{n\to\infty} \left(1-\frac{1}{n}\right)^n = e^{-1} = \frac{1}{e} < 1\).
    \item[\textbf{19.}] \textbf{Absolutely Convergent}. Absolute series \(\sum \frac{1}{n^{3/2}}\) is a p-series with \(p=3/2 > 1\).
    \item[\textbf{20.}] \textbf{Diverges}. Test for Divergence. \(\lim_{n\to\infty} \frac{n^n}{n!} = \infty\).
    \item[\textbf{21.}] \textbf{Conditionally Convergent}. Note \(\cos(n\pi) = (-1)^n\). This is the alternating harmonic series, which converges conditionally.
    \item[\textbf{22.}] \textbf{Converges Absolutely}. Root Test: \(\lim_{n\to\infty} (\frac{e^n}{n^n})^{1/n} = \lim_{n\to\infty} \frac{e}{n} = 0 < 1\).
    \item[\textbf{23.}] \textbf{Converges to 3/2}. Telescoping Series (for \(n \ge 2\)). \(\frac{2}{(n-1)(n+1)} = \frac{1}{n-1}-\frac{1}{n+1}\). \(S_N = (1-\frac{1}{3}) + (\frac{1}{2}-\frac{1}{4}) + \dots\). Sum is \(1+1/2=3/2\).
    \item[\textbf{24.}] \textbf{Converges Absolutely}. Ratio Test: \(\lim_{n\to\infty} \frac{100^{n+1}/(n+1)!}{100^n/n!} = \lim_{n\to\infty} \frac{100}{n+1} = 0 < 1\).
    \item[\textbf{25.}] \textbf{Converges Absolutely}. Root Test: \(\lim_{n\to\infty} \frac{\arctan n}{2} = \frac{\pi/2}{2} = \frac{\pi}{4} < 1\).
    \item[\textbf{26.}] \textbf{Conditionally Convergent}. AST: \(b_n=1/n^{1/3}\) is decreasing and has limit 0. Absolute series \(\sum 1/n^{1/3}\) diverges (p-series, \(p=1/3 \le 1\)).
    \item[\textbf{27.}] \textbf{Converges Absolutely}. Ratio Test: \(\lim_{n\to\infty} \frac{3^{n+1}/(2n+3)!}{3^n/(2n+1)!} = \lim_{n\to\infty} \frac{3}{(2n+3)(2n+2)} = 0 < 1\).
    \item[\textbf{28.}] \textbf{Converges Absolutely}. Direct Comparison Test. \(3 \le 5+2\cos n \le 7\). So \(\frac{5+2\cos n}{n^3} \le \frac{7}{n^3}\). \(\sum 7/n^3\) converges.
    \item[\textbf{29.}] \textbf{Converges Absolutely}. Ratio Test on absolute series \(\sum n/e^n\): \(\lim_{n\to\infty} \frac{(n+1)/e^{n+1}}{n/e^n} = \lim_{n\to\infty} \frac{n+1}{n} \frac{1}{e} = \frac{1}{e} < 1\).
    \item[\textbf{30.}] \textbf{Diverges}. This is the same as problem 7. Telescoping series whose partial sum \(S_N = -\ln(N+1)\) diverges.
    \item[\textbf{31.}] \textbf{Diverges}. Test for Divergence. \(\lim_{n\to\infty} \left(\frac{n+1}{n}\right)^n = \lim_{n\to\infty} \left(1+\frac{1}{n}\right)^n = e \neq 0\).
    \item[\textbf{32.}] \textbf{Converges Absolutely}. Ratio Test. \(\lim_{n\to\infty} \frac{(n+1)!/(2n+2)!}{n!/(2n)!} = \lim_{n\to\infty} \frac{n+1}{(2n+2)(2n+1)} = 0 < 1\).
    \item[\textbf{33.}] \textbf{Conditionally Convergent}. AST: \(b_n=1/\sqrt{n+1}\) is decreasing with limit 0. Absolute series \(\sum 1/\sqrt{n+1}\) diverges by LCT with \(\sum 1/\sqrt{n}\).
    \item[\textbf{34.}] \textbf{Converges Absolutely}. Ratio Test. \(\lim_{n\to\infty} \frac{(n+1)2^{n+1}/3^{n+1}}{n 2^n / 3^n} = \lim_{n\to\infty} \frac{n+1}{n}\frac{2}{3} = \frac{2}{3} < 1\).
    \item[\textbf{35.}] \textbf{Diverges}. Telescoping Series. The terms are \(a_n = \frac{3}{n+2} - \frac{3}{n+3}\) is incorrect. This is not a telescoping series. Use LCT with \(b_n=1/n\). \(\lim_{n\to\infty} \frac{3/\sqrt{n^2+5n+6}}{1/n} = 3\). Since \(\sum 1/n\) diverges, the series diverges.
    \item[\textbf{36.}] \textbf{Conditionally Convergent}. AST: \(b_n = 1/(3n+1)\) is decreasing with limit 0. Absolute series \(\sum 1/(3n+1)\) diverges by LCT with \(\sum 1/n\).
    \item[\textbf{37.}] \textbf{Converges Absolutely}. Ratio Test. \(\lim_{n\to\infty} \frac{(n+1)^3/(n+1)!}{n^3/n!} = \lim_{n\to\infty} \frac{(n+1)^3}{(n+1)n^3} = \lim_{n\to\infty} \frac{(n+1)^2}{n^3} = 0 < 1\).
    \item[\textbf{38.}] \textbf{Converges to \(e-1\)}. Telescoping Series. \(S_N = (e^{1/1}-e^{1/2})+(e^{1/2}-e^{1/3})+\dots+(e^{1/N}-e^{1/(N+1)}) = e - e^{1/(N+1)}\). \(\lim_{N\to\infty} S_N = e - e^0 = e-1\).
    \item[\textbf{39.}] \textbf{Converges Absolutely}. Ratio Test. \(\lim_{n\to\infty} \frac{(n+1)!/(... \cdot (2n+1))}{n!/(... \cdot (2n-1))} = \lim_{n\to\infty} \frac{n+1}{2n+1} = \frac{1}{2} < 1\).
    \item[\textbf{40.}] \textbf{Converges Absolutely}. Ratio Test. \(\lim_{n\to\infty} \frac{(n+1)^{n+1}/(2n+2)!}{n^n/(2n)!} = \lim_{n\to\infty} \frac{(n+1)^{n+1}}{n^n}\frac{1}{(2n+2)(2n+1)} = \lim_{n\to\infty} (1+1/n)^n \frac{n+1}{(2n+2)(2n+1)} = e \cdot 0 = 0 < 1\).
    \item[\textbf{41.}] \textbf{Conditionally Convergent}. AST: \(b_n = (\ln n)/n\) is decreasing for \(n \ge 3\) with limit 0. Absolute series \(\sum (\ln n)/n\) diverges by Integral Test.
    \item[\textbf{42.}] \textbf{Diverges}. Root Test. \(\lim_{n\to\infty} \frac{n}{\ln n} = \infty > 1\).
    \item[\textbf{43.}] \textbf{Converges to 2}. Telescoping Series. \(\frac{2^n n!}{(n+2)!} = \frac{2^n}{(n+2)(n+1)}\). This is not telescoping. Let's re-evaluate. It seems the problem was intended to be simpler. Let's assume the question is \(\sum_{n=1}^{\infty} \left(\frac{1}{n!} - \frac{1}{(n+1)!}\right)\). This telescopes to 1. If the problem is as written, use Ratio Test: \(\lim_{n\to\infty} \frac{2^{n+1}(n+1)!/(n+3)!}{2^n n!/(n+2)!} = \lim_{n\to\infty} \frac{2(n+1)}{n+3} = 2 > 1\). Diverges.
    \item[\textbf{44.}] \textbf{Converges Absolutely}. Direct Comparison. \(1 \le 2+(-1)^n \le 3\). So \(\frac{2+(-1)^n}{n^2} \le \frac{3}{n^2}\). Converges.
    \item[\textbf{45.}] \textbf{Converges Absolutely}. Ratio Test. Limit is \(1/27 < 1\).
    \item[\textbf{46.}] \textbf{Diverges}. Telescoping Series. \(S_N = (\sqrt{2}-1)+(\sqrt{3}-\sqrt{2})+\dots+(\sqrt{N+1}-\sqrt{N}) = \sqrt{N+1}-1\). \(\lim_{N\to\infty} S_N = \infty\).
    \item[\textbf{47.}] \textbf{Converges Absolutely}. Root Test. \(\lim_{n\to\infty} \frac{4}{n+1} = 0 < 1\).
    \item[\textbf{48.}] \textbf{Absolutely Convergent}. The series for \(e^{-1}\) is \(\sum_{n=0}^{\infty} \frac{(-1)^n}{n!}\). It converges absolutely.
    \item[\textbf{49.}] \textbf{Conditionally Convergent}. AST: \(b_n = 1/(1+\ln n)\) is decreasing with limit 0. Absolute series \(\sum 1/(1+\ln n)\) diverges by LCT with \(\sum 1/n\).
    \item[\textbf{50.}] \textbf{Converges to 1/4}. Telescoping Series. Use partial fractions: \(\frac{1}{n(n+1)(n+2)} = \frac{1/2}{n} - \frac{1}{n+1} + \frac{1/2}{n+2}\). Sum is \(1/4\).
    \item[\textbf{51.}] \textbf{Converges Absolutely}. Root Test. \(\lim_{n\to\infty} \frac{n^2+1}{2n^2+1} = \frac{1}{2} < 1\).
    \item[\textbf{52.}] \textbf{Converges Absolutely}. Direct Comparison. \(|\cos n| \le 1\). So \(\frac{|\cos n|}{2^n} \le \frac{1}{2^n}\).
    \item[\textbf{53.}] \textbf{Conditionally Convergent}. AST: \(b_n = 1/(2n+1)\) is decreasing with limit 0. Absolute series diverges by LCT with \(\sum 1/n\).
    \item[\textbf{54.}] \textbf{Converges Absolutely}. Ratio Test. \(\lim_{n\to\infty} \frac{(n+2)5^{n+1}/((n+1)3^{2n+2})}{(n+1)5^n/(n 3^{2n})} = \lim_{n\to\infty} \frac{(n+2)n}{(n+1)^2} \frac{5}{9} = \frac{5}{9} < 1\).
    \item[\textbf{55.}] \textbf{Converges Absolutely}. Root Test. \(\lim_{n\to\infty} (\sqrt[n]{n}-1) = 1-1 = 0 < 1\).
    \item[\textbf{56.}] \textbf{Absolutely Convergent}. Absolute series \(\sum \frac{\arctan n}{n^2}\). Use Direct Comparison: \(\frac{\arctan n}{n^2} < \frac{\pi/2}{n^2}\).
    \item[\textbf{57.}] \textbf{Converges to 2}. Telescoping Series (for \(n \ge 2\)). \(\frac{1}{n(2n-1)} = \frac{2}{2n-1} - \frac{1}{n}\) is not telescoping. Try \(\frac{1}{n(n-1)} = \frac{1}{n-1}-\frac{1}{n}\). The sum for \(\sum_{n=2} \frac{1}{n(n-1)}\) is 1. The original series converges by LCT with \(1/n^2\).
    \item[\textbf{58.}] \textbf{Converges Absolutely}. Ratio Test. \(\lim_{n\to\infty} \frac{3^{n+1}(n+1)^2/(n+1)!}{3^n n^2 / n!} = \lim_{n\to\infty} \frac{3(n+1)^2}{(n+1)n^2} = \lim_{n\to\infty} \frac{3(n+1)}{n^2} = 0 < 1\).
    \item[\textbf{59.}] \textbf{Conditionally Convergent}. AST: \(b_n = \frac{n-1}{n^2+1}\) is decreasing for \(n\ge2\) with limit 0. Absolute series diverges by LCT with \(\sum 1/n\).
    \item[\textbf{60.}] \textbf{Converges Absolutely}. Ratio Test. \(\lim_{n\to\infty} \frac{5^{n+1}/((n+1)!)^2}{5^n/(n!)^2} = \lim_{n\to\infty} \frac{5}{(n+1)^2} = 0 < 1\).
    \item[\textbf{61.}] \textbf{Converges to 1/2}. Telescoping Series. \(S_N = (\frac{1}{2}-\frac{1}{4})+(\frac{1}{4}-\frac{1}{8})+\dots+(\frac{1}{2^N}-\frac{1}{2^{N+1}}) = \frac{1}{2}-\frac{1}{2^{N+1}}\). Sum is 1/2.
    \item[\textbf{62.}] \textbf{Diverges}. Ratio Test. \(\lim_{n\to\infty} \frac{(2n+2)!/ (2^{n+1}(n+1)!) }{(2n)! / (2^n n!)} = \lim_{n\to\infty} \frac{(2n+2)(2n+1)}{2(n+1)} = \lim_{n\to\infty} (2n+1) = \infty\).
    \item[\textbf{63.}] \textbf{Diverges}. Test for Divergence. \(\lim_{n\to\infty} (1+k/n)^n = e^k \neq 0\).
    \item[\textbf{64.}] \textbf{Absolutely Convergent}. Absolute series \(\sum n/(n^3+1)\) converges by LCT with \(\sum 1/n^2\).
    \item[\textbf{65.}] \textbf{Absolutely Convergent}. Note \(\sin(n\pi/2)\) sequence is \(1, 0, -1, 0, 1, \dots\). Absolute series is \(\sum_{k=0}^{\infty} \frac{1}{(2k+1)!}\). Compare with \(\sum 1/n!\), which converges.
    \item[\textbf{66.}] \textbf{Converges Absolutely}. Ratio Test. \(\lim_{n\to\infty} \frac{(n+1)!/e^{(n+1)^2}}{n!/e^{n^2}} = \lim_{n\to\infty} \frac{n+1}{e^{2n+1}} = 0 < 1\).
    \item[\textbf{67.}] \textbf{Converges to \(\cos(1)-1\)}. Telescoping Series. \(S_N = (\cos(1)-\cos(1/2)) + \dots = \cos(1/(N+1)) - \cos(1)\) is incorrect. \(S_N = \cos(1) - \cos(1/(N+1))\). Sum is \(\cos(1)-1\).
    \item[\textbf{68.}] \textbf{Diverges}. Test for Divergence (alternating). \(\lim_{n\to\infty} n!/100^n = \infty \neq 0\).
    \item[\textbf{69.}] \textbf{Converges Absolutely}. Root Test. \(\lim_{n\to\infty} (\frac{n}{n+1})^n = \lim_{n\to\infty} (1-\frac{1}{n+1})^n = 1/e < 1\).
    \item[\textbf{70.}] \textbf{Absolutely Convergent}. Absolute series \(\sum 1/n^{1.01}\) is a convergent p-series.
    \item[\textbf{71.}] \textbf{Diverges}. Ratio Test. \(\lim_{n\to\infty} \frac{4(n+1)!/(... \cdot (3n+1))}{4n!/(... \cdot (3n-2))} = \lim_{n\to\infty} \frac{4(n+1)}{3n+1} = 4/3 > 1\).
    \item[\textbf{72.}] \textbf{Diverges}. Test for Divergence. \(\lim_{n\to\infty} \frac{n^2+1}{2n^2+n-1} = 1/2 \neq 0\).
    \item[\textbf{73.}] \textbf{Diverges}. Test for Divergence. \(a_n = \frac{2^n n!}{n!} = 2^n \to \infty\).
    \item[\textbf{74.}] \textbf{Converges Absolutely}. Ratio Test. \(\lim_{n\to\infty} \frac{(2n+2)!/(n+1)^{2n+2}}{(2n)!/n^{2n}} = \lim_{n\to\infty} \frac{(2n+2)(2n+1)}{(n+1)^2} (\frac{n}{n+1})^{2n} = 4(1/e^2) < 1\).
    \item[\textbf{75.}] \textbf{Converges Absolutely}. Ratio Test. Limit of ratio is \(3/n \to 0\).
    \item[\textbf{76.}] \textbf{Converges Absolutely}. LCT with \(\sum 1/n^2\).
    \item[\textbf{77.}] \textbf{Analysis for p-series}. Conditionally convergent for \(0<p\le 1\), absolutely convergent for \(p>1\).
    \item[\textbf{78.}] \textbf{Converges Absolutely}. Root Test. \(\lim_{n\to\infty} (\frac{(n!)^n}{n^{n^2}})^{1/n} = \lim_{n\to\infty} \frac{n!}{n^n} = 0 < 1\).
    \item[\textbf{79.}] \textbf{Converges Absolutely}. Ratio Test. Limit of ratio is 0.
    \item[\textbf{80.}] \textbf{Converges to 1/4}. Same as problem 50.
    \item[\textbf{81.}] \textbf{Converges Absolutely}. Comparison with \(\sum 1/n^2\), since \(n! > n^2\) for \(n \ge 4\).
    \item[\textbf{82.}] \textbf{Converges Absolutely}. LCT with \(\sum 1/n^2\).
    \item[\textbf{83.}] \textbf{Diverges}. LCT with \(\sum 1/n\). Note \(\lim_{n\to\infty} \frac{\sqrt[n]{2}-1}{1/n} = \ln 2\).
    \item[\textbf{84.}] \textbf{Conditionally Convergent}. AST holds. Absolute series \(\sum 1/n^{1/4}\) diverges.
    \item[\textbf{85.}] \textbf{Diverges}. Test for Divergence.
    \item[\textbf{86.}] \textbf{Converges Absolutely}. AST holds. Absolute series \(\sum e^{-n}/n\) converges by comparison with \(\sum e^{-n}\).
    \item[\textbf{87.}] \textbf{Diverges}. Ratio Test. Limit is \(27/4 > 1\).
    \item[\textbf{88.}] \textbf{Diverges}. Root Test. Limit is \(3/2 > 1\).
    \item[\textbf{89.}] \textbf{Conditionally Convergent}. AST holds. Absolute series diverges by integral test.
    \item[\textbf{90.}] \textbf{Converges to 1/3}. Telescoping. Simplify \(a_n = \frac{n \cdot n!}{(n+2)!} = \frac{n}{(n+2)(n+1)}\). Use partial fractions.
    \item[\textbf{91.}] \textbf{Converges Absolutely}. LCT with convergent geometric series \(\sum (4/5)^n\).
    \item[\textbf{92.}] \textbf{Converges Absolutely to 0}. Note \(\sin(n\pi) = 0\) for all integers n. All terms are 0.
    \item[\textbf{93.}] \textbf{Diverges}. Test for Divergence.
    \item[\textbf{94.}] \textbf{Conditionally Convergent}. AST holds. Absolute series diverges by LCT with \(\sum 1/n\).
    \item[\textbf{95.}] \textbf{Diverges}. Test for Divergence. \(\lim_{n\to\infty} 1/\sqrt[n]{n!} = \lim_{n\to\infty} n/n! = \infty\) is wrong. Use \(n! \approx (n/e)^n\). \(\lim 1/(n/e) = 0\). Test is inconclusive. Diverges by LCT with \(\sum 1/n\).
    \item[\textbf{96.}] \textbf{Diverges}. Test for Divergence. \(\lim_{n\to\infty} 1/\arctan(n) = 2/\pi \neq 0\).
    \item[\textbf{97.}] \textbf{Converges Absolutely}. Ratio Test. Limit is \(2/4=1/2 < 1\).
    \item[\textbf{98.}] \textbf{Converges to 1}. Telescoping. Sum is 1.
    \item[\textbf{99.}] \textbf{Diverges}. Test for Divergence.
    \item[\textbf{100.}] \textbf{Converges Absolutely}. Root Test. Limit is \(\ln(n)/n \to 0\).
    \item[\textbf{101.}] \textbf{Conditionally Convergent}. AST holds. Absolute series diverges by LCT with \(\sum 1/n\).
    \item[\textbf{102.}] \textbf{Converges Absolutely}. Ratio Test. Limit is 0.
    \item[\textbf{103.}] \textbf{Converges Absolutely}. Absolute value is \(\sum \frac{n^2 \cos n}{n^4+1}\). Use comparison \(\le \frac{n^2}{n^4+1}\), which converges by LCT with \(\sum 1/n^2\).
    \item[\textbf{104.}] \textbf{Conditionally Convergent}. AST holds. Absolute series diverges by LCT with \(\sum 1/n\).
    \item[\textbf{105.}] \textbf{Converges Absolutely}. Ratio Test. Limit is \(1/e < 1\).
    \item[\textbf{106.}] \textbf{Absolutely Convergent}. Telescoping series. Converges.
    \item[\textbf{107.}] \textbf{Diverges}. Test for Divergence. \(\lim (1+1/n^2)^n = 1\).
    \item[\textbf{108.}] \textbf{Diverges}. Ratio Test. Limit is \(2 > 1\).
    \item[\textbf{109.}] \textbf{Converges Absolutely}. Ratio Test. Limit is 0.
    \item[\textbf{110.}] \textbf{Converges Absolutely}. Root Test or Direct comparison with \(1/n^n \le 1/2^n\).
    \item[\textbf{111.}] \textbf{Diverges}. Test for Divergence. Limit is 1.
    \item[\textbf{112.}] \textbf{Diverges}. Use conjugate. \(a_n = \frac{1}{\sqrt{n^2+1}+n}\). LCT with \(1/n\). Diverges.
    \item[\textbf{113.}] \textbf{Converges Absolutely}. Ratio Test with absolute series. Limit is \(1/4\).
    \item[\textbf{114.}] \textbf{Diverges}. Integral test.
    \item[\textbf{115.}] \textbf{Converges Absolutely}. Note \(n/n! = 1/(n-1)!\). Comparison with \(1/n^2\).
    \item[\textbf{116.}] \textbf{Converges to 1}. Telescoping series.
    \item[\textbf{117.}] \textbf{Diverges}. Test for Divergence. Limit of terms is \(-1, -1, -1, \dots\).
    \item[\textbf{118.}] \textbf{Converges Absolutely}. Ratio Test. Limit is 0.
    \item[\textbf{119.}] \textbf{Absolutely Convergent}. LCT with \(\sum 1/n^2\).
    \item[\textbf{120.}] \textbf{Converges Absolutely}. Ratio Test. \(\lim \frac{1/2^{n^2+2n+1}}{1/2^{n^2}} = 0\).
    \item[\textbf{121.}] \textbf{Converges Absolutely}. AST. Absolute series \(\sum \frac{(2n)!}{4^n(n!)^2}\). Ratio test gives limit 1 (inconclusive). Use Stirling's approx. Diverges.
    \item[\textbf{122.}] \textbf{Diverges}. Ratio Test. Limit is \(\infty\).
    \item[\textbf{123.}] \textbf{Diverges}. Integral Test.
    \item[\textbf{124.}] \textbf{Converges Absolutely}. Ratio Test. Limit is 0.
    \item[\textbf{125.}] \textbf{Diverges}. Use conjugate. \(a_n = \frac{1}{(\dots)^{2/3}+(\dots)^{1/3}(\dots)^{1/3}+(\dots)^{2/3}}\). LCT with \(1/n^{2/3}\).
    \item[\textbf{126.}] \textbf{Absolutely Convergent}. The absolute series converges by Direct Comparison Test with \(\sum 1/n^2\).
    \item[\textbf{127.}] \textbf{Converges Absolutely}. Root Test. \(\lim (2n/n^2)^{1/n} = \lim (2/n)^{1/n} = 1\). Root test inconclusive. Ratio Test: \(\lim \frac{(2(n+1))^{n+1}/(n+1)^{2n+2}}{ (2n)^n/n^{2n} } = \lim \frac{2(n+1)}{n^2} (1+1/n)^n (n/(n+1))^{2n} = 0\).
    \item[\textbf{128.}] \textbf{Absolutely Convergent}. LCT with \(\sum 1/n^{1.5}\).
    \item[\textbf{129.}] \textbf{Converges to 2}. Telescoping. \(a_n = \frac{2}{n} - \frac{2}{n+2}\). Sum is \(2(1+1/2)=3\).
    \item[\textbf{130.}] \textbf{Diverges}. Test for Divergence. \(\lim 3n!/(n+1)! = \lim 3/(n+1)=0\). The test is inconclusive. LCT with \(\sum 1/n\). Diverges.
    \item[\textbf{131.}] \textbf{Absolutely Convergent}. Geometric series with \(r=-1/e\).
    \item[\textbf{132.}] \textbf{Converges Absolutely}. Ratio Test.
    \item[\textbf{133.}] \textbf{Diverges}. LCT with \(\sum 1/n\).
    \item[\textbf{134.}] \textbf{Converges Absolutely}. Ratio test. Limit is 0.
    \item[\textbf{135.}] \textbf{Diverges}. LCT with \(\sum 1/n\).
    \item[\textbf{136.}] \textbf{Conditionally Convergent}. AST holds. Absolute series diverges by LCT with \(1/n\).
    \item[\textbf{137.}] \textbf{Converges Absolutely}. Root Test. Limit is \(3/e > 1\). Wait, \(3/e^n\). Limit is \(3/\infty = 0\). Converges.
    \item[\textbf{138.}] \textbf{Converges Absolutely}. LCT with \(\sum 1/n^2\).
    \item[\textbf{139.}] \textbf{Converges Absolutely}. Ratio Test. Limit is \(2/3 < 1\).
    \item[\textbf{140.}] \textbf{Diverges}. Test for Divergence. Limit is 1.
    \item[\textbf{141.}] \textbf{Diverges}. Test for Divergence. The terms are always \(\ge 1/3\).
    \item[\textbf{142.}] \textbf{Converges Absolutely}. Ratio Test. Limit is 0.
    \item[\textbf{143.}] \textbf{Conditionally Convergent}. AST holds. Absolute series diverges by LCT with \(1/n\).
    \item[\textbf{144.}] \textbf{Converges Absolutely}. LCT with \(\sum 1/2^n\).
    \item[\textbf{145.}] \textbf{Converges Absolutely}. Root Test. \(\lim (2n+1)/n^2 = 0 < 1\).
    \item[\textbf{146.}] \textbf{Diverges}. LCT with \(\sum 1/n\).
    \item[\textbf{147.}] \textbf{Converges Absolutely}. LCT with \(\sum 1/5^n\).
    \item[\textbf{148.}] \textbf{Diverges}. LCT with \(\sum 1/n\).
    \item[\textbf{149.}] \textbf{Converges Absolutely}. Ratio Test. Limit is \(1/3\).
    \item[\textbf{150.}] \textbf{Converges Absolutely}. LCT with \(\sum 1/3^n\).
    \item[\textbf{151.}] \textbf{Converges to 3/2}. Telescoping.
    \item[\textbf{152.}] \textbf{Conditionally Convergent}. AST holds. Absolute series diverges by LCT with \(1/n\).
    \item[\textbf{153.}] \textbf{Converges Absolutely}. Ratio test. Limit is 0.
    \item[\textbf{154.}] \textbf{Converges Absolutely}. LCT with \(\sum 1/n^{1.5}\).
    \item[\textbf{155.}] \textbf{Diverges}. Integral test.
    \item[\textbf{156.}] \textbf{Converges Absolutely}. Ratio test. Limit is \(2/n \to 0\).
    \item[\textbf{157.}] \textbf{Diverges}. Test for Divergence. Limit is 1.
    \item[\textbf{158.}] \textbf{Diverges}. Test for Divergence. Limit of \(b_n\) is 1.
    \item[\textbf{159.}] \textbf{Converges Absolutely}. Ratio test. Limit is \(1/4\).
    \item[\textbf{160.}] \textbf{Converges Absolutely}. LCT with \(\sum 1/n^{1.5}\).
    \item[\textbf{161.}] \textbf{Converges Absolutely}. Direct comparison with \(\sum 3/n^2\).
    \item[\textbf{162.}] \textbf{Converges Absolutely}. Ratio test. Limit is 1/4.
    \item[\textbf{163.}] \textbf{Converges Absolutely}. LCT with \(\sum 1/n^2\).
    \item[\textbf{164.}] \textbf{Diverges}. p-series, \(p=1/2 \le 1\).
    \item[\textbf{165.}] \textbf{Conditionally Convergent}. AST holds. Absolute series diverges.
    \item[\textbf{166.}] \textbf{Converges Absolutely}. Ratio test. Limit is 0.
    \item[\textbf{167.}] \textbf{Converges Absolutely}. LCT with \(\sum 1/n^{1.5}\).
    \item[\textbf{168.}] \textbf{Converges Absolutely}. Ratio test. Limit is 0.
    \item[\textbf{169.}] \textbf{Converges Absolutely}. Root Test. Limit is 0.
    \item[\textbf{170.}] \textbf{Diverges}. LCT with \(\sum 1/n\).
    
\end{enumerate}
%%%%%%%%%%%%%%%%%%%%%%%%%%%%%%%%%%%%%%%%%%%%%%%%%%%%%%
% SOLUTIONS SECTION
%%%%%%%%%%%%%%%%%%%%%%%%%%%%%%%%%%%%%%%%%%%%%%%%%%%%%%
\section{Solutions - Power Series}
\begin{enumerate}
    \item[\textbf{1.}] \(f(x) = \frac{1}{1-(-x)} = \sum_{n=0}^{\infty} (-x)^n = \sum_{n=0}^{\infty} (-1)^n x^n\). Converges for \(|-x|<1 \Rightarrow |x|<1\). Endpoints \(x=1\) (\(\sum (-1)^n\)) and \(x=-1\) (\(\sum 1\)) diverge. \textbf{R=1, I=(-1, 1)}.
    
    \item[\textbf{2.}] \(f(x) = \frac{1}{1-(x^2)} = \sum_{n=0}^{\infty} (x^2)^n = \sum_{n=0}^{\infty} x^{2n}\). Converges for \(|x^2|<1 \Rightarrow |x|<1\). Endpoints \(x=\pm 1\) (\(\sum 1\)) diverge. \textbf{R=1, I=(-1, 1)}.
    
    \item[\textbf{3.}] \(f(x) = \frac{1}{1-(-4x^2)} = \sum_{n=0}^{\infty} (-4x^2)^n = \sum_{n=0}^{\infty} (-1)^n 4^n x^{2n}\). Converges for \(|-4x^2|<1 \Rightarrow |x|<1/2\). Endpoints \(x=\pm 1/2\) (\(\sum (-1)^n\)) diverge. \textbf{R=1/2, I=(-1/2, 1/2)}.
    
    \item[\textbf{4.}] \(f(x) = \frac{1}{4(1-x/4)} = \frac{1}{4} \sum_{n=0}^{\infty} \left(\frac{x}{4}\right)^n = \sum_{n=0}^{\infty} \frac{x^n}{4^{n+1}}\). Converges for \(|x/4|<1 \Rightarrow |x|<4\). Endpoints diverge. \textbf{R=4, I=(-4, 4)}.
    
    \item[\textbf{5.}] \(f(x) = x \frac{1}{1-x} = x \sum_{n=0}^{\infty} x^n = \sum_{n=0}^{\infty} x^{n+1}\). Converges for \(|x|<1\). Endpoints diverge. \textbf{R=1, I=(-1, 1)}.
    
    \item[\textbf{6.}] \(f(x) = x^2 \frac{1}{1-(-x^3)} = x^2 \sum_{n=0}^{\infty} (-x^3)^n = \sum_{n=0}^{\infty} (-1)^n x^{3n+2}\). Converges for \(|-x^3|<1 \Rightarrow |x|<1\). Endpoints diverge. \textbf{R=1, I=(-1, 1)}.
    
    \item[\textbf{7.}] \(f(x) = \frac{3}{2(1+x/2)} = \frac{3}{2} \sum_{n=0}^{\infty} \left(-\frac{x}{2}\right)^n = \sum_{n=0}^{\infty} \frac{3(-1)^n x^n}{2^{n+1}}\). Converges for \(|-x/2|<1 \Rightarrow |x|<2\). Endpoints diverge. \textbf{R=2, I=(-2, 2)}.
    
    \item[\textbf{8.}] Differentiate \(g(x) = \frac{1}{1-x} = \sum_{n=0}^{\infty} x^n\). \(f(x) = g'(x) = \sum_{n=1}^{\infty} nx^{n-1}\). Radius is the same as for g(x). \textbf{R=1, I=(-1, 1)}.
    
    \item[\textbf{9.}] Start with \(g(x) = \frac{1}{1+x} = \sum_{n=0}^{\infty} (-1)^n x^n\). Differentiate twice. \(g'(x) = \frac{-1}{(1+x)^2} = \sum_{n=1}^{\infty} (-1)^n nx^{n-1}\). \(g''(x) = \frac{2}{(1+x)^3} = \sum_{n=2}^{\infty} (-1)^n n(n-1)x^{n-2}\). So \(f(x) = \frac{1}{2}g''(x) = \sum_{n=2}^{\infty} \frac{(-1)^n n(n-1)}{2} x^{n-2}\). \textbf{R=1, I=(-1, 1)}.
    
    \item[\textbf{10.}] Start with \(g(x) = \frac{1}{1-x^2} = \sum_{n=0}^{\infty} x^{2n}\). Differentiate: \(g'(x) = \frac{2x}{(1-x^2)^2} = \sum_{n=1}^{\infty} 2nx^{2n-1}\). Then \(f(x) = \frac{1}{2}g'(x) = \sum_{n=1}^{\infty} nx^{2n-1}\). \textbf{R=1, I=(-1, 1)}.
    
    \item[\textbf{11.}] Integrate \(g(x) = \frac{1}{1-x} = \sum_{n=0}^{\infty} x^n\). \(\int \frac{1}{1-x}dx = -\ln(1-x) = C + \sum_{n=0}^{\infty} \frac{x^{n+1}}{n+1}\). At \(x=0\), \(\ln(1)=0=C\). So \(f(x) = -\sum_{n=0}^{\infty} \frac{x^{n+1}}{n+1} = -\sum_{k=1}^{\infty} \frac{x^k}{k}\). \textbf{R=1, I=[-1, 1)}.
    
    \item[\textbf{12.}] Integrate \(g(x) = \frac{1}{1+x^2} = \sum_{n=0}^{\infty} (-1)^n x^{2n}\). \(\int \frac{2x}{1+x^2}dx = \ln(1+x^2)\). So, integrate \(2x g(x) = \sum_{n=0}^{\infty} 2(-1)^n x^{2n+1}\). \(f(x) = C + \sum_{n=0}^{\infty} \frac{2(-1)^n x^{2n+2}}{2n+2} = \sum_{n=0}^{\infty} \frac{(-1)^n x^{2n+2}}{n+1}\). At \(x=0\), C=0. \textbf{R=1, I=[-1, 1]}.
    
    \item[\textbf{13.}] Integrate \(g(x) = \frac{1}{1+x^2} = \sum_{n=0}^{\infty} (-1)^n x^{2n}\). \(f(x) = \int g(x) dx = C + \sum_{n=0}^{\infty} \frac{(-1)^n x^{2n+1}}{2n+1}\). At \(x=0\), C=0. \textbf{R=1, I=[-1, 1]}.
    
    \item[\textbf{14.}] Start with \(\arctan(u) = \sum_{n=0}^{\infty} \frac{(-1)^n u^{2n+1}}{2n+1}\). Let \(u=x^2\). \(\arctan(x^2) = \sum_{n=0}^{\infty} \frac{(-1)^n x^{4n+2}}{2n+1}\). Then \(f(x) = x \arctan(x^2) = \sum_{n=0}^{\infty} \frac{(-1)^n x^{4n+3}}{2n+1}\). \textbf{R=1, I=[-1, 1]}.
    
    \item[\textbf{15.}] \(f(x) = \ln(5(1+x/5)) = \ln(5) + \ln(1+x/5)\). Use series for \(\ln(1+u)\) with \(u=x/5\). \(f(x) = \ln(5) + \sum_{n=1}^{\infty} \frac{(-1)^{n-1}(x/5)^n}{n} = \ln(5) + \sum_{n=1}^{\infty} \frac{(-1)^{n-1}x^n}{n 5^n}\). \textbf{R=5, I=(-5, 5]}.
    
    \item[\textbf{16.}] \(f(x) = -\frac{1}{5-x} = -\frac{1}{5(1-x/5)} = -\frac{1}{5} \sum_{n=0}^{\infty} (\frac{x}{5})^n = -\sum_{n=0}^{\infty} \frac{x^n}{5^{n+1}}\). \textbf{R=5, I=(-5, 5)}.
    
    \item[\textbf{17.}] \(f(x) = x^3 \frac{1}{1-(-x)} = x^3 \sum_{n=0}^{\infty} (-1)^n x^n = \sum_{n=0}^{\infty} (-1)^n x^{n+3}\). \textbf{R=1, I=(-1, 1)}.
    
    \item[\textbf{18.}] Differentiate \(g(x)=\frac{-1/2}{1+2x} = -\frac{1}{2} \sum (-2x)^n\). \(g'(x) = \frac{1}{(1+2x)^2} = \sum_{n=1}^{\infty} (-1)^{n+1} n (2x)^{n-1} = \sum_{n=1}^{\infty} (-1)^{n+1} n 2^{n-1} x^{n-1}\). \textbf{R=1/2, I=(-1/2, 1/2)}.
    
    \item[\textbf{19.}] Use series for \(\ln(1+x) = \sum_{n=1}^{\infty} \frac{(-1)^{n-1}x^n}{n}\). Then \(f(x) = x \sum_{n=1}^{\infty} \frac{(-1)^{n-1}x^n}{n} = \sum_{n=1}^{\infty} \frac{(-1)^{n-1}x^{n+1}}{n}\). \textbf{R=1, I=(-1, 1]}.
    
    \item[\textbf{20.}] Start with \(\frac{1}{1+x^4} = \sum_{n=0}^{\infty} (-x^4)^n = \sum_{n=0}^{\infty} (-1)^n x^{4n}\). Integrate: \(f(x) = C + \sum_{n=0}^{\infty} \frac{(-1)^n x^{4n+1}}{4n+1}\). \textbf{R=1, I=[-1, 1]}.
    
    % ... The remaining solutions follow a similar pattern of derivation and analysis.
    % To save space, the rest will be more condensed but follow the same logic.
    
    \item[\textbf{21.}] Partial Fractions: \(\frac{1}{1-x} + \frac{1}{1+x} = \sum x^n + \sum (-1)^n x^n = \sum (1+(-1)^n)x^n = \sum_{k=0}^{\infty} 2x^{2k}\). \textbf{R=1, I=(-1,1)}.
    \item[\textbf{22.}] \(f(x) = \frac{x}{2(1-x/2)} = \frac{x}{2} \sum (x/2)^n = \sum_{n=0}^{\infty} \frac{x^{n+1}}{2^{n+1}}\). \textbf{R=2, I=(-2,2)}.
    \item[\textbf{23.}] Differentiate \(\sum (-1)^n x^{3n}\): \(f(x) = \sum_{n=1}^{\infty} (-1)^n 3n x^{3n-1}\). \textbf{R=1, I=(-1,1)}.
    \item[\textbf{24.}] Integrate \(\frac{-4x^3}{1-x^4}\): \(f(x) = \int \sum -4x^{4n+3} dx = \sum_{n=0}^{\infty} \frac{-4x^{4n+4}}{4n+4} = -\sum_{n=0}^{\infty} \frac{x^{4n+4}}{n+1}\). \textbf{R=1, I=[-1,1)}.
    \item[\textbf{25.}] Long division: \( -1 + \frac{2}{1-x} = -1 + 2\sum x^n = 1 + \sum_{n=1}^{\infty} 2x^n\). \textbf{R=1, I=(-1,1)}.
    \item[\textbf{26.}] \(f(x) = \frac{x^2}{16(1-(x/4)^2)} = \frac{x^2}{16} \sum (x/4)^{2n} = \sum_{n=0}^{\infty} \frac{x^{2n+2}}{16^{n+1}}\). \textbf{R=4, I=(-4,4)}.
    \item[\textbf{27.}] \(f(x) = \frac{1}{9(1+(x/3)^2)} = \frac{1}{9} \sum (-1)^n (x/3)^{2n} = \sum_{n=0}^{\infty} \frac{(-1)^n x^{2n}}{9^{n+1}}\). \textbf{R=3, I=(-3,3)}.
    \item[\textbf{28.}] Integrate \(\ln(1+x^2)\): \(f(x) = x \sum (-1)^n x^{2n} = \sum_{n=0}^{\infty} (-1)^n x^{2n+1}\). \textbf{R=1, I=(-1,1)}.
    \item[\textbf{29.}] Differentiate \(\frac{1/3}{1-3x}\): \(\frac{1}{(1-3x)^2} = \frac{1}{3} \frac{d}{dx} \sum (3x)^n = \sum_{n=1}^{\infty} n 3^{n-1} x^{n-1}\). \textbf{R=1/3, I=(-1/3,1/3)}.
    \item[\textbf{30.}] \(f(x) = x^2(-\sum_{n=1}^{\infty} x^n/n) = -\sum_{n=1}^{\infty} \frac{x^{n+2}}{n}\). \textbf{R=1, I=[-1,1)}.
    \item[\textbf{31.}] Substitute \(u=x/3\) into arctan series: \(f(x) = \sum_{n=0}^{\infty} \frac{(-1)^n (x/3)^{2n+1}}{2n+1} = \sum_{n=0}^{\infty} \frac{(-1)^n x^{2n+1}}{(2n+1)3^{2n+1}}\). \textbf{R=3, I=[-3,3]}.
    \item[\textbf{32.}] Partial Fractions: \(\frac{1}{1-2x} - \frac{1}{1-x} = \sum (2x)^n - \sum x^n = \sum_{n=0}^{\infty} (2^n-1)x^n\). \textbf{R=1/2, I=(-1/2,1/2)}.
    \item[\textbf{33.}] Partial Fractions: \(\frac{1/2}{x+1} - \frac{1/2}{x+3} = \frac{1}{2}\sum(-x)^n - \frac{1}{6}\sum(-x/3)^n = \sum_{n=0}^{\infty} \frac{1}{2}\left((-1)^n - \frac{(-1)^n}{3^{n+1}}\right)x^n\). \textbf{R=1, I=(-1,1)}.
    \item[\textbf{34.}] \(\frac{1}{3(1-2x/3)} = \frac{1}{3} \sum (2x/3)^n = \sum_{n=0}^{\infty} \frac{2^n x^n}{3^{n+1}}\). \textbf{R=3/2, I=(-3/2,3/2)}.
    \item[\textbf{35.}] Same as problem 12. \textbf{R=1, I=[-1, 1]}.
    \item[\textbf{36.}] \(x \frac{1}{1-x^4} = x \sum (x^4)^n = \sum_{n=0}^{\infty} x^{4n+1}\). \textbf{R=1, I=(-1,1)}.
    \item[\textbf{37.}] Same as problem 9. \textbf{R=1, I=(-1, 1)}.
    \item[\textbf{38.}] \(e^x = \sum x^n/n!\). So \(f(x) = x \sum x^n/n! = \sum_{n=0}^{\infty} \frac{x^{n+1}}{n!}\). \textbf{R=\(\infty\), I=(-\(\infty\),\(\infty\))}.
    \item[\textbf{39.}] \(\cos(u) = \sum (-1)^n u^{2n}/(2n)!\). Let \(u=x^2\). \(f(x) = \sum_{n=0}^{\infty} \frac{(-1)^n x^{4n}}{(2n)!}\). \textbf{R=\(\infty\), I=(-\(\infty\),\(\infty\))}.
    \item[\textbf{40.}] \(\sin(x) = \sum (-1)^n x^{2n+1}/(2n+1)!\). \(f(x) = \frac{1}{x} \sum \dots = \sum_{n=0}^{\infty} \frac{(-1)^n x^{2n}}{(2n+1)!}\). \textbf{R=\(\infty\), I=(-\(\infty\),\(\infty\))}.
    \item[\textbf{41.}] \(-\frac{1}{5(1-2x/5)} = -\frac{1}{5} \sum (2x/5)^n = -\sum_{n=0}^{\infty} \frac{2^n x^n}{5^{n+1}}\). \textbf{R=5/2, I=(-5/2,5/2)}.
    \item[\textbf{42.}] \(\sum_{n=1}^{\infty} \frac{(-1)^{n-1}x^n}{n} - x = \sum_{n=2}^{\infty} \frac{(-1)^{n-1}x^n}{n}\). \textbf{R=1, I=(-1,1]}.
    \item[\textbf{43.}] Integrate \(x \arctan(x)\). Start with \(\arctan x = \sum \frac{(-1)^n x^{2n+1}}{2n+1}\). \(x \arctan x = \sum \frac{(-1)^n x^{2n+2}}{2n+1}\). Integrate: \(C + \sum_{n=0}^{\infty} \frac{(-1)^n x^{2n+3}}{(2n+1)(2n+3)}\). \textbf{R=1, I=[-1,1]}.
    \item[\textbf{44.}] Differentiate \(\ln(1-u)\) with \(u=x^2\). \(f(x) = \frac{-2x}{1-x^2} = -2x \sum (x^2)^n = \sum_{n=0}^{\infty} -2x^{2n+1}\). \textbf{R=1, I=(-1,1)}.
    \item[\textbf{45.}] \(1 + \frac{1}{2-x} = 1 + \frac{1}{2(1-x/2)} = 1 + \frac{1}{2}\sum (x/2)^n = 1 + \sum_{n=0}^{\infty} \frac{x^n}{2^{n+1}}\). \textbf{R=2, I=(-2,2)}.
    \item[\textbf{46.}] Use identity \(\sin^2 x = \frac{1-\cos(2x)}{2}\). \(\frac{1}{2} - \frac{1}{2} \sum_{n=0}^{\infty} \frac{(-1)^n (2x)^{2n}}{(2n)!} = \sum_{n=1}^{\infty} \frac{(-1)^{n+1} 2^{2n-1} x^{2n}}{(2n)!}\). \textbf{R=\(\infty\), I=(-\(\infty\),\(\infty\))}.
    \item[\textbf{47.}] Substitute \(u=-x^2\) into \(e^u = \sum u^n/n!\): \(f(x) = \sum_{n=0}^{\infty} \frac{(-x^2)^n}{n!} = \sum_{n=0}^{\infty} \frac{(-1)^n x^{2n}}{n!}\). \textbf{R=\(\infty\), I=(-\(\infty\),\(\infty\))}.
    \item[\textbf{48.}] \(-1 + \frac{2}{1-x^2} = -1 + 2\sum (x^2)^n = 1 + \sum_{n=1}^{\infty} 2x^{2n}\). \textbf{R=1, I=(-1,1)}.
    \item[\textbf{49.}] \(\ln(1+x)-\ln(1-x) = \sum \frac{(-1)^{n-1}x^n}{n} - \sum \frac{-x^n}{n} = \sum_{n=1}^{\infty} \frac{(-1)^{n-1}+1}{n}x^n = \sum_{k=0}^{\infty} \frac{2x^{2k+1}}{2k+1}\). \textbf{R=1, I=(-1,1)}.
    \item[\textbf{50.}] \(f(x) = x^2 (\frac{1}{1+x})^2 = x^2 \frac{d}{dx}(\frac{-1}{1+x}) = x^2 \sum_{n=1}^{\infty} (-1)^{n-1} n (-x)^{n-1} = \sum_{n=1}^{\infty} n(-1)^{n-1}x^{n+1}\). \textbf{R=1, I=(-1,1)}.
\end{enumerate}

\end{document}