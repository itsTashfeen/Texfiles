\documentclass{article}
\usepackage{amsmath}
\usepackage{graphicx}
\usepackage{amssymb}
\usepackage{geometry}
\geometry{a4paper, margin=1in}

\title{10.1: Parametric Equations - Problem Set}
\author{Tashfeen Omran}
\date{October 2025}
\begin{document}

\maketitle

\section*{Parametric Equations Problem Set}

\subsection*{Problems}
\begin{enumerate}
    % Evaluating Points
    \item For the parametric equations $x = 3t^2 - 1$, $y = t^3 - t$, find the coordinates of the point for $t = -2$.
    \item For the parametric equations $x = e^{2t}$, $y = \ln(t+1)$, find the coordinates of the point for $t = 0$.

    % Eliminating the Parameter (Algebraic)
    \item Eliminate the parameter to find the Cartesian equation for $x = 2t + 5$, $y = 4t - 1$.
    \item Eliminate the parameter to find the Cartesian equation for $x = \sqrt{t-3}$, $y = t+1$. State the domain for the resulting equation.
    \item Eliminate the parameter to find the Cartesian equation for $x = e^{-t}$, $y = 3e^{2t}$.
    \item Eliminate the parameter to find the Cartesian equation for $x = \frac{1}{t+1}$, $y = \frac{t}{t+1}$.

    % Eliminating the Parameter (Trigonometric)
    \item Eliminate the parameter to find the Cartesian equation for $x = 5\cos(t)$, $y = 5\sin(t)$.
    \item Eliminate the parameter to find the Cartesian equation for $x = 4\cos(t) + 1$, $y = 3\sin(t) - 2$.
    \item Eliminate the parameter to find the Cartesian equation for $x = 3\sec(t)$, $y = 4\tan(t)$.
    \item Eliminate the parameter to find the Cartesian equation for $x = \cos(2t)$, $y = \cos(t)$. (Hint: Use a double-angle identity).

    % Sketching and Orientation
    \item Sketch the curve for $x = t-1$, $y = t^2+4$ for $-1 \le t \le 2$. Indicate the orientation with an arrow.
    \item Sketch the curve for $x = t^3 - 3t$, $y = t^2$. Indicate the orientation.
    \item Sketch the curve for $x = 2\sin(t)$, $y = \cos^2(t)$. Indicate the orientation.
    \item Sketch the curve for $x = \sqrt{t}$, $y = t - 2$. What portion of the Cartesian curve is traced? Indicate the orientation.
    \item Sketch the curve for $x = 4\sin(t)$, $y = 4\cos(t)$ for $0 \le t \le \pi$. Indicate the orientation.
    \item Sketch the curve for $x = 1 + \ln(t)$, $y = t^2$ for $t > 0$. Indicate the orientation.
    \item The path of a particle is given by $x=2-t^2$, $y=t$. Sketch the curve and indicate the direction of motion as $t$ increases.
    \item Sketch the curve defined by $x = e^t$, $y = e^{-t}$. Indicate the orientation.

    % Analysis of Motion
    \item A particle moves according to $x = 6\cos(\pi t)$, $y = 6\sin(\pi t)$. How long does it take to complete one full revolution? Is the motion clockwise or counter-clockwise?
    \item A particle moves on an ellipse given by $x = 5\sin(t)$, $y = 2\cos(t)$, for $0 \le t \le 4\pi$. Describe the motion.
    \item The position of a particle is given by $x = 2t$, $y = \cos(\pi t)$. Describe the particle's horizontal and vertical motion. Is the overall motion periodic?
    \item A Lissajous figure is created by $x = \sin(t)$, $y = \sin(2t)$. Sketch the curve for $0 \le t \le 2\pi$.

    % Parameterizing a Cartesian Equation
    \item Find a set of parametric equations for the line $y = 7x - 3$.
    \item Find a set of parametric equations for the parabola $x = y^2 - 4y + 1$.
    \item Find a set of parametric equations for the ellipse $\frac{(x-2)^2}{25} + \frac{(y+4)^2}{9} = 1$.

    % Applications
    \item Find the parametric equations for the line segment starting at $(1, 6)$ and ending at $(-3, 2)$.
    \item A projectile is launched from ground level with an initial speed of 100 m/s at an angle of $30^\circ$. Using $g \approx 9.8 \text{ m/s}^2$, the parametric equations are $x(t) = (100\cos(30^\circ))t$ and $y(t) = (100\sin(30^\circ))t - \frac{1}{2}(9.8)t^2$. Find how long the projectile is in the air.
    \item The equations for a cycloid (the path traced by a point on a rolling circle of radius $r$) are $x = r(\theta - \sin\theta)$, $y = r(1-\cos\theta)$. Find the position of the point when the circle has rolled a quarter of a turn ($\theta = \pi/2$) if the radius is 2.

    % Advanced Concepts
    \item Two particles have paths given by $\mathbf{r}_1(t) = \langle t+3, t^2 \rangle$ and $\mathbf{r}_2(s) = \langle s-1, 2s \rangle$. Find any intersection points of their paths. Do they collide?
    \item For the curve given by $x = t^3 - 3t$ and $y = 3t^2 - 9$, find the slope of the tangent line at $t=2$.
\end{enumerate}

\newpage
\section*{Solutions}

\subsection*{Problem 1}
Given $x = 3t^2 - 1$, $y = t^3 - t$. For $t = -2$: \\
$x = 3(-2)^2 - 1 = 3(4) - 1 = 12 - 1 = 11$. \\
$y = (-2)^3 - (-2) = -8 + 2 = -6$. \\
The point is \textbf{(11, -6)}.

\subsection*{Problem 2}
Given $x = e^{2t}$, $y = \ln(t+1)$. For $t = 0$: \\
$x = e^{2(0)} = e^0 = 1$. \\
$y = \ln(0+1) = \ln(1) = 0$. \\
The point is \textbf{(1, 0)}.

\subsection*{Problem 3}
From $x = 2t + 5$, solve for $t$: $t = \frac{x-5}{2}$. \\
Substitute into the $y$ equation: $y = 4\left(\frac{x-5}{2}\right) - 1 = 2(x-5) - 1 = 2x - 10 - 1$. \\
The Cartesian equation is $\mathbf{y = 2x - 11}$.

\subsection*{Problem 4}
From $x = \sqrt{t-3}$, square both sides: $x^2 = t-3$, so $t = x^2+3$. \\
Substitute into the $y$ equation: $y = (x^2+3)+1$. \\
The Cartesian equation is $\mathbf{y = x^2+4}$. \\
Since $x = \sqrt{t-3}$, $x$ must be non-negative. The domain is $\mathbf{x \ge 0}$.

\subsection*{Problem 5}
From $x = e^{-t}$, we can write $t = -\ln(x)$.
Alternatively, notice $x = e^{-t} \implies \frac{1}{x} = e^t$.
Also $y = 3e^{2t} = 3(e^t)^2$.
Substitute $e^t = \frac{1}{x}$: $y = 3\left(\frac{1}{x}\right)^2$.
The Cartesian equation is $\mathbf{y = \frac{3}{x^2}}$.

\subsection*{Problem 6}
From $x = \frac{1}{t+1}$, solve for $t$: $x(t+1) = 1 \implies xt + x = 1 \implies t = \frac{1-x}{x}$. \\
Substitute into the $y$ equation: $y = \frac{\frac{1-x}{x}}{\frac{1-x}{x}+1} = \frac{\frac{1-x}{x}}{\frac{1-x+x}{x}} = \frac{\frac{1-x}{x}}{\frac{1}{x}} = 1-x$. \\
A simpler way: Notice that $x+y = \frac{1}{t+1} + \frac{t}{t+1} = \frac{1+t}{t+1} = 1$.
The Cartesian equation is $\mathbf{y = 1-x}$.

\subsection*{Problem 7}
Recognize that this fits the Pythagorean identity.
$\cos(t) = x/5$ and $\sin(t) = y/5$. \\
Since $\cos^2(t) + \sin^2(t) = 1$, we have $(\frac{x}{5})^2 + (\frac{y}{5})^2 = 1$. \\
The Cartesian equation is $\mathbf{x^2 + y^2 = 25}$, a circle centered at the origin with radius 5.

\subsection*{Problem 8}
Isolate the trigonometric terms: $\cos(t) = \frac{x-1}{4}$ and $\sin(t) = \frac{y+2}{3}$. \\
Using $\cos^2(t) + \sin^2(t) = 1$: \\
$\left(\frac{x-1}{4}\right)^2 + \left(\frac{y+2}{3}\right)^2 = 1$. \\
This is the equation of an ellipse centered at $(1, -2)$.

\subsection*{Problem 9}
Isolate the trigonometric terms: $\sec(t) = x/3$ and $\tan(t) = y/4$. \\
Use the identity $\sec^2(t) - \tan^2(t) = 1$. \\
$\left(\frac{x}{3}\right)^2 - \left(\frac{y}{4}\right)^2 = 1$. \\
This is the equation of a hyperbola.

\subsection*{Problem 10}
Use the double-angle identity for cosine: $\cos(2t) = 2\cos^2(t) - 1$. \\
From the parametric equations, we have $x = \cos(2t)$ and $y = \cos(t)$. \\
Substitute these into the identity: $x = 2y^2 - 1$.
This is the equation of a parabola opening to the right. Since $y=\cos(t)$, $-1 \le y \le 1$.

\subsection*{Problem 11}
Points: $t=-1 \implies (-2, 5)$, $t=0 \implies (-1, 4)$, $t=2 \implies (1, 8)$. The curve is a parabola ($y = (x+1)^2+4$) opening upwards. The orientation is from left to right.
\begin{center} \includegraphics[width=0.5\textwidth]{{problem11_sketch.png}} \end{center}

\subsection*{Problem 12}
This is a self-intersecting curve. At $t=0$, point is $(0,0)$. At $t=\pm\sqrt{3}$, $x=0$, so it crosses the y-axis. The curve starts from the bottom left, moves up and right, loops at the origin, and then moves up and left.
\begin{center} \includegraphics[width=0.5\textwidth]{{problem12_sketch.png}} \end{center}

\subsection*{Problem 13}
Eliminate parameter: $x = 2\sin(t) \implies \sin(t) = x/2$. $y = \cos^2(t) = 1-\sin^2(t) = 1-(x/2)^2 = 1-x^2/4$. This is a parabola opening downwards. Since $x=2\sin(t)$, we have $-2 \le x \le 2$. The particle oscillates back and forth along this parabolic arc. At $t=0$, point is $(0,1)$. At $t=\pi/2$, point is $(2,0)$. At $t=\pi$, point is $(0,1)$. The orientation moves from $(0,1)$ to $(2,0)$ and back.
\begin{center} \includegraphics[width=0.5\textwidth]{{problem13_sketch.png}} \end{center}

\subsection*{Problem 14}
Eliminate parameter: $x=\sqrt{t} \implies t=x^2$. Substitute: $y=x^2-2$. This is a parabola.
Restriction: Since $x=\sqrt{t}$, $t \ge 0$ and $x \ge 0$. So, only the right half of the parabola is traced.
Orientation: $t=0 \implies (0, -2)$, $t=4 \implies (2, 2)$. The curve moves upwards and to the right.
\begin{center} \includegraphics[width=0.5\textwidth]{{problem14_sketch.png}} \end{center}

\subsection*{Problem 15}
This is a circle $x^2+y^2=16$. The interval $0 \le t \le \pi$ traces a semi-circle.
$t=0 \implies (0, 4)$. $t=\pi/2 \implies (4, 0)$. $t=\pi \implies (0, -4)$.
The orientation is \textbf{clockwise} along the right semi-circle.
\begin{center} \includegraphics[width=0.5\textwidth]{{problem15_sketch.png}} \end{center}

\subsection*{Problem 16}
Eliminate parameter: $x=1+\ln(t) \implies \ln(t)=x-1 \implies t = e^{x-1}$.
Substitute into y: $y = (e^{x-1})^2 = e^{2x-2}$. This is an exponential curve.
As $t$ increases from near 0 to $\infty$, $\ln(t)$ goes from $-\infty$ to $\infty$, so $x$ covers all real numbers. The orientation is from left to right.
\begin{center} \includegraphics[width=0.5\textwidth]{{problem16_sketch.png}} \end{center}

\subsection*{Problem 17}
Eliminate parameter: $t=y$. Substitute into x: $x=2-y^2$. This is a parabola opening to the left with vertex at $(2,0)$.
Orientation: As $t$ increases, $y$ increases. The particle moves up along the parabola.
\begin{center} \includegraphics[width=0.5\textwidth]{{problem17_sketch.png}} \end{center}

\subsection*{Problem 18}
Notice that $y = e^{-t} = 1/e^t = 1/x$. The curve is the hyperbola $y=1/x$.
Restriction: Since $e^t > 0$ for all $t$, both $x$ and $y$ are positive. The curve is restricted to the first quadrant.
Orientation: As $t$ increases from $-\infty$ to $\infty$, $x=e^t$ increases from $0$ to $\infty$. The orientation is from left to right along the hyperbola branch.
\begin{center} \includegraphics[width=0.5\textwidth]{{problem18_sketch.png}} \end{center}

\subsection*{Problem 19}
The equations describe a circle of radius 6. The period $T$ is found when the argument of sine/cosine completes a $2\pi$ cycle.
$\pi T = 2\pi \implies T = 2$. It takes \textbf{2 seconds} to complete one revolution.
To find direction, check points: $t=0 \implies (6,0)$. $t=0.5 \implies (0,6)$. The motion is from the positive x-axis to the positive y-axis, which is \textbf{counter-clockwise}.

\subsection*{Problem 20}
The curve is an ellipse $\frac{x^2}{25} + \frac{y^2}{4} = 1$. The interval length is $4\pi$, which is two full $2\pi$ cycles.
Direction: $t=0 \implies (0,2)$. $t=\pi/2 \implies (5,0)$. The motion is from the positive y-axis to the positive x-axis, which is \textbf{clockwise}.
The particle traverses the entire ellipse \textbf{twice in a clockwise direction}.

\subsection*{Problem 21}
Horizontal motion: $x=2t$. The particle moves to the right at a constant speed.
Vertical motion: $y=\cos(\pi t)$. The particle oscillates vertically between -1 and 1 with a period of $T = 2\pi/\pi = 2$.
The overall motion is not periodic in the sense of returning to a starting point, because the $x$ coordinate always increases. The particle moves along a cosine wave that is stretched horizontally.

\subsection*{Problem 22}
This curve traces a "figure-eight" shape. It starts at $(0,0)$, moves into the first quadrant, crosses the origin at $t=\pi$, moves into the fourth quadrant, and returns to the origin at $t=2\pi$.
\begin{center} \includegraphics[width=0.5\textwidth]{{problem22_sketch.png}} \end{center}

\subsection*{Problem 23}
The simplest parameterization is to let $x=t$. Then substitute into the equation to find $y$.
$\mathbf{x=t, y=7t-3}$.

\subsection*{Problem 24}
Since the equation gives $x$ in terms of $y$, it's easiest to let $y=t$.
$\mathbf{y=t, x=t^2 - 4t + 1}$.

\subsection*{Problem 25}
This is an ellipse centered at $(2, -4)$ with semi-major axis $a=5$ and semi-minor axis $b=3$.
Use the standard parameterization for an ellipse:
$\frac{x-h}{a} = \cos(t)$ and $\frac{y-k}{b} = \sin(t)$.
$\mathbf{x = 2 + 5\cos(t), y = -4 + 3\sin(t)}$ for $0 \le t \le 2\pi$.

\subsection*{Problem 26}
Use the formula $x(t) = x_1 + (x_2-x_1)t$ and $y(t) = y_1 + (y_2-y_1)t$ for $0 \le t \le 1$.
$x(t) = 1 + (-3-1)t = 1 - 4t$.
$y(t) = 6 + (2-6)t = 6 - 4t$.
So, $\mathbf{x=1-4t, y=6-4t}$ for $0 \le t \le 1$.

\subsection*{Problem 27}
The projectile is in the air until $y(t) = 0$.
$y(t) = (100\sin(30^\circ))t - 4.9t^2 = (100 \cdot 0.5)t - 4.9t^2 = 50t - 4.9t^2$.
Set $y(t) = 0$: $t(50 - 4.9t) = 0$.
The solutions are $t=0$ (launch) and $t = 50/4.9 \approx 10.2$.
The projectile is in the air for approximately \textbf{10.2 seconds}.

\subsection*{Problem 28}
Given $r=2$ and $\theta=\pi/2$.
$x = 2(\pi/2 - \sin(\pi/2)) = 2(\pi/2 - 1) = \pi - 2$.
$y = 2(1 - \cos(\pi/2)) = 2(1 - 0) = 2$.
The position is $\mathbf{(\pi-2, 2)}$.

\subsection*{Problem 29}
Intersection points occur when coordinates are equal, but not necessarily at the same time parameter.
Set $x_1(t) = x_2(s)$ and $y_1(t) = y_2(s)$.
$t+3 = s-1 \implies s = t+4$.
$t^2 = 2s$.
Substitute $s$ into the second equation: $t^2 = 2(t+4) \implies t^2 = 2t+8 \implies t^2 - 2t - 8 = 0$.
$(t-4)(t+2) = 0$, so $t=4$ or $t=-2$.
If $t=4$, the point on path 1 is $(4+3, 4^2) = (7, 16)$.
If $t=-2$, the point on path 1 is $(-2+3, (-2)^2) = (1, 4)$.
The intersection points are \textbf{(7, 16)} and \textbf{(1, 4)}.

Collision: Does $t=s$? Set $x_1(t)=x_2(t)$ and $y_1(t)=y_2(t)$.
$t+3 = t-1 \implies 3 = -1$, which is impossible.
There is \textbf{no collision}.

\subsection*{Problem 30}
The slope of the tangent line is given by $\frac{dy}{dx} = \frac{dy/dt}{dx/dt}$.
$x = t^3 - 3t \implies \frac{dx}{dt} = 3t^2 - 3$.
$y = 3t^2 - 9 \implies \frac{dy}{dt} = 6t$.
So, $\frac{dy}{dx} = \frac{6t}{3t^2 - 3} = \frac{2t}{t^2-1}$.
At $t=2$, the slope is $\frac{2(2)}{2^2-1} = \frac{4}{4-1} = \frac{4}{3}$.
The slope at $t=2$ is $\mathbf{4/3}$.

\newpage
\section*{Concept Checklist}
\begin{itemize}
    \item \textbf{Evaluating Points from Parametric Equations:} Problems 1, 2
    \item \textbf{Eliminating the Parameter (Algebraic Methods):}
        \begin{itemize}
            \item Linear/Polynomial: Problem 3
            \item Radical Expressions: Problem 4
            \item Exponential/Logarithmic Expressions: Problems 5, 16
            \item Rational Expressions: Problem 6
        \end{itemize}
    \item \textbf{Eliminating the Parameter (Trigonometric Identities):}
        \begin{itemize}
            \item Circles ($\sin^2+\cos^2=1$): Problem 7
            \item Ellipses ($\sin^2+\cos^2=1$): Problem 8
            \item Hyperbolas ($\sec^2-\tan^2=1$): Problem 9
            \item Double-Angle Identities: Problem 10
        \end{itemize}
    \item \textbf{Sketching Curves and Determining Orientation:}
        \begin{itemize}
            \item Parabolas: Problems 11, 14, 17
            \item Self-Intersecting Curves: Problem 12
            \item Oscillating Motion on an Arc: Problem 13
            \item Semi-circles/Arcs: Problem 15
            \item Hyperbolas: Problem 18
            \item Lissajous Figures: Problem 22
        \end{itemize}
    \item \textbf{Analyzing Motion (Period, Direction, Description):} Problems 19, 20, 21
    \item \textbf{Parameterizing a Cartesian Equation:}
        \begin{itemize}
            \item Line: Problem 23
            \item Parabola: Problem 24
            \item Ellipse: Problem 25
        \end{itemize}
    \item \textbf{Applications:}
        \begin{itemize}
            \item Line Segments: Problem 26
            \item Projectile Motion: Problem 27
            \item Cycloid: Problem 28
        \end{itemize}
    \item \textbf{Advanced Topics:}
        \begin{itemize}
            \item Intersection vs. Collision: Problem 29
            \item Calculus (Derivatives/Tangent Slopes): Problem 30
        \end{itemize}
\end{itemize}
% Note: Images for sketches are placeholders. In a real LaTeX document, you would generate these images.
% For problem 11: y=(x+1)^2+4
% For problem 12: x=t^3-3t, y=t^2
% For problem 13: x=2sin(t), y=cos^2(t)
% For problem 14: y=x^2-2 for x>=0
% For problem 15: x=4sin(t), y=4cos(t) for t in [0, pi]
% For problem 16: y=e^(2x-2)
% For problem 17: x=2-y^2
% For problem 18: y=1/x for x>0
% For problem 22: x=sin(t), y=sin(2t)

\end{document}