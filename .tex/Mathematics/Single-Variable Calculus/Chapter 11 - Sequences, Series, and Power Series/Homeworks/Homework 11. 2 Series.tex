\documentclass{article}
\usepackage{amsmath}
\usepackage{amssymb}
\usepackage{geometry}
\geometry{a4paper, margin=1in}

\title{Homework 11.2 Series}
\author{Tashfeen Omran}
\date{\today}

\begin{document}

\maketitle

\part{Comprehensive Introduction, Context, and Prerequisites}
\section{Core Concepts}
This chapter transitions from the study of sequences (ordered lists of numbers) to \textbf{series} (the sums of those numbers). While a sequence is a list $a_1, a_2, a_3, \dots$, a series is the addition of these terms: $a_1 + a_2 + a_3 + \dots$.

\begin{itemize}
    \item \textbf{Infinite Series:} The central idea is to understand what it means to sum an infinite number of terms. We denote an infinite series using summation notation:
    \[ \sum_{n=1}^{\infty} a_n = a_1 + a_2 + a_3 + \dots + a_n + \dots \]
    \item \textbf{Sequence of Partial Sums:} We can't add infinite terms directly. Instead, we examine the \textbf{sequence of partial sums}, $\{S_n\}$. Each term $S_n$ is the sum of the first $n$ terms of the series:
    \begin{align*}
        S_1 &= a_1 \\
        S_2 &= a_1 + a_2 \\
        S_3 &= a_1 + a_2 + a_3 \\
        &\vdots \\
        S_n &= a_1 + a_2 + \dots + a_n = \sum_{k=1}^{n} a_k
    \end{align*}
    \item \textbf{Convergence and Divergence:} The core question of this chapter is whether a series \textbf{converges} or \textbf{diverges}.
    \begin{itemize}
        \item A series $\sum a_n$ is said to \textbf{converge} if its sequence of partial sums $\{S_n\}$ converges to a finite limit $S$. This limit $S$ is called the \textbf{sum of the series}.
        \[ \text{If } \lim_{n \to \infty} S_n = S \text{ (where } S \text{ is finite), then the series converges and } \sum_{n=1}^{\infty} a_n = S. \]
        \item If the sequence of partial sums $\{S_n\}$ diverges (i.e., the limit is infinite or does not exist), then the series is said to \textbf{diverge}. A divergent series does not have a sum.
    \end{itemize}
\end{itemize}

\section{Intuition and Derivation}
The "why" behind series is about making sense of infinite processes. Consider Zeno's famous paradox: to walk across a room, you must first walk half the distance, then half of the remaining distance, then half of that remainder, and so on. You would have to complete an infinite number of tasks:
\[ \frac{1}{2} + \frac{1}{4} + \frac{1}{8} + \frac{1}{16} + \dots \]
Intuitively, we know you can cross the room, so this sum must be a finite number (it's 1). The concept of partial sums formalizes this. The first step takes you $S_1 = 1/2$ of the way. The second step takes you to $S_2 = 1/2 + 1/4 = 3/4$. The third step gets you to $S_3 = 7/8$. We can see that the sequence of partial sums $\{1/2, 3/4, 7/8, \dots, (2^n-1)/2^n, \dots\}$ is approaching 1. Therefore, the infinite sum is 1. We are turning an infinite addition problem into a limit problem, which we already have the tools to solve.

\section{Historical Context and Motivation}
The study of infinite series has roots in antiquity with Greek mathematicians like Archimedes, who used the "method of exhaustion" to calculate the area of a parabolic segment. He did this by summing the areas of an infinite sequence of triangles, effectively calculating the sum of a geometric series. However, the formal development of series blossomed in the 17th and 18th centuries with the invention of calculus.

Mathematicians like Isaac Newton and Gottfried Wilhelm Leibniz used infinite series (specifically, power series) to represent functions, solve differential equations, and calculate values like $\pi$ and logarithms to high precision. The Bernoulli family and Leonhard Euler made monumental contributions, with Euler solving the famous Basel problem (the sum of the reciprocals of the squares). The rigorous foundation for convergence and divergence, however, was laid later by Augustin-Louis Cauchy in the 19th century, who formalized the limit-based definitions we use today. The motivation was clear: many functions and physical phenomena could not be described by simple formulas but could be understood as the sum of an infinite number of simpler parts.

\section{Key Formulas}
\begin{itemize}
    \item \textbf{Sum of a Series:} $S = \lim_{n \to \infty} S_n$, where $S_n = \sum_{k=1}^{n} a_k$.
    \item \textbf{Geometric Series:} A series of the form $\sum_{n=1}^{\infty} a r^{n-1} = a + ar + ar^2 + \dots$.
    \begin{itemize}
        \item It \textbf{converges} to the sum $S = \frac{a}{1-r}$ if the common ratio $r$ satisfies $|r| < 1$.
        \item It \textbf{diverges} if $|r| \ge 1$.
    \end{itemize}
    \item \textbf{The $n$-th Term Test for Divergence:}
    \[ \text{If } \lim_{n \to \infty} a_n \neq 0 \text{ or if this limit does not exist, then the series } \sum_{n=1}^{\infty} a_n \text{ diverges.} \]
    \textbf{Crucial Warning:} If $\lim_{n \to \infty} a_n = 0$, this test is \textbf{inconclusive}. The series might converge or it might diverge. (The harmonic series $\sum 1/n$ is the classic example of a series where $a_n \to 0$ but the series diverges).
\end{itemize}

\section{Prerequisites}
To succeed with series, you must be proficient in the following:
\begin{itemize}
    \item \textbf{Sequences:} Understanding the definition of a sequence, its notation $\{a_n\}$, and how to determine if it converges or diverges.
    \item \textbf{Limits at Infinity:} The entire concept of series convergence is built on the limit of the sequence of partial sums. You must be able to compute $\lim_{n \to \infty} f(n)$, especially for rational functions and exponential functions.
    \item \textbf{Algebra:} You need strong skills in manipulating exponents (e.g., $a^{n+1} = a \cdot a^n$), fractions, and summation notation. Partial fraction decomposition will be needed for telescoping series.
    \item \textbf{Function Recognition:} Identifying the type of series (e.g., geometric) is the first and most critical step in many problems.
\end{itemize}

\part{Detailed Homework Solutions}

\subsection*{Problem 1}
\textbf{(a) What is the difference between a sequence and a series?}
\begin{itemize}
    \item A \textbf{sequence} is an ordered list of numbers, like $\{a_1, a_2, a_3, \dots\}$.
    \item A \textbf{series} is the sum of the terms of a sequence, like $a_1 + a_2 + a_3 + \dots$.
\end{itemize}
\textbf{Correct Answer:} A sequence is an ordered list of numbers whereas a series is the sum of a list of numbers.

\textbf{(b) What is a convergent series? What is a divergent series?}
\begin{itemize}
    \item A series is \textbf{convergent} if the sequence of its partial sums $\{S_n\}$ converges to a finite number.
    \item A series is \textbf{divergent} if the sequence of its partial sums does not converge to a finite number.
\end{itemize}
\textbf{Correct Answer:} A series is convergent if the sequence of partial sums is a convergent sequence. A series is divergent if it is not convergent.

\subsection*{Problem 2}
\textbf{Calculate the sum of the series $\sum_{n=1}^{\infty} a_n$ whose partial sums are given by $S_n = 7 - 2(0.6)^n$.}
\textbf{Solution:} The sum of a series is defined as the limit of its sequence of partial sums as $n \to \infty$.
\[ S = \lim_{n \to \infty} S_n = \lim_{n \to \infty} \left(7 - 2(0.6)^n\right) \]
As $n \to \infty$, the term $(0.6)^n$ approaches 0, because this is an exponential function with a base whose absolute value is less than 1.
\[ S = 7 - 2 \cdot \lim_{n \to \infty} (0.6)^n = 7 - 2(0) = 7 \]
\textbf{Final Answer:} 7

\subsection*{Problem 3}
\textbf{Calculate the sum of the series $\sum_{n=1}^{\infty} a_n$ whose partial sums are given by $S_n = \frac{n^2 - 1}{7n^2 + 1}$.}
\textbf{Solution:} The sum is the limit of the partial sums.
\[ S = \lim_{n \to \infty} S_n = \lim_{n \to \infty} \frac{n^2 - 1}{7n^2 + 1} \]
This is a limit of a rational function where the degree of the numerator is equal to the degree of the denominator. The limit is the ratio of the leading coefficients.
Alternatively, divide the numerator and denominator by the highest power of $n$, which is $n^2$:
\[ S = \lim_{n \to \infty} \frac{\frac{n^2}{n^2} - \frac{1}{n^2}}{\frac{7n^2}{n^2} + \frac{1}{n^2}} = \lim_{n \to \infty} \frac{1 - \frac{1}{n^2}}{7 + \frac{1}{n^2}} \]
As $n \to \infty$, the terms $1/n^2$ go to 0.
\[ S = \frac{1 - 0}{7 + 0} = \frac{1}{7} \]
\textbf{Final Answer:} 1/7

\subsection*{Problem 4}
\textbf{Calculate the first eight terms of the sequence of partial sums for the series $\sum_{n=1}^{\infty} (-1)^n 9n$. Does it appear the series is convergent or divergent?}
\textbf{Solution:} The terms of the series are $a_n = (-1)^n 9n$.
\begin{align*}
    a_1 &= (-1)^1 9(1) = -9 \\
    a_2 &= (-1)^2 9(2) = 18 \\
    a_3 &= (-1)^3 9(3) = -27 \\
    a_4 &= (-1)^4 9(4) = 36 \\
    a_5 &= -45, a_6 = 54, a_7 = -63, a_8 = 72
\end{align*}
Now, calculate the partial sums $S_n = a_1 + \dots + a_n$.
\begin{align*}
    S_1 &= a_1 = \mathbf{-9.0000} \\
    S_2 &= S_1 + a_2 = -9 + 18 = \mathbf{9.0000} \\
    S_3 &= S_2 + a_3 = 9 - 27 = \mathbf{-18.0000} \\
    S_4 &= S_3 + a_4 = -18 + 36 = \mathbf{18.0000} \\
    S_5 &= S_4 + a_5 = 18 - 45 = \mathbf{-27.0000} \\
    S_6 &= S_5 + a_6 = -27 + 54 = \mathbf{27.0000} \\
    S_7 &= S_6 + a_7 = 27 - 63 = \mathbf{-36.0000} \\
    S_8 &= S_7 + a_8 = -36 + 72 = \mathbf{36.0000}
\end{align*}
The sequence of partial sums is $\{-9, 9, -18, 18, -27, 27, \dots\}$. The terms are oscillating and their magnitude is growing. The limit $\lim_{n \to \infty} S_n$ does not exist.
\textbf{Conclusion:} The series is \textbf{divergent}.

\subsection*{Problem 5}
\textbf{Calculate the first eight terms of the sequence of partial sums for the series $\sum_{n=1}^{\infty} \frac{(-1)^{n-1}}{n!}$. Does it appear the series is convergent or divergent?}
\textbf{Solution:} The terms are $a_n = \frac{(-1)^{n-1}}{n!}$.
\begin{align*}
    a_1 &= \frac{(-1)^0}{1!} = 1 \\
    a_2 &= \frac{(-1)^1}{2!} = -1/2 = -0.5 \\
    a_3 &= \frac{(-1)^2}{3!} = 1/6 \approx 0.16667 \\
    a_4 &= \frac{(-1)^3}{4!} = -1/24 \approx -0.04167 \\
    a_5 &= 1/120 \approx 0.00833 \\
    a_6 &= -1/720 \approx -0.00139 \\
    a_7 &= 1/5040 \approx 0.00020 \\
    a_8 &= -1/40320 \approx -0.00002
\end{align*}
Now, calculate the partial sums $S_n$.
\begin{align*}
    S_1 &= 1 = \mathbf{1.0000} \\
    S_2 &= 1 - 0.5 = \mathbf{0.5000} \\
    S_3 &= 0.5 + 1/6 \approx 0.6667 \\
    S_4 &= 0.6667 - 1/24 \approx \mathbf{0.6250} \\
    S_5 &= 0.6250 + 1/120 \approx \mathbf{0.6333} \\
    S_6 &= 0.6333 - 1/720 \approx \mathbf{0.6319} \\
    S_7 &= 0.6319 + 1/5040 \approx \mathbf{0.6321} \\
    S_8 &= 0.6321 - 1/40320 \approx \mathbf{0.6321}
\end{align*}
The sequence of partial sums appears to be oscillating with decreasing amplitude, homing in on a value around 0.6321.
\textbf{Conclusion:} The series is \textbf{convergent}. (This series converges to $1 - 1/e$).

\subsection*{Problem 6}
\textbf{Let $a_n = \frac{7n}{4n+1}$. (a) Determine whether $\{a_n\}$ is convergent. (b) Determine whether $\sum a_n$ is convergent.}
\textbf{Solution:}
\textbf{(a)} To determine if the sequence $\{a_n\}$ converges, we take the limit of $a_n$.
\[ \lim_{n \to \infty} a_n = \lim_{n \to \infty} \frac{7n}{4n+1} = \frac{7}{4} \]
Since the limit is a finite number, the sequence $\{a_n\}$ is \textbf{convergent}.

\textbf{(b)} To determine if the series $\sum a_n$ is convergent, we use the Test for Divergence. This test checks the limit of the terms $a_n$.
\[ \lim_{n \to \infty} a_n = \frac{7}{4} \]
Since this limit is not equal to 0, the series $\sum a_n$ is \textbf{divergent}.

\subsection*{Problem 7}
\textbf{Let $a_n = \frac{4n}{6n+1}$. (a) Find the limit of the sequence $\{a_n\}$. (b) Determine whether the series $\sum a_n$ is convergent.}
\textbf{Solution:}
\textbf{(a)} Find the limit of the sequence.
\[ \lim_{n \to \infty} a_n = \lim_{n \to \infty} \frac{4n}{6n+1} = \frac{4}{6} = \frac{2}{3} \]
\textbf{Answer:} 2/3

\textbf{(b)} We use the Test for Divergence. Since $\lim_{n \to \infty} a_n = 2/3 \neq 0$, the series diverges.
\textbf{Correct Answer:} Diverges; the limit of the terms $a_n$ is not 0 as $n$ goes to $\infty$.

\subsection*{Problem 8}
\textbf{Determine whether the geometric series $7 - 8 + \frac{64}{7} - \frac{512}{49} + \dots$ is convergent or divergent. If it is convergent, find its sum.}
\textbf{Solution:} This is a geometric series.
The first term is $a = 7$.
The common ratio is $r = \frac{\text{second term}}{\text{first term}} = \frac{-8}{7}$.
To check for convergence, we examine the absolute value of the ratio:
\[ |r| = \left|-\frac{8}{7}\right| = \frac{8}{7} \]
Since $|r| = 8/7 > 1$, the series is \textbf{divergent}.
\textbf{Final Answer:} DIVERGES

\subsection*{Problem 9}
\textbf{Determine whether the geometric series $8 - 9 + \frac{81}{8} - \frac{729}{64} + \dots$ is convergent or divergent. If it is convergent, find its sum.}
\textbf{Solution:} This is a geometric series.
The first term is $a = 8$.
The common ratio is $r = \frac{-9}{8}$.
The absolute value of the ratio is $|r| = \left|-\frac{9}{8}\right| = \frac{9}{8}$.
Since $|r| > 1$, the series is \textbf{divergent}.
\textbf{Final Answer:} DIVERGES

\subsection*{Problem 10}
\textbf{Determine whether the geometric series $5 + 4 + \frac{16}{5} + \frac{64}{25} + \dots$ is convergent or divergent. If it is convergent, find its sum.}
\textbf{Solution:} This is a geometric series.
The first term is $a = 5$.
The common ratio is $r = \frac{4}{5}$.
The absolute value of the ratio is $|r| = \frac{4}{5}$.
Since $|r| < 1$, the series is \textbf{convergent}.
The sum is given by the formula $S = \frac{a}{1-r}$.
\[ S = \frac{5}{1 - \frac{4}{5}} = \frac{5}{\frac{1}{5}} = 5 \cdot 5 = 25 \]
\textbf{Final Answer:} 25

\subsection*{Problem 11}
\textbf{Determine whether the geometric series $\sum_{n=1}^{\infty} \frac{2}{\pi^n}$ is convergent or divergent. If it is convergent, find its sum.}
\textbf{Solution:} Let's rewrite the term to fit the standard geometric series form.
\[ a_n = \frac{2}{\pi^n} = 2 \left(\frac{1}{\pi}\right)^n \]
This is a geometric series. Let's find the first term and the common ratio.
The first term (at $n=1$) is $a = 2 \left(\frac{1}{\pi}\right)^1 = \frac{2}{\pi}$.
The common ratio is $r = \frac{1}{\pi}$.
Since $\pi \approx 3.14159$, we have $|r| = 1/\pi < 1$. Therefore, the series is \textbf{convergent}.
The sum is $S = \frac{a}{1-r}$.
\[ S = \frac{\frac{2}{\pi}}{1 - \frac{1}{\pi}} = \frac{\frac{2}{\pi}}{\frac{\pi - 1}{\pi}} = \frac{2}{\pi} \cdot \frac{\pi}{\pi - 1} = \frac{2}{\pi - 1} \]
\textbf{Final Answer:} The prompt shows the answer as $2\pi - 1$, which seems to be a typo in the provided image. The correct sum is $\frac{2}{\pi - 1}$. Let's re-read the summation. $\sum_{n=1}^\infty \frac{2}{\pi^n}$. My calculation is correct. The answer shown in the image might be incorrect or for a different problem. Let's assume the sum starts from $n=0$: $\sum_{n=0}^\infty 2(\frac{1}{\pi})^n$. Then $a=2(\frac{1}{\pi})^0 = 2$. $S = \frac{2}{1-1/\pi} = \frac{2\pi}{\pi - 1}$. Still not matching. Let's stick with the correct derivation.
The sum is $\frac{2}{\pi-1}$.

\subsection*{Problem 12}
\textbf{Determine whether the geometric series $\sum_{n=1}^{\infty} \frac{(-2)^{n-1}}{5^n}$ is convergent or divergent. If it is convergent, find its sum.}
\textbf{Solution:} Let's rewrite the term.
\[ a_n = \frac{(-2)^{n-1}}{5^n} = \frac{(-2)^{n-1}}{5 \cdot 5^{n-1}} = \frac{1}{5} \left(\frac{-2}{5}\right)^{n-1} \]
This is in the form $\sum ar^{n-1}$.
The first term is $a = \frac{1}{5}$.
The common ratio is $r = -\frac{2}{5}$.
The absolute value of the ratio is $|r| = |-\frac{2}{5}| = \frac{2}{5}$.
Since $|r| < 1$, the series is \textbf{convergent}.
The sum is $S = \frac{a}{1-r}$.
\[ S = \frac{\frac{1}{5}}{1 - (-\frac{2}{5})} = \frac{\frac{1}{5}}{1 + \frac{2}{5}} = \frac{\frac{1}{5}}{\frac{7}{5}} = \frac{1}{5} \cdot \frac{5}{7} = \frac{1}{7} \]
\textbf{Final Answer:} 1/7

\subsection*{Problem 13}
\textbf{Determine whether the series $\frac{1}{8} + \frac{1}{16} + \frac{1}{24} + \frac{1}{32} + \dots$ is convergent or divergent.}
\textbf{Solution:} Let's write the series in summation notation. The terms are $\frac{1}{8\cdot 1}, \frac{1}{8\cdot 2}, \frac{1}{8\cdot 3}, \dots$.
\[ \sum_{n=1}^{\infty} \frac{1}{8n} \]
We can factor out the constant $\frac{1}{8}$:
\[ \frac{1}{8} \sum_{n=1}^{\infty} \frac{1}{n} \]
The series $\sum_{n=1}^{\infty} \frac{1}{n}$ is the \textbf{harmonic series}, which is a famous divergent series. A constant multiple of a divergent series is also divergent.
\textbf{Final Answer:} DIVERGES

\subsection*{Problem 14}
\textbf{Determine whether the series $\frac{1}{2} + \frac{2}{3} + \frac{3}{4} + \frac{4}{5} + \dots$ is convergent or divergent.}
\textbf{Solution:} The general term is $a_n = \frac{n}{n+1}$.
We use the Test for Divergence.
\[ \lim_{n \to \infty} a_n = \lim_{n \to \infty} \frac{n}{n+1} = 1 \]
Since the limit is $1 \neq 0$, the series is \textbf{divergent}.
\textbf{Final Answer:} DIVERGES

\subsection*{Problem 15}
\textbf{Determine whether the series $\frac{2}{9} + \frac{4}{81} + \frac{8}{729} + \dots$ is convergent or divergent. If convergent, find its sum.}
\textbf{Solution:} Let's examine the terms. $a_1 = \frac{2}{9}$, $a_2 = \frac{4}{81} = \frac{2^2}{9^2}$, $a_3 = \frac{8}{729} = \frac{2^3}{9^3}$.
The general term is $a_n = \frac{2^n}{9^n} = \left(\frac{2}{9}\right)^n$.
This is a geometric series.
The first term (at $n=1$) is $a = \frac{2}{9}$.
The common ratio is $r = \frac{2}{9}$.
Since $|r| = 2/9 < 1$, the series is \textbf{convergent}.
The sum is $S = \frac{a}{1-r}$.
\[ S = \frac{\frac{2}{9}}{1 - \frac{2}{9}} = \frac{\frac{2}{9}}{\frac{7}{9}} = \frac{2}{9} \cdot \frac{9}{7} = \frac{2}{7} \]
\textbf{Final Answer:} The prompt shows the answer as 27. Let me re-read the problem. The terms listed are $\frac{2}{9}, \frac{4}{81}, \frac{8}{729}, \frac{16}{6561}, \frac{32}{59049}$. This corresponds to $a_n = \frac{2^n}{9^n}$. My work is correct. The given answer of 27 must be for a different problem. For this series, the sum is $2/7$.

\subsection*{Problem 16}
\textbf{Determine whether the series $\frac{1}{2} + \frac{3}{4} + \frac{1}{8} + \frac{3}{16} + \frac{1}{32} + \frac{3}{64} + \dots$ is convergent or divergent. If it is convergent, find its sum.}
\textbf{Solution:} The terms of this series do not form a single geometric series. Let's try rearranging the terms into two separate series.
\[ \left(\frac{1}{2} + \frac{1}{8} + \frac{1}{32} + \dots\right) + \left(\frac{3}{4} + \frac{3}{16} + \frac{3}{64} + \dots\right) \]
The first series is geometric with $a_1 = 1/2$ and $r_1 = \frac{1/8}{1/2} = 1/4$. Since $|r_1| < 1$, it converges.
Its sum is $S_1 = \frac{a_1}{1-r_1} = \frac{1/2}{1-1/4} = \frac{1/2}{3/4} = \frac{1}{2}\cdot\frac{4}{3} = \frac{2}{3}$.
The second series is geometric with $a_2 = 3/4$ and $r_2 = \frac{3/16}{3/4} = 1/4$. Since $|r_2| < 1$, it converges.
Its sum is $S_2 = \frac{a_2}{1-r_2} = \frac{3/4}{1-1/4} = \frac{3/4}{3/4} = 1$.
The sum of the original series is the sum of these two convergent series: $S = S_1 + S_2 = \frac{2}{3} + 1 = \frac{5}{3}$.
Since the series converges to a finite sum, it is \textbf{convergent}.
\textbf{Final Answer:} 5/3

\subsection*{Problem 17}
\textbf{Determine whether the series $\sum_{k=1}^{\infty} \frac{k^2}{k^2 - 4k + 6}$ is convergent or divergent.}
\textbf{Solution:} We use the Test for Divergence. We must find the limit of the general term $a_k = \frac{k^2}{k^2 - 4k + 6}$.
\[ \lim_{k \to \infty} a_k = \lim_{k \to \infty} \frac{k^2}{k^2 - 4k + 6} \]
The degrees of the numerator and denominator are the same, so the limit is the ratio of the leading coefficients.
\[ \lim_{k \to \infty} a_k = \frac{1}{1} = 1 \]
Since the limit is $1 \neq 0$, the series is \textbf{divergent}.
\textbf{Final Answer:} Divergent. (The sum is DIVERGES).

\subsection*{Problem 18}
\textbf{Determine whether the series $\sum_{n=1}^{\infty} (6^n + 17^{-n})$ is convergent or divergent. If it is convergent, find its sum.}
\textbf{Solution:} We can split the series into two:
\[ \sum_{n=1}^{\infty} 6^n + \sum_{n=1}^{\infty} 17^{-n} \]
Let's analyze the first series, $\sum_{n=1}^{\infty} 6^n$. This is a geometric series with first term $a=6$ and common ratio $r=6$. Since $|r| = 6 \ge 1$, this series \textbf{diverges}.
The second series is $\sum_{n=1}^{\infty} 17^{-n} = \sum_{n=1}^{\infty} (\frac{1}{17})^n$. This is a geometric series with first term $a=1/17$ and common ratio $r=1/17$. Since $|r|<1$, this series converges.
The original series is the sum of a divergent series and a convergent series. The sum of a divergent and convergent series is always \textbf{divergent}.

Alternatively, use the Test for Divergence on the original series term $a_n = 6^n + 17^{-n}$.
\[ \lim_{n \to \infty} (6^n + 17^{-n}) = \lim_{n \to \infty} 6^n + \lim_{n \to \infty} \frac{1}{17^n} = \infty + 0 = \infty \]
Since the limit of the terms does not exist (it is infinite), the series diverges.

\textbf{Note on the WebAssign Image:} The provided image shows the correct answer for this problem as `36`. This is mathematically incorrect for the series $\sum (6^n + 17^{-n})$. However, the answer 36 is correct for the very similar-looking problem $\sum_{n=1}^{\infty} 6^{n+1}7^{-n}$. Let's solve that one:
\[ \sum_{n=1}^{\infty} 6^{n+1}7^{-n} = \sum_{n=1}^{\infty} 6 \cdot 6^n \cdot \frac{1}{7^n} = \sum_{n=1}^{\infty} 6 \left(\frac{6}{7}\right)^n \]
This is a geometric series with ratio $r = 6/7$. Since $|r|<1$, it converges. The first term is $a = 6(6/7)^1 = 36/7$. The sum is:
\[ S = \frac{a}{1-r} = \frac{36/7}{1 - 6/7} = \frac{36/7}{1/7} = 36 \]
This confirms the answer in the box belongs to a different, but similar, problem. For the problem as written in the screenshot, the answer is DIVERGES.
\textbf{Final Answer (for the problem as written):} DIVERGES

\subsection*{Problem 19}
\textbf{Determine whether the series $\sum_{n=1}^{\infty} \ln\left(\frac{n^2+6}{2n^2+5}\right)$ is convergent or divergent.}
\textbf{Solution:} We use the Test for Divergence. We find the limit of the term $a_n = \ln\left(\frac{n^2+6}{2n^2+5}\right)$.
First, find the limit of the argument of the logarithm:
\[ \lim_{n \to \infty} \frac{n^2+6}{2n^2+5} = \frac{1}{2} \]
Since the natural logarithm function $\ln(x)$ is continuous for $x > 0$, we can bring the limit inside:
\[ \lim_{n \to \infty} a_n = \ln\left(\lim_{n \to \infty} \frac{n^2+6}{2n^2+5}\right) = \ln\left(\frac{1}{2}\right) \]
Since $\ln(1/2) = -\ln(2) \neq 0$, the series is \textbf{divergent} by the Test for Divergence.
\textbf{Final Answer:} DIVERGES

\subsection*{Problem 20}
\textbf{Express the number $0.\overline{4} = 0.4444\dots$ as a ratio of integers.}
\textbf{Solution:} We can write the repeating decimal as an infinite geometric series.
\[ 0.4444\dots = 0.4 + 0.04 + 0.004 + 0.0004 + \dots \]
\[ = \frac{4}{10} + \frac{4}{100} + \frac{4}{1000} + \frac{4}{10000} + \dots \]
This is a geometric series with first term $a = 4/10$ and common ratio $r = 1/10$.
Since $|r| < 1$, the series converges. The sum is:
\[ S = \frac{a}{1-r} = \frac{4/10}{1 - 1/10} = \frac{4/10}{9/10} = \frac{4}{9} \]
\textbf{Final Answer:} 4/9

\subsection*{Problem 21}
\textbf{Express the number $0.\overline{57} = 0.575757\dots$ as a ratio of integers.}
\textbf{Solution:} We write the repeating decimal as an infinite geometric series. The repeating block has two digits.
\[ 0.575757\dots = 0.57 + 0.0057 + 0.000057 + \dots \]
\[ = \frac{57}{100} + \frac{57}{10000} + \frac{57}{1000000} + \dots \]
This is a geometric series with first term $a = 57/100$ and common ratio $r = 1/100$.
Since $|r| < 1$, the series converges. The sum is:
\[ S = \frac{a}{1-r} = \frac{57/100}{1 - 1/100} = \frac{57/100}{99/100} = \frac{57}{99} \]
We can simplify this fraction by dividing the numerator and denominator by 3.
\[ \frac{57 \div 3}{99 \div 3} = \frac{19}{33} \]
\textbf{Final Answer:} 19/33

\part{In-Depth Analysis of Problems and Techniques}
\section{Problem Types and General Approach}
\begin{itemize}
    \item \textbf{Type 1: Conceptual Definitions (Problem 1)}
    \begin{itemize}
        \item \textbf{Description:} These problems test the fundamental definitions of sequences, series, convergence, and divergence.
        \item \textbf{Approach:} Memorize and understand the core definitions. Know that a series is a sum and its convergence depends on the limit of its partial sums.
    \end{itemize}
    \item \textbf{Type 2: Sum from Partial Sums (Problems 2, 3)}
    \begin{itemize}
        \item \textbf{Description:} You are given the formula for the $n$-th partial sum, $S_n$, and asked to find the sum of the series.
        \item \textbf{Approach:} This is a straightforward limit problem. The sum $S$ is simply $\lim_{n \to \infty} S_n$. Use standard techniques for limits at infinity.
    \end{itemize}
    \item \textbf{Type 3: The Test for Divergence (Problems 6, 7, 14, 17, 19)}
    \begin{itemize}
        \item \textbf{Description:} These problems present a series where the limit of the individual terms, $a_n$, is not zero.
        \item \textbf{Approach:} This should often be the first test you try. Calculate $\lim_{n \to \infty} a_n$. If this limit is anything other than 0 (or does not exist), you can immediately conclude the series diverges. No further work is needed.
    \end{itemize}
    \item \textbf{Type 4: Geometric Series (Problems 8, 9, 10, 11, 12, 15)}
    \begin{itemize}
        \item \textbf{Description:} The series has a common ratio $r$ between successive terms.
        \item \textbf{Approach:}
        1.  Identify the first term $a$ (by plugging in the starting value of $n$) and the common ratio $r$ (by dividing the second term by the first).
        2.  Check the condition for convergence: $|r| < 1$.
        3.  If it converges, use the sum formula $S = \frac{a}{1-r}$. If it diverges, state that.
    \end{itemize}
    \item \textbf{Type 5: Composite or Disguised Series (Problems 13, 16, 18)}
    \begin{itemize}
        \item \textbf{Description:} The series is a sum/multiple of other, more basic series (like harmonic or geometric).
        \item \textbf{Approach:} Use properties of series to split it into simpler parts. For $\sum(a_n + b_n)$, analyze $\sum a_n$ and $\sum b_n$ separately. A constant multiple can be factored out, as in Problem 13 ($\sum c \cdot a_n = c \sum a_n$).
    \end{itemize}
    \item \textbf{Type 6: Application to Repeating Decimals (Problems 20, 21)}
    \begin{itemize}
        \item \textbf{Description:} Convert a repeating decimal into a fraction.
        \item \textbf{Approach:} Write the decimal as the sum of its parts (e.g., $0.444 = 0.4 + 0.04 + \dots$). This will always form a convergent geometric series. Identify $a$ and $r$ and use the sum formula.
    \end{itemize}
     \item \textbf{Type 7: Numerical Investigation (Problems 4, 5)}
    \begin{itemize}
        \item \textbf{Description:} Calculate the first several partial sums to guess whether the series converges or diverges.
        \item \textbf{Approach:} Be methodical. First, write down the first few terms of the series, $a_n$. Then, compute the partial sums cumulatively: $S_1=a_1$, $S_2=S_1+a_2$, $S_3=S_2+a_3$, etc. Observe the trend in the sequence $\{S_n\}$.
    \end{itemize}
\end{itemize}

\section{Key Algebraic and Calculus Manipulations}
\begin{itemize}
    \item \textbf{Rewriting Terms for Geometric Series (Problems 11, 12, 15):} This is the most common algebraic trick. The goal is to isolate a term of the form $r^n$ or $r^{n-1}$.
    \begin{itemize}
        \item \textbf{Example (Problem 12):} The term was $a_n = \frac{(-2)^{n-1}}{5^n}$. Recognizing that $5^n = 5 \cdot 5^{n-1}$ was crucial to rewrite it as $\frac{1}{5} \left(\frac{-2}{5}\right)^{n-1}$, which perfectly matches the standard form $ar^{n-1}$.
    \end{itemize}
    \item \textbf{Evaluating Limits of Rational Functions (Problems 3, 6, 7, 17, 19):} A core calculus skill used repeatedly for the Test for Divergence and for finding sums from partial sums.
    \begin{itemize}
        \item \textbf{Example (Problem 3):} For $S_n = \frac{n^2 - 1}{7n^2 + 1}$, the fastest way to find $\lim_{n \to \infty} S_n$ is to identify that the degrees are equal and the limit is the ratio of leading coefficients, $1/7$.
    \end{itemize}
    \item \textbf{Factoring out Constants (Problem 13):} A simple but powerful property.
    \begin{itemize}
        \item \textbf{Example (Problem 13):} The series $\sum \frac{1}{8n}$ was simplified to $\frac{1}{8} \sum \frac{1}{n}$. This immediately revealed its connection to the divergent harmonic series.
    \end{itemize}
    \item \textbf{Splitting a Series (Problems 16, 18):} A series of a sum can be split into a sum of series, $\sum(a_n+b_n) = \sum a_n + \sum b_n$, provided both sub-series converge. If one diverges, the whole thing diverges.
    \begin{itemize}
        \item \textbf{Example (Problem 16):} The series $\frac{1}{2} + \frac{3}{4} + \frac{1}{8} + \dots$ seemed complex until it was split into two separate, simple geometric series. This technique unlocked the solution.
    \end{itemize}
    \item \textbf{Continuity of Functions for Limits (Problem 19):} To evaluate the limit of a composition of functions, like $\ln(f(n))$, we can use the property $\lim \ln(f(n)) = \ln(\lim f(n))$.
    \begin{itemize}
        \item \textbf{Example (Problem 19):} We didn't evaluate the limit of the whole logarithmic term directly. We first found the limit of its simpler argument, $\frac{n^2+6}{2n^2+5} \to \frac{1}{2}$, and then applied the logarithm to get the final answer, $\ln(1/2)$.
    \end{itemize}
\end{itemize}

\part{"Cheatsheet" and Tips for Success}
\section{Summary of Formulas and Tests}
\begin{itemize}
    \item \textbf{Geometric Series:} $\sum_{n=1}^{\infty} ar^{n-1}$. Converges to $\frac{a}{1-r}$ if $|r|<1$. Diverges if $|r|\ge 1$.
    \item \textbf{Test for Divergence:} If $\lim_{n \to \infty} a_n \neq 0$, the series $\sum a_n$ diverges. (If the limit is 0, the test is inconclusive).
    \item \textbf{Sum from Partial Sums:} If you know $S_n$, the sum is $S = \lim_{n \to \infty} S_n$.
\end{itemize}

\section{Tricks and Shortcuts}
\begin{itemize}
    \item \textbf{Test for Divergence First:} If the terms of the series look "heavy" at the top (i.e., the limit of $a_n$ obviously isn't zero), check that first. It's often the quickest way to solve a divergence problem.
    \item \textbf{Geometric Series Recognition:} Anything that looks like (number)$^n$ is a candidate for a geometric series. Do the algebra to get it into the form $ar^{n-1}$.
    \item \textbf{Repeating Decimals:} A number with a repeating block of $k$ digits, $d_1d_2...d_k$, can be quickly written as the fraction $\frac{d_1d_2...d_k}{10^k - 1}$. For example, $0.\overline{57}$ has $k=2$ digits, so it's $\frac{57}{10^2-1} = \frac{57}{99}$. For $0.\overline{4}$, $k=1$, so it's $\frac{4}{10^1-1} = \frac{4}{9}$.
\end{itemize}

\section{Common Pitfalls and How to Avoid Them}
\begin{itemize}
    \item \textbf{Mistake:} Assuming if $\lim_{n \to \infty} a_n = 0$, the series must converge.
    \begin{itemize}
        \item \textbf{Avoidance:} Burn the harmonic series $\sum \frac{1}{n}$ into your memory. Here, $a_n=1/n \to 0$, but the series diverges. This is the ultimate counterexample. The Test for Divergence is a one-way street.
    \end{itemize}
    \item \textbf{Mistake:} Messing up the first term, $a$, in a geometric series.
    \begin{itemize}
        \item \textbf{Avoidance:} Don't just assume $a$ is the coefficient in front. Always find the first term by plugging the starting index (e.g., $n=1$) into your formula for the terms of the series.
    \end{itemize}
    \item \textbf{Mistake:} Confusing a sequence with a series.
    \begin{itemize}
        \item \textbf{Avoidance:} Remember, $\{a_n\}$ is a list, $\sum a_n$ is a sum. The convergence of one does not imply the convergence of the other (as seen in Problem 6, where the sequence $\{7n/(4n+1)\}$ converges but the series $\sum 7n/(4n+1)$ diverges).
    \end{itemize}
\end{itemize}

\part{Conceptual Synthesis and The "Big Picture"}
\section{Thematic Connections}
The core theme of this topic is \textbf{making sense of infinite sums by turning them into limit problems}. This is a powerful and recurring idea in calculus. We saw it first with derivatives, where the instantaneous rate of change was the limit of average rates of change. We saw it again with definite integrals, where the exact area under a curve was the limit of the sum of the areas of a finite number of rectangles (a Riemann sum).

Here, we take the seemingly impossible task of adding infinitely many numbers and reframe it as a more manageable question: "What is the limit of the sequence of finite partial sums?" This strategy—approximating an infinite process with a sequence of finite steps and then taking the limit—is arguably the single most important theme in all of calculus.

\section{Forward and Backward Links}
\begin{itemize}
    \item \textbf{Backward Links (Foundations):} This topic is a direct and logical extension of our study of \textbf{sequences and limits}. A series $\sum a_n$ cannot be understood without first understanding the sequence of its terms $\{a_n\}$ and, more importantly, the sequence of its partial sums $\{S_n\}$. The convergence of a series is \textit{defined} by the convergence of the sequence $\{S_n\}$. All the techniques you learned for evaluating limits at infinity are now directly applied to finding the sums of series or testing them for divergence.
    \item \textbf{Forward Links (Future Applications):} Understanding the basics of series convergence is absolutely critical for what comes next. This topic is the foundation for:
    \begin{itemize}
        \item \textbf{Taylor and Maclaurin Series:} These are infinite polynomial representations of more complicated functions (like $e^x$, $\sin x$, or $\ln(1+x)$). They are one of the most powerful tools in all of applied mathematics, used for approximating functions, solving differential equations, and evaluating difficult integrals. To understand when these polynomial approximations are valid, you must understand the convergence of the series that defines them.
        \item \textbf{Power Series:} These are functions defined as infinite series, and they are used to find solutions to differential equations that model physical systems in engineering and physics.
        \item \textbf{Fourier Series:} In signal processing and physics, these series are used to break down complex periodic signals (like a sound wave) into a sum of simple sine and cosine waves.
    \end{itemize}
\end{itemize}

\part{Real-World Application and Modeling}
\section{Concrete Scenarios in Finance and Economics}
\begin{itemize}
    \item \textbf{Scenario 1: Valuing a Perpetuity (Finance):} A perpetuity is a type of financial instrument that promises to pay a fixed amount of money, $C$, at regular intervals (e.g., yearly) forever. Examples include certain government bonds or preferred stocks. To find out what this instrument is worth today (its Present Value, PV), you must sum the discounted value of every future payment. If the interest rate is $i$, a payment of $C$ received in $n$ years is worth $C/(1+i)^n$ today. The total PV is an infinite geometric series:
    \[ PV = \frac{C}{1+i} + \frac{C}{(1+i)^2} + \frac{C}{(1+i)^3} + \dots \]
    This allows an infinite stream of payments to have a finite, calculable value.

    \item \textbf{Scenario 2: The Keynesian Multiplier (Economics):} When the government injects \$1 billion into the economy, that money doesn't just get spent once. The people who receive it will save some and spend the rest. Suppose people have a Marginal Propensity to Consume (MPC) of 0.8, meaning they spend 80\% of any new income. The initial \$1B is spent. The recipients then spend 80\% of that (\$800M). The next recipients spend 80\% of that (\$640M), and so on. The total impact on the economy is an infinite geometric series:
    \[ \text{Total Impact} = \$1B + \$1B(0.8) + \$1B(0.8)^2 + \$1B(0.8)^3 + \dots \]
    Summing this series tells economists the total "bang for the buck" of the stimulus.

    \item \textbf{Scenario 3: Dividend Discount Model (Stock Valuation):} A common way to value a company's stock is to assume its price is the sum of the present values of all future dividends it will ever pay. If a company's dividend is expected to grow at a constant rate, $g$, forever, the price of the stock becomes the sum of a convergent geometric series. This provides a theoretical price for the stock based on its future earnings potential.
\end{itemize}

\section{Model Problem Setup: Gordon Growth Model}
Let's set up the model for the third scenario: valuing a stock with constantly growing dividends.

\begin{itemize}
    \item \textbf{Problem:} A company is expected to pay a dividend of \$2.00 per share next year. Investors require a 10\% rate of return on this stock, and the company's dividend is expected to grow at a steady 3\% per year forever. What is the theoretical price of one share of this stock?
    \item \textbf{Variables:}
    \begin{itemize}
        \item $D_1 = \$2.00$ (dividend in year 1)
        \item $k = 0.10$ (required rate of return, or discount rate)
        \item $g = 0.03$ (constant growth rate of dividends)
    \end{itemize}
    \item \textbf{Model Formulation:} The price, $P$, is the sum of the present values of all future dividends.
    \begin{itemize}
        \item Dividend in year 2: $D_2 = D_1(1+g)$
        \item Dividend in year $n$: $D_n = D_1(1+g)^{n-1}$
        \item Present value of dividend in year $n$: $PV(D_n) = \frac{D_n}{(1+k)^n} = \frac{D_1(1+g)^{n-1}}{(1+k)^n}$
    \end{itemize}
    \item \textbf{The Series to be Solved:} The price is the sum of these present values from year 1 to infinity.
    \[ P = \sum_{n=1}^{\infty} \frac{D_1(1+g)^{n-1}}{(1+k)^n} \]
    This is a geometric series. To find its sum, we identify the first term ($a$) and the ratio ($r$).
    \begin{itemize}
        \item First term ($n=1$): $a = \frac{D_1(1+g)^0}{(1+k)^1} = \frac{D_1}{1+k}$
        \item Ratio: $r = \frac{1+g}{1+k}$
    \end{itemize}
    The sum is $P = \frac{a}{1-r} = \frac{D_1/(1+k)}{1 - (1+g)/(1+k)} = \frac{D_1/(1+k)}{(k-g)/(1+k)} = \frac{D_1}{k-g}$.
    Plugging in the numbers: $P = \frac{2.00}{0.10 - 0.03} = \frac{2.00}{0.07} \approx \$28.57$. The series allows us to assign a finite value to an infinite stream of growing payments.
\end{itemize}

\part{Common Variations and Untested Concepts}
My homework assignment focused on geometric series and the Test for Divergence. Here are two other fundamental types of series that were not included but are crucial to know.

\section{Telescoping Series}
A telescoping series is one where, in the sequence of partial sums, most of the terms cancel out, leaving only the first few and last few terms. The key is to use partial fraction decomposition.

\begin{itemize}
    \item \textbf{Explanation:} Consider a series of the form $\sum (f(n) - f(n+1))$.
    The $n$-th partial sum is:
    $S_n = (f(1) - f(2)) + (f(2) - f(3)) + (f(3) - f(4)) + \dots + (f(n) - f(n+1))$.
    Notice the internal cancellation: the $-f(2)$ cancels with $f(2)$, $-f(3)$ with $f(3)$, and so on. We are left with just:
    $S_n = f(1) - f(n+1)$.
    The sum of the series is then $S = \lim_{n \to \infty} S_n = \lim_{n \to \infty} (f(1) - f(n+1))$.
    \item \textbf{Worked Example:} Find the sum of the series $\sum_{n=1}^{\infty} \frac{1}{n(n+1)}$.
    \begin{enumerate}
        \item \textbf{Partial Fractions:} First, decompose the term $a_n = \frac{1}{n(n+1)}$.
        \[ \frac{1}{n(n+1)} = \frac{A}{n} + \frac{B}{n+1} \implies 1 = A(n+1) + Bn \]
        If $n=0$, $1 = A(1)$, so $A=1$. If $n=-1$, $1 = B(-1)$, so $B=-1$.
        Thus, $a_n = \frac{1}{n} - \frac{1}{n+1}$.
        \item \textbf{Write out the Partial Sum $S_n$:}
        \[ S_n = \left(1 - \frac{1}{2}\right) + \left(\frac{1}{2} - \frac{1}{3}\right) + \left(\frac{1}{3} - \frac{1}{4}\right) + \dots + \left(\frac{1}{n} - \frac{1}{n+1}\right) \]
        \item \textbf{Cancel Terms:} All the middle terms cancel out.
        \[ S_n = 1 - \frac{1}{n+1} \]
        \item \textbf{Take the Limit:}
        \[ S = \lim_{n \to \infty} S_n = \lim_{n \to \infty} \left(1 - \frac{1}{n+1}\right) = 1 - 0 = 1 \]
        So, the series converges to 1.
    \end{enumerate}
\end{itemize}

\section{The Harmonic Series}
This is not a type of series to solve, but rather a specific, critical example to memorize.
\begin{itemize}
    \item \textbf{Definition:} The harmonic series is $\sum_{n=1}^{\infty} \frac{1}{n} = 1 + \frac{1}{2} + \frac{1}{3} + \frac{1}{4} + \dots$.
    \item \textbf{Importance:} This is the most famous example of a series that \textbf{DIVERGES} even though its terms approach zero ($\lim_{n \to \infty} 1/n = 0$). This proves that the converse of the Test for Divergence is false. Just because the terms go to zero does not guarantee convergence. You will use this as a benchmark for comparison tests in later sections.
\end{itemize}


\part{Advanced Diagnostic Testing: "Find the Flaw"}
For each problem below, a flawed solution is presented. Your task is to find the single critical error, explain why it is an error in one sentence, and then provide the correct step and final answer.

\subsection*{Problem 1}
Determine if the series $\sum_{n=1}^{\infty} 5 \left(\frac{3}{2}\right)^n$ converges or diverges.
\begin{itemize}
    \item \textbf{Flawed Solution:}
    \begin{enumerate}
        \item This is a geometric series with $a = 5$ and $r = 3/2$.
        \item The sum of a geometric series is $S = \frac{a}{1-r}$.
        \item $S = \frac{5}{1 - 3/2} = \frac{5}{-1/2} = -10$.
        \item The series converges to -10.
    \end{enumerate}
\end{itemize}
\textit{My Analysis:}
\begin{itemize}
    \item \textbf{The Flaw is in Step 2:} The formula for the sum of a geometric series is only valid when $|r| < 1$.
    \item \textbf{Correction:} Here, $|r| = |3/2| = 3/2 > 1$. Therefore, the series diverges. The sum formula cannot be applied.
    \item \textbf{Correct Answer:} DIVERGES.
\end{itemize}

\subsection*{Problem 2}
Find the sum of the series $\sum_{n=0}^{\infty} \frac{3}{4^n}$.
\begin{itemize}
    \item \textbf{Flawed Solution:}
    \begin{enumerate}
        \item This is a geometric series $\sum_{n=0}^{\infty} 3 \left(\frac{1}{4}\right)^n$.
        \item The ratio is $r=1/4$. Since $|r|<1$, the series converges.
        \item The first term is $a = 3$.
        \item The sum is $S = \frac{a}{1-r} = \frac{3}{1 - 1/4} = \frac{3}{3/4} = 4$.
    \end{enumerate}
\end{itemize}
\textit{My Analysis:}
\begin{itemize}
    \item \textbf{The Flaw is in Step 3:} The first term 'a' is calculated by plugging in the starting index, which is n=0, not n=1. The first term is $3(\frac{1}{4})^0 = 3 \cdot 1 = 3$. Oh wait, the solution *did* use $a=3$. Let me re-read. Ah, the provided solution is actually correct. I need to create a flaw. Let me rewrite the flawed solution.
\end{itemize}
\textbf{REVISED Problem 2}
Find the sum of the series $\sum_{n=1}^{\infty} 3 \left(\frac{1}{4}\right)^n$.
\begin{itemize}
    \item \textbf{Flawed Solution:}
    \begin{enumerate}
        \item This is a geometric series with ratio $r=1/4$. It converges.
        \item The first term is $a=3$.
        \item The sum is $S = \frac{a}{1-r} = \frac{3}{1 - 1/4} = \frac{3}{3/4} = 4$.
    \end{enumerate}
\end{itemize}
\textit{My Analysis:}
\begin{itemize}
    \item \textbf{The Flaw is in Step 2:} The first term $a$ is found by substituting the starting index $n=1$, not by taking the coefficient.
    \item \textbf{Correction:} The first term is $a = 3(\frac{1}{4})^1 = \frac{3}{4}$. The sum is $S = \frac{3/4}{1-1/4} = \frac{3/4}{3/4} = 1$.
    \item \textbf{Correct Answer:} 1.
\end{itemize}

\subsection*{Problem 3}
Determine if the series $\sum_{n=1}^{\infty} \frac{3n+1}{n}$ converges or diverges.
\begin{itemize}
    \item \textbf{Flawed Solution:}
    \begin{enumerate}
        \item Let's check the limit of the terms: $a_n = \frac{3n+1}{n} = 3 + \frac{1}{n}$.
        \item $\lim_{n \to \infty} a_n = \lim_{n \to \infty} (3 + \frac{1}{n}) = 3 + 0 = 3$.
        \item The terms approach 3.
        \item We can rewrite the series as $\sum (3 + \frac{1}{n}) = \sum 3 + \sum \frac{1}{n}$. Since $\sum \frac{1}{n}$ is the harmonic series which diverges, the whole series diverges.
    \end{enumerate}
\end{itemize}
\textit{My Analysis:}
\begin{itemize}
    \item \textbf{The Flaw is in Step 4:} While the conclusion is correct, the reasoning is incomplete and misses the most direct path; the Test for Divergence should have been invoked immediately.
    \item \textbf{Correction:} In Step 2, we found that $\lim_{n \to \infty} a_n = 3$. Since this limit is not 0, the series diverges by the Test for Divergence. There is no need to split the series.
    \item \textbf{Correct Answer:} DIVERGES.
\end{itemize}

\subsection*{Problem 4}
Find the sum of the telescoping series $\sum_{n=1}^{\infty} \left(\frac{1}{n+1} - \frac{1}{n}\right)$.
\begin{itemize}
    \item \textbf{Flawed Solution:}
    \begin{enumerate}
        \item Let's write out the $n$-th partial sum, $S_n$.
        \item $S_n = (\frac{1}{2} - 1) + (\frac{1}{3} - \frac{1}{2}) + (\frac{1}{4} - \frac{1}{3}) + \dots + (\frac{1}{n+1} - \frac{1}{n})$.
        \item The terms cancel out, leaving $S_n = 1 - \frac{1}{n+1}$.
        \item The sum is $S = \lim_{n \to \infty} (1 - \frac{1}{n+1}) = 1$.
    \end{enumerate}
\end{itemize}
\textit{My Analysis:}
\begin{itemize}
    \item \textbf{The Flaw is in Step 3:} The cancellation was done incorrectly; the terms that remain are the very first term and the very last term.
    \item \textbf{Correction:} In the sum $S_n = (\frac{1}{2} - 1) + (\frac{1}{3} - \frac{1}{2}) + \dots$, the $\frac{1}{2}$ cancels with the $-\frac{1}{2}$, etc., leaving the $-1$ from the first bracket and the $\frac{1}{n+1}$ from the last bracket. So, $S_n = -1 + \frac{1}{n+1}$.
    The correct sum is $S = \lim_{n \to \infty} (-1 + \frac{1}{n+1}) = -1$.
    \item \textbf{Correct Answer:} -1.
\end{itemize}

\subsection*{Problem 5}
Express $0.\overline{123}$ as a ratio of integers.
\begin{itemize}
    \item \textbf{Flawed Solution:}
    \begin{enumerate}
        \item We can write this as the series $0.123 + 0.000123 + \dots = \frac{123}{1000} + \frac{123}{1000000} + \dots$.
        \item This is a geometric series with first term $a = 123/1000$ and ratio $r=1/100$.
        \item The sum is $S = \frac{a}{1-r} = \frac{123/1000}{1 - 1/100} = \frac{123/1000}{99/100} = \frac{123}{1000} \cdot \frac{100}{99} = \frac{123}{990}$.
    \end{enumerate}
\end{itemize}
\textit{My Analysis:}
\begin{itemize}
    \item \textbf{The Flaw is in Step 2:} The common ratio is incorrect because each term is smaller by a factor of 1000, not 100.
    \item \textbf{Correction:} The ratio is $r = \frac{a_2}{a_1} = \frac{123/1000000}{123/1000} = \frac{1}{1000}$. The sum is $S = \frac{123/1000}{1 - 1/1000} = \frac{123/1000}{999/1000} = \frac{123}{999}$.
    \item \textbf{Correct Answer:} 123/999 (which simplifies to 41/333).
\end{itemize}

\end{document}