\documentclass{article}
\usepackage{amsmath}
\usepackage{amssymb}
\usepackage{geometry}
\geometry{a4paper, margin=1in}

\title{Homework 11.10 Taylor and Maclaurin Series}
\author{Tashfeen Omran}
\date{November 2025}

\begin{document}

\maketitle

\part{Comprehensive Introduction, Context, and Prerequisites}

\section{Core Concepts}
The central idea behind Taylor and Maclaurin series is one of the most powerful in calculus: \textbf{approximating complicated functions with simpler ones}. Specifically, we use polynomials to approximate functions like $f(x) = \sin(x)$ or $f(x) = e^x$. Polynomials are ideal because they are easy to evaluate, differentiate, and integrate.

A \textbf{Taylor series} is an infinite polynomial that is constructed to be a perfect match for a function $f(x)$ around a specific point, $x=a$. For the series to match the function, we force all of its derivatives (zeroth derivative, first derivative, second, etc.) to be equal to the derivatives of $f(x)$ at that point $a$.

The general form of a Taylor series for a function $f(x)$ centered at $x=a$ is:
\[ f(x) = \sum_{n=0}^{\infty} c_n (x-a)^n = c_0 + c_1(x-a) + c_2(x-a)^2 + c_3(x-a)^3 + \dots \]
The coefficients $c_n$ are precisely calculated to make the derivatives match. The formula for these coefficients is:
\[ c_n = \frac{f^{(n)}(a)}{n!} \]
where $f^{(n)}(a)$ is the $n$-th derivative of $f(x)$ evaluated at the point $a$, and $n!$ is the factorial of $n$.

A \textbf{Maclaurin series} is simply a special case of the Taylor series where the center point is $a=0$. This is the most common type of series expansion. The formula simplifies slightly:
\[ f(x) = \sum_{n=0}^{\infty} \frac{f^{(n)}(0)}{n!} x^n = f(0) + \frac{f'(0)}{1!}x + \frac{f''(0)}{2!}x^2 + \frac{f'''(0)}{3!}x^3 + \dots \]

A \textbf{Taylor polynomial} is a finite version of the series. The $N$-th degree Taylor polynomial, denoted $T_N(x)$, is the sum of the first $N+1$ terms and serves as an approximation of the function. As the degree $N$ increases, the approximation generally gets better.

\section{Intuition and Derivation}
Imagine you want to approximate a function $f(x)$ near a point $x=a$.
\begin{enumerate}
    \item \textbf{Zeroth-Order Approximation (Constant)}: The simplest approximation is just the function's value at that point: $f(x) \approx f(a)$. This is a horizontal line. It's only good exactly at $x=a$.

    \item \textbf{First-Order Approximation (Linear)}: To do better, we can use the tangent line at $x=a$. The equation of the tangent line is $L(x) = f(a) + f'(a)(x-a)$. This is the first-degree Taylor polynomial, $T_1(x)$. It matches the function's value and its slope (first derivative) at $x=a$.

    \item \textbf{Second-Order Approximation (Quadratic)}: To improve further, we can create a parabola that not only has the same value and slope at $x=a$, but also the same concavity (second derivative). This parabola is $T_2(x) = f(a) + f'(a)(x-a) + \frac{f''(a)}{2}(x-a)^2$. Notice the coefficient $\frac{f''(a)}{2!}$. If you take two derivatives of $T_2(x)$ and evaluate at $x=a$, you get exactly $f''(a)$, as desired.

    \item \textbf{Generalizing to the n-th Order}: We continue this process. To match the $n$-th derivative of $f(x)$ at $x=a$, we need to add a term of the form $c_n(x-a)^n$. If we take $n$ derivatives of this term, we get $c_n \cdot n!$. We want this to equal $f^{(n)}(a)$. So, we must have $c_n \cdot n! = f^{(n)}(a)$, which gives us the general formula for the coefficients: $c_n = \frac{f^{(n)}(a)}{n!}$.
\end{enumerate}
This logical progression ensures that the Taylor series is the "best" possible polynomial representation of a function around a point, because it's constructed to match the function in every possible way (value, slope, concavity, jerk, etc.) at that specific point.

\section{Historical Context and Motivation}
The ideas behind Taylor series were developed long before they were formalized. In the 14th century, mathematicians in the Kerala school of astronomy and mathematics in India found series for sine, cosine, and arctangent to aid in astronomical calculations.

In 17th century Europe, the primary motivation was computational. Mathematicians like Newton and Gregory needed ways to calculate values for transcendental functions (like logarithms, trigonometric, and exponential functions) that appeared in physics, astronomy, and navigation. Before calculators, computing $\sin(0.1)$ was a difficult task. However, if $\sin(x)$ could be represented as an infinite polynomial, $\sin(x) = x - \frac{x^3}{6} + \frac{x^5}{120} - \dots$, then calculating $\sin(0.1)$ becomes a matter of simple arithmetic: $0.1 - \frac{0.001}{6} + \frac{0.00001}{120} - \dots$. This provided a powerful method for creating accurate mathematical tables.

The English mathematician \textbf{Brook Taylor} formally introduced the general method in his 1715 book \textit{Methodus Incrementorum Directa et Inversa}. A few decades later, the Scottish mathematician \textbf{Colin Maclaurin} made extensive use of the special case where the series is centered at zero, and this important special case was named after him.

\section{Key Formulas}
\begin{itemize}
    \item \textbf{Taylor Series of $f$ centered at $a$:}
        \[ f(x) = \sum_{n=0}^{\infty} \frac{f^{(n)}(a)}{n!} (x-a)^n \]

    \item \textbf{Maclaurin Series of $f$ (Taylor series centered at $a=0$):}
        \[ f(x) = \sum_{n=0}^{\infty} \frac{f^{(n)}(0)}{n!} x^n \]

    \item \textbf{Radius of Convergence (from Ratio Test):} For a power series $\sum a_n(x-a)^n$, the radius of convergence $R$ is typically found by requiring the limit of the ratio of consecutive terms to be less than 1:
        \[ L = \lim_{n \to \infty} \left| \frac{a_{n+1}(x-a)^{n+1}}{a_n(x-a)^n} \right| = |x-a| \lim_{n \to \infty} \left| \frac{a_{n+1}}{a_n} \right| < 1 \]
        This gives $|x-a| < R$, where $R = \frac{1}{\lim_{n \to \infty} |a_{n+1}/a_n|} = \lim_{n \to \infty} \left| \frac{a_n}{a_{n+1}} \right|$.

    \item \textbf{Some Important Maclaurin Series to Memorize:}
        \begin{align*}
            e^x &= \sum_{n=0}^{\infty} \frac{x^n}{n!} = 1 + x + \frac{x^2}{2!} + \frac{x^3}{3!} + \dots \quad (R=\infty) \\
            \sin(x) &= \sum_{n=0}^{\infty} \frac{(-1)^n x^{2n+1}}{(2n+1)!} = x - \frac{x^3}{3!} + \frac{x^5}{5!} - \dots \quad (R=\infty) \\
            \cos(x) &= \sum_{n=0}^{\infty} \frac{(-1)^n x^{2n}}{(2n)!} = 1 - \frac{x^2}{2!} + \frac{x^4}{4!} - \dots \quad (R=\infty) \\
            \frac{1}{1-x} &= \sum_{n=0}^{\infty} x^n = 1 + x + x^2 + x^3 + \dots \quad (R=1) \\
            \ln(1+x) &= \sum_{n=1}^{\infty} \frac{(-1)^{n-1} x^n}{n} = x - \frac{x^2}{2} + \frac{x^3}{3} - \dots \quad (R=1)
        \end{align*}
\end{itemize}

\section{Prerequisites}
To master Taylor and Maclaurin series, the following skills are essential:
\begin{itemize}
    \item \textbf{Differentiation}: You must be proficient at finding derivatives, including higher-order derivatives (second, third, fourth, etc.), and using rules like the chain rule, product rule, and quotient rule flawlessly.
    \item \textbf{Factorials}: A solid understanding of factorial notation ($n!$) and how to simplify expressions involving factorials (e.g., $\frac{(n+1)!}{n!} = n+1$).
    \item \textbf{Power Series}: You should already be familiar with the concept of a power series, its general form $\sum c_n(x-a)^n$, and the idea of a radius and interval of convergence.
    \item \textbf{Ratio Test}: The Ratio Test is the primary tool for finding the radius of convergence of a Taylor series. You must be able to apply it correctly.
    \item \textbf{Algebraic Manipulation}: Strong skills in simplifying complex fractions and handling limits are crucial, especially for the Ratio Test.
    \item \textbf{Sigma Notation}: Comfort with reading, interpreting, and manipulating sums in sigma notation ($\sum$) is necessary.
\end{itemize}

\part{Detailed Homework Solutions}

\subsection{Problem 1}
\textbf{Question:} If $f(x) = \sum_{n=0}^{\infty} b_n(x-3)^n$ for all $x$, write a formula for $b_9$.

\textbf{Solution:}
\begin{enumerate}
    \item The given series is a Taylor series for the function $f(x)$ centered at $a=3$. The general formula for the coefficients of a Taylor series centered at $a$ is:
        \[ b_n = \frac{f^{(n)}(a)}{n!} \]
    \item In this problem, the center is $a=3$, and we are asked to find the specific coefficient $b_9$.
    \item We substitute $n=9$ and $a=3$ into the general formula:
        \[ b_9 = \frac{f^{(9)}(3)}{9!} \]
    \item Comparing this result to the given options, we find that it matches the second option.
\end{enumerate}
\textbf{Final Answer:} $b_9 = \frac{f^{(9)}(3)}{9!}$

\subsection{Problem 2}
\textbf{Question:} If $f^{(n)}(0) = (n+1)!$ for $n = 0, 1, 2, \dots$, find the Maclaurin series for $f$ and its radius of convergence $R$.

\textbf{Solution:}
\begin{enumerate}
    \item \textbf{Find the Maclaurin Series:}
        The formula for a Maclaurin series is:
        \[ f(x) = \sum_{n=0}^{\infty} \frac{f^{(n)}(0)}{n!} x^n \]
        We are given that $f^{(n)}(0) = (n+1)!$. We substitute this into the formula:
        \[ f(x) = \sum_{n=0}^{\infty} \frac{(n+1)!}{n!} x^n \]
        Now, we simplify the coefficient. Recall that $(n+1)! = (n+1) \cdot n \cdot (n-1) \cdots 1 = (n+1) \cdot n!$.
        \[ \frac{(n+1)!}{n!} = \frac{(n+1) \cdot n!}{n!} = n+1 \]
        So, the Maclaurin series is:
        \[ \sum_{n=0}^{\infty} (n+1)x^n \]

    \item \textbf{Find the Radius of Convergence R:}
        We use the Ratio Test. Let $a_n = (n+1)x^n$.
        \begin{align*}
            L &= \lim_{n \to \infty} \left| \frac{a_{n+1}}{a_n} \right| \\
            &= \lim_{n \to \infty} \left| \frac{(n+1+1)x^{n+1}}{(n+1)x^n} \right| \\
            &= \lim_{n \to \infty} \left| \frac{(n+2)x^{n+1}}{(n+1)x^n} \right| \\
            &= \lim_{n \to \infty} \left| \frac{n+2}{n+1} \cdot \frac{x^{n+1}}{x^n} \right| \\
            &= |x| \lim_{n \to \infty} \frac{n+2}{n+1}
        \end{align*}
        To evaluate the limit, we divide the numerator and denominator by the highest power of $n$, which is $n$:
        \[ |x| \lim_{n \to \infty} \frac{1+2/n}{1+1/n} = |x| \frac{1+0}{1+0} = |x| \]
        For the series to converge, the result of the Ratio Test must be less than 1:
        \[ L = |x| < 1 \]
        This inequality describes the interval of convergence. The radius of convergence $R$ is the number on the right side.
        \[ R = 1 \]
\end{enumerate}
\textbf{Final Answers:}
\begin{itemize}
    \item Maclaurin Series: $\sum_{n=0}^{\infty} (n+1)x^n$
    \item Radius of Convergence: $R = 1$
\end{itemize}

\subsection{Problem 3}
\textbf{Question:} Find the Taylor series for $f$ centered at 9 if $f^{(n)}(9) = \frac{(-1)^n n!}{5^n(n+3)}$. What is the radius of convergence $R$ of the Taylor series?

\textbf{Solution:}
\begin{enumerate}
    \item \textbf{Find the Taylor Series:}
        The formula for a Taylor series centered at $a=9$ is:
        \[ f(x) = \sum_{n=0}^{\infty} \frac{f^{(n)}(9)}{n!} (x-9)^n \]
        We are given the formula for the $n$-th derivative at 9. We substitute it in:
        \[ f(x) = \sum_{n=0}^{\infty} \frac{\frac{(-1)^n n!}{5^n(n+3)}}{n!} (x-9)^n \]
        Now, we simplify the coefficient by canceling the $n!$ terms:
        \[ f(x) = \sum_{n=0}^{\infty} \frac{(-1)^n}{5^n(n+3)} (x-9)^n \]

    \item \textbf{Find the Radius of Convergence R:}
        We use the Ratio Test. Let the entire term be $a_n$.
        \begin{align*}
            L &= \lim_{n \to \infty} \left| \frac{a_{n+1}}{a_n} \right| \\
            &= \lim_{n \to \infty} \left| \frac{\frac{(-1)^{n+1}(x-9)^{n+1}}{5^{n+1}(n+1+3)}}{\frac{(-1)^n(x-9)^n}{5^n(n+3)}} \right| \\
            &= \lim_{n \to \infty} \left| \frac{(x-9)^{n+1}}{5^{n+1}(n+4)} \cdot \frac{5^n(n+3)}{(x-9)^n} \right| \\
            &= \lim_{n \to \infty} \left| \frac{x-9}{5} \cdot \frac{n+3}{n+4} \right| \\
            &= \frac{|x-9|}{5} \lim_{n \to \infty} \frac{n+3}{n+4}
        \end{align*}
        Evaluating the limit:
        \[ \frac{|x-9|}{5} \lim_{n \to \infty} \frac{1+3/n}{1+4/n} = \frac{|x-9|}{5} \cdot 1 = \frac{|x-9|}{5} \]
        For convergence, this limit must be less than 1:
        \[ \frac{|x-9|}{5} < 1 \]
        \[ |x-9| < 5 \]
        From this inequality, we can see that the radius of convergence is 5.
\end{enumerate}
\textbf{Final Answers:}
\begin{itemize}
    \item Taylor Series: $\sum_{n=0}^{\infty} \frac{(-1)^n(x-9)^n}{5^n(n+3)}$
    \item Radius of Convergence: $R = 5$
\end{itemize}

\subsection{Problem 4}
\textbf{Question:} Find the Maclaurin series for $f(x) = 7(1-x)^{-2}$ and its associated radius of convergence $R$.

\textbf{Solution:}
\begin{enumerate}
    \item \textbf{Find the Maclaurin Series (Method 1: Differentiation):}
        We need to find a pattern for $f^{(n)}(0)$.
        \begin{align*}
            f(x) &= 7(1-x)^{-2} & f(0) &= 7 \\
            f'(x) &= 7(-2)(1-x)^{-3}(-1) = 14(1-x)^{-3} & f'(0) &= 14 \\
            f''(x) &= 14(-3)(1-x)^{-4}(-1) = 42(1-x)^{-4} & f''(0) &= 42 \\
            f'''(x) &= 42(-4)(1-x)^{-5}(-1) = 168(1-x)^{-5} & f'''(0) &= 168
        \end{align*}
        Let's rewrite these in terms of factorials to find a pattern:
        $f(0) = 7 = 7 \cdot 1!$
        $f'(0) = 14 = 7 \cdot 2!$
        $f''(0) = 42 = 7 \cdot 3!$
        $f'''(0) = 168 = 7 \cdot 4!$
        The pattern is $f^{(n)}(0) = 7 \cdot (n+1)!$.
        Now substitute this into the Maclaurin series formula:
        \[ f(x) = \sum_{n=0}^{\infty} \frac{f^{(n)}(0)}{n!} x^n = \sum_{n=0}^{\infty} \frac{7 \cdot (n+1)!}{n!} x^n = \sum_{n=0}^{\infty} 7(n+1)x^n \]

    \item \textbf{Find the Maclaurin Series (Method 2: Differentiating a Known Series):}
        We know the geometric series:
        \[ \frac{1}{1-x} = \sum_{n=0}^{\infty} x^n \quad (\text{for } |x|<1) \]
        Notice that our function $f(x)$ is related to the derivative of this.
        \[ \frac{d}{dx}\left(\frac{1}{1-x}\right) = \frac{d}{dx}(1-x)^{-1} = -1(1-x)^{-2}(-1) = \frac{1}{(1-x)^2} \]
        So, $f(x) = 7 \cdot \frac{d}{dx}\left(\frac{1}{1-x}\right)$. We can differentiate the series term-by-term:
        \[ \frac{d}{dx}\left(\sum_{n=0}^{\infty} x^n\right) = \sum_{n=0}^{\infty} \frac{d}{dx}(x^n) = \sum_{n=1}^{\infty} nx^{n-1} \]
        (Note the index starts at $n=1$ because the derivative of the constant term $x^0$ is zero).
        Let's re-index this sum to start at $n=0$. Let $k=n-1$, so $n=k+1$.
        \[ \sum_{k=0}^{\infty} (k+1)x^k \]
        This is the series for $\frac{1}{(1-x)^2}$. Our function is 7 times this:
        \[ f(x) = 7 \sum_{n=0}^{\infty} (n+1)x^n = \sum_{n=0}^{\infty} 7(n+1)x^n \]

    \item \textbf{Find the Radius of Convergence R:}
        Using the Ratio Test on $a_n = 7(n+1)x^n$:
        \begin{align*}
            L &= \lim_{n \to \infty} \left| \frac{7(n+2)x^{n+1}}{7(n+1)x^n} \right| \\
            &= |x| \lim_{n \to \infty} \frac{n+2}{n+1} = |x| \cdot 1 = |x|
        \end{align*}
        For convergence, $|x| < 1$, so the radius of convergence is $R=1$.
        (Alternatively, differentiating a power series does not change its radius of convergence. Since the geometric series for $\frac{1}{1-x}$ has $R=1$, our resulting series also has $R=1$).
\end{enumerate}
\textbf{Final Answers:}
\begin{itemize}
    \item Maclaurin Series: $\sum_{n=0}^{\infty} 7(n+1)x^n$
    \item Radius of Convergence: $R = 1$
\end{itemize}

\subsection{Problem 5}
This is a tutorial version of Problem 4 with $f(x) = 6(1-x)^{-2}$. The steps and logic are identical, just with the constant 6 instead of 7.
\textbf{Final Answers:}
\begin{itemize}
    \item Maclaurin Series: $\sum_{n=0}^{\infty} 6(n+1)x^n$
    \item Radius of Convergence: $R = 1$
\end{itemize}

\subsection{Problem 6}
\textbf{Question:} Find the Maclaurin series for $f(x) = \ln(1+3x)$ and its associated radius of convergence $R$.

\textbf{Solution:}
\begin{enumerate}
    \item \textbf{Find the Maclaurin Series (Method: Integrating a Known Series):}
        We start with the geometric series formula:
        \[ \frac{1}{1-u} = \sum_{n=0}^{\infty} u^n \quad (\text{for } |u|<1) \]
        Substitute $u = -3x$:
        \[ \frac{1}{1-(-3x)} = \frac{1}{1+3x} = \sum_{n=0}^{\infty} (-3x)^n = \sum_{n=0}^{\infty} (-1)^n 3^n x^n \]
        Now, notice that $\int \frac{1}{1+3x} dx = \frac{1}{3}\ln(1+3x)$. This means our function is related to the integral of this series.
        \[ \ln(1+3x) = 3 \int \frac{1}{1+3x} dx \]
        We can integrate the power series term-by-term:
        \begin{align*}
            \int \left( \sum_{n=0}^{\infty} (-1)^n 3^n x^n \right) dx &= \sum_{n=0}^{\infty} \int (-1)^n 3^n x^n dx \\
            &= \sum_{n=0}^{\infty} (-1)^n 3^n \frac{x^{n+1}}{n+1} + C
        \end{align*}
        So, $\frac{1}{3}\ln(1+3x) = \sum_{n=0}^{\infty} (-1)^n 3^n \frac{x^{n+1}}{n+1} + C$. To find $C$, we plug in $x=0$.
        $\frac{1}{3}\ln(1) = 0+C \implies 0=C$.
        Now multiply by 3:
        \[ \ln(1+3x) = \sum_{n=0}^{\infty} (-1)^n 3^{n+1} \frac{x^{n+1}}{n+1} \]
        The problem in the image shows the summation starting from $n=1$. Let's re-index. Let $k = n+1$, so $n=k-1$.
        \[ \sum_{k=1}^{\infty} (-1)^{k-1} 3^k \frac{x^k}{k} \]
        Switching the index variable back to $n$:
        \[ \sum_{n=1}^{\infty} \frac{(-1)^{n-1} 3^n x^n}{n} \]

    \item \textbf{Find the Radius of Convergence R:}
        Integrating a power series does not change its radius of convergence. The series we started with, for $\frac{1}{1+3x}$, converges for $|-3x|<1$, which means $|x| < 1/3$. Therefore, the radius of convergence for the integrated series is also $R=1/3$.
        Let's verify with the Ratio Test on our final series. Let $a_n = \frac{(-1)^{n-1} 3^n x^n}{n}$.
        \begin{align*}
            L &= \lim_{n \to \infty} \left| \frac{a_{n+1}}{a_n} \right| \\
            &= \lim_{n \to \infty} \left| \frac{3^{n+1}x^{n+1}}{n+1} \cdot \frac{n}{3^n x^n} \right| \\
            &= |3x| \lim_{n \to \infty} \frac{n}{n+1} = |3x| \cdot 1 = 3|x|
        \end{align*}
        For convergence, $3|x| < 1$, which gives $|x| < 1/3$.
        So the radius of convergence is $R=1/3$.
\end{enumerate}
\textbf{Final Answers:}
\begin{itemize}
    \item Maclaurin Series: $\sum_{n=1}^{\infty} \frac{(-1)^{n-1} 3^n x^n}{n}$
    \item Radius of Convergence: $R = 1/3$
\end{itemize}

\subsection{Problem 7}
\textbf{Question:} Consider $f(x)=\ln(1+4x)$. Complete the table of derivatives, find the Maclaurin series, and find the radius of convergence.

\textbf{Solution:}
This problem is very similar to Problem 6, with a constant of 4 instead of 3.

\begin{enumerate}
    \item \textbf{Complete the Table:}
        \begin{align*}
            f^{(0)}(x) = \ln(1+4x) &\implies f^{(0)}(0) = \ln(1) = 0 \\
            f^{(1)}(x) = \frac{4}{1+4x} &\implies f^{(1)}(0) = \frac{4}{1} = 4 \\
            f^{(2)}(x) = -4(1+4x)^{-2}(4) = -16(1+4x)^{-2} &\implies f^{(2)}(0) = -16 \\
            f^{(3)}(x) = -16(-2)(1+4x)^{-3}(4) = 128(1+4x)^{-3} &\implies f^{(3)}(0) = 128 \\
            f^{(4)}(x) = 128(-3)(1+4x)^{-4}(4) = -1536(1+4x)^{-4} &\implies f^{(4)}(0) = -1536
        \end{align*}

    \item \textbf{Find the Maclaurin Series:}
        Using the same integration method as in Problem 6, we would find the series is:
        \[ \sum_{n=1}^{\infty} \frac{(-1)^{n-1} 4^n x^n}{n} \]

    \item \textbf{Find the Radius of Convergence R:}
        The geometric series for $\frac{1}{1+4x}$ converges when $|-4x|<1$, which is $|x| < 1/4$. Integrating does not change the radius.
        \[ R = 1/4 \]
\end{enumerate}
\textbf{Final Answers:}
\begin{itemize}
    \item Table: $f^{(0)}(0)=0$, $f^{(1)}(0)=4$, $f^{(2)}(0)=-16$, $f^{(3)}(0)=128$, $f^{(4)}(0)=-1536$
    \item Maclaurin Series: $\sum_{n=1}^{\infty} \frac{(-1)^{n-1} 4^n x^n}{n}$
    \item Radius of Convergence: $R = 1/4$
\end{itemize}

\subsection{Problem 8}
\textbf{Question:} Find the Maclaurin series for $f(x) = \sin(x)$ and its associated radius of convergence $R$.

\textbf{Solution:}
\begin{enumerate}
    \item \textbf{Find the Maclaurin Series:}
        This is a standard, fundamental series. We find it by calculating the derivatives at $a=0$.
        \begin{align*}
            f(x) &= \sin(x) & f(0) &= 0 \\
            f'(x) &= \cos(x) & f'(0) &= 1 \\
            f''(x) &= -\sin(x) & f''(0) &= 0 \\
            f'''(x) &= -\cos(x) & f'''(0) &= -1 \\
            f^{(4)}(x) &= \sin(x) & f^{(4)}(0) &= 0
        \end{align*}
        The pattern of derivatives at 0 is $0, 1, 0, -1, 0, 1, 0, -1, \dots$.
        Now we build the series using the formula $f(x) = \sum \frac{f^{(n)}(0)}{n!} x^n$:
        \begin{align*}
            f(x) &= \frac{0}{0!}x^0 + \frac{1}{1!}x^1 + \frac{0}{2!}x^2 + \frac{-1}{3!}x^3 + \frac{0}{4!}x^4 + \frac{1}{5!}x^5 + \dots \\
            &= x - \frac{x^3}{3!} + \frac{x^5}{5!} - \frac{x^7}{7!} + \dots
        \end{align*}
        Only the odd-powered terms are non-zero. The signs alternate. The general term corresponds to an odd exponent, which we can write as $2n+1$. The series in sigma notation is:
        \[ \sum_{n=0}^{\infty} \frac{(-1)^n x^{2n+1}}{(2n+1)!} \]

    \item \textbf{Find the Radius of Convergence R:}
        We use the Ratio Test.
        \begin{align*}
            L &= \lim_{n \to \infty} \left| \frac{a_{n+1}}{a_n} \right| \\
            &= \lim_{n \to \infty} \left| \frac{\frac{(-1)^{n+1}x^{2(n+1)+1}}{(2(n+1)+1)!}}{\frac{(-1)^n x^{2n+1}}{(2n+1)!}} \right| \\
            &= \lim_{n \to \infty} \left| \frac{x^{2n+3}}{(2n+3)!} \cdot \frac{(2n+1)!}{x^{2n+1}} \right| \\
            &= \lim_{n \to \infty} \left| x^2 \cdot \frac{(2n+1)!}{(2n+3)(2n+2)(2n+1)!} \right| \\
            &= |x^2| \lim_{n \to \infty} \frac{1}{(2n+3)(2n+2)}
        \end{align*}
        As $n \to \infty$, the denominator goes to infinity, so the limit is 0.
        \[ L = |x^2| \cdot 0 = 0 \]
        The condition for convergence is $L < 1$. Since $0 < 1$ for all possible values of $x$, the series converges for all $x$. This means the radius of convergence is infinite.
        \[ R = \infty \]
\end{enumerate}
\textbf{Final Answers:}
\begin{itemize}
    \item Maclaurin Series: $\sum_{n=0}^{\infty} \frac{(-1)^n x^{2n+1}}{(2n+1)!}$
    \item Radius of Convergence: $R = \infty$
\end{itemize}

\part{In-Depth Analysis of Problems and Techniques}

\section{Problem Types and General Approach}
The homework problems can be categorized into four main types:

\begin{itemize}
    \item \textbf{Type 1: Direct Definition of Coefficients (Problem 1)}
        \begin{itemize}
            \item \textbf{Description:} These problems test the fundamental definition of Taylor coefficients without requiring you to know the function itself.
            \item \textbf{Approach:} Directly apply the formula $c_n = \frac{f^{(n)}(a)}{n!}$. Identify the given values of $n$ (the index of the coefficient) and $a$ (the center of the series) and substitute them.
        \end{itemize}

    \item \textbf{Type 2: Series from a Given Derivative Formula (Problems 2, 3)}
        \begin{itemize}
            \item \textbf{Description:} You are provided with a general formula for $f^{(n)}(a)$ and asked to construct the series.
            \item \textbf{Approach:} Substitute the given formula into the general Taylor/Maclaurin series formula $\sum \frac{f^{(n)}(a)}{n!} (x-a)^n$. Simplify the resulting coefficient as much as possible (often involving canceling factorials). Then, apply the Ratio Test to find the radius of convergence.
        \end{itemize}

    \item \textbf{Type 3: Building a Series from a Known Function (Problems 4, 5, 6, 7, 8)}
        \begin{itemize}
            \item \textbf{Description:} These are the most common problems. You are given a function $f(x)$ and must find its series representation.
            \item \textbf{Approach:} There are two main strategies:
                \begin{enumerate}
                    \item \textbf{From Scratch:} Calculate the first few derivatives of $f(x)$ and evaluate them at the center $a$. Look for a pattern in the sequence $f^{(n)}(a)$. Use this pattern to write a general formula for the $n$-th term of the series. This was necessary for Problem 8 ($f(x)=\sin(x)$) and used as one method in Problem 4.
                    \item \textbf{From a Known Series:} This is a powerful shortcut. Recognize that your function can be obtained by substituting into, differentiating, or integrating a known series (like the geometric series). Problems 4, 5, 6, and 7 are all related to the geometric series $\frac{1}{1-x}$. This approach is often faster and less error-prone than finding a pattern from scratch.
                \end{enumerate}
            Once the series is found, always use the Ratio Test to determine the radius of convergence.
        \end{itemize}
\end{itemize}

\section{Key Algebraic and Calculus Manipulations}

\begin{itemize}
    \item \textbf{Factorial Simplification:} This is a critical algebraic skill for simplifying coefficients.
        \begin{itemize}
            \item \textbf{Example (Problem 2):} Simplifying $\frac{(n+1)!}{n!}$ to $n+1$. This was the key step to finding the coefficient in the series $\sum (n+1)x^n$.
            \item \textbf{Example (Problem 8):} Simplifying $\frac{(2n+1)!}{(2n+3)!}$ to $\frac{1}{(2n+3)(2n+2)}$ during the Ratio Test. This step is necessary to evaluate the limit correctly.
        \end{itemize}

    \item \textbf{Pattern Recognition in Derivatives:} This is the core calculus skill for deriving series from first principles.
        \begin{itemize}
            \item \textbf{Example (Problem 8):} For $f(x)=\sin(x)$, recognizing that the sequence of derivatives at zero, $f^{(n)}(0)$, follows the repeating pattern $0, 1, 0, -1, \dots$. This recognition is what allows you to construct the series $x - \frac{x^3}{3!} + \dots$.
            \item \textbf{Example (Problem 4):} For $f(x)=7(1-x)^{-2}$, finding that $f^{(n)}(0) = 7(n+1)!$. This requires careful application of the chain rule and identifying how the constants and exponents evolve with each derivative.
        \end{itemize}

    \item \textbf{Applying the Ratio Test for Radius of Convergence:} This is the standard procedure for finding $R$.
        \begin{itemize}
            \item \textbf{Example (Problem 3):} The setup required simplifying the ratio of terms with factors of $5^n$, $(n+3)$, and $(x-9)^n$. The crucial step was separating the limit into parts: $\frac{|x-9|}{5} \lim_{n \to \infty} \frac{n+3}{n+4}$. This isolates the part dependent on $x$ from the part that becomes 1. This allowed the final inequality $|x-9|<5$ to be found, revealing $R=5$.
        \end{itemize}

    \item \textbf{Term-by-Term Differentiation and Integration of Series:} This is the most elegant technique for many problems.
        \begin{itemize}
            \item \textbf{Example (Problem 4):} Recognizing that $\frac{1}{(1-x)^2}$ is the derivative of the geometric series $\sum x^n$. Differentiating the series term-by-term, $\frac{d}{dx}\sum x^n = \sum nx^{n-1}$, provides the answer much more quickly than repeated differentiation of the function.
            \item \textbf{Example (Problem 6):} Realizing that $\ln(1+3x)$ is the integral of $\frac{3}{1+3x}$. By finding the series for $\frac{1}{1+3x}$ and integrating it term-by-term, $\int \sum (-1)^n 3^n x^n dx = \sum (-1)^n 3^n \frac{x^{n+1}}{n+1}$, we found the required series without calculating a single derivative of the logarithm function itself.
        \end{itemize}

    \item \textbf{Re-indexing a Series:} This is often needed to match a required format or to simplify an expression after differentiation/integration.
        \begin{itemize}
            \item \textbf{Example (Problem 4):} After differentiating $\sum x^n$ to get $\sum_{n=1}^{\infty} nx^{n-1}$, we used the substitution $k=n-1$ to rewrite the sum as $\sum_{k=0}^{\infty} (k+1)x^k$, which is a more standard power series form starting from an index of 0.
        \end{itemize}
\end{itemize}

\part{"Cheatsheet" and Tips for Success}

\section{Summary of Important Formulas}
\begin{itemize}
    \item \textbf{Taylor Series at $a$:} $f(x) = \sum_{n=0}^{\infty} \frac{f^{(n)}(a)}{n!} (x-a)^n$
    \item \textbf{Maclaurin Series (at $a=0$):} $f(x) = \sum_{n=0}^{\infty} \frac{f^{(n)}(0)}{n!} x^n$
    \item \textbf{Memorize These Five Series:}
        \begin{enumerate}
            \item $\frac{1}{1-x} = \sum_{n=0}^{\infty} x^n$, $R=1$
            \item $e^x = \sum_{n=0}^{\infty} \frac{x^n}{n!}$, $R=\infty$
            \item $\sin(x) = \sum_{n=0}^{\infty} \frac{(-1)^n x^{2n+1}}{(2n+1)!}$, $R=\infty$
            \item $\cos(x) = \sum_{n=0}^{\infty} \frac{(-1)^n x^{2n}}{(2n)!}$, $R=\infty$
            \item $\ln(1+x) = \sum_{n=1}^{\infty} \frac{(-1)^{n-1} x^n}{n}$, $R=1$
        \end{enumerate}
\end{itemize}

\section{Cheats, Tricks, and Shortcuts}
\begin{itemize}
    \item \textbf{Don't Reinvent the Wheel:} Before starting a long series of derivatives, check if the function is a simple variation of one of the five essential series listed above.
    \item \textbf{Substitute:} To find the series for $e^{x^2}$, don't take derivatives! Just take the series for $e^u = \sum \frac{u^n}{n!}$ and substitute $u=x^2$. This gives $\sum \frac{(x^2)^n}{n!} = \sum \frac{x^{2n}}{n!}$.
    \item \textbf{Differentiate:} To find the series for $\frac{1}{(1-x)^2}$, notice it's the derivative of $\frac{1}{1-x}$. Differentiate the series $\sum x^n$ to get $\sum nx^{n-1}$. This is much faster.
    \item \textbf{Integrate:} To find the series for $\arctan(x)$, notice it's the integral of $\frac{1}{1+x^2}$. Find the series for $\frac{1}{1+x^2}$ (by substituting $u=-x^2$ into the geometric series) and then integrate that series term-by-term.
    \item \textbf{Radius of Convergence Rules:} Differentiating or integrating a power series \textbf{does not change} its radius of convergence.
\end{itemize}

\section{Common Pitfalls and How to Avoid Them}
\begin{itemize}
    \item \textbf{Forgetting the $n!$:} This is the most common mistake. The coefficient is always the $n$-th derivative \textbf{divided by} $n!$. Write the formula down every time.
    \item \textbf{Errors in the Ratio Test:} Be very careful with algebra when simplifying the ratio $|a_{n+1}/a_n|$. Write out every step and be meticulous with fractions and exponents.
    \item \textbf{Mixing up Center $a$ and Variable $x$:} Remember that $a$ is a constant. In the formula $\frac{f^{(n)}(a)}{n!} (x-a)^n$, you evaluate the derivative at the constant $a$. The variable $x$ remains a variable.
    \item \textbf{Chain Rule Errors:} When finding derivatives of composite functions like $\ln(1+4x)$, it's easy to forget the extra factor from the chain rule in higher derivatives.
    \item \textbf{Off-by-One Index Errors:} When differentiating or integrating, the starting index of the sum may change (e.g., from $n=0$ to $n=1$). Be mindful of this. If in doubt, write out the first few terms of the original and new series to check your work.
\end{itemize}

\part{Conceptual Synthesis and The "Big Picture"}

\section{Thematic Connections}
The core theme of this topic is \textbf{infinite polynomial approximation}. We are trading complex, transcendental functions for "infinitely long" but structurally simple polynomials. This powerful idea of approximation is a central theme in calculus and connects to several other topics:
\begin{itemize}
    \item \textbf{Linear Approximation (Tangent Lines):} The first-degree Taylor Polynomial, $T_1(x) = f(a) + f'(a)(x-a)$, is precisely the tangent line approximation you learned in first-semester calculus. Taylor series are a direct and powerful generalization of this fundamental idea.
    \item \textbf{Numerical Integration (Simpson's Rule, Trapezoid Rule):} In numerical integration, we approximate a function with simple shapes (rectangles, trapezoids, parabolas) to estimate the area under its curve. Taylor series work similarly, but they create a global approximation (a single polynomial) rather than a piecewise one.
    \item \textbf{L'Hôpital's Rule:} For indeterminate limits like $\lim_{x\to 0} \frac{\sin(x)}{x}$, replacing the functions with the first few terms of their Maclaurin series can provide great intuition. For example, $\lim_{x\to 0} \frac{x - x^3/6 + \dots}{x} = \lim_{x\to 0} (1 - x^2/6 + \dots) = 1$. This shows a deep connection between a function's local behavior (derivatives) and its limiting behavior.
\end{itemize}

\section{Forward and Backward Links}
\begin{itemize}
    \item \textbf{Backward Link (Dependence on Previous Concepts):}
        Taylor series are the culmination of several threads from earlier in calculus. They are a specific type of \textbf{Power Series}, so all the theory of convergence (especially the \textbf{Ratio Test}) applies directly. The very definition of the coefficients, $c_n = \frac{f^{(n)}(a)}{n!}$, is entirely dependent on the concept of \textbf{higher-order derivatives}. Without a mastery of differentiation and power series, Taylor series are impossible to understand.

    \item \textbf{Forward Link (Foundation for Future Topics):}
        The skills and concepts from Taylor series are not just an endpoint; they are a critical foundation for more advanced mathematics, science, and engineering.
        \begin{itemize}
            \item \textbf{Differential Equations:} Many important differential equations cannot be solved with simple functions. A standard technique is to assume the solution is a power series and use the differential equation to find a formula for the coefficients. This is the "series solution" method.
            \item \textbf{Complex Analysis:} In the study of functions with complex numbers, Taylor series are generalized to represent analytic functions, forming one ofthe cornerstones of the field.
            \item \textbf{Physics and Engineering:} In physics, complex equations are often simplified or "linearized" by taking the first one or two terms of a Taylor expansion. The famous small-angle approximation $\sin(\theta) \approx \theta$ is just the first term of the sine function's Maclaurin series. This is used everywhere from optics to mechanics.
            \item \textbf{Probability and Statistics:} The "moment-generating function" of a probability distribution is directly related to the Maclaurin series of an exponential function and is used to find key properties like the mean and variance.
        \end{itemize}
\end{itemize}

\part{Real-World Application and Modeling}

\section{Concrete Scenarios in Finance and Economics}
Taylor series are not just a mathematical curiosity; they are a fundamental tool in quantitative finance for approximating complex relationships and managing risk.

\begin{enumerate}
    \item \textbf{Derivative Pricing and Risk Management (The "Greeks"):} The price of a financial option is a complex function of several variables, including the underlying stock price, time, and market volatility. The "Greeks" (Delta, Gamma, Vega, etc.) are the partial derivatives of the option's price with respect to these variables. Portfolio managers use a second-order Taylor expansion to approximate the change in an option's value without running a full, computationally expensive pricing model. This is called a Delta-Gamma approximation and is essential for daily risk management.

    \item \textbf{Stochastic Calculus (Itô's Lemma):} The modeling of stock prices over time involves randomness, which is the domain of stochastic calculus. The foundational formula in this field, Itô's Lemma, describes how a function of a random process changes. It can be intuitively understood as a Taylor series expansion that, because of the nature of randomness, must retain the second-order term. This lemma is the bedrock upon which models like the famous Black-Scholes option pricing model are built.

    \item \textbf{Bond Pricing and Duration/Convexity:} The price of a bond is a non-linear function of interest rates. Financial analysts use a Taylor series expansion to approximate this relationship. The first derivative of price with respect to yield is related to a concept called "duration," and the second derivative is related to "convexity." A bond portfolio's sensitivity to interest rate changes is modeled as: $\Delta \text{Price} \approx (-\text{Duration}) \cdot \Delta \text{Yield} + \frac{1}{2} \text{Convexity} \cdot (\Delta \text{Yield})^2$. This is a direct application of a second-order Taylor polynomial.
\end{enumerate}

\section{Model Problem Setup: Delta-Gamma Approximation}
Let's set up a concrete model for the first scenario.

\begin{itemize}
    \item \textbf{Scenario:} A quantitative analyst wants to quickly estimate the change in the price of a call option if the underlying stock price moves by a small amount.
    \item \textbf{Variables:}
        \begin{itemize}
            \item $V(S)$: The price of the option, which is a function of the stock price $S$.
            \item $S_0$: The current stock price.
            \item $\Delta S$: The projected small change in the stock price (e.g., $+\$0.50$).
        \end{itemize}
    \item \textbf{The Model (Taylor Expansion):}
        We can model the new option price, $V(S_0 + \Delta S)$, using a second-order Taylor expansion around the current price $S_0$. The general formula is:
        \[ V(S) \approx V(S_0) + \frac{V'(S_0)}{1!} (S - S_0) + \frac{V''(S_0)}{2!} (S - S_0)^2 \]
        Letting $S = S_0 + \Delta S$, we get:
        \[ V(S_0 + \Delta S) \approx V(S_0) + V'(S_0) \Delta S + \frac{1}{2} V''(S_0) (\Delta S)^2 \]

    \item \textbf{Financial Formulation:}
        In finance, the derivatives have special names:
        \begin{itemize}
            \item \textbf{Delta ($\Delta$):} The first derivative of the option price with respect to the stock price. $\Delta = V'(S_0)$.
            \item \textbf{Gamma ($\Gamma$):} The second derivative. $\Gamma = V''(S_0)$.
        \end{itemize}
        Substituting these into the model gives the formula used by traders worldwide:
        \[ \text{New Option Price} \approx \text{Current Price} + \Delta \cdot (\Delta S) + \frac{1}{2} \Gamma \cdot (\Delta S)^2 \]
        This allows for a very fast and reasonably accurate estimation of profit or loss due to small market movements, which is critical in high-speed trading environments.
\end{itemize}

\part{Common Variations and Untested Concepts}
The provided homework set covers the core mechanics of Taylor series but omits a few important related concepts.

\section{Taylor's Inequality and the Remainder}
Your homework always assumed that the series converged to the function. Taylor's Inequality provides a way to prove this by bounding the error, or \textbf{remainder}, $R_n(x) = f(x) - T_n(x)$, where $T_n(x)$ is the $n$-th degree Taylor polynomial.

\textbf{Taylor's Inequality:} If $|f^{(n+1)}(t)| \le M$ for all $t$ between $a$ and $x$, then the remainder $R_n(x)$ satisfies the inequality:
\[ |R_n(x)| \le \frac{M}{(n+1)!} |x-a|^{n+1} \]
A Taylor series converges to $f(x)$ if and only if $\lim_{n \to \infty} R_n(x) = 0$.

\textbf{Worked Example:} Prove that the Maclaurin series for $\cos(x)$ converges to $\cos(x)$ for all $x$.
\begin{enumerate}
    \item The derivatives of $f(x)=\cos(x)$ are $\pm \sin(x)$ or $\pm \cos(x)$.
    \item Therefore, for any $n$, the $(n+1)$-th derivative, $f^{(n+1)}(t)$, will be one of these four functions.
    \item In all cases, $|f^{(n+1)}(t)| \le 1$ for all $t$. So we can choose $M=1$.
    \item Applying Taylor's Inequality with $a=0$:
        \[ |R_n(x)| \le \frac{1}{(n+1)!} |x|^{n+1} \]
    \item We need to show this goes to zero as $n \to \infty$. For any fixed value of $x$, we know that $\lim_{n \to \infty} \frac{|x|^{n+1}}{(n+1)!} = 0$, because the factorial in the denominator grows much faster than the exponential in the numerator.
    \item Since the remainder goes to zero, the Maclaurin series for $\cos(x)$ converges to $\cos(x)$ for all $x$.
\end{enumerate}

\section{The Binomial Series}
The function $f(x)=(1+x)^k$, where $k$ is any real number, has a very important Maclaurin series known as the \textbf{binomial series}. Problems 4 and 5 were special cases of this where $k=-2$.

The formula is:
\[ (1+x)^k = \sum_{n=0}^{\infty} \binom{k}{n} x^n = 1 + kx + \frac{k(k-1)}{2!}x^2 + \frac{k(k-1)(k-2)}{3!}x^3 + \dots \]
This series converges for $|x|<1$. The term $\binom{k}{n}$ is the generalized binomial coefficient.

\textbf{Worked Example:} Find the first four terms of the Maclaurin series for $f(x) = \sqrt{1+x}$.
\begin{enumerate}
    \item Rewrite the function as $f(x) = (1+x)^{1/2}$. Here, $k=1/2$.
    \item Apply the binomial series formula:
        \begin{align*}
            (1+x)^{1/2} &= 1 + \frac{1}{2}x + \frac{(\frac{1}{2})(\frac{1}{2}-1)}{2!}x^2 + \frac{(\frac{1}{2})(\frac{1}{2}-1)(\frac{1}{2}-2)}{3!}x^3 + \dots \\
            &= 1 + \frac{1}{2}x + \frac{(\frac{1}{2})(-\frac{1}{2})}{2}x^2 + \frac{(\frac{1}{2})(-\frac{1}{2})(-\frac{3}{2})}{6}x^3 + \dots \\
            &= 1 + \frac{1}{2}x - \frac{1}{8}x^2 + \frac{1}{16}x^3 - \dots
        \end{align*}
\end{enumerate}


\part{Advanced Diagnostic Testing: "Find the Flaw"}
The following five problems have complete solutions that contain a single, subtle error. Your task is to find the flaw, explain it, and provide the correct solution.

\subsection{Problem 1}
\textbf{Question:} Find the Maclaurin series for $f(x) = x\cos(x)$.

\textbf{Flawed Solution:}
The Maclaurin series for $\cos(x)$ is $\sum_{n=0}^{\infty} \frac{(-1)^n x^{2n}}{(2n)!} = 1 - \frac{x^2}{2!} + \frac{x^4}{4!} - \dots$.
To find the series for $x\cos(x)$, we multiply the series for $\cos(x)$ by $x$:
\[ x \cdot \left( \sum_{n=0}^{\infty} \frac{(-1)^n x^{2n}}{(2n)!} \right) = \sum_{n=0}^{\infty} \frac{(-1)^n x^{2n+1}}{(2n)!} \]
The first few terms are:
\[ x - \frac{x^3}{2!} + \frac{x^5}{4!} - \frac{x^7}{6!} + \dots \]
\textbf{Final Answer:} $\sum_{n=0}^{\infty} \frac{(-1)^n x^{2n+1}}{(2n)!}$

---
\textbf{Find the Flaw Here:}
\begin{itemize}
    \item \textbf{The Flaw Is:} The denominator in the general term is incorrect.
    \item \textbf{Explanation:} The denominator should be $(2n)!$, matching the original cosine series, but the flawed solution incorrectly keeps it as $(2n)!$ instead of using the denominator corresponding to the new power. The correct series has denominators that match the original cosine series terms. For example, the term from $\cos(x)$ with denominator $2!$ is $x^2/2!$. Multiplying by $x$ gives $x^3/2!$. The denominator remains $2!$. The general form is correct. *Correction*: My initial analysis was slightly off. The sigma notation is correct as written. The error is more subtle. Let's re-examine the expansion.
    $\cos(x) = \frac{(-1)^0 x^0}{0!} + \frac{(-1)^1 x^2}{2!} + \frac{(-1)^2 x^4}{4!} + \dots$
    $x\cos(x) = x\left(\frac{x^0}{0!} - \frac{x^2}{2!} + \frac{x^4}{4!} - \dots\right) = \frac{x^1}{0!} - \frac{x^3}{2!} + \frac{x^5}{4!} - \dots$
    The flawed solution has $\sum \frac{(-1)^n x^{2n+1}}{(2n)!}$.
    For n=0: $\frac{(-1)^0 x^1}{0!} = x$. Correct.
    For n=1: $\frac{(-1)^1 x^3}{2!} = -x^3/2!$. Correct.
    For n=2: $\frac{(-1)^2 x^5}{4!} = x^5/4!$. Correct.
    Ah, the solution provided is actually correct. Let me create a new flawed problem.

\textbf{New Problem 1}
\textbf{Question:} Find the Maclaurin series for $f(x) = \cos(x^2)$.
\textbf{Flawed Solution:}
The Maclaurin series for $\cos(x)$ is $\sum_{n=0}^{\infty} \frac{(-1)^n x^{2n}}{(2n)!}$.
We can find the series for $\cos(x^2)$ by differentiating the series for $\cos(x)$ and evaluating at $x^2$.
$\frac{d}{dx} \cos(x) = -\sin(x) = \sum_{n=1}^{\infty} \frac{(-1)^n (2n) x^{2n-1}}{(2n)!} = \sum_{n=1}^{\infty} \frac{(-1)^n x^{2n-1}}{(2n-1)!}$.
Substituting $x^2$ gives: $\sum_{n=1}^{\infty} \frac{(-1)^n (x^2)^{2n-1}}{(2n-1)!} = \sum_{n=1}^{\infty} \frac{(-1)^n x^{4n-2}}{(2n-1)!}$.
\textbf{Final Answer:} $\sum_{n=1}^{\infty} \frac{(-1)^n x^{4n-2}}{(2n-1)!}$
---
\textbf{Find the Flaw Here:}
\begin{itemize}
    \item \textbf{The Flaw Is:} The method used is conceptually incorrect; one should use substitution, not differentiation and substitution.
    \item \textbf{Explanation:} To find the series for a composite function like $\cos(x^2)$, you substitute the inner function ($u=x^2$) directly into the known series for the outer function ($\cos(u)$).
    \item \textbf{Correct Step and Solution:}
        Take the series for $\cos(u) = \sum_{n=0}^{\infty} \frac{(-1)^n u^{2n}}{(2n)!}$.
        Substitute $u=x^2$:
        \[ \cos(x^2) = \sum_{n=0}^{\infty} \frac{(-1)^n (x^2)^{2n}}{(2n)!} = \sum_{n=0}^{\infty} \frac{(-1)^n x^{4n}}{(2n)!} \]
        \textbf{Correct Final Answer:} $\sum_{n=0}^{\infty} \frac{(-1)^n x^{4n}}{(2n)!}$
\end{itemize}

\subsection{Problem 2}
\textbf{Question:} Find the Taylor series for $f(x) = \frac{1}{x}$ centered at $a=2$.

\textbf{Flawed Solution:}
We find the derivatives of $f(x) = x^{-1}$.
$f(x) = x^{-1} \implies f(2) = 1/2$
$f'(x) = -x^{-2} \implies f'(2) = -1/4$
$f''(x) = 2x^{-3} \implies f''(2) = 2/8 = 1/4$
$f'''(x) = 6x^{-4} \implies f'''(2) = 6/16 = 3/8$
The pattern for the $n$-th derivative is $f^{(n)}(x) = \frac{n!}{x^{n+1}}$.
So, $f^{(n)}(2) = \frac{n!}{2^{n+1}}$.
The Taylor series is $\sum_{n=0}^{\infty} \frac{f^{(n)}(2)}{n!}(x-2)^n = \sum_{n=0}^{\infty} \frac{n!/2^{n+1}}{n!}(x-2)^n = \sum_{n=0}^{\infty} \frac{(x-2)^n}{2^{n+1}}$.
\textbf{Final Answer:} $\sum_{n=0}^{\infty} \frac{(x-2)^n}{2^{n+1}}$

---
\textbf{Find the Flaw Here:}
\begin{itemize}
    \item \textbf{The Flaw Is:} The pattern identified for the $n$-th derivative is missing the alternating sign.
    \item \textbf{Explanation:} The derivatives are $f' = -x^{-2}$, $f'' = +2x^{-3}$, $f''' = -6x^{-4}$. The sign alternates. The correct pattern for the $n$-th derivative is $f^{(n)}(x) = \frac{(-1)^n n!}{x^{n+1}}$.
    \item \textbf{Correct Step and Solution:}
        The correct formula for the derivative at the center is $f^{(n)}(2) = \frac{(-1)^n n!}{2^{n+1}}$.
        Substituting this into the Taylor series formula:
        \[ \sum_{n=0}^{\infty} \frac{\frac{(-1)^n n!}{2^{n+1}}}{n!}(x-2)^n = \sum_{n=0}^{\infty} \frac{(-1)^n (x-2)^n}{2^{n+1}} \]
        \textbf{Correct Final Answer:} $\sum_{n=0}^{\infty} \frac{(-1)^n (x-2)^n}{2^{n+1}}$
\end{itemize}

\subsection{Problem 3}
\textbf{Question:} Find the radius of convergence for the series $\sum_{n=0}^{\infty} \frac{(3x-1)^n}{n^2+1}$.

\textbf{Flawed Solution:}
We use the Ratio Test.
\begin{align*}
    L &= \lim_{n \to \infty} \left| \frac{(3x-1)^{n+1}}{((n+1)^2+1)} \cdot \frac{(n^2+1)}{(3x-1)^n} \right| \\
    &= |3x-1| \lim_{n \to \infty} \frac{n^2+1}{n^2+2n+2} \\
    &= |3x-1| \cdot 1 = |3x-1|
\end{align*}
For convergence, we need $|3x-1| < 1$.
This means $-1 < 3x-1 < 1$, which simplifies to $0 < 3x < 2$, so $0 < x < 2/3$.
The interval is $(0, 2/3)$, so the radius of convergence is $R=1$.
\textbf{Final Answer:} $R=1$

---
\textbf{Find the Flaw Here:}
\begin{itemize}
    \item \textbf{The Flaw Is:} The radius of convergence was incorrectly identified from the final inequality.
    \item \textbf{Explanation:} The radius of convergence must be calculated from the standard form $|x-a| < R$. The inequality $|3x-1| < 1$ must first be rewritten as $|3(x-1/3)| < 1$, which simplifies to $|x-1/3| < 1/3$.
    \item \textbf{Correct Step and Solution:}
        Starting from the result of the ratio test, $|3x-1| < 1$:
        \[ |3(x-1/3)| < 1 \implies 3|x-1/3| < 1 \implies |x-1/3| < \frac{1}{3} \]
        This inequality is in the form $|x-a|<R$. The center is $a=1/3$ and the radius is $R=1/3$.
        \textbf{Correct Final Answer:} $R = 1/3$
\end{itemize}

\subsection{Problem 4}
\textbf{Question:} Find the first three non-zero terms of the Maclaurin series for $f(x)=e^x \ln(1+x)$.

\textbf{Flawed Solution:}
We know the series:
$e^x = 1 + x + \frac{x^2}{2} + \dots$
$\ln(1+x) = x - \frac{x^2}{2} + \frac{x^3}{3} - \dots$
To find the series for the product, we multiply the series term-by-term:
\begin{align*}
    e^x \ln(1+x) &= (1 \cdot x) + (x \cdot (-\frac{x^2}{2})) + (\frac{x^2}{2} \cdot \frac{x^3}{3}) + \dots \\
    &= x - \frac{x^3}{2} + \frac{x^5}{6} + \dots
\end{align*}
\textbf{Final Answer:} $x - \frac{x^3}{2} + \frac{x^5}{6}$

---
\textbf{Find the Flaw Here:}
\begin{itemize}
    \item \textbf{The Flaw Is:} The multiplication of the two series was done incorrectly.
    \item \textbf{Explanation:} When multiplying polynomials (or power series), every term from the first series must be multiplied by every term from the second series (like FOIL, but for infinite series). The flawed solution only multiplied terms with corresponding positions.
    \item \textbf{Correct Step and Solution:}
        We must distribute properly and collect like terms by degree.
        \[ (1 + x + \frac{x^2}{2} + \dots)(x - \frac{x^2}{2} + \frac{x^3}{3} - \dots) \]
        \textbf{x-term:} $(1)(x) = x$
        \textbf{$x^2$-term:} $(1)(-\frac{x^2}{2}) + (x)(x) = -\frac{1}{2}x^2 + x^2 = \frac{1}{2}x^2$
        \textbf{$x^3$-term:} $(1)(\frac{x^3}{3}) + (x)(-\frac{x^2}{2}) + (\frac{x^2}{2})(x) = \frac{1}{3}x^3 - \frac{1}{2}x^3 + \frac{1}{2}x^3 = \frac{1}{3}x^3$
        The first three non-zero terms are $x + \frac{1}{2}x^2 + \frac{1}{3}x^3$.
        \textbf{Correct Final Answer:} $x + \frac{1}{2}x^2 + \frac{1}{3}x^3$
\end{itemize}

\subsection{Problem 5}
\textbf{Question:} Find the Maclaurin series for $f(x) = \sin(x) + \cos(x)$.

\textbf{Flawed Solution:}
We find the derivatives at $a=0$.
$f(x) = \sin(x) + \cos(x) \implies f(0) = 1$
$f'(x) = \cos(x) - \sin(x) \implies f'(0) = 1$
$f''(x) = -\sin(x) - \cos(x) \implies f''(0) = -1$
$f'''(x) = -\cos(x) + \sin(x) \implies f'''(0) = -1$
The series is $\sum \frac{f^{(n)}(0)}{n!}x^n$:
\[ 1 + \frac{1}{1!}x - \frac{1}{2!}x^2 - \frac{1}{3!}x^3 + \dots \]
The general term is $\frac{x^n}{n!}$ with a strange sign pattern. Let's just write out the first four terms.
\textbf{Final Answer:} $1 + x - \frac{x^2}{2} - \frac{x^3}{6}$

---
\textbf{Find the Flaw Here:}
\begin{itemize}
    \item \textbf{The Flaw Is:} A subtle but critical error was made in the zeroth term of the series.
    \item \textbf{Explanation:} The zeroth coefficient is $c_0 = \frac{f^{(0)}(0)}{0!}$. The solution correctly found $f(0)=1$, but implicitly divided by $1!$ instead of $0!$. Since $0! = 1$, the value is correct, but this reveals a potential misunderstanding. Let's find a more concrete flaw. The approach is valid but cumbersome. A much simpler method exists. Let's re-evaluate the premise. The solution itself is correct. Let's create a better flawed problem.
\end{itemize}

\textbf{New Problem 5}
\textbf{Question:} Find the interval of convergence for the series for $\ln(1+x)$.
\textbf{Flawed Solution:}
The series for $\ln(1+x)$ is $\sum_{n=1}^{\infty} \frac{(-1)^{n-1}x^n}{n}$.
Using the Ratio Test, we find the radius of convergence is $R=1$. This means the series converges for $|x|<1$, or $-1<x<1$.
The interval of convergence is $(-1, 1)$.
\textbf{Final Answer:} $(-1, 1)$
---
\textbf{Find the Flaw Here:}
\begin{itemize}
    \item \textbf{The Flaw Is:} The solution failed to test the endpoints of the interval.
    \item \textbf{Explanation:} The Ratio Test is inconclusive when the limit equals 1, which occurs at the endpoints $x=-1$ and $x=1$. These points must be tested separately by plugging them back into the original series.
    \item \textbf{Correct Step and Solution:}
        \textbf{Test $x=1$:} The series becomes $\sum_{n=1}^{\infty} \frac{(-1)^{n-1}(1)^n}{n} = \sum_{n=1}^{\infty} \frac{(-1)^{n-1}}{n}$. This is the Alternating Harmonic Series, which is known to converge by the Alternating Series Test. So, $x=1$ is included.
        \textbf{Test $x=-1$:} The series becomes $\sum_{n=1}^{\infty} \frac{(-1)^{n-1}(-1)^n}{n} = \sum_{n=1}^{\infty} \frac{(-1)^{2n-1}}{n} = \sum_{n=1}^{\infty} \frac{-1}{n}$. This is $-1$ times the Harmonic Series, which is known to diverge. So, $x=-1$ is not included.
        The interval of convergence includes $x=1$ but not $x=-1$.
        \textbf{Correct Final Answer:} $(-1, 1]$
\end{itemize}

\end{document}