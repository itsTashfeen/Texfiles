\documentclass{article}
\usepackage[margin=1in]{geometry}
\usepackage{amsmath}
\usepackage{amssymb}
\usepackage{graphicx}

\title{Homework 11.4: The Comparison Tests}
\author{Tashfeen Omran}
\date{\November 2025}

\begin{document}

\maketitle

\section{Comprehensive Introduction, Context, and Prerequisites}

Welcome to this guide on the Comparison Tests for infinite series. These tests are powerful tools that allow us to determine if a series converges or diverges by comparing it to another series whose behavior is already known. This is a fundamental skill in the study of series, as finding the exact sum of a series is often impossible, but determining its convergence is both crucial and achievable.

\subsection{Core Concepts}

The central idea is to analyze an unknown series, \(\sum a_n\), by comparing its terms to the terms of a known series, \(\sum b_n\). We primarily use two well-understood types of series for comparison:
\begin{itemize}
    \item \textbf{p-series:} A series of the form \(\sum_{n=1}^{\infty} \frac{1}{n^p}\). It converges if \(p > 1\) and diverges if \(p \leq 1\).
    \item \textbf{Geometric Series:} A series of the form \(\sum_{n=1}^{\infty} ar^{n-1}\). It converges if \(|r| < 1\) and diverges if \(|r| \geq 1\).
\end{itemize}

There are two main comparison tests:

\subsubsection{The Direct Comparison Test (DCT)}
This test is based on a simple, intuitive idea:
\begin{itemize}
    \item If the terms of your series \(a_n\) are always smaller than the terms of a known \textit{convergent} series \(b_n\), then your series cannot possibly sum to infinity. It is "squeezed" from above and must also converge.
    \item If the terms of your series \(a_n\) are always larger than the terms of a known \textit{divergent} series \(b_n\), then your series must be "pushed" to infinity and must also diverge.
\end{itemize}
The key is setting up the correct inequality. For the tests to apply, all terms \(a_n\) and \(b_n\) must be positive.

\subsubsection{The Limit Comparison Test (LCT)}
The LCT is a more powerful and often easier-to-use test, especially when setting up a direct inequality is difficult. It formalizes the idea of two series "behaving similarly" in the long run.
\begin{itemize}
    \item We compute the limit of the ratio of the terms: \( c = \lim_{n \to \infty} \frac{a_n}{b_n} \).
    \item If \(c\) is a finite, positive number (\(0 < c < \infty\)), it means that for large \(n\), \(a_n \approx c \cdot b_n\). Since the series \(\sum c \cdot b_n\) has the same convergence behavior as \(\sum b_n\), we can conclude that \(\sum a_n\) and \(\sum b_n\) do the same thing: either they both converge or they both diverge.
\end{itemize}
This test is particularly effective for series involving rational functions of \(n\), where we can choose \(b_n\) by looking at the ratio of the highest-power (dominant) terms.

\subsection{Historical Context and Motivation}
The rigorous study of infinite series began in the 19th century with mathematicians like Augustin-Louis Cauchy. Before this, mathematicians like Newton and Euler used series extensively but often with a more intuitive, less formal approach. This sometimes led to paradoxical results. For example, the series \(1 - 1 + 1 - 1 + \dots\) was assigned different values depending on how it was grouped.

The core motivation for developing convergence tests was to put the theory of infinite series on a solid logical foundation. Physicists and astronomers were increasingly using series, such as Fourier series, to model phenomena like heat flow and planetary orbits. It became critical to know which of these infinite sums represented a real, finite quantity and which were mathematical nonsense. The Comparison Tests were among the first and most important tools developed by Cauchy and others to bring this necessary rigor, allowing mathematicians to confidently determine the validity of a series without needing to calculate its exact sum.

\subsection{Key Formulas}
For all tests, assume \(\sum a_n\) and \(\sum b_n\) are series with positive terms.

\subsubsection{The Direct Comparison Test (DCT)}
\begin{enumerate}
    \item If \(a_n \leq b_n\) for all \(n\) (or for all \(n \geq N\)) and \(\sum b_n\) converges, then \(\sum a_n\) converges.
    \item If \(a_n \geq b_n\) for all \(n\) (or for all \(n \geq N\)) and \(\sum b_n\) diverges, then \(\sum a_n\) diverges.
\end{enumerate}

\subsubsection{The Limit Comparison Test (LCT)}
Let \( c = \lim_{n \to \infty} \frac{a_n}{b_n} \).
\begin{enumerate}
    \item If \(c\) is a finite positive number (\(0 < c < \infty\)), then \(\sum a_n\) and \(\sum b_n\) either both converge or both diverge.
    \item \textbf{Special Case 1:} If \(c = 0\) and \(\sum b_n\) converges, then \(\sum a_n\) converges. (This means \(a_n\) is significantly smaller than \(b_n\)).
    \item \textbf{Special Case 2:} If \(c = \infty\) and \(\sum b_n\) diverges, then \(\sum a_n\) diverges. (This means \(a_n\) is significantly larger than \(b_n\)).
\end{enumerate}

\subsection{Prerequisites}
\begin{itemize}
    \item \textbf{Sequences and Series:} A solid understanding of the definitions of sequences and series, and the concept of convergence (the limit of partial sums).
    \item \textbf{Standard Series:} You must know the convergence rules for p-series and geometric series by heart. These are your primary tools for comparison.
    \item \textbf{Limits at Infinity:} Proficiency in calculating limits of sequences as \(n \to \infty\), especially for rational functions. The technique of dividing by the highest power of \(n\) is essential.
    \item \textbf{Algebraic Manipulation:} Skill in manipulating inequalities and simplifying complex fractions.
    \item \textbf{Function Growth Rates:} An intuitive understanding of which functions grow faster (e.g., exponentials > polynomials > logarithms). This helps in choosing the right comparison series.
\end{itemize}

\section{Detailed Homework Solutions}

\subsection{Problems 1-6}

\subsubsection*{Problem 1}
\textbf{(a)} If \(a_n > b_n\) and \(\sum b_n\) converges, you can say nothing about \(\sum a_n\). It might converge or diverge. The DCT is inconclusive.
\textbf{(b)} If \(a_n < b_n\) and \(\sum b_n\) converges, then \(\sum a_n\) must also converge by the Direct Comparison Test.

\subsubsection*{Problem 2}
\textbf{(a)} If \(a_n > b_n\) and \(\sum b_n\) diverges, then \(\sum a_n\) must also diverge by the Direct Comparison Test.
\textbf{(b)} If \(a_n < b_n\) and \(\sum b_n\) diverges, you can say nothing about \(\sum a_n\). It might converge or diverge. The DCT is inconclusive.

\subsubsection*{Problem 3}
\textbf{(a) DCT:} The problem in the text has a typo, it compares \(\sum \frac{n}{n^3-5}\) to \(\sum \frac{1}{n^3+5}\) and \(\sum \frac{1}{n^2}\). Let's show convergence of \(\sum_{n=2}^{\infty} \frac{1}{n^3+5}\) by comparing to \(\sum_{n=2}^{\infty} \frac{1}{n^3}\). Since \(n^3+5 > n^3\), we have \(\frac{1}{n^3+5} < \frac{1}{n^3}\). \(\sum \frac{1}{n^3}\) is a convergent p-series (\(p=3>1\)). Therefore, \(\sum \frac{1}{n^3+5}\) converges by DCT.
\textbf{(b) LCT:} Let's test \(\sum_{n=2}^{\infty} \frac{n}{n^3-5}\) by comparing to \(\sum_{n=2}^{\infty} \frac{1}{n^2}\). Let \(a_n = \frac{n}{n^3-5}\) and \(b_n = \frac{1}{n^2}\).
\[ \lim_{n \to \infty} \frac{a_n}{b_n} = \lim_{n \to \infty} \frac{n/(n^3-5)}{1/n^2} = \lim_{n \to \infty} \frac{n^3}{n^3-5} = 1 \]
Since the limit is 1 (\(0<1<\infty\)) and \(\sum \frac{1}{n^2}\) converges, \(\sum \frac{n}{n^3-5}\) converges by LCT.

\subsubsection*{Problem 4}
\textbf{(a) DCT:} To show \(\sum_{n=2}^{\infty} \frac{n^2+n}{n^3-2}\) diverges by comparing with \(\sum_{n=2}^{\infty} \frac{1}{n}\). We need \(\frac{n^2+n}{n^3-2} \geq \frac{1}{n}\). This means \(n(n^2+n) \geq n^3-2 \implies n^3+n^2 \geq n^3-2 \implies n^2 \geq -2\), which is true for all \(n\). Since \(\sum \frac{1}{n}\) diverges (harmonic series), \(\sum \frac{n^2+n}{n^3-2}\) diverges by DCT.
\textbf{(b) LCT:} Test \(\sum_{n=2}^{\infty} \frac{n^2-n}{n^3+2}\) by comparing with \(\sum_{n=2}^{\infty} \frac{1}{n}\). Let \(a_n = \frac{n^2-n}{n^3+2}\) and \(b_n = \frac{1}{n}\).
\[ \lim_{n \to \infty} \frac{a_n}{b_n} = \lim_{n \to \infty} \frac{(n^2-n)/(n^3+2)}{1/n} = \lim_{n \to \infty} \frac{n(n^2-n)}{n^3+2} = \lim_{n \to \infty} \frac{n^3-n^2}{n^3+2} = 1 \]
Since the limit is 1 and \(\sum \frac{1}{n}\) diverges, the series diverges by LCT.

\subsubsection*{Problem 5}
To show \(\sum_{n=1}^{\infty} \frac{n}{n^3+1}\) converges. We compare to \(\sum \frac{1}{n^2}\).
\textbf{(c)} is the correct inequality for DCT. For \(n \ge 1\), \(n^3+1 > n^3\), so \(\frac{n}{n^3+1} < \frac{n}{n^3} = \frac{1}{n^2}\). Since \(\sum \frac{1}{n^2}\) converges, our series converges. (a) and (b) are incorrect or inconclusive.

\subsubsection*{Problem 6}
To show \(\sum_{n=1}^{\infty} \frac{n}{n^2+1}\) diverges. We compare to \(\sum \frac{1}{n}\) or a multiple of it.
\textbf{(c)} is the correct inequality for DCT. We have \(n^2+1 < 2n^2\) for \(n>1\). Thus, \(\frac{n}{n^2+1} > \frac{n}{2n^2} = \frac{1}{2n}\). Since \(\sum \frac{1}{2n}\) diverges (multiple of harmonic series), our series diverges. (a) is inconclusive. (b) is false.

\subsection{Problems 7-40: Convergence or Divergence}
\textbf{7. \(\sum_{n=1}^{\infty} \frac{1}{n^3+8}\):} Compare with \(\sum \frac{1}{n^3}\). Since \(\frac{1}{n^3+8} < \frac{1}{n^3}\) and \(\sum \frac{1}{n^3}\) is a convergent p-series (\(p=3>1\)), the series \textbf{converges} by DCT.

\textbf{8. \(\sum_{n=2}^{\infty} \frac{1}{\sqrt{n}-1}\):} Compare with \(\sum \frac{1}{\sqrt{n}}\). Since \(\sqrt{n}-1 < \sqrt{n}\), we have \(\frac{1}{\sqrt{n}-1} > \frac{1}{\sqrt{n}}\). \(\sum \frac{1}{\sqrt{n}}\) is a divergent p-series (\(p=1/2 \le 1\)). The series \textbf{diverges} by DCT.

\textbf{9. \(\sum_{n=1}^{\infty} \frac{n+1}{n\sqrt{n}}\):} Simplify to \(\sum (\frac{1}{\sqrt{n}} + \frac{1}{n^{3/2}})\). This is the sum of a divergent p-series (\(p=1/2\)) and a convergent p-series (\(p=3/2\)). The sum of a divergent and convergent series is divergent. The series \textbf{diverges}.

\textbf{10. \(\sum_{n=1}^{\infty} \frac{n-1}{n^3+1}\):} Use LCT with \(b_n = \frac{n}{n^3} = \frac{1}{n^2}\). \(\lim_{n \to \infty} \frac{(n-1)/(n^3+1)}{1/n^2} = \lim_{n \to \infty} \frac{n^2(n-1)}{n^3+1} = 1\). Since \(\sum \frac{1}{n^2}\) converges, the series \textbf{converges}.

\textbf{11. \(\sum_{n=1}^{\infty} \frac{9^n}{3+10^n}\):} Use LCT with \(b_n = (\frac{9}{10})^n\). \(\lim_{n \to \infty} \frac{9^n/(3+10^n)}{(9/10)^n} = \lim_{n \to \infty} \frac{10^n}{3+10^n} = 1\). Since \(\sum (\frac{9}{10})^n\) is a convergent geometric series, the series \textbf{converges}.

\textbf{12. \(\sum_{n=1}^{\infty} \frac{6^n}{5^n-1}\):} Use LCT with \(b_n = (\frac{6}{5})^n\). \(\lim_{n \to \infty} \frac{6^n/(5^n-1)}{(6/5)^n} = \lim_{n \to \infty} \frac{5^n}{5^n-1} = 1\). Since \(\sum (\frac{6}{5})^n\) is a divergent geometric series, the series \textbf{diverges}.

\textbf{13. \(\sum_{n=2}^{\infty} \frac{1}{\ln n}\):} For \(n>1\), \(\ln n < n\), so \(\frac{1}{\ln n} > \frac{1}{n}\). Since \(\sum \frac{1}{n}\) diverges, the series \textbf{diverges} by DCT.

\textbf{14. \(\sum_{k=1}^{\infty} \frac{k \sin^2 k}{1+k^3}\):} Since \(0 \le \sin^2 k \le 1\), we have \(\frac{k \sin^2 k}{1+k^3} \le \frac{k}{1+k^3}\). For large k, \(\frac{k}{1+k^3} \approx \frac{1}{k^2}\). \(\sum \frac{1}{k^2}\) converges. The series \textbf{converges} by DCT.

\textbf{15. \(\sum_{k=1}^{\infty} \frac{\sqrt{k}}{k^3+4k+3}\):} Use LCT with \(b_k = \frac{\sqrt{k}}{k^3} = \frac{1}{k^{2.5}}\). \(\sum \frac{1}{k^{2.5}}\) is a convergent p-series (\(p=2.5>1\)). The limit of the ratio is 1. The series \textbf{converges}.

\textbf{16. \(\sum_{k=1}^{\infty} \frac{(2k-1)(k^2-1)}{(k+1)(k^2+4)^2}\):} The degree of the numerator is 3. The degree of the denominator is 5. So the terms behave like \(k^3/k^5 = 1/k^2\). Use LCT with \(b_k = \frac{1}{k^2}\). \(\sum \frac{1}{k^2}\) converges, so the series \textbf{converges}.

\textbf{17. \(\sum_{n=1}^{\infty} \frac{1+\cos n}{e^n}\):} Since \(-1 \le \cos n \le 1\), we have \(0 \le 1+\cos n \le 2\). Thus, \(0 \le \frac{1+\cos n}{e^n} \le \frac{2}{e^n}\). \(\sum \frac{2}{e^n}\) is a convergent geometric series (\(r=1/e < 1\)). The series \textbf{converges} by DCT.

\textbf{18. \(\sum_{n=1}^{\infty} \frac{1}{\sqrt[3]{3n^4+1}}\):} Use LCT with \(b_n = \frac{1}{\sqrt[3]{n^4}} = \frac{1}{n^{4/3}}\). \(\sum \frac{1}{n^{4/3}}\) is a convergent p-series (\(p=4/3>1\)). The limit of the ratio is \(1/\sqrt[3]{3}\). The series \textbf{converges}.

\textbf{19. \(\sum_{n=1}^{\infty} \frac{4^{n+1}}{3^n-2}\):} Use LCT with \(b_n = \frac{4^n}{3^n} = (\frac{4}{3})^n\). \(\lim_{n \to \infty} \frac{4 \cdot 4^n/(3^n-2)}{(4/3)^n} = \lim_{n \to \infty} \frac{4 \cdot 3^n}{3^n-2} = 4\). Since \(\sum (\frac{4}{3})^n\) is a divergent geometric series, the series \textbf{diverges}.

\textbf{20. \(\sum_{n=1}^{\infty} \frac{1}{n^n}\):} For \(n \ge 2\), \(n^n \ge n^2\), so \(\frac{1}{n^n} \le \frac{1}{n^2}\). Since \(\sum \frac{1}{n^2}\) converges, the series \textbf{converges} by DCT.

\textbf{21. \(\sum_{n=1}^{\infty} \frac{1}{\sqrt{n^2+1}}\):} Use LCT with \(b_n = \frac{1}{\sqrt{n^2}} = \frac{1}{n}\). \(\lim_{n \to \infty} \frac{1/\sqrt{n^2+1}}{1/n} = \lim_{n \to \infty} \frac{n}{\sqrt{n^2+1}}=1\). Since \(\sum \frac{1}{n}\) diverges, the series \textbf{diverges}.

\textbf{22. \(\sum_{n=1}^{\infty} \frac{2}{\sqrt{n}+2}\):} Compare with \(\sum \frac{1}{\sqrt{n}}\). Use LCT. \(\lim_{n \to \infty} \frac{2/(\sqrt{n}+2)}{1/\sqrt{n}} = \lim_{n \to \infty} \frac{2\sqrt{n}}{\sqrt{n}+2} = 2\). Since \(\sum \frac{1}{\sqrt{n}}\) diverges (\(p=1/2 \le 1\)), the series \textbf{diverges}.

\textbf{23. \(\sum_{n=1}^{\infty} \frac{n+1}{n^3+n}\):} Use LCT with \(b_n = \frac{n}{n^3} = \frac{1}{n^2}\). The limit of the ratio is 1. Since \(\sum \frac{1}{n^2}\) converges, the series \textbf{converges}.

\textbf{24. \(\sum_{n=1}^{\infty} \frac{n^2+n+1}{n^4+n^2}\):} Use LCT with \(b_n = \frac{n^2}{n^4} = \frac{1}{n^2}\). The limit of the ratio is 1. Since \(\sum \frac{1}{n^2}\) converges, the series \textbf{converges}.

\textbf{25. \(\sum_{n=1}^{\infty} \frac{\sqrt{1+n}}{2+n}\):} Use LCT with \(b_n = \frac{\sqrt{n}}{n} = \frac{1}{\sqrt{n}}\). The limit is 1. Since \(\sum \frac{1}{\sqrt{n}}\) diverges, the series \textbf{diverges}.

\textbf{26. \(\sum_{n=3}^{\infty} \frac{n+2}{(n+1)^3}\):} Use LCT with \(b_n = \frac{n}{n^3} = \frac{1}{n^2}\). The limit is 1. Since \(\sum \frac{1}{n^2}\) converges, the series \textbf{converges}.

\textbf{27. \(\sum_{n=1}^{\infty} \frac{5+2n}{(1+n^2)^2}\):} Use LCT with \(b_n = \frac{n}{n^4} = \frac{1}{n^3}\). The limit is 2. Since \(\sum \frac{1}{n^3}\) converges, the series \textbf{converges}.

\textbf{28. \(\sum_{n=1}^{\infty} \frac{n+3^n}{n+2^n}\):} For large \(n\), the terms behave like \(\frac{3^n}{2^n} = (\frac{3}{2})^n\). The limit of the terms is \(\infty\). By the Test for Divergence, the series \textbf{diverges}.

\textbf{29. \(\sum_{n=1}^{\infty} \frac{e^n+1}{ne^n+1}\):} Use LCT with \(b_n = \frac{1}{n}\). \(\lim_{n \to \infty} \frac{(e^n+1)/(ne^n+1)}{1/n} = \lim_{n \to \infty} \frac{n(e^n+1)}{ne^n+1} = \lim_{n \to \infty} \frac{ne^n+n}{ne^n+1} = 1\). Since \(\sum \frac{1}{n}\) diverges, the series \textbf{diverges}.

\textbf{30. \(\sum_{n=2}^{\infty} \frac{1}{n\sqrt{n^2-1}}\):} Use LCT with \(b_n = \frac{1}{n\sqrt{n^2}} = \frac{1}{n^2}\). The limit is 1. Since \(\sum \frac{1}{n^2}\) converges, the series \textbf{converges}.

\textbf{31. \(\sum_{n=1}^{\infty} \frac{2+\sin n}{n^2}\):} \(1 \le 2+\sin n \le 3\). So \(\frac{1}{n^2} \le \frac{2+\sin n}{n^2} \le \frac{3}{n^2}\). Since \(\sum \frac{3}{n^2}\) converges, the series \textbf{converges} by DCT.

\textbf{32. \(\sum_{n=1}^{\infty} \frac{n^2+\cos^2 n}{n^3}\):} Split into \(\sum (\frac{1}{n} + \frac{\cos^2 n}{n^3})\). \(\sum \frac{1}{n}\) diverges. \(\sum \frac{\cos^2 n}{n^3}\) converges by DCT with \(\sum \frac{1}{n^3}\). The sum of a divergent and convergent series is divergent. The series \textbf{diverges}.

\textbf{33. \(\sum_{n=1}^{\infty} (1+\frac{1}{n})^2 e^{-n}\):} \(\lim_{n \to \infty} (1+\frac{1}{n})^2 = 1\). The terms behave like \(e^{-n} = (\frac{1}{e})^n\). Use LCT with \(b_n = (\frac{1}{e})^n\). The limit is 1. \(\sum b_n\) is a convergent geometric series. The series \textbf{converges}.

\textbf{34. \(\sum_{n=1}^{\infty} \frac{e^{1/n}}{n}\):} Use LCT with \(b_n = \frac{1}{n}\). \(\lim_{n \to \infty} \frac{e^{1/n}/n}{1/n} = \lim_{n \to \infty} e^{1/n} = e^0 = 1\). Since \(\sum \frac{1}{n}\) diverges, the series \textbf{diverges}.

\textbf{35. \(\sum_{n=1}^{\infty} \frac{1}{n!}\):} For \(n \ge 4\), \(n! > n^2\). So \(\frac{1}{n!} < \frac{1}{n^2}\). Since \(\sum \frac{1}{n^2}\) converges, the series \textbf{converges} by DCT.

\textbf{36. \(\sum_{n=1}^{\infty} \frac{n!}{n^n}\):} For \(n \ge 1\), \(n^n = n \cdot n \cdot \dots \cdot n \ge n \cdot (n-1) \cdot \dots \cdot 1 \cdot 1 = n!\). But this inequality is wrong. A better one is \(n^n \ge 2^n\) for \(n \ge 2\). The Ratio Test is better, but with comparison: \(\frac{n!}{n^n} = \frac{1 \cdot 2 \cdot \dots \cdot n}{n \cdot n \cdot \dots \cdot n} = \frac{1}{n} \frac{2}{n} \dots \frac{n}{n} \le \frac{1}{n} \frac{2}{n} = \frac{2}{n^2}\). Since \(\sum \frac{2}{n^2}\) converges, the series \textbf{converges} by DCT.

\textbf{37. \(\sum_{n=1}^{\infty} \sin(1/n)\):} Use LCT with \(b_n = \frac{1}{n}\). Since \(\lim_{x \to 0} \frac{\sin x}{x} = 1\), we have \(\lim_{n \to \infty} \frac{\sin(1/n)}{1/n} = 1\). Since \(\sum \frac{1}{n}\) diverges, the series \textbf{diverges}.

\textbf{38. \(\sum_{n=1}^{\infty} \sin^2(1/n)\):} Use LCT with \(b_n = \frac{1}{n^2}\). \(\lim_{n \to \infty} \frac{\sin^2(1/n)}{(1/n)^2} = (\lim_{n \to \infty} \frac{\sin(1/n)}{1/n})^2 = 1^2 = 1\). Since \(\sum \frac{1}{n^2}\) converges, the series \textbf{converges}.

\textbf{39. \(\sum_{n=1}^{\infty} \frac{1}{n} \tan(\frac{1}{n})\):} Use LCT with \(b_n = \frac{1}{n^2}\). Since \(\lim_{x \to 0} \frac{\tan x}{x} = 1\), \(\lim_{n \to \infty} \frac{\tan(1/n)}{1/n} = 1\). The limit of the ratios is \(\lim_{n \to \infty} \frac{(1/n)\tan(1/n)}{1/n^2} = \lim_{n \to \infty} \frac{\tan(1/n)}{1/n} = 1\). Since \(\sum \frac{1}{n^2}\) converges, the series \textbf{converges}.

\textbf{40. \(\sum_{n=1}^{\infty} \frac{1}{n^{1+1/n}}\):} Use LCT with \(b_n = \frac{1}{n}\). \(\lim_{n \to \infty} \frac{1/n^{1+1/n}}{1/n} = \lim_{n \to \infty} \frac{n}{n \cdot n^{1/n}} = \lim_{n \to \infty} \frac{1}{n^{1/n}} = \frac{1}{1} = 1\), since \(\lim n^{1/n}=1\). Since \(\sum \frac{1}{n}\) diverges, the series \textbf{diverges}.

\subsection{Problems 41-44: Approximations and Error Estimates}
The error \(R_N\) is bounded by the remainder \(T_N\) of the comparison series. \(R_N \le T_N \le \int_N^\infty f(x)dx\).

\textbf{41. \(\sum_{n=1}^{\infty} \frac{1}{5+n^5}\):} We compare with \(\sum \frac{1}{n^5}\). Error \(R_{10} \le \int_{10}^\infty \frac{1}{x^5} dx = [-\frac{1}{4x^4}]_{10}^\infty = 0 - (-\frac{1}{4(10)^4}) = \frac{1}{40000} = 0.000025\).

\textbf{42. \(\sum_{n=1}^{\infty} \frac{e^{1/n}}{n^4}\):} We compare with \(\sum \frac{C}{n^4}\) for some constant C. Since \(\lim_{n \to \infty} e^{1/n} = 1\), for large n, \(e^{1/n} \approx 1\). Let's use a looser bound. For \(n \ge 1\), \(1/n \le 1\), so \(e^{1/n} \le e^1 = e\). Thus \(\frac{e^{1/n}}{n^4} \le \frac{e}{n^4}\). Error \(R_{10} \le \int_{10}^\infty \frac{e}{x^4} dx = e[-\frac{1}{3x^3}]_{10}^\infty = \frac{e}{3(10)^3} = \frac{e}{3000} \approx 0.0009\).

\textbf{43. \(\sum_{n=1}^{\infty} 5^{-n} \cos^2 n\):} \(a_n = \frac{\cos^2 n}{5^n} \le \frac{1}{5^n}\). We compare with the geometric series \(\sum \frac{1}{5^n}\). The remainder \(T_{10}\) is the sum of a geometric series starting at \(n=11\): \(T_{10} = \sum_{n=11}^\infty (\frac{1}{5})^n = \frac{(1/5)^{11}}{1-1/5} = \frac{(1/5)^{11}}{4/5} = \frac{1}{4 \cdot 5^{10}} \approx 2.56 \times 10^{-8}\).

\textbf{44. \(\sum_{n=1}^{\infty} \frac{1}{3^n+4^n}\):} Compare with \(\sum \frac{1}{4^n}\) since \(3^n+4^n > 4^n\). The remainder \(R_{10} \le T_{10} = \sum_{n=11}^\infty (\frac{1}{4})^n = \frac{(1/4)^{11}}{1-1/4} = \frac{(1/4)^{11}}{3/4} = \frac{1}{3 \cdot 4^{10}} \approx 3.17 \times 10^{-7}\).

\subsection{Problems 45-55: Theoretical and Advanced Problems}

\textbf{45. \(0.d_1 d_2 d_3 \dots = \sum_{n=1}^\infty \frac{d_n}{10^n}\):} Since \(d_n\) is a digit, \(0 \le d_n \le 9\). So we have the inequality \(\frac{d_n}{10^n} \le \frac{9}{10^n}\). We can compare the series to \(\sum_{n=1}^\infty \frac{9}{10^n} = 9 \sum_{n=1}^\infty (\frac{1}{10})^n\). This is a constant times a convergent geometric series (\(r=1/10 < 1\)). Therefore, the series for any decimal representation converges by DCT.

\textbf{46. \(\sum_{n=2}^{\infty} \frac{1}{n^p \ln n}\):}
\begin{itemize}
    \item If \(p > 1\): For \(n \ge 3\), \(\ln n > 1\), so \(n^p \ln n > n^p\). Thus \(\frac{1}{n^p \ln n} < \frac{1}{n^p}\). Since \(\sum \frac{1}{n^p}\) converges for \(p>1\), the series converges by DCT.
    \item If \(p = 1\): The series is \(\sum \frac{1}{n \ln n}\). We use the Integral Test. \(\int_2^\infty \frac{dx}{x \ln x}\). Let \(u=\ln x, du=dx/x\). \(\int_{\ln 2}^\infty \frac{du}{u} = [\ln u]_{\ln 2}^\infty\), which diverges.
    \item If \(p < 1\): For any \(n\), \(n^p < n\). Thus \(n^p \ln n < n \ln n\), which implies \(\frac{1}{n^p \ln n} > \frac{1}{n \ln n}\). Since \(\sum \frac{1}{n \ln n}\) diverges, the series diverges by DCT.
\end{itemize}
Conclusion: The series converges for \(p>1\) and diverges for \(p \le 1\).

\textbf{47. \(\sum_{n=2}^{\infty} \frac{1}{(\ln n)^{\ln n}}\):} Use the identity \(x^y = e^{y \ln x}\).
\((\ln n)^{\ln n} = e^{\ln n \cdot \ln(\ln n)} = (e^{\ln n})^{\ln(\ln n)} = n^{\ln(\ln n)}\).
The series is \(\sum_{n=2}^{\infty} \frac{1}{n^{\ln(\ln n)}}\).
As \(n \to \infty\), \(\ln(\ln n) \to \infty\). So, for a large enough integer \(N\), if \(n \ge N\), then \(\ln(\ln n) > 2\).
For \(n \ge N\), we have \(n^{\ln(\ln n)} > n^2\), which means \(\frac{1}{n^{\ln(\ln n)}} < \frac{1}{n^2}\).
Since \(\sum \frac{1}{n^2}\) is a convergent p-series, the series \(\sum \frac{1}{(\ln n)^{\ln n}}\) \textbf{converges} by the Direct Comparison Test.

\textbf{48. Proof (LCT Case \(c=0\)):}
\textbf{(a)} Given \(\lim_{n \to \infty} \frac{a_n}{b_n} = 0\) and \(\sum b_n\) converges. By the definition of a limit, for any \(\epsilon > 0\), there exists an integer \(N\) such that for \(n>N\), \(|\frac{a_n}{b_n} - 0| < \epsilon\). Let's choose \(\epsilon = 1\). Then for \(n > N\), \(\frac{a_n}{b_n} < 1\), which implies \(a_n < b_n\) (since terms are positive). Since \(\sum b_n\) converges, the series \(\sum_{n=N+1}^\infty b_n\) also converges. By DCT, since \(a_n < b_n\) for \(n>N\), the series \(\sum_{n=N+1}^\infty a_n\) converges. Since convergence is not affected by a finite number of terms, \(\sum a_n\) converges.
\textbf{(b) (i) \(\sum \frac{\ln n}{n^3}\):} Compare with \(b_n = \frac{1}{n^2}\) (convergent p-series). \(\lim_{n \to \infty} \frac{(\ln n)/n^3}{1/n^2} = \lim_{n \to \infty} \frac{\ln n}{n} = 0\). By part (a), the series converges.
\textbf{(ii) \(\sum (1-\cos(1/n))\):} Compare with \(b_n = \frac{1}{n^2}\) (convergent p-series). Use L'Hôpital's Rule on \(\lim_{x \to 0} \frac{1-\cos x}{x^2} = \lim_{x \to 0} \frac{\sin x}{2x} = \frac{1}{2}\). So the limit of the ratios is \(1/2\). This is a standard LCT case, not the \(c=0\) case. But it converges.

\textbf{49. Proof (LCT Case \(c=\infty\)):}
\textbf{(a)} Given \(\lim_{n \to \infty} \frac{a_n}{b_n} = \infty\) and \(\sum b_n\) diverges. By the definition of an infinite limit, for any \(M > 0\), there exists an integer \(N\) such that for \(n>N\), \(\frac{a_n}{b_n} > M\). Let's choose \(M=1\). Then for \(n>N\), \(\frac{a_n}{b_n} > 1\), which implies \(a_n > b_n\). Since \(\sum b_n\) diverges, \(\sum_{n=N+1}^\infty b_n\) also diverges. By DCT, since \(a_n > b_n\) for \(n>N\), the series \(\sum_{n=N+1}^\infty a_n\) diverges. Therefore, \(\sum a_n\) diverges.
\textbf{(b) (i) \(\sum \frac{1}{\ln n}\):} Compare with \(b_n = \frac{1}{n}\) (divergent). \(\lim_{n \to \infty} \frac{1/\ln n}{1/n} = \lim_{n \to \infty} \frac{n}{\ln n} = \infty\). By part (a), the series diverges.
\textbf{(ii) \(\sum \frac{\ln n}{n}\):} Compare with \(b_n = \frac{1}{n}\) (divergent). \(\lim_{n \to \infty} \frac{(\ln n)/n}{1/n} = \lim_{n \to \infty} \ln n = \infty\). By part (a), the series diverges.

\textbf{50. Counterexample:} We need \(\lim (a_n/b_n) = 0\), \(\sum b_n\) diverges, but \(\sum a_n\) converges. Let \(a_n = \frac{1}{n^2}\) and \(b_n = \frac{1}{n}\). Then \(\sum a_n\) converges, \(\sum b_n\) diverges, and \(\lim_{n \to \infty} \frac{1/n^2}{1/n} = \lim_{n \to \infty} \frac{1}{n} = 0\).

\textbf{51. Proof:} Compare \(\sum a_n\) with the divergent harmonic series \(\sum \frac{1}{n}\) using LCT. Let \(L = \lim_{n \to \infty} n a_n\). We are given \(L \ne 0\). If \(L\) is finite and positive, then by LCT, since \(\sum \frac{1}{n}\) diverges, \(\sum a_n\) must also diverge. If \(L = \infty\), then by the special case of LCT (Problem 49), since \(\sum \frac{1}{n}\) diverges, \(\sum a_n\) must also diverge. The only case not covered is a limit that does not exist, but the core idea holds.

\textbf{52. Proof:} Given \(\sum a_n\) converges (with \(a_n > 0\)). This implies \(\lim_{n \to \infty} a_n = 0\). We want to test \(\sum \ln(1+a_n)\). Use LCT, comparing with \(\sum a_n\). We need to evaluate \(\lim_{n \to \infty} \frac{\ln(1+a_n)}{a_n}\). Since \(a_n \to 0\), let \(x=a_n\). This is equivalent to \(\lim_{x \to 0} \frac{\ln(1+x)}{x}\). Using L'Hôpital's Rule, this is \(\lim_{x \to 0} \frac{1/(1+x)}{1} = 1\). Since the limit is 1 (\(0<1<\infty\)) and \(\sum a_n\) converges, \(\sum \ln(1+a_n)\) also converges.

\textbf{53. True/False:} True. Given \(\sum a_n\) converges (with \(a_n > 0\)), so \(\lim_{n \to \infty} a_n = 0\). We test \(\sum \sin(a_n)\). Use LCT, comparing with \(\sum a_n\). We evaluate \(\lim_{n \to \infty} \frac{\sin(a_n)}{a_n}\). Since \(a_n \to 0\), this is equivalent to the fundamental limit \(\lim_{x \to 0} \frac{\sin x}{x} = 1\). Since the limit is 1 and \(\sum a_n\) converges, \(\sum \sin(a_n)\) also converges.

\textbf{54. Proof:} Given \(\sum a_n\) converges (with \(a_n \ge 0\)). This implies \(\lim_{n \to \infty} a_n = 0\). Because the limit is 0, there must exist an integer \(N\) such that for all \(n > N\), we have \(0 \le a_n < 1\). For these values of \(n\), multiplying by \(a_n\) (a positive number less than 1) gives \(a_n \cdot a_n < 1 \cdot a_n\), or \(a_n^2 < a_n\). Since \(\sum_{n=N+1}^\infty a_n\) converges, and \(0 \le a_n^2 < a_n\) for \(n>N\), the series \(\sum_{n=N+1}^\infty a_n^2\) must converge by DCT. Adding a finite number of terms does not affect convergence, so \(\sum a_n^2\) converges.

\textbf{55. True/False:}
\textbf{(a) False.} Counterexample: Let \(a_n = 1/n\) and \(b_n = 1/n\). \(\sum a_n\) and \(\sum b_n\) both diverge. But \(\sum a_n b_n = \sum (1/n^2)\) converges.
\textbf{(b) False.} Counterexample: Let \(a_n = 1/n^3\) (converges) and \(b_n = n\) (diverges). Then \(\sum a_n b_n = \sum (1/n^2)\) converges.
\textbf{(c) True.} If \(\sum b_n\) converges, then \(\lim_{n \to \infty} b_n = 0\). This implies that for \(n > N\), \(0 \le b_n < 1\). Thus, for \(n>N\), \(a_n b_n < a_n\). Since \(\sum a_n\) converges, \(\sum a_n b_n\) converges by DCT.

\section{In-Depth Analysis of Problems and Techniques}

\subsection{Problem Types and General Approach}
\begin{itemize}
    \item \textbf{Type 1: Rational and Algebraic Functions of n} (Problems 7-10, 15, 16, 18, 21-27, etc.)
    \begin{itemize}
        \item \textbf{General Approach:} These are the ideal candidates for the Limit Comparison Test. The strategy is to create a comparison p-series \(\sum b_n\) by taking the ratio of the highest power of \(n\) from the numerator and the denominator of \(a_n\). This "dominant term" approach almost always yields a finite, positive limit, making the conclusion straightforward.
    \end{itemize}
    \item \textbf{Type 2: Exponential Functions} (Problems 11, 12, 19, 28, 29, 33, 43, 44)
    \begin{itemize}
        \item \textbf{General Approach:} Compare with a geometric series. The base of the exponential term is key. For \(a_n = \frac{9^n}{3+10^n}\), the behavior is dominated by \(\frac{9^n}{10^n}\). LCT is usually cleanest. For terms like in Problem 28, a simple limit check using the Test for Divergence is often fastest.
    \end{itemize}
    \item \textbf{Type 3: Logarithmic and Factorial Functions} (Problems 13, 35, 36, 46, 47)
    \begin{itemize}
        \item \textbf{General Approach:} Use Direct Comparison based on growth rates. Key facts: \(\ln n < n\), and for large \(n\), \(n! > n^p\) for any \(p\). These allow for creating simple bounding inequalities.
    \end{itemize}
    \item \textbf{Type 4: Trigonometric and Special Functions} (Problems 14, 17, 31, 37, 38, 39)
    \begin{itemize}
        \item \textbf{General Approach:} For bounded functions like \(\sin(n)\) or \(\cos(n)\), use DCT by bounding the trig part (e.g., between -1 and 1). For functions where the argument goes to zero, like \(\sin(1/n)\) or \(\tan(1/n)\), use LCT with a p-series and fundamental trig limits (\(\lim_{x \to 0} \sin x / x = 1\)).
    \end{itemize}
    \item \textbf{Type 5: Theoretical Proofs and Counterexamples} (Problems 48-55)
    \begin{itemize}
        \item \textbf{General Approach:} These require a deep understanding of the test definitions and properties of convergent sequences. For proofs, start from the formal definition of the limit. For counterexamples, use simple, well-known series (like \(\sum 1/n\) and \(\sum 1/n^2\)) as building blocks.
    \end{itemize}
\end{itemize}

\subsection{Key Algebraic and Calculus Manipulations}
\begin{itemize}
    \item \textbf{Identifying Dominant Terms:} This was the cornerstone of most LCT problems. For any mix of polynomial, exponential, and logarithmic terms, you must identify the fastest-growing function in the numerator and denominator to choose your comparison series.
    \item \textbf{Strategic Bounding for DCT:} This was crucial for trigonometric functions and simple rational functions. In Problem 7, the step \(n^3+8 > n^3 \implies \frac{1}{n^3+8} < \frac{1}{n^3}\) is a classic example of making the denominator smaller to make the fraction larger, establishing an upper bound.
    \item \textbf{Limit Evaluation via Division by Highest Power:} In nearly every LCT calculation, this technique was used.
    \item \textbf{Using Standard Calculus Limits:} Problems 37-39 relied entirely on knowing \(\lim_{x \to 0} \frac{\sin x}{x} = 1\) and \(\lim_{x \to 0} \frac{\tan x}{x} = 1\). Problem 52 used \(\lim_{x \to 0} \frac{\ln(1+x)}{x}=1\). Recognizing when to apply these fundamental limits is key.
    \item \textbf{Integral Test for Advanced Cases:} Problem 46 (\(\sum 1/(n^p \ln n)\)) is a classic problem where the Integral Test is the most definitive way to handle the parameter \(p\).
\end{itemize}

\section{"Cheatsheet" and Tips for Success}

\subsection{Summary of Formulas}
\begin{itemize}
    \item \textbf{p-series} \(\sum \frac{1}{n^p}\): Converges if \(p>1\), Diverges if \(p \leq 1\).
    \item \textbf{Geometric Series} \(\sum ar^{n-1}\): Converges if \(|r|<1\), Diverges if \(|r| \geq 1\).
    \item \textbf{Direct Comparison Test (DCT):}
        \begin{itemize}
            \item If \(0 < a_n \leq b_n\) and \(\sum b_n\) converges \(\implies \sum a_n\) converges.
            \item If \(a_n \geq b_n > 0\) and \(\sum b_n\) diverges \(\implies \sum a_n\) diverges.
        \end{itemize}
    \item \textbf{Limit Comparison Test (LCT):} Let \(c = \lim_{n \to \infty} \frac{a_n}{b_n}\).
        \begin{itemize}
            \item If \(0 < c < \infty\), then \(\sum a_n\) and \(\sum b_n\) share the same fate.
            \item If \(c=0\) and \(\sum b_n\) converges, then \(\sum a_n\) converges.
            \item If \(c=\infty\) and \(\sum b_n\) diverges, then \(\sum a_n\) diverges.
        \end{itemize}
\end{itemize}

\subsection{Tips, Tricks, and Shortcuts}
\begin{itemize}
    \item \textbf{LCT is your workhorse:} When in doubt, try the LCT. It's often more robust than the DCT.
    \item \textbf{Dominant Term Shortcut:} For any rational function of \(n\), your comparison series \(\sum b_n\) is simply \(\sum \frac{n^{\text{degree of num}}}{n^{\text{degree of den}}}\).
    \item \textbf{When to use DCT:} Use DCT when the inequality is simple and obvious. For example, removing positive terms from a denominator makes the fraction bigger. Adding positive terms makes it smaller.
    \item \textbf{Recognizing Patterns:}
        \begin{itemize}
            \item Polynomials/Roots \(\to\) LCT with a p-series.
            \item Exponentials (\(c^n\)) \(\to\) LCT with a geometric series.
            \item \(\sin(n), \cos(n)\) \(\to\) DCT by bounding the term.
            \item \(\sin(1/n), \tan(1/n)\) \(\to\) LCT with \(\sum 1/n\).
        \end{itemize}
\end{itemize}

\subsection{Common Pitfalls to Avoid}
\begin{itemize}
    \item \textbf{Wrong-way inequality:} Do not try to use DCT if your series is SMALLER than a divergent series or LARGER than a convergent series. This is inconclusive.
    \item \textbf{Forgetting Positivity:} The Comparison Tests only apply to series with positive terms.
    \item \textbf{Misremembering Convergence Rules:} Do not mix up the condition for p-series (\(p>1\)) with the one for geometric series (\(|r|<1\)). A common mistake is thinking \(\sum 1/n\) converges. It does not.
\end{itemize}

\section{Conceptual Synthesis and The "Big Picture"}

\subsection{Thematic Connections}
The core theme of this topic is \textbf{determining the qualitative behavior of a system without explicit calculation}. We are not finding the sum of the series; we are only asking the question: "Is the sum finite?" This theme is central to many areas of mathematics and science. It is analogous to determining if a system is stable without needing to know its exact state at every future moment. This idea of comparing an unknown object to a known one is a fundamental problem-solving technique. The Integral Test does the same thing, but compares the discrete series to a continuous integral. In a broader sense, this is the heart of asymptotic analysis: understanding how functions and series behave "at infinity."

\subsection{Forward and Backward Links}
\begin{itemize}
    \item \textbf{Backward Links:} The Comparison Tests are a logical next step after learning the definition of a series and the most basic test, the Test for Divergence. They rely completely on your prerequisite knowledge of specific "benchmark" series—namely, p-series and geometric series. Without knowing how these benchmarks behave, comparison is impossible.
    \item \textbf{Forward Links:} The skill of identifying the dominant behavior of a term \(a_n\) is absolutely crucial for the Ratio and Root Tests, which are essentially more advanced forms of comparison used for more complicated series, especially power series. When you study the Interval of Convergence for a power series \(\sum c_n x^n\), you are fundamentally asking "For which values of x does this series converge?" and the techniques learned here will be essential.
\end{itemize}

\section{Real-World Application and Modeling}

\subsection{Concrete Scenarios in Finance and Economics}
\begin{enumerate}
    \item \textbf{Valuation of Perpetual Annuities:} In finance, a perpetuity is a stream of fixed payments that continues forever. Its present value is given by the series \(\sum_{n=1}^{\infty} \frac{C}{(1+r)^n}\), where \(C\) is the payment and \(r\) is the discount rate. This is a convergent geometric series. However, if the payments are not constant but grow, say \(C_n = \frac{C}{n}\) (a declining payment), the valuation model becomes \(\sum \frac{C}{n(1+r)^n}\). An analyst must use a comparison test (e.g., comparing to the convergent p-series \(\sum 1/n^2\)) to confirm that the security still has a finite value.
    \item \textbf{Economic Growth Models:} Some macroeconomic models describe a country's total economic output as the sum of outputs from infinitely many past investments. The output from an investment made \(n\) years ago might contribute \(I_n = \frac{I_0}{\sqrt{n^3+1}}\) to today's economy due to depreciation and other factors. To see if the total accumulated capital is finite, an economist would need to test the convergence of the series \(\sum I_n\). The Limit Comparison Test with \(\sum 1/n^{3/2}\) would quickly show this series converges, implying a stable, finite capital stock.
    \item \textbf{Stochastic Processes (Random Walks):} In quantitative finance, the price of a stock is often modeled as a random walk. Certain properties of these walks, like the probability of ever returning to the starting price, can be expressed as an infinite series. For a simple 1D random walk, the probability of return involves a term related to \(\frac{1}{\sqrt{n}}\). The divergence of \(\sum \frac{1}{\sqrt{n}}\), shown via comparison to \(\sum 1/n\), proves that the walk is "recurrent" and will return to the origin with probability 1.
\end{enumerate}

\subsection{Model Problem Setup}
Let's model the valuation of a company whose dividend payouts are expected to grow but at a decreasing rate.

\begin{itemize}
    \item \textbf{Scenario:} A mature tech company plans to issue dividends forever. Due to increasing competition, its annual dividend per share in year \(n\), denoted \(D_n\), is projected to be \( D_n = \frac{5n+2}{n^2+1} \). An investor wants to determine if the total present value of all future dividends is finite, using a discount rate of \(r=10\%\).
    \item \textbf{Model Setup:}
        \begin{itemize}
            \item \textbf{Variables:} \(D_n = \frac{5n+2}{n^2+1}\) is the dividend in year \(n\). \(r = 0.10\) is the discount rate.
            \item \textbf{Function:} The present value of the dividend from year \(n\) is \(PV_n = \frac{D_n}{(1+r)^n} = \frac{(5n+2)}{(n^2+1)(1.1)^n}\).
            \item \textbf{Series to Analyze:} The total present value (TPV) of the stock is the sum of all future discounted dividends:
            \[ TPV = \sum_{n=1}^{\infty} PV_n = \sum_{n=1}^{\infty} \frac{5n+2}{(n^2+1)(1.1)^n} \]
            \item \textbf{Convergence Test Setup:} To determine if the TPV is finite, we must test if this series converges. We can use the Limit Comparison Test. The dominant part of the rational function is \(\frac{5n}{n^2} = \frac{5}{n}\). The exponential term \((1.1)^n\) is the most dominant overall. Let's compare to a simpler series that captures this exponential decay, \(b_n = (\frac{1}{1.05})^n\). The exponential decay in the denominator is so powerful that it guarantees convergence. The LCT would show that \(\lim_{n \to \infty} a_n/b_n = 0\), and since \(\sum b_n\) converges, \(\sum a_n\) also converges by the special case of the LCT. The investor can conclude the stock has a finite theoretical value.
        \end{itemize}
\end{itemize}

\section{Common Variations and Untested Concepts}
The provided homework was quite comprehensive, covering the special LCT cases (Problems 48-49) and advanced comparison types (Problems 46-47). A concept that is related but not explicitly tested is applying these tests to series that are not initially positive.

\subsection{Absolute Convergence}
A key application of these tests is in determining the \textbf{absolute convergence} of a series that has both positive and negative terms. A series \(\sum a_n\) is said to converge absolutely if the series of its absolute values, \(\sum |a_n|\), converges.
Since \(\sum |a_n|\) is a series of positive terms, we can use the Comparison Tests on it.
\begin{itemize}
    \item \textbf{Theorem:} If a series converges absolutely, then it converges.
    \item \textbf{Example:} Determine if \(\sum_{n=1}^\infty \frac{\sin n}{n^2}\) converges.
    \begin{itemize}
        \item This series has positive and negative terms. We can't use the Comparison Test directly.
        \item Instead, we test the series of absolute values: \(\sum_{n=1}^\infty \left|\frac{\sin n}{n^2}\right| = \sum_{n=1}^\infty \frac{|\sin n|}{n^2}\).
        \item This is now a series of positive terms, so we can use the Direct Comparison Test.
        \item We know \(|\sin n| \le 1\). Therefore, \(\frac{|\sin n|}{n^2} \le \frac{1}{n^2}\).
        \item The comparison series \(\sum \frac{1}{n^2}\) is a convergent p-series (\(p=2>1\)).
        \item By the DCT, \(\sum \frac{|\sin n|}{n^2}\) converges.
        \item \textbf{Conclusion:} Since the series of absolute values converges, the original series \(\sum \frac{\sin n}{n^2}\) converges absolutely, and therefore it converges.
    \end{itemize}
\end{itemize}

\section{Advanced Diagnostic Testing: "Find the Flaw"}
The following five problems have solutions that appear correct but contain one subtle, critical error. Your task is to find the flaw, explain it, and provide the correct solution.

\subsection{Problem 1}
Determine if \(\sum_{n=2}^{\infty} \frac{n+5}{n^3-2}\) converges or diverges.

\textbf{Flawed Solution:}
Let \(a_n = \frac{n+5}{n^3-2}\). We will use the Limit Comparison Test. The dominant terms are \(n\) in the numerator and \(n^3\) in the denominator, so we compare with \(b_n = \frac{n}{n^3} = \frac{1}{n^2}\).
The comparison series \(\sum b_n = \sum \frac{1}{n^2}\) is a p-series with \(p=2\), so it converges.
Now we compute the limit:
\[ L = \lim_{n \to \infty} \frac{a_n}{b_n} = \lim_{n \to \infty} \frac{(n+5)/(n^3-2)}{1/n^2} = \lim_{n \to \infty} \frac{n^2(n+5)}{n^3-2} = \lim_{n \to \infty} \frac{n^3+5n^2}{n^3-2} = 1 \]
Since \(L=1\) is a finite positive number and \(\sum b_n\) converges, the series \(\sum a_n\) converges.

\subsection{Problem 2}
Determine if \(\sum_{n=1}^{\infty} \frac{3-\cos n}{\sqrt{n}}\) converges or diverges.

\textbf{Flawed Solution:}
Let \(a_n = \frac{3-\cos n}{\sqrt{n}}\). We will use the Direct Comparison Test.
We know that \(-1 \leq \cos n \leq 1\). Multiplying by -1 reverses the inequalities: \(1 \geq -\cos n \geq -1\).
Adding 3, we get \(3+1 \geq 3-\cos n \geq 3-1\), which is \(4 \geq 3-\cos n \geq 2\).
So, we can bound our term:
\[ \frac{3-\cos n}{\sqrt{n}} \leq \frac{4}{\sqrt{n}} \]
We compare with the series \(\sum b_n = \sum \frac{4}{\sqrt{n}}\). This is a constant times a p-series with \(p=1/2\). Since \(p \leq 1\), \(\sum b_n\) diverges.
Since our series is less than a divergent series, the Direct Comparison Test is inconclusive.

\subsection{Problem 3}
Determine if \(\sum_{n=1}^{\infty} \frac{5^n}{4^n+n^3}\) converges or diverges.

\textbf{Flawed Solution:}
Let \(a_n = \frac{5^n}{4^n+n^3}\). We use the Limit Comparison Test.
The dominant term in the denominator is \(4^n\). So, we compare with \(b_n = \frac{5^n}{4^n} = \left(\frac{5}{4}\right)^n\).
The comparison series \(\sum b_n\) is a geometric series with \(r = 5/4\). Since \(|r|>1\), it diverges.
Let's find the limit:
\[ L = \lim_{n \to \infty} \frac{a_n}{b_n} = \lim_{n \to \infty} \frac{5^n/(4^n+n^3)}{(5/4)^n} = \lim_{n \to \infty} \frac{5^n}{4^n+n^3} \cdot \frac{4^n}{5^n} = \lim_{n \to \infty} \frac{4^n}{4^n+n^3} \]
\[ L = \lim_{n \to \infty} \frac{1}{1 + n^3/4^n} = \frac{1}{1+0} = 1 \]
Since \(L=1\) is a finite positive number and \(\sum b_n\) diverges, the series \(\sum a_n\) diverges.

\subsection{Problem 4}
Determine if \(\sum_{n=1}^{\infty} \tan\left(\frac{1}{n^2}\right)\) converges or diverges.

\textbf{Flawed Solution:}
Let \(a_n = \tan(1/n^2)\). We use the Limit Comparison Test with \(b_n = \frac{1}{n}\).
The comparison series \(\sum b_n = \sum \frac{1}{n}\) is the harmonic series, which diverges.
We compute the limit using the substitution \(x = 1/n\). As \(n \to \infty\), \(x \to 0\).
\[ L = \lim_{n \to \infty} \frac{\tan(1/n^2)}{1/n} = \lim_{x \to 0} \frac{\tan(x^2)}{x} = \lim_{x \to 0} \frac{\sin(x^2)}{x \cos(x^2)} \]
\[ L = \lim_{x \to 0} \frac{\sin(x^2)}{x^2} \cdot \frac{x}{\cos(x^2)} = 1 \cdot \frac{0}{1} = 0 \]
Since the limit is \(L=0\) and the comparison series \(\sum b_n\) diverges, the test is inconclusive.

\subsection{Problem 5}
Determine if \(\sum_{n=1}^{\infty} \frac{1}{5^n - 3^n}\) converges or diverges.

\textbf{Flawed Solution:}
Let \(a_n = \frac{1}{5^n - 3^n}\). We will use the Direct Comparison Test.
For \(n \geq 1\), we know \(5^n - 3^n < 5^n\).
Taking the reciprocal reverses the inequality:
\[ \frac{1}{5^n - 3^n} > \frac{1}{5^n} \]
The comparison series is \(\sum b_n = \sum \frac{1}{5^n} = \sum (\frac{1}{5})^n\). This is a convergent geometric series since \(|r|=1/5 < 1\).
Our series \(\sum a_n\) has terms that are greater than the terms of a convergent series, so the Direct Comparison Test is inconclusive. Therefore, we cannot determine the convergence of this series.


\end{document}