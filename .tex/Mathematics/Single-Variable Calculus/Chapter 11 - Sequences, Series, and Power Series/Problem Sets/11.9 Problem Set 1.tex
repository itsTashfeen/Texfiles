\documentclass{article}
\usepackage{amsmath}
\usepackage{amssymb}
\usepackage[margin=1in]{geometry}

\title{Extensive Problem Set for Chapter 11.9: Representation of Functions as Power Series}
\author{Generated for Tashfeen Omran}
\date{November 2025}

\begin{document}

\maketitle

\section*{Practice Problems}
This problem set contains 65 problems designed to test all concepts related to representing functions as power series, as detailed in your study guide. The problems gradually increase in complexity.

% --- Problems ---

\subsection*{Problem 1}
Find a power series representation for the function $f(x) = \frac{1}{1+x}$ and determine its interval of convergence.

\subsection*{Problem 2}
Find a power series representation for the function $f(x) = \frac{3}{1-x}$ and determine its interval of convergence.

\subsection*{Problem 3}
Find a power series representation for the function $f(x) = \frac{1}{2-x}$ and determine its interval of convergence.

\subsection*{Problem 4}
Find a power series representation for the function $f(x) = \frac{5}{3+x}$ and determine its interval of convergence.

\subsection*{Problem 5}
Find a power series representation for the function $f(x) = \frac{1}{1-x^3}$ and determine its interval of convergence.

\subsection*{Problem 6}
Find a power series representation for the function $f(x) = \frac{1}{1+x^4}$ and determine its interval of convergence.

\subsection*{Problem 7}
Find a power series representation for the function $f(x) = \frac{1}{4-x^2}$ and determine its interval of convergence.

\subsection*{Problem 8}
Find a power series representation for the function $f(x) = \frac{x}{1-x}$ and determine its interval of convergence.

\subsection*{Problem 9}
Find a power series representation for the function $f(x) = \frac{x^2}{1+x}$ and determine its interval of convergence.

\subsection*{Problem 10}
Find a power series representation for the function $f(x) = \frac{3x^4}{1-x^2}$ and determine its interval of convergence.

\subsection*{Problem 11}
Find a power series representation for the function $f(x) = \frac{x}{9+x^2}$ and determine its interval of convergence.

\subsection*{Problem 12}
Find a power series representation for the function $f(x) = \frac{x^3}{2-x^2}$ and determine its interval of convergence.

\subsection*{Problem 13}
Use differentiation to find a power series representation for $f(x) = \frac{1}{(1-x)^2}$. What is the radius of convergence?

\subsection*{Problem 14}
Use differentiation to find a power series representation for $f(x) = \frac{1}{(1+x)^2}$. What is the radius of convergence?

\subsection*{Problem 15}
Use differentiation to find a power series representation for $f(x) = \frac{2}{(5-x)^2}$. What is the radius of convergence?

\subsection*{Problem 16}
Use the result from Problem 13 to find a power series for $f(x) = \frac{2x}{(1-x)^2}$.

\subsection*{Problem 17}
Use differentiation to find a power series representation for $f(x) = \frac{x}{(1+x^2)^2}$. What is the radius of convergence?

\subsection*{Problem 18}
Use differentiation twice to find a power series for $f(x) = \frac{2}{(1-x)^3}$. What is the radius of convergence?

\subsection*{Problem 19}
Use integration to find a power series representation for $f(x) = \ln(1+x)$. What is the radius of convergence?

\subsection*{Problem 20}
Use integration to find a power series representation for $f(x) = \ln(1-x)$. What is the radius of convergence?

\subsection*{Problem 21}
Use integration to find a power series representation for $f(x) = \ln(3+x)$. What is the radius of convergence?

\subsection*{Problem 22}
Use integration to find a power series representation for $f(x) = \ln(1+x^2)$. What is the radius of convergence?

\subsection*{Problem 23}
Find a power series representation for the function $f(x) = x \ln(1+x)$.

\subsection*{Problem 24}
Find a power series representation for the function $f(x) = x^2 \ln(1-x^2)$.

\subsection*{Problem 25}
Find a power series representation for the function $f(x) = \arctan(x)$. What is the radius of convergence?

\subsection*{Problem 26}
Find a power series representation for the function $f(x) = \arctan(x^3)$.

\subsection*{Problem 27}
Find a power series representation for the function $f(x) = x \arctan(x)$.

\subsection*{Problem 28}
Find a power series representation for the function $f(x) = \frac{x+2}{x-1}$. (Hint: Use algebraic manipulation first).

\subsection*{Problem 29}
Find a power series representation for the function $f(x) = \frac{x^2}{x+3}$. (Hint: Use algebraic manipulation first).

\subsection*{Problem 30}
Evaluate the indefinite integral $\int \frac{1}{1+x^5} dx$ as a power series. What is the radius of convergence?

\subsection*{Problem 31}
Evaluate the indefinite integral $\int \frac{x}{1-x^4} dx$ as a power series. What is the radius of convergence?

\subsection*{Problem 32}
Evaluate the indefinite integral $\int \ln(1-x) dx$ as a power series.

\subsection*{Problem 33}
Evaluate the indefinite integral $\int x \arctan(x^2) dx$ as a power series.

\subsection*{Problem 34}
Use the first three non-zero terms of the series for $\arctan(x)$ to approximate $\int_0^{0.5} \arctan(x) dx$.

\subsection*{Problem 35}
Find the sum of the series $\sum_{n=1}^\infty n x^{n-1}$ for $|x|<1$.

\subsection*{Problem 36}
Find the sum of the series $\sum_{n=0}^\infty \frac{x^n}{n!}$ by recognizing it as the Maclaurin series for a known function.

\subsection*{Problem 37}
Find the sum of the series $\sum_{n=0}^\infty (-1)^n \frac{x^{2n}}{(2n)!}$.

\subsection*{Problem 38}
Find the sum of the series $\sum_{n=0}^\infty \frac{3^n}{5^n n!}$.

\subsection*{Problem 39}
Find the sum of the series $1 - \ln(2) + \frac{(\ln 2)^2}{2!} - \frac{(\ln 2)^3}{3!} + \dots$.

\subsection*{Problem 40}
Find the sum of the series $\sum_{n=2}^\infty \frac{x^n}{n}$. (Hint: Differentiate the series).

\subsection*{Problem 41}
Use the binomial series to find the Maclaurin series for $f(x) = \sqrt{1+x}$. What is the radius of convergence?

\subsection*{Problem 42}
Use the binomial series to find the Maclaurin series for $f(x) = \frac{1}{\sqrt{1+x}}$. What is the radius of convergence?

\subsection*{Problem 43}
Use the binomial series to find the Maclaurin series for $f(x) = \sqrt[3]{1+x}$. What is the radius of convergence?

\subsection*{Problem 44}
Use the binomial series to find the Maclaurin series for $f(x) = \frac{1}{\sqrt{1-x^2}}$. What is the radius of convergence?

\subsection*{Problem 45}
Use the binomial series to find the Maclaurin series for $f(x) = (1+x)^{3/2}$.

\subsection*{Problem 46}
Find a power series representation for $f(x) = \frac{\ln(1+x)}{x}$.

\subsection*{Problem 47}
Find a power series for $f(x) = \frac{d}{dx}\left(\frac{1}{1+x^3}\right)$.

\subsection*{Problem 48}
Find a power series for $f(x) = \frac{x^2-1}{x-2}$.

\subsection*{Problem 49}
Find a power series representation for $f(x) = \ln\left(\frac{1+x}{1-x}\right)$. (Hint: Use properties of logarithms).

\subsection*{Problem 50}
Find a power series representation for $f(x) = \frac{1}{(4-x)^3}$. What is the radius of convergence?

\subsection*{Problem 51}
Evaluate $\int \frac{e^x-1}{x} dx$ as a power series.

\subsection*{Problem 52}
Find a power series for $f(x) = \sin(x^2)$.

\subsection*{Problem 53}
Use the series for $e^x$ to find the series for $f(x) = e^{-x^2}$.

\subsection*{Problem 54}
Evaluate $\int e^{-x^2} dx$ as a power series. This integral is fundamental in statistics.

\subsection*{Problem 55}
Find a power series for $f(x) = \cos(\sqrt{x})$. (Assume $x \ge 0$).

\subsection*{Problem 56}
Find a power series for $f(x) = \frac{1+x^2}{1-x^2}$. (Hint: Split the fraction).

\subsection*{Problem 57}
Use the binomial series to find the first four terms of the Maclaurin series for $f(x) = \sqrt[4]{16-x}$.

\subsection*{Problem 58}
Find a power series for the function $f(x) = \frac{5x-1}{x^2-x-2}$. (Hint: Use partial fraction decomposition).

\subsection*{Problem 59}
Find the sum of the series $\sum_{n=1}^\infty \frac{n}{2^n}$. (Hint: Consider a function $f(x) = \sum_{n=1}^\infty n x^n$).

\subsection*{Problem 60}
Find the sum of the series $\sum_{n=1}^\infty \frac{(-1)^{n-1}}{n \cdot 3^n}$. (Hint: Recognize the series for $\ln(1+x)$).

\subsection*{Problem 61}
Find a power series for $f(x) = \frac{x^3}{(1-2x)^2}$.

\subsection*{Problem 62}
Find a power series for $f(x) = \ln(x^2+4)$.

\subsection*{Problem 63}
Find the sum of the series $\frac{\pi}{2} - \frac{\pi^3}{2^3 \cdot 3!} + \frac{\pi^5}{2^5 \cdot 5!} - \dots$.

\subsection*{Problem 64}
Evaluate the indefinite integral $\int \frac{\arctan(x)}{x} dx$ as a power series.

\subsection*{Problem 65}
Use a power series to find the limit $\lim_{x \to 0} \frac{x - \ln(1+x)}{x^2}$.


\newpage
\section*{Solutions}

% --- Solutions ---

\subsection*{Solution to Problem 1}
The function is in the form $\frac{1}{1-r}$ with $r = -x$.
\[ f(x) = \frac{1}{1-(-x)} = \sum_{n=0}^{\infty} (-x)^n = \sum_{n=0}^{\infty} (-1)^n x^n \]
The series converges for $|r|<1 \implies |-x|<1 \implies |x|<1$.
The interval of convergence is $(-1, 1)$.

\subsection*{Solution to Problem 2}
The function is in the form $\frac{a}{1-r}$ with $a=3, r=x$.
\[ f(x) = \frac{3}{1-x} = 3 \sum_{n=0}^{\infty} x^n = \sum_{n=0}^{\infty} 3x^n \]
Converges for $|x|<1$. Interval: $(-1, 1)$.

\subsection*{Solution to Problem 3}
Factor out 2 from the denominator:
\[ f(x) = \frac{1}{2(1 - x/2)} = \frac{1}{2} \cdot \frac{1}{1 - (x/2)} \]
This is a geometric series with $a=1/2, r=x/2$.
\[ f(x) = \frac{1}{2} \sum_{n=0}^{\infty} \left(\frac{x}{2}\right)^n = \sum_{n=0}^{\infty} \frac{x^n}{2^{n+1}} \]
Converges for $|x/2|<1 \implies |x|<2$. Interval: $(-2, 2)$.

\subsection*{Solution to Problem 4}
Factor out 3:
\[ f(x) = \frac{5}{3(1 + x/3)} = \frac{5}{3} \cdot \frac{1}{1 - (-x/3)} \]
Geometric series with $a=5/3, r=-x/3$.
\[ f(x) = \frac{5}{3} \sum_{n=0}^{\infty} \left(-\frac{x}{3}\right)^n = \sum_{n=0}^{\infty} \frac{5(-1)^n x^n}{3^{n+1}} \]
Converges for $|-x/3|<1 \implies |x|<3$. Interval: $(-3, 3)$.

\subsection*{Solution to Problem 5}
This is a geometric series with $r=x^3$.
\[ f(x) = \frac{1}{1-x^3} = \sum_{n=0}^{\infty} (x^3)^n = \sum_{n=0}^{\infty} x^{3n} \]
Converges for $|x^3|<1 \implies |x|<1$. Interval: $(-1, 1)$.

\subsection*{Solution to Problem 6}
Geometric series with $r=-x^4$.
\[ f(x) = \frac{1}{1-(-x^4)} = \sum_{n=0}^{\infty} (-x^4)^n = \sum_{n=0}^{\infty} (-1)^n x^{4n} \]
Converges for $|-x^4|<1 \implies |x|<1$. Interval: $(-1, 1)$.

\subsection*{Solution to Problem 7}
Factor out 4:
\[ f(x) = \frac{1}{4(1-x^2/4)} = \frac{1}{4} \cdot \frac{1}{1 - (x^2/4)} \]
Geometric series with $a=1/4, r=x^2/4$.
\[ f(x) = \frac{1}{4} \sum_{n=0}^{\infty} \left(\frac{x^2}{4}\right)^n = \sum_{n=0}^{\infty} \frac{x^{2n}}{4^{n+1}} \]
Converges for $|x^2/4|<1 \implies x^2<4 \implies |x|<2$. Interval: $(-2, 2)$.

\subsection*{Solution to Problem 8}
\[ f(x) = x \cdot \frac{1}{1-x} = x \sum_{n=0}^{\infty} x^n = \sum_{n=0}^{\infty} x^{n+1} \]
Converges for $|x|<1$. Interval: $(-1, 1)$.

\subsection*{Solution to Problem 9}
\[ f(x) = x^2 \cdot \frac{1}{1-(-x)} = x^2 \sum_{n=0}^{\infty} (-x)^n = \sum_{n=0}^{\infty} (-1)^n x^{n+2} \]
Converges for $|-x|<1 \implies |x|<1$. Interval: $(-1, 1)$.

\subsection*{Solution to Problem 10}
\[ f(x) = 3x^4 \cdot \frac{1}{1-x^2} = 3x^4 \sum_{n=0}^{\infty} (x^2)^n = \sum_{n=0}^{\infty} 3x^{2n+4} \]
Converges for $|x^2|<1 \implies |x|<1$. Interval: $(-1, 1)$.

\subsection*{Solution to Problem 11}
\[ f(x) = x \cdot \frac{1}{9(1+x^2/9)} = \frac{x}{9} \cdot \frac{1}{1-(-x^2/9)} = \frac{x}{9} \sum_{n=0}^{\infty} \left(-\frac{x^2}{9}\right)^n = \sum_{n=0}^{\infty} \frac{(-1)^n x^{2n+1}}{9^{n+1}} \]
Converges for $|-x^2/9|<1 \implies x^2<9 \implies |x|<3$. Interval: $(-3, 3)$.

\subsection*{Solution to Problem 12}
\[ f(x) = x^3 \cdot \frac{1}{2(1-x^2/2)} = \frac{x^3}{2} \sum_{n=0}^{\infty} \left(\frac{x^2}{2}\right)^n = \sum_{n=0}^{\infty} \frac{x^{2n+3}}{2^{n+1}} \]
Converges for $|x^2/2|<1 \implies x^2<2 \implies |x|<\sqrt{2}$. Interval: $(-\sqrt{2}, \sqrt{2})$.

\subsection*{Solution to Problem 13}
Note that $\frac{d}{dx}\left(\frac{1}{1-x}\right) = \frac{1}{(1-x)^2}$.
\[ f(x) = \frac{d}{dx} \left(\sum_{n=0}^{\infty} x^n\right) = \sum_{n=1}^{\infty} nx^{n-1} \]
The radius of convergence remains $R=1$.

\subsection*{Solution to Problem 14}
Note that $\frac{d}{dx}\left(\frac{-1}{1+x}\right) = \frac{1}{(1+x)^2}$.
The series for $\frac{-1}{1+x} = - \sum_{n=0}^\infty (-x)^n = \sum_{n=0}^\infty (-1)^{n+1} x^n$.
\[ f(x) = \frac{d}{dx} \left(\sum_{n=0}^{\infty} (-1)^{n+1} x^n\right) = \sum_{n=1}^{\infty} (-1)^{n+1} n x^{n-1} \]
The radius of convergence remains $R=1$.

\subsection*{Solution to Problem 15}
Note that $\frac{d}{dx}\left(\frac{1}{5-x}\right) = \frac{1}{(5-x)^2}$. The series for $\frac{1}{5-x} = \frac{1}{5(1-x/5)} = \sum_{n=0}^\infty \frac{x^n}{5^{n+1}}$.
So, $\frac{1}{(5-x)^2} = \frac{d}{dx} \left(\sum_{n=0}^\infty \frac{x^n}{5^{n+1}}\right) = \sum_{n=1}^\infty \frac{nx^{n-1}}{5^{n+1}}$.
\[ f(x) = 2 \cdot \frac{1}{(5-x)^2} = \sum_{n=1}^\infty \frac{2nx^{n-1}}{5^{n+1}} \]
Radius of convergence remains $R=5$.

\subsection*{Solution to Problem 16}
Take the series from Problem 13 and multiply by $2x$:
\[ f(x) = 2x \sum_{n=1}^{\infty} nx^{n-1} = \sum_{n=1}^{\infty} 2nx^n \]

\subsection*{Solution to Problem 17}
Note that $\frac{d}{dx}\left(\frac{-1}{2(1+x^2)}\right) = \frac{x}{(1+x^2)^2}$.
The series for $\frac{-1}{2(1+x^2)} = -\frac{1}{2} \sum_{n=0}^\infty (-x^2)^n = \sum_{n=0}^\infty \frac{(-1)^{n+1}}{2}x^{2n}$.
\[ f(x) = \frac{d}{dx}\left(\sum_{n=0}^\infty \frac{(-1)^{n+1}}{2}x^{2n}\right) = \sum_{n=1}^\infty \frac{(-1)^{n+1}}{2}(2n)x^{2n-1} = \sum_{n=1}^\infty (-1)^{n+1} n x^{2n-1} \]
Radius of convergence remains $R=1$.

\subsection*{Solution to Problem 18}
Note that $\frac{d^2}{dx^2}\left(\frac{1}{1-x}\right) = \frac{2}{(1-x)^3}$.
We differentiate the series for $\frac{1}{1-x}$ twice.
First derivative: $\sum_{n=1}^\infty nx^{n-1}$.
Second derivative: $\frac{d}{dx}\left(\sum_{n=1}^\infty nx^{n-1}\right) = \sum_{n=2}^\infty n(n-1)x^{n-2}$.
Radius remains $R=1$.

\subsection*{Solution to Problem 19}
Note that $\int \frac{1}{1+x} dx = \ln(1+x) + C$.
The series for $\frac{1}{1+x} = \sum_{n=0}^\infty (-1)^n x^n$.
\[ \ln(1+x) = \int \left(\sum_{n=0}^\infty (-1)^n x^n\right) dx = C_0 + \sum_{n=0}^\infty \frac{(-1)^n x^{n+1}}{n+1} \]
At $x=0$, $\ln(1)=0$, so $C_0=0$.
\[ \ln(1+x) = \sum_{n=0}^\infty \frac{(-1)^n x^{n+1}}{n+1} = \sum_{k=1}^\infty \frac{(-1)^{k-1} x^{k}}{k} \]
Radius of convergence remains $R=1$.

\subsection*{Solution to Problem 20}
Note that $\int \frac{-1}{1-x} dx = \ln(1-x) + C$.
The series for $\frac{-1}{1-x} = - \sum_{n=0}^\infty x^n = \sum_{n=0}^\infty -x^n$.
\[ \ln(1-x) = \int \left(\sum_{n=0}^\infty -x^n\right) dx = C_0 + \sum_{n=0}^\infty -\frac{x^{n+1}}{n+1} \]
At $x=0$, $\ln(1)=0$, so $C_0=0$.
\[ \ln(1-x) = -\sum_{n=0}^\infty \frac{x^{n+1}}{n+1} = -\sum_{k=1}^\infty \frac{x^k}{k} \]
Radius of convergence remains $R=1$.

\subsection*{Solution to Problem 21}
Note that $\int \frac{1}{3+x} dx = \ln(3+x) + C$.
Series for $\frac{1}{3+x} = \frac{1}{3(1+x/3)} = \sum_{n=0}^\infty \frac{(-1)^n x^n}{3^{n+1}}$.
\[ \ln(3+x) = \int \left(\sum_{n=0}^\infty \frac{(-1)^n x^n}{3^{n+1}}\right) dx = C_0 + \sum_{n=0}^\infty \frac{(-1)^n x^{n+1}}{(n+1)3^{n+1}} \]
At $x=0$, $\ln(3)=C_0$.
\[ \ln(3+x) = \ln(3) + \sum_{n=0}^\infty \frac{(-1)^n x^{n+1}}{(n+1)3^{n+1}} = \ln(3) + \sum_{k=1}^\infty \frac{(-1)^{k-1} x^k}{k \cdot 3^k} \]
Radius remains $R=3$.

\subsection*{Solution to Problem 22}
Note that $\int \frac{2x}{1+x^2} dx = \ln(1+x^2) + C$. First find series for $\frac{2x}{1+x^2}$.
$\frac{2x}{1+x^2} = 2x \sum_{n=0}^\infty (-x^2)^n = \sum_{n=0}^\infty 2(-1)^n x^{2n+1}$.
\[ \ln(1+x^2) = \int \left(\sum_{n=0}^\infty 2(-1)^n x^{2n+1}\right) dx = C_0 + \sum_{n=0}^\infty \frac{2(-1)^n x^{2n+2}}{2n+2} = C_0 + \sum_{n=0}^\infty \frac{(-1)^n x^{2n+2}}{n+1} \]
At $x=0$, $\ln(1)=0$, so $C_0=0$.
\[ \ln(1+x^2) = \sum_{n=0}^\infty \frac{(-1)^n x^{2n+2}}{n+1} = \sum_{k=1}^\infty \frac{(-1)^{k-1} x^{2k}}{k} \]
Radius remains $R=1$.

\subsection*{Solution to Problem 23}
From Problem 19, $\ln(1+x) = \sum_{n=1}^\infty \frac{(-1)^{n-1} x^n}{n}$.
\[ x \ln(1+x) = x \sum_{n=1}^\infty \frac{(-1)^{n-1} x^n}{n} = \sum_{n=1}^\infty \frac{(-1)^{n-1} x^{n+1}}{n} \]

\subsection*{Solution to Problem 24}
From Problem 22, $\ln(1+u) = \sum_{k=1}^\infty \frac{(-1)^{k-1} u^k}{k}$. Let $u=-x^2$.
$\ln(1-x^2) = \sum_{k=1}^\infty \frac{(-1)^{k-1} (-x^2)^k}{k} = \sum_{k=1}^\infty \frac{(-1)^{k-1} (-1)^k x^{2k}}{k} = \sum_{k=1}^\infty \frac{-x^{2k}}{k}$.
\[ x^2 \ln(1-x^2) = x^2 \sum_{k=1}^\infty \frac{-x^{2k}}{k} = \sum_{k=1}^\infty \frac{-x^{2k+2}}{k} \]

\subsection*{Solution to Problem 25}
Note that $\int \frac{1}{1+x^2} dx = \arctan(x) + C$.
Series for $\frac{1}{1+x^2} = \sum_{n=0}^\infty (-x^2)^n = \sum_{n=0}^\infty (-1)^n x^{2n}$.
\[ \arctan(x) = \int \left(\sum_{n=0}^\infty (-1)^n x^{2n}\right) dx = C_0 + \sum_{n=0}^\infty \frac{(-1)^n x^{2n+1}}{2n+1} \]
At $x=0$, $\arctan(0)=0$, so $C_0=0$.
\[ \arctan(x) = \sum_{n=0}^\infty \frac{(-1)^n x^{2n+1}}{2n+1} \]
Radius of convergence remains $R=1$.

\subsection*{Solution to Problem 26}
Use the result from Problem 25, substituting $x^3$ for $x$.
\[ \arctan(x^3) = \sum_{n=0}^\infty \frac{(-1)^n (x^3)^{2n+1}}{2n+1} = \sum_{n=0}^\infty \frac{(-1)^n x^{6n+3}}{2n+1} \]

\subsection*{Solution to Problem 27}
Use the result from Problem 25 and multiply by $x$.
\[ x \arctan(x) = x \sum_{n=0}^\infty \frac{(-1)^n x^{2n+1}}{2n+1} = \sum_{n=0}^\infty \frac{(-1)^n x^{2n+2}}{2n+1} \]

\subsection*{Solution to Problem 28}
\[ f(x) = \frac{x-1+3}{x-1} = 1 + \frac{3}{x-1} = 1 - \frac{3}{1-x} = 1 - 3\sum_{n=0}^\infty x^n = 1 - \sum_{n=0}^\infty 3x^n \]

\subsection*{Solution to Problem 29}
Using polynomial long division or synthetic division, $\frac{x^2}{x+3} = x-3 + \frac{9}{x+3}$.
Series for $\frac{9}{x+3} = \frac{9}{3(1+x/3)} = 3\sum_{n=0}^\infty (-x/3)^n = \sum_{n=0}^\infty \frac{3(-1)^n x^n}{3^n}$.
\[ f(x) = x-3 + \sum_{n=0}^\infty \frac{(-1)^n x^n}{3^{n-1}} \]

\subsection*{Solution to Problem 30}
The integrand is $\frac{1}{1-(-x^5)} = \sum_{n=0}^\infty (-x^5)^n = \sum_{n=0}^\infty (-1)^n x^{5n}$.
\[ \int \left(\sum_{n=0}^\infty (-1)^n x^{5n}\right) dx = C + \sum_{n=0}^\infty \frac{(-1)^n x^{5n+1}}{5n+1} \]
Radius of convergence is $R=1$.

\subsection*{Solution to Problem 31}
The integrand is $x \cdot \frac{1}{1-x^4} = x \sum_{n=0}^\infty (x^4)^n = \sum_{n=0}^\infty x^{4n+1}$.
\[ \int \left(\sum_{n=0}^\infty x^{4n+1}\right) dx = C + \sum_{n=0}^\infty \frac{x^{4n+2}}{4n+2} \]
Radius of convergence is $R=1$.

\subsection*{Solution to Problem 32}
From Problem 20, $\ln(1-x) = -\sum_{k=1}^\infty \frac{x^k}{k}$.
\[ \int \left(-\sum_{k=1}^\infty \frac{x^k}{k}\right) dx = C - \sum_{k=1}^\infty \frac{x^{k+1}}{k(k+1)} \]

\subsection*{Solution to Problem 33}
From Problem 26, $\arctan(x^2) = \sum_{n=0}^\infty \frac{(-1)^n x^{4n+2}}{2n+1}$.
Integrand is $x \arctan(x^2) = \sum_{n=0}^\infty \frac{(-1)^n x^{4n+3}}{2n+1}$.
\[ \int \left(\sum_{n=0}^\infty \frac{(-1)^n x^{4n+3}}{2n+1}\right) dx = C + \sum_{n=0}^\infty \frac{(-1)^n x^{4n+4}}{(2n+1)(4n+4)} \]

\subsection*{Solution to Problem 34}
From Problem 25, $\arctan(x) = x - \frac{x^3}{3} + \frac{x^5}{5} - \dots$.
\[ \int_0^{0.5} \left(x - \frac{x^3}{3} + \frac{x^5}{5}\right) dx = \left[\frac{x^2}{2} - \frac{x^4}{12} + \frac{x^6}{30}\right]_0^{0.5} \]
\[ = \frac{(0.5)^2}{2} - \frac{(0.5)^4}{12} + \frac{(0.5)^6}{30} = \frac{0.25}{2} - \frac{0.0625}{12} + \frac{0.015625}{30} \approx 0.125 - 0.0052 + 0.0005 = 0.1203 \]

\subsection*{Solution to Problem 35}
The series $\sum_{n=1}^\infty n x^{n-1}$ is the derivative of $\sum_{n=0}^\infty x^n$.
The sum of $\sum_{n=0}^\infty x^n$ is $\frac{1}{1-x}$.
So, the sum of the series is $\frac{d}{dx}\left(\frac{1}{1-x}\right) = \frac{1}{(1-x)^2}$.

\subsection*{Solution to Problem 36}
This is the Maclaurin series for $e^x$.

\subsection*{Solution to Problem 37}
This is the Maclaurin series for $\cos(x)$.

\subsection*{Solution to Problem 38}
This is the series for $e^x$ with $x=3/5$. So the sum is $e^{3/5}$.
\[ \sum_{n=0}^\infty \frac{(3/5)^n}{n!} = e^{3/5} \]

\subsection*{Solution to Problem 39}
This is the series for $e^x$ with $x = -\ln(2)$. So the sum is $e^{-\ln(2)} = e^{\ln(2^{-1})} = 2^{-1} = 1/2$.

\subsection*{Solution to Problem 40}
Let $S(x) = \sum_{n=2}^\infty \frac{x^n}{n}$. Then $S'(x) = \sum_{n=2}^\infty x^{n-1} = x+x^2+x^3+\dots$.
This is a geometric series with $a=x, r=x$. The sum is $\frac{x}{1-x}$.
So $S(x) = \int \frac{x}{1-x} dx = \int (-1 + \frac{1}{1-x}) dx = -x - \ln(1-x) + C$.
Since $S(0)=0$, we have $0 = -0-\ln(1)+C \implies C=0$.
The sum is $-x - \ln(1-x)$.

\subsection*{Solution to Problem 41}
$f(x) = (1+x)^{1/2}$. Here $k=1/2$.
\[ (1+x)^{1/2} = 1 + \frac{1}{2}x + \frac{(1/2)(-1/2)}{2!}x^2 + \frac{(1/2)(-1/2)(-3/2)}{3!}x^3 + \dots \]
\[ = 1 + \frac{1}{2}x - \frac{1}{8}x^2 + \frac{1}{16}x^3 - \dots \]
Radius is $R=1$.

\subsection*{Solution to Problem 42}
$f(x) = (1+x)^{-1/2}$. Here $k=-1/2$.
\[ (1+x)^{-1/2} = 1 - \frac{1}{2}x + \frac{(-1/2)(-3/2)}{2!}x^2 + \frac{(-1/2)(-3/2)(-5/2)}{3!}x^3 + \dots \]
\[ = 1 - \frac{1}{2}x + \frac{3}{8}x^2 - \frac{5}{16}x^3 + \dots \]
Radius is $R=1$.

\subsection*{Solution to Problem 43}
$f(x) = (1+x)^{1/3}$. Here $k=1/3$.
\[ (1+x)^{1/3} = 1 + \frac{1}{3}x + \frac{(1/3)(-2/3)}{2!}x^2 + \frac{(1/3)(-2/3)(-5/3)}{3!}x^3 + \dots \]
\[ = 1 + \frac{1}{3}x - \frac{1}{9}x^2 + \frac{5}{81}x^3 - \dots \]
Radius is $R=1$.

\subsection*{Solution to Problem 44}
$f(x) = (1-x^2)^{-1/2}$. Use result from Problem 42 with $x \to -x^2$.
\[ = 1 - \frac{1}{2}(-x^2) + \frac{3}{8}(-x^2)^2 - \frac{5}{16}(-x^2)^3 + \dots = 1 + \frac{1}{2}x^2 + \frac{3}{8}x^4 + \frac{5}{16}x^6 + \dots \]
Radius is $R=1$.

\subsection*{Solution to Problem 45}
$f(x) = (1+x)^{3/2}$. Here $k=3/2$.
\[ (1+x)^{3/2} = 1 + \frac{3}{2}x + \frac{(3/2)(1/2)}{2!}x^2 + \frac{(3/2)(1/2)(-1/2)}{3!}x^3 + \dots \]
\[ = 1 + \frac{3}{2}x + \frac{3}{8}x^2 - \frac{1}{16}x^3 + \dots \]

\subsection*{Solution to Problem 46}
\[ f(x) = \frac{1}{x} \ln(1+x) = \frac{1}{x} \sum_{n=1}^\infty \frac{(-1)^{n-1}x^n}{n} = \sum_{n=1}^\infty \frac{(-1)^{n-1}x^{n-1}}{n} \]

\subsection*{Solution to Problem 47}
Series for $\frac{1}{1+x^3} = \sum_{n=0}^\infty (-x^3)^n = \sum_{n=0}^\infty (-1)^n x^{3n}$.
\[ f(x) = \frac{d}{dx} \left( \sum_{n=0}^\infty (-1)^n x^{3n} \right) = \sum_{n=1}^\infty (-1)^n (3n) x^{3n-1} \]

\subsection*{Solution to Problem 48}
$f(x) = \frac{(x-1)(x+1)}{x-2} = (x+1) \frac{x-1}{x-2} = (x+1)\left(1+\frac{1}{x-2}\right) = (x+1)\left(1-\frac{1}{2-x}\right)$.
$\frac{1}{2-x} = \sum_{n=0}^\infty \frac{x^n}{2^{n+1}}$.
So $f(x) = (x+1)\left(1 - \sum_{n=0}^\infty \frac{x^n}{2^{n+1}}\right)$. This is more complex to write in a single sum. An easier way: $f(x) = \frac{x^2-1}{x-2} = x+2 + \frac{3}{x-2} = x+2 - \frac{3}{2-x} = x+2 - \sum_{n=0}^\infty \frac{3x^n}{2^{n+1}}$.

\subsection*{Solution to Problem 49}
$f(x) = \ln(1+x) - \ln(1-x)$. Use series from Problems 19 and 20.
\[ f(x) = \left(\sum_{n=1}^\infty \frac{(-1)^{n-1} x^{n}}{n}\right) - \left(-\sum_{n=1}^\infty \frac{x^n}{n}\right) = \sum_{n=1}^\infty \frac{x^n}{n}((-1)^{n-1} + 1) \]
If $n$ is even, the term is 0. If $n$ is odd, the term is $2x^n/n$.
\[ f(x) = 2 \sum_{k=0}^\infty \frac{x^{2k+1}}{2k+1} \]

\subsection*{Solution to Problem 50}
We need to differentiate $\frac{1}{4-x}$ twice. The second derivative is $\frac{2}{(4-x)^3}$. So $f(x) = \frac{1}{2}\frac{d^2}{dx^2}\left(\frac{1}{4-x}\right)$.
Series for $\frac{1}{4-x} = \sum_{n=0}^\infty \frac{x^n}{4^{n+1}}$.
First derivative: $\sum_{n=1}^\infty \frac{nx^{n-1}}{4^{n+1}}$. Second derivative: $\sum_{n=2}^\infty \frac{n(n-1)x^{n-2}}{4^{n+1}}$.
\[ f(x) = \frac{1}{2} \sum_{n=2}^\infty \frac{n(n-1)x^{n-2}}{4^{n+1}} \]
Radius is $R=4$.

\subsection*{Solution to Problem 51}
$e^x = \sum_{n=0}^\infty \frac{x^n}{n!} = 1 + x + \frac{x^2}{2!} + \dots$.
$e^x-1 = \sum_{n=1}^\infty \frac{x^n}{n!}$.
$\frac{e^x-1}{x} = \sum_{n=1}^\infty \frac{x^{n-1}}{n!}$.
\[ \int \left(\sum_{n=1}^\infty \frac{x^{n-1}}{n!}\right) dx = C + \sum_{n=1}^\infty \frac{x^{n}}{n \cdot n!} \]

\subsection*{Solution to Problem 52}
The series for $\sin(u) = \sum_{n=0}^\infty \frac{(-1)^n u^{2n+1}}{(2n+1)!}$. Let $u=x^2$.
\[ \sin(x^2) = \sum_{n=0}^\infty \frac{(-1)^n (x^2)^{2n+1}}{(2n+1)!} = \sum_{n=0}^\infty \frac{(-1)^n x^{4n+2}}{(2n+1)!} \]

\subsection*{Solution to Problem 53}
The series for $e^u = \sum_{n=0}^\infty \frac{u^n}{n!}$. Let $u=-x^2$.
\[ e^{-x^2} = \sum_{n=0}^\infty \frac{(-x^2)^n}{n!} = \sum_{n=0}^\infty \frac{(-1)^n x^{2n}}{n!} \]

\subsection*{Solution to Problem 54}
Using the result from Problem 53:
\[ \int e^{-x^2} dx = \int \left(\sum_{n=0}^\infty \frac{(-1)^n x^{2n}}{n!}\right) dx = C + \sum_{n=0}^\infty \frac{(-1)^n x^{2n+1}}{n!(2n+1)} \]

\subsection*{Solution to Problem 55}
The series for $\cos(u) = \sum_{n=0}^\infty \frac{(-1)^n u^{2n}}{(2n)!}$. Let $u=\sqrt{x}$.
\[ \cos(\sqrt{x}) = \sum_{n=0}^\infty \frac{(-1)^n (\sqrt{x})^{2n}}{(2n)!} = \sum_{n=0}^\infty \frac{(-1)^n x^{n}}{(2n)!} \]

\subsection*{Solution to Problem 56}
\[ f(x) = \frac{1-x^2+2x^2}{1-x^2} = \frac{1-x^2}{1-x^2} + \frac{2x^2}{1-x^2} = 1 + 2x^2 \sum_{n=0}^\infty (x^2)^n = 1 + \sum_{n=0}^\infty 2x^{2n+2} \]

\subsection*{Solution to Problem 57}
$f(x) = (16-x)^{1/4} = 16^{1/4}(1 - x/16)^{1/4} = 2(1 - x/16)^{1/4}$.
Use binomial series for $(1+u)^k$ with $u=-x/16, k=1/4$.
\[ 2 \left[ 1 + \frac{1}{4}(-\frac{x}{16}) + \frac{(1/4)(-3/4)}{2!}(-\frac{x}{16})^2 + \frac{(1/4)(-3/4)(-7/4)}{3!}(-\frac{x}{16})^3 + \dots \right] \]
\[ = 2 \left[ 1 - \frac{x}{64} - \frac{3x^2}{8192} - \frac{7x^3}{786432} - \dots \right] = 2 - \frac{x}{32} - \frac{3x^2}{4096} - \frac{7x^3}{393216} - \dots \]

\subsection*{Solution to Problem 58}
Partial fractions: $\frac{5x-1}{(x-2)(x+1)} = \frac{A}{x-2} + \frac{B}{x+1}$. This gives $A=3, B=2$.
$f(x) = \frac{3}{x-2} + \frac{2}{x+1} = \frac{-3}{2-x} + \frac{2}{1+x} = -\frac{3}{2}\frac{1}{1-x/2} + 2\frac{1}{1-(-x)}$.
\[ f(x) = -\frac{3}{2}\sum_{n=0}^\infty \left(\frac{x}{2}\right)^n + 2\sum_{n=0}^\infty (-x)^n = \sum_{n=0}^\infty \left(-\frac{3}{2^{n+1}} + 2(-1)^n\right)x^n \]

\subsection*{Solution to Problem 59}
Consider $f(x) = \sum_{n=1}^\infty n x^n = x \sum_{n=1}^\infty n x^{n-1} = x \frac{d}{dx}\left(\frac{1}{1-x}\right) = \frac{x}{(1-x)^2}$.
The sum is $f(1/2) = \frac{1/2}{(1-1/2)^2} = \frac{1/2}{(1/2)^2} = 2$.

\subsection*{Solution to Problem 60}
This is the series for $\ln(1+x) = \sum_{n=1}^\infty \frac{(-1)^{n-1} x^n}{n}$ with $x=1/3$.
The sum is $\ln(1+1/3) = \ln(4/3)$.

\subsection*{Solution to Problem 61}
Series for $\frac{1}{(1-u)^2} = \sum_{n=1}^\infty nu^{n-1}$. Let $u=2x$.
$\frac{1}{(1-2x)^2} = \sum_{n=1}^\infty n(2x)^{n-1} = \sum_{n=1}^\infty n 2^{n-1} x^{n-1}$.
\[ f(x) = x^3 \sum_{n=1}^\infty n 2^{n-1} x^{n-1} = \sum_{n=1}^\infty n 2^{n-1} x^{n+2} \]

\subsection*{Solution to Problem 62}
$f(x) = \ln(4(1+x^2/4)) = \ln(4) + \ln(1+x^2/4)$.
Use series for $\ln(1+u)$ with $u=x^2/4$.
\[ f(x) = \ln(4) + \sum_{n=1}^\infty \frac{(-1)^{n-1}(x^2/4)^n}{n} = \ln(4) + \sum_{n=1}^\infty \frac{(-1)^{n-1}x^{2n}}{n \cdot 4^n} \]

\subsection*{Solution to Problem 63}
This matches the series for $\sin(x) = x - x^3/3! + x^5/5! - \dots$ evaluated at $x=\pi/2$.
The sum is $\sin(\pi/2) = 1$.

\subsection*{Solution to Problem 64}
Series for $\arctan(x) = \sum_{n=0}^\infty \frac{(-1)^n x^{2n+1}}{2n+1}$.
$\frac{\arctan(x)}{x} = \sum_{n=0}^\infty \frac{(-1)^n x^{2n}}{2n+1}$.
\[ \int \frac{\arctan(x)}{x} dx = C + \sum_{n=0}^\infty \frac{(-1)^n x^{2n+1}}{(2n+1)^2} \]

\subsection*{Solution to Problem 65}
$\ln(1+x) = x - \frac{x^2}{2} + \frac{x^3}{3} - \dots$.
$x - \ln(1+x) = x - \left(x - \frac{x^2}{2} + \frac{x^3}{3} - \dots\right) = \frac{x^2}{2} - \frac{x^3}{3} + \dots$.
\[ \frac{x - \ln(1+x)}{x^2} = \frac{\frac{x^2}{2} - \frac{x^3}{3} + \dots}{x^2} = \frac{1}{2} - \frac{x}{3} + \dots \]
As $x \to 0$, the limit is $1/2$.


\newpage
\section*{Concept and Problem Cross-Reference}

\subsection*{Concept Checklist}
Below is the list of concepts and problem types tested in this problem set, with the corresponding problem numbers for reference.

\begin{itemize}
    \item \textbf{Basic Geometric Series Transformation ($1/(a \pm bx)$)}
        \begin{itemize}
            \item Problems: 1, 2, 3, 4
        \end{itemize}
    \item \textbf{Geometric Series with Powers of x ($1/(a \pm bx^k)$)}
        \begin{itemize}
            \item Problems: 5, 6, 7
        \end{itemize}
    \item \textbf{Multiplying a Series by a Polynomial ($x^k \cdot f(x)$)}
        \begin{itemize}
            \item Problems: 8, 9, 10, 11, 12, 16, 23, 24, 27, 46, 61
        \end{itemize}
    \item \textbf{Algebraic Pre-processing (Long Division, Partial Fractions, etc.)}
        \begin{itemize}
            \item Problems: 28, 29, 48, 56, 58
        \end{itemize}
    \item \textbf{Term-by-Term Differentiation (to find series for $1/(a+bx)^k$)}
        \begin{itemize}
            \item Problems: 13, 14, 15, 17, 18, 47, 50
        \end{itemize}
    \item \textbf{Term-by-Term Integration (to find series for $\ln(\dots)$ and $\arctan(\dots)$)}
        \begin{itemize}
            \item Problems: 19, 20, 21, 22, 25, 26, 49, 62
        \end{itemize}
    \item \textbf{Representing an Integral as a Power Series}
        \begin{itemize}
            \item Problems: 30, 31, 32, 33, 51, 54, 64
        \end{itemize}
    \item \textbf{The Binomial Series ($ (1+x)^k $)}
        \begin{itemize}
            \item Problems: 41, 42, 43, 44, 45, 57
        \end{itemize}
    \item \textbf{Recognizing and Summing Known Maclaurin Series}
        \begin{itemize}
            \item Problems: 35, 36, 37, 38, 39, 59, 60, 63
        \end{itemize}
    \item \textbf{Substituting into Known Series ($x \to x^k$, $x \to \sqrt{x}$)}
        \begin{itemize}
            \item Problems: 26, 44, 52, 53, 55
        \end{itemize}
    \item \textbf{Multi-Step and Combination Problems}
        \begin{itemize}
            \item Problems: 17, 24, 33, 40, 49, 50, 56, 58, 61
        \end{itemize}
    \item \textbf{Applications (Approximation, Limits)}
        \begin{itemize}
            \item Problems: 34, 65
        \end{itemize}
\end{itemize}

\end{document}