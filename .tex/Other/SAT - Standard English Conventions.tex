\documentclass[11pt, a4paper]{article}
\usepackage[margin=0.75in]{geometry}
\usepackage[utf8]{inputenc}
\usepackage{xcolor}
\usepackage{tcolorbox}
\usepackage{enumitem}
\usepackage{amsmath}
\usepackage{amssymb}
\usepackage{titlesec}
\usepackage{parskip}

% Color Definitions
\definecolor{satblue}{RGB}{0, 51, 102}
\definecolor{sattrap}{RGB}{178, 34, 34}
\definecolor{satgreen}{RGB}{34, 139, 34}

% Custom Commands
\newcommand{\IC}{\textbf{\textcolor{satblue}{IC}}}
\newcommand{\DC}{\textbf{\textcolor{gray}{DC}}}
\newcommand{\correct}[1]{\textcolor{satgreen}{\textbf{\checkmark\ Correct:} #1}}
\newcommand{\wrong}[1]{\textcolor{sattrap}{\textbf{\ding{55} Incorrect:} #1}}
\newcommand{\trap}[1]{\textbf{\textcolor{sattrap}{TRAP:}} #1}

% Title Format
\titleformat{\section}
{\normalfont\Large\bfseries\color{satblue}}{\thesection}{1em}{}

\begin{document}

\begin{center}
    {\Huge \textbf{SAT Grammar Cheat Sheet}} \\
    \vspace{0.5em}
    {\Large Standard English Conventions} \\
    \vspace{1em}
    \textit{Think of SAT Grammar not as ``reading'' but as \textbf{Math}. Every punctuation mark is a variable that must fit a specific formula.}
\end{center}

\vspace{0.5cm}

% SECTION I
\section{The Legends (Definitions)}
Before applying formulas, you must identify the pieces:

\begin{tcolorbox}[colback=blue!5!white, colframe=satblue, title=\textbf{The Components}]
    \begin{itemize}[leftmargin=*]
        \item \IC{} \textbf{(Independent Clause)}: A complete sentence. It has a \textbf{Subject}, a \textbf{Verb}, and expresses a \textbf{Complete Thought}.
        \begin{itemize}
            \item \textit{Test:} Can it stand alone as a sentence? (e.g., \textit{The dog ran.})
        \end{itemize}
        
        \item \DC{} \textbf{(Dependent Clause/Phrase)}: A fragment. It is missing a subject, a verb, or starts with a ``subordinating word'' (like \textit{because, although, when, which}).
        \begin{itemize}
            \item \textit{Test:} Does it leave you hanging? (e.g., \textit{Because the dog ran...})
        \end{itemize}
    \end{itemize}
\end{tcolorbox}

% SECTION II
\section{The Punctuation Formulas}

\subsection*{1. Separating Two Sentences ($\IC + \IC$)}
You generally cannot join two complete sentences with just a comma. That is a \textbf{Comma Splice} (a major error).

\begin{itemize}
    \item \textbf{The Period:} $\IC . \IC$
    \item \textbf{The Semicolon:} $\IC ; \IC$
    \begin{quote}
        \textbf{Note:} On the SAT, a period and a semicolon are grammatically \textbf{identical}. If Choice A uses a period and Choice B uses a semicolon (and the words are the same), \textbf{both are wrong}.
    \end{quote}
    \item \textbf{Comma + FANBOYS:} $\IC, \text{(for/and/nor/but/or/yet/so)} \IC$
    \item \textbf{The Colon:} $\IC : \IC$ (Only if the second explains the first).
\end{itemize}

\subsection*{2. Connecting Dependent \& Independent ($\IC + \DC$)}
\begin{itemize}
    \item \textbf{Introductory Phrase:} $\DC, \IC$
    \item \textbf{Trailing Phrase:} $\IC, \DC$
    \item \textbf{Essential Phrase (Restrictive):} $\IC \DC$ (No punctuation)
    \begin{quote}
        \textit{Example:} The boy \textbf{who runs} is fast.
    \end{quote}
\end{itemize}

\subsection*{3. The ``Sandwich'' Rule (Parenthetical Information)}
If you insert extra information in the middle of a sentence, you need two pieces of punctuation (bookends). You must \textbf{match} them.

\begin{tcolorbox}[colback=yellow!10!white, colframe=orange!80!black]
    \begin{itemize}
        \item \textbf{Commas:} ... word, [extra info], word ...
        \item \textbf{Dashes:} ... word --- [extra info] --- word ...
        \item \textbf{Parentheses:} ... word ([extra info]) word ...
    \end{itemize}
    \trap{Do not mix and match (e.g., \textit{word, extra info --- word} is WRONG).} \\
    \textbf{Strategy:} Look at the punctuation \textit{before} the blank. If you see an opening dash earlier, the answer must have a closing dash.
\end{tcolorbox}

\subsection*{4. The Colon (:) Rule}
\textbf{Formula:} $\IC : [\text{List, Explanation, or Definition}]$
\begin{itemize}
    \item \textbf{The Rule:} The \textbf{Left Side} must be a complete sentence. The \textbf{Right Side} can be anything (a sentence, a fragment, a single word).
    \item \textbf{Usage:} Use it when the second part ``defines'' or ``illustrates'' the first part.
\end{itemize}

\subsection*{5. The Dash (---) Rule}
\begin{itemize}
    \item \textbf{Emphasis:} Acts like a colon but with more drama. $\IC \text{ --- Definition}$
    \item \textbf{Interruption:} Acts like parentheses. $\IC \text{ --- [interruption] --- } \IC$
\end{itemize}

% SECTION III
\section{Specific SAT Traps \& Patterns}

\begin{enumerate}
    \item \textbf{The ``So... That...'' Rule} \\
    Correlative conjunctions are teams. They don't like to be separated by commas.
    \begin{itemize}
        \item \textbf{Pattern:} So [adjective] that [result]
        \item \textbf{Rule:} \textbf{NO} comma before ``that.''
        \item \correct{It was \textbf{so} cold \textbf{that} I froze.}
        \item \wrong{It was \textbf{so} cold, \textbf{that} I froze.}
    \end{itemize}

    \item \textbf{Restrictive Appositives (Names \& Titles)} \\
    When you have a category followed by a specific name, ask: \textbf{Is the name essential?}
    \begin{itemize}
        \item \textbf{Essential (Generic Category):} The poet [Name]
        \begin{itemize}
            \item \textbf{Rule:} No commas.
            \item \textit{Example:} The poet \textbf{Maya Angelou} wrote this. (There are many poets; we need her name).
        \end{itemize}
        \item \textbf{Non-Essential (Specific Category):} My mother, [Name],
        \begin{itemize}
            \item \textbf{Rule:} Use commas.
            \item \textit{Example:} My mother, \textbf{Susan}, is here. (I only have one mother; the name is extra info).
        \end{itemize}
        \item \textbf{SAT Shortcut:} If the choice is \textit{Compound Name} (no commas), it is usually right.
    \end{itemize}

    \item \textbf{Lists and ``AND''}
    \begin{itemize}
        \item \textbf{List of 3+ items:} A, B, and C (Use commas).
        \item \textbf{List of 2 items:} A and B (No commas).
        \item \trap{The researchers determined [Item 1] and [Item 2]. Do not put a comma before ``and'' here.}
        \item \trap{Because of [Noun 1] and [Noun 2]. No commas.}
    \end{itemize}

    \item \textbf{Dangling Modifiers} \\
    If a sentence starts with a descriptive phrase ($\DC, \IC$), the very next word \textbf{must} be the thing being described.
    \begin{itemize}
        \item \wrong{Walking down the street, the trees looked beautiful. (The trees were walking?)}
        \item \correct{Walking down the street, \textbf{I} thought the trees looked beautiful.}
        \item \textbf{Trailing Modifier:} $\IC, \text{[verb]-ing...}$ The action in the ``-ing'' phrase refers to the \textbf{Subject} of the main sentence.
    \end{itemize}

    \item \textbf{``It's'' vs. ``Its''}
    \begin{itemize}
        \item \textbf{It's} = \textbf{It is}. (Contains a Verb. Creates an Independent Clause).
        \item \textbf{Its} = \textbf{Possessive}. (Adjective. Does not create a clause).
        \item \trap{\textit{The car is fast; \textbf{it's} engine is loud.} (Correct, because ``it's'' = ``it is'', making two sentences. Use a semicolon).}
    \end{itemize}

    \item \textbf{Transition Words (However, Therefore, Thus)} \\
    Placement matters.
    \begin{itemize}
        \item \textbf{Beginning of sentence:} $\IC. \text{ However, } \IC.$
        \item \textbf{Middle of sentence (Sandwich):} $\IC, \text{ however, is...}$
        \item \textbf{The Logic Trap:} Sometimes ``however'' belongs to the \textbf{first} sentence.
        \begin{itemize}
            \item \textit{Example:} $\IC, \text{ however}; \IC$ 
            \item \textit{Meaning:} ``That is true, however. On the other hand, this is also true.''
            \item \textbf{Test:} Read the sentences separately to see which one ``owns'' the transition word.
        \end{itemize}
    \end{itemize}

    \item \textbf{``That'' vs. ``Which''}
    \begin{itemize}
        \item \textbf{That:} Essential info. No commas. (\textit{The bike \textbf{that} I broke.})
        \item \textbf{Which:} Extra info. Needs commas. (\textit{The bike, \textbf{which} I broke, is red.})
        \item \trap{You virtually \textbf{NEVER} put a comma after the word ``that'' (e.g., \textit{...the fact that, the water...} is WRONG).}
    \end{itemize}
\end{enumerate}

% SECTION IV
\section{Final Strategy Checklist}

\begin{tcolorbox}[colback=green!5!white, colframe=satgreen, title=\textbf{Strategy Protocol}]
\begin{enumerate}
    \item \textbf{Read the whole sentence.} Don't just look at the blank. You need to find the subject and main verb of the \textit{entire} text to know if you are dealing with a fragment or a sentence.
    
    \item \textbf{Process of Elimination:}
    \begin{itemize}
        \item If \textit{Answer A} = Period and \textit{Answer B} = Semicolon, \textbf{Cross them both out}.
        \item If \textit{Answer A} creates a sentence fragment (missing verb), cross it out.
    \end{itemize}
    
    \item \textbf{The ``Removal Test'':}
    \begin{itemize}
        \item If you think an answer choice is creating a parenthetical sandwich (commas or dashes), cross out the text between the punctuation. Does the remaining sentence make sense?
        \item \textbf{Yes?} The punctuation is correct.
        \item \textbf{No?} The punctuation is wrong.
    \end{itemize}
    
    \item \textbf{Check for ``Comma Splices'':} This is the most common wrong answer. Always check if you are connecting two complete sentences with a weak comma.
\end{enumerate}
\end{tcolorbox}

\end{document}