\documentclass{article}
\usepackage{amsmath, amssymb, geometry, hyperref, xcolor, graphicx, tcolorbox}
\geometry{a4paper, margin=1in}
\title{Comprehensive Study Guide: Deriving the Black-Scholes-Merton Framework}
\author{Generated by Gemini 3 Pro}
\date{\today}

\begin{document}
\maketitle
\tableofcontents

\section{Conceptual Foundations: The Delta-Hedged Portfolio}

The derivation of the Black-Scholes-Merton (BSM) model begins not with the formula itself, but with the construction of a specific portfolio designed to eliminate directional risk.

\subsection{The Portfolio Construction}
To mitigate directional risk, option traders combine options with a position in the underlying asset. We define a portfolio $\Pi$ consisting of:
\begin{itemize}
    \item A long position in a call option ($C$).
    \item A short position in the underlying stock.
\end{itemize}

The value of this portfolio at time $t$ is given by:
\[
\Pi_t = C - \Delta S_t
\]
Where:
\begin{itemize}
    \item $C$ is the value of the call option.
    \item $S_t$ is the price of the underlying at time $t$.
    \item $\Delta$ is the number of shares sold short. Crucially, in this context, $\Delta$ represents the hedge ratio (the partial derivative $\frac{\partial C}{\partial S}$).
\end{itemize}

\subsection{Analyzing Portfolio Changes and Financing}
Over a small time step, the change in the portfolio's value includes the change in asset prices and the cost of financing the position. The total change is expressed as:

\[
\text{Change} = \underbrace{[C(S_{t+1}) - C(S_t)]}_{\text{Change in Option}} - \underbrace{\Delta(S_{t+1} - S_t)}_{\text{P/L from Short Stock}} - \underbrace{r(C - \Delta S_t)}_{\text{Financing Cost}}
\]

\paragraph{Understanding the Financing Term:}
The term $-r(C - \Delta S_t)$ represents the interest paid or received on the net value of the portfolio.
\begin{itemize}
    \item \textbf{Cost of Call ($-rC$):} We must borrow money to buy the option, incurring a debit of interest.
    \item \textbf{Interest on Short Stock ($+r\Delta S_t$):} When we short $\Delta$ shares, we receive cash proceeds equal to $\Delta S_t$. This cash earns the risk-free rate $r$.
\end{itemize}
Thus, shorting the stock provides financing that offsets the cost of holding the long option.

\section{The Mathematical Engine: Taylor Expansions and The Greeks}

To solve for the option price, we must approximate how the option's value $C$ changes as $S$ and $t$ change. We utilize a \textbf{Second-Order Taylor Expansion}.

\subsection{The Taylor Expansion of Option Value}
The change in the option value $dC$ is approximated by expanding the function $C(S, t)$:

\[
dC \approx \underbrace{\frac{\partial C}{\partial t} dt}_{\text{Time Decay}} + \underbrace{\frac{\partial C}{\partial S} dS}_{\text{Linear Price Move}} + \underbrace{\frac{1}{2} \frac{\partial^2 C}{\partial S^2} (dS)^2}_{\text{Curvature}}
\]

Mapping this to the Greeks:
\begin{align*}
    \Theta (\text{Theta}) &= \frac{\partial C}{\partial t} \\
    \Delta (\text{Delta}) &= \frac{\partial C}{\partial S} \\
    \Gamma (\text{Gamma}) &= \frac{\partial^2 C}{\partial S^2}
\end{align*}

Substituting these into the portfolio change equation, the linear delta terms ($\Delta dS$) cancel out perfectly. This is the definition of being "Delta Neutral." We are left with:

\[
\text{Portfolio Change} \approx \Theta dt + \frac{1}{2}\Gamma (dS)^2 - r(C - \Delta S)dt
\]

\subsection{Deep Dive: The Sign of Theta}
A common confusion arises from the term $+ \Theta$ in the equation, given that option holders typically lose money due to time decay.

\begin{tcolorbox}[colback=gray!10, title=Clarification: Why is Theta Positive in the Equation?]
The symbol $\Theta$ represents the partial derivative $\frac{\partial C}{\partial t}$. For a long option position, the value decreases as time passes. Therefore, the mathematical value of the variable $\Theta$ is \textbf{negative} (e.g., $\Theta = -0.05$).

The equation adds the variable $\Theta$. Since $\Theta$ is a negative number, adding it results in a reduction of value. The text states "the option holder loses money," which is consistent with adding a negative parameter.
\end{tcolorbox}

\section{The Stochastic Leap: From Realized Returns to Volatility}

The derivation relies on a critical substitution derived from \textbf{Ito's Lemma}. This is the bridge between randomness and deterministic pricing.

\subsection{Realized Squared Change vs. Variance}
In the Taylor expansion, we encounter the term $(dS)^2$, or $(S_{t+1} - S_t)^2$.
\begin{itemize}
    \item \textbf{The Random Variable:} $(S_{t+1} - S_t)^2$ is the \textit{realized} squared price change over a single time step. It is unknown until it happens.
    \item \textbf{The Parameter:} $\sigma^2$ (Variance) is the \textit{expected} dispersion of returns over time.
\end{itemize}

\subsection{The Fundamental Substitution}
Under the assumption that stock prices follow a Geometric Brownian Motion, Ito's Lemma allows us to replace the random realized squared change with a deterministic expected value as the time step $dt \to 0$:

\[
(S_{t+1} - S_t)^2 \approx \sigma^2 S^2 dt
\]

\textbf{Important Distinctions:}
\begin{itemize}
    \item $\sigma^2$ is the Variance.
    \item $\sigma$ is the Standard Deviation (Volatility).
    \item The $S^2$ term exists because volatility acts on returns (percentages), so the dollar variance scales with the square of the stock price ($S^2$).
\end{itemize}

By substituting $\sigma^2 S^2 dt$ for $(dS)^2$, we eliminate the random component of the portfolio's change. The portfolio becomes risk-free.

\section{Drift, Risk Neutrality, and No Arbitrage}

\subsection{The Irrelevance of Drift ($\mu$)}
A counter-intuitive result of the BSM derivation is that the expected return of the stock (drift) does not appear in the final equation.
\begin{itemize}
    \item While a call option benefits from upward drift, the delta-hedged portfolio includes a short stock position that \textit{loses} from upward drift.
    \item By holding the correct ratio ($\Delta$), the effects of drift cancel out perfectly.
    \item Since drift is hedged away, the market offers no risk premium for it.
\end{itemize}

\subsection{The Law of One Price}
Because the delta-hedged portfolio has had its random risk terms ($\Gamma$) converted to deterministic terms (via Ito's Lemma) and its directional risk ($\Delta$) eliminated, it is a \textbf{risk-free asset}.

By the \textbf{No Arbitrage} principle, a risk-free portfolio must earn exactly the risk-free rate $r$. If it earned more, arbitrageurs would borrow at $r$ to buy the portfolio; if less, they would short the portfolio to lend at $r$.

\[
\text{Change in Portfolio} = r \times (\text{Portfolio Value}) dt
\]

Setting the derived change equal to the risk-free return yields the Black-Scholes PDE.

\section{Practical Application: The Reality of Hedging}

In the theoretical derivation, we assume continuous rebalancing. In practice, traders face the "Trader's Dilemma."

\subsection{Static vs. Dynamic Hedging}
\begin{itemize}
    \item \textbf{Static Hedging:} Setting a hedge and leaving it. This fails because Delta changes as Spot changes (Gamma).
    \item \textbf{Dynamic Hedging:} Continuously adjusting the hedge ratio.
    \item \textbf{Transaction Costs:} In reality, continuous hedging implies infinite transaction costs. Traders must balance the cost of rebalancing against the risk of being unhedged (Gamma risk).
\end{itemize}

\subsection{Trading Volatility}
A delta-hedged portfolio is not "safe" from all risks; it transforms directional risk into volatility risk.
\begin{itemize}
    \item \textbf{Long Gamma/Vega:} The trader profits if realized volatility (magnitude of moves) exceeds the implied volatility paid for the option.
    \item \textbf{Short Theta:} The trader pays "rent" (time decay) every day for the privilege of holding the Gamma/Vega exposure.
\end{itemize}

\subsection{Hedging the Greeks}
\begin{itemize}
    \item \textbf{Delta Neutral:} Hedged against small directional moves (via stock).
    \item \textbf{Gamma/Vega Neutral:} Requires buying/selling other options to offset curvature and volatility sensitivity.
    \item \textbf{Theta Neutral:} Generally impossible without exiting the position. Theta is the cost of doing business.
\end{itemize}

\end{document}