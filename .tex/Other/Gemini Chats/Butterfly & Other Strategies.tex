\documentclass{article}
\usepackage{amsmath, amssymb, geometry, hyperref, xcolor, graphicx, fancyhdr}

% Layout settings
\geometry{a4paper, margin=1in}
\setlength{\parindent}{0pt}
\setlength{\parskip}{1em}

% Title Information
\title{Comprehensive Study Guide: Option Pricing Mechanics \& Strategies}
\author{Generated by Gemini 3 Pro}
\date{\today}

\begin{document}

\maketitle
\tableofcontents
\newpage

\section{Phase 1: Conceptual Foundations}

\subsection{Components of Option Premium}
An option's total price (premium) is the sum of two distinct components:
\[
\text{Total Premium} = \text{Intrinsic Value} + \text{Extrinsic Value}
\]

\subsubsection{Intrinsic Value (Moneyness)}
Intrinsic value represents the tangible equity in an option if it were exercised immediately. It is strictly mathematical and cannot be negative.

\begin{itemize}
    \item \textbf{Call Option:} Intrinsic Value = $\max(S - K, 0)$
    \item \textbf{Put Option:} Intrinsic Value = $\max(K - S, 0)$
\end{itemize}

Where $S$ is the Underlying Price and $K$ is the Strike Price.

\textbf{Moneyness Definitions:}
\begin{itemize}
    \item \textbf{In-the-Money (ITM):} Has intrinsic value $> 0$.
    \item \textbf{Out-of-the-Money (OTM):} Has intrinsic value $= 0$.
    \item \textbf{At-the-Money (ATM):} $S \approx K$.
\end{itemize}

\subsubsection{Extrinsic Value (Time Value)}
Extrinsic value is the premium paid above the intrinsic value. It represents the "risk premium" or the probability that the option will move further ITM before expiration.

The three primary factors influencing Extrinsic Value are:
\begin{enumerate}
    \item \textbf{Time to Expiry:} More time = higher value (due to higher probability of price movement).
    \item \textbf{Volatility ($\sigma$):} Higher volatility = higher value (greater range of potential outcomes).
    \item \textbf{Strike Price:} Extrinsic value is generally highest when the option is At-the-Money (ATM).
\end{enumerate}

\section{Phase 2: Core Mathematics \& Logic}

\subsection{Put-Call Parity (PCP)}
Put-Call Parity describes the no-arbitrage relationship between the prices of European calls ($C$) and puts ($P$) with the same strike ($K$) and expiration. In a simplified model (ignoring interest rates and dividends for short-term variations), the equation is:

\begin{equation}
C + K = P + U
\end{equation}

Where:
\begin{itemize}
    \item $C$ = Call Price
    \item $K$ = Strike Price
    \item $P$ = Put Price
    \item $U$ = Underlying Asset Price
\end{itemize}

\subsection{Vega and Price Sensitivity}
\textbf{Vega} measures an option's sensitivity to changes in implied volatility. A critical property derived from Put-Call Parity is that for the same strike and expiration:
\[
\text{Vega}_{\text{call}} = \text{Vega}_{\text{put}}
\]
This means if implied volatility increases, causing the Put price to rise by amount $X$, the Call price must also rise by exactly amount $X$ (assuming the underlying price remains constant).

\section{Phase 3: Option Strategies}

\subsection{The Straddle}
A straddle is a volatility strategy involving the purchase of both a Call and a Put at the same strike $K$ and expiration.
\[
\text{Straddle Value} = \text{Call Price} + \text{Put Price}
\]
The strategy profits if the stock makes a significant move in \textit{either} direction.

\subsection{The Long Butterfly Spread}
A long butterfly spread is a neutral strategy constructed by:
\begin{itemize}
    \item Buying 1 Lower Strike Call ($K_L$)
    \item Selling 2 Middle Strike Calls ($K_M$)
    \item Buying 1 Higher Strike Call ($K_H$)
\end{itemize}

\textbf{Visual Analysis (Payoff Diagram):}
The payoff diagram resembles a "tent."
\begin{itemize}
    \item \textbf{Max Loss:} Occurs outside the "wings" ($< K_L$ or $> K_H$). The max loss is equal to the \textbf{Net Debit} paid to enter the trade.
    \item \textbf{Max Profit:} Occurs exactly at the middle strike ($K_M$).
    \item \textbf{Breakeven Points:}
    \begin{align*}
        \text{Lower BE} &= K_L + \text{Net Debit} \\
        \text{Upper BE} &= K_H - \text{Net Debit}
    \end{align*}
\end{itemize}

\section{Phase 4: Applied Exercises}

\subsection{Exercise 1: Intrinsic Value Calculation}
\textbf{Scenario:} A February soybean future ($U$) is trading at \$1005.25. Consider a January 950 strike Call.

\textbf{Analysis:}
\begin{itemize}
    \item Is $U > K$? Yes ($1005.25 > 950$). Therefore, the option is \textbf{ITM}.
    \item Intrinsic Value = $1005.25 - 950.00 = \mathbf{55.25}$.
\end{itemize}

\subsection{Exercise 2: Volatility Impact (Vega)}
\textbf{Scenario:} Based on Exercise 1, if implied volatility increases such that the 950 Put price increases by \$0.50, what is the new Call price?

\textbf{Solution:}
Because $\text{Vega}_{\text{call}} = \text{Vega}_{\text{put}}$, the Call price increases by the exact same amount.
\begin{align*}
    \text{New Call Price} &= \text{Old Call Price} + \Delta \text{Volatility Premium} \\
    &= 55.25 (\text{Intrinsic}) + 0.50 \\
    &= \mathbf{55.75}
\end{align*}
\textit{Note: This assumes the original call price was purely intrinsic, or uses the intrinsic base to show the relative change.}

\subsection{Exercise 3: Solving for Variables with PCP}
\textbf{Scenario:} $K = 10$, $U = 12.40$, $C = 3.55$. Find $P$.

\textbf{Solution:}
Using $C + K = P + U$:
\begin{align*}
    3.55 + 10 &= P + 12.40 \\
    13.55 &= P + 12.40 \\
    P &= 13.55 - 12.40 \\
    P &= \mathbf{1.15}
\end{align*}

\subsection{Exercise 4: Pricing a Straddle via Parity}
\textbf{Scenario:} Underlying ($U$) = 98.20, Call ($K=100$) = 4.00. Find the value of the 100 Strike Straddle.

\textbf{Step 1: Find the Put Price.}
Using PCP ($C + K = P + U$):
\begin{align*}
    4.00 + 100 &= P + 98.20 \\
    104.00 &= P + 98.20 \\
    P &= 104.00 - 98.20 \\
    P &= 5.80
\end{align*}

\textbf{Step 2: Calculate Straddle Value.}
\begin{align*}
    \text{Straddle} &= C + P \\
    &= 4.00 + 5.80 \\
    &= \mathbf{9.80}
\end{align*}

\subsection{Exercise 5: Butterfly Spread Breakevens}
\textbf{Scenario:} A payoff diagram shows strikes at 310, 330, and 350. The max loss line is at -\$7.00.

\textbf{Solution:}
\begin{itemize}
    \item \textbf{Net Debit (Max Loss):} \$7.00
    \item \textbf{Lower Strike ($K_L$):} \$310
    \item \textbf{Upper Strike ($K_H$):} \$350
\end{itemize}

The breakeven points are:
\begin{align*}
    \text{Lower BE} &= 310 + 7.00 = \$317 \\
    \text{Upper BE} &= 350 - 7.00 = \mathbf{\$343}
\end{align*}

\end{document}