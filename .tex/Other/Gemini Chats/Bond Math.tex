\documentclass{article}
\usepackage{amsmath, amssymb, geometry, hyperref, xcolor, graphicx, array, booktabs}
\geometry{a4paper, margin=1in}

\title{Comprehensive Study Guide: Treasury Bond Mechanics, Risk, and Strategy}
\author{Generated by Gemini 3 Pro}
\date{\today}

\begin{document}

\maketitle
\tableofcontents

\newpage

\section{Phase 1: Conceptual Foundations}

\subsection{The Treasury Ecosystem}
The U.S. Department of the Treasury issues debt securities to fund government operations. The primary distinction between these instruments lies in their maturity and interest payment structure.

\begin{itemize}
    \item \textbf{Treasury Bills (T-Bills):} Short-term securities maturing in one year or less (e.g., 1, 3, 6, 12 months). They are \textit{zero-coupon} instruments sold at a discount to par value. The return is generated solely by the difference between the purchase price and the face value at maturity.
    \item \textbf{Treasury Notes (T-Notes):} Medium-term securities maturing between 2 and 10 years. They pay a fixed coupon semi-annually.
    \item \textbf{Treasury Bonds (T-Bonds):} Long-term securities maturing in 20 or 30 years. Like notes, they pay fixed semi-annual coupons but possess significantly higher duration risk.
\end{itemize}

\subsection{The Core Mechanism: Price and Yield}
A fundamental axiom of fixed-income investing is the inverse relationship between price and yield.
\begin{itemize}
    \item \textbf{If Interest Rates Rise:} New bonds are issued with higher coupons. Existing bonds with lower coupons become less attractive, forcing their market price \textbf{down} to align their yield with current rates.
    \item \textbf{If Interest Rates Fall:} Existing bonds with higher coupons become more valuable, driving their market price \textbf{up}.
\end{itemize}

\subsection{The Yield Curve}
The yield curve plots the yields of Treasury securities against their maturities.
\begin{itemize}
    \item \textbf{Normal Curve:} Upward sloping. Investors demand higher yields for locking up capital for longer periods (e.g., 30-year yield $>$ 1-month yield).
    \item \textbf{Inverted Curve:} Downward sloping or "humped." Short-term yields exceed long-term yields (e.g., 1-month yield $>$ 10-year yield). This anomaly often signals an impending economic slowdown or recession.
    \item \textbf{Market Implication:} An inversion suggests investors expect the Federal Reserve to cut interest rates in the future. They rush to buy long-term bonds to lock in yields before they fall, driving long-term prices up and yields down.
\end{itemize}

\section{Phase 2: The Core Mathematics of Risk}

\subsection{Duration: The Linear Approximation}
Duration is the primary measure of a bond's sensitivity to interest rate changes. It represents the approximate percentage change in a bond's price for a $1\%$ change in yield.

The linear approximation formula is:
\[
\frac{\Delta P}{P} \approx -D \times \Delta y
\]
Where:
\begin{itemize}
    \item $\Delta P$ is the change in price.
    \item $P$ is the current price.
    \item $D$ is the Modified Duration (in years).
    \item $\Delta y$ is the change in yield (in decimal form).
\end{itemize}

\textbf{Example:} Consider a 30-year bond with a duration of roughly 18 years. If rates fall by $1\%$ ($0.01$):
\[
\%\Delta P \approx -18 \times (-0.01) = +18\%
\]
Conversely, if rates rise by $1\%$, the bond loses approximately $18\%$ of its value.

\subsection{Convexity: The Curvature Correction}
The user correctly identified that strict linearity is an assumption. The actual relationship between price and yield is a curve, not a straight line. This curvature is called \textbf{Convexity}.

\begin{itemize}
    \item \textbf{The Flaw of Linearity:} Duration (the tangent line) accurately predicts price changes for small yield shifts. For large shifts (e.g., 200+ basis points), duration underestimates the price rise when rates fall and overestimates the price drop when rates rise.
    \item \textbf{Positive Convexity:} Standard Treasuries exhibit positive convexity. This acts as a "cushion":
    \begin{itemize}
        \item When yields rise, prices fall \textit{less} than duration predicts.
        \item When yields fall, prices rise \textit{more} than duration predicts.
    \end{itemize}
\end{itemize}

\section{Phase 3: Deep Dives \& Nuances}

\subsection{Yield to Maturity vs. Total Return}
A critical distinction exists between the yield an investor sees (Income) and the total money made (Total Return).

\textbf{The Investor's Dilemma:} Why buy a 10-year note yielding $3.5\%$ when a 6-month T-Bill yields $4.0\%$?

\textbf{Answer:} The investor is speculating on capital appreciation driven by duration.
\begin{align*}
    \text{Total Return} &= \text{Income (Coupons)} + \text{Capital Gain/Loss (Price Change)}
\end{align*}

\paragraph{Scenario: Betting on Rate Cuts}
If an investor expects a recession and subsequent rate cuts, they utilize the high duration of long-term bonds.
\begin{itemize}
    \item \textbf{T-Bill Strategy:} Holds value, pays steady 4\% annualized. Minimal price fluctuation.
    \item \textbf{Long Bond Strategy:} Pays lower 3.5\% income, but if rates drop by $1\%$, the price may jump $9\%$ (assuming duration of 9).
\end{itemize}

\textbf{Comparative Analysis (6-Month Horizon):}
\begin{center}
\begin{tabular}{l c c c}
\toprule
\textbf{Strategy} & \textbf{Income (Yield)} & \textbf{Price Gain (Duration Effect)} & \textbf{Total Return} \\
\midrule
T-Bill (Short Duration) & $\sim 2.0\%$ & $\sim 0\%$ & $\sim 2.0\%$ \\
Long Bond (High Duration) & $\sim 1.75\%$ & $+9.0\%$ & \textbf{$\sim 10.75\%$} \\
\bottomrule
\end{tabular}
\end{center}
The long bond investor sacrifices yield for the potential of significant price appreciation.

\section{Phase 4: Practical Applications \& Case Studies}

\subsection{Case Study 1: The 2022 Rate Hike Cycle (Duration Risk Realized)}
Historically, the Federal Reserve rarely cuts rates by 100bps instantly. However, in 2022, they raised rates aggressively (over 400bps in a year). This serves as a reverse example of duration power.

\textbf{Scenario:}
\begin{itemize}
    \item \textbf{Jan 2022:} Investor buys 10-Year Note at $1.63\%$ yield. Duration $\approx 9$.
    \item \textbf{Jan 2023:} Rates rise; new 10-Year Note yields $3.50\%$.
    \item \textbf{Yield Change ($\Delta y$):} $3.50\% - 1.63\% = +1.87\%$.
\end{itemize}

\textbf{Linear Calculation:}
\[
\%\Delta P \approx -9 \times 0.0187 = -16.83\%
\]
\textbf{Actual Market Outcome:} The bond price fell roughly $14.5\%$ to $15\%$. The discrepancy (loss was less severe than predicted) is due to \textbf{convexity}. Despite the convexity cushion, the total return was deeply negative ($\approx -13\%$), dwarfing the small coupon income.

\subsection{Case Study 2: The Collapse of Silicon Valley Bank (SVB)}
The collapse of SVB in 2023 is a textbook example of mismanaged duration risk and accounting classifications.

\subsubsection{The Setup}
\begin{enumerate}
    \item \textbf{Asset Accumulation:} SVB saw deposits triple during 2020-2021.
    \item \textbf{The Trade:} To generate yield in a low-rate environment, they purchased billions in long-dated Treasuries and Mortgage-Backed Securities (MBS). These assets have high duration.
    \item \textbf{The Accounting Shield:} SVB classified $\sim\$91$ billion of these assets as \textbf{Held-to-Maturity (HTM)}.
    \begin{itemize}
        \item \textit{Available-for-Sale (AFS):} Marked-to-market. Gains/losses appear on the balance sheet.
        \item \textit{Held-to-Maturity (HTM):} Carried at amortized cost. Unrealized losses are hidden from regulatory capital calculations unless sold.
    \end{itemize}
\end{enumerate}

\subsubsection{The Unwinding}
As the Fed hiked rates in 2022, the value of SVB's long-term bonds plummeted (via the duration mechanism).
\begin{itemize}
    \item By Dec 2022, the HTM portfolio had \textbf{unrealized losses of $\sim\$15$ billion}.
    \item SVB's total equity capital was only $\sim\$16$ billion.
    \item \textbf{Insolvency:} The hidden losses effectively wiped out the bank's equity.
\end{itemize}

When a liquidity crunch forced SVB to sell assets to cover withdrawals, they could no longer hide behind HTM accounting. The realization of losses triggered a panic (bank run), causing the bank's collapse.

\end{document}