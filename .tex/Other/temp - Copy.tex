\documentclass{article}
\usepackage{amsmath}
\usepackage{amssymb}
\usepackage[margin=1in]{geometry}

\title{Homework 11: Gas Laws and Kinetic Molecular Theory}
\author{Tashfeen Omran}
\date{October 2025}

\begin{document}

\maketitle

\part{Comprehensive Introduction, Context, and Prerequisites}

\section{Core Concepts: The Gaseous State of Matter}
This chapter explores the behavior of gases, one of the fundamental states of matter. Unlike solids and liquids, gases have unique properties: they are highly compressible, they expand to fill the entire volume of their container, and they have very low densities. The study of gases involves understanding the relationship between four key measurable properties:
\begin{itemize}
    \item \textbf{Pressure (P):} Defined as the force exerted by the gas per unit area. It arises from the countless collisions of gas particles with the walls of their container. Common units include atmospheres (atm), millimeters of mercury (mmHg), torr, and pascals (Pa).
    \item \textbf{Volume (V):} The amount of space the gas occupies. Since a gas fills its container, the volume of the gas is the volume of the container. Common units are liters (L) and milliliters (mL).
    \item \textbf{Temperature (T):} A measure of the average kinetic energy of the gas particles. Higher temperatures mean particles are moving, on average, more rapidly. For all gas law calculations, temperature \textbf{must} be expressed in Kelvin (K), the absolute temperature scale.
    \item \textbf{Amount (n):} The quantity of gas, measured in moles (mol).
\end{itemize}
The laws in this chapter are mathematical models that describe how these four variables are interrelated.

\subsection{The Empirical Gas Laws}
These laws were developed from experimental observations and describe the relationship between two variables while the other two are held constant.
\begin{itemize}
    \item \textbf{Boyle's Law (constant n, T):} Describes the inverse relationship between pressure and volume. As pressure on a gas increases, its volume decreases proportionally.
    \item \textbf{Charles's Law (constant n, P):} Describes the direct relationship between volume and absolute temperature. As the temperature of a gas increases, its volume increases proportionally.
    \item \textbf{Gay-Lussac's Law (constant n, V):} Describes the direct relationship between pressure and absolute temperature. As the temperature of a gas increases, its pressure increases proportionally.
    \item \textbf{Avogadro's Law (constant P, T):} Describes the direct relationship between volume and the amount of gas. Equal volumes of all gases, at the same temperature and pressure, have the same number of moles.
\end{itemize}

\subsection{The Ideal Gas Law}
The empirical laws can be synthesized into a single, more powerful equation called the \textbf{Ideal Gas Law}. This law describes the state of a hypothetical "ideal" gas. An ideal gas is a theoretical construct whose particles are assumed to have no volume and no intermolecular attractive forces. This model works remarkably well for real gases under most conditions (low pressure and high temperature). The law relates all four variables (P, V, T, n) through a proportionality constant, R, known as the ideal gas constant.

\subsection{The Kinetic Molecular Theory of Gases (KMT)}
KMT is a scientific model that explains the macroscopic behavior of gases (as described by the gas laws) by considering the behavior of the gas particles at the microscopic (molecular) level. The core postulates of KMT are:
\begin{enumerate}
    \item Gases consist of a large number of particles (atoms or molecules) that are in continuous, random motion.
    \item The volume of the individual gas particles is negligible compared to the total volume of the container.
    \item The attractive and repulsive forces between gas particles are negligible.
    \item Collisions between gas particles and with the container walls are perfectly elastic (i.e., kinetic energy is conserved).
    \item The average kinetic energy of the gas particles is directly proportional to the absolute temperature (in Kelvin).
\end{enumerate}
These postulates explain why pressure increases with temperature (particles move faster and hit walls harder) and why lighter gases move faster than heavier gases at the same temperature (to have the same kinetic energy, a lighter particle must have a higher velocity).

\section{Intuition and Derivation}
The gas laws are not arbitrary; they are derived from the physical behavior of particles.
\begin{itemize}
    \item \textbf{Boyle's Law (P $\propto$ 1/V):} Imagine particles in a cylinder with a piston. If you compress the cylinder to half its volume, the particles are now confined to a smaller space. They will collide with the walls twice as often, leading to double the pressure.
    \item \textbf{Charles's Law (V $\propto$ T):} Now imagine heating the cylinder. The particles gain kinetic energy and move faster. They strike the piston with more force and more frequently. To keep the pressure constant (i.e., the force per area on the piston), the piston must move outward, increasing the volume.
    \item \textbf{Graham's Law of Effusion:} KMT states that at a given temperature T, all gases have the same average kinetic energy ($\overline{KE}$). The formula for kinetic energy is $KE = \frac{1}{2}mv^2$. Therefore, for two different gases, A and B, at the same temperature:
    \[ \overline{KE}_A = \overline{KE}_B \]
    \[ \frac{1}{2} M_A (u_{\text{rms,A}})^2 = \frac{1}{2} M_B (u_{\text{rms,B}})^2 \]
    where M is the molar mass and $u_{\text{rms}}$ is the root-mean-square speed. Rearranging this equation to compare the speeds gives Graham's Law:
    \[ \frac{u_{\text{rms,A}}}{u_{\text{rms,B}}} = \sqrt{\frac{M_B}{M_A}} \]
    This shows that the speed of a gas is inversely proportional to the square root of its molar mass. Lighter gases move faster.
\end{itemize}

\section{Historical Context and Motivation}
The study of gases was a central part of the Scientific Revolution in the 17th and 18th centuries. Before this, air was considered one of the classical "elements," but its physical properties were not understood quantitatively. The invention of the barometer by Evangelista Torricelli in 1643 allowed for the measurement of atmospheric pressure, opening the door for systematic investigation.

Robert Boyle, an Irish scientist, used a J-shaped tube to study the relationship between the pressure and volume of trapped air, leading to Boyle's Law in 1662. Over a century later, French scientists Jacques Charles and Joseph Louis Gay-Lussac investigated how the volume and pressure of gases change with temperature, spurred by the new technology of hot air balloons. Their work established the direct proportionality between volume/pressure and absolute temperature. Around the same time, Amedeo Avogadro proposed his hypothesis connecting the volume of a gas to the number of particles, a crucial step in developing the concept of the mole. These empirical findings provided the foundation for the Ideal Gas Law and the later development of the Kinetic Molecular Theory by physicists like James Clerk Maxwell and Ludwig Boltzmann, who sought to explain these macroscopic laws from the fundamental principles of particle motion.

\section{Key Formulas}
\begin{itemize}
    \item \textbf{Temperature Conversion:} $T_K = T_{^{\circ}C} + 273.15$
    \item \textbf{Boyle's Law:} $P_1V_1 = P_2V_2$
    \item \textbf{Charles's Law:} $\frac{V_1}{T_1} = \frac{V_2}{T_2}$
    \item \textbf{Gay-Lussac's Law:} $\frac{P_1}{T_1} = \frac{P_2}{T_2}$
    \item \textbf{Combined Gas Law (constant n):} $\frac{P_1V_1}{T_1} = \frac{P_2V_2}{T_2}$
    \item \textbf{Ideal Gas Law:} $PV = nRT$
        \begin{itemize}
            \item $R = 0.08206 \frac{\text{L} \cdot \text{atm}}{\text{mol} \cdot \text{K}}$ (most common in chemistry)
            \item $R = 8.314 \frac{\text{J}}{\text{mol} \cdot \text{K}}$
        \end{itemize}
    \item \textbf{Molar Mass and Density from Ideal Gas Law:}
        \[ \mathcal{M} = \frac{dRT}{P} \quad \text{or} \quad d = \frac{P\mathcal{M}}{RT} \]
        where $\mathcal{M}$ is molar mass (g/mol) and $d$ is density (g/L).
    \item \textbf{Dalton's Law of Partial Pressures:} $P_{\text{total}} = P_A + P_B + P_C + \dots$
    \item \textbf{Mole Fraction ($\chi$):} $\chi_A = \frac{n_A}{n_{\text{total}}} = \frac{P_A}{P_{\text{total}}}$
    \item \textbf{Graham's Law of Effusion/Diffusion:} $\frac{\text{rate}_A}{\text{rate}_B} = \frac{u_{\text{rms},A}}{u_{\text{rms},B}} = \sqrt{\frac{\mathcal{M}_B}{\mathcal{M}_A}}$
\end{itemize}

\section{Prerequisites}
To succeed in this chapter, you must be proficient in the following:
\begin{enumerate}
    \item \textbf{Algebra:} Solving multi-variable equations for an unknown, working with proportions and ratios.
    \item \textbf{Unit Conversions:} Converting between different units of pressure (atm, mmHg, torr) and temperature (Celsius to Kelvin).
    \item \textbf{Stoichiometry:} Calculating molar mass, converting between grams and moles, and using mole ratios from balanced chemical equations.
    \item \textbf{Scientific Notation:} Working with very large and very small numbers.
\end{enumerate}

\part{Detailed Homework Solutions}
\textit{Note: The provided materials include several sets of problems that function as the homework assignment. Each problem is solved below in full detail.}

\subsection{Worksheet: Kinetic Molecular Theory}

\subsubsection{Problem 1}
The velocity distribution below details the same molecule moving at 25$^{\circ}$C, 50$^{\circ}$C, and 75$^{\circ}$C.

\textbf{Analysis of the Graph:}
According to the Kinetic Molecular Theory, temperature is a measure of the average kinetic energy of the molecules. Higher temperature means higher average kinetic energy, which in turn means higher average molecular speed. A distribution curve for a higher temperature will be broader (a wider range of speeds) and its peak will be shifted to the right (a higher most probable speed).
\begin{itemize}
    \item Curve A: Tallest and narrowest peak, located at the lowest molecular velocity. This corresponds to the \textbf{lowest temperature (25$^{\circ}$C)}.
    \item Curve B: Intermediate height, width, and peak position. This corresponds to the \textbf{middle temperature (50$^{\circ}$C)}.
    \item Curve C: Shortest and broadest peak, located at the highest molecular velocity. This corresponds to the \textbf{highest temperature (75$^{\circ}$C)}.
\end{itemize}

\textbf{a) Which curve represents the molecule at 50$^{\circ}$C?}
\begin{description}
    \item[Solution:] Curve B represents the intermediate temperature.
    \item[Final Answer:] Curve B
\end{description}

\textbf{b) Which curve represents the molecule at 75$^{\circ}$C?}
\begin{description}
    \item[Solution:] Curve C is the broadest and shifted furthest to the right, representing the highest temperature and fastest movement.
    \item[Final Answer:] Curve C
\end{description}

\textbf{c) Which curve represents the molecule at 25$^{\circ}$C?}
\begin{description}
    \item[Solution:] Curve A is the narrowest and shifted furthest to the left, representing the lowest temperature and slowest movement.
    \item[Final Answer:] Curve A
\end{description}

\textbf{d) Which curve represents the molecule moving at the highest speed?}
\begin{description}
    \item[Solution:] The highest speed corresponds to the highest temperature.
    \item[Final Answer:] Curve C
\end{description}

\textbf{e) Which curve represents the molecule moving at the lowest speed?}
\begin{description}
    \item[Solution:] The lowest speed corresponds to the lowest temperature.
    \item[Final Answer:] Curve A
\end{description}

\subsubsection{Problem 2}
Consider the graph below comparing the velocity distribution of PH$_3$ and Kr at the same temperature.

\textbf{Analysis of the Graph:}
At the same temperature, molecules of different gases have the same average kinetic energy. Since $KE = \frac{1}{2}mv^2$, for a heavier molecule (larger molar mass, $\mathcal{M}$) to have the same KE, it must have a lower average speed. Therefore, the distribution curve for the heavier gas will have its peak at a lower velocity.
\begin{itemize}
    \item Molar Mass of Kr: $\approx 83.80$ g/mol
    \item Molar Mass of PH$_3$: $30.97 + 3(1.01) \approx 34.00$ g/mol
\end{itemize}
Krypton (Kr) is significantly heavier than phosphine (PH$_3$).
\begin{itemize}
    \item Curve A: Peak is at a lower velocity. This must be the heavier gas, \textbf{Kr}.
    \item Curve B: Peak is at a higher velocity. This must be the lighter gas, \textbf{PH$_3$}.
\end{itemize}

\textbf{a) Identify which curve represents Kr.}
\begin{description}
    \item[Solution:] Kr has a larger molar mass, so it will have a lower average speed.
    \item[Final Answer:] Curve A
\end{description}

\textbf{b) Identify which curve represents PH$_3$.}
\begin{description}
    \item[Solution:] PH$_3$ has a smaller molar mass, so it will have a higher average speed.
    \item[Final Answer:] Curve B
\end{description}

\textbf{c) Which curve represents the gas that diffuses more slowly?}
\begin{description}
    \item[Solution:] According to Graham's Law, the rate of diffusion is inversely proportional to the square root of the molar mass. The heavier gas diffuses more slowly. Kr is heavier.
    \item[Final Answer:] Curve A
\end{description}

\textbf{d) Which curve represents a higher velocity?}
\begin{description}
    \item[Solution:] The lighter gas, PH$_3$, has the higher average velocity.
    \item[Final Answer:] Curve B
\end{description}

\textbf{e) Which curve represents the gas with a lower molar mass?}
\begin{description}
    \item[Solution:] The gas with the lower molar mass is PH$_3$.
    \item[Final Answer:] Curve B
\end{description}

\subsubsection{Problem 3}
The velocity distribution below details four different molecules (A, B, C, and D) moving at the same temperature.

\textbf{Analysis of the Graph:}
Similar to Problem 2, at the same temperature, heavier molecules move slower. The peak of the distribution curve indicates the most probable speed. A peak further to the left means a lower speed and therefore a higher molar mass.
Order of speeds: D $>$ C $>$ B $>$ A.
Therefore, order of molar masses: A $>$ B $>$ C $>$ D.

\textbf{a) Which gas has the largest molar mass?}
\begin{description}
    \item[Solution:] Gas A has the distribution peak at the lowest speed, so it must have the largest molar mass.
    \item[Final Answer:] Gas A
\end{description}

\textbf{b) Which gas effuses most rapidly?}
\begin{description}
    \item[Solution:] The gas that effuses most rapidly is the one with the highest speed, which corresponds to the lowest molar mass.
    \item[Final Answer:] Gas D
\end{description}

\textbf{c) True or false: Gas "A" has a smaller molar mass than gas "D".}
\begin{description}
    \item[Solution:] Gas A moves slowest, so it has the largest molar mass. Gas D moves fastest, so it has the smallest molar mass. Therefore, the statement is false.
    \item[Final Answer:] False
\end{description}

\subsubsection{Problem 4}
An unknown gas diffuses 1.12 times faster than argon (Ar) through a porous membrane. Calculate the molar mass of the unknown gas.

\textbf{Setup:}
Let the unknown gas be 'x'. We are given:
\[ \frac{\text{rate}_x}{\text{rate}_{\text{Ar}}} = 1.12 \]
From Graham's Law:
\[ \frac{\text{rate}_x}{\text{rate}_{\text{Ar}}} = \sqrt{\frac{\mathcal{M}_{\text{Ar}}}{\mathcal{M}_x}} \]
The molar mass of Argon (Ar) is $\mathcal{M}_{\text{Ar}} = 39.95$ g/mol.

\textbf{Solution:}
\begin{align*}
    1.12 &= \sqrt{\frac{39.95 \text{ g/mol}}{\mathcal{M}_x}} \\
    (1.12)^2 &= \frac{39.95 \text{ g/mol}}{\mathcal{M}_x} \\
    1.2544 &= \frac{39.95 \text{ g/mol}}{\mathcal{M}_x} \\
    \mathcal{M}_x &= \frac{39.95 \text{ g/mol}}{1.2544} \\
    \mathcal{M}_x &\approx 31.85 \text{ g/mol}
\end{align*}

\textbf{Final Answer:} The molar mass of the unknown gas is approximately 31.9 g/mol.

\subsubsection{Problem 5}
Calculate the diffusion rate of hydrobromic acid (HBr) relative to Xenon (Xe).

\textbf{Setup:}
We need to find the ratio $\frac{\text{rate}_{\text{HBr}}}{\text{rate}_{\text{Xe}}}$.
Molar masses:
\begin{itemize}
    \item $\mathcal{M}_{\text{HBr}} \approx 1.01 + 79.90 = 80.91$ g/mol
    \item $\mathcal{M}_{\text{Xe}} \approx 131.29$ g/mol
\end{itemize}
Using Graham's Law:
\[ \frac{\text{rate}_{\text{HBr}}}{\text{rate}_{\text{Xe}}} = \sqrt{\frac{\mathcal{M}_{\text{Xe}}}{\mathcal{M}_{\text{HBr}}}} \]

\textbf{Solution:}
\[ \frac{\text{rate}_{\text{HBr}}}{\text{rate}_{\text{Xe}}} = \sqrt{\frac{131.29 \text{ g/mol}}{80.91 \text{ g/mol}}} = \sqrt{1.6226} \approx 1.27 \]

\textbf{Final Answer:} HBr diffuses approximately 1.27 times faster than Xe.

\subsubsection{Problem 6}
Arrange the following species in order of increasing rate of effusion: Cl$_2$, Ar, HI.

\textbf{Setup:}
The rate of effusion is inversely proportional to the square root of the molar mass. A higher molar mass means a lower rate of effusion. We need to calculate the molar masses and then order them from largest to smallest, which will correspond to the order of increasing effusion rate.
\begin{itemize}
    \item $\mathcal{M}_{\text{HI}} \approx 1.01 + 126.90 = 127.91$ g/mol (Slowest)
    \item $\mathcal{M}_{\text{Cl}_2} \approx 2 \times 35.45 = 70.90$ g/mol (Intermediate)
    \item $\mathcal{M}_{\text{Ar}} \approx 39.95$ g/mol (Fastest)
\end{itemize}

\textbf{Solution:}
The order of increasing molar mass is Ar $<$ Cl$_2$ $<$ HI.
Therefore, the order of increasing rate of effusion is the reverse.

\textbf{Final Answer:} HI $<$ Cl$_2$ $<$ Ar

\subsubsection{Problem 7}
It takes 54 seconds for N$_2$O gas to effuse from a container. How much time would it take for the same amount of I$_2$ gas to effuse under the same conditions?

\textbf{Setup:}
The time it takes for a gas to effuse is inversely proportional to its rate. Therefore, we can write a version of Graham's Law for time:
\[ \frac{\text{time}_A}{\text{time}_B} = \frac{\text{rate}_B}{\text{rate}_A} = \sqrt{\frac{\mathcal{M}_A}{\mathcal{M}_B}} \]
Let A = I$_2$ and B = N$_2$O. We need to find time$_{I_2}$.
\begin{itemize}
    \item $\mathcal{M}_{\text{I}_2} \approx 2 \times 126.90 = 253.8$ g/mol
    \item $\mathcal{M}_{\text{N}_2\text{O}} \approx 2(14.01) + 16.00 = 44.02$ g/mol
    \item time$_{\text{N}_2\text{O}} = 54$ s
\end{itemize}

\textbf{Solution:}
\begin{align*}
    \frac{\text{time}_{\text{I}_2}}{\text{time}_{\text{N}_2\text{O}}} &= \sqrt{\frac{\mathcal{M}_{\text{I}_2}}{\mathcal{M}_{\text{N}_2\text{O}}}} \\
    \frac{\text{time}_{\text{I}_2}}{54 \text{ s}} &= \sqrt{\frac{253.8 \text{ g/mol}}{44.02 \text{ g/mol}}} \\
    \frac{\text{time}_{\text{I}_2}}{54 \text{ s}} &= \sqrt{5.7655} \approx 2.40 \\
    \text{time}_{\text{I}_2} &= 2.40 \times 54 \text{ s} \approx 129.6 \text{ s}
\end{align*}

\textbf{Final Answer:} It would take approximately 130 seconds for the I$_2$ gas to effuse.

\subsection{Worksheet: Molar Mass, Density, Gas Stoichiometry, Gas Mixtures}

\subsubsection{Problem 1}
Determine the molar mass of an unknown gas in a container at -50.0$^{\circ}$C and 6.00 atm pressure. The density of this gas is 14.5 g/L.

\textbf{Setup:}
We use the formula relating molar mass ($\mathcal{M}$) and density ($d$):
\[ \mathcal{M} = \frac{dRT}{P} \]
Given:
\begin{itemize}
    \item $d = 14.5$ g/L
    \item $P = 6.00$ atm
    \item $T = -50.0^{\circ}\text{C} + 273.15 = 223.15$ K
    \item $R = 0.08206 \frac{\text{L} \cdot \text{atm}}{\text{mol} \cdot \text{K}}$
\end{itemize}

\textbf{Solution:}
\[ \mathcal{M} = \frac{(14.5 \text{ g/L})(0.08206 \frac{\text{L} \cdot \text{atm}}{\text{mol} \cdot \text{K}})(223.15 \text{ K})}{6.00 \text{ atm}} \]
\[ \mathcal{M} = \frac{265.55}{6.00} \text{ g/mol} \approx 44.26 \text{ g/mol} \]

\textbf{Final Answer:} The molar mass is approximately 44.3 g/mol.

\subsubsection{Problem 2}
At STP, what is the density of a gas (in g/L) if the molar mass of this gas is 66.1 g/mole?

\textbf{Setup:}
Standard Temperature and Pressure (STP) is defined as $T = 0^{\circ}\text{C} = 273.15$ K and $P = 1.00$ atm. We use the rearranged density formula:
\[ d = \frac{P\mathcal{M}}{RT} \]
Given:
\begin{itemize}
    \item $\mathcal{M} = 66.1$ g/mol
    \item $P = 1.00$ atm
    \item $T = 273.15$ K
    \item $R = 0.08206 \frac{\text{L} \cdot \text{atm}}{\text{mol} \cdot \text{K}}$
\end{itemize}

\textbf{Solution:}
\[ d = \frac{(1.00 \text{ atm})(66.1 \text{ g/mol})}{(0.08206 \frac{\text{L} \cdot \text{atm}}{\text{mol} \cdot \text{K}})(273.15 \text{ K})} \]
\[ d = \frac{66.1}{22.414} \text{ g/L} \approx 2.949 \text{ g/L} \]

\textbf{Final Answer:} The density is approximately 2.95 g/L.

\subsubsection{Problem 3}
Calculate the volume of 110.4 g CH$_4$ gas (in L) at STP.

\textbf{Setup:}
This is a two-step problem. First, convert the mass of methane (CH$_4$) to moles. Second, use the Ideal Gas Law to find the volume.
Molar mass of CH$_4$ = $12.01 + 4(1.01) = 16.05$ g/mol.
STP conditions: $P=1.00$ atm, $T=273.15$ K.

\textbf{Solution:}
\begin{enumerate}
    \item \textbf{Calculate moles (n):}
    \[ n = 110.4 \text{ g CH}_4 \times \frac{1 \text{ mol CH}_4}{16.05 \text{ g CH}_4} \approx 6.879 \text{ mol CH}_4 \]
    \item \textbf{Calculate volume (V) using PV=nRT:}
    \[ V = \frac{nRT}{P} = \frac{(6.879 \text{ mol})(0.08206 \frac{\text{L} \cdot \text{atm}}{\text{mol} \cdot \text{K}})(273.15 \text{ K})}{1.00 \text{ atm}} \]
    \[ V \approx 154.2 \text{ L} \]
\end{enumerate}

\textit{Alternative Method using Molar Volume at STP:} At STP, 1 mole of any ideal gas occupies 22.4 L.
\[ V = 6.879 \text{ mol} \times \frac{22.4 \text{ L}}{1 \text{ mol}} \approx 154.1 \text{ L} \]

\textbf{Final Answer:} The volume is approximately 154 L.

\subsubsection{Problem 4}
A 15.0 L sample of an unknown diatomic gas has a mass of 21.45 g at a pressure of 1.00 atm at 273.15 K. What is the molar mass of the diatomic gas? Which diatomic gas is it?

\textbf{Setup:}
We can calculate the molar mass using the ideal gas law in the form $\mathcal{M} = \frac{mRT}{PV}$, where $m$ is mass.
Given:
\begin{itemize}
    \item $m = 21.45$ g
    \item $V = 15.0$ L
    \item $P = 1.00$ atm
    \item $T = 273.15$ K (STP conditions)
    \item $R = 0.08206 \frac{\text{L} \cdot \text{atm}}{\text{mol} \cdot \text{K}}$
\end{itemize}

\textbf{Solution:}
\begin{align*}
    \mathcal{M} &= \frac{(21.45 \text{ g})(0.08206 \frac{\text{L} \cdot \text{atm}}{\text{mol} \cdot \text{K}})(273.15 \text{ K})}{(1.00 \text{ atm})(15.0 \text{ L})} \\
    \mathcal{M} &= \frac{480.8}{15.0} \text{ g/mol} \approx 32.05 \text{ g/mol}
\end{align*}
The molar mass is approximately 32.0 g/mol. A common diatomic gas with this molar mass is oxygen, O$_2$ ($\mathcal{M} \approx 2 \times 16.00 = 32.00$ g/mol).

\textbf{Final Answer:} The molar mass is $\approx 32.1$ g/mol. The diatomic gas is likely Oxygen (O$_2$).

\subsubsection{Problem 5}
If 15.0 moles of CH$_4$ react with oxygen, what volume of CO$_2$ (in L) is produced at 23.0$^{\circ}$C and 0.985 atm?
Reaction: CH$_4$(g) + 2O$_2$(g) $\rightarrow$ CO$_2$(g) + 2H$_2$O(l)

\textbf{Setup:}
This is a stoichiometry problem.
\begin{enumerate}
    \item Use the mole ratio from the balanced equation to find moles of CO$_2$ produced.
    \item Use the Ideal Gas Law to find the volume of the CO$_2$.
\end{enumerate}
Given conditions for the product: $T = 23.0^{\circ}\text{C} = 296.15$ K, $P = 0.985$ atm.

\textbf{Solution:}
\begin{enumerate}
    \item \textbf{Calculate moles of CO$_2$:} The mole ratio of CH$_4$ to CO$_2$ is 1:1.
    \[ n_{\text{CO}_2} = 15.0 \text{ mol CH}_4 \times \frac{1 \text{ mol CO}_2}{1 \text{ mol CH}_4} = 15.0 \text{ mol CO}_2 \]
    \item \textbf{Calculate volume of CO$_2$:}
    \[ V = \frac{nRT}{P} = \frac{(15.0 \text{ mol})(0.08206 \frac{\text{L} \cdot \text{atm}}{\text{mol} \cdot \text{K}})(296.15 \text{ K})}{0.985 \text{ atm}} \]
    \[ V = \frac{364.5}{0.985} \text{ L} \approx 370.0 \text{ L} \]
\end{enumerate}

\textbf{Final Answer:} The volume of CO$_2$ produced is approximately 370 L.

\subsubsection{Problem 6}
What volume of NH$_3$ (in L) must be used at 244 torr and 35$^{\circ}$C to produce 2.3 kg of HCl gas?
Reaction: 2NH$_3$(g) + 3Cl$_2$(g) $\rightarrow$ N$_2$(g) + 6HCl(g)

\textbf{Setup:}
This is a gas stoichiometry problem working backwards from product to reactant.
\begin{enumerate}
    \item Convert mass of HCl to moles of HCl.
    \item Use the mole ratio to find the moles of NH$_3$ required.
    \item Use the Ideal Gas Law with the given conditions for NH$_3$ to find its volume.
\end{enumerate}
Given conditions for NH$_3$:
\begin{itemize}
    \item $P = 244 \text{ torr} \times \frac{1 \text{ atm}}{760 \text{ torr}} \approx 0.321$ atm
    \item $T = 35^{\circ}\text{C} + 273.15 = 308.15$ K
\end{itemize}
Molar mass of HCl $\approx 1.01 + 35.45 = 36.46$ g/mol.
Mass of HCl = $2.3 \text{ kg} = 2300$ g.

\textbf{Solution:}
\begin{enumerate}
    \item \textbf{Calculate moles of HCl:}
    \[ n_{\text{HCl}} = 2300 \text{ g HCl} \times \frac{1 \text{ mol HCl}}{36.46 \text{ g HCl}} \approx 63.08 \text{ mol HCl} \]
    \item \textbf{Calculate moles of NH$_3$:} The mole ratio of NH$_3$ to HCl is 2:6 or 1:3.
    \[ n_{\text{NH}_3} = 63.08 \text{ mol HCl} \times \frac{2 \text{ mol NH}_3}{6 \text{ mol HCl}} \approx 21.03 \text{ mol NH}_3 \]
    \item \textbf{Calculate volume of NH$_3$:}
    \[ V = \frac{nRT}{P} = \frac{(21.03 \text{ mol})(0.08206 \frac{\text{L} \cdot \text{atm}}{\text{mol} \cdot \text{K}})(308.15 \text{ K})}{0.321 \text{ atm}} \]
    \[ V = \frac{531.6}{0.321} \text{ L} \approx 1656 \text{ L} \approx 1.7 \times 10^3 \text{ L} \]
\end{enumerate}

\textbf{Final Answer:} Approximately $1.7 \times 10^3$ L of NH$_3$ must be used.

\subsubsection{Problem 7}
A cylinder of compressed natural gases has a volume of 20.0 L and contains 1813 g of CH$_4$ and 336 g of O$_2$. Answer the following questions about the gases.

\textbf{a) Calculate the partial pressure of each gas and the total pressure in the cylinder at 22.0$^{\circ}$C.}

\textbf{Setup:}
We need to calculate the moles of each gas, then use the Ideal Gas Law for each gas to find its partial pressure ($P_i = n_iRT/V$). The total pressure is the sum of the partial pressures.
Molar masses: CH$_4 \approx 16.05$ g/mol; O$_2 \approx 32.00$ g/mol.
Conditions:
\begin{itemize}
    \item $V = 20.0$ L
    \item $T = 22.0^{\circ}\text{C} + 273.15 = 295.15$ K
\end{itemize}

\textbf{Solution:}
\begin{enumerate}
    \item \textbf{Calculate moles of each gas:}
    \[ n_{\text{CH}_4} = 1813 \text{ g CH}_4 \times \frac{1 \text{ mol}}{16.05 \text{ g}} \approx 113.0 \text{ mol CH}_4 \]
    \[ n_{\text{O}_2} = 336 \text{ g O}_2 \times \frac{1 \text{ mol}}{32.00 \text{ g}} = 10.5 \text{ mol O}_2 \]
    \item \textbf{Calculate partial pressures:}
    \[ P_{\text{CH}_4} = \frac{n_{\text{CH}_4}RT}{V} = \frac{(113.0 \text{ mol})(0.08206)(295.15 \text{ K})}{20.0 \text{ L}} \approx 136.9 \text{ atm} \]
    \[ P_{\text{O}_2} = \frac{n_{\text{O}_2}RT}{V} = \frac{(10.5 \text{ mol})(0.08206)(295.15 \text{ K})}{20.0 \text{ L}} \approx 12.7 \text{ atm} \]
    \item \textbf{Calculate total pressure:}
    \[ P_{\text{total}} = P_{\text{CH}_4} + P_{\text{O}_2} = 136.9 \text{ atm} + 12.7 \text{ atm} = 149.6 \text{ atm} \]
\end{enumerate}

\textbf{Final Answer:} $P_{\text{CH}_4} \approx 137$ atm, $P_{\text{O}_2} \approx 12.7$ atm, $P_{\text{total}} \approx 150$ atm.

\textbf{b) Calculate the mole fraction of CH$_4$ within the cylinder.}

\textbf{Setup:}
Mole fraction $\chi_{\text{CH}_4}$ can be calculated in two ways: using the ratio of moles or the ratio of pressures.
\[ \chi_{\text{CH}_4} = \frac{n_{\text{CH}_4}}{n_{\text{total}}} = \frac{P_{\text{CH}_4}}{P_{\text{total}}} \]

\textbf{Solution (using moles):}
\begin{align*}
    n_{\text{total}} &= n_{\text{CH}_4} + n_{\text{O}_2} = 113.0 \text{ mol} + 10.5 \text{ mol} = 123.5 \text{ mol} \\
    \chi_{\text{CH}_4} &= \frac{113.0 \text{ mol}}{123.5 \text{ mol}} \approx 0.915
\end{align*}

\textbf{Solution (using pressures):}
\[ \chi_{\text{CH}_4} = \frac{136.9 \text{ atm}}{149.6 \text{ atm}} \approx 0.915 \]

\textbf{Final Answer:} The mole fraction of CH$_4$ is approximately 0.915.

\subsection{Worksheet: Gas Law Problems}

\subsubsection{Problem 1 (Gay-Lussac's Law)}
A gas sample in a sealed flask has a pressure of 792.84 torr at 21.26$^{\circ}$C. Assuming the volume of the flask cannot change, what is the pressure if the flask is heated to 75.35$^{\circ}$C?

\textbf{Setup:}
Volume and moles are constant. This is a pressure-temperature relationship (Gay-Lussac's Law).
\[ \frac{P_1}{T_1} = \frac{P_2}{T_2} \]
Given:
\begin{itemize}
    \item $P_1 = 792.84$ torr
    \item $T_1 = 21.26^{\circ}\text{C} + 273.15 = 294.41$ K
    \item $T_2 = 75.35^{\circ}\text{C} + 273.15 = 348.50$ K
\end{itemize}
We need to solve for $P_2$.

\textbf{Solution:}
\[ P_2 = P_1 \times \frac{T_2}{T_1} = 792.84 \text{ torr} \times \frac{348.50 \text{ K}}{294.41 \text{ K}} \]
\[ P_2 \approx 938.5 \text{ torr} \]

\textbf{Final Answer:} The new pressure is approximately 938.5 torr.

\subsubsection{Problem 2 (Boyle's Law)}
A gas sample in a syringe has a volume of 35.26 mL at 0.9854 atm. The plunger is depressed such that the volume is reduced to 22.33 mL. What is the new pressure, assuming no change in temperature?

\textbf{Setup:}
Temperature and moles are constant. This is a pressure-volume relationship (Boyle's Law).
\[ P_1V_1 = P_2V_2 \]
Given:
\begin{itemize}
    \item $P_1 = 0.9854$ atm
    \item $V_1 = 35.26$ mL
    \item $V_2 = 22.33$ mL
\end{itemize}
We need to solve for $P_2$.

\textbf{Solution:}
\[ P_2 = \frac{P_1V_1}{V_2} = \frac{(0.9854 \text{ atm})(35.26 \text{ mL})}{22.33 \text{ mL}} \]
\[ P_2 \approx 1.556 \text{ atm} \]

\textbf{Final Answer:} The new pressure is approximately 1.556 atm.

\subsubsection{Problem 3 (Charles's Law)}
A sample of nitrogen in a flexible container has a volume of 12.748 L at -24.36$^{\circ}$C. The temperature is increased to 8.97$^{\circ}$C. Assuming pressure remains constant, what is the new volume?

\textbf{Setup:}
Pressure and moles are constant. This is a volume-temperature relationship (Charles's Law).
\[ \frac{V_1}{T_1} = \frac{V_2}{T_2} \]
Given:
\begin{itemize}
    \item $V_1 = 12.748$ L
    \item $T_1 = -24.36^{\circ}\text{C} + 273.15 = 248.79$ K
    \item $T_2 = 8.97^{\circ}\text{C} + 273.15 = 282.12$ K
\end{itemize}
We need to solve for $V_2$.

\textbf{Solution:}
\[ V_2 = V_1 \times \frac{T_2}{T_1} = 12.748 \text{ L} \times \frac{282.12 \text{ K}}{248.79 \text{ K}} \]
\[ V_2 \approx 14.456 \text{ L} \]

\textbf{Final Answer:} The new volume is approximately 14.46 L.

\subsubsection{Problem 5 (Combined Gas Law)}
A 1.500 L balloon is filled with helium at 305.9 K and 745.9 mmHg. The balloon rises to an altitude where the pressure is 154.2 mmHg and the temperature is 250.6 K. What is the volume of the balloon at this altitude?

\textbf{Setup:}
The amount of gas (n) is constant, but P, V, and T all change. We use the Combined Gas Law.
\[ \frac{P_1V_1}{T_1} = \frac{P_2V_2}{T_2} \]
Given:
\begin{itemize}
    \item $P_1 = 745.9$ mmHg
    \item $V_1 = 1.500$ L
    \item $T_1 = 305.9$ K
    \item $P_2 = 154.2$ mmHg
    \item $T_2 = 250.6$ K
\end{itemize}
We need to solve for $V_2$.

\textbf{Solution:}
\[ V_2 = \frac{P_1V_1T_2}{T_1P_2} = \frac{(745.9 \text{ mmHg})(1.500 \text{ L})(250.6 \text{ K})}{(305.9 \text{ K})(154.2 \text{ mmHg})} \]
\[ V_2 = \frac{281140}{47169.8} \text{ L} \approx 5.96 \text{ L} \]

\textbf{Final Answer:} The new volume is approximately 5.96 L.

\part{In-Depth Analysis of Problems and Techniques}

\section{Problem Types and General Approach}
The homework problems can be categorized into several distinct types:

\begin{itemize}
    \item \textbf{Type 1: Kinetic Molecular Theory (KMT) Graph Interpretation}
        \begin{itemize}
            \item \textbf{Examples:} KMT Worksheet, Problems 1, 2, 3.
            \item \textbf{General Approach:} These problems test conceptual understanding. Remember two key rules:
                \begin{enumerate}
                    \item \textbf{Effect of Temperature:} For the *same gas*, a higher temperature results in a higher average speed. The distribution curve flattens, broadens, and its peak shifts to the right (higher velocity).
                    \item \textbf{Effect of Molar Mass:} At the *same temperature*, lighter gases (lower molar mass) have a higher average speed. The curve for the lighter gas will have its peak shifted to the right compared to the heavier gas.
                \end{enumerate}
        \end{itemize}

    \item \textbf{Type 2: Two-Variable Empirical Gas Laws}
        \begin{itemize}
            \item \textbf{Examples:} Gas Law Problems Worksheet, Problems 1 (Gay-Lussac), 2 (Boyle), 3 (Charles).
            \item \textbf{General Approach:} Identify which two variables are changing and which two are constant. Select the corresponding law ($P_1V_1=P_2V_2$, $V_1/T_1=V_2/T_2$, or $P_1/T_1=P_2/T_2$). The most critical step is to convert all temperatures to Kelvin. Then, algebraically solve for the unknown variable.
        \end{itemize}

    \item \textbf{Type 3: Combined Gas Law}
        \begin{itemize}
            \item \textbf{Example:} Gas Law Problems Worksheet, Problem 5.
            \item \textbf{General Approach:} Use this when P, V, and T are all changing for a fixed amount of gas (n is constant). The formula $\frac{P_1V_1}{T_1} = \frac{P_2V_2}{T_2}$ is a direct application. Ensure all units are consistent for P and V on both sides and that T is in Kelvin.
        \end{itemize}

    \item \textbf{Type 4: Ideal Gas Law (Single State)}
        \begin{itemize}
            \item \textbf{Examples:} Molar Mass Worksheet, Problems 1, 2, 4.
            \item \textbf{General Approach:} Use $PV=nRT$ when given three of the four variables and asked to find the fourth for a single set of conditions. Pay close attention to units to ensure they match the units of your chosen R constant (usually $0.08206 \frac{\text{L} \cdot \text{atm}}{\text{mol} \cdot \text{K}}$). This law is also used to find derived properties like molar mass ($\mathcal{M} = mRT/PV$) or density ($d = P\mathcal{M}/RT$).
        \end{itemize}

    \item \textbf{Type 5: Gas Stoichiometry}
        \begin{itemize}
            \item \textbf{Examples:} Molar Mass Worksheet, Problems 3, 5, 6.
            \item \textbf{General Approach:} This is a multi-step process combining stoichiometry with gas laws.
                \begin{enumerate}
                    \item Write and balance the chemical equation.
                    \item Convert the given quantity of reactant/product (e.g., grams) into moles.
                    \item Use the mole ratio from the balanced equation to find the moles of the desired substance.
                    \item Use the Ideal Gas Law ($V=nRT/P$) with the moles just calculated and the given conditions (P, T) to find the volume of the gaseous substance.
                \end{enumerate}
        \end{itemize}

    \item \textbf{Type 6: Dalton's Law and Gas Mixtures}
        \begin{itemize}
            \item \textbf{Example:} Molar Mass Worksheet, Problem 7.
            \item \textbf{General Approach:} Treat each gas in the mixture as if it occupies the entire volume by itself. You can calculate the partial pressure of each gas using $P_i=n_iRT/V$. The total pressure is the sum of these partial pressures. Mole fraction ($\chi_i$) can be found by dividing the moles of one gas by the total moles, or the partial pressure of one gas by the total pressure.
        \end{itemize}

    \item \textbf{Type 7: Graham's Law of Effusion/Diffusion}
        \begin{itemize}
            \item \textbf{Examples:} KMT Worksheet, Problems 4, 5, 6, 7.
            \item \textbf{General Approach:} These problems relate the rate of movement (or time of travel) of gases to their molar masses. The key is setting up the correct ratio: the rate is inversely proportional to the square root of the molar mass ($\text{rate}_A/\text{rate}_B = \sqrt{\mathcal{M}_B/\mathcal{M}_A}$), while the time is directly proportional ($\text{time}_A/\text{time}_B = \sqrt{\mathcal{M}_A/\mathcal{M}_B}$).
        \end{itemize}
\end{itemize}

\section{Key Algebraic and Calculus Manipulations}
No calculus is required for this topic, but specific algebraic manipulations are essential.
\begin{enumerate}
    \item \textbf{Absolute Temperature Conversion (Mandatory):} This is the most crucial manipulation. For any formula involving temperature T (Charles's, Gay-Lussac's, Combined, Ideal), you must convert from Celsius to Kelvin: $T_K = T_{^{\circ}C} + 273.15$. In \textbf{Gas Law Problems \#3}, converting $-24.36^{\circ}$C to 248.79 K was the first and most important step. Failure to do this is the most common error.
    
    \item \textbf{Algebraic Rearrangement of Ideal Gas Law:} You must be comfortable solving $PV=nRT$ for any of its variables. In \textbf{Molar Mass Worksheet \#4}, solving for molar mass required combining $n=m/\mathcal{M}$ with the ideal gas law to get $\mathcal{M} = \frac{mRT}{PV}$.
    
    \item \textbf{Solving Proportions:} For the empirical and combined gas laws, you are solving proportions. For example, in \textbf{Gas Law Problems \#1}, solving $P_1/T_1 = P_2/T_2$ for $P_2$ requires multiplying both sides by $T_2$ to get $P_2 = P_1(T_2/T_1)$.

    \item \textbf{Solving Square Root Equations (Graham's Law):} In \textbf{KMT Worksheet \#4}, we had to solve for $\mathcal{M}_x$ inside a square root. This required squaring both sides of the equation: $(1.12)^2 = (\sqrt{\dots})^2$. This eliminates the radical and allows you to solve for the variable.
    
    \item \textbf{Unit Analysis and Cancellation:} This is a critical error-checking tool. In \textbf{Molar Mass Worksheet \#1}, the calculation $\mathcal{M} = \frac{(14.5 \text{ g/L})(0.08206 \frac{\text{L} \cdot \text{atm}}{\text{mol} \cdot \text{K}})(223.15 \text{ K})}{6.00 \text{ atm}}$ shows how L, atm, and K all cancel out, leaving the correct final units of g/mol. Always write your units and ensure they cancel properly.
    
    \item \textbf{Stoichiometric Bridge (Mole Ratio):} The central step of any gas stoichiometry problem. In \textbf{Molar Mass Worksheet \#6}, the ratio $\frac{2 \text{ mol NH}_3}{6 \text{ mol HCl}}$ from the balanced equation was the essential link to get from the known amount of product (HCl) to the unknown amount of reactant (NH$_3$).
\end{enumerate}

\part{"Cheatsheet" and Tips for Success}

\section{Summary of Most Important Formulas}
\begin{itemize}
    \item \textbf{Ideal Gas Law:} $PV = nRT$ (Your most versatile tool)
    \item \textbf{Combined Gas Law:} $\frac{P_1V_1}{T_1} = \frac{P_2V_2}{T_2}$ (For changing conditions, constant moles)
    \item \textbf{Molar Mass/Density Link:} $\mathcal{M} = \frac{dRT}{P}$
    \item \textbf{Dalton's Law:} $P_{\text{total}} = P_A + P_B + \dots$ and $P_A = \chi_A P_{\text{total}}$
    \item \textbf{Graham's Law:} $\frac{\text{rate}_A}{\text{rate}_B} = \sqrt{\frac{\mathcal{M}_B}{\mathcal{M}_A}}$
\end{itemize}

\section{Tricks and Shortcuts}
\begin{itemize}
    \item \textbf{Recognizing Relationships:} Before you calculate, think about the relationship. In a Boyle's law problem, if pressure goes down, volume MUST go up. If your answer shows volume also going down, you made a mistake.
    \item \textbf{STP Means 22.4 L/mol:} If a problem mentions STP and asks for volume from moles (or vice-versa), you can use the shortcut that 1 mole = 22.4 L. This avoids a full PV=nRT calculation. (Used in \textbf{Molar Mass Worksheet \#3}).
    \item \textbf{Graham's Law Logic Check:} The lighter gas is always faster. The gas with the smaller molar mass should have the larger rate value. If you calculate that He is slower than Xe, you flipped the ratio.
    \item \textbf{Combined Gas Law as the Parent:} Boyle's, Charles's, and Gay-Lussac's laws are all just simplified versions of the Combined Gas Law. If volume is constant, $V_1=V_2$, so they cancel out, leaving $P_1/T_1 = P_2/T_2$ (Gay-Lussac's Law). You can always start with the Combined Gas Law and cancel the variables that are held constant.
\end{itemize}

\section{Common Pitfalls and How to Avoid Them}
\begin{enumerate}
    \item \textbf{THE KELVIN TRAP:} \textbf{ALWAYS convert temperature to Kelvin.} There are no exceptions. Write `T(K) = T($^\circ$C) + 273.15` as the first step on every problem involving temperature.
    \item \textbf{Unit Mismatch with R:} If you use $R=0.08206$, your pressure MUST be in atm, and volume MUST be in L. If the problem gives you pressure in torr or mmHg, convert it to atm BEFORE plugging it into the Ideal Gas Law.
    \item \textbf{Flipping Graham's Law Ratio:} Remember the inverse relationship. Rate is on top, its Molar Mass is on the bottom inside the square root. $\text{rate}_A/\text{rate}_B = \sqrt{\mathcal{M}_B/\mathcal{M}_A}$. Write it down every time.
    \item \textbf{Confusing Total Moles vs. Component Moles:} In mixture problems, be careful whether the question asks for a property of a single gas (use $n_i, P_i$) or the whole mixture (use $n_{\text{total}}, P_{\text{total}}$).
    \item \textbf{Stoichiometry Errors:} Forgetting to use the coefficients from the balanced equation is a common mistake. Always double-check your mole ratios.
\end{enumerate}

\part{Conceptual Synthesis and The "Big Picture"}

\section{Thematic Connections}
The core theme of this chapter is \textbf{modeling the macroscopic world from microscopic principles}. We observe macroscopic properties like pressure and temperature, which seem continuous and smooth. The Kinetic Molecular Theory, however, explains that these properties are the emergent result of an immense number of discrete, chaotic, microscopic events—the random motion and collision of individual particles.

This theme is central to many areas of science:
\begin{itemize}
    \item \textbf{Thermodynamics:} The concepts of heat and work are macroscopic, but they are explained by the transfer of kinetic energy between microscopic particles.
    \item \textbf{Kinetics:} The rate of a chemical reaction (macroscopic) is explained by Collision Theory, which depends on the frequency and energy of collisions between reactant molecules (microscopic).
    \item \textbf{Calculus:} The very idea of an integral is to find a macroscopic quantity (like area) by summing up an infinite number of infinitesimally small, microscopic pieces.
\end{itemize}
The gas laws are our first major foray into creating a powerful mathematical model to connect these two scales.

\section{Forward and Backward Links}
\begin{itemize}
    \item \textbf{Backward Links (Foundations):} This chapter is a direct application of the \textbf{mole concept and stoichiometry} from introductory chemistry. Without a firm grasp of converting between mass and moles, and using mole ratios, the gas stoichiometry problems are impossible. It also relies heavily on \textbf{basic algebra} for manipulating the gas law equations.

    \item \textbf{Forward Links (Building Blocks):} The concepts from this chapter are absolutely essential for future topics:
        \begin{itemize}
            \item \textbf{Chemical Equilibrium:} The equilibrium constant for gases, $K_p$, is expressed in terms of the \textbf{partial pressures} of the reactants and products. Understanding Dalton's Law is a prerequisite.
            \item \textbf{Thermodynamics:} The first law of thermodynamics often involves calculating the work done by an expanding gas ($w = -P\Delta V$). Understanding the relationship between P, V, and T is fundamental.
            \item \textbf{Physical Chemistry:} This entire field is, in many ways, an extension of the ideas in this chapter—using physics and advanced mathematics to model the behavior of chemical systems, with the ideal gas being the simplest starting point.
        \end{itemize}
\end{itemize}

\part{Real-World Application and Modeling}

\section{Concrete Scenarios in Economics and Finance}
While gas laws directly model physical systems, their underlying principles of modeling complex systems with simplified equations and understanding emergent behavior from random individual actions have strong parallels in quantitative finance and economics.

\begin{enumerate}
    \item \textbf{Derivative Pricing and the Black-Scholes Model:} The Black-Scholes model is a foundational equation used to price stock options. It is essentially an "ideal gas law" for finance. Just as the ideal gas law models gas behavior by making simplifying assumptions (particles have no volume, no intermolecular forces), the Black-Scholes model makes assumptions about the market (prices follow a random walk, volatility is constant, no transaction costs). In both cases, the "ideal" model provides a powerful and surprisingly accurate baseline, but "real" markets (like "real" gases) exhibit deviations that lead to more advanced models. The variable for temperature in KMT, which represents random kinetic energy, is analogous to the \textbf{volatility} term in Black-Scholes, which represents the magnitude of random price fluctuations.

    \item \textbf{Stochastic Calculus and Asset Price Modeling:} The random, chaotic motion of gas particles described by KMT is a classic example of a \textbf{stochastic process}. Financial mathematicians use a type of stochastic calculus called It\^{o} calculus to model the price of stocks and other securities. They model the price path not as a predictable line, but as a "random walk" with a certain drift and volatility, very similar to how a physicist would model the path of a single pollen grain being bombarded by water molecules (Brownian motion), which itself is related to the kinetic theory of fluids.

    \item \textbf{Economic Modeling of Resource Consumption:} The consumption of a finite resource like natural gas can be modeled using variables analogous to the gas laws. The total amount of gas in a reserve is the number of moles ($n$). The volume ($V$) could represent the size of the reserve. The economic "pressure" ($P$) to extract the gas is driven by market demand and price. A high price creates high pressure, leading to a faster rate of consumption (flow, analogous to effusion). As the resource depletes ($n$ decreases), the pressure required to extract the remaining gas might increase, linking all the variables in a dynamic system.
\end{enumerate}

\section{Model Problem Setup: A Financial Analogy}
Let's set up a simplified model connecting the random walk of stock prices to the kinetic theory of gases.

\textbf{Scenario:} A quantitative analyst wants to estimate the "economic pressure" on a stock, which they define as a measure of its trading activity. They model this using an analogy to gas pressure, where the "particles" are individual buy/sell orders.

\textbf{Model Setup:}
\begin{itemize}
    \item Let $N$ be the total number of shares traded in a day (analogous to the number of particles, $n$).
    \item Let $V$ be the "market volume," which could be a measure of the total market capitalization or liquidity (analogous to physical volume, $V$).
    \item Let $\sigma$ (volatility) be a measure of the average magnitude of price swings. In the Black-Scholes model, the term $\sigma^2$ is called variance and is analogous to the mean-square speed of particles. We can therefore say volatility is analogous to temperature, as both drive the system's "energy": $\sigma \propto T$.
    \item We want to find the "Economic Pressure," $P_{\text{econ}}$.
\end{itemize}
Using a direct analogy to the Ideal Gas Law, $PV=nRT$, the analyst could propose a model:
\[ P_{\text{econ}}V = N k \sigma \]
Here, $k$ is a new "Economic Gas Constant" that would need to be determined empirically from market data. This equation posits that the pressure (trading intensity) is directly proportional to the number of trades and the market volatility, and inversely proportional to the market size. While not a physical law, this type of modeling uses the framework of well-understood physical systems to create testable hypotheses about complex economic systems.

\part{Common Variations and Untested Concepts}
Your homework set was very comprehensive. However, a standard chemistry curriculum often includes the following topic, which was not explicitly tested.

\section{Real Gases and the van der Waals Equation}
The Ideal Gas Law assumes gas particles have no volume and no intermolecular attractions. This is a good approximation at low pressure (particles are far apart) and high temperature (particles move too fast to interact). At high pressure or low temperature, these assumptions fail, and we must use a more sophisticated model for "real" gases.

The \textbf{van der Waals equation} adjusts the ideal gas law to account for these factors:
\[ \left( P + \frac{an^2}{V^2} \right) (V - nb) = nRT \]
\begin{itemize}
    \item The term $\frac{an^2}{V^2}$ corrects the measured pressure \textbf{upwards}. The constant '$a$' accounts for intermolecular attractions. These attractions pull molecules together, reducing the force of their collisions with the container wall and thus lowering the measured pressure compared to the ideal case. We add this term back to the measured pressure to get the "ideal" pressure.
    \item The term $nb$ corrects the container volume \textbf{downwards}. The constant '$b$' represents the volume of the gas particles themselves. The free space available for the particles to move in is the container volume minus the volume the particles themselves take up.
\end{itemize}
Constants '$a$' and '$b$' are unique for each gas.

\textbf{Worked Example:}
Calculate the pressure exerted by 1.00 mole of ammonia (NH$_3$) in a 5.00 L container at 300 K using both the ideal gas law and the van der Waals equation. For NH$_3$, $a = 4.17 \frac{\text{L}^2\text{atm}}{\text{mol}^2}$ and $b = 0.0371 \frac{\text{L}}{\text{mol}}$.

\textbf{A) Ideal Gas Law Solution:}
\[ P = \frac{nRT}{V} = \frac{(1.00 \text{ mol})(0.08206 \frac{\text{L} \cdot \text{atm}}{\text{mol} \cdot \text{K}})(300 \text{ K})}{5.00 \text{ L}} = 4.92 \text{ atm} \]

\textbf{B) van der Waals Solution:}
\begin{align*}
    P &= \frac{nRT}{V-nb} - \frac{an^2}{V^2} \\
    P &= \frac{(1.00)(0.08206)(300)}{5.00 - (1.00)(0.0371)} - \frac{(4.17)(1.00)^2}{(5.00)^2} \\
    P &= \frac{24.618}{4.9629} - \frac{4.17}{25.00} \\
    P &= 4.960 - 0.1668 \\
    P &= 4.79 \text{ atm}
\end{align*}
In this case, the real pressure is slightly lower than the ideal pressure, indicating that for ammonia under these conditions, the effect of intermolecular attractions (the '$a$' term) is more significant than the effect of molecular volume (the '$b$' term).

\part{Advanced Diagnostic Testing: "Find the Flaw"}
For each problem below, a complete solution is provided. However, there is one subtle but critical error in each solution. Your task is to find the flaw, explain why it is wrong, and provide the correct step and final answer.

\subsection{Problem 1}
A 2.50 L container of gas at 1.00 atm is heated from 25$^{\circ}$C to 100$^{\circ}$C. What is the final pressure in the container?

\textbf{Flawed Solution:}
This is a pressure-temperature problem, so we use Gay-Lussac's Law: $\frac{P_1}{T_1} = \frac{P_2}{T_2}$.
\begin{itemize}
    \item $P_1 = 1.00$ atm
    \item $T_1 = 25^{\circ}$C
    \item $T_2 = 100^{\circ}$C
\end{itemize}
Solving for $P_2$:
\[ P_2 = P_1 \times \frac{T_2}{T_1} = 1.00 \text{ atm} \times \frac{100^{\circ}\text{C}}{25^{\circ}\text{C}} = 4.00 \text{ atm} \]
\textbf{Final Answer:} 4.00 atm.

---
\textbf{Your Analysis:}
\begin{itemize}
    \item \textbf{Identify the Flaw:}
    \item \textbf{Explain the Error:}
    \item \textbf{Provide the Correction:}
\end{itemize}
\rule{\textwidth}{0.4pt}

\subsection{Problem 2}
What is the molar mass of a gas that has a density of 1.50 g/L at a pressure of 745 torr and a temperature of 298 K?

\textbf{Flawed Solution:}
We use the formula $\mathcal{M} = \frac{dRT}{P}$.
\begin{itemize}
    \item $d = 1.50$ g/L
    \item $R = 0.08206 \frac{\text{L} \cdot \text{atm}}{\text{mol} \cdot \text{K}}$
    \item $T = 298$ K
    \item $P = 745$ torr
\end{itemize}
Plugging in the values:
\[ \mathcal{M} = \frac{(1.50 \text{ g/L})(0.08206 \frac{\text{L} \cdot \text{atm}}{\text{mol} \cdot \text{K}})(298 \text{ K})}{745 \text{ torr}} = 0.0492 \text{ g/mol} \]
\textbf{Final Answer:} 0.0492 g/mol.

---
\textbf{Your Analysis:}
\begin{itemize}
    \item \textbf{Identify the Flaw:}
    \item \textbf{Explain the Error:}
    \item \textbf{Provide the Correction:}
\end{itemize}
\rule{\textwidth}{0.4pt}

\subsection{Problem 3}
An unknown gas effuses at a rate of 25.0 mL/min, while nitrogen gas (N$_2$) effuses at 35.0 mL/min under the same conditions. What is the molar mass of the unknown gas?

\textbf{Flawed Solution:}
We use Graham's Law: $\frac{\text{rate}_A}{\text{rate}_B} = \sqrt{\frac{\mathcal{M}_B}{\mathcal{M}_A}}$.
Let A be the unknown gas (Unk) and B be N$_2$. $\mathcal{M}_{\text{N}_2} = 28.02$ g/mol.
\[ \frac{\text{rate}_{\text{Unk}}}{\text{rate}_{\text{N}_2}} = \frac{25.0}{35.0} \approx 0.714 \]
\[ 0.714 = \sqrt{\frac{\mathcal{M}_{\text{Unk}}}{\mathcal{M}_{\text{N}_2}}} = \sqrt{\frac{\mathcal{M}_{\text{Unk}}}{28.02 \text{ g/mol}}} \]
Square both sides:
\[ (0.714)^2 = 0.510 = \frac{\mathcal{M}_{\text{Unk}}}{28.02 \text{ g/mol}} \]
\[ \mathcal{M}_{\text{Unk}} = 0.510 \times 28.02 \text{ g/mol} = 14.3 \text{ g/mol} \]
\textbf{Final Answer:} 14.3 g/mol.

---
\textbf{Your Analysis:}
\begin{itemize}
    \item \textbf{Identify the Flaw:}
    \item \textbf{Explain the Error:}
    \item \textbf{Provide the Correction:}
\end{itemize}
\rule{\textwidth}{0.4pt}

\subsection{Problem 4}
How many liters of O$_2$ gas at STP are required to completely combust 32.0 grams of methane (CH$_4$)?
Reaction: CH$_4$(g) + 2O$_2$(g) $\rightarrow$ CO$_2$(g) + 2H$_2$O(g)

\textbf{Flawed Solution:}
\begin{enumerate}
    \item Convert grams of CH$_4$ to moles. Molar mass of CH$_4$ is 16.05 g/mol.
    \[ 32.0 \text{ g CH}_4 \times \frac{1 \text{ mol CH}_4}{16.05 \text{ g CH}_4} = 1.99 \text{ mol CH}_4 \]
    \item Use the mole ratio to find moles of O$_2$. The ratio is 1 CH$_4$ : 1 CO$_2$. Oh wait, we need O$_2$. The ratio is 1 CH$_4$ : 2 O$_2$. Let's just use the 1:1 ratio.
    \[ 1.99 \text{ mol CH}_4 \times \frac{1 \text{ mol O}_2}{1 \text{ mol CH}_4} = 1.99 \text{ mol O}_2 \]
    \item Convert moles of O$_2$ to Liters at STP using the molar volume.
    \[ 1.99 \text{ mol O}_2 \times \frac{22.4 \text{ L}}{1 \text{ mol}} = 44.6 \text{ L} \]
\end{enumerate}
\textbf{Final Answer:} 44.6 L.

---
\textbf{Your Analysis:}
\begin{itemize}
    \item \textbf{Identify the Flaw:}
    \item \textbf{Explain the Error:}
    \item \textbf{Provide the Correction:}
\end{itemize}
\rule{\textwidth}{0.4pt}

\subsection{Problem 5}
A gas mixture contains 0.50 moles of He and 1.50 moles of Ar in a 10.0 L container at 273 K. What is the partial pressure of He?

\textbf{Flawed Solution:}
First, calculate the total pressure using the total moles.
$n_{\text{total}} = 0.50 + 1.50 = 2.00$ moles.
\[ P_{\text{total}} = \frac{n_{\text{total}}RT}{V} = \frac{(2.00 \text{ mol})(0.08206 \frac{\text{L} \cdot \text{atm}}{\text{mol} \cdot \text{K}})(273 \text{ K})}{10.0 \text{ L}} = 4.48 \text{ atm} \]
The partial pressure is the total pressure divided by the number of components.
\[ P_{\text{He}} = \frac{P_{\text{total}}}{2} = \frac{4.48 \text{ atm}}{2} = 2.24 \text{ atm} \]
\textbf{Final Answer:} 2.24 atm.

---
\textbf{Your Analysis:}
\begin{itemize}
    \item \textbf{Identify the Flaw:}
    \item \textbf{Explain the Error:}
    \item \textbf{Provide the Correction:}
\end{itemize}
\rule{\textwidth}{0.4pt}

\end{document}