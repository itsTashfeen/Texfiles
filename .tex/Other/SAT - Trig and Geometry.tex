\documentclass[11pt, a4paper]{article}
\usepackage[margin=0.75in]{geometry}
\usepackage[utf8]{inputenc}
\usepackage{xcolor}
\usepackage{tcolorbox}
\usepackage{enumitem}
\usepackage{amsmath}
\usepackage{amssymb}
\usepackage{titlesec}
\usepackage{parskip}

% Color Definitions
\definecolor{mathnavy}{RGB}{0, 51, 102}
\definecolor{mathred}{RGB}{178, 34, 34}
\definecolor{mathgreen}{RGB}{34, 139, 34}
\definecolor{mathorange}{RGB}{204, 102, 0}

% Custom Commands
\newcommand{\rulebox}[2]{
    \begin{tcolorbox}[colback=blue!5!white, colframe=mathnavy, title=\textbf{#1}]
    #2
    \end{tcolorbox}
}

\newcommand{\alertbox}[2]{
    \begin{tcolorbox}[colback=red!5!white, colframe=mathred, title=\textbf{#1}]
    #2
    \end{tcolorbox}
}

\newcommand{\tip}[1]{\textbf{\textcolor{mathgreen}{PRO TIP:}} #1}

% Title Format
\titleformat{\section}
{\normalfont\Large\bfseries\color{mathnavy}}{\thesection}{1em}{}

\begin{document}

\begin{center}
    {\Huge \textbf{SAT Math Cheat Sheet}} \\
    \vspace{0.5em}
    {\Large Geometry, Trigonometry \& Inscribed Shapes} \\
    \vspace{1em}
    \textit{The SAT Math section relies on \textbf{patterns}. Memorize these rules to solve problems in seconds.}
\end{center}

\vspace{0.5cm}

% SECTION I: TRIG IDENTITIES
\section{The ``Must-Memorize'' Identities}
These rules allow you to solve trig problems without drawing a triangle.

\rulebox{1. The Complementary Rule (Cofunction)}{
    For any right triangle with acute angles $A$ and $B$ (where $A+B=90^\circ$):
    \begin{itemize}
        \item $\sin(A) = \cos(B)$
        \item $\sin(A) = \cos(90^\circ - A)$
    \end{itemize}
    \textbf{Reciprocal Tangent Rule:}
    \begin{itemize}
        \item $\tan(A) = \frac{1}{\tan(B)}$
    \end{itemize}
}

\rulebox{2. The Pythagorean Identity}{
    For any angle $\theta$:
    \[ \sin^2(\theta) + \cos^2(\theta) = 1 \]
}

% SECTION II: RIGHT TRIANGLES
\section{Right Triangles (The Core)}

\subsection*{A. Pythagorean Triples}
Memorize these integer sets. If you see two, you know the third instantly.
\begin{itemize}
    \item \textbf{3 - 4 - 5} (and multiples like 6 - 8 - 10)
    \item \textbf{5 - 12 - 13}
    \item \textbf{8 - 15 - 17}
    \item \textbf{7 - 24 - 25}
\end{itemize}

\subsection*{B. Special Right Triangles}
\begin{tcolorbox}[colback=yellow!10!white, colframe=orange!80!black, title=\textbf{Shortcuts}]
    \textbf{30-60-90 Triangle}
    \begin{itemize}
        \item Side opposite $30^\circ$ ($x$)
        \item Side opposite $60^\circ$ ($x\sqrt{3}$)
        \item Hypotenuse ($2x$) \quad \textit{(Hypotenuse is double the short leg)}
    \end{itemize}
    \textbf{45-45-90 Triangle (Isosceles)}
    \begin{itemize}
        \item Legs ($x$)
        \item Hypotenuse ($x\sqrt{2}$)
    \end{itemize}
\end{tcolorbox}

% SECTION III: INSCRIBED SHAPES
\section{Inscribed Shapes (Shapes Inside Shapes)}
\textit{This is essentially the ``Hidden Rules'' section.}

\rulebox{1. Square Inscribed in a Circle}{
    \textbf{Visual:} A square inside a circle, corners touching the edge.
    \begin{itemize}
        \item \textbf{Rule:} The \textbf{Diagonal} of the square = The \textbf{Diameter} of the circle.
        \item \textbf{Math:} If side is $s$: \quad Diameter $d = s\sqrt{2}$
    \end{itemize}
}

\rulebox{2. Circle Inscribed in a Square}{
    \textbf{Visual:} A circle inside a square, edges touching the sides.
    \begin{itemize}
        \item \textbf{Rule:} The \textbf{Side Length} of the square = The \textbf{Diameter} of the circle.
        \item \textbf{Math:} If side is $s$: \quad Diameter $d = s$ \quad (Radius $r = s/2$)
    \end{itemize}
}

\rulebox{3. Hexagon Inscribed in a Circle}{
    \textbf{Visual:} A 6-sided regular polygon inside a circle.
    \begin{itemize}
        \item \textbf{Rule:} The \textbf{Side Length} of the hexagon = The \textbf{Radius} of the circle.
    \end{itemize}
}

\rulebox{4. Triangle on Diameter (Thales's Theorem)}{
    \textbf{Visual:} A triangle inside a circle where one side is the Diameter.
    \begin{itemize}
        \item \textbf{Rule:} The angle opposite the diameter is always $\mathbf{90^\circ}$.
        \item \textbf{Strategy:} This automatically creates a Right Triangle. Use Pythagorean Theorem.
    \end{itemize}
}

% SECTION IV: GENERAL TRIANGLE RULES
\section{Non-Right Triangles \& Similarity}

\subsection*{A. Similarity \& Scale Factors}
If two triangles have the \textbf{same angles}, they are \textbf{Similar}.
\begin{itemize}
    \item \textbf{Side Lengths:} Scale by factor $k$.
    \item \textbf{Perimeter:} Scales by factor $k$.
    \item \textbf{Area:} Scales by factor $\mathbf{k^2}$.
    \item \textbf{Trig Functions:} Do \textbf{NOT} change. $\sin(A)$ in a tiny triangle is equal to $\sin(A)$ in a huge similar triangle.
\end{itemize}

\subsection*{B. Triangle Inequality Theorem}
The sum of any two sides must be greater than the third side.
\[ \text{Difference} < \text{Third Side} < \text{Sum} \]

\subsection*{C. Equilateral Triangle Area Shortcut}
\[ \text{Area} = \frac{s^2\sqrt{3}}{4} \]

% SECTION V: CIRCLES & RADIANS
\section{Circles \& Radians}

\subsection*{A. Conversions}
\begin{itemize}
    \item $\pi \text{ radians} = 180^\circ$
    \item \textbf{Degrees} $\to$ \textbf{Radians}: Multiply by $\frac{\pi}{180}$
    \item \textbf{Radians} $\to$ \textbf{Degrees}: Multiply by $\frac{180}{\pi}$
\end{itemize}

\subsection*{B. Arc Length \& Sector Area}
If $\theta$ is in \textbf{Radians}:
\begin{itemize}
    \item \textbf{Arc Length:} $s = r\theta$
    \item \textbf{Sector Area:} $A = \frac{1}{2}r^2\theta$
\end{itemize}

\end{document}