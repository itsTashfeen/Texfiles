\documentclass[a4paper, 11pt]{article}

% PACKAGES
\usepackage[utf8]{inputenc}
\usepackage{geometry}
\usepackage{listings}
\usepackage{xcolor}
\usepackage{hyperref}
\usepackage{amsmath}

% PAGE LAYOUT
\geometry{a4paper, margin=1in}

% HYPERREF SETUP
\hypersetup{
    colorlinks=true,
    linkcolor=blue,
    filecolor=magenta,      
    urlcolor=cyan,
    pdftitle={yt-dlp Command Reference},
    pdfpagemode=FullScreen,
}

% LISTINGS (CODE BLOCK) STYLE
\definecolor{codegray}{gray}{0.95}
\definecolor{codepurple}{rgb}{0.58,0,0.82}
\definecolor{codeblue}{rgb}{0.1,0.1,0.7}

\lstdefinestyle{mystyle}{
    backgroundcolor=\color{codegray},
    commentstyle=\color{green!40!black},
    keywordstyle=\color{codeblue},
    numberstyle=\tiny\color{gray},
    stringstyle=\color{codepurple},
    basicstyle=\ttfamily\footnotesize,
    breakatwhitespace=false,
    breaklines=true,
    captionpos=b,
    keepspaces=true,
    numbers=left,
    numbersep=5pt,
    showspaces=false,
    showstringspaces=false,
    showtabs=false,
    tabsize=2,
    frame=single,
    framerule=0pt,
    rulecolor=\color{black},
    title=\lstname
}
\lstset{style=mystyle, language=bash}

% DOCUMENT METADATA
\title{Definitive yt-dlp Command Reference}
\author{Generated based on user queries and error resolution}
\date{\today}

% DOCUMENT START
\begin{document}
\maketitle

\begin{abstract}
This document provides a curated list of robust \texttt{yt-dlp} commands for common downloading tasks. The commands have been refined to handle typical errors (like format selection and HTTP 403 Forbidden errors) and specific user requirements (like subtitle handling).
\end{abstract}

\tableofcontents
\newpage

\section{Essential First Step: How to Update yt-dlp}
Before running any download command, it is crucial to ensure \texttt{yt-dlp} is up-to-date. YouTube frequently changes its backend, which can cause "HTTP Error 403: Forbidden" errors. Running the update command is the primary solution to this problem.

\begin{lstlisting}[caption={yt-dlp update command}, label={lst:update}]
yt-dlp -U
\end{lstlisting}

\section{Command Reference List}
\subsection{1. Download a Full Playlist (Video, Audio \& Subtitles)}
This command downloads an entire playlist, selecting the best available video and audio for each item. It also downloads the manually uploaded English subtitle first, and if not available, falls back to the auto-generated one. Subtitles are converted to the \texttt{.srt} format.

\begin{lstlisting}[caption={Download full playlist with subtitles}, label={lst:playlist-subs}]
yt-dlp --write-subs --write-auto-subs --sub-lang "en" --convert-subs srt -f "bestvideo+bestaudio/best" -o "\%(playlist)s/\%(playlist_index)s - \%(title)s.\%(ext)s" "PLAYLIST_URL"
\end{lstlisting}

\subsection{2. Download a Full Playlist (Video \& Audio Only)}
This command downloads the entire playlist with the best video and audio but completely skips searching for or downloading any subtitles.

\begin{lstlisting}[caption={Download full playlist without subtitles}, label={lst:playlist-no-subs}]
cyt-dlp -f "bestvideo+bestaudio/best" -o "\%(playlist)s/\%(playlist_index)s - \%(title)s.\%(ext)s" "PLAYLIST_URL"
\end{lstlisting}

\subsection{3. Download Only Subtitles for a Full Playlist}
This command skips downloading the video and audio content entirely. It only saves the English \texttt{.srt} subtitle files for every video in the playlist, using the same directory structure as the full download.

\begin{lstlisting}[caption={Download only subtitles for a playlist}, label={lst:playlist-only-subs}]
yt-dlp --write-subs --write-auto-subs --sub-lang "en" --convert-subs srt --skip-download -o "\%(playlist)s/\%(playlist_index)s - \%(title)s.\%(ext)s" "PLAYLIST_URL"
\end{lstlisting}

\subsection{4. Download a Single Video (Video, Audio \& Subtitles)}
This command downloads a single video with the best quality video and audio, along with its English \texttt{.srt} subtitle file. The output is named after the video title.

\begin{lstlisting}[caption={Download a single video with subtitles}, label={lst:single-subs}]
yt-dlp --write-subs --write-auto-subs --sub-lang "en" --convert-subs srt -f "bestvideo+bestaudio/best" -o "\%(title)s.\%(ext)s" "SINGLE_VIDEO_URL"
\end{lstlisting}

\subsection{5. Download Only Subtitles for a Single Video}
This command skips the video and audio download for a single video and saves only the English \texttt{.srt} subtitle file, naming it after the video's title.

\begin{lstlisting}[caption={Download only subtitles for a single video}, label={lst:single-only-subs}]
yt-dlp --write-subs --write-auto-subs --sub-lang "en" --convert-subs srt --skip-download -o "\%(title)s.\%(ext)s" "SINGLE_VIDEO_URL"
\end{lstlisting}

\end{document}
% DOCUMENT END